%! Author = Mateusz Redosz
%! Date = 20/09/2025

% Słownik pojęć

\newglossaryentry{backend}
{
    name={Backend},
    description={Część aplikacji odpowiedzialna za logikę biznesową, przetwarzanie danych i komunikację z bazą danych. Działa po stronie serwera i obsługuje żądania wysyłane przez frontend.}
}

\newglossaryentry{frontend}
{
    name={Frontend},
    description={Warstwa aplikacji odpowiedzialna za interfejs użytkownika oraz interakcję z użytkownikiem. Zazwyczaj tworzona przy użyciu technologii takich jak HTML, CSS i JavaScript.}
}

\newglossaryentry{baza-danych}
{
    name={Baza danych},
    description={Zbiór uporządkowanych danych przechowywanych w sposób umożliwiający ich łatwe wyszukiwanie, modyfikowanie i analizowanie. W aplikacjach najczęściej wykorzystywane są relacyjne lub nierelacyjne bazy danych.}
}

\newglossaryentry{framework}
{
    name={Framework},
    description={Zestaw narzędzi, bibliotek i struktur wspomagających tworzenie aplikacji. Ułatwia programowanie poprzez dostarczenie gotowych komponentów oraz określenie zasad organizacji kodu.}
}

\newglossaryentry{review-kodu}
{
    name={Review kodu},
    description={Proces polegający na wzajemnym przeglądzie kodu źródłowego przez programistów w celu wykrycia błędów, poprawy jakości oraz zwiększenia spójności projektu.}
}

\newglossaryentry{jwt}
{
    name={JWT},
    description={Skrót od \textit{JSON Web Token}. Standard służący do bezpiecznego przekazywania informacji między stronami w formacie JSON, często używany w procesach autoryzacji użytkowników.}
}

\newglossaryentry{cicd}
{
    name={CI/CD},
    description={Skrót od \textit{Continuous Integration/Continuous Deployment}. Praktyka programistyczna polegająca na automatyzacji procesu budowania, testowania i wdrażania oprogramowania.}
}

\newglossaryentry{commit}
{
    name={Commit},
    description={Zapis zmian w repozytorium systemu kontroli wersji, który dokumentuje stan projektu w określonym momencie.}
}

\newglossaryentry{push}
{
    name={Push},
    description={Operacja w systemie kontroli wersji polegająca na wysłaniu lokalnych zmian (commitów) do zdalnego repozytorium.}
}

\newglossaryentry{spot}
{
    name={Spot},
    description={Spotkanie zespołu projektowego, zazwyczaj krótkie i regularne, służące omówieniu postępów prac, problemów oraz planów na najbliższy okres.}
}

\newglossaryentry{sidebar}
{
    name={Sidebar},
    description={Boczny panel w interfejsie użytkownika, zawierający menu nawigacyjne lub dodatkowe opcje funkcjonalne aplikacji.}
}

\newglossaryentry{design}
{
    name={Design},
    description={Etap lub proces projektowania wyglądu i funkcjonalności aplikacji, obejmujący zarówno aspekty wizualne, jak i użytkowe (UX/UI).}
}

\newglossaryentry{DAD_LLC}{
    name={Disciplined Agile Delivery - Lean Life Cycle},
    description={
        Disciplined Agile Delivery w wariancie Lean Life Cycle to sposób prowadzenia projektu,
        który łączy elastyczność Agile z przewidywalnością Waterfalla,
        ale bez stałych sprintów — praca toczy się w ciągłym przepływie.
        Na starcie zakłada mocniejszą fazę przygotowawczą: doprecyzowanie zakresu,
        szkic architektury, identyfikację ryzyk i kryteria jakości. W realizacji następuje ciągłe doprecyzowywanie wymagań
        i backlogu, oparte na regularnym feedbacku udziałowców, co pozwala korygować obrany kurs.
        Całość opiera się na praktykach Lean — koncentracji na wartości, wizualizacji pracy
        i małych partiach dostaw — oraz lekkim governance: peer review
        i regularnych przeglądach postępów.
    }
}

\newglossaryentry{operator-drona}
{
    name={Operator drona},
    description={Osoba lub podmiot będący właścicielem floty dronów. Może posiadać jeden lub wiele dronów. Droniarz rekreacyjny jest zazwyczaj jednocześnie operatorem floty oraz pilotem.}
}

\newglossaryentry{pilot-drona}
{
    name={Pilot drona},
    description={Osoba posiadająca uprawnienia do pilotowania drona (jeżeli są wymagane) i wykonująca loty dronem. Droniarz rekreacyjny jest zazwyczaj jednocześnie pilotem oraz operatorem floty.}
}

\newglossaryentry{droniarz}
{
    name={Droniarz},
    description={Potoczne określenie osoby, która jest jednocześnie pilotem oraz operatorem drona. Zwykle entuzjasta dronów.}
}

\newglossaryentry{pilot-fpv}
{
    name={Pilot FPV},
    description={Pilot drona latający w trybie \textit{First Person View} (FPV), korzystający z gogli przekazujących obraz z kamery pokładowej, co pozwala sterować z perspektywy pierwszej osoby.}
}
\newglossaryentry{droniarz-fpv}{alias=pilot-fpv}

\newglossaryentry{pilot-foto}
{
    name={Pilot foto},
    description={Pilot wykorzystujący drony fotograficzne/filmowe do rejestracji materiałów wizualnych (zdjęcia, wideo), zwykle z naciskiem na stabilizację i jakość obrazu.}
}
\newglossaryentry{droniarz-foto}{alias=pilot-foto}
\newglossaryentry{droniarz-fotograf}{alias=pilot-foto}

\newglossaryentry{azure-blob-storage}
{
    name={Azure Blob Storage},
    description={Usługa magazynu obiektowego w chmurze Microsoft Azure do przechowywania nieustrukturyzowanych danych (\textit{blobs}) takich jak obrazy, wideo i pliki. Udostępnia kontenery, warstwy dostępu, wersjonowanie oraz tokeny SAS; często używana do hostowania multimediów w aplikacjach webowych.}
}
