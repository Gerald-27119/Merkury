%! Author = Mateusz Redosz
%! Date = 20/09/2025

% Słownik pojęć

\newglossaryentry{backend}
{
    name={Backend},
    description={Część aplikacji odpowiedzialna za logikę biznesową, przetwarzanie danych i komunikację z bazą danych. Działa po stronie serwera i obsługuje żądania wysyłane przez frontend}
}

\newglossaryentry{frontend}
{
    name={Frontend},
    description={Warstwa aplikacji odpowiedzialna za interfejs użytkownika oraz interakcję z użytkownikiem. Zazwyczaj tworzona przy użyciu technologii takich jak HTML, CSS i JavaScript}
}

\newglossaryentry{baza-danych}
{
    name={Baza danych},
    description={Zbiór uporządkowanych danych przechowywanych w sposób umożliwiający ich łatwe wyszukiwanie, modyfikowanie i analizowanie. W aplikacjach najczęściej wykorzystywane są relacyjne lub nierelacyjne bazy danych}
}

\newglossaryentry{framework}
{
    name={Framework},
    description={Zestaw narzędzi, bibliotek i struktur wspomagających tworzenie aplikacji. Ułatwia programowanie poprzez dostarczenie gotowych komponentów oraz określenie zasad organizacji kodu}
}

\newglossaryentry{review-kodu}
{
    name={Review kodu},
    description={Proces polegający na wzajemnym przeglądzie kodu źródłowego przez programistów w celu wykrycia błędów, poprawy jakości oraz zwiększenia spójności projektu}
}

\newglossaryentry{jwt}
{
    name={JWT},
    description={Skrót od \textit{JSON Web Token}. Standard służący do bezpiecznego przekazywania informacji między stronami w formacie JSON, często używany w procesach autoryzacji użytkowników}
}

\newglossaryentry{cicd}
{
    name={CI/CD},
    description={Skrót od \textit{Continuous Integration/Continuous Deployment}. Praktyka programistyczna polegająca na automatyzacji procesu budowania, testowania i wdrażania oprogramowania}
}

\newglossaryentry{commit}
{
    name={Commit},
    description={Zapis zmian w repozytorium systemu kontroli wersji, który dokumentuje stan projektu w określonym momencie}
}

\newglossaryentry{push}
{
    name={Push},
    description={Operacja w systemie kontroli wersji polegająca na wysłaniu lokalnych zmian (commitów) do zdalnego repozytorium}
}

\newglossaryentry{spot}
{
    name={Spot},
    description={Spotkanie zespołu projektowego, zazwyczaj krótkie i regularne, służące omówieniu postępów prac, problemów oraz planów na najbliższy okres}
}

\newglossaryentry{sidebar}
{
    name={Sidebar},
    description={Boczny panel w interfejsie użytkownika, zawierający menu nawigacyjne lub dodatkowe opcje funkcjonalne aplikacji}
}

\newglossaryentry{design}
{
    name={Design},
    description={Etap lub proces projektowania wyglądu i funkcjonalności aplikacji, obejmujący zarówno aspekty wizualne, jak i użytkowe (UX/UI)}
}

\newglossaryentry{folder-by-type}
{
    name={Folder by type},
    description={Sposób organizowania struktury katalogów w projekcie, w którym pliki są grupowane według rodzaju (typu) zasobu, a nie według funkcjonalności. Na przykład wszystkie komponenty trafiają do jednego folderu, wszystkie style do innego itd}
}

\newglossaryentry{biblioteka}
{
    name={Biblioteka},
    description={Zewnętrzny lub wewnętrzny zestaw gotowych funkcji, klas, komponentów lub modułów, który można wielokrotnie wykorzystywać w projekcie zamiast pisać wszystko od zera}
}

\newglossaryentry{protected-route}
{
    name={Protected route},
    description={Trasa w aplikacji, do której dostęp jest ograniczony, zwykle tylko dla zalogowanych użytkowników lub użytkowników z odpowiednimi uprawnieniami. Jeżeli użytkownik nie spełnia warunków, jest przekierowywany (np. na stronę główną)}
}

\newglossaryentry{wzorzec}
{
    name={Wzorzec},
    description={Powtarzalne, sprawdzone rozwiązanie typowego problemu projektowego lub architektonicznego. Wzorzec opisuje \emph{jak} coś organizować lub implementować, żeby było czytelne, skalowalne i łatwe w utrzymaniu}
}

\newglossaryentry{DAD_LLC}
{
    name={Disciplined Agile Delivery - Lean Life Cycle},
    description={
        Disciplined Agile Delivery w wariancie Lean Life Cycle to sposób prowadzenia projektu,
        który łączy elastyczność Agile z przewidywalnością Waterfalla,
        ale bez stałych sprintów — praca toczy się w ciągłym przepływie.
        Na starcie zakłada mocniejszą fazę przygotowawczą: doprecyzowanie zakresu,
        szkic architektury, identyfikację ryzyk i kryteria jakości.
        W realizacji następuje ciągłe doprecyzowywanie wymagań
        i backlogu, oparte na regularnym feedbacku udziałowców.
        Całość opiera się na praktykach Lean oraz lekkim governance:
        code review i regularnych przeglądach postępów.
    }
}

\newglossaryentry{operator-drona}
{
    name={Operator drona},
    description={Osoba lub podmiot będący właścicielem floty dronów. Może posiadać jeden lub wiele dronów. Droniarz rekreacyjny jest zazwyczaj jednocześnie operatorem floty oraz pilotem.}
}

\newglossaryentry{pilot-drona}
{
    name={Pilot drona},
    description={Osoba posiadająca uprawnienia do pilotowania drona (jeżeli są wymagane) i wykonująca loty dronem. Droniarz rekreacyjny jest zazwyczaj jednocześnie pilotem oraz operatorem floty.}
}

\newglossaryentry{droniarz}
{
    name={Droniarz},
    plural={droniarze},
    description={Potoczne określenie osoby, która jest jednocześnie pilotem oraz operatorem drona. Zwykle entuzjasta dronów.}
}

\newglossaryentry{PANSA}
{
    name={PANSA},
    description={
        Polish Air Navigation Services Agency, pol. Polska Agencja Żeglugi Powietrznej.
        Instytucja ta zapewnia m.in. mapę z zaznaczonymi strefami lotów.
        Każda strefa ma swoje właściwości prawne.
    }
}

\newglossaryentry{pilot-fpv}
{
    name={Pilot FPV},
    description={Pilot drona latający w trybie \textit{First Person View} (FPV),
    korzystający z gogli przekazujących obraz z kamery pokładowej.}
}

\newglossaryentry{droniarz-fpv}
{
    name={Droniarz FPV},
    description={\glsentrydesc{pilot-fpv}}
}

\newglossaryentry{droniarz-foto-video}
{
    name={Droniarz foto/video},
    description={Pilot wykorzystujący drony fotograficzne/filmowe do rejestracji materiałów wizualnych
    (zdjęcia, wideo), zwykle z naciskiem na stabilizację i jakość obrazu.},
    user1={droniarzem foto/video}
}

\newglossaryentry{droniarz-foto}
{
    name={Droniarz foto},
    description={\glsentrydesc{droniarz-foto-video}}
}

\newglossaryentry{droniarz-fotograf}
{
    name={Droniarz fotograf},
    description={\glsentrydesc{droniarz-foto-video}}
}

\newglossaryentry{pilot-foto}
{
    name={Pilot foto},
    description={\glsentrydesc{droniarz-foto-video}}
}

\newglossaryentry{azure-blob-storage}
{
    name={Azure Blob Storage},
    description={Usługa magazynu obiektowego w chmurze Microsoft Azure do przechowywania
    nieustrukturyzowanych danych (\textit{blobs}) takich jak obrazy, wideo i pliki.
    Udostępnia kontenery, warstwy dostępu, wersjonowanie oraz tokeny SAS; często używana
    do hostowania multimediów w aplikacjach webowych.}
}

\newglossaryentry{stan}{
    name={Stan},
    description={Aktualny zestaw danych przechowywanych przez aplikację lub komponent, na podstawie którego renderowany jest interfejs użytkownika. Stan może być lokalny (utrzymywany w pojedynczym komponencie) albo globalny (wspólny dla wielu komponentów).}
}

\newglossaryentry{ui}{
    name={UI},
    description={Interfejs użytkownika (ang. \textit{User Interface}); warstwa prezentacji odpowiedzialna za sposób wyświetlania danych oraz interakcji użytkownika z aplikacją.}
}

\newglossaryentry{hook}{
    name={Hook (React)},
    description={Prosta funkcja w React, która „dodaje” możliwości do elementu interfejsu — np. pozwala mu coś zapamiętać (stan) albo zrobić coś po zmianie/załadowaniu. Wszystkie hooki zaczynają się od \texttt{use...} (np. \texttt{useState}, \texttt{useEffect}).}
}

\newglossaryentry{css}{
    name={CSS},
    description={Kaskadowe arkusze stylów (Cascading Style Sheets) — język opisu prezentacji dokumentów (np. HTML). Definiuje wygląd interfejsu: układ, kolory, typografię, odstępy, animacje i zachowania responsywne, oddzielając warstwę treści od warstwy prezentacji.}
}

\newglossaryentry{responsywnosc}{
    name={Responsywność},
    description={Określenie związane z projektowaniem responsywnym (Responsive Web Design, RWD), czyli dostosowywaniem interfejsu do różnych rozmiarów i parametrów ekranów. Obejmuje m.in. elastyczne siatki, grafiki i \gls{media-queries}, tak aby układ i czytelność były zachowane na telefonach, tabletach i desktopach.}
}

\newglossaryentry{props}{
    name={Props},
    description={Właściwości przekazywane do komponentu React przez komponent nadrzędny; służą do konfiguracji i przekazywania danych. Powinny być traktowane jako tylko do odczytu (read-only) wewnątrz komponentu potomnego.}
}

\newglossaryentry{react}{
    name={React},
    description={Biblioteka JavaScript do budowy interfejsów użytkownika w oparciu o komponenty deklaratywne i wirtualny DOM. Zapewnia jednokierunkowy przepływ danych oraz zarządzanie stanem komponentów.}
}

\newglossaryentry{type-script}{
    name={TypeScript},
    description={Rozszerzenie do języka JavaScript dodający statyczne typowanie, interfejsy i narzędzia do większych projektów. Kompiluje się do czystego JavaScript, ułatwiając wykrywanie błędów w czasie kompilacji i refaktoryzację.}
}

\newglossaryentry{redux}{
    name={Redux},
    description={Biblioteka do przewidywalnego zarządzania stanem aplikacji. Opiera się na jednym \emph{store}, akcjach i czystych \emph{reducerach}, promuje niemutowalność i jednokierunkowy przepływ danych. Często używana z Reactem, ale niezależna od niego.}
}

\newglossaryentry{media-queries}{
    name={Media queries},
    description={Konstrukcja CSS pozwalająca stosować reguły stylów w zależności od cech urządzenia/okna (np. szerokości ekranu, orientacji, preferencji użytkownika). Podstawa responsywnego projektowania (\emph{responsive design}).}
}

\newglossaryentry{backlog}{
    name={Backlog},
    description={Lista zadań, które należy wykonać w ramach projektu, używane w metodykach zwinnych.}
}

\newglossaryentry{tablica_kanban}{
    name={Tablica Kanban},
    description={Narzędzie do zarządzania przepływem pracy, które pomaga zespołom śledzić zadania oraz ich postępy.
    Składa się z kolumn reprezentujących stan etapu prac, na przykład „Do zrobienia” lub „W trakcie”.}
}

\newglossaryentry{ide}{
    name={IDE},
    description={(ang. \textit{integrated development environment}); to zintegrowane środowisko programistyczne, służące do
    tworzenia, modyfikowania, testowania i konserwacji oprogramowania}
}

\newglossaryentry{api}{
    name={API},
    description={(ang. \textit{application programming interface}); zbiór reguł i operacji do komunikacji z oprogramowaniem.}
}

\longnewglossaryentry{rest_api}{
    name={REST API}
}{
    Architektura budowania usług sieciowych komunikujących się poprzez metody protokołu HTTP (GET, PUT, POST, DELETE, PATCH).
    Wymiana danych występuje często w formacie JSON lub XML.
    \\REST API musi spełniać następujące reguły:
    \begin{enumerate}[nosep]
        \item \textbf{Rozdzielenie klient-serwer} \textemdash \space klient i serwer są od siebie niezależne, komunikują się poprzez interfejs.
        \item \textbf{Bezstanowość} \textemdash \space każde żądanie przez klienta zawiera wszystkie informacje niezbędne do jego obsłużenia.
        Po otrzymaniu żądania serwer nie przechowuje o nim żadnych informacji.
        \item \textbf{Buforowalność (cache)} \textemdash \space odpowiedzi z API powinny informować, czy dane można cache’ować.
        Jeśli tak, to przy kolejnym żądaniu mogą być zwrócone z cache’a.
        \item \textbf{Jednolity interfejs}:
        \begin{itemize}[nosep]
            \item \textbf{Identyfikacja zasobów} \textemdash \space każdy zasób musi być jednoznacznie zidentyfikowany w interakcji klient-serwer.
            \item \textbf{Manipulacja zasobów poprzez reprezentację} \textemdash \space po otrzymaniu reprezentacji klient może zmienić stan zasobu przesyłając zmodyfikowaną reprezentację.
            \item \textbf{Samoopisujące się wiadomości} \textemdash \space każde żądanie i odpowiedź powinny zawierać informacje do jego poprawnego przetworzenia.
            \item \textbf{Hypermedia jako silnik stanu aplikacji (HATEOAS)} \textemdash \space po otrzymaniu odpowiedzi klient powinien móc dynamicznie poznać inne interakcje przez linki.
        \end{itemize}
        \item \textbf{Warstwowość} \textemdash \space klient nie wie czy komunikuje się bezpośrednio z serwerem, czy poprzez pośrednika
        (np. proxy) oraz nie wie z czym komunikuje się obsługująca go warstwa.
        \item \textbf{Kod na żądanie (opcjonalnie)} \textemdash \space serwer może przesłać fragment kodu, który zostanie wykonany przez klienta.
    \end{enumerate}
}

\newglossaryentry{uml}{
    name={UML},
    description={(ang. \textit{Unified Modeling Language});
    graficzny język wizualizacji, specyfikowania oraz dokumentowania składników systemów informatycznych. }
}

\newglossaryentry{bpmn}{
    name={BPMN},
    description={(ang. \textit{Business Process Model and Notation});
    standardowa notacja graficzna, która umożliwia szczegółowe przedstawienie i dokumentowanie procesów biznesowych.}
}

\newglossaryentry{infinite-scroll}{
    name={Infinite scroll},
    description={Wzorzec interfejsu użytkownika, w którym kolejne porcje treści są automatycznie doładowywane podczas przewijania strony w dół, zamiast być podzielone na odrębne, ręcznie przełączane strony}
}

\newglossaryentry{redis}{
    name={Redis},
    description={Baza danych typu klucz–wartość wykorzystywana jako szybka warstwa \glslink{cache}{cache}}
}

\newglossaryentry{moscow}{
    name={MoSCoW},
    description={Metoda nadawania priorytetów wymaganiom, wyróżniająca kategorie:
    \emph{Must have}, \emph{Should have}, \emph{Could have} oraz
    \emph{Won't have this time}}
}

\newglossaryentry{cache}{
    name={Cache},
    description={Mechanizm przechowywania danych w celu przyspieszenia ich ponownego odczytu}
}

\newglossaryentry{jvm}{
    name={JVM},
    description={(ang. \textit{Java Virtual Machine}); maszyna wirtualna oraz środowisko do wykonywania kodu bajtowego Javy}
}

\newglossaryentry{spring-framework}{
    name={Spring Framework},
    description={Framework, który służy do tworzenia aplikacji w Javie.
    Zajmuje się między innymi ustawianiem konfiguracji oraz zarządzaniem zależnośćiami,
    np.: wstrzykiwaniem (ang. \textit{Dependency Injection}), odwróceniem kontroli (ang. \textit{Inversion of Control}).
    Oferuje wiele modułów wspierających, które ułatwiają rozwój projektów}
}

\newglossaryentry{convention-over-configuration}{
    name={Convention Over Configuration},
    description={Zasada programowania polegająca na przyjmowaniu domyślnych, bazowych regół, zamiast ręcznego implementowania konfiguracji}
}

\newglossaryentry{dom}{
    name={DOM},
    description={(ang. \textit{Document Object Model}); interfejs reprezentacji stron internetowych jako węzły i obiekty.
    Dzięki temu skrypty, np. napisane w JavaScript, mogą wchodzić w interakcje ze stroną
    (np. modyfikować strukturę, treść lub styl)}
}

\newglossaryentry{query-params}{
    name={Parametry zapytania (query params)},
    description={Pary \texttt{klucz=wartość} przekazywane w części adresu URL po znaku zapytania \texttt{?}, służące m.in. do filtrowania, sortowania, paginacji wyników lub przekazywania dodatkowych opcji żądania do serwera, np. \texttt{?param1=val1\&param2=val2}}
}

\newglossaryentry{spa}{
    name={SPA},
    description={(ang. \textit{Single Page Application}); aplikacja webowa, która zawiera jeden plik HTML.
    Po jednorazowym wczytaniu jej zawartość jest zmieniana bez ponownego przeładowania}
}

\newglossaryentry{wolumen}{
    name={Wolumen},
    description={Zarządzane przez Dockera magazyny danych dla kontenerów, które
    są odizolowane od maszyny hosta.}
}

\newglossaryentry{srp}{
    name={Single Responsibilty Principle},
    description={(tłum. \textit{Zasada Pojedynczej Odpowiedzialności}); zasada należąca do wytycznych \glslink{solid}{SOLID}.
    Zgodnie z nią klasa lub funkcja powinna mieć dokładnie jeden powód do zmiany, czyli być odpowiedzalną za jedną rzecz}
}

\newglossaryentry{solid}{
    name={SOLID},
    description={Mnemonik zdefiniowany przez Roberta C. Martina, zawierający 5 zasad projektowania,
    które mają poprawić elastyczność i prostotę utrzymania oprogramowania.
    \begin{itemize}
        \item \textbf{S} \textendash \space Single Responsibilty Principle (tłum. \textit{Zasada Pojedynczej Odpowiedzialności})
        \item \textbf{O} \textendash \space Open/Closed Principle (tłum. \textit{Zasada Otwarte/Zamknięte})
        \item \textbf{L} \textendash \space Liskov Substitution Principle (tłum. \textit{Zasada Podstawienia Liskov})
        \item \textbf{I} \textendash \space Interaface Segregation Principle (tłum. \textit{Zasada Segregacji Interfejsów})
        \item \textbf{D} \textendash \space Dependency Inversion Principle (tłum. \textit{Zasada Odwrócenia Zależności})
    \end{itemize}
    Definicja została napisana na podstawie treści książki \cite{wzorce-projektowe}
    }
}

\newglossaryentry{ioc}{
    name={IoC},
    description={Inversion of Control (tłum. \textit{Odwrócenie kontroli}); paradygmat programowania, w którym kontrola nad
    zarządzaniem obiektami i zależnościami przekazywana jest do zewnętrznego kontenera lub \glslink{framework}{frameworka}}
}

\newglossaryentry{bean}{
    name={Bean},
    description={To obiekt w \glslink{framework}{frameworku} \glslink{spring-framework}{Spring},
    który jest tworzony i zarządzany przez kontener Spring \glslink{ioc}{IoC}}
}

\newglossaryentry{annotaion}{
    name={Adnotacja},
    description={To sposób dodania metadanych do kodu Java, nie zmieniają bezpośrednio działania programu, ale są wykorzystywane przez kompilator i
    \glslink{framework}{frameworki}.
    Zaczynają się symbolem \textit{@}}
}

\newglossaryentry{react-component}{
    name={Komponent React},
    description={TODO}
}
