%! Author = Mateusz Redosz
%! Date = 20/09/2025

% Słownik pojęć

\newglossaryentry{backend}
{
    name={Backend},
    description={Część aplikacji odpowiedzialna za logikę biznesową, przetwarzanie danych i komunikację z bazą danych. Działa po stronie serwera i obsługuje żądania wysyłane przez frontend}
}

\newglossaryentry{frontend}
{
    name={Frontend},
    description={Warstwa aplikacji odpowiedzialna za interfejs użytkownika oraz interakcję z użytkownikiem. Zazwyczaj tworzona przy użyciu technologii takich jak HTML, CSS i JavaScript}
}

\newglossaryentry{baza-danych}
{
    name={Baza danych},
    description={Zbiór uporządkowanych danych przechowywanych w sposób umożliwiający ich łatwe wyszukiwanie, modyfikowanie i analizowanie. W aplikacjach najczęściej wykorzystywane są relacyjne lub nierelacyjne bazy danych}
}

\newglossaryentry{framework}
{
    name={Framework},
    description={Zestaw narzędzi, bibliotek i struktur wspomagających tworzenie aplikacji. Ułatwia programowanie poprzez dostarczenie gotowych komponentów oraz określenie zasad organizacji kodu}
}

\newglossaryentry{review-kodu}
{
    name={Review kodu},
    description={Proces polegający na wzajemnym przeglądzie kodu źródłowego przez programistów w celu wykrycia błędów, poprawy jakości oraz zwiększenia spójności projektu}
}

\newglossaryentry{jwt}
{
    name={JWT},
    description={Skrót od \textit{JSON Web Token}. Standard służący do bezpiecznego przekazywania informacji między stronami w formacie JSON, często używany w procesach autoryzacji użytkowników}
}

\newglossaryentry{oauth}
{
    name={OAuth},
    description={Standard autoryzacji umożliwiający aplikacjom zewnętrznym uzyskanie dostępu do zasobów użytkownika bez przekazywania jego hasła, często wykorzystywany przy logowaniu za pomocą dostawców takich jak Google czy GitHub}
}

\newglossaryentry{cicd}
{
    name={CI/CD},
    description={Skrót od \textit{Continuous Integration/Continuous Deployment}. Praktyka programistyczna polegająca na automatyzacji procesu budowania, testowania i wdrażania oprogramowania}
}

\newglossaryentry{github}
{
    name={GitHub},
    description={Platforma hostingu repozytoriów \textit{Git} w chmurze, oferująca m.in. pull requesty, system zgłoszeń (issues), zarządzanie wersjami oraz integrację z narzędziami CI/CD}
}

\newglossaryentry{jira}
{
    name={Jira},
    description={Narzędzie firmy Atlassian do zarządzania projektami i zadaniami, szeroko stosowane w metodykach zwinnych. Umożliwia pracę z epikami, taskami, podtaskami oraz tablicami Scrum i Kanban}
}

\newglossaryentry{spring-boot}
{
    name={Spring Boot},
    description={Framework w ekosystemie Spring dla języka Java, ułatwiający tworzenie aplikacji backendowych dzięki automatycznej konfiguracji, wbudowanemu serwerowi aplikacyjnemu oraz zestawowi gotowych starterów}
}

\newglossaryentry{spring-security}
{
    name={Spring Security},
    description={Moduł bezpieczeństwa w ekosystemie Spring odpowiedzialny za uwierzytelnianie i autoryzację użytkowników. Zapewnia obsługę różnych mechanizmów logowania, ról i uprawnień oraz integrację z różnymi źródłami danych}
}

\newglossaryentry{docker}
{
    name={Docker},
    description={Platforma do konteneryzacji aplikacji. Pozwala uruchamiać oprogramowanie w lekkich, izolowanych kontenerach tworzonych na podstawie obrazów, co upraszcza wdrażanie i utrzymanie spójnego środowiska}
}

\newglossaryentry{cors}
{
    name={CORS},
    description={Skrót od \textit{Cross-Origin Resource Sharing}. Mechanizm bezpieczeństwa w przeglądarkach, który kontroluje, czy aplikacja z jednej domeny może wykonywać zapytania HTTP do serwera w innej domenie; konfigurowany za pomocą nagłówków HTTP}
}

\newglossaryentry{http-only-cookie}
{
    name={Ciasteczko HttpOnly},
    description={Ciasteczko HTTP ustawione z flagą \texttt{HttpOnly}, dzięki czemu nie jest dostępne z poziomu JavaScriptu. Zmniejsza ryzyko kradzieży tokenów (np. JWT) w przypadku ataków typu XSS}
}

\newglossaryentry{tailwind-css}
{
    name={Tailwind CSS},
    description={Framework CSS typu \textit{utility-first}, dostarczający gotowe klasy narzędziowe do określania wyglądu (kolory, odstępy, layout). Umożliwia szybkie prototypowanie i spójne stylowanie komponentów bez pisania rozbudowanych arkuszy CSS}
}

\newglossaryentry{prettier}
{
    name={Prettier},
    description={Narzędzie do automatycznego formatowania kodu (np. JavaScript, TypeScript, CSS, HTML). Narzuca spójny styl formatowania, zastępując ręczne ustawianie wcięć i łamań linii}
}

\newglossaryentry{eslint}
{
    name={ESLint},
    description={Statyczny analizator kodu JavaScript/TypeScript. Umożliwia wykrywanie błędów, niespójności stylu oraz potencjalnych problemów poprzez zestaw reguł, które można dostosować do projektu}
}

\newglossaryentry{tanstack-query}
{
    name={TanStack Query},
    description={Biblioteka do obsługi zapytań do serwera i cachowania danych w aplikacjach frontendowych (m.in. React). Ułatwia zarządzanie stanem danych z backendu: pobieranie, odświeżanie, invalidację i obsługę błędów}
}

\newglossaryentry{leaflet}
{
    name={Leaflet},
    description={Lekka biblioteka JavaScript do tworzenia interaktywnych map w przeglądarce, często używana z danymi z OpenStreetMap. Umożliwia dodawanie znaczników, warstw oraz obsługę interakcji użytkownika}
}

\newglossaryentry{e2e-tests}
{
    name={Testy E2E},
    description={Testy \textit{end-to-end}, które sprawdzają działanie systemu od strony użytkownika, przechodząc przez wszystkie warstwy aplikacji (frontend, backend, baza danych) i symulując rzeczywiste scenariusze użycia}
}

\newglossaryentry{dto}
{
    name={DTO},
    description={Skrót od \textit{Data Transfer Object}. Prosty obiekt przenoszący dane między warstwami systemu lub między usługami. Zawiera pola danych, zazwyczaj bez logiki biznesowej}
}

\newglossaryentry{modal}
{
    name={Modal},
    description={Okno dialogowe (okno modalne), które pojawia się na wierzchu interfejsu i blokuje interakcję z resztą aplikacji, dopóki użytkownik go nie zamknie. Służy do prezentowania ważnych komunikatów lub formularzy}
}

\newglossaryentry{skeleton-loader}
{
    name={Skeleton loader},
    description={Wzorzec prezentowania stanu ładowania, w którym zamiast klasycznego „spinnera” wyświetlane są szare prostokąty imitujące docelowy układ treści. Poprawia subiektywne odczucie szybkości działania aplikacji}
}

\newglossaryentry{z-index}
{
    name={z-index},
    description={Właściwość CSS określająca kolejność nakładania się elementów (oś Z). Wyższa wartość powoduje wyświetlenie elementu „nad” elementami o niższych wartościach}
}

\newglossaryentry{intersection-observer}
{
    name={Intersection Observer},
    description={API przeglądarkowe umożliwiające reagowanie na momenty, gdy dany element pojawia się w polu widzenia użytkownika (viewport) lub opuszcza je. Wykorzystywane m.in. do implementacji \gls{infinite-scroll} i lazy loadingu}
}

\newglossaryentry{latex}
{
    name={LaTeX},
    description={System składu tekstu wykorzystywany do przygotowywania profesjonalnych dokumentów technicznych i naukowych. Umożliwia precyzyjne formatowanie, zarządzanie odwołaniami, bibliografią i wzorami matematycznymi}
}

\newglossaryentry{commit}
{
    name={Commit},
    @@ -60,7 +180,7 @@
\newglossaryentry{spot}
{
    name={Spot},
    description={Potencjalne miejsce do latania dronem, zaznaczone na mapie.}
}

\newglossaryentry{sidebar}
@@ -300,7 +420,197 @@
description={(ang. \textit{Business Process Model and Notation});
standardowa notacja graficzna, która umożliwia szczegółowe przedstawienie i dokumentowanie procesów biznesowych.}
}

\newglossaryentry{infinite-scroll}{
    name={Infinite scroll},
    description={Wzorzec interfejsu użytkownika, w którym kolejne porcje treści są automatycznie doładowywane podczas przewijania strony w dół, zamiast być podzielone na odrębne, ręcznie przełączane strony}
}
\newglossaryentry{cdn}
{
    name={CDN},
    description={Skrót od \textit{Content Delivery Network}. Rozproszona sieć serwerów
    służąca do szybkiego dostarczania statycznych zasobów (np. obrazów, arkuszy CSS,
    skryptów JavaScript) z węzłów geograficznie najbliższych użytkownikowi, co zmniejsza
    opóźnienia i odciąża serwer aplikacji}
}

\newglossaryentry{react-maplibre}
{
    name={React-MapLibre},
    description={Otwartoźródłowa biblioteka do renderowania interaktywnych map
    wektorowych w przeglądarce, rozwijana jako niezależna kontynuacja Mapbox GL JS.
    Umożliwia wyświetlanie kafelków mapowych, znaczników i warstw z danymi
    geoprzestrzennymi}
}

\newglossaryentry{websocket}
{
    name={WebSocket},
    description={Protokół komunikacyjny umożliwiający dwukierunkową komunikację
    w czasie rzeczywistym między przeglądarką a serwerem po pojedynczym,
    utrzymywanym połączeniu TCP. Często wykorzystywany m.in. w czatach i aplikacjach
    działających w czasie rzeczywistym.}
}

\newglossaryentry{docker-compose}
{
    name={Docker Compose},
    description={Narzędzie do definiowania i uruchamiania wielokontenerowych aplikacji \gls{docker}
    za pomocą pliku konfiguracyjnego (np. \texttt{docker-compose.yml}). Umożliwia jednoczesne
    uruchamianie powiązanych usług (np. \gls{backend}, baza danych, usługi pomocnicze) jednym poleceniem}
}

\newglossaryentry{pro}
{
    name={PRO},
    description={Przedmiot realizowany na 5. semestrze studiów, prowadzony przez dr. inż. Martę Łabudę. W ramach przedmiotu
    wybrano temat projektu oraz
    wytworzono wstępną dokumentację projektu w tym m.in. wymagania.}
}

\newglossaryentry{prz1}
{
    name={PRZ 1},
    description={Przedmiot realizowany na 6. semestrze studiów, prowadzony w przypadku zespołu projektowego przez mgr. inż. Adama Urbanowicza. W ramach przedmiotu
    wytworzono projekt interfejsu użytkownika.}
}

\newglossaryentry{prz2}
{
    name={PRZ 2},
    description={Przedmiot realizowany na 7. semestrze studiów, prowadzony w przypadku zespołu projektowego przez mgr. inż. Adama Urbanowicza. W ramach przedmiotu
    dokończono prace nad pracą inżynierską. Pan Adam Urbanowicz jako promotor doradzał zespołowi projektowemu.}
}

\newglossaryentry{psem}
{
    name={PSEM},
    description={Przedmiot realizowany na 7. semestrze studiów, prowadzony w przypadku zespołu projektowego przez dr. inż. Marka Bednarczyka. W ramach przedmiotu
    dokończono wytwarzanie dokumentacji.}
}

\newglossaryentry{spa}
{
    name={SPA},
    description={SPA (Single Page Application) to aplikacja webowa, w której cała strona ładuje się raz,
    a późniejsze zmiany widoku odbywają się dynamicznie po stronie przeglądarki bez pełnego przeładowania strony.}
}

\newglossaryentry{routing}
{
    name={routing},
    description={Routing w \gls{spa} to warstwa w kliencie odpowiedzialna za zarządzanie stanem “aktualnej strony” na podstawie URL-a,
    zwykle z wykorzystaniem historii przeglądarki,
    tak aby interfejs reagował na zmianę ścieżki bez przeładowań z serwera.}
}



\newglossaryentry{unit-tests}
{
    name={testy jednostkowe},
    description={Testy sprawdzające poprawność działania pojedynczych, małych fragmentów kodu (np. funkcji, metod, klas) w izolacji od reszty systemu.}
}

\newglossaryentry{jakarta-validation}
{
    name={jakarta validation},
    description={Jakarta Validation to specyfikacja (i zestaw adnotacji, typu @NotNull, @Size itd.) służąca do automatycznego sprawdzania poprawności danych w aplikacjach stworzonych za pomocą Java/Jakarta EE/Spring, np. przy walidacji pól DTO, encji czy parametrów metod.}
}
\newglossaryentry{intellij-idea}
{
    name={IntelliJ IDEA},
    description={Zintegrowane środowisko programistyczne (IDE) firmy JetBrains, szeroko stosowane przy tworzeniu aplikacji backendowych w ekosystemie Spring. Oferuje m.in. podpowiedzi składni, refaktoryzację kodu, debugger oraz integrację z systemami kontroli wersji}
}

\newglossaryentry{dockerfile}
{
    name={Dockerfile},
    description={Plik tekstowy zawierający instrukcje opisujące, jak zbudować obraz Dockera (jakiej podstawy użyć, jakie pliki skopiować, jakie polecenia uruchomić). Na jego podstawie narzędzie Docker tworzy gotowy obraz kontenera}
}

\newglossaryentry{redis}
{
    name={Redis},
    description={Szybka baza danych typu klucz–wartość przechowywana głównie w pamięci operacyjnej. Często wykorzystywana jako pamięć podręczna (cache), magazyn sesji lub prosty mechanizm komunikatów między usługami}
}

\newglossaryentry{gif}
{
    name={GIF},
    description={Format graficzny \textit{Graphics Interchange Format} obsługujący krótkie, zapętlone animacje. W aplikacjach czatowych wykorzystywany do wysyłania „reakcji” w postaci ruchomych obrazków}
}

\newglossaryentry{emoji}{
    name={emoji},
    description={Małe graficzne ikonki używane do wyrażania emocji
    lub pojęć w komunikacji cyfrowej (np. uśmiechnięta buźka, kciuk w górę,
    symbol serca).}
}


\newglossaryentry{url}
{
    name={URL},
    description={Adres zasobu w internecie (ang. \textit{Uniform Resource Locator}), np. adres strony, widoku w aplikacji webowej lub konkretnego posta na forum}
}

\newglossaryentry{slug}
{
    name={Slug},
    description={Przyjazny dla użytkownika fragment adresu URL, zwykle oparty na tytule (np. \texttt{/post/jak-zaczac-latac-dronem}), ułatwiający identyfikację treści i pozycjonowanie w wyszukiwarkach}
}

\newglossaryentry{tinymce}
{
    name={TinyMCE},
    description={Popularny edytor \textit{rich text} osadzany w przeglądarce. Pozwala użytkownikowi formatować tekst (pogrubienia, listy, nagłówki, linki) w sposób przypominający klasyczny edytor tekstu, zapisując wynik zwykle w HTML}
}

\newglossaryentry{rich-text-editor}
{
    name={Rich text editor},
    description={Edytor treści, który zamiast „surowego” tekstu umożliwia stosowanie formatowania (np. pogrubienie, kursywa, listy, nagłówki, linki), dzięki czemu użytkownik może tworzyć czytelne, sformatowane wpisy}
}

\newglossaryentry{tiptap}
{
    name={Tiptap},
    description={Nowoczesny, rozszerzalny edytor \textit{rich text} dla aplikacji webowych oparty na silniku ProseMirror. Umożliwia budowanie rozbudowanych, modularnych edytorów treści, np. do postów na forum}
}

\newglossaryentry{integration-tests}
{
    name={Testy integracyjne},
    description={Testy sprawdzające, czy połączone ze sobą moduły lub usługi współpracują poprawnie — na przykład czy warstwa backendowa poprawnie komunikuje się z bazą danych, warstwą sieciową i pozostałymi komponentami systemu}
}
\newglossaryentry{endpoint}
{
    name={endpoint},
    description={Endpoint to konkretny adres (np. \gls{url}) i metoda protokołu HTTP
    w \gls{api}, które razem odpowiadają za realizację jednej, dobrze zdefiniowanej
    operacji (np. pobrania listy spotów, dodania komentarza, wyszukania spotów).}
}

\newglossaryentry{redux-slice}
{
    name={slice Redux},
    description={Slice Redux to wydzielona część globalnego stanu w \gls{redux},
    wraz z powiązanymi akcjami i reduktorami, odpowiedzialna za jeden obszar domeny
    (np. konto użytkownika, czat, mapę czy listę znajomych).}
}
\newglossaryentry{jsoup}{
    name={jsoup},
    description={Biblioteka \textit{Java} do przetwarzania dokumentów HTML,
    umożliwiająca parsowanie, przeszukiwanie i modyfikowanie struktury dokumentu
    w sposób zbliżony do pracy z DOM-em i selektorami CSS.}
}
\newglossaryentry{paginacja}
{
    name={paginacja},
    description={Mechanizm dzielenia dużych zbiorów danych
    (np. list postów, wyników wyszukiwania, komentarzy)
    na mniejsze strony, które są pobierane i wyświetlane stopniowo,
    zamiast ładowania wszystkich elementów jednocześnie.}
}
