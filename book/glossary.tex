%! Author = Mateusz Redosz
%! Date = 20/09/2025

% Słownik pojęć

\newglossaryentry{backend}
{
    name={Backend},
    description={Część aplikacji odpowiedzialna za logikę biznesową, przetwarzanie danych i komunikację z bazą danych. Działa po stronie serwera i obsługuje żądania wysyłane przez frontend}
}

\newglossaryentry{frontend}
{
    name={Frontend},
    description={Warstwa aplikacji odpowiedzialna za interfejs użytkownika oraz interakcję z użytkownikiem. Zazwyczaj tworzona przy użyciu technologii takich jak HTML, CSS i JavaScript}
}

\newglossaryentry{baza-danych}
{
    name={Baza danych},
    description={Zbiór uporządkowanych danych przechowywanych w sposób umożliwiający ich łatwe wyszukiwanie, modyfikowanie i analizowanie. W aplikacjach najczęściej wykorzystywane są relacyjne lub nierelacyjne bazy danych}
}

\newglossaryentry{framework}
{
    name={Framework},
    description={Zestaw narzędzi, bibliotek i struktur wspomagających tworzenie aplikacji. Ułatwia programowanie poprzez dostarczenie gotowych komponentów oraz określenie zasad organizacji kodu}
}

\newglossaryentry{review-kodu}
{
    name={Review kodu},
    description={Proces polegający na wzajemnym przeglądzie kodu źródłowego przez programistów w celu wykrycia błędów, poprawy jakości oraz zwiększenia spójności projektu}
}

\newglossaryentry{jwt}
{
    name={JWT},
    description={Skrót od \textit{JSON Web Token}. Standard służący do bezpiecznego przekazywania informacji między stronami w formacie JSON, często używany w procesach autoryzacji użytkowników}
}

\newglossaryentry{cicd}
{
    name={CI/CD},
    description={Skrót od \textit{Continuous Integration/Continuous Deployment}. Praktyka programistyczna polegająca na automatyzacji procesu budowania, testowania i wdrażania oprogramowania}
}

\newglossaryentry{commit}
{
    name={Commit},
    description={Zapis zmian w repozytorium systemu kontroli wersji, który dokumentuje stan projektu w określonym momencie}
}

\newglossaryentry{push}
{
    name={Push},
    description={Operacja w systemie kontroli wersji polegająca na wysłaniu lokalnych zmian (commitów) do zdalnego repozytorium}
}

\newglossaryentry{spot}
{
    name={Spot},
    description={Spotkanie zespołu projektowego, zazwyczaj krótkie i regularne, służące omówieniu postępów prac, problemów oraz planów na najbliższy okres}
}

\newglossaryentry{sidebar}
{
    name={Sidebar},
    description={Boczny panel w interfejsie użytkownika, zawierający menu nawigacyjne lub dodatkowe opcje funkcjonalne aplikacji}
}

\newglossaryentry{design}
{
    name={Design},
    description={Etap lub proces projektowania wyglądu i funkcjonalności aplikacji, obejmujący zarówno aspekty wizualne, jak i użytkowe (UX/UI)}
}

\newglossaryentry{folder-by-type}
{
    name={Folder by type},
    description={Sposób organizowania struktury katalogów w projekcie, w którym pliki są grupowane według rodzaju (typu) zasobu, a nie według funkcjonalności. Na przykład wszystkie komponenty trafiają do jednego folderu, wszystkie style do innego itd}
}

\newglossaryentry{biblioteka}
{
    name={Biblioteka},
    description={Zewnętrzny lub wewnętrzny zestaw gotowych funkcji, klas, komponentów lub modułów, który można wielokrotnie wykorzystywać w projekcie zamiast pisać wszystko od zera}
}

\newglossaryentry{protected-route}
{
    name={Protected route},
    description={Trasa w aplikacji, do której dostęp jest ograniczony, zwykle tylko dla zalogowanych użytkowników lub użytkowników z odpowiednimi uprawnieniami. Jeżeli użytkownik nie spełnia warunków, jest przekierowywany (np. na stronę główną)}
}

\newglossaryentry{wzorzec}
{
    name={Wzorzec},
    description={Powtarzalne, sprawdzone rozwiązanie typowego problemu projektowego lub architektonicznego. Wzorzec opisuje \emph{jak} coś organizować lub implementować, żeby było czytelne, skalowalne i łatwe w utrzymaniu}
}

\newglossaryentry{DAD_LLC}{
    name={Disciplined Agile Delivery - Lean Life Cycle},
    description={Kanbanowy tryb pracy bez sprintów: zespół pobiera małe zadania z tablicy, pilnuje limitów pracy w toku i często wdraża małe zmiany.
    Planowanie jest lekkie i na bieżąco; priorytet mają rzeczy o najwyższej wartości.
    Skupiamy się na płynności—skraca się czas przejścia zadań, usuwa blokady, usprawnia proces małymi krokami.
    Jakość jest wbudowana: code review, testy automatyczne i prosta definicja „gotowe”}
}

\newglossaryentry{folder-by-type}
{
    name={Folder by type},
    description={Sposób organizowania struktury katalogów w projekcie, w którym pliki są grupowane według rodzaju (typu) zasobu, a nie według funkcjonalności. Na przykład wszystkie komponenty trafiają do jednego folderu, wszystkie style do innego itd.}
}

\newglossaryentry{biblioteka}
{
    name={Biblioteka},
    description={Zewnętrzny lub wewnętrzny zestaw gotowych funkcji, klas, komponentów lub modułów, który można wielokrotnie wykorzystywać w projekcie zamiast pisać wszystko od zera.}
}

\newglossaryentry{protected-route}
{
    name={Protected route},
    description={Trasa w aplikacji, do której dostęp jest ograniczony, zwykle tylko dla zalogowanych użytkowników lub użytkowników z odpowiednimi uprawnieniami. Jeżeli użytkownik nie spełnia warunków, jest przekierowywany (np. na stronę główną.}
}

\newglossaryentry{wzorzec}
{
    name={Wzorzec},
    description={Powtarzalne, sprawdzone rozwiązanie typowego problemu projektowego lub architektonicznego. Wzorzec opisuje \emph{jak} coś organizować lub implementować, żeby było czytelne, skalowalne i łatwe w utrzymaniu.}
}

\newglossaryentry{stan}{
    name={Stan},
    description={Aktualny zestaw danych przechowywanych przez aplikację lub komponent, na podstawie którego renderowany jest interfejs użytkownika. Stan może być lokalny (utrzymywany w pojedynczym komponencie) albo globalny (wspólny dla wielu komponentów).}
}

\newglossaryentry{ui}{
    name={UI},
    description={Interfejs użytkownika (ang. \textit{User Interface}); warstwa prezentacji odpowiedzialna za sposób wyświetlania danych oraz interakcji użytkownika z aplikacją.}
}

\newglossaryentry{hook}{
    name={Hook (React)},
    description={Prosta funkcja w React, która „dodaje” możliwości do elementu interfejsu — np. pozwala mu coś zapamiętać (stan) albo zrobić coś po zmianie/załadowaniu. Większość hooków zaczyna się od \texttt{use...} (np. \texttt{useState}, \texttt{useEffect}).}
}

\newglossaryentry{css}{
    name={CSS},
    description={Kaskadowe arkusze stylów (Cascading Style Sheets) — język opisu prezentacji dokumentów (np. HTML). Definiuje wygląd interfejsu: układ, kolory, typografię, odstępy, animacje i zachowania responsywne, oddzielając warstwę treści od warstwy prezentacji.}
}

\newglossaryentry{responsywnosc}{
    name={Responsywność},
    description={Określenie związane z projektowaniem responsywnym (Responsive Web Design, RWD), czyli dostosowywaniem interfejsu do różnych rozmiarów i parametrów ekranów. Obejmuje m.in. elastyczne siatki, grafiki i \gls{media-queries}, tak aby układ i czytelność były zachowane na telefonach, tabletach i desktopach.}
}

\newglossaryentry{props}{
    name={Props},
    description={Właściwości przekazywane do komponentu React przez komponent nadrzędny; służą do konfiguracji i przekazywania danych. Powinny być traktowane jako tylko do odczytu (read-only) wewnątrz komponentu potomnego.}
}

\newglossaryentry{react}{
    name={React},
    description={Biblioteka JavaScript do budowy interfejsów użytkownika w oparciu o komponenty deklaratywne i wirtualny DOM. Zapewnia jednokierunkowy przepływ danych oraz zarządzanie stanem komponentów.}
}

\newglossaryentry{type-script}{
    name={TypeScript},
    description={Rozszerzenie do języka JavaScript dodający statyczne typowanie, interfejsy i narzędzia do większych projektów. Kompiluje się do czystego JavaScript, ułatwiając wykrywanie błędów w czasie kompilacji i refaktoryzację.}
}

\newglossaryentry{redux}{
    name={Redux},
    description={Biblioteka do przewidywalnego zarządzania stanem aplikacji. Opiera się na jednym \emph{store}, akcjach i czystych \emph{reducerach}, promuje niemutowalność i jednokierunkowy przepływ danych. Często używana z Reactem, ale niezależna od niego.}
}

\newglossaryentry{media-queries}{
    name={Media queries},
    description={Konstrukcja CSS pozwalająca stosować reguły stylów w zależności od cech urządzenia/okna (np. szerokości ekranu, orientacji, preferencji użytkownika). Podstawa responsywnego projektowania (\emph{responsive design}).}
}

