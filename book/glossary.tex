%! Author = Mateusz Redosz
%! Date = 20/09/2025

% Słownik pojęć

\newglossaryentry{backend}
{
    name={Backend},
    description={Część aplikacji odpowiedzialna za logikę biznesową, przetwarzanie danych i komunikację z bazą danych. Działa po stronie serwera i obsługuje żądania wysyłane przez frontend}
}

\newglossaryentry{frontend}
{
    name={Frontend},
    description={Warstwa aplikacji odpowiedzialna za interfejs użytkownika oraz interakcję z użytkownikiem. Zazwyczaj tworzona przy użyciu technologii takich jak HTML, CSS i JavaScript}
}

\newglossaryentry{baza-danych}
{
    name={Baza danych},
    description={Zbiór uporządkowanych danych przechowywanych w sposób umożliwiający ich łatwe wyszukiwanie, modyfikowanie i analizowanie. W aplikacjach najczęściej wykorzystywane są relacyjne lub nierelacyjne bazy danych}
}

\newglossaryentry{framework}
{
    name={Framework},
    description={Zestaw narzędzi, bibliotek i struktur wspomagających tworzenie aplikacji. Ułatwia programowanie poprzez dostarczenie gotowych komponentów oraz określenie zasad organizacji kodu}
}

\newglossaryentry{review-kodu}
{
    name={Review kodu},
    description={Proces polegający na wzajemnym przeglądzie kodu źródłowego przez programistów w celu wykrycia błędów, poprawy jakości oraz zwiększenia spójności projektu}
}

\newglossaryentry{jwt}
{
    name={JWT},
    description={Skrót od \textit{JSON Web Token}. Standard służący do bezpiecznego przekazywania informacji między stronami w formacie JSON, często używany w procesach autoryzacji użytkowników}
}

\newglossaryentry{oauth}
{
    name={OAuth},
    description={Standard autoryzacji umożliwiający aplikacjom zewnętrznym uzyskanie dostępu do zasobów użytkownika bez przekazywania jego hasła, często wykorzystywany przy logowaniu za pomocą dostawców takich jak Google czy GitHub}
}

\newglossaryentry{cicd}
{
    name={CI/CD},
    description={Skrót od \textit{Continuous Integration/Continuous Deployment}. Praktyka programistyczna polegająca na automatyzacji procesu budowania, testowania i wdrażania oprogramowania}
}

\newglossaryentry{github}
{
    name={GitHub},
    description={Platforma hostingu repozytoriów \textit{Git} w chmurze, oferująca m.in. pull requesty, system zgłoszeń (issues), zarządzanie wersjami oraz integrację z narzędziami CI/CD}
}

\newglossaryentry{jira}
{
    name={Jira},
    description={Narzędzie firmy Atlassian do zarządzania projektami i zadaniami, szeroko stosowane w metodykach zwinnych. Umożliwia pracę z epikami, taskami, podtaskami oraz tablicami Scrum i Kanban}
}

\newglossaryentry{spring-boot}
{
    name={Spring Boot},
    description={Framework w ekosystemie Spring dla języka Java, ułatwiający tworzenie aplikacji backendowych dzięki automatycznej konfiguracji, wbudowanemu serwerowi aplikacyjnemu oraz zestawowi gotowych starterów}
}

\newglossaryentry{spring-security}
{
    name={Spring Security},
    description={Moduł bezpieczeństwa w ekosystemie Spring odpowiedzialny za uwierzytelnianie i autoryzację użytkowników. Zapewnia obsługę różnych mechanizmów logowania, ról i uprawnień oraz integrację z różnymi źródłami danych}
}

\newglossaryentry{cors}
{
    name={CORS},
    description={Skrót od \textit{Cross-Origin Resource Sharing}. Mechanizm bezpieczeństwa w przeglądarkach, który kontroluje, czy aplikacja z jednej domeny może wykonywać zapytania HTTP do serwera w innej domenie; konfigurowany za pomocą nagłówków HTTP}
}

\newglossaryentry{http-only-cookie}
{
    name={Ciasteczko HttpOnly},
    description={Ciasteczko HTTP ustawione z flagą \texttt{HttpOnly}, dzięki czemu nie jest dostępne z poziomu JavaScriptu. Zmniejsza ryzyko kradzieży tokenów (np. JWT) w przypadku ataków typu XSS}
}

\newglossaryentry{tailwind-css}
{
    name={Tailwind CSS},
    description={Framework CSS typu \textit{utility-first}, dostarczający gotowe klasy narzędziowe do określania wyglądu (kolory, odstępy, layout). Umożliwia szybkie prototypowanie i spójne stylowanie komponentów bez pisania rozbudowanych arkuszy CSS}
}

\newglossaryentry{prettier}
{
    name={Prettier},
    description={Narzędzie do automatycznego formatowania kodu (np. JavaScript, TypeScript, CSS, HTML). Narzuca spójny styl formatowania, zastępując ręczne ustawianie wcięć i łamań linii}
}

\newglossaryentry{eslint}
{
    name={ESLint},
    description={Statyczny analizator kodu JavaScript/TypeScript. Umożliwia wykrywanie błędów, niespójności stylu oraz potencjalnych problemów poprzez zestaw reguł, które można dostosować do projektu}
}

\newglossaryentry{tanstack-query}
{
    name={TanStack Query},
    description={Biblioteka do obsługi zapytań do serwera i cachowania danych w aplikacjach frontendowych (m.in. React). Ułatwia zarządzanie stanem danych z backendu: pobieranie, odświeżanie i obsługę błędów}
}

\newglossaryentry{leaflet}
{
    name={Leaflet},
    description={Lekka biblioteka JavaScript do tworzenia interaktywnych map w przeglądarce, często używana z danymi z OpenStreetMap. Umożliwia dodawanie znaczników, warstw oraz obsługę interakcji użytkownika}
}

\newglossaryentry{e2e-tests}
{
    name={Testy E2E},
    description={Testy \textit{end-to-end}, które sprawdzają działanie systemu od strony użytkownika, przechodząc przez wszystkie warstwy aplikacji (frontend, backend, baza danych) i symulując rzeczywiste scenariusze użycia}
}

\newglossaryentry{dto}
{
    name={DTO},
    description={Skrót od \textit{Data Transfer Object}. Obiekt przenoszący dane między warstwami systemu lub między usługami. Zawiera pola danych, zazwyczaj bez logiki biznesowej}
}

\newglossaryentry{modal}
{
    name={Modal},
    description={Okno dialogowe (okno modalne), które pojawia się na wierzchu interfejsu i blokuje interakcję z resztą aplikacji, dopóki użytkownik go nie zamknie. Służy do prezentowania ważnych komunikatów lub formularzy}
}

\newglossaryentry{skeleton-loader}
{
    name={Skeleton loader},
    description={Wzorzec prezentowania stanu ładowania, w którym zamiast klasycznego „spinnera” wyświetlane są szare prostokąty imitujące docelowy układ treści. Poprawia subiektywne odczucie szybkości działania aplikacji}
}

\newglossaryentry{z-index}
{
    name={z-index},
    description={Właściwość CSS określająca kolejność nakładania się elementów (oś Z). Wyższa wartość powoduje wyświetlenie elementu „nad” elementami o niższych wartościach}
}

\newglossaryentry{intersection-observer}
{
    name={Intersection Observer},
    description={API przeglądarkowe umożliwiające reagowanie na momenty, gdy dany element pojawia się w polu widzenia użytkownika (viewport) lub opuszcza je. Wykorzystywane m.in. do implementacji \gls{infinite-scroll} i lazy loadingu}
}

\newglossaryentry{latex}
{
    name={LaTeX},
    description={System składu tekstu wykorzystywany do przygotowywania profesjonalnych dokumentów technicznych i naukowych. Umożliwia precyzyjne formatowanie, zarządzanie odwołaniami, bibliografią i wzorami matematycznymi}
}

\newglossaryentry{commit}
{
    name={Commit},
    description={Zapis zmian w repozytorium systemu kontroli wersji, który dokumentuje stan projektu w określonym momencie}
}

\newglossaryentry{push}
{
    name={Push},
    description={Operacja w systemie kontroli wersji polegająca na wysłaniu lokalnych zmian (commitów) do zdalnego repozytorium}
}

\newglossaryentry{spot}
{
    name={Spot},
    description={Potencjalne miejsce do latania dronem, zaznaczone na mapie.}
}

\newglossaryentry{sidebar}
{
    name={Sidebar},
    description={Boczny panel w interfejsie użytkownika, zawierający menu nawigacyjne lub dodatkowe opcje funkcjonalne aplikacji}
}

\newglossaryentry{design}
{
    name={Design},
    description={Etap lub proces projektowania wyglądu i funkcjonalności aplikacji, obejmujący zarówno aspekty wizualne, jak i użytkowe (UX/UI)}
}

\newglossaryentry{folder-by-type}
{
    name={Folder by type},
    description={Sposób organizowania struktury katalogów w projekcie, w którym pliki są grupowane według rodzaju (typu) zasobu, a nie według funkcjonalności. Na przykład wszystkie komponenty trafiają do jednego folderu, wszystkie style do innego itd.}
}

\newglossaryentry{biblioteka}
{
    name={Biblioteka},
    description={Zewnętrzny lub wewnętrzny zestaw gotowych funkcji, klas, komponentów lub modułów, który można wielokrotnie wykorzystywać w projekcie zamiast pisać wszystko od zera}
}

\newglossaryentry{protected-route}
{
    name={Protected route},
    description={Trasa w aplikacji, do której dostęp jest ograniczony, zwykle tylko dla zalogowanych użytkowników lub użytkowników z odpowiednimi uprawnieniami. Jeżeli użytkownik nie spełnia warunków, jest przekierowywany (np. na stronę główną)}
}

\newglossaryentry{wzorzec}
{
    name={Wzorzec},
    description={Powtarzalne, sprawdzone rozwiązanie typowego problemu projektowego lub architektonicznego. Wzorzec opisuje \emph{jak} coś organizować lub implementować, żeby było czytelne, skalowalne i łatwe w utrzymaniu}
}

\newglossaryentry{DAD_LLC}
{
    name={Disciplined Agile Delivery - Lean Life Cycle},
    description={
        Disciplined Agile Delivery w wariancie Lean Life Cycle to sposób prowadzenia projektu,
        który łączy elastyczność Agile z przewidywalnością Waterfalla,
        ale bez stałych sprintów — praca toczy się w ciągłym przepływie.
        Na starcie zakłada mocniejszą fazę przygotowawczą: doprecyzowanie zakresu,
        szkic architektury, identyfikację ryzyk i kryteria jakości.
        W realizacji następuje ciągłe doprecyzowywanie wymagań
        i backlogu, oparte na regularnym feedbacku udziałowców.
        Całość opiera się na praktykach Lean oraz lekkim governance:
        code review i regularnych przeglądach postępów.
    }
}

\newglossaryentry{operator-drona}
{
    name={Operator drona},
    description={Osoba lub podmiot będący właścicielem floty dronów. Może posiadać jeden lub wiele dronów. Droniarz rekreacyjny jest zazwyczaj jednocześnie operatorem floty oraz pilotem.}
}

\newglossaryentry{pilot-drona}
{
    name={Pilot drona},
    description={Osoba posiadająca uprawnienia do pilotowania drona (jeżeli są wymagane) i wykonująca loty dronem. Droniarz rekreacyjny jest zazwyczaj jednocześnie pilotem oraz operatorem floty.}
}

\newglossaryentry{droniarz}
{
    name={Droniarz},
    plural={droniarze},
    description={Potoczne określenie osoby, która jest jednocześnie pilotem oraz operatorem drona. Zwykle entuzjasta dronów.}
}

\newglossaryentry{PANSA}
{
    name={PANSA},
    description={
        Polish Air Navigation Services Agency, pol. Polska Agencja Żeglugi Powietrznej.
        Instytucja ta zapewnia m.in. mapę z zaznaczonymi strefami lotów.
        Każda strefa ma swoje właściwości prawne.
    }
}

\newglossaryentry{pilot-fpv}
{
    name={Pilot FPV},
    description={Pilot drona latający w trybie \textit{First Person View} (FPV),
    korzystający z gogli przekazujących obraz z kamery pokładowej.}
}

\newglossaryentry{droniarz-fpv}
{
    name={Droniarz FPV},
    description={\glsentrydesc{pilot-fpv}}
}

\newglossaryentry{droniarz-foto-video}
{
    name={Droniarz foto/video},
    description={Pilot wykorzystujący drony fotograficzne/filmowe do rejestracji materiałów wizualnych
    (zdjęcia, wideo), zwykle z naciskiem na stabilizację i jakość obrazu.},
    user1={droniarzem foto/video}
}

\newglossaryentry{droniarz-foto}
{
    name={Droniarz foto},
    description={\glsentrydesc{droniarz-foto-video}}
}

\newglossaryentry{droniarz-fotograf}
{
    name={Droniarz fotograf},
    description={\glsentrydesc{droniarz-foto-video}}
}

\newglossaryentry{pilot-foto}
{
    name={Pilot foto},
    description={\glsentrydesc{droniarz-foto-video}}
}

\newglossaryentry{azure-blob-storage}
{
    name={Azure Blob Storage},
    description={Usługa magazynu obiektowego w chmurze Microsoft Azure do przechowywania
    nieustrukturyzowanych danych (\textit{blobs}) takich jak obrazy, wideo i pliki.
    Udostępnia kontenery, warstwy dostępu, wersjonowanie oraz tokeny SAS; często używana
    do hostowania multimediów w aplikacjach webowych.}
}

\newglossaryentry{stan}{
    name={Stan},
    description={Aktualny zestaw danych przechowywanych przez aplikację lub komponent, na podstawie którego renderowany jest interfejs użytkownika. Stan może być lokalny (utrzymywany w pojedynczym komponencie) albo globalny (wspólny dla wielu komponentów).}
}

\newglossaryentry{ui}{
    name={UI},
    description={Interfejs użytkownika (ang. \textit{User Interface}); warstwa prezentacji odpowiedzialna za sposób wyświetlania danych oraz interakcji użytkownika z aplikacją.}
}

\newglossaryentry{hook}{
    name={Hook (React)},
    description={Prosta funkcja w React, która „dodaje” możliwości do elementu interfejsu — np. pozwala mu coś zapamiętać (stan) albo zrobić coś po zmianie/załadowaniu. Wszystkie hooki zaczynają się od \texttt{use...} (np. \texttt{useState}, \texttt{useEffect}).}
}

\newglossaryentry{css}{
    name={CSS},
    description={Kaskadowe arkusze stylów (Cascading Style Sheets) — język opisu prezentacji dokumentów (np. HTML). Definiuje wygląd interfejsu: układ, kolory, typografię, odstępy, animacje i zachowania responsywne, oddzielając warstwę treści od warstwy prezentacji.}
}

\newglossaryentry{responsywnosc}{
    name={Responsywność},
    description={Określenie związane z projektowaniem responsywnym (Responsive Web Design, RWD), czyli dostosowywaniem interfejsu do różnych rozmiarów i parametrów ekranów. Obejmuje m.in. elastyczne siatki, grafiki i \gls{media-queries}, tak aby układ i czytelność były zachowane na telefonach, tabletach i desktopach.}
}

\newglossaryentry{props}{
    name={Props},
    description={Właściwości przekazywane do komponentu React przez komponent nadrzędny; służą do konfiguracji i przekazywania danych. Powinny być traktowane jako tylko do odczytu (read-only) wewnątrz komponentu potomnego.}
}

\newglossaryentry{react}{
    name={React},
    description={Biblioteka JavaScript do budowy interfejsów użytkownika w oparciu o komponenty deklaratywne i wirtualny DOM. Zapewnia jednokierunkowy przepływ danych oraz zarządzanie stanem komponentów.}
}

\newglossaryentry{type-script}{
    name={TypeScript},
    description={Rozszerzenie do języka JavaScript dodający statyczne typowanie, interfejsy i narzędzia do większych projektów. Kompiluje się do czystego JavaScript, ułatwiając wykrywanie błędów w czasie kompilacji i refaktoryzację.}
}

\newglossaryentry{redux}{
    name={Redux},
    description={Biblioteka do przewidywalnego zarządzania stanem aplikacji. Opiera się na jednym \emph{store}, akcjach i czystych \emph{reducerach}, promuje niemutowalność i jednokierunkowy przepływ danych. Często używana z Reactem, ale niezależna od niego.}
}

\newglossaryentry{media-queries}{
    name={Media queries},
    description={Konstrukcja CSS pozwalająca stosować reguły stylów w zależności od cech urządzenia/okna (np. szerokości ekranu, orientacji, preferencji użytkownika). Podstawa responsywnego projektowania (\emph{responsive design}).}
}

\newglossaryentry{backlog}{
    name={Backlog},
    description={Lista zadań, które należy wykonać w ramach projektu, używana w metodykach zwinnych.}
}

\newglossaryentry{tablica_kanban}{
    name={Tablica Kanban},
    description={Narzędzie do zarządzania przepływem pracy, które pomaga zespołom śledzić zadania oraz ich postępy.
    Składa się z kolumn reprezentujących stan etapu prac, na przykład „Do zrobienia” lub „W trakcie”.}
}

\newglossaryentry{ide}{
    name={IDE},
    description={(ang. \textit{integrated development environment}); zintegrowane środowisko programistyczne, służące do
    tworzenia, modyfikowania, testowania i konserwacji oprogramowania}
}

\newglossaryentry{api}{
    name={API},
    description={(ang. \textit{application programming interface}); zbiór reguł i operacji do komunikacji z oprogramowaniem.}
}

\longnewglossaryentry{rest_api}{
    name={REST API}
}{
    Architektura budowania usług sieciowych komunikujących się poprzez metody protokołu HTTP (GET, PUT, POST, DELETE, PATCH).
    Wymiana danych występuje często w formacie JSON lub XML.
    \\REST API musi spełniać następujące reguły:
    \begin{enumerate}[nosep]
        \item \textbf{Rozdzielenie klient-serwer} \textemdash \space klient i serwer są od siebie niezależne, komunikują się poprzez interfejs.
        \item \textbf{Bezstanowość} \textemdash \space każde żądanie przez klienta zawiera wszystkie informacje niezbędne do jego obsłużenia.
        Po otrzymaniu żądania serwer nie przechowuje o nim żadnych informacji.
        \item \textbf{Buforowalność (cache)} \textemdash \space odpowiedzi z API powinny informować, czy dane można cache’ować.
        Jeśli tak, to przy kolejnym żądaniu mogą być zwrócone z cache’a.
        \item \textbf{Jednolity interfejs}:
        \begin{itemize}[nosep]
            \item \textbf{Identyfikacja zasobów} \textemdash \space każdy zasób musi być jednoznacznie zidentyfikowany w interakcji klient-serwer.
            \item \textbf{Manipulacja zasobów poprzez reprezentację} \textemdash \space po otrzymaniu reprezentacji klient może zmienić stan zasobu przesyłając zmodyfikowaną reprezentację.
            \item \textbf{Samoopisujące się wiadomości} \textemdash \space każde żądanie i odpowiedź powinny zawierać informacje do jego poprawnego przetworzenia.
            \item \textbf{Hypermedia jako silnik stanu aplikacji (HATEOAS)} \textemdash \space po otrzymaniu odpowiedzi klient powinien móc dynamicznie poznać inne interakcje przez linki.
        \end{itemize}
        \item \textbf{Warstwowość} \textemdash \space klient nie wie, czy komunikuje się bezpośrednio z serwerem, czy poprzez pośrednika
        (np. proxy) oraz nie wie, z czym komunikuje się obsługująca go warstwa.
        \item \textbf{Kod na żądanie (opcjonalnie)} \textemdash \space serwer może przesłać fragment kodu, który zostanie wykonany przez klienta.
    \end{enumerate}
}

\newglossaryentry{uml}{
    name={UML},
    description={(ang. \textit{Unified Modeling Language});
    graficzny język wizualizacji, specyfikowania oraz dokumentowania składników systemów informatycznych.}
}

\newglossaryentry{bpmn}{
    name={BPMN},
    description={(ang. \textit{Business Process Model and Notation});
    standardowa notacja graficzna, która umożliwia szczegółowe przedstawienie i dokumentowanie procesów biznesowych.}
}

\newglossaryentry{infinite-scroll}{
    name={Infinite scroll},
    description={Wzorzec interfejsu użytkownika, w którym kolejne porcje treści są automatycznie doładowywane podczas przewijania strony w dół, zamiast być podzielone na odrębne, ręcznie przełączane strony}
}

\newglossaryentry{cdn}
{
    name={CDN},
    description={Skrót od \textit{Content Delivery Network}. Rozproszona sieć serwerów
    służąca do szybkiego dostarczania statycznych zasobów (np. obrazów, arkuszy CSS,
    skryptów JavaScript) z węzłów geograficznie najbliższych użytkownikowi, co zmniejsza
    opóźnienia i odciąża serwer aplikacji}
}

\newglossaryentry{react-maplibre}
{
    name={React-MapLibre},
    description={Otwartoźródłowa biblioteka do renderowania interaktywnych map
    wektorowych w przeglądarce, rozwijana jako niezależna kontynuacja Mapbox GL JS.
    Umożliwia wyświetlanie kafelków mapowych, znaczników i warstw z danymi
    geoprzestrzennymi}
}

\newglossaryentry{docker-compose}
{
    name={Docker Compose},
    description={Narzędzie do definiowania i uruchamiania aplikacji składających się z wielu kontenerów Docker.
    Konfiguracja usług (m.in. obrazy, porty, wolumeny i sieci) opisywana jest w pliku YAML (np. \texttt{docker-compose.yml})
    i umożliwia jednoczesne uruchamianie powiązanych usług (np. \gls{backend}, baza danych, usługi pomocnicze) jednym poleceniem}
}

\newglossaryentry{pro}
{
    name={PRO},
    description={Przedmiot realizowany na 5. semestrze studiów, prowadzony przez dr. inż. Martę Łabudę. W ramach przedmiotu
    wybrano temat projektu oraz wytworzono wstępną dokumentację projektu, w tym m.in. wymagania.}
}

\newglossaryentry{prz1}
{
    name={PRZ 1},
    description={Przedmiot realizowany na 6. semestrze studiów, prowadzony w przypadku zespołu projektowego przez mgr. inż. Adama Urbanowicza. W ramach przedmiotu
    wytworzono projekt interfejsu użytkownika.}
}

\newglossaryentry{prz2}
{
    name={PRZ 2},
    description={Przedmiot realizowany na 7. semestrze studiów, prowadzony w przypadku zespołu projektowego przez mgr. inż. Adama Urbanowicza. W ramach przedmiotu
    dokończono prace nad pracą inżynierską. Pan Adam Urbanowicz jako promotor doradzał zespołowi projektowemu.}
}

\newglossaryentry{psem}
{
    name={PSEM},
    description={Przedmiot realizowany na 7. semestrze studiów, prowadzony w przypadku zespołu projektowego przez dr. inż. Marka Bednarczyka. W ramach przedmiotu
    dokończono wytwarzanie dokumentacji.}
}

\newglossaryentry{spa}
{
    name={SPA},
    description={Single Page Application to aplikacja webowa, w której cała strona ładuje się raz,
    a późniejsze zmiany widoku odbywają się dynamicznie po stronie przeglądarki bez pełnego przeładowania strony.}
}

\newglossaryentry{routing}
{
    name={routing},
    description={Routing w \gls{spa} to warstwa w kliencie odpowiedzialna za zarządzanie stanem “aktualnej strony” na podstawie URL-a,
    zwykle z wykorzystaniem historii przeglądarki,
    tak aby interfejs reagował na zmianę ścieżki bez przeładowań z serwera.}
}

\newglossaryentry{unit-tests}
{
    name={testy jednostkowe},
    description={Testy sprawdzające poprawność działania pojedynczych, małych fragmentów kodu (np. funkcji, metod, klas) w izolacji od reszty systemu.}
}

\newglossaryentry{jakarta-validation}
{
    name={jakarta validation},
    description={Jakarta Validation to specyfikacja (i zestaw adnotacji, typu @NotNull, @Size itd.) służąca do automatycznego sprawdzania poprawności danych w aplikacjach stworzonych za pomocą Java/Jakarta EE/Spring, np. przy walidacji pól DTO, encji czy parametrów metod.}
}

\newglossaryentry{intellij-idea}
{
    name={IntelliJ IDEA},
    description={Zintegrowane środowisko programistyczne (IDE) firmy JetBrains, szeroko stosowane przy tworzeniu aplikacji backendowych w ekosystemie Spring. Oferuje m.in. podpowiedzi składni, refaktoryzację kodu, debugger oraz integrację z systemami kontroli wersji}
}

\newglossaryentry{dockerfile}
{
    name={Dockerfile},
    description={Plik tekstowy zawierający instrukcje opisujące, jak zbudować obraz Dockera (jakiej podstawy użyć, jakie pliki skopiować, jakie polecenia uruchomić). Na jego podstawie Docker tworzy gotowy obraz kontenera}
}

\newglossaryentry{redis}
{
    name={Redis},
    description={Szybka baza danych typu klucz–wartość przechowywana głównie w pamięci operacyjnej. Często wykorzystywana jako pamięć podręczna (cache), magazyn sesji lub prosty mechanizm komunikatów między usługami}
}

\newglossaryentry{gif}
{
    name={GIF},
    description={Format graficzny \textit{Graphics Interchange Format} obsługujący krótkie, zapętlone animacje. W aplikacjach czatowych wykorzystywany do wysyłania „reakcji” w postaci ruchomych obrazków}
}

\newglossaryentry{emoji}{
    name={emoji},
    description={Małe graficzne ikonki używane do wyrażania emocji
    lub pojęć w komunikacji cyfrowej (np. uśmiechnięta buźka, kciuk w górę,
    symbol serca).}
}

\newglossaryentry{url}
{
    name={URL},
    description={Adres zasobu w internecie (ang. \textit{Uniform Resource Locator}), np. adres strony, widoku w aplikacji webowej lub konkretnego posta na forum}
}

\newglossaryentry{slug}
{
    name={Slug},
    description={Przyjazny dla użytkownika fragment adresu URL, zwykle oparty na tytule (np. \texttt{/post/jak-zaczac-latac-dronem}), ułatwiający identyfikację treści i pozycjonowanie w wyszukiwarkach}
}

\newglossaryentry{tinymce}
{
    name={TinyMCE},
    description={Popularny edytor \textit{rich text} osadzany w przeglądarce. Pozwala użytkownikowi formatować tekst (pogrubienia, listy, nagłówki, linki) w sposób przypominający klasyczny edytor tekstu, zapisując wynik zwykle w HTML}
}

\newglossaryentry{rich-text-editor}
{
    name={Rich text editor},
    description={Edytor treści, który zamiast „surowego” tekstu umożliwia stosowanie formatowania (np. pogrubienie, kursywa, listy, nagłówki, linki), dzięki czemu użytkownik może tworzyć czytelne, sformatowane wpisy}
}

\newglossaryentry{tiptap}
{
    name={Tiptap},
    description={Nowoczesny, rozszerzalny edytor \textit{rich text} dla aplikacji webowych oparty na silniku ProseMirror. Umożliwia budowanie rozbudowanych, modularnych edytorów treści, np. do postów na forum}
}

\newglossaryentry{integration-tests}
{
    name={Testy integracyjne},
    description={Testy sprawdzające, czy połączone ze sobą moduły lub usługi współpracują poprawnie — na przykład czy warstwa backendowa poprawnie komunikuje się z bazą danych, warstwą sieciową i pozostałymi komponentami systemu}
}

\newglossaryentry{endpoint}
{
    name={endpoint},
    description={Endpoint to konkretny adres (np. \gls{url}) i metoda protokołu HTTP
    w \gls{api}, które razem odpowiadają za realizację jednej, dobrze zdefiniowanej
    operacji (np. pobrania listy spotów, dodania komentarza, wyszukania spotów).}
}

\newglossaryentry{redux-slice}
{
    name={slice Redux},
    description={Slice Redux to wydzielona część globalnego stanu w \gls{redux},
    wraz z powiązanymi akcjami i reduktorami, odpowiedzialna za jeden obszar domeny
    (np. konto użytkownika, czat, mapę czy listę znajomych).}
}

\newglossaryentry{jsoup}{
    name={jsoup},
    description={Biblioteka \textit{Java} do przetwarzania dokumentów HTML,
    umożliwiająca parsowanie, przeszukiwanie i modyfikowanie struktury dokumentu
    w sposób zbliżony do pracy z DOM-em i selektorami CSS.}
}

\newglossaryentry{paginacja}
{
    name={paginacja},
    description={Mechanizm dzielenia dużych zbiorów danych
    (np. list postów, wyników wyszukiwania, komentarzy)
    na mniejsze strony, które są pobierane i wyświetlane stopniowo,
    zamiast ładowania wszystkich elementów jednocześnie.}
}

\newglossaryentry{gantt-chart}
{
    name={diagram Gantta},
    description={Graficzne narzędzie do planowania i monitorowania przebiegu projektu, przedstawiające zadania jako poziome paski na osi czasu wraz z ich początkiem, końcem oraz ewentualnymi zależnościami.}
}

\newglossaryentry{moscow}{
    name={MoSCoW},
    description={Metoda nadawania priorytetów wymaganiom, wyróżniająca kategorie:
    \emph{Must have}, \emph{Should have}, \emph{Could have} oraz
    \emph{Won't have this time}}
}

\newglossaryentry{cache}{
    name={Cache},
    description={Mechanizm przechowywania danych w celu przyspieszenia ich ponownego odczytu}
}

\newglossaryentry{jvm}{
    name={JVM},
    description={(ang. \textit{Java Virtual Machine}); maszyna wirtualna oraz środowisko do wykonywania kodu bajtowego Javy}
}

\newglossaryentry{spring-framework}{
    name={Spring Framework},
    description={Framework, który służy do tworzenia aplikacji w Javie.
    Zajmuje się między innymi ustawianiem konfiguracji oraz zarządzaniem zależnościami,
    np. wstrzykiwaniem (ang. \textit{Dependency Injection}), odwróceniem kontroli (ang. \textit{Inversion of Control}).
    Oferuje wiele modułów wspierających, które ułatwiają rozwój projektów}
}

\newglossaryentry{convention-over-configuration}{
    name={Convention Over Configuration},
    description={Zasada programowania polegająca na przyjmowaniu domyślnych, bazowych reguł, zamiast ręcznego implementowania konfiguracji}
}

\newglossaryentry{dom}{
    name={DOM},
    description={(ang. \textit{Document Object Model}); interfejs reprezentacji stron internetowych jako węzły i obiekty.
    Dzięki temu skrypty, np. napisane w JavaScript, mogą wchodzić w interakcje ze stroną
    (np. modyfikować strukturę, treść lub styl)}
}

\newglossaryentry{query-params}{
    name={Parametry zapytania (query params)},
    description={Pary \texttt{klucz=wartość} przekazywane w części adresu URL po znaku zapytania \texttt{?}, służące m.in. do filtrowania, sortowania, paginacji wyników lub przekazywania dodatkowych opcji żądania do serwera, np. \texttt{?param1=val1\&param2=val2}}
}

\newglossaryentry{docker}{
    name={Docker},
    description={Platforma do konteneryzacji aplikacji wykorzystująca wirtualizację na poziomie systemu operacyjnego.
    Umożliwia uruchamianie oprogramowania w lekkich, odizolowanych kontenerach wraz z wszystkimi zależnościami,
    co ułatwia przenoszenie i powtarzalne odtwarzanie środowisk uruchomieniowych}
}

\newglossaryentry{docker-hub}{
    name={Docker Hub},
    description={Publiczne repozytorium (rejestr) obrazów kontenerów Docker,
    umożliwiające przechowywanie, udostępnianie oraz dystrybucję obrazów.
    Użytkownicy mogą korzystać z oficjalnych obrazów przygotowanych przez społeczność
    i dostawców oprogramowania lub publikować własne obrazy}
}

\newglossaryentry{kontener}{
    name={kontener},
    description={Lekka, odizolowana jednostka uruchomieniowa, w której umieszczana jest aplikacja wraz z jej zależnościami
    (bibliotekami, konfiguracją itp.). Kontenery współdzielą jądro systemu operacyjnego, ale posiadają własną
    przestrzeń procesów, systemu plików i sieci, co zapewnia powtarzalne i przenośne środowisko uruchomieniowe,
    najczęściej zarządzane przez platformę taką jak \glslink{docker}{docker}}
}

\newglossaryentry{wolumen}{
    name={Wolumen},
    description={Zarządzane przez Dockera magazyny danych dla kontenerów, które
    są odizolowane od maszyny hosta.}
}

\newglossaryentry{websocket}{
    name={WebSocket},
    description={Protokół komunikacyjny umożliwiający utrzymanie stałego połączenia pomiędzy \gls{klient}em a \gls{serwer}em oraz dwukierunkową wymianę danych w czasie zbliżonym do rzeczywistego.}
}

\newglossaryentry{stomp}{
    name={STOMP},
    description={Protokół wiadomościowy działający „nad” \gls{websocket}. Wprowadza pojęcia \glslink{destynacja}{destynacji}, \glslink{subskrypcja}{subskrypcji} i wysyłania komunikatów w modelu \gls{publish-subscribe}.}
}

\newglossaryentry{http}{
    name={HTTP},
    description={Podstawowy protokół komunikacji w sieci WWW, wykorzystywany m.in. przez \gls{rest_api}. Zwykle działa w modelu „żądanie–odpowiedź”, bez stałego połączenia.}
}

\newglossaryentry{ietf}{
    name={IETF},
    description={Organizacja opracowująca i publikująca standardy internetowe, m.in. dokumenty typu \gls{rfc}.}
}

\newglossaryentry{rfc}{
    name={RFC},
    description={Request for Comments to numerowane publikacje opisujące ustalenia dotyczące działania Internetu — m.in. protokoły, formaty danych i dobre praktyki.
    Są wydawane w ramach społeczności \gls{ietf}.}
}

\newglossaryentry{handshake}{
    name={handshake},
    description={Etap początkowy zestawiania połączenia (np. \gls{websocket}), w którym \gls{klient} i \gls{serwer} uzgadniają parametry komunikacji.}
}

\newglossaryentry{ack}{
    name={ACK},
    description={Skrót od \textit{Acknowledgement}; komunikat potwierdzający odbiór lub przetworzenie wiadomości. W \gls{stomp} może służyć do potwierdzania wiadomości po stronie \gls{klient}a.}
}

\newglossaryentry{sockjs}{
    name={SockJS},
    description={Mechanizm kompatybilności dla \gls{websocket}: gdy przeglądarka lub sieć nie obsługuje WebSocket, SockJS może użyć alternatywnego sposobu komunikacji, aby zachować podobne zachowanie.}
}

\newglossaryentry{polling}{
    name={polling},
    description={Sposób komunikacji, w którym \gls{klient} cyklicznie „odpytuje” \gls{serwer} (np. co kilka sekund) o nowe dane, zwykle poprzez \gls{http}.}
}

\newglossaryentry{long-polling}{
    name={long polling},
    description={Odmiana \gls{polling}u: \gls{klient} wysyła żądanie, a \gls{serwer} wstrzymuje odpowiedź do momentu pojawienia się nowych danych (lub upłynięcia limitu czasu).}
}

\newglossaryentry{publish-subscribe}{
    name={publish--subscribe},
    description={Model komunikacji, w którym nadawca publikuje zdarzenia na kanał, a odbiorcy, którzy go \glslink{subskrypcja}{subskrybują}, automatycznie otrzymują wiadomości.}
}

\newglossaryentry{destynacja}{
    name={destynacja},
    description={Adres logiczny (ścieżka) w komunikacji \gls{stomp}, na który wysyła się wiadomości lub który się \glslink{subskrypcja}{subskrybuje} (np. \texttt{/app/...} albo \texttt{/subscribe/...}).}
}

\newglossaryentry{subskrypcja}{
    name={subskrypcja},
    description={Mechanizm „zapisania się” na kanał (\glslink{destynacja}{destynację}) w celu automatycznego odbierania wiadomości publikowanych przez \gls{serwer}.}
}

\newglossaryentry{prefiks}{
    name={prefiks},
    description={Ustalony początek ścieżki (np. \texttt{/app} lub \texttt{/subscribe}), który porządkuje i rozróżnia typy komunikacji w aplikacji.}
}

\newglossaryentry{ramka}{
    name={ramka},
    description={Podstawowa jednostka danych przesyłana w ramach połączenia \gls{websocket}.}
}

\newglossaryentry{serwer}{
    name={serwer},
    description={Część systemu (zwykle po stronie \glslink{backend}{backendu}), która przetwarza żądania i udostępnia dane lub funkcje \glslink{klient}{klientom}.}
}

\newglossaryentry{klient}{
    name={klient},
    description={Część systemu (zwykle po stronie \gls{frontend}u), która korzysta z usług \glslink{serwer}{serwera} i prezentuje dane użytkownikowi.}
}
\newglossaryentry{naglowek}{
    name={Nagłówek},
    description={Metadane dołączane do wiadomości lub żądania (np. dodatkowe pola opisujące typ komunikatu, identyfikator, autoryzację).}
}

\newglossaryentry{broker}{
    name={Broker},
    description={Element pośredniczący w komunikacji wiadomościowej (np. STOMP), który odbiera wiadomości i rozsyła je do odpowiednich subskrybentów.}
}

\newglossaryentry{callback}{
    name={callback},
    description={Funkcja przekazywana jako parametr, wywoływana automatycznie przy wystąpieniu zdarzenia (np. gdy nadejdzie wiadomość z WebSocket).}
}

\newglossaryentry{store}{
    name={Store (Redux)},
    description={Centralne miejsce przechowywania stanu aplikacji w Redux. Pozwala odczytywać aktualny stan oraz wysyłać akcje zmieniające ten stan.}
}
\newglossaryentry{propsy}
{
    name={propsy},
    description={Zestaw właściwości przekazywanych do komponentu React przez komponent nadrzędny.}
}

\newglossaryentry{kontekst}{
    name={kontekst React'a},
    description={Mechanizm w React pozwalający udostępnić wspólny obiekt wielu komponentom bez przekazywania go przez \gls{propsy}.}
}

\newglossaryentry{reconnect}{
    name={reconnect},
    description={Automatyczne ponowne nawiązanie połączenia po jego zerwaniu (np. po chwilowej utracie internetu).}
}

\newglossaryentry{payload}{
    name={Payload},
    description={Właściwa „treść” wiadomości (dane), którą przesyłamy, np. obiekt wiadomości czatu zapisany jako JSON.}
}

\newglossaryentry{stomp_client}{
    name={klient STOMP},
    description={Biblioteka/komponent po stronie klienta realizujący protokół STOMP (subskrypcje, publikacja, ramki, nagłówki) nad transportem WebSocket/SockJS}
}

\newglossaryentry{stompjs}{
    name={StompJS},
    description={Biblioteka JavaScript/TypeScript (np. \texttt{@stomp/stompjs}) implementująca klienta protokołu STOMP}
}

\newglossaryentry{json}{
    name={JSON},
    description={Format tekstowy do wymiany danych (JavaScript Object Notation), często używany jako reprezentacja komunikatów API i wiadomości w aplikacjach webowych}
}

\newglossaryentry{serializacja}{
    name={Serializacja},
    description={Proces zamiany obiektu (np. w TypeScript/Java) na format przenośny, np. JSON, w celu przesłania lub zapisu}
}

\newglossaryentry{deserializacja}{
    name={deserializacja},
    description={Proces odtworzenia obiektu z formatu przenośnego (np. JSON) po stronie odbiorcy}
}

\newglossaryentry{cleanup}{
    name={cleanup},
    description={„Sprzątanie” zasobów: usuwanie subskrypcji, rozłączanie połączeń i czyszczenie struktur pomocniczych, aby nie pozostawiać aktywnych uchwytów}
}

\newglossaryentry{cykl_zycia}{
    name={cykl życia komponentu},
    description={Etapy działania komponentu (np. montowanie, aktualizacja, odmontowanie) determinujące momenty inicjalizacji i zwalniania zasobów}
}

\newglossaryentry{use_effect}{
    name={useEffect},
    description={Hook React służący do uruchamiania efektów ubocznych po renderowaniu (oraz do definiowania funkcji sprzątającej wykonywanej przy odmontowaniu lub zmianie zależności)}
}

\newglossaryentry{montowanie}{
    name={montowanie},
    description={Moment „podpięcia” komponentu do drzewa UI.}
}

\newglossaryentry{odmontowanie}{
    name={odmontowanie},
    description={Usunięcie komponentu z drzewa UI.}
}

\newglossaryentry{tablica_zaleznosci}{
    name={tablica zależności},
    description={Lista wartości (dependencies) sterująca ponownym uruchamianiem \texttt{useEffect} w React; zmiana zależności powoduje wykonanie sprzątania i ponowne uruchomienie efektu}
}

\newglossaryentry{zmienna_srodowiskowa}{
    name={zmienna środowiskowa},
    description={Parametr konfiguracyjny przekazywany aplikacji z zewnątrz (np. plik \texttt{.env} lub konfiguracja CI/CD), pozwalający zmieniać zachowanie bez modyfikacji kodu}
}

\newglossaryentry{singleton}{
    name={Singleton},
    description={Wzorzec projektowy gwarantujący istnienie jednej, współdzielonej instancji obiektu w aplikacji}
}

\newglossaryentry{redux_dispatch}{
    name={dispatch (Redux)},
    description={Funkcja w Redux służąca do wysyłania akcji, które modyfikują stan przechowywany w store}
}

\newglossaryentry{akcja_redux}{
    name={akcja Redux},
    description={Obiekt opisujący zdarzenie/operację zmiany stanu w Redux, obsługiwany przez reducery (np. \texttt{chatActions.setLastMessage(...)})}
}

\newglossaryentry{fallback}
{
    name={Fallback},
    description={Mechanizm awaryjnego przełączania na alternatywny sposób działania, gdy preferowana metoda jest niedostępna. W kontekście WebSocket/SockJS oznacza automatyczne przejście z WebSocket na techniki oparte o HTTP (np. streaming lub long polling), aby utrzymać komunikację.}
}


\newglossaryentry{optimistic-ui}
{
    name={Optimistic UI},
    description={Technika w aplikacjach klienckich polegająca na natychmiastowym zaktualizowaniu interfejsu jeszcze przed otrzymaniem potwierdzenia z serwera. Poprawia wrażenie responsywności,
    a w przypadku błędu po stronie serwera zmiana jest cofana lub oznaczana jako nieudana. W niniejszym projekcie technikę tę zastosowano dla wysyłania wiadomości na czacie.}
}





