%! Author = Adam
%! Date = 01/11/2025

\subsection{Aktorzy}
\label{subsec:aktorzy}

\renewcommand{\arraystretch}{1.15} %to nie jest grozne?

\begin{longtable}{|L{0.34\textwidth}|p{0.62\textwidth}|}
    \hline
    \textbf{Aktor} & \textbf{Krótki opis} \\ \hline
    \endfirsthead
    \hline
    \textbf{Aktor} & \textbf{Krótki opis} \\ \hline
    \endhead
    \hline \multicolumn{2}{r}{\small Kontynuacja na następnej stronie} \\
    \endfoot
    \hline
    \endlastfoot

    Użytkownik niezalogowany & Gość przeglądający publiczne treści (mapa, spoty, forum), może się zarejestrować/zalogować. \\ \hline
    Użytkownik (nie premium) & Zarejestrowany użytkownik podstawowy: zarządza kontem i ulubionymi, dodaje treści/komentarze, korzysta z czatu. \\ \hline
    Użytkownik premium & Użytkownik z wykupioną subskrypcją; ma dostęp do funkcji premium (np. widok premium, rozszerzone wyszukiwanie). \\ \hline
    Moderator & Nadzór treści: przegląda zgłoszenia, akceptuje/usuwa posty i komentarze, blokuje nadużycia. \\ \hline
    Deweloper & Osoba utrzymująca system: inicjuje pipeline’y, monitoruje wdrożenia i infrastrukturę. \\ \hline
    Dostawca tożsamości (OAuth) & Zewnętrzny provider logowania (SSO) używany do weryfikacji kont użytkowników. \\ \hline
    Bramka płatnicza & Zewnętrzny system obsługujący płatności/subskrypcje (autoryzacja, statusy). \\ \hline
    Usługa mailowa (SMTP) & Serwis do wysyłki wiadomości e-mail (rejestracja, reset hasła, powiadomienia). \\ \hline
    Dostawca API do map & Zewnętrzny dostawca map/tili/geokodowania wykorzystywany w widoku mapy. \\ \hline
    Dostawca API pogodowego & Zewnętrzne API prognoz pogody używane do prezentacji warunków na spotach. \\ \hline
    Azure Blob Storage & Usługa chmurowa do przechowywania multimediów (zdjęcia spotów, pliki z czatu). \\ \hline
    Baza danych (DB) & System przechowywania danych aplikacji. \\ \hline
    GitHub (repo + CI/CD) & Repozytorium kodu i pipeline’y automatyzujące build/test/deploy. \\ \hline

    \caption{Aktorzy systemu – przegląd\label{tab:aktorzy}}\\

\end{longtable}
