%! Author = Adam
%! Date = 22/11/2025

\subsubsection{Scenariusze przypadków użycia dla wyszukiwarki}

\usecasecard{tab:pu10-globalna-wyszukiwarka}{Wyszukiwanie spota w globalnej wyszukiwarce}{%
    \ucpriority{S}
    \ucactors{Użytkownik, Usługa do wyświetlania mapy, Usługa do pogody}
    \ucdesc{Użytkownik wyszukuje spoty za pomocą globalnej wyszukiwarki w aplikacji.}
    \ucpre{Użytkownik znajduje się na stronie głównej z wyszukwiarką.}
    \ucpost{Użytkownik otrzymuje listę znalezionych spotów.}
    \ucmain{%
        \begin{enumerate}[nosep,leftmargin=16pt,labelindent=0pt]
            \item Użytkownik wpisuje frazę w globalnej wyszukiwarce.
            \item System wyszukuje spoty spełniające kryteria.
            \item System wyświetla listę wyników.
        \end{enumerate}
    }
    \ucalt{%
        \begin{enumerate}[nosep,leftmargin=21pt,labelindent=0pt,label={}]
            \item[3a.] System informuje o braku wyników spełniających kryteria.
        \end{enumerate}
    }
}

\usecasecard{tab:pu11-przejscie-z-wynikow}{Przejście do spota na mapie z wyszukiwarki}{%
    \ucpriority{S}
    \ucactors{Użytkownik}
    \ucdesc{Użytkownik przechodzi z wyników wyszukiwarki do widoku mapy ustawionego na konkretny spot.}
    \ucpre{Wyświetlona jest lista wyników wyszukiwania spotów.}
    \ucpost{Mapa jest przybliżona do wybranego spota, a jego szczegóły są dostępne.}
    \ucmain{%
        \begin{enumerate}[nosep,leftmargin=16pt,labelindent=0pt]
            \item Użytkownik wybiera spota z listy wyników.
            \item System przełącza widok na moduł mapy.
            \item System ustawia mapę na lokalizację spota i otwiera jego szczegóły.
        \end{enumerate}
    }
    \ucalt{Brak istotnych alternatywnych przepływów.}
}
