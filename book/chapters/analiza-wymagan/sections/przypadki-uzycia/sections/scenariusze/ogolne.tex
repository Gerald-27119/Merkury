%! Author = Adam
%! Date = 22/11/2025

\subsubsection{Scenariusze przypadków użycia – funkcje ogólne}

\usecasecard{tab:pu1-rejestracja}{Rejestracja użytkownika}{%
    \ucpriority{M}
    \ucactors{Użytkownik}
    \ucdesc{Użytkownik zakłada konto poprzez formularz rejestracji.}
    \ucpre{Użytkownik znajduje się na stronie z formularzem rejestracji.}
    \ucpost{Użytkownik posiada konto w systemie.}
    \ucmain{%
        \begin{enumerate}[nosep,leftmargin=16pt,labelindent=0pt]
            \item Użytkownik wypełnia formularz rejestracyjny.
            \item Użytkownik naciska przycisk rejestracji.
            \item System tworzy konto użytkownika.
            \item System loguje użytkownika i przenosi go na ostatnio dowiedzoną podstronę.
        \end{enumerate}
    }
    \ucalt{%
        \begin{enumerate}[nosep,leftmargin=21pt,labelindent=0pt,label={}]
            \item[1a.] Podane dane są niepoprawne – system wyświetla
            komunikat o błędzie oraz podświetla pola wymagające poprawy.
            \item[2a.] Nazwa użytkownika jest już zajęta – system wyświetla
            komunikat o błędzie.
            \item[2b.] Adres email jest już zajęty – system wyświetla
            komunikat o błędzie.
        \end{enumerate}
    }
}

\usecasecard{tab:pu2-logowanie}{Logowanie użytkownika}{%
    \ucpriority{M}
    \ucactors{Użytkownik}
    \ucdesc{Użytkownik loguje się do systemu, podając login i hasło.}
    \ucpre{Użytkownik znajduje się na stronie logowania.}
    \ucpost{Użytkownik jest zalogowany i przeniesiony na ostatnio dowiedzoną podstronę.}
    \ucmain{%
        \begin{enumerate}[nosep,leftmargin=16pt,labelindent=0pt]
            \item Użytkownik wypełnia formularz logowania.
            \item Użytkownik naciska przycisk logowania.
            \item System loguje użytkownika i przenosi go na ostatnio dowiedzoną podstronę.
        \end{enumerate}
    }
    \ucalt{%
        \begin{enumerate}[nosep,leftmargin=21pt,labelindent=0pt,label={}]
            \item[2a.] Podane dane są niepoprawne – system wyświetla komunikat o błędzie.
        \end{enumerate}
    }
}

\usecasecard{tab:pu3-reset-hasla}{Resetowanie hasła}{%
    \ucpriority{S}
    \ucactors{Użytkownik, Usługa SMTP}
    \ucdesc{Użytkownik inicjuje reset hasła, aby odzyskać dostęp do konta.}
    \ucpre{Użytkownik znajduje się na ekranie resetu hasła.}
    \ucpost{Użytkownik otrzymuje wiadomość e-mail z linkiem do ustawienia nowego hasła.}
    \ucmain{%
        \begin{enumerate}[nosep,leftmargin=16pt,labelindent=0pt]
            \item Użytkownik wpisuje adres e-mail powiązany z kontem.
            \item Użytkownik zatwierdza żądanie resetu hasła.
            \item System generuje token resetu hasła.
            \item System wysyła e-mail z linkiem do zmiany hasła.
        \end{enumerate}
    }
    \ucalt{%
        \begin{enumerate}[nosep,leftmargin=21pt,labelindent=0pt,label={}]
            \item[2a.] Nie istnieje konto dla podanego adresu – system wyświetla komunikat o błędzie.
            \item[4a.] Występuje błąd połączenia z usługą SMTP – system informuje użytkownika o problemie technicznym.
        \end{enumerate}
    }
}

\usecasecard{tab:pu4-zmiana-hasla}{Zmiana hasła w ustawieniach konta}{%
    \ucpriority{S}
    \ucactors{Użytkownik}
    \ucdesc{Użytkownik zmienia hasło do konta z poziomu ustawień profilu.}
    \ucpre{Użytkownik jest zalogowany i znajduje się na ekranie zmiany danych konta.}
    \ucpost{Hasło do konta użytkownika zostało zaktualizowane.}
    \ucmain{%
        \begin{enumerate}[nosep,leftmargin=16pt,labelindent=0pt]
            \item Użytkownik wpisuje aktualne hasło.
            \item Użytkownik wpisuje nowe hasło i powtarza je.
            \item Użytkownik zatwierdza formularz zmiany hasła.
            \item System zapisuje nowe hasło i informuje o powodzeniu operacji.
        \end{enumerate}
    }
    \ucalt{%
        \begin{enumerate}[nosep,leftmargin=21pt,labelindent=0pt,label={}]
            \item[3a.] Aktualne hasło jest nieprawidłowe – system wyświetla komunikat i nie zapisuje zmian.
            \item[3b.] Nowe hasło nie spełnia wymagań bezpieczeństwa – system informuje o błędzie i podświetla pola do poprawy.
        \end{enumerate}
    }
}

\usecasecard{tab:pu5-wylogowanie}{Wylogowanie użytkownika}{%
    \ucpriority{S}
    \ucactors{Użytkownik}
    \ucdesc{Użytkownik wylogowuje się z aplikacji.}
    \ucpre{Użytkownik jest zalogowany.}
    \ucpost{Sesja użytkownika została zakończona, użytkownik widzi stronę główną dla niezalogowanych.}
    \ucmain{%
        \begin{enumerate}[nosep,leftmargin=16pt,labelindent=0pt]
            \item Użytkownik wybiera opcję wylogowania z menu.
            \item System unieważnia token dostępu użytkownika.
            \item System przenosi użytkownika na stronę główną aplikacji.
        \end{enumerate}
    }
    \ucalt{Brak istotnych alternatywnych przepływów.}
}

\usecasecard{tab:pu6-powiadomienia}{Przeglądanie powiadomień}{%
    \ucpriority{W}
    \ucactors{Użytkownik}
    \ucdesc{Użytkownik przegląda listę powiadomień.}
    \ucpre{Użytkownik jest na ekranie centra powiadomień.}
    \ucpost{Powiadomienia zostały wyświetlone, a wybrane oznaczone jako przeczytane.}
    \ucmain{%
        \begin{enumerate}[nosep,leftmargin=16pt,labelindent=0pt]
            \item System wyświetla powiadomienia w odwróconym porządku chronologicznym.
            \item Użytkownik otwiera wybrane powiadomienie.
            \item System oznacza powiadomienie jako przeczytane i ewentualnie przenosi użytkownika do powiązanego widoku.
        \end{enumerate}
    }
    \ucalt{%
        \begin{enumerate}[nosep,leftmargin=21pt,labelindent=0pt,label={}]
            \item[1a.] System nie może pobrać powiadomień (błąd serwera) – użytkownik otrzymuje komunikat o błędzie i może spróbować ponownie.
        \end{enumerate}
    }
}
