\subsubsection{Scenariusze przypadków użycia dla profilu użytkownika}

\usecasecard{tab:pu27-dodaj-spota}{Dodanie spota w profilu użytkownika}{%
    \ucpriority{Wysoki}
    \ucactors{Użytkownik zalogowany, Usługa do wyświetlania mapy, Usługa do przechowywania plików w chmurze}
    \ucdesc{Użytkownik dodaje nowy spot poprzez swój profil.}
    \ucpre{Użytkownik jest zalogowany i znajduje się w widoku swojego profilu.}
    \ucpost{Nowy spot został zapisany i widoczny na mapie oraz w profilu użytkownika.}
    \ucmain{%
        \begin{enumerate}[nosep,leftmargin=16pt,labelindent=0pt]
            \item Użytkownik wybiera opcję „Dodaj spota”.
            \item Użytkownik uzupełnia podstawowe informacje o spocie (nazwa, opis, typ).
            \item Użytkownik wskazuje lokalizację spota na mapie.
            \item (Opcjonalnie) Użytkownik dodaje zdjęcia/filmy do spota.
            \item Użytkownik zapisuje spota.
            \item System zapisuje dane spota (oraz pliki w chmurze) i aktualizuje mapę oraz profil użytkownika.
        \end{enumerate}
    }
    \ucalt{%
        \begin{enumerate}[nosep,leftmargin=21pt,labelindent=0pt,label={}]
            \item[2a.] Formularz zawiera błędy – system wyświetla komunikat i zaznacza wymagające poprawy pola.
        \end{enumerate}
    }
}

\usecasecard{tab:pu28-profil-wlasny}{Przeglądanie profilu użytkownika}{%
    \ucpriority{Wysoki}
    \ucactors{Użytkownik zalogowany}
    \ucdesc{Użytkownik przegląda swój profil (lista spotów, media, podstawowe dane).}
    \ucpre{Użytkownik jest zalogowany.}
    \ucpost{Wyświetlony jest widok profilu użytkownika wraz z jego zawartością.}
    \ucmain{%
        \begin{enumerate}[nosep,leftmargin=16pt,labelindent=0pt]
            \item Użytkownik otwiera swój profil.
            \item System pobiera dane profilu (informacje podstawowe, spoty, media).
            \item System wyświetla dane w odpowiednich sekcjach (spoty, zdjęcia, filmy, komentarze).
        \end{enumerate}
    }
    \ucalt{Brak istotnych alternatywnych przepływów.}
}

\usecasecard{tab:pu29-profil-innego}{Przeglądanie profilu innego użytkownika}{%
    \ucpriority{Średni}
    \ucactors{Użytkownik zalogowany}
    \ucdesc{Użytkownik ogląda profil innego użytkownika (np. z mapy, forum lub społeczności).}
    \ucpre{Użytkownik jest zalogowany i ma dostęp do odnośnika do profilu innego użytkownika.}
    \ucpost{Profil innego użytkownika został wyświetlony.}
    \ucmain{%
        \begin{enumerate}[nosep,leftmargin=16pt,labelindent=0pt]
            \item Użytkownik wybiera odnośnik do profilu innego użytkownika.
            \item System pobiera dane profilu docelowego użytkownika.
            \item System wyświetla profil (media, podstawowe informacje).
        \end{enumerate}
    }
    \ucalt{%
        \begin{enumerate}[nosep,leftmargin=21pt,labelindent=0pt,label={}]
            \item[2a.] Wystąpił błąd podczas pobierania danych użytkownika – system wyświetla informację o błędzie.
        \end{enumerate}
    }
}

\usecasecard{tab:pu30-dodaj-znajomego}{Dodanie użytkownika do znajomych}{%
    \ucpriority{Średni}
    \ucactors{Użytkownik zalogowany}
    \ucdesc{Użytkownik wysyła lub akceptuje zaproszenie do znajomych.}
    \ucpre{Użytkownik jest zalogowany i przegląda profil innego użytkownika.}
    \ucpost{Relacja „znajomy” została utworzona lub zaproszenie czeka na akceptację.}
    \ucmain{%
        \begin{enumerate}[nosep,leftmargin=16pt,labelindent=0pt]
            \item Użytkownik klika przycisk „Dodaj do znajomych”.
            \item System sprawdza, czy relacja już istnieje.
            \item System tworzy nowe zaproszenie.
            \item System informuje o statusie o wysłaniu zaproszenia.
        \end{enumerate}
    }
    \ucalt{Brak istotnych alternatywnych przepływów.}
}

\usecasecard{tab:pu31-spolecznosci}{Przeglądanie społeczności}{%
    \ucpriority{Średni}
    \ucactors{Użytkownik zalogowany}
    \ucdesc{Użytkownik przegląda społeczności, grupy lub listy znajomych powiązane z aplikacją.}
    \ucpre{Użytkownik jest zalogowany.}
    \ucpost{Lista społeczności lub znajomych została wyświetlona.}
    \ucmain{%
        \begin{enumerate}[nosep,leftmargin=16pt,labelindent=0pt]
            \item Użytkownik przechodzi do sekcji społeczności.
            \item System pobiera listę społeczności i znajomych użytkownika.
            \item System wyświetla listę z możliwością przechodzenia do profili i czatów.
        \end{enumerate}
    }
    \ucalt{Brak istotnych alternatywnych przepływów.}
}

\usecasecard{tab:pu39-przegladaj-dodane-spoty}{Przeglądanie dodanych spotów}{%
    \ucpriority{Wysoki}
    \ucactors{Użytkownik zalogowany, Usługa do wyświetlania mapy}
    \ucdesc{Użytkownik przegląda listę/siatkę spotów, które sam dodał.}
    \ucpre{Użytkownik jest zalogowany i znajduje się w widoku swojego profilu lub sekcji „Moje spoty”.}
    \ucpost{Lista dodanych spotów użytkownika została wyświetlona (np. na mapie i/lub w formie listy).}
    \ucmain{%
        \begin{enumerate}[nosep,leftmargin=16pt,labelindent=0pt]
            \item Użytkownik przechodzi do sekcji „Moje spoty”.
            \item System pobiera listę spotów dodanych przez użytkownika.
            \item System wyświetla listę spotów oraz znaczniki na mapie.
            \item Użytkownik wybiera spota, aby przejść do jego szczegółów.
        \end{enumerate}
    }
    \ucalt{%
        \begin{enumerate}[nosep,leftmargin=21pt,labelindent=0pt,label={}]
            \item[2a.] Użytkownik nie dodał jeszcze żadnego spota – system wyświetla komunikat i proponuje dodanie pierwszego spota.
        \end{enumerate}
    }
}

\usecasecard{tab:pu40-edytuj-dane-uzytkownika}{Edycja danych użytkownika}{%
    \ucpriority{Wysoki}
    \ucactors{Użytkownik zalogowany}
    \ucdesc{Użytkownik modyfikuje swoje dane profilu (np. nazwę, opis, avatar).}
    \ucpre{Użytkownik jest zalogowany i znajduje się w widoku edycji profilu.}
    \ucpost{Zaktualizowane dane profilu są zapisane i widoczne w aplikacji.}
    \ucmain{%
        \begin{enumerate}[nosep,leftmargin=16pt,labelindent=0pt]
            \item Użytkownik otwiera widok edycji profilu.
            \item Użytkownik wprowadza zmiany w danych profilu.
            \item Użytkownik zapisuje zmiany.
            \item System waliduje dane i zapisuje zaktualizowany profil.
        \end{enumerate}
    }
    \ucalt{%
        \begin{enumerate}[nosep,leftmargin=21pt,labelindent=0pt,label={}]
            \item[4a.] Dane są niepoprawne lub niekompletne – system wyświetla komunikat o błędzie i zaznacza pola do poprawy.
        \end{enumerate}
    }
}

\usecasecard{tab:pu41-przegladaj-zdjecia-spotow}{Przeglądanie dodanych zdjęć do spotów}{%
    \ucpriority{Średni}
    \ucactors{Użytkownik zalogowany, Usługa do przechowywania plików w chmurze}
    \ucdesc{Użytkownik przegląda wszystkie zdjęcia powiązane ze spotami, które dodał.}
    \ucpre{Użytkownik jest zalogowany i znajduje się w sekcji mediów (np. „Moje zdjęcia”).}
    \ucpost{Lista lub galeria zdjęć powiązanych ze spotami użytkownika została wyświetlona.}
    \ucmain{%
        \begin{enumerate}[nosep,leftmargin=16pt,labelindent=0pt]
            \item Użytkownik otwiera sekcję przeglądania zdjęć ze spotów.
            \item System pobiera metadane zdjęć z usługi przechowywania plików.
            \item System wyświetla galerię zdjęć z podstawowymi informacjami (np. nazwa spota, data dodania).
            \item Użytkownik wybiera zdjęcie, aby zobaczyć je w powiększeniu lub przejść do spota.
        \end{enumerate}
    }
    \ucalt{%
        \begin{enumerate}[nosep,leftmargin=21pt,labelindent=0pt,label={}]
            \item[2a.] Użytkownik nie dodał jeszcze zdjęć – system wyświetla informację o braku zdjęć.
        \end{enumerate}
    }
}

\usecasecard{tab:pu42-przegladaj-filmy-spotow}{Przeglądanie dodanych filmów do spotów}{%
    \ucpriority{Średni}
    \ucactors{Użytkownik zalogowany, Usługa do przechowywania plików w chmurze}
    \ucdesc{Użytkownik przegląda filmy powiązane ze spotami, które dodał.}
    \ucpre{Użytkownik jest zalogowany i znajduje się w sekcji mediów (np. „Moje filmy”).}
    \ucpost{Lista lub galeria filmów powiązanych ze spotami użytkownika została wyświetlona.}
    \ucmain{%
        \begin{enumerate}[nosep,leftmargin=16pt,labelindent=0pt]
            \item Użytkownik otwiera sekcję przeglądania filmów ze spotów.
            \item System pobiera metadane filmów z usługi przechowywania plików.
            \item System wyświetla listę/galerię filmów z podstawowymi informacjami.
            \item Użytkownik wybiera film, aby go odtworzyć lub przejść do spota.
        \end{enumerate}
    }
    \ucalt{%
        \begin{enumerate}[nosep,leftmargin=21pt,labelindent=0pt,label={}]
            \item[2a.] Użytkownik nie dodał jeszcze filmów – system wyświetla informację o braku filmów.
        \end{enumerate}
    }
}

\usecasecard{tab:pu43-przegladaj-komentarze-spotow}{Przeglądanie dodanych komentarzy do spotów}{%
    \ucpriority{Średni}
    \ucactors{Użytkownik zalogowany}
    \ucdesc{Użytkownik przegląda komentarze dodane do spotów, które sam utworzył.}
    \ucpre{Użytkownik jest zalogowany i otwiera sekcję komentarzy do swoich spotów.}
    \ucpost{Lista komentarzy do spotów użytkownika została wyświetlona.}
    \ucmain{%
        \begin{enumerate}[nosep,leftmargin=16pt,labelindent=0pt]
            \item Użytkownik przechodzi do sekcji komentarzy do własnych spotów.
            \item System pobiera komentarze powiązane ze spotami użytkownika.
            \item System wyświetla komentarze (np. w kolejności chronologicznej lub według popularności).
        \end{enumerate}
    }
    \ucalt{%
        \begin{enumerate}[nosep,leftmargin=21pt,labelindent=0pt,label={}]
            \item[2a.] Żaden z spotów użytkownika nie ma komentarzy – system wyświetla odpowiednią informację.
        \end{enumerate}
    }
}

