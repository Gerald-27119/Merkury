%! Author = Adam
%! Date = 22/11/2025

\subsubsection{Scenariusze przypadków użycia dla mapy}

\usecasecard{tab:pu8-mapa}{Przeglądanie mapy spotów}{%
    \ucpriority{Wysoki}
    \ucactors{Użytkownik niezalogowany, Usługa do wyświetlania mapy}
    \ucdesc{Użytkownik przegląda mapę spotów.}
    \ucpre{Użytkownik znajduje się w module mapy.}
    \ucpost{Mapa ze spotami została wyświetlona, a użytkownik może przybliżać, oddalać i przesuwać widok.}
    \ucmain{%
        \begin{enumerate}[nosep,leftmargin=16pt,labelindent=0pt]
            \item System inicjuje widok mapy z domyślnym obszarem.
            \item System pobiera listę spotów.
            \item System rysuje znaczniki spotów na mapie.
        \end{enumerate}
    }
    \ucalt{%
        \begin{enumerate}[nosep,leftmargin=21pt,labelindent=0pt,label={}]
            \item[2a.] Usługa mapy jest niedostępna – system wyświetla komunikat o błędzie.
        \end{enumerate}
    }
}

\usecasecard{tab:pu11-szczegoly-spota}{Otwarcie szczegółów spota}{%
    \ucpriority{Wysoki}
    \ucactors{Użytkownik niezalogowany, Użytkownik zalogowany}
    \ucdesc{Użytkownik otwiera widok szczegółów wybranego spota.}
    \ucpre{Użytkownik widzi mapę spotów.}
    \ucpost{Wyświetlony został widok szczegółów spota z podstawowymi informacjami oraz jego lokalizacją na mapie.}
    \ucmain{%
        \begin{enumerate}[nosep,leftmargin=16pt,labelindent=0pt]
            \item Użytkownik wybiera spota z mapy.
            \item System pobiera dane szczegółowe spota (informacje opisowe, lokalizacja).
            \item System otwiera widok szczegółów spota.
            \item System prezentuje informacje o spocie oraz mapę przybliżoną do jego lokalizacji.
        \end{enumerate}
    }
    \ucalt{%
        \begin{enumerate}[nosep,leftmargin=21pt,labelindent=0pt,label={}]
            \item[2a.] Spot nie istnieje (został usunięty lub ukryty) – system informuje użytkownika i powraca do poprzedniego widoku.
            \item[2b.] Wystąpił błąd podczas pobierania danych spota – system wyświetla komunikat o błędzie i umożliwia ponowną próbę.
        \end{enumerate}
    }
}

\usecasecard{tab:pu12-komentarze-spota}{Przeglądanie komentarzy do spota}{%
    \ucpriority{Średni}
    \ucactors{Użytkownik niezalogowany}
    \ucdesc{Użytkownik czyta komentarze pod wybranym spotem.}
    \ucpre{Wyświetlany jest widok szczegółów spota.}
    \ucpost{Lista komentarzy do spota została wyświetlona.}
    \ucmain{%
        \begin{enumerate}[nosep,leftmargin=16pt,labelindent=0pt]
            \item System pobiera komentarze powiązane ze spotem.
            \item System wyświetla komentarze w kolejności chronologicznej lub według popularności.
            \item Użytkownik przewija listę komentarzy.
        \end{enumerate}
    }
    \ucalt{%
        \begin{enumerate}[nosep,leftmargin=21pt,labelindent=0pt,label={}]
            \item[1a.] Spot nie ma jeszcze komentarzy – system wyświetla odpowiednią informację.
        \end{enumerate}
    }
}

\usecasecard{tab:pu13-pogoda}{Przeglądanie pogody na spocie}{%
    \ucpriority{Średni}
    \ucactors{Użytkownik zalogowany, Usługa danych pogodowych}
    \ucdesc{Użytkownik sprawdza prognozę pogody dla lokalizacji spota.}
    \ucpre{Wyświetlany jest widok szczegółów spota.}
    \ucpost{Prognoza pogody dla spota została wyświetlona.}
    \ucmain{%
        \begin{enumerate}[nosep,leftmargin=16pt,labelindent=0pt]
            \item Użytkownik otwiera zakładkę pogody.
            \item System wysyła zapytanie do usługi pogodowej z lokalizacją spota.
            \item System odbiera prognozę i prezentuje ją (temperatura, prędkość wiatru, opady).
        \end{enumerate}
    }
    \ucalt{%
        \begin{enumerate}[nosep,leftmargin=21pt,labelindent=0pt,label={}]
            \item[2a.] Usługa pogodowa jest niedostępna – system wyświetla komunikat o braku danych pogodowych.
        \end{enumerate}
    }
}

\usecasecard{tab:pu9-szukaj-na-mapie}{Wyszukiwanie spota na mapie}{%
    \ucpriority{Wysoki}
    \ucactors{Użytkownik niezalogowany}
    \ucdesc{Użytkownik wyszukuje spota po nazwie korzystając z pola wyszukiwania na mapie.}
    \ucpre{Użytkownik widzi mapę spotów.}
    \ucpost{Mapa zostaje ustawiona na wybranego spota lub listę dopasowań.}
    \ucmain{%
        \begin{enumerate}[nosep,leftmargin=16pt,labelindent=0pt]
            \item Użytkownik wpisuje frazę w polu wyszukiwania na mapie.
            \item System podpowiada listę pasujących spotów.
            \item Użytkownik wybiera spota z listy.
            \item System przenosi użytkownika na mapie do wybranego spota.
        \end{enumerate}
    }
    \ucalt{%
        \begin{enumerate}[nosep,leftmargin=21pt,labelindent=0pt,label={}]
            \item[2a.] Brak wyników dla podanej frazy – system informuje użytkownika o braku dopasowań.
        \end{enumerate}
    }
}
