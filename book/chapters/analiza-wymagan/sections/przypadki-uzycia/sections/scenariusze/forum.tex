%! Author = Adam
%! Date = 22/11/2025

\subsubsection{Scenariusze przypadków użycia dla forum}

\usecasecard{tab:pu14-posty-forum}{Przeglądanie postów na forum}{%
    \ucpriority{S}
    \ucactors{Użytkownik}
    \ucdesc{Użytkownik przegląda listę postów na forum z możliwością sortowania wyników.}
    \ucpre{Użytkownik znajduje się w module forum.}
    \ucpost{Lista postów forum jest wyświetlona, a użytkownik może przechodzić do szczegółów wybranego posta.}
    \ucmain{%
        \begin{enumerate}[nosep,leftmargin=16pt,labelindent=0pt]
            \item Użytkownik przechodzi do widoku listy postów forum.
            \item System pobiera listę postów i domyślnie wyświetla je w kolejności od najnowszych.
            \item Użytkownik wybiera sposób sortowania listy.
            \item System aktualizuje listę postów zgodnie z wybranym kryterium sortowania.
            \item Użytkownik wybiera post, który chce przeczytać.
            \item System otwiera szczegółowy widok wybranego posta.
        \end{enumerate}
    }
    \ucalt{%
        \begin{enumerate}[nosep,leftmargin=21pt,labelindent=0pt,label={}]
            \item[6a.] System nie może pobrać szczegółów posta – system wyświetla komunikat o błędzie.
        \end{enumerate}
    }
}

\usecasecard{tab:pu-forum-wyszukiwanie-postow}{Wyszukiwanie postów na forum}{%
    \ucpriority{S}
    \ucactors{Użytkownik}
    \ucdesc{Użytkownik wyszukuje posty na forum na podstawie tytułu, kategorii, tagów oraz autora.}
    \ucpre{Użytkownik znajduje się w module forum lub na stronie wyszukiwarki postów.}
    \ucpost{Lista postów spełniających zadane kryteria wyszukiwania jest wyświetlona.}
    \ucmain{%
        \begin{enumerate}[nosep,leftmargin=16pt,labelindent=0pt]
            \item Użytkownik otwiera panel wyszukiwania postów na forum.
            \item Użytkownik określa kryteria wyszukiwania (np. tytuł, kategoria, tagi, autor).
            \item Użytkownik uruchamia wyszukiwanie.
            \item System filtruje posty zgodnie z podanymi kryteriami i wyświetla listę wyników.
        \end{enumerate}
    }
    \ucalt{%
        \begin{enumerate}[nosep,leftmargin=21pt,labelindent=0pt,label={}]
            \item[4a.] Brak postów spełniających zadane kryteria – system wyświetla informację o braku wyników.
            \item[4b.] Wystąpił błąd podczas wyszukiwania – system wyświetla komunikat o błędzie i umożliwia ponowną próbę.
        \end{enumerate}
    }
}

\usecasecard{tab:pu15-dodaj-post}{Dodanie posta na forum}{%
    \ucpriority{S}
    \ucactors{Użytkownik, Usługa do przechowywania plików w chmurze}
    \ucdesc{Użytkownik publikuje nowy post na forum, określając jego treść, kategorię, tagi oraz opcjonalne załączniki.}
    \ucpre{Użytkownik znajduje się w module forum.}
    \ucpost{Nowy post jest poprawnie zapisany i widoczny na forum.}
    \ucmain{%
        \begin{enumerate}[nosep,leftmargin=16pt,labelindent=0pt]
            \item Użytkownik wybiera opcję dodania nowego posta.
            \item Użytkownik wpisuje tytuł i treść posta.
            \item Użytkownik wybiera kategorię posta.
            \item (Opcjonalnie) Użytkownik wybiera tagi przypisane do posta.
            \item (Opcjonalnie) Użytkownik dodaje załączniki (zdjęcia/filmy) do posta.
            \item Użytkownik publikuje posta.
            \item System zapisuje posta (oraz poprawne załączniki w chmurze) i wyświetla go na liście postów.
        \end{enumerate}
    }
    \ucalt{%
        \begin{enumerate}[nosep,leftmargin=21pt,labelindent=0pt,label={}]
            \item[6a.] Załącznik nie może zostać zapisany lub nie spełnia wymagań (np. zbyt duży rozmiar, nieobsługiwany format) – system informuje użytkownika o błędzie, blokuje publikację posta i wymaga usunięcia lub podmiany problematycznego pliku.
            \item[6b.] Formularz zawiera błędne lub niekompletne dane (np. brak tytułu lub treści) – system wyświetla komunikat i prosi o poprawę danych przed publikacją.
        \end{enumerate}
    }
}

\usecasecard{tab:pu16-dodaj-komentarz}{Dodanie komentarza na forum}{%
    \ucpriority{S}
    \ucactors{Użytkownik}
    \ucdesc{Użytkownik dodaje komentarz pod postem na forum, opcjonalnie z załącznikami (zdjęcia/filmy).}
    \ucpre{Użytkownik jest zalogowany i widzi szczegóły posta.}
    \ucpost{Nowy komentarz został zapisany i jest widoczny pod postem.}
    \ucmain{%
        \begin{enumerate}[nosep,leftmargin=16pt,labelindent=0pt]
            \item Użytkownik wpisuje treść komentarza w formularzu pod postem.
            \item (Opcjonalnie) Użytkownik dodaje załączniki (zdjęcia/filmy) do komentarza.
            \item Użytkownik publikuje komentarz.
            \item System zapisuje komentarz (oraz poprawne załączniki) i odświeża listę komentarzy.
        \end{enumerate}
    }
    \ucalt{%
        \begin{enumerate}[nosep,leftmargin=21pt,labelindent=0pt,label={}]
            \item[3a.] Treść komentarza lub załączniki są niepoprawne (np. naruszają walidację) – system wyświetla komunikat o błędzie i blokuje publikację do czasu poprawy danych.
        \end{enumerate}
    }
}

\usecasecard{tab:pu17-historia-postow}{Przeglądanie historii interakcji z postami}{%
    \ucpriority{S}
    \ucactors{Użytkownik}
    \ucdesc{Użytkownik przegląda historię swoich aktywności na forum (dodane posty, komentarze, reakcje).}
    \ucpre{Użytkownik jest zalogowany.}
    \ucpost{Lista interakcji użytkownika z postami jest wyświetlona.}
    \ucmain{%
        \begin{enumerate}[nosep,leftmargin=16pt,labelindent=0pt]
            \item Użytkownik przechodzi do sekcji historii aktywności.
            \item System pobiera historię interakcji użytkownika.
            \item System wyświetla listę interakcji z możliwością filtrowania.
        \end{enumerate}
    }
    \ucalt{Brak istotnych alternatywnych przepływów.}
}

\usecasecard{tab:pu33-zarzadzaj-komentarzami-forum}{Zarządzanie komentarzami na forum}{%
    \ucpriority{C}
    \ucactors{Użytkownik}
    \ucdesc{Użytkownik zarządza komentarzami pod postami forum (edycja, usuwanie, zgłaszanie komentarzy innych użytkowników).}
    \ucpre{Użytkownik jest zalogowany i ma dostęp do danego wątku forum.}
    \ucpost{Komentarze zostały zaktualizowane zgodnie z działaniami użytkownika.}
    \ucmain{%
        \begin{enumerate}[nosep,leftmargin=16pt,labelindent=0pt]
            \item Użytkownik otwiera widok komentarzy pod postem.
            \item Użytkownik wybiera komentarz i odpowiednią akcję (edycja, usunięcie, zgłoszenie).
            \item System weryfikuje uprawnienia użytkownika oraz zgodność akcji z jego rolą.
            \item System wykonuje wybraną akcję (np. zapisuje zmiany, usuwa komentarz lub przygotowuje zgłoszenie) i aktualizuje widok.
        \end{enumerate}
    }
    \ucalt{%
        \begin{enumerate}[nosep,leftmargin=21pt,labelindent=0pt,label={}]
            \item[2a.] Użytkownik próbuje zgłosić własny komentarz – system blokuje operację i informuje, że nie można zgłaszać własnych treści.
            \item[3a.] Użytkownik nie ma wymaganych uprawnień do wykonania wybranej akcji – system blokuje operację i informuje o braku uprawnień.
        \end{enumerate}
    }
}

\usecasecard{tab:pu35-zglos-komentarz}{Zgłoszenie komentarza naruszającego regulamin}{%
    \ucpriority{C}
    \ucactors{Użytkownik}
    \ucdesc{Użytkownik zgłasza komentarz na forum jako naruszający regulamin.}
    \ucpre{Użytkownik widzi komentarz w aplikacji.}
    \ucpost{Zgłoszenie komentarza zostało zapisane i trafiło do kolejki moderacyjnej.}
    \ucmain{%
        \begin{enumerate}[nosep,leftmargin=16pt,labelindent=0pt]
            \item Użytkownik wybiera opcję „Zgłoś komentarz”.
            \item Użytkownik wybiera kategorię naruszenia, podaje szczegóły i potwierdza zgłoszenie.
            \item System zapisuje zgłoszenie i wiąże je z komentarzem oraz zgłaszającym użytkownikiem.
        \end{enumerate}
    }
    \ucalt{%
        \begin{enumerate}[nosep,leftmargin=21pt,labelindent=0pt,label={}]
            \item[1a.] Użytkownik próbuje zgłosić własny komentarz – system blokuje operację i wyświetla komunikat, że nie można zgłaszać własnych treści.
            \item[3a.] Komentarz został już wcześniej zgłoszony – system informuje użytkownika, że komentarz znajduje się już w kolejce moderacyjnej i nie zapisuje kolejnego zgłoszenia.
        \end{enumerate}
    }
}

\usecasecard{tab:pu36-zglos-posta}{Zgłoszenie posta na forum}{%
    \ucpriority{C}
    \ucactors{Użytkownik}
    \ucdesc{Użytkownik zgłasza post forum jako naruszający regulamin lub tematykę.}
    \ucpre{Wyświetlony jest widok posta na forum.}
    \ucpost{Zgłoszenie posta zostało zapisane}
    \ucmain{%
        \begin{enumerate}[nosep,leftmargin=16pt,labelindent=0pt]
            \item Użytkownik wybiera opcję „Zgłoś post”.
            \item Użytkownik wybiera kategorię naruszenia, podaje szczegóły i potwierdza zgłoszenie.
            \item System zapisuje zgłoszenie i oznacza post jako zgłoszony.
        \end{enumerate}
    }
    \ucalt{%
        \begin{enumerate}[nosep,leftmargin=21pt,labelindent=0pt,label={}]
            \item[1a.] Użytkownik próbuje zgłosić własny post – system blokuje operację i wyświetla komunikat, że nie można zgłaszać własnych treści.
            \item[3a.] Post został już wcześniej zgłoszony – system informuje użytkownika, że post jest już zgłoszony i nie zapisuje kolejnego zgłoszenia.
        \end{enumerate}
    }
}


\usecasecard{tab:pu-forum-przegladaj-komentarze}{Przeglądanie komentarzy pod postem}{%
    \ucpriority{S}
    \ucactors{Użytkownik}
    \ucdesc{Użytkownik przegląda komentarze dodane pod wybranym postem na forum z możliwością zmiany kolejności ich wyświetlania.}
    \ucpre{Wyświetlany jest szczegółowy widok posta na forum.}
    \ucpost{Lista komentarzy powiązanych z postem została wyświetlona zgodnie z wybranym kryterium sortowania.}
    \ucmain{%
        \begin{enumerate}[nosep,leftmargin=16pt,labelindent=0pt]
            \item System pobiera komentarze powiązane z wybranym postem i domyślnie wyświetla je w kolejności od najnowszych.
            \item Użytkownik wybiera sposób sortowania komentarzy.
            \item System aktualizuje listę komentarzy zgodnie z wybranym kryterium sortowania.
            \item Użytkownik przewija listę komentarzy i zapoznaje się z ich treścią.
        \end{enumerate}
    }
    \ucalt{%
        \begin{enumerate}[nosep,leftmargin=21pt,labelindent=0pt,label={}]
            \item[1a.] Post nie ma jeszcze komentarzy – system wyświetla informację o braku komentarzy.
            \item[1b.] Wystąpił błąd podczas pobierania komentarzy – system wyświetla komunikat o błędzie i umożliwia ponowną próbę.
        \end{enumerate}
    }
}
