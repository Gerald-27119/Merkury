%! Author = Adam
%! Date = 22/11/2025

\subsubsection{Scenariusze przypadków użycia dla panelu użytkownika}

\usecasecard{tab:pu27-dodaj-spota}{Dodanie spota w panelu użytkownika}{%
    \ucpriority{M}
    \ucactors{Użytkownik, Usługa do wyświetlania mapy, Usługa do przechowywania plików w chmurze}
    \ucdesc{Użytkownik dodaje nowy spot poprzez panel.}
    \ucpre{Użytkownik jest zalogowany i znajduje się w widoku panelu użytkownika.}
    \ucpost{Nowy spot został zapisany i jest widoczny na mapie oraz w panelu użytkownika.}
    \ucmain{%
        \begin{enumerate}[nosep,leftmargin=16pt,labelindent=0pt]
            \item Użytkownik wybiera opcję „Dodaj spota”.
            \item Użytkownik uzupełnia podstawowe informacje o spocie (nazwa, opis, tagi).
            \item Użytkownik wskazuje lokalizację spota na mapie.
            \item Użytkownik dodaje zdjęcia/filmy do spota.
            \item Użytkownik zapisuje spota.
            \item System zapisuje dane spota (oraz pliki w chmurze) i aktualizuje mapę.
        \end{enumerate}
    }
    \ucalt{%
        \begin{enumerate}[nosep,leftmargin=21pt,labelindent=0pt,label={}]
            \item[3a.] Podane dane wejściowe są niepoprawne – system wyświetla komunikat i zaznacza wymagające poprawy pola.
        \end{enumerate}
    }
}

\usecasecard{tab:pu28-profil-wlasny}{Przeglądanie profilu użytkownika}{%
    \ucpriority{M}
    \ucactors{Użytkownik}
    \ucdesc{Użytkownik przegląda profil (lista spotów, media, podstawowe dane).}
    \ucpre{Brak}
    \ucpost{Wyświetlony jest widok profilu użytkownika wraz z jego zawartością.}
    \ucmain{%
        \begin{enumerate}[nosep,leftmargin=16pt,labelindent=0pt]
            \item Użytkownik otwiera profil.
            \item System pobiera dane profilu (informacje podstawowe, spoty, media).
            \item System wyświetla dane w odpowiednich sekcjach (spoty, zdjęcia, filmy, komentarze).
        \end{enumerate}
    }
    \ucalt{%
        \begin{enumerate}[nosep,leftmargin=21pt,labelindent=0pt,label={}]
            \item[2a.] Wystąpił błąd podczas pobierania danych użytkownika – system wyświetla informację o błędzie.
        \end{enumerate}
    }
}

\usecasecard{tab:pu30-dodaj-znajomego}{Dodanie użytkownika do znajomych}{%
    \ucpriority{M}
    \ucactors{Użytkownik}
    \ucdesc{Użytkownik wysyła lub akceptuje zaproszenie do znajomych.}
    \ucpre{Użytkownik jest zalogowany i przegląda profil innego użytkownika.}
    \ucpost{Relacja „znajomy” została utworzona lub zaproszenie czeka na akceptację.}
    \ucmain{%
        \begin{enumerate}[nosep,leftmargin=16pt,labelindent=0pt]
            \item Użytkownik klika przycisk „Dodaj do znajomych”.
            \item System sprawdza, czy relacja już istnieje.
            \item System tworzy nowe zaproszenie.
            \item System informuje o statusie o wysłaniu zaproszenia.
        \end{enumerate}
    }
    \ucalt{Brak istotnych alternatywnych przepływów.}
}

\usecasecard{tab:pu31-spolecznosci}{Przeglądanie społeczności}{%
    \ucpriority{M}
    \ucactors{Użytkownik}
    \ucdesc{Użytkownik przegląda społeczności, grupy lub listy znajomych powiązane z aplikacją.}
    \ucpre{Użytkownik jest zalogowany.}
    \ucpost{Lista społeczności lub znajomych została wyświetlona.}
    \ucmain{%
        \begin{enumerate}[nosep,leftmargin=16pt,labelindent=0pt]
            \item Użytkownik przechodzi do sekcji społeczności.
            \item System pobiera listę społeczności i znajomych użytkownika.
            \item System wyświetla listę z możliwością przechodzenia do profili i czatów.
        \end{enumerate}
    }
    \ucalt{%
        \begin{enumerate}[nosep,leftmargin=21pt,labelindent=0pt,label={}]
            \item[2a.] Nie udało się pobrać danych – system wyświetla komunikat o błędzie.
        \end{enumerate}
    }
}

\usecasecard{tab:pu39-przegladaj-dodane-spoty}{Przeglądanie dodanych spotów}{%
    \ucpriority{M}
    \ucactors{Użytkownik, Usługa do wyświetlania mapy}
    \ucdesc{Użytkownik przegląda listę/siatkę spotów, które sam dodał.}
    \ucpre{Użytkownik jest zalogowany i znajduje się w widoku panelu użytkownika lub sekcji „Moje spoty”.}
    \ucpost{Lista dodanych spotów użytkownika została wyświetlona (np. na mapie i/lub w formie listy).}
    \ucmain{%
        \begin{enumerate}[nosep,leftmargin=16pt,labelindent=0pt]
            \item Użytkownik przechodzi do sekcji „Moje spoty”.
            \item System pobiera listę spotów dodanych przez użytkownika.
            \item System wyświetla listę spotów oraz znaczniki na mapie.
            \item Użytkownik wybiera spota, aby przejść do jego szczegółów.
        \end{enumerate}
    }
    \ucalt{%
        \begin{enumerate}[nosep,leftmargin=21pt,labelindent=0pt,label={}]
            \item[2a.] Użytkownik nie dodał jeszcze żadnego spota – system wyświetla komunikat i proponuje dodanie pierwszego spota.
        \end{enumerate}
    }
}

\usecasecard{tab:pu40-edytuj-dane-uzytkownika}{Edycja danych użytkownika}{%
    \ucpriority{M}
    \ucactors{Użytkownik}
    \ucdesc{Użytkownik modyfikuje swoje dane profilu (np. nazwę, opis, avatar).}
    \ucpre{Użytkownik jest zalogowany i znajduje się w widoku edycji profilu.}
    \ucpost{Zaktualizowane dane profilu są zapisane i widoczne w aplikacji.}
    \ucmain{%
        \begin{enumerate}[nosep,leftmargin=16pt,labelindent=0pt]
            \item Użytkownik otwiera widok edycji profilu.
            \item Użytkownik wprowadza zmiany w danych profilu.
            \item Użytkownik zapisuje zmiany.
            \item System waliduje dane i zapisuje zaktualizowany profil.
        \end{enumerate}
    }
    \ucalt{%
        \begin{enumerate}[nosep,leftmargin=21pt,labelindent=0pt,label={}]
            \item[4a.] Dane są niepoprawne lub niekompletne – system wyświetla komunikat o błędzie i zaznacza pola do poprawy.
        \end{enumerate}
    }
}

\usecasecard{tab:pu41-przegladaj-media-spotow}{Przeglądanie dodanych mediów do spotów}{%
    \ucpriority{M}
    \ucactors{Użytkownik, Usługa do przechowywania plików w chmurze}
    \ucdesc{Użytkownik przegląda media (zdjęcia i filmy) powiązane ze spotami, które dodał.}
    \ucpre{Użytkownik jest zalogowany i znajduje się w sekcji mediów.}
    \ucpost{Lista lub galeria mediów powiązanych ze spotami użytkownika została wyświetlona; w przypadku filmów możliwe jest ich odtworzenie.}
    \ucmain{%
        \begin{enumerate}[nosep,leftmargin=16pt,labelindent=0pt]
            \item Użytkownik otwiera sekcję przeglądania mediów dodanych do spotów.
            \item System pobiera metadane mediów z usługi przechowywania plików.
            \item System wyświetla listę/galerię mediów z podstawowymi informacjami (typ medium, data dodania).
            \item Użytkownik wybiera element listy; w przypadku filmu system uruchamia jego odtwarzanie.
        \end{enumerate}
    }
    \ucalt{%
        \begin{enumerate}[nosep,leftmargin=21pt,labelindent=0pt,label={}]
            \item[2a.] Użytkownik nie dodał jeszcze żadnych mediów – system wyświetla informację o braku mediów.
            \item[2b.] Nie udało się pobrać danych z usługi przechowywania plików – system wyświetla komunikat o błędzie.
        \end{enumerate}
    }
}

\usecasecard{tab:pu43-przegladaj-komentarze-spotow}{Przeglądanie dodanych komentarzy do spotów}{%
    \ucpriority{M}
    \ucactors{Użytkownik}
    \ucdesc{Użytkownik przegląda komentarze dodane do spotów, które sam utworzył.}
    \ucpre{Użytkownik jest zalogowany i otwiera sekcję komentarzy do swoich spotów.}
    \ucpost{Lista komentarzy do spotów użytkownika została wyświetlona.}
    \ucmain{%
        \begin{enumerate}[nosep,leftmargin=16pt,labelindent=0pt]
            \item Użytkownik przechodzi do sekcji komentarzy do własnych spotów.
            \item System pobiera komentarze powiązane ze spotami użytkownika.
            \item System wyświetla komentarze (np. w kolejności chronologicznej).
        \end{enumerate}
    }
    \ucalt{%
        \begin{enumerate}[nosep,leftmargin=21pt,labelindent=0pt,label={}]
            \item[2a.] Żaden z spotów użytkownika nie ma komentarzy – system wyświetla odpowiednią informację.
            \item[3a.] Nie udało się pobrać komentarzy – system wyświetla komunikat o błędzie.
        \end{enumerate}
    }
}

