%%! Author = Adam
%%! Date = 22/11/2025

\subsection{Scenariusze przypadków użycia}
\label{subsec:scenariusz-przypadkow-uzycia}

Niniejszy rozdział zawiera scenariusze przypadków użycia.
Zostały one wykonane dla wybranych przypadków użycia.
\\
Każdy ze scenariuszy posiada jeden z trzech możliwych priorytetów implementacji: wysoki, średni albo niski.

% --- Licznik dla scenariuszy przypadków użycia ---
\newcounter{usecase}[chapter]
\renewcommand{\theusecase}{\thechapter.\arabic{usecase}}

\newcommand{\ucpriority}[1]{\textbf{Priorytet:} & #1 \\ \hline}
\newcommand{\ucactors}[1]{\textbf{Aktorzy:} & #1 \\ \hline}
\newcommand{\ucdesc}[1]{\textbf{Opis:} & #1 \\ \hline}
\newcommand{\ucpre}[1]{\textbf{Warunki\newline wstępne:} & #1 \\ \hline}
\newcommand{\ucpost}[1]{\textbf{Warunki\newline końcowe:} & #1 \\ \hline}
\newcommand{\ucmain}[1]{\textbf{Główny\newline przepływ\newline zdarzeń:} & #1 \\ \hline}
\newcommand{\ucalt}[1]{\textbf{Alternatywne\newline przepływy\newline zdarzeń:} & #1 \\ \hline}

% --- Karta scenariusza PU ---
\newcommand{\usecasecard}[4]{%
    \refstepcounter{usecase}%
    \begin{center}
    \renewcommand{\arraystretch}{1.15}%
    \begin{tabularx}{\textwidth}{|>{\columncolor{lightgray}}p{0.20\textwidth}|X|}
    \rowcolor{lightgray}
    \multicolumn{2}{|c|}{\textbf{KARTA SCENARIUSZA PRZYPADKU UŻYCIA}} \\ \hline
    \textbf{Identyfikator:} & #3 \\ \hline
    \textbf{Nazwa:}         & #2 \\ \hline
    #4
    \end{tabularx}
    \vspace{3pt}
    \textbf{Tabela \theusecase:} Scenariusz przypadku użycia: #2\label{#1}
    \end{center}%
    \addcontentsline{lot}{table}{Tabela \theusecase: Scenariusz przypadku użycia: #2}%
}

% Scenariusze PU:

\usecasecard{tab:pu1-rejestracja}{Rejestracja użytkownika}{PU1}{%
    \ucpriority{Wysoki}
    \ucactors{Użytkownik niezalogowany}
    \ucdesc{Użytkownik zakłada konto poprzez formularz rejestracji.}
    \ucpre{Użytkownik znajduje się na stronie z formularzem rejestracji.}
    \ucpost{Użytkownik posiada konto w systemie.}
    \ucmain{%
        \begin{enumerate}[nosep,leftmargin=16pt,labelindent=0pt]
            \item Użytkownik wypełnia formularz rejestracyjny.
            \item Użytkownik naciska przycisk rejestracji.
            \item System tworzy konto użytkownika.
            \item System loguje użytkownika i przenosi go na stronę główną aplikacji.
        \end{enumerate}
    }
    \ucalt{%
        \begin{enumerate}[nosep,leftmargin=21pt,labelindent=0pt,label={}]
            \item[1a.] Podane dane są niepoprawne – system wyświetla
            komunikat o błędzie oraz podświetla pola wymagające poprawy.
            \item[2a.] Nazwa użytkownika jest już zajęta – system wyświetla
            komunikat o błędzie.
        \end{enumerate}
    }
}

\usecasecard{tab:pu2-logowanie}{Logowanie użytkownika}{PU2}{%
    \ucpriority{Wysoki}
    \ucactors{Użytkownik niezalogowany}
    \ucdesc{Użytkownik loguje się do systemu, podając login i hasło.}
    \ucpre{Użytkownik znajduje się na stronie logowania.}
    \ucpost{Użytkownik jest zalogowany i przeniesiony na stronę główną aplikacji.}
    \ucmain{%
        \begin{enumerate}[nosep,leftmargin=16pt,labelindent=0pt]
            \item Użytkownik wypełnia formularz logowania.
            \item Użytkownik naciska przycisk logowania.
            \item System loguje użytkownika i przenosi go na ostatnią odwiedzoną przez niego stronę aplikacji.
        \end{enumerate}
    }
    \ucalt{%
        \begin{enumerate}[nosep,leftmargin=21pt,labelindent=0pt,label={}]
            \item[2a.] Podane dane są niepoprawne – system wyświetla komunikat o błędzie.
        \end{enumerate}
    }
}

\usecasecard{tab:pu3-reset-hasla}{Resetowanie hasła}{PU3}{%
    \ucpriority{Wysoki}
    \ucactors{Użytkownik niezalogowany, Usługa SMTP}
    \ucdesc{Użytkownik inicjuje reset hasła, aby odzyskać dostęp do konta.}
    \ucpre{Użytkownik znajduje się na ekranie resetu hasła.}
    \ucpost{Użytkownik otrzymuje wiadomość e-mail z linkiem do ustawienia nowego hasła.}
    \ucmain{%
        \begin{enumerate}[nosep,leftmargin=16pt,labelindent=0pt]
            \item Użytkownik wpisuje adres e-mail powiązany z kontem.
            \item Użytkownik zatwierdza żądanie resetu hasła.
            \item System generuje token resetu hasła.
            \item System wysyła e-mail z linkiem do zmiany hasła.
        \end{enumerate}
    }
    \ucalt{%
        \begin{enumerate}[nosep,leftmargin=21pt,labelindent=0pt,label={}]
            \item[2a.] Nie istnieje konto dla podanego adresu – system wyświetla komunikat o błędzie.
            \item[4a.] Występuje błąd połączenia z usługą SMTP – system informuje użytkownika o problemie technicznym.
        \end{enumerate}
    }
}

%TODO sprwawdz porpawnosc
\usecasecard{tab:pu4-zmiana-hasla}{Zmiana hasła w ustawieniach konta}{PU4}{%
    \ucpriority{Wysoki}
    \ucactors{Użytkownik zalogowany}
    \ucdesc{Użytkownik zmienia hasło do konta z poziomu ustawień profilu.}
    \ucpre{Użytkownik jest zalogowany i znajduje się na ekranie zmiany danych konta.}
    \ucpost{Hasło do konta użytkownika zostało zaktualizowane.}
    \ucmain{%
        \begin{enumerate}[nosep,leftmargin=16pt,labelindent=0pt]
            \item Użytkownik wpisuje aktualne hasło.
            \item Użytkownik wpisuje nowe hasło i powtarza je.
            \item Użytkownik zatwierdza formularz zmiany hasła.
            \item System zapisuje nowe hasło i informuje o powodzeniu operacji.
        \end{enumerate}
    }
    \ucalt{%
        \begin{enumerate}[nosep,leftmargin=21pt,labelindent=0pt,label={}]
            \item[3a.] Aktualne hasło jest nieprawidłowe – system wyświetla komunikat i nie zapisuje zmian.
            \item[3b.] Nowe hasło nie spełnia wymagań bezpieczeństwa – system informuje o błędzie i podświetla pola do poprawy.
        \end{enumerate}
    }
}

\usecasecard{tab:pu5-wylogowanie}{Wylogowanie użytkownika}{PU5}{%
    \ucpriority{Wysoki}
    \ucactors{Użytkownik zalogowany}
    \ucdesc{Użytkownik wylogowuje się z aplikacji.}
    \ucpre{Użytkownik jest zalogowany.}
    \ucpost{Sesja użytkownika została zakończona, użytkownik widzi stronę główną dla niezalogowanych.}
    \ucmain{%
        \begin{enumerate}[nosep,leftmargin=16pt,labelindent=0pt]
            \item Użytkownik wybiera opcję wylogowania z menu.
            \item System unieważnia token dostępu użytkownika.
            \item System przenosi użytkownika na stronę główną aplikacji.
        \end{enumerate}
    }
    \ucalt{Brak istotnych alternatywnych przepływów.}
}

\usecasecard{tab:pu6-powiadomienia}{Przeglądanie powiadomień}{PU6}{%
    \ucpriority{Niski}
    \ucactors{Użytkownik zalogowany}
    \ucdesc{Użytkownik przegląda listę powiadomień.}
    \ucpre{Użytkownik jest na ekranie centra powiadomień.}
    \ucpost{Powiadomienia zostały wyświetlone, a wybrane oznaczone jako przeczytane.}
    \ucmain{%
        \begin{enumerate}[nosep,leftmargin=16pt,labelindent=0pt]
            \item System wyświetla powiadomienia w odwróconym porządku chronologicznym.
            \item Użytkownik otwiera wybrane powiadomienie.
            \item System oznacza powiadomienie jako przeczytane i ewentualnie przenosi użytkownika do powiązanego widoku.
        \end{enumerate}
    }
    \ucalt{%
        \begin{enumerate}[nosep,leftmargin=21pt,labelindent=0pt,label={}]
            \item[1a.] System nie może pobrać powiadomień (błąd serwera) – użytkownik otrzymuje komunikat o błędzie i może spróbować ponownie.
        \end{enumerate}
    }
}

\usecasecard{tab:pu7-subskrypcja}{Wykupienie subskrypcji premium}{PU7}{%
    \ucpriority{Niski}
    \ucactors{Użytkownik zalogowany, Bramka płatnicza, System finansowo-księgowy}
    \ucdesc{Użytkownik opłaca subskrypcję premium w celu uzyskania dodatkowych funkcji.}
    \ucpre{Użytkownik jest zalogowany i znajduje się w module subskrypcji.}
    \ucpost{Subskrypcja premium jest aktywna, a użytkownik ma dostęp do funkcji premium.}
    \ucmain{%
        \begin{enumerate}[nosep,leftmargin=16pt,labelindent=0pt]
            \item Użytkownik wybiera plan subskrypcji.
            \item Użytkownik przechodzi do bramki płatniczej.
            \item Użytkownik podaje dane płatnicze i zatwierdza transakcję.
            \item Bramka płatnicza przetwarza płatność i zwraca wynik do systemu.
            \item System zapisuje informację o opłaconej subskrypcji i aktualizuje uprawnienia.
            \item System generuje wpis w systemie finansowo-księgowym.
        \end{enumerate}
    }
    \ucalt{%
        \begin{enumerate}[nosep,leftmargin=21pt,labelindent=0pt,label={}]
            \item[4a.] Płatność nie powiodła się – system informuje użytkownika i umożliwia ponowną próbę.
            \item[5a.] W czasie aktualizacji subskrypcji wystąpił błąd – system cofa zmiany i wyświetla komunikat o problemie.
        \end{enumerate}
    }
}

\usecasecard{tab:pu8-mapa}{Przeglądanie mapy spotów}{PU8}{%
    \ucpriority{Wysoki}
    \ucactors{Użytkownik niezalogowany, Usługa do wyświetlania mapy}
    \ucdesc{Użytkownik przegląda mapę spotów.}
    \ucpre{Użytkownik znajduje się w module mapy.}
    \ucpost{Mapa ze spotami została wyświetlona, a użytkownik może przybliżać, oddalać i przesuwać widok.}
    \ucmain{%
        \begin{enumerate}[nosep,leftmargin=16pt,labelindent=0pt]
            \item System inicjuje widok mapy z domyślnym obszarem.
            \item System pobiera listę spotów w aktualnym zakresie mapy.
            \item System rysuje znaczniki spotów na mapie.
            \item Użytkownik przesuwa lub skaluje mapę.
            \item System pobiera spoty dla nowego zakresu.
        \end{enumerate}
    }
    \ucalt{%
        \begin{enumerate}[nosep,leftmargin=21pt,labelindent=0pt,label={}]
            \item[2a.] Usługa mapy jest niedostępna – system wyświetla komunikat o błędzie.
        \end{enumerate}
    }
}

\usecasecard{tab:pu9-szukaj-na-mapie}{Wyszukiwanie spota na mapie}{PU9}{%
    \ucpriority{Wysoki}
    \ucactors{Użytkownik niezalogowany}
    \ucdesc{Użytkownik wyszukuje spota po nazwie korzystając z pola wyszukiwania na mapie.}
    \ucpre{Użytkownik widzi mapę spotów.}
    \ucpost{Mapa zostaje ustawiona na wybranego spota lub listę dopasowań.}
    \ucmain{%
        \begin{enumerate}[nosep,leftmargin=16pt,labelindent=0pt]
            \item Użytkownik wpisuje frazę w polu wyszukiwania na mapie.
            \item System podpowiada listę pasujących spotów.
            \item Użytkownik wybiera spota z listy.
            \item System przybliża mapę do wybranego spota i podświetla jego znacznik.
        \end{enumerate}
    }
    \ucalt{%
        \begin{enumerate}[nosep,leftmargin=21pt,labelindent=0pt,label={}]
            \item[2a.] Brak wyników dla podanej frazy – system informuje użytkownika o braku dopasowań.
        \end{enumerate}
    }
}

\usecasecard{tab:pu10-globalna-wyszukiwarka}{Wyszukiwanie spota w globalnej wyszukiwarce}{PU10}{%
    \ucpriority{Wysoki}
    \ucactors{Użytkownik niezalogowany, Usługa do wyświetlania mapy, Usługa do pogody}
    \ucdesc{Użytkownik wyszukuje spoty za pomocą globalnej wyszukiwarki w aplikacji.}
    \ucpre{Użytkownik znajduje sie na stronie głównej z wyszukwiarką.}
    \ucpost{Użytkownik otrzymuje listę znalezionych spotów.}
    \ucmain{%
        \begin{enumerate}[nosep,leftmargin=16pt,labelindent=0pt]
            \item Użytkownik wpisuje frazę w globalnej wyszukiwarce.
            \item System wyszukuje spoty spełniające kryteria.
            \item System wyświetla listę wyników.
        \end{enumerate}
    }
    \ucalt{%
        \begin{enumerate}[nosep,leftmargin=21pt,labelindent=0pt,label={}]
            \item[3a.] Brak wyników – system wyświetla komunikat i proponuje zmianę kryteriów wyszukiwania.
        \end{enumerate}
    }
}

\usecasecard{tab:pu11-przejscie-z-wynikow}{Przejście do spota na mapie z wyszukiwarki}{PU11}{%
    \ucpriority{Wysoki}
    \ucactors{Użytkownik niezalogowany}
    \ucdesc{Użytkownik przechodzi z wyników wyszukiwarki do widoku mapy ustawionego na konkretny spot.}
    \ucpre{Wyświetlona jest lista wyników wyszukiwania spotów.}
    \ucpost{Mapa jest przybliżona do wybranego spota, a jego szczegóły są dostępne.}
    \ucmain{%
        \begin{enumerate}[nosep,leftmargin=16pt,labelindent=0pt]
            \item Użytkownik wybiera spota z listy wyników.
            \item System przełącza widok na moduł mapy.
            \item System ustawia mapę na lokalizację spota i otwiera jego szczegóły.
        \end{enumerate}
    }
    \ucalt{Brak istotnych alternatywnych przepływów.}
}

\usecasecard{tab:pu12-komentarze-spota}{Przeglądanie komentarzy do spota}{PU12}{%
    \ucpriority{Średni}
    \ucactors{Użytkownik niezalogowany}
    \ucdesc{Użytkownik czyta komentarze pod wybranym spotem.}
    \ucpre{Wyświetlany jest widok szczegółów spota.}
    \ucpost{Lista komentarzy do spota została wyświetlona.}
    \ucmain{%
        \begin{enumerate}[nosep,leftmargin=16pt,labelindent=0pt]
            \item System pobiera komentarze powiązane ze spotem.
            \item System wyświetla komentarze w kolejności chronologicznej lub według popularności.
            \item Użytkownik przewija listę komentarzy.
        \end{enumerate}
    }
    \ucalt{%
        \begin{enumerate}[nosep,leftmargin=21pt,labelindent=0pt,label={}]
            \item[1a.] Spot nie ma jeszcze komentarzy – system wyświetla odpowiednią informację.
        \end{enumerate}
    }
}

\usecasecard{tab:pu13-pogoda}{Przeglądanie pogody na spocie}{PU13}{%
    \ucpriority{Średni}
    \ucactors{Użytkownik zalogowany, Usługa danych pogodowych}
    \ucdesc{Użytkownik sprawdza prognozę pogody dla lokalizacji spota.}
    \ucpre{Wyświetlany jest widok szczegółów spota.}
    \ucpost{Prognoza pogody dla spota została wyświetlona.}
    \ucmain{%
        \begin{enumerate}[nosep,leftmargin=16pt,labelindent=0pt]
            \item Użytkownik otwiera zakładkę pogody.
            \item System wysyła zapytanie do usługi pogodowej z lokalizacją spota.
            \item System odbiera prognozę i prezentuje ją (temperatura, prędkość wiatru, opady).
        \end{enumerate}
    }
    \ucalt{%
        \begin{enumerate}[nosep,leftmargin=21pt,labelindent=0pt,label={}]
            \item[2a.] Usługa pogodowa jest niedostępna – system wyświetla komunikat o braku danych pogodowych.
        \end{enumerate}
    }
}

\usecasecard{tab:pu14-posty-forum}{Przeglądanie postów na forum}{PU14}{%
    \ucpriority{Wysoki}
    \ucactors{Użytkownik niezalogowany}
    \ucdesc{Użytkownik przegląda listę postów na forum.}
    \ucpre{Użytkownik znajduje się w module forum.}
    \ucpost{Lista postów forum jest wyświetlona, a użytkownik może przechodzić do szczegółów.}
    \ucmain{%
        \begin{enumerate}[nosep,leftmargin=16pt,labelindent=0pt]
            \item System pobiera listę postów.
            \item System wyświetla posty z podstawowymi informacjami.
            \item Użytkownik wybiera post, który chce przeczytać.
            \item System otwiera szczegółowy widok posta.
        \end{enumerate}
    }
    \ucalt{%
        \begin{enumerate}[nosep,leftmargin=21pt,labelindent=0pt,label={}]
            \item[3a.] System nie może pobrać szczegółów psota – system wyświetla komunikat o błędzie.
        \end{enumerate}
    }
}

\usecasecard{tab:pu15-dodaj-post}{Dodanie posta na forum}{PU15}{%
    \ucpriority{Wysoki}
    \ucactors{Użytkownik zalogowany, Usługa do przechowywania plików w chmurze}
    \ucdesc{Użytkownik publikuje nowy post na forum.}
    \ucpre{Użytkownik znajduje się w module forum.}
    \ucpost{Nowy post jest widoczny na forum.}
    \ucmain{%
        \begin{enumerate}[nosep,leftmargin=16pt,labelindent=0pt]
            \item Użytkownik wybiera opcję dodania nowego posta.
            \item Użytkownik wpisuje tytuł i treść posta.
            \item (Opcjonalnie) Użytkownik dodaje załączniki (zdjęcia/filmy) do posta.
            \item Użytkownik publikuje posta.
            \item System zapisuje posta (oraz załączniki w chmurze) i wyświetla go na liście postów.
        \end{enumerate}
    }
    \ucalt{%
        \begin{enumerate}[nosep,leftmargin=21pt,labelindent=0pt,label={}]
            \item[3a.] Załącznik nie może zostać zapisany – system informuje o błędzie i pozwala opublikować posta bez pliku.
            \item[4a.] Formularz zawiera błędne lub niekompletne dane – system wyświetla komunikat i prosi o poprawę.
        \end{enumerate}
    }
}

\usecasecard{tab:pu16-dodaj-komentarz}{Dodanie komentarza na forum}{PU16}{%
    \ucpriority{Wysoki}
    \ucactors{Użytkownik zalogowany}
    \ucdesc{Użytkownik dodaje komentarz pod postem na forum.}
    \ucpre{Użytkownik jest zalogowany i widzi szczegóły posta.}
    \ucpost{Nowy komentarz został zapisany i widoczny pod postem.}
    \ucmain{%
        \begin{enumerate}[nosep,leftmargin=16pt,labelindent=0pt]
            \item Użytkownik wpisuje treść komentarza w formularzu pod postem.
            \item Użytkownik publikuje komentarz.
            \item System zapisuje komentarz i odświeża listę komentarzy.
        \end{enumerate}
    }
    \ucalt{%
        \begin{enumerate}[nosep,leftmargin=21pt,labelindent=0pt,label={}]
            \item[2a.] Treść komentarza jest niepoprawa – system wyświetla komunikat o błędzie.
        \end{enumerate}
    }
}

\usecasecard{tab:pu17-historia-postow}{Przeglądanie historii interakcji z postami}{PU17}{%
    \ucpriority{Średni}
    \ucactors{Użytkownik zalogowany}
    \ucdesc{Użytkownik przegląda historię swoich aktywności na forum (dodane posty, komentarze, reakcje).}
    \ucpre{Użytkownik jest zalogowany.}
    \ucpost{Lista interakcji użytkownika z postami jest wyświetlona.}
    \ucmain{%
        \begin{enumerate}[nosep,leftmargin=16pt,labelindent=0pt]
            \item Użytkownik przechodzi do sekcji historii aktywności.
            \item System pobiera historię interakcji użytkownika.
            \item System wyświetla listę interakcji z możliwością filtrowania.
        \end{enumerate}
    }
    \ucalt{Brak istotnych alternatywnych przepływów.}
}

\usecasecard{tab:pu18-czat-prywatny}{Utworzenie prywatnego czatu}{PU18}{%
    \ucpriority{Wysoki}
    \ucactors{Użytkownik zalogowany}
    \ucdesc{Użytkownik tworzy prywatną konwersację z innym użytkownikiem.}
    \ucpre{Użytkownik jest zalogowany i znajduje się w zakładce społeczność.}
    \ucpost{Nowy czat prywatny został utworzony i wyświetlony użytkownikowi.}
    \ucmain{%
        \begin{enumerate}[nosep,leftmargin=16pt,labelindent=0pt]
            \item Użytkownik wybiera opcję utworzenia nowego czatu.
            \item System tworzy nowy czat (jeśli nie istnieje).
            \item System otwiera widok nowego czatu.
        \end{enumerate}
    }
    \ucalt{%
        \begin{enumerate}[nosep,leftmargin=21pt,labelindent=0pt,label={}]
            \item[1a.] Taki czat już istnieje – system zamiast tworzyć nowy, otwiera istniejącą konwersację.
        \end{enumerate}
    }
}

\usecasecard{tab:pu19-czat-grupowy}{Utworzenie czatu grupowego}{PU19}{%
    \ucpriority{Średni}
    \ucactors{Użytkownik zalogowany}
    \ucdesc{Użytkownik tworzy nowy czat grupowy z kilkoma uczestnikami.}
    \ucpre{Użytkownik jest zalogowany i znajduje się na dowolnym czacie prywatnym.}
    \ucpost{Czat grupowy został utworzony i otwarty.}
    \ucmain{%
        \begin{enumerate}[nosep,leftmargin=16pt,labelindent=0pt]
            \item Użytkownik wybiera opcję utworzenia czatu grupowego.
            \item Użytkownik wybiera uczestników grupy.
            \item Użytkownik zatwierdza utworzenie czatu.
            \item System tworzy czat grupowy i dodaje do niego wskazanych użytkowników.
            \item System otwiera widok nowego czatu grupowego.
        \end{enumerate}
    }
    \ucalt{%
        \begin{enumerate}[nosep,leftmargin=21pt,labelindent=0pt,label={}]
            \item[3a.] System nie może utworzyć czatu – aplikacja informuje o błędzie.
        \end{enumerate}
    }
}

\usecasecard{tab:pu20-lista-czatow}{Przeglądanie listy czatów}{PU20}{%
    \ucpriority{Wysoki}
    \ucactors{Użytkownik zalogowany}
    \ucdesc{Użytkownik przegląda listę swoich czatów prywatnych i grupowych.}
    \ucpre{Użytkownik jest zalogowany i otwiera moduł czatu.}
    \ucpost{Lista czatów użytkownika została wyświetlona.}
    \ucmain{%
        \begin{enumerate}[nosep,leftmargin=16pt,labelindent=0pt]
            \item System pobiera listę czatów użytkownika.
            \item System wyświetla listę czatów z podstawowymi informacjami.
            \item Użytkownik wybiera czat z listy.
            \item System otwiera widok wybranego czatu.
        \end{enumerate}
    }
    \ucalt{Brak istotnych alternatywnych przepływów.}
}

\usecasecard{tab:pu20-wyslij-wiadomosc}{Wysyłanie wiadomości na czacie}{PU20}{%
    \ucpriority{Wysoki}
    \ucactors{Użytkownik zalogowany}
    \ucdesc{Użytkownik wysyła wiadomość tekstową na czacie.}
    \ucpre{Użytkownik jest zalogowany i znajduje się w widoku konkretnego czatu.}
    \ucpost{Nowa wiadomość jest zapisana i widoczna w historii czatu.}
    \ucmain{%
        \begin{enumerate}[nosep,leftmargin=16pt,labelindent=0pt]
            \item Użytkownik wpisuje treść wiadomości.
            \item Użytkownik wysyła wiadomość.
            \item System zapisuje wiadomość i dostarcza ją do uczestników czatu.
            \item System wyświetla wiadomość na liście wiadomości.
        \end{enumerate}
    }
    \ucalt{%
        \begin{enumerate}[nosep,leftmargin=21pt,labelindent=0pt,label={}]
            \item[2a.] Treść wiadomości jest pusta – system blokuje wysłanie i pozostaje w tym samym widoku.
        \end{enumerate}
    }
}

\usecasecard{tab:pu21-wyslij-gifa}{Wysyłanie GIF-a na czacie}{PU25}{%
    \ucpriority{Średni}
    \ucactors{Użytkownik zalogowany, Usługa GIF-ów}
    \ucdesc{Użytkownik wysyła animację GIF w konwersacji czatowej.}
    \ucpre{Użytkownik jest zalogowany i znajduje się w widoku czatu.}
    \ucpost{Wybrany GIF został dodany jako wiadomość w czacie.}
    \ucmain{%
        \begin{enumerate}[nosep,leftmargin=16pt,labelindent=0pt]
            \item Użytkownik wybiera opcję dodania GIF-a.
            \item System otwiera okno wyszukiwarki GIF-ów.
            \item Użytkownik wybiera lub wyszukuje GIF-a.
            \item Użytkownik zatwierdza wysłanie GIF-a.
            \item System dodaje GIF-a jako wiadomość na czacie.
        \end{enumerate}
    }
    \ucalt{%
        \begin{enumerate}[nosep,leftmargin=21pt,labelindent=0pt,label={}]
            \item[2a.] Usługa GIF-ów jest niedostępna – system informuje o braku możliwości wysłania GIF-a.
        \end{enumerate}
    }
}

\usecasecard{tab:pu22-wyslij-plik}{Wysyłanie pliku na czacie}{PU22}{%
    \ucpriority{Średni}
    \ucactors{Użytkownik zalogowany, Usługa do przechowywania plików w chmurze}
    \ucdesc{Użytkownik wysyła plik (np. zdjęcie, film) w czacie.}
    \ucpre{Użytkownik jest zalogowany i znajduje się w widoku czatu.}
    \ucpost{Plik został zapisany w chmurze i powiązany z wiadomością na czacie.}
    \ucmain{%
        \begin{enumerate}[nosep,leftmargin=16pt,labelindent=0pt]
            \item Użytkownik wybiera opcję dodania pliku.
            \item Użytkownik wybiera plik z urządzenia.
            \item System przesyła plik do usługi przechowywania w chmurze.
            \item System tworzy wiadomość z odnośnikiem do pliku.
            \item System wyświetla wiadomość na liście czatu.
        \end{enumerate}
    }
    \ucalt{%
        \begin{enumerate}[nosep,leftmargin=21pt,labelindent=0pt,label={}]
            \item[3a.] Przesyłanie pliku nie powiodło się – system informuje użytkownika i umożliwia ponowną próbę.
        \end{enumerate}
    }
}

\usecasecard{tab:pu23-edycja-czatu}{Edycja ustawień czatu}{PU23}{%
    \ucpriority{Niski}
    \ucactors{Użytkownik zalogowany}
    \ucdesc{Użytkownik modyfikuje ustawienia czatu (np. nazwę, avatar, tryb powiadomień).}
    \ucpre{Użytkownik jest zalogowany i ma uprawnienia do edycji danego czatu.}
    \ucpost{Zaktualizowane ustawienia czatu są zapisane i widoczne dla uczestników.}
    \ucmain{%
        \begin{enumerate}[nosep,leftmargin=16pt,labelindent=0pt]
            \item Użytkownik otwiera panel ustawień czatu.
            \item Użytkownik wprowadza zmiany (np. nazwę, opis, avatar).
            \item Użytkownik zapisuje zmiany.
            \item System waliduje dane i aktualizuje konfigurację czatu.
        \end{enumerate}
    }
    \ucalt{Brak istotnych alternatywnych przepływów poza walidacją pól.}
}

\usecasecard{tab:pu24-dodaj-czlonka}{Dodanie członka do czatu grupowego}{PU24}{%
    \ucpriority{Średni}
    \ucactors{Użytkownik zalogowany}
    \ucdesc{Użytkownik z uprawnieniami administratora dodaje nowego uczestnika do czatu grupowego.}
    \ucpre{Użytkownik jest zalogowany, znajduje się w czacie grupowym i ma prawo zarządzać członkami.}
    \ucpost{Nowy uczestnik został dodany do czatu grupowego.}
    \ucmain{%
        \begin{enumerate}[nosep,leftmargin=16pt,labelindent=0pt]
            \item Użytkownik otwiera listę uczestników czatu grupowego.
            \item Użytkownik wybiera opcję dodania nowego członka.
            \item Użytkownik wskazuje użytkownika do dodania i zatwierdza wybór.
            \item System dodaje wskazanego użytkownika do czatu grupowego.
        \end{enumerate}
    }
    \ucalt{%
        \begin{enumerate}[nosep,leftmargin=21pt,labelindent=0pt,label={}]
            \item[3a.] Operacja nie powiodła się – system informuje o błędzie.
        \end{enumerate}
    }
}

\usecasecard{tab:pu25-historia-czatu}{Przeszukiwanie historii czatu}{PU25}{%
    \ucpriority{Niski}
    \ucactors{Użytkownik premium}
    \ucdesc{Użytkownik wyszukuje konkretne wiadomości w historii czatu.}
    \ucpre{Użytkownik jest zalogowany jako użytkownik premium i znajduje się w widoku czatu.}
    \ucpost{Wiadomości spełniające kryteria wyszukiwania zostały wyświetlone.}
    \ucmain{%
        \begin{enumerate}[nosep,leftmargin=16pt,labelindent=0pt]
            \item Użytkownik otwiera pole wyszukiwania historii w czacie.
            \item Użytkownik wpisuje frazę lub filtr (np. zakres dat, autor).
            \item System filtruje wiadomości zgodnie z kryteriami.
            \item System prezentuje listę dopasowanych fragmentów rozmowy.
        \end{enumerate}
    }
    \ucalt{Brak istotnych alternatywnych przepływów.}
}

\usecasecard{tab:pu26-wyslane-pliki}{Przeglądanie wysłanych plików na czacie}{PU26}{%
    \ucpriority{Niski}
    \ucactors{Użytkownik premium, Usługa do przechowywania plików w chmurze}
    \ucdesc{Użytkownik przegląda listę plików wysłanych w ramach czatów.}
    \ucpre{Użytkownik jest zalogowany jako użytkownik premium.}
    \ucpost{Użytkownik widzi listę wysłanych plików i może przechodzić do powiązanych czatów.}
    \ucmain{%
        \begin{enumerate}[nosep,leftmargin=16pt,labelindent=0pt]
            \item Użytkownik otwiera sekcję „Wysłane pliki”.
            \item System pobiera metadane plików z usługi przechowywania.
            \item System wyświetla listę plików z podstawowymi informacjami (nazwa, typ, data).
            \item Użytkownik wybiera plik, aby otworzyć go lub przejść do powiązanego czatu.
        \end{enumerate}
    }
    \ucalt{Brak istotnych alternatywnych przepływów.}
}

\usecasecard{tab:pu27-dodaj-spota}{Dodanie spota w profilu użytkownika}{PU27}{%
    \ucpriority{Wysoki}
    \ucactors{Użytkownik zalogowany, Usługa do wyświetlania mapy, Usługa do przechowywania plików w chmurze}
    \ucdesc{Użytkownik dodaje nowy spot poprzez swój profil.}
    \ucpre{Użytkownik jest zalogowany i znajduje się w widoku swojego profilu.}
    \ucpost{Nowy spot został zapisany i widoczny na mapie oraz w profilu użytkownika.}
    \ucmain{%
        \begin{enumerate}[nosep,leftmargin=16pt,labelindent=0pt]
            \item Użytkownik wybiera opcję „Dodaj spota”.
            \item Użytkownik uzupełnia podstawowe informacje o spocie (nazwa, opis, typ).
            \item Użytkownik wskazuje lokalizację spota na mapie.
            \item (Opcjonalnie) Użytkownik dodaje zdjęcia/filmy do spota.
            \item Użytkownik zapisuje spota.
            \item System zapisuje dane spota (oraz pliki w chmurze) i aktualizuje mapę oraz profil użytkownika.
        \end{enumerate}
    }
    \ucalt{%
        \begin{enumerate}[nosep,leftmargin=21pt,labelindent=0pt,label={}]
            \item[2a.] Formularz zawiera błędy – system wyświetla komunikat i zaznacza wymagające poprawy pola.
        \end{enumerate}
    }
}

\usecasecard{tab:pu28-profil-wlasny}{Przeglądanie profilu użytkownika}{PU28}{%
    \ucpriority{Wysoki}
    \ucactors{Użytkownik zalogowany}
    \ucdesc{Użytkownik przegląda swój profil (lista spotów, media, podstawowe dane).}
    \ucpre{Użytkownik jest zalogowany.}
    \ucpost{Wyświetlony jest widok profilu użytkownika wraz z jego zawartością.}
    \ucmain{%
        \begin{enumerate}[nosep,leftmargin=16pt,labelindent=0pt]
            \item Użytkownik otwiera swój profil.
            \item System pobiera dane profilu (informacje podstawowe, spoty, media).
            \item System wyświetla dane w odpowiednich sekcjach (spoty, zdjęcia, filmy, komentarze).
        \end{enumerate}
    }
    \ucalt{Brak istotnych alternatywnych przepływów.}
}

\usecasecard{tab:pu29-profil-innego}{Przeglądanie profilu innego użytkownika}{PU29}{%
    \ucpriority{Średni}
    \ucactors{Użytkownik zalogowany}
    \ucdesc{Użytkownik ogląda profil innego użytkownika (np. z mapy, forum lub społeczności).}
    \ucpre{Użytkownik jest zalogowany i ma dostęp do odnośnika do profilu innego użytkownika.}
    \ucpost{Profil innego użytkownika został wyświetlony.}
    \ucmain{%
        \begin{enumerate}[nosep,leftmargin=16pt,labelindent=0pt]
            \item Użytkownik wybiera odnośnik do profilu innego użytkownika.
            \item System pobiera dane profilu docelowego użytkownika.
            \item System wyświetla profil (media, podstawowe informacje).
        \end{enumerate}
    }
    \ucalt{%
        \begin{enumerate}[nosep,leftmargin=21pt,labelindent=0pt,label={}]
            \item[2a.] Wystąpił błąd podczas pobierania danych użytkownika  – system wyświetla informację o błędzie.
        \end{enumerate}
    }
}


\usecasecard{tab:pu30-dodaj-znajomego}{Dodanie użytkownika do znajomych}{PU30}{%
    \ucpriority{Średni}
    \ucactors{Użytkownik zalogowany}
    \ucdesc{Użytkownik wysyła lub akceptuje zaproszenie do znajomych.}
    \ucpre{Użytkownik jest zalogowany i przegląda profil innego użytkownika.}
    \ucpost{Relacja „znajomy” została utworzona lub zaproszenie czeka na akceptację.}
    \ucmain{%
        \begin{enumerate}[nosep,leftmargin=16pt,labelindent=0pt]
            \item Użytkownik klika przycisk „Dodaj do znajomych”.
            \item System sprawdza, czy relacja już istnieje.
            \item System tworzy nowe zaproszenie.
            \item System informuje o statusie o wysłaniu zaproszenia.
        \end{enumerate}
    }
    \ucalt{Brak istotnych alternatywnych przepływów.}
    }
}

\usecasecard{tab:pu31-spolecznosci}{Przeglądanie społeczności}{PU31}{%
    \ucpriority{Średni}
    \ucactors{Użytkownik zalogowany}
    \ucdesc{Użytkownik przegląda społeczności, grupy lub listy znajomych powiązane z aplikacją.}
    \ucpre{Użytkownik jest zalogowany.}
    \ucpost{Lista społeczności lub znajomych została wyświetlona.}
    \ucmain{%
        \begin{enumerate}[nosep,leftmargin=16pt,labelindent=0pt]
            \item Użytkownik przechodzi do sekcji społeczności.
            \item System pobiera listę społeczności i znajomych użytkownika.
            \item System wyświetla listę z możliwością przechodzenia do profili i czatów.
        \end{enumerate}
    }
    \ucalt{Brak istotnych alternatywnych przepływów.}
}

\usecasecard{tab:pu32-zarzadzaj-komentarzami-spot}{Zarządzanie komentarzami do spotów}{PU32}{%
    \ucpriority{Niski}
    \ucactors{Użytkownik zalogowany (właściciel spota lub moderator)}
    \ucdesc{Użytkownik zarządza komentarzami dodanymi do spota (edycja, usuwanie, ukrywanie).}
    \ucpre{Użytkownik jest zalogowany i wyświetla szczegóły spota.}
    \ucpost{Wybrane komentarze zostały zaktualizowane lub ukryte zgodnie z działaniem użytkownika.}
    \ucmain{%
        \begin{enumerate}[nosep,leftmargin=16pt,labelindent=0pt]
            \item Użytkownik otwiera panel zarządzania komentarzami dla danego spota.
            \item System pobiera listę komentarzy wraz z możliwymi akcjami.
            \item Użytkownik wybiera komentarz i akcję (np. edytuj, usuń, ukryj).
            \item System wykonuje wybraną akcję na komentarzu.
            \item System odświeża listę komentarzy.
        \end{enumerate}
    }
    \ucalt{%
        \begin{enumerate}[nosep,leftmargin=21pt,labelindent=0pt,label={}]
            \item[3a.] Użytkownik nie ma uprawnień do zarządzania komentarzem – system wyświetla komunikat o braku uprawnień.
        \end{enumerate}
    }
}

\usecasecard{tab:pu33-zarzadzaj-komentarzami-forum}{Zarządzanie komentarzami na forum}{PU33}{%
    \ucpriority{Niski}
    \ucactors{Użytkownik zalogowany (autor posta lub moderator)}
    \ucdesc{Użytkownik zarządza komentarzami pod postami forum (edycja, usuwanie, przypinanie).}
    \ucpre{Użytkownik jest zalogowany i ma dostęp do danego wątku forum.}
    \ucpost{Komentarze zostały zaktualizowane zgodnie z działaniami użytkownika.}
    \ucmain{%
        \begin{enumerate}[nosep,leftmargin=16pt,labelindent=0pt]
            \item Użytkownik otwiera widok komentarzy pod postem.
            \item Użytkownik wybiera komentarz i odpowiednią akcję.
            \item System weryfikuje uprawnienia użytkownika.
            \item System wykonuje wybraną akcję i aktualizuje widok.
        \end{enumerate}
    }
    \ucalt{%
        \begin{enumerate}[nosep,leftmargin=21pt,labelindent=0pt,label={}]
            \item[3a.] Użytkownik nie ma wymaganych uprawnień – system blokuje operację i informuje o tym.
        \end{enumerate}
    }
}

\usecasecard{tab:pu34-zarzadzaj-postami}{Zarządzanie postami na forum}{PU34}{%
    \ucpriority{Niski}
    \ucactors{Użytkownik zalogowany (autor posta lub moderator)}
    \ucdesc{Użytkownik edytuje, archiwizuje lub usuwa własne posty na forum.}
    \ucpre{Użytkownik jest zalogowany i otwiera listę swoich postów lub moderowany dział forum.}
    \ucpost{Status wybranych postów został zaktualizowany.}
    \ucmain{%
        \begin{enumerate}[nosep,leftmargin=16pt,labelindent=0pt]
            \item Użytkownik przechodzi do sekcji zarządzania postami.
            \item System pobiera listę postów użytkownika (lub działu).
            \item Użytkownik wybiera post i żądaną akcję (edycja, archiwizacja, usunięcie).
            \item System zapisuje zmiany i aktualizuje listę postów.
        \end{enumerate}
    }
    \ucalt{%
        \begin{enumerate}[nosep,leftmargin=21pt,labelindent=0pt,label={}]
            \item[3a.] Użytkownik próbuje usunąć post z zablokowanego wątku – system odmawia wykonania operacji.
        \end{enumerate}
    }
}

\usecasecard{tab:pu35-zglos-komentarz}{Zgłoszenie komentarza naruszającego regulamin}{PU35}{%
    \ucpriority{Średni}
    \ucactors{Użytkownik zalogowany}
    \ucdesc{Użytkownik zgłasza komentarz na forum.}
    \ucpre{Użytkownik widzi komentarz w aplikacji.}
    \ucpost{Zgłoszenie komentarza zostało zapisane i trafiło do kolejki moderacyjnej.}
    \ucmain{%
        \begin{enumerate}[nosep,leftmargin=16pt,labelindent=0pt]
            \item Użytkownik wybiera opcję „Zgłoś komentarz”.
            \item Użytkownik określa powód zgłoszenia.
            \item System zapisuje zgłoszenie i wiąże je z komentarzem i zgłaszającym.
        \end{enumerate}
    }
    \ucalt{Brak istotnych alternatywnych przepływów.}
}

\usecasecard{tab:pu36-zglos-posta}{Zgłoszenie posta na forum}{PU36}{%
    \ucpriority{Średni}
    \ucactors{Użytkownik zalogowany}
    \ucdesc{Użytkownik zgłasza post forum jako naruszający regulamin lub tematykę.}
    \ucpre{Wyświetlony jest widok posta na forum.}
    \ucpost{Zgłoszenie posta zostało zapisane i przekazane moderatorom.}
    \ucmain{%
        \begin{enumerate}[nosep,leftmargin=16pt,labelindent=0pt]
            \item Użytkownik wybiera opcję „Zgłoś post”.
            \item Użytkownik wybiera kategorię naruszenia i potwierdza zgłoszenie.
            \item System zapisuje zgłoszenie i oznacza post jako zgłoszony.
        \end{enumerate}
    }
    \ucalt{Brak istotnych alternatywnych przepływów.}
}

\usecasecard{tab:pu37-zmien-typ-mapy}{Zmiana typu mapy}{PU37}{%
    \ucpriority{Niski}
    \ucactors{Użytkownik premium, Usługa do wyświetlania mapy}
    \ucdesc{Użytkownik zmienia typ mapy (np. standardowa, satelitarna, hybrydowa).}
    \ucpre{Użytkownik premium jest na ekranie mapy.}
    \ucpost{Mapa jest wyświetlana w wybranym typie.}
    \ucmain{%
        \begin{enumerate}[nosep,leftmargin=16pt,labelindent=0pt]
            \item Użytkownik otwiera ustawienia widoku mapy.
            \item Użytkownik wybiera typ mapy z dostępnej listy.
            \item System przełącza widok mapy na wybrany typ.
        \end{enumerate}
    }
    \ucalt{%
        \begin{enumerate}[nosep,leftmargin=21pt,labelindent=0pt,label={}]
            \item[3a.] Wybrany typ mapy nie jest dostępny (błąd usługi mapowej) – system przywraca poprzedni typ i informuje o błędzie.
        \end{enumerate}
    }
}

\usecasecard{tab:pu38-strefy-pansa}{Przeglądanie stref PANSA}{PU38}{%
    \ucpriority{Niski}
    \ucactors{Użytkownik premium, Usługa do wyświetlania mapy}
    \ucdesc{Użytkownik wyświetla na mapie strefy przestrzeni powietrznej \gls{PANSA}.}
    \ucpre{Użytkownik premium ma otwarty moduł mapy.}
    \ucpost{Strefy PANSA zostały zwizualizowane na mapie.}
    \ucmain{%
        \begin{enumerate}[nosep,leftmargin=16pt,labelindent=0pt]
            \item Użytkownik włącza warstwę „Strefy PANSA”.
            \item System pobiera dane o strefach.
            \item System nakłada kontury stref na mapę.
        \end{enumerate}
    }
    \ucalt{%
        \begin{enumerate}[nosep,leftmargin=21pt,labelindent=0pt,label={}]
            \item[2a.] Dane o strefach są chwilowo niedostępne – system komunikuje problem i nie włącza warstwy.
        \end{enumerate}
    }
}
