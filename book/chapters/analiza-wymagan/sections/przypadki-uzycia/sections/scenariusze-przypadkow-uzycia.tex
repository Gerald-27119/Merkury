%%! Author = Adam
%%! Date = 22/11/2025

\subsection{Scenariusze przypadków użycia}
\label{subsec:scenariusz-przypadkow-uzycia}

Niniejszy rozdział zawiera scenariusze przypadków użycia.
Zostały one wykonane dla wybranych przypadków użycia.

% --- Licznik dla scenariuszy przypadków użycia ---
\newcounter{usecase}[chapter]
\renewcommand{\theusecase}{\thechapter.\arabic{usecase}}

\newcommand{\ucpriority}[1]{\textbf{Priorytet:} & #1 \\ \hline}
\newcommand{\ucactors}[1]{\textbf{Aktorzy:} & #1 \\ \hline}
\newcommand{\ucdesc}[1]{\textbf{Opis:} & #1 \\ \hline}
\newcommand{\ucpre}[1]{\textbf{Warunki\newline wstępne:} & #1 \\ \hline}
\newcommand{\ucpost}[1]{\textbf{Warunki\newline końcowe:} & #1 \\ \hline}
\newcommand{\ucmain}[1]{\textbf{Główny\newline przepływ\newline zdarzeń:} & #1 \\ \hline}
\newcommand{\ucalt}[1]{\textbf{Alternatywne\newline przepływy\newline zdarzeń:} & #1 \\ \hline}

% --- Karta scenariusza PU ---
\newcommand{\usecasecard}[4]{%
    \refstepcounter{usecase}%
    \begin{center}
    \renewcommand{\arraystretch}{1.15}%
    \begin{tabularx}{\textwidth}{|>{\columncolor{lightgray}}p{0.20\textwidth}|X|}
    \rowcolor{lightgray}
    \multicolumn{2}{|c|}{\textbf{KARTA SCENARIUSZA PRZYPADKU UŻYCIA}} \\ \hline
    \textbf{Identyfikator:} & #3 \\ \hline
    \textbf{Nazwa:}         & #2 \\ \hline
    #4
    \end{tabularx}
    \vspace{3pt}
    \textbf{Tabela \theusecase:} Scenariusz przypadku użycia: #2\label{#1}
    \end{center}%
    \addcontentsline{lot}{table}{Tabela \theusecase: Scenariusz przypadku użycia: #2}%
}

% Scenariusze PU:

\usecasecard{tab:pu1-rejestracja}{Rejestracja użytkownika}{PU1}{%
    \ucpriority{Wysoki}
    \ucactors{Użytkownik niezalogowany}
    \ucdesc{Użytkownik zakłada konto poprzez formularz rejestracji.}
    \ucpre{Użytkownik znajduje się na stronie z formularzem rejestracji.}
    \ucpost{Użytkownik posiada konto w systemie.}
    \ucmain{%
        \begin{enumerate}[nosep,leftmargin=16pt,labelindent=0pt]
            \item Użytkownik wypełnia formularz rejestracyjny.
            \item Użytkownik naciska przycisk rejestracji.
            \item System tworzy konto użytkownika.
            \item System loguje użytkownika i przenosi go na stronę główną aplikacji.
        \end{enumerate}
    }
    \ucalt{%
        \begin{enumerate}[nosep,leftmargin=21pt,labelindent=0pt,label={}]
            \item[1a.] Formularz zawiera niepoprawne dane – system wyświetla
            komunikat o błędzie oraz podświetla pola wymagające poprawy.
            \item[2a.] Nazwa użytkownika jest już zajęta – system wyświetla
            komunikat o błędzie.
        \end{enumerate}
    }
}

\usecasecard{tab:pu2-logowanie}{Logowanie użytkownika}{PU2}{%
    \ucpriority{Wysoki}
    \ucactors{Użytkownik niezalogowany}
    \ucdesc{Użytkownik loguje się do systemu, podając login i hasło.}
    \ucpre{Użytkownik znajduje się na stronie logowania.}
    \ucpost{Użytkownik jest zalogowany i przeniesiony na stronę główną aplikacji.}
    \ucmain{%
        \begin{enumerate}[nosep,leftmargin=16pt,labelindent=0pt]
            \item Użytkownik wpisuje nazwę użytkownika lub adres e-mail.
            \item Użytkownik wpisuje hasło.
            \item Użytkownik naciska przycisk logowania.
            \item System weryfikuje dane logowania.
            \item System loguje użytkownika i przenosi go na stronę główną.
        \end{enumerate}
    }
    \ucalt{%
        \begin{enumerate}[nosep,leftmargin=21pt,labelindent=0pt,label={}]
            \item[4a.] Podane dane są niepoprawne – system wyświetla komunikat o błędzie i pozwala ponowić próbę.
            \item[4b.] Konto użytkownika jest zablokowane – system informuje o blokadzie i sugeruje kontakt z administratorem.
        \end{enumerate}
    }
}

\usecasecard{tab:pu3-reset-hasla}{Resetowanie hasła}{PU3}{%
    \ucpriority{Wysoki}
    \ucactors{Użytkownik niezalogowany, Usługa SMTP}
    \ucdesc{Użytkownik inicjuje reset hasła, aby odzyskać dostęp do konta.}
    \ucpre{Użytkownik znajduje się na ekranie resetu hasła.}
    \ucpost{Użytkownik otrzymuje wiadomość e-mail z linkiem do ustawienia nowego hasła.}
    \ucmain{%
        \begin{enumerate}[nosep,leftmargin=16pt,labelindent=0pt]
            \item Użytkownik wpisuje adres e-mail powiązany z kontem.
            \item Użytkownik zatwierdza żądanie resetu hasła.
            \item System sprawdza, czy istnieje konto powiązane z podanym adresem.
            \item System generuje token resetu hasła.
            \item System wysyła e-mail z linkiem do zmiany hasła.
        \end{enumerate}
    }
    \ucalt{%
        \begin{enumerate}[nosep,leftmargin=21pt,labelindent=0pt,label={}]
            \item[3a.] Nie istnieje konto dla podanego adresu – system wyświetla komunikat o błędzie.
            \item[5a.] Występuje błąd połączenia z usługą SMTP – system informuje użytkownika o problemie technicznym.
        \end{enumerate}
    }
}

\usecasecard{tab:pu4-zmiana-hasla}{Zmiana hasła w ustawieniach konta}{PU4}{%
    \ucpriority{Wysoki}
    \ucactors{Użytkownik zalogowany}
    \ucdesc{Użytkownik zmienia hasło do konta z poziomu ustawień profilu.}
    \ucpre{Użytkownik jest zalogowany i znajduje się na ekranie ustawień konta.}
    \ucpost{Hasło do konta użytkownika zostało zaktualizowane.}
    \ucmain{%
        \begin{enumerate}[nosep,leftmargin=16pt,labelindent=0pt]
            \item Użytkownik przechodzi do sekcji zmiany hasła.
            \item Użytkownik wpisuje aktualne hasło.
            \item Użytkownik wpisuje nowe hasło i powtarza je.
            \item Użytkownik zatwierdza formularz zmiany hasła.
            \item System weryfikuje aktualne hasło i poprawność nowego.
            \item System zapisuje nowe hasło i informuje o powodzeniu operacji.
        \end{enumerate}
    }
    \ucalt{%
        \begin{enumerate}[nosep,leftmargin=21pt,labelindent=0pt,label={}]
            \item[5a.] Aktualne hasło jest nieprawidłowe – system wyświetla komunikat i nie zapisuje zmian.
            \item[5b.] Nowe hasło nie spełnia wymagań bezpieczeństwa – system informuje o błędzie i podświetla pola do poprawy.
        \end{enumerate}
    }
}

\usecasecard{tab:pu5-zmiana-motywu}{Zmiana motywu interfejsu}{PU5}{%
    \ucpriority{Średni}
    \ucactors{Użytkownik niezalogowany}
    \ucdesc{Użytkownik zmienia motyw kolorystyczny interfejsu (np. jasny/ciemny).}
    \ucpre{Użytkownik znajduje się w aplikacji webowej.}
    \ucpost{Wybrany motyw jest zastosowany w interfejsie i zapamiętany w ustawieniach.}
    \ucmain{%
        \begin{enumerate}[nosep,leftmargin=16pt,labelindent=0pt]
            \item Użytkownik otwiera menu ustawień wyglądu.
            \item Użytkownik wybiera dostępny motyw interfejsu.
            \item System stosuje wybrany motyw.
            \item System zapisuje preferencję (np. w local storage lub profilu użytkownika).
        \end{enumerate}
    }
    \ucalt{Brak istotnych alternatywnych przepływów.}
}

\usecasecard{tab:pu6-wylogowanie}{Wylogowanie użytkownika}{PU6}{%
    \ucpriority{Wysoki}
    \ucactors{Użytkownik zalogowany}
    \ucdesc{Użytkownik wylogowuje się z aplikacji.}
    \ucpre{Użytkownik jest zalogowany.}
    \ucpost{Sesja użytkownika została zakończona, użytkownik widzi stronę startową dla niezalogowanych.}
    \ucmain{%
        \begin{enumerate}[nosep,leftmargin=16pt,labelindent=0pt]
            \item Użytkownik wybiera opcję wylogowania z menu.
            \item System unieważnia sesję użytkownika.
            \item System przenosi użytkownika na stronę startową aplikacji.
        \end{enumerate}
    }
    \ucalt{Brak istotnych alternatywnych przepływów.}
}

\usecasecard{tab:pu7-powiadomienia}{Przeglądanie powiadomień}{PU7}{%
    \ucpriority{Wysoki}
    \ucactors{Użytkownik zalogowany}
    \ucdesc{Użytkownik przegląda listę powiadomień (np. o nowych komentarzach, wiadomościach, płatnościach).}
    \ucpre{Użytkownik jest zalogowany.}
    \ucpost{Powiadomienia zostały wyświetlone, a wybrane oznaczone jako przeczytane.}
    \ucmain{%
        \begin{enumerate}[nosep,leftmargin=16pt,labelindent=0pt]
            \item Użytkownik otwiera panel powiadomień.
            \item System pobiera listę powiadomień użytkownika.
            \item System wyświetla powiadomienia w odwróconym porządku chronologicznym.
            \item Użytkownik otwiera wybrane powiadomienie.
            \item System oznacza powiadomienie jako przeczytane i ewentualnie przenosi użytkownika do powiązanego widoku (np. postu, spota, czatu).
        \end{enumerate}
    }
    \ucalt{%
        \begin{enumerate}[nosep,leftmargin=21pt,labelindent=0pt,label={}]
            \item[2a.] System nie może pobrać powiadomień (błąd serwera) – użytkownik otrzymuje komunikat o błędzie i może spróbować ponownie.
        \end{enumerate}
    }
}

\usecasecard{tab:pu8-edycja-konta}{Edycja danych konta}{PU8}{%
    \ucpriority{Wysoki}
    \ucactors{Użytkownik zalogowany}
    \ucdesc{Użytkownik modyfikuje podstawowe dane konta (np. imię, avatar, opis).}
    \ucpre{Użytkownik jest zalogowany i znajduje się w sekcji ustawień konta.}
    \ucpost{Zaktualizowane dane konta są zapisane i widoczne w profilu.}
    \ucmain{%
        \begin{enumerate}[nosep,leftmargin=16pt,labelindent=0pt]
            \item Użytkownik otwiera formularz edycji danych konta.
            \item Użytkownik wprowadza zmiany w dostępnych polach.
            \item Użytkownik zatwierdza formularz.
            \item System waliduje dane i zapisuje zmiany.
            \item System informuje użytkownika o poprawnym zapisaniu danych.
        \end{enumerate}
    }
    \ucalt{%
        \begin{enumerate}[nosep,leftmargin=21pt,labelindent=0pt,label={}]
            \item[4a.] Formularz zawiera błędne dane – system wyświetla komunikat i podświetla pola wymagające poprawy.
        \end{enumerate}
    }
}

\usecasecard{tab:pu9-subskrypcja}{Wykupienie subskrypcji premium}{PU9}{%
    \ucpriority{Wysoki}
    \ucactors{Użytkownik zalogowany, Bramka płatnicza, System finansowo-księgowy}
    \ucdesc{Użytkownik opłaca subskrypcję premium w celu uzyskania dodatkowych funkcji.}
    \ucpre{Użytkownik jest zalogowany i znajduje się w module subskrypcji.}
    \ucpost{Subskrypcja premium jest aktywna, a użytkownik ma dostęp do funkcji premium.}
    \ucmain{%
        \begin{enumerate}[nosep,leftmargin=16pt,labelindent=0pt]
            \item Użytkownik wybiera plan subskrypcji.
            \item Użytkownik przechodzi do bramki płatniczej.
            \item Użytkownik podaje dane płatnicze i zatwierdza transakcję.
            \item Bramka płatnicza przetwarza płatność i zwraca wynik do systemu.
            \item System zapisuje informację o opłaconej subskrypcji i aktualizuje uprawnienia.
            \item System generuje wpis w systemie finansowo-księgowym.
        \end{enumerate}
    }
    \ucalt{%
        \begin{enumerate}[nosep,leftmargin=21pt,labelindent=0pt,label={}]
            \item[4a.] Płatność nie powiodła się – system informuje użytkownika i umożliwia ponowną próbę.
            \item[5a.] W czasie aktualizacji subskrypcji wystąpił błąd – system cofa zmiany i wyświetla komunikat o problemie.
        \end{enumerate}
    }
}

\usecasecard{tab:pu10-mapa}{Przeglądanie mapy spotów}{PU10}{%
    \ucpriority{Wysoki}
    \ucactors{Użytkownik niezalogowany, Użytkownik zalogowany, Usługa do wyświetlania mapy}
    \ucdesc{Użytkownik przegląda mapę z zaznaczonymi spotami.}
    \ucpre{Użytkownik znajduje się w module mapy.}
    \ucpost{Mapa ze spotami została wyświetlona, a użytkownik może przybliżać, oddalać i przesuwać widok.}
    \ucmain{%
        \begin{enumerate}[nosep,leftmargin=16pt,labelindent=0pt]
            \item System inicjuje widok mapy z domyślnym obszarem.
            \item System pobiera listę spotów w aktualnym zakresie mapy.
            \item System rysuje znaczniki spotów na mapie.
            \item Użytkownik przesuwa lub skaluje mapę.
            \item System dociąga i aktualizuje listę spotów dla nowego zakresu.
        \end{enumerate}
    }
    \ucalt{%
        \begin{enumerate}[nosep,leftmargin=21pt,labelindent=0pt,label={}]
            \item[2a.] Usługa mapy jest niedostępna – system wyświetla komunikat o błędzie i prosty widok zastępczy.
        \end{enumerate}
    }
}

\usecasecard{tab:pu11-szukaj-na-mapie}{Wyszukiwanie spota na mapie}{PU11}{%
    \ucpriority{Wysoki}
    \ucactors{Użytkownik niezalogowany, Użytkownik zalogowany}
    \ucdesc{Użytkownik wyszukuje spota korzystając z pola wyszukiwania na mapie (np. po nazwie lub lokalizacji).}
    \ucpre{Użytkownik widzi mapę spotów.}
    \ucpost{Mapa zostaje ustawiona na wybranego spota lub listę dopasowań.}
    \ucmain{%
        \begin{enumerate}[nosep,leftmargin=16pt,labelindent=0pt]
            \item Użytkownik wpisuje frazę w polu wyszukiwania na mapie.
            \item System podpowiada listę pasujących spotów.
            \item Użytkownik wybiera spota z listy.
            \item System przybliża mapę do wybranego spota i podświetla jego znacznik.
        \end{enumerate}
    }
    \ucalt{%
        \begin{enumerate}[nosep,leftmargin=21pt,labelindent=0pt,label={}]
            \item[2a.] Brak wyników dla podanej frazy – system informuje użytkownika o braku dopasowań.
        \end{enumerate}
    }
}

\usecasecard{tab:pu12-globalna-wyszukiwarka}{Wyszukiwanie spota w globalnej wyszukiwarce}{PU12}{%
    \ucpriority{Wysoki}
    \ucactors{Użytkownik}
    \ucdesc{Użytkownik wyszukuje spoty za pomocą globalnej wyszukiwarki w aplikacji.}
    \ucpre{Użytkownik ma dostęp do globalnego pola wyszukiwania.}
    \ucpost{Użytkownik otrzymuje listę znalezionych spotów.}
    \ucmain{%
        \begin{enumerate}[nosep,leftmargin=16pt,labelindent=0pt]
            \item Użytkownik wpisuje frazę w globalnej wyszukiwarce.
            \item System wyszukuje spoty spełniające kryteria.
            \item System wyświetla listę wyników.
        \end{enumerate}
    }
    \ucalt{%
        \begin{enumerate}[nosep,leftmargin=21pt,labelindent=0pt,label={}]
            \item[3a.] Brak wyników – system wyświetla komunikat i proponuje zmianę kryteriów wyszukiwania.
        \end{enumerate}
    }
}

\usecasecard{tab:pu13-przejscie-z-wynikow}{Przejście do spota na mapie z wyszukiwarki}{PU13}{%
    \ucpriority{Wysoki}
    \ucactors{Użytkownik}
    \ucdesc{Użytkownik przechodzi z wyników wyszukiwarki do widoku mapy ustawionego na konkretny spot.}
    \ucpre{Wyświetlona jest lista wyników wyszukiwania spotów.}
    \ucpost{Mapa jest przybliżona do wybranego spota, a jego szczegóły są dostępne.}
    \ucmain{%
        \begin{enumerate}[nosep,leftmargin=16pt,labelindent=0pt]
            \item Użytkownik wybiera spota z listy wyników.
            \item System przełącza widok na moduł mapy.
            \item System ustawia mapę na lokalizację spota i podświetla jego znacznik.
        \end{enumerate}
    }
    \ucalt{Brak istotnych alternatywnych przepływów.}
}

\usecasecard{tab:pu14-szczegoly-spota}{Wyświetlanie szczegółów spota}{PU14}{%
    \ucpriority{Wysoki}
    \ucactors{Użytkownik niezalogowany, Użytkownik zalogowany}
    \ucdesc{Użytkownik otwiera widok szczegółów wybranego spota.}
    \ucpre{Użytkownik ma przed sobą mapę lub listę spotów.}
    \ucpost{Wyświetlony jest ekran ze szczegółami spota (opis, media, komentarze).}
    \ucmain{%
        \begin{enumerate}[nosep,leftmargin=16pt,labelindent=0pt]
            \item Użytkownik wybiera spota z mapy lub listy.
            \item System pobiera szczegółowe dane spota.
            \item System wyświetla kartę szczegółów spota.
        \end{enumerate}
    }
    \ucalt{%
        \begin{enumerate}[nosep,leftmargin=21pt,labelindent=0pt,label={}]
            \item[2a.] Spot został usunięty – system informuje użytkownika o niedostępności spota.
        \end{enumerate}
    }
}

\usecasecard{tab:pu15-komentarze-spota}{Przeglądanie komentarzy do spota}{PU15}{%
    \ucpriority{Wysoki}
    \ucactors{Użytkownik niezalogowany, Użytkownik zalogowany}
    \ucdesc{Użytkownik czyta komentarze pod wybranym spotem.}
    \ucpre{Wyświetlany jest widok szczegółów spota.}
    \ucpost{Lista komentarzy do spota została wyświetlona.}
    \ucmain{%
        \begin{enumerate}[nosep,leftmargin=16pt,labelindent=0pt]
            \item System pobiera komentarze powiązane ze spotem.
            \item System wyświetla komentarze w kolejności chronologicznej lub według popularności.
            \item Użytkownik przewija listę komentarzy.
        \end{enumerate}
    }
    \ucalt{%
        \begin{enumerate}[nosep,leftmargin=21pt,labelindent=0pt,label={}]
            \item[1a.] Spot nie ma jeszcze komentarzy – system wyświetla odpowiednią informację.
        \end{enumerate}
    }
}

\usecasecard{tab:pu16-pogoda}{Przeglądanie pogody na spocie}{PU16}{%
    \ucpriority{Średni}
    \ucactors{Użytkownik zalogowany, Usługa danych pogodowych}
    \ucdesc{Użytkownik sprawdza prognozę pogody dla lokalizacji spota.}
    \ucpre{Wyświetlany jest widok szczegółów spota z dostępną lokalizacją.}
    \ucpost{Prognoza pogody dla spota została wyświetlona.}
    \ucmain{%
        \begin{enumerate}[nosep,leftmargin=16pt,labelindent=0pt]
            \item Użytkownik przechodzi do zakładki z pogodą dla spota.
            \item System wysyła zapytanie do usługi pogodowej z lokalizacją spota.
            \item System odbiera prognozę i prezentuje ją (np. temperatura, wiatr, opady).
        \end{enumerate}
    }
    \ucalt{%
        \begin{enumerate}[nosep,leftmargin=21pt,labelindent=0pt,label={}]
            \item[2a.] Usługa pogodowa jest niedostępna – system wyświetla komunikat o braku danych pogodowych.
        \end{enumerate}
    }
}

\usecasecard{tab:pu17-posty-forum}{Przeglądanie postów na forum}{PU17}{%
    \ucpriority{Wysoki}
    \ucactors{Użytkownik niezalogowany, Użytkownik zalogowany}
    \ucdesc{Użytkownik przegląda listę postów na forum.}
    \ucpre{Użytkownik znajduje się w module forum.}
    \ucpost{Lista postów forum jest wyświetlona, a użytkownik może przechodzić do szczegółów.}
    \ucmain{%
        \begin{enumerate}[nosep,leftmargin=16pt,labelindent=0pt]
            \item System pobiera listę postów z wybranego działu forum.
            \item System wyświetla posty z podstawowymi informacjami (autor, data, liczba komentarzy).
            \item Użytkownik wybiera post, który chce przeczytać.
            \item System otwiera szczegółowy widok posta.
        \end{enumerate}
    }
    \ucalt{%
        \begin{enumerate}[nosep,leftmargin=21pt,labelindent=0pt,label={}]
            \item[1a.] Brak postów w danym dziale – system wyświetla odpowiednią informację.
        \end{enumerate}
    }
}

\usecasecard{tab:pu18-dodaj-post}{Dodanie posta na forum}{PU18}{%
    \ucpriority{Wysoki}
    \ucactors{Użytkownik zalogowany, Usługa do przechowywania plików w chmurze}
    \ucdesc{Użytkownik publikuje nowy post na forum.}
    \ucpre{Użytkownik jest zalogowany i znajduje się w module forum.}
    \ucpost{Nowy post jest widoczny na forum.}
    \ucmain{%
        \begin{enumerate}[nosep,leftmargin=16pt,labelindent=0pt]
            \item Użytkownik wybiera opcję dodania nowego posta.
            \item Użytkownik wpisuje tytuł i treść posta.
            \item (Opcjonalnie) Użytkownik dodaje załączniki (zdjęcia/filmy) do posta.
            \item Użytkownik publikuje posta.
            \item System zapisuje posta (oraz załączniki w chmurze) i wyświetla go na liście postów.
        \end{enumerate}
    }
    \ucalt{%
        \begin{enumerate}[nosep,leftmargin=21pt,labelindent=0pt,label={}]
            \item[3a.] Załącznik nie może zostać zapisany – system informuje o błędzie i pozwala opublikować posta bez pliku.
            \item[4a.] Formularz zawiera błędne lub niekompletne dane – system wyświetla komunikat i prosi o poprawę.
        \end{enumerate}
    }
}

\usecasecard{tab:pu19-dodaj-komentarz}{Dodanie komentarza na forum}{PU19}{%
    \ucpriority{Wysoki}
    \ucactors{Użytkownik zalogowany}
    \ucdesc{Użytkownik dodaje komentarz pod postem na forum.}
    \ucpre{Użytkownik jest zalogowany i widzi szczegóły posta.}
    \ucpost{Nowy komentarz został zapisany i widoczny pod postem.}
    \ucmain{%
        \begin{enumerate}[nosep,leftmargin=16pt,labelindent=0pt]
            \item Użytkownik wpisuje treść komentarza w formularzu pod postem.
            \item Użytkownik publikuje komentarz.
            \item System zapisuje komentarz i odświeża listę komentarzy.
        \end{enumerate}
    }
    \ucalt{%
        \begin{enumerate}[nosep,leftmargin=21pt,labelindent=0pt,label={}]
            \item[2a.] Treść komentarza jest pusta lub przekracza limit znaków – system wyświetla komunikat o błędzie.
        \end{enumerate}
    }
}

\usecasecard{tab:pu20-historia-postow}{Przeglądanie historii interakcji z postami}{PU20}{%
    \ucpriority{Średni}
    \ucactors{Użytkownik zalogowany}
    \ucdesc{Użytkownik przegląda historię swoich aktywności na forum (dodane posty, komentarze, reakcje).}
    \ucpre{Użytkownik jest zalogowany.}
    \ucpost{Lista interakcji użytkownika z postami jest wyświetlona.}
    \ucmain{%
        \begin{enumerate}[nosep,leftmargin=16pt,labelindent=0pt]
            \item Użytkownik przechodzi do sekcji historii aktywności.
            \item System pobiera historię interakcji użytkownika.
            \item System wyświetla listę interakcji z możliwością filtrowania (np. posty, komentarze).
        \end{enumerate}
    }
    \ucalt{Brak istotnych alternatywnych przepływów.}
}

\usecasecard{tab:pu21-czat-prywatny}{Utworzenie prywatnego czatu}{PU21}{%
    \ucpriority{Wysoki}
    \ucactors{Użytkownik zalogowany}
    \ucdesc{Użytkownik tworzy prywatną konwersację z innym użytkownikiem.}
    \ucpre{Użytkownik jest zalogowany i znajduje się w module czatu.}
    \ucpost{Nowy czat prywatny został utworzony i widoczny na liście czatów.}
    \ucmain{%
        \begin{enumerate}[nosep,leftmargin=16pt,labelindent=0pt]
            \item Użytkownik wybiera opcję utworzenia nowego czatu.
            \item Użytkownik wskazuje docelowego użytkownika (np. z listy znajomych lub wyszukiwarki).
            \item System sprawdza, czy istnieje już prywatny czat między tymi użytkownikami.
            \item System tworzy nowy czat (jeśli nie istnieje) i dodaje do niego uczestników.
            \item System otwiera widok nowego czatu.
        \end{enumerate}
    }
    \ucalt{%
        \begin{enumerate}[nosep,leftmargin=21pt,labelindent=0pt,label={}]
            \item[3a.] Taki czat już istnieje – system zamiast tworzyć nowy, otwiera istniejącą konwersację.
        \end{enumerate}
    }
}

\usecasecard{tab:pu22-czat-grupowy}{Utworzenie czatu grupowego}{PU22}{%
    \ucpriority{Wysoki}
    \ucactors{Użytkownik zalogowany}
    \ucdesc{Użytkownik tworzy nowy czat grupowy z kilkoma uczestnikami.}
    \ucpre{Użytkownik jest zalogowany i znajduje się w module czatu.}
    \ucpost{Czat grupowy został utworzony i widoczny na liście czatów.}
    \ucmain{%
        \begin{enumerate}[nosep,leftmargin=16pt,labelindent=0pt]
            \item Użytkownik wybiera opcję utworzenia czatu grupowego.
            \item Użytkownik podaje nazwę czatu grupowego.
            \item Użytkownik wybiera uczestników grupy.
            \item Użytkownik zatwierdza utworzenie czatu.
            \item System tworzy czat grupowy i dodaje do niego wskazanych użytkowników.
            \item System otwiera widok nowego czatu grupowego.
        \end{enumerate}
    }
    \ucalt{%
        \begin{enumerate}[nosep,leftmargin=21pt,labelindent=0pt,label={}]
            \item[3a.] Użytkownik nie wybrał żadnego dodatkowego uczestnika – system informuje, że czat grupowy wymaga co najmniej dwóch uczestników poza twórcą.
        \end{enumerate}
    }
}

\usecasecard{tab:pu23-lista-czatow}{Przeglądanie listy czatów}{PU23}{%
    \ucpriority{Wysoki}
    \ucactors{Użytkownik zalogowany}
    \ucdesc{Użytkownik przegląda listę swoich czatów prywatnych i grupowych.}
    \ucpre{Użytkownik jest zalogowany i otwiera moduł czatu.}
    \ucpost{Lista czatów użytkownika została wyświetlona.}
    \ucmain{%
        \begin{enumerate}[nosep,leftmargin=16pt,labelindent=0pt]
            \item System pobiera listę czatów użytkownika.
            \item System wyświetla listę czatów z podstawowymi informacjami (nazwa, ostatnia wiadomość, liczba nieprzeczytanych).
            \item Użytkownik wybiera czat z listy.
            \item System otwiera widok wybranego czatu.
        \end{enumerate}
    }
    \ucalt{Brak istotnych alternatywnych przepływów.}
}

\usecasecard{tab:pu24-wyslij-wiadomosc}{Wysyłanie wiadomości na czacie}{PU24}{%
    \ucpriority{Wysoki}
    \ucactors{Użytkownik zalogowany}
    \ucdesc{Użytkownik wysyła wiadomość tekstową na czacie prywatnym lub grupowym.}
    \ucpre{Użytkownik jest zalogowany i znajduje się w widoku konkretnego czatu.}
    \ucpost{Nowa wiadomość jest zapisana i widoczna w historii czatu.}
    \ucmain{%
        \begin{enumerate}[nosep,leftmargin=16pt,labelindent=0pt]
            \item Użytkownik wpisuje treść wiadomości w polu edycji.
            \item Użytkownik wysyła wiadomość (np. naciskając Enter lub ikonę wysyłania).
            \item System zapisuje wiadomość i dostarcza ją do uczestników czatu.
            \item System wyświetla wiadomość na liście wiadomości.
        \end{enumerate}
    }
    \ucalt{%
        \begin{enumerate}[nosep,leftmargin=21pt,labelindent=0pt,label={}]
            \item[2a.] Treść wiadomości jest pusta – system blokuje wysłanie i pozostaje w tym samym widoku.
        \end{enumerate}
    }
}

\usecasecard{tab:pu25-wyslij-gifa}{Wysyłanie GIF-a na czacie}{PU25}{%
    \ucpriority{Średni}
    \ucactors{Użytkownik zalogowany, Usługa GIF-ów}
    \ucdesc{Użytkownik wysyła animację GIF w konwersacji czatowej.}
    \ucpre{Użytkownik jest zalogowany i znajduje się w widoku czatu.}
    \ucpost{Wybrany GIF został dodany jako wiadomość w czacie.}
    \ucmain{%
        \begin{enumerate}[nosep,leftmargin=16pt,labelindent=0pt]
            \item Użytkownik wybiera opcję dodania GIF-a.
            \item System otwiera okno wyszukiwarki GIF-ów.
            \item Użytkownik wybiera lub wyszukuje GIF-a.
            \item Użytkownik zatwierdza wysłanie GIF-a.
            \item System dodaje GIF-a jako wiadomość na czacie.
        \end{enumerate}
    }
    \ucalt{%
        \begin{enumerate}[nosep,leftmargin=21pt,labelindent=0pt,label={}]
            \item[2a.] Usługa GIF-ów jest niedostępna – system informuje o braku możliwości wysłania GIF-a.
        \end{enumerate}
    }
}

\usecasecard{tab:pu26-wyslij-plik}{Wysyłanie pliku na czacie}{PU26}{%
    \ucpriority{Wysoki}
    \ucactors{Użytkownik zalogowany, Usługa do przechowywania plików w chmurze}
    \ucdesc{Użytkownik wysyła plik (np. zdjęcie, film) w czacie.}
    \ucpre{Użytkownik jest zalogowany i znajduje się w widoku czatu.}
    \ucpost{Plik został zapisany w chmurze i powiązany z wiadomością na czacie.}
    \ucmain{%
        \begin{enumerate}[nosep,leftmargin=16pt,labelindent=0pt]
            \item Użytkownik wybiera opcję dodania pliku.
            \item Użytkownik wybiera plik z urządzenia.
            \item System przesyła plik do usługi przechowywania w chmurze.
            \item System tworzy wiadomość z odnośnikiem do pliku.
            \item System wyświetla wiadomość na liście czatu.
        \end{enumerate}
    }
    \ucalt{%
        \begin{enumerate}[nosep,leftmargin=21pt,labelindent=0pt,label={}]
            \item[3a.] Przesyłanie pliku nie powiodło się – system informuje użytkownika i umożliwia ponowną próbę.
        \end{enumerate}
    }
}

\usecasecard{tab:pu27-edycja-czatu}{Edycja ustawień czatu}{PU27}{%
    \ucpriority{Średni}
    \ucactors{Użytkownik zalogowany}
    \ucdesc{Użytkownik modyfikuje ustawienia czatu (np. nazwę, avatar, tryb powiadomień).}
    \ucpre{Użytkownik jest zalogowany i ma uprawnienia do edycji danego czatu.}
    \ucpost{Zaktualizowane ustawienia czatu są zapisane i widoczne dla uczestników.}
    \ucmain{%
        \begin{enumerate}[nosep,leftmargin=16pt,labelindent=0pt]
            \item Użytkownik otwiera panel ustawień czatu.
            \item Użytkownik wprowadza zmiany (np. nazwę, opis, avatar).
            \item Użytkownik zapisuje zmiany.
            \item System waliduje dane i aktualizuje konfigurację czatu.
        \end{enumerate}
    }
    \ucalt{Brak istotnych alternatywnych przepływów poza walidacją pól.}
}

\usecasecard{tab:pu28-dodaj-czlonka}{Dodanie członka do czatu grupowego}{PU28}{%
    \ucpriority{Średni}
    \ucactors{Użytkownik zalogowany}
    \ucdesc{Użytkownik z uprawnieniami administratora dodaje nowego uczestnika do czatu grupowego.}
    \ucpre{Użytkownik jest zalogowany, znajduje się w czacie grupowym i ma prawo zarządzać członkami.}
    \ucpost{Nowy uczestnik został dodany do czatu grupowego.}
    \ucmain{%
        \begin{enumerate}[nosep,leftmargin=16pt,labelindent=0pt]
            \item Użytkownik otwiera listę uczestników czatu grupowego.
            \item Użytkownik wybiera opcję dodania nowego członka.
            \item Użytkownik wskazuje użytkownika do dodania.
            \item System dodaje wskazanego użytkownika do czatu grupowego.
        \end{enumerate}
    }
    \ucalt{%
        \begin{enumerate}[nosep,leftmargin=21pt,labelindent=0pt,label={}]
            \item[3a.] Użytkownik, którego chcemy dodać, zablokował zapraszającego – system informuje o braku możliwości dodania.
        \end{enumerate}
    }
}

\usecasecard{tab:pu29-historia-czatu}{Przeszukiwanie historii czatu}{PU29}{%
    \ucpriority{Średni}
    \ucactors{Użytkownik premium}
    \ucdesc{Użytkownik wyszukuje konkretne wiadomości w historii czatu.}
    \ucpre{Użytkownik jest zalogowany jako użytkownik premium i znajduje się w widoku czatu.}
    \ucpost{Wiadomości spełniające kryteria wyszukiwania zostały wyświetlone.}
    \ucmain{%
        \begin{enumerate}[nosep,leftmargin=16pt,labelindent=0pt]
            \item Użytkownik otwiera pole wyszukiwania historii w czacie.
            \item Użytkownik wpisuje frazę lub filtr (np. zakres dat, autor).
            \item System filtruje wiadomości zgodnie z kryteriami.
            \item System prezentuje listę dopasowanych fragmentów rozmowy.
        \end{enumerate}
    }
    \ucalt{Brak istotnych alternatywnych przepływów.}
}

\usecasecard{tab:pu30-wyslane-pliki}{Przeglądanie wysłanych plików na czacie}{PU30}{%
    \ucpriority{Średni}
    \ucactors{Użytkownik premium, Usługa do przechowywania plików w chmurze}
    \ucdesc{Użytkownik przegląda listę plików wysłanych w ramach czatów.}
    \ucpre{Użytkownik jest zalogowany jako użytkownik premium.}
    \ucpost{Użytkownik widzi listę wysłanych plików i może przechodzić do powiązanych czatów.}
    \ucmain{%
        \begin{enumerate}[nosep,leftmargin=16pt,labelindent=0pt]
            \item Użytkownik otwiera sekcję „Wysłane pliki”.
            \item System pobiera metadane plików z usługi przechowywania.
            \item System wyświetla listę plików z podstawowymi informacjami (nazwa, typ, data).
            \item Użytkownik wybiera plik, aby otworzyć go lub przejść do powiązanego czatu.
        \end{enumerate}
    }
    \ucalt{Brak istotnych alternatywnych przepływów.}
}

\usecasecard{tab:pu31-dodaj-spota}{Dodanie spota w profilu użytkownika}{PU31}{%
    \ucpriority{Wysoki}
    \ucactors{Użytkownik zalogowany, Usługa do wyświetlania mapy, Usługa do przechowywania plików w chmurze}
    \ucdesc{Użytkownik dodaje nowy spot poprzez swój profil.}
    \ucpre{Użytkownik jest zalogowany i znajduje się w widoku swojego profilu.}
    \ucpost{Nowy spot został zapisany i widoczny na mapie oraz w profilu użytkownika.}
    \ucmain{%
        \begin{enumerate}[nosep,leftmargin=16pt,labelindent=0pt]
            \item Użytkownik wybiera opcję „Dodaj spota”.
            \item Użytkownik uzupełnia podstawowe informacje o spocie (nazwa, opis, typ).
            \item Użytkownik wskazuje lokalizację spota na mapie.
            \item (Opcjonalnie) Użytkownik dodaje zdjęcia/filmy do spota.
            \item Użytkownik zapisuje spota.
            \item System zapisuje dane spota (oraz pliki w chmurze) i aktualizuje mapę oraz profil użytkownika.
        \end{enumerate}
    }
    \ucalt{%
        \begin{enumerate}[nosep,leftmargin=21pt,labelindent=0pt,label={}]
            \item[2a.] Formularz zawiera błędy – system wyświetla komunikat i zaznacza wymagające poprawy pola.
        \end{enumerate}
    }
}

\usecasecard{tab:pu32-profil-wlasny}{Przeglądanie profilu użytkownika}{PU32}{%
    \ucpriority{Wysoki}
    \ucactors{Użytkownik zalogowany}
    \ucdesc{Użytkownik przegląda swój profil (lista spotów, media, podstawowe dane).}
    \ucpre{Użytkownik jest zalogowany.}
    \ucpost{Wyświetlony jest widok profilu użytkownika wraz z jego zawartością.}
    \ucmain{%
        \begin{enumerate}[nosep,leftmargin=16pt,labelindent=0pt]
            \item Użytkownik otwiera swój profil.
            \item System pobiera dane profilu (informacje podstawowe, spoty, media).
            \item System wyświetla dane w odpowiednich sekcjach (spoty, zdjęcia, filmy, komentarze).
        \end{enumerate}
    }
    \ucalt{Brak istotnych alternatywnych przepływów.}
}

\usecasecard{tab:pu33-profil-innego}{Przeglądanie profilu innego użytkownika}{PU33}{%
    \ucpriority{Średni}
    \ucactors{Użytkownik zalogowany}
    \ucdesc{Użytkownik ogląda profil innego użytkownika (np. z mapy, forum lub społeczności).}
    \ucpre{Użytkownik jest zalogowany i ma dostęp do odnośnika do profilu innego użytkownika.}
    \ucpost{Profil innego użytkownika został wyświetlony.}
    \ucmain{%
        \begin{enumerate}[nosep,leftmargin=16pt,labelindent=0pt]
            \item Użytkownik wybiera odnośnik do profilu innego użytkownika.
            \item System pobiera publiczne dane profilu docelowego użytkownika.
            \item System wyświetla profil (spoty, media, podstawowe informacje).
        \end{enumerate}
    }
    \ucalt{%
        \begin{enumerate}[nosep,leftmargin=21pt,labelindent=0pt,label={}]
            \item[2a.] Profil jest ustawiony jako prywatny – system wyświetla ograniczoną ilość informacji i komunikat o prywatności.
        \end{enumerate}
    }
}

\usecasecard{tab:pu34-multimedia-spotow}{Przeglądanie multimediów powiązanych ze spotami}{PU34}{%
    \ucpriority{Średni}
    \ucactors{Użytkownik zalogowany}
    \ucdesc{Użytkownik przegląda zdjęcia i filmy dodane do spotów.}
    \ucpre{Użytkownik znajduje się w swoim profilu lub w widoku szczegółów spota.}
    \ucpost{Lista zdjęć i filmów powiązanych ze spotami została wyświetlona.}
    \ucmain{%
        \begin{enumerate}[nosep,leftmargin=16pt,labelindent=0pt]
            \item Użytkownik przechodzi do zakładki multimediów.
            \item System pobiera listę multimediów z wybranych spotów.
            \item System wyświetla miniatury zdjęć i filmów.
            \item Użytkownik otwiera wybrane medium w powiększeniu.
        \end{enumerate}
    }
    \ucalt{Brak istotnych alternatywnych przepływów.}
}

\usecasecard{tab:pu35-dodaj-znajomego}{Dodanie użytkownika do znajomych}{PU35}{%
    \ucpriority{Średni}
    \ucactors{Użytkownik zalogowany}
    \ucdesc{Użytkownik wysyła lub akceptuje zaproszenie do znajomych.}
    \ucpre{Użytkownik jest zalogowany i przegląda profil innego użytkownika.}
    \ucpost{Relacja „znajomy” została utworzona lub zaproszenie czeka na akceptację.}
    \ucmain{%
        \begin{enumerate}[nosep,leftmargin=16pt,labelindent=0pt]
            \item Użytkownik klika przycisk „Dodaj do znajomych”.
            \item System sprawdza, czy relacja już istnieje.
            \item System tworzy nowe zaproszenie lub bezpośrednio ustanawia relację (w zależności od modelu).
            \item System informuje o statusie (wysłano zaproszenie lub dodano do znajomych).
        \end{enumerate}
    }
    \ucalt{%
        \begin{enumerate}[nosep,leftmargin=21pt,labelindent=0pt,label={}]
            \item[2a.] Użytkownik docelowy zablokował nadawcę – system informuje, że nie można wysłać zaproszenia.
        \end{enumerate}
    }
}

\usecasecard{tab:pu36-spolecznosci}{Przeglądanie społeczności (social)}{PU36}{%
    \ucpriority{Średni}
    \ucactors{Użytkownik zalogowany}
    \ucdesc{Użytkownik przegląda społeczności, grupy lub listy znajomych powiązane z aplikacją.}
    \ucpre{Użytkownik jest zalogowany.}
    \ucpost{Lista społeczności lub znajomych została wyświetlona.}
    \ucmain{%
        \begin{enumerate}[nosep,leftmargin=16pt,labelindent=0pt]
            \item Użytkownik przechodzi do sekcji społeczności.
            \item System pobiera listę społeczności i znajomych użytkownika.
            \item System wyświetla listę z możliwością przechodzenia do profili i czatów.
        \end{enumerate}
    }
    \ucalt{Brak istotnych alternatywnych przepływów.}
}

\usecasecard{tab:pu37-zarzadzaj-komentarzami-spot}{Zarządzanie komentarzami do spotów}{PU37}{%
    \ucpriority{Wysoki}
    \ucactors{Użytkownik zalogowany (właściciel spota lub moderator)}
    \ucdesc{Użytkownik zarządza komentarzami dodanymi do spota (edycja, usuwanie, ukrywanie).}
    \ucpre{Użytkownik jest zalogowany i wyświetla szczegóły spota.}
    \ucpost{Wybrane komentarze zostały zaktualizowane lub ukryte zgodnie z działaniem użytkownika.}
    \ucmain{%
        \begin{enumerate}[nosep,leftmargin=16pt,labelindent=0pt]
            \item Użytkownik otwiera panel zarządzania komentarzami dla danego spota.
            \item System pobiera listę komentarzy wraz z możliwymi akcjami.
            \item Użytkownik wybiera komentarz i akcję (np. edytuj, usuń, ukryj).
            \item System wykonuje wybraną akcję na komentarzu.
            \item System odświeża listę komentarzy.
        \end{enumerate}
    }
    \ucalt{%
        \begin{enumerate}[nosep,leftmargin=21pt,labelindent=0pt,label={}]
            \item[3a.] Użytkownik nie ma uprawnień do zarządzania komentarzem – system wyświetla komunikat o braku uprawnień.
        \end{enumerate}
    }
}

\usecasecard{tab:pu38-zarzadzaj-komentarzami-forum}{Zarządzanie komentarzami na forum}{PU38}{%
    \ucpriority{Wysoki}
    \ucactors{Użytkownik zalogowany (autor posta lub moderator)}
    \ucdesc{Użytkownik zarządza komentarzami pod postami forum (edycja, usuwanie, przypinanie).}
    \ucpre{Użytkownik jest zalogowany i ma dostęp do danego wątku forum.}
    \ucpost{Komentarze zostały zaktualizowane zgodnie z działaniami użytkownika.}
    \ucmain{%
        \begin{enumerate}[nosep,leftmargin=16pt,labelindent=0pt]
            \item Użytkownik otwiera widok komentarzy pod postem.
            \item Użytkownik wybiera komentarz i odpowiednią akcję.
            \item System weryfikuje uprawnienia użytkownika.
            \item System wykonuje wybraną akcję i aktualizuje widok.
        \end{enumerate}
    }
    \ucalt{%
        \begin{enumerate}[nosep,leftmargin=21pt,labelindent=0pt,label={}]
            \item[3a.] Użytkownik nie ma wymaganych uprawnień – system blokuje operację i informuje o tym.
        \end{enumerate}
    }
}

\usecasecard{tab:pu39-zarzadzaj-postami}{Zarządzanie postami na forum}{PU39}{%
    \ucpriority{Wysoki}
    \ucactors{Użytkownik zalogowany (autor posta lub moderator)}
    \ucdesc{Użytkownik edytuje, archiwizuje lub usuwa własne posty na forum.}
    \ucpre{Użytkownik jest zalogowany i otwiera listę swoich postów lub moderowany dział forum.}
    \ucpost{Status wybranych postów został zaktualizowany.}
    \ucmain{%
        \begin{enumerate}[nosep,leftmargin=16pt,labelindent=0pt]
            \item Użytkownik przechodzi do sekcji zarządzania postami.
            \item System pobiera listę postów użytkownika (lub działu).
            \item Użytkownik wybiera post i żądaną akcję (edycja, archiwizacja, usunięcie).
            \item System zapisuje zmiany i aktualizuje listę postów.
        \end{enumerate}
    }
    \ucalt{%
        \begin{enumerate}[nosep,leftmargin=21pt,labelindent=0pt,label={}]
            \item[3a.] Użytkownik próbuje usunąć post z zablokowanego wątku – system odmawia wykonania operacji.
        \end{enumerate}
    }
}

\usecasecard{tab:pu40-zglos-komentarz}{Zgłoszenie komentarza naruszającego regulamin}{PU40}{%
    \ucpriority{Średni}
    \ucactors{Użytkownik zalogowany}
    \ucdesc{Użytkownik zgłasza komentarz (na forum lub pod spotem) jako naruszający regulamin.}
    \ucpre{Użytkownik widzi komentarz w aplikacji.}
    \ucpost{Zgłoszenie komentarza zostało zapisane i trafiło do kolejki moderacyjnej.}
    \ucmain{%
        \begin{enumerate}[nosep,leftmargin=16pt,labelindent=0pt]
            \item Użytkownik wybiera opcję „Zgłoś komentarz”.
            \item Użytkownik określa powód zgłoszenia.
            \item System zapisuje zgłoszenie i wiąże je z komentarzem i zgłaszającym.
        \end{enumerate}
    }
    \ucalt{Brak istotnych alternatywnych przepływów.}
}

\usecasecard{tab:pu41-zglos-spota}{Zgłoszenie spota}{PU41}{%
    \ucpriority{Średni}
    \ucactors{Użytkownik zalogowany}
    \ucdesc{Użytkownik zgłasza spota jako nieaktualnego, niebezpiecznego lub naruszającego przepisy.}
    \ucpre{Wyświetlony jest widok szczegółów spota.}
    \ucpost{Zgłoszenie dotyczące spota zostało zapisane w systemie.}
    \ucmain{%
        \begin{enumerate}[nosep,leftmargin=16pt,labelindent=0pt]
            \item Użytkownik wybiera opcję „Zgłoś spota”.
            \item Użytkownik wskazuje powód zgłoszenia i dodaje opis.
            \item System zapisuje zgłoszenie i informuje, że zostanie ono przeanalizowane.
        \end{enumerate}
    }
    \ucalt{Brak istotnych alternatywnych przepływów.}
}

\usecasecard{tab:pu42-zglos-posta}{Zgłoszenie posta na forum}{PU42}{%
    \ucpriority{Średni}
    \ucactors{Użytkownik zalogowany}
    \ucdesc{Użytkownik zgłasza post forum jako naruszający regulamin lub tematykę.}
    \ucpre{Wyświetlony jest widok posta na forum.}
    \ucpost{Zgłoszenie posta zostało zapisane i przekazane moderatorom.}
    \ucmain{%
        \begin{enumerate}[nosep,leftmargin=16pt,labelindent=0pt]
            \item Użytkownik wybiera opcję „Zgłoś post”.
            \item Użytkownik wybiera kategorię naruszenia i potwierdza zgłoszenie.
            \item System zapisuje zgłoszenie i oznacza post jako zgłoszony.
        \end{enumerate}
    }
    \ucalt{Brak istotnych alternatywnych przepływów.}
}

\usecasecard{tab:pu43-moderacja-zgloszen}{Moderacja zgłoszonych treści}{PU43}{%
    \ucpriority{Wysoki}
    \ucactors{Użytkownik zalogowany (moderator)}
    \ucdesc{Moderator przegląda zgłoszone posty, spoty i komentarze oraz podejmuje decyzje moderacyjne.}
    \ucpre{Moderator jest zalogowany i ma dostęp do panelu zgłoszeń.}
    \ucpost{Treści zostały oznaczone jako zaakceptowane, usunięte lub zablokowane.}
    \ucmain{%
        \begin{enumerate}[nosep,leftmargin=16pt,labelindent=0pt]
            \item Moderator otwiera panel zgłoszeń.
            \item System wyświetla listę zgłoszonych treści z podstawowymi informacjami.
            \item Moderator wybiera zgłoszenie i analizuje treść.
            \item Moderator podejmuje decyzję (np. odrzuć, usuń treść, zablokuj użytkownika).
            \item System zapisuje decyzję i aktualizuje stan treści.
        \end{enumerate}
    }
    \ucalt{Brak istotnych alternatywnych przepływów.}
}

\usecasecard{tab:pu44-zmien-typ-mapy}{Zmiana typu mapy}{PU44}{%
    \ucpriority{Średni}
    \ucactors{Użytkownik premium, Usługa do wyświetlania mapy}
    \ucdesc{Użytkownik zmienia typ mapy (np. standardowa, satelitarna, hybrydowa).}
    \ucpre{Użytkownik premium jest na ekranie mapy.}
    \ucpost{Mapa jest wyświetlana w wybranym typie.}
    \ucmain{%
        \begin{enumerate}[nosep,leftmargin=16pt,labelindent=0pt]
            \item Użytkownik otwiera ustawienia widoku mapy.
            \item Użytkownik wybiera typ mapy z dostępnej listy.
            \item System przełącza widok mapy na wybrany typ.
        \end{enumerate}
    }
    \ucalt{%
        \begin{enumerate}[nosep,leftmargin=21pt,labelindent=0pt,label={}]
            \item[3a.] Wybrany typ mapy nie jest dostępny (błąd usługi mapowej) – system przywraca poprzedni typ i informuje o błędzie.
        \end{enumerate}
    }
}

\usecasecard{tab:pu45-strefy-pansa}{Przeglądanie stref PANSA}{PU45}{%
    \ucpriority{Wysoki}
    \ucactors{Użytkownik premium, Usługa do wyświetlania mapy}
    \ucdesc{Użytkownik wyświetla na mapie strefy przestrzeni powietrznej \gls{PANSA}.}
    \ucpre{Użytkownik premium ma otwarty moduł mapy.}
    \ucpost{Strefy PANSA zostały zwizualizowane na mapie.}
    \ucmain{%
        \begin{enumerate}[nosep,leftmargin=16pt,labelindent=0pt]
            \item Użytkownik włącza warstwę „Strefy PANSA”.
            \item System pobiera dane o strefach z odpowiedniego źródła.
            \item System nakłada kontury stref na mapę.
        \end{enumerate}
    }
    \ucalt{%
        \begin{enumerate}[nosep,leftmargin=21pt,labelindent=0pt,label={}]
            \item[2a.] Dane o strefach są chwilowo niedostępne – system komunikuje problem i nie włącza warstwy.
        \end{enumerate}
    }
}

\usecasecard{tab:pu46-mapa-ze-strefami}{Korzystanie z mapy ze strefami PANSA}{PU46}{%
    \ucpriority{Średni}
    \ucactors{Użytkownik premium}
    \ucdesc{Użytkownik planuje lot, uwzględniając strefy PANSA wyświetlane na mapie.}
    \ucpre{Na mapie są włączone strefy PANSA.}
    \ucpost{Użytkownik zaplanował lokalizację lotu poza strefami ograniczeń lub świadomie w ich obrębie.}
    \ucmain{%
        \begin{enumerate}[nosep,leftmargin=16pt,labelindent=0pt]
            \item Użytkownik analizuje przebieg stref PANSA na mapie.
            \item Użytkownik wybiera spota lub lokalizację planowanego lotu.
            \item System informuje, czy lokalizacja znajduje się w strefie i jaki jest jej typ.
        \end{enumerate}
    }
    \ucalt{Brak istotnych alternatywnych przepływów.}
}

\usecasecard{tab:pu47-korzystaj-wyszukiwarka-spotow}{Korzystanie z wyszukiwarki spotów}{PU47}{%
    \ucpriority{Wysoki}
    \ucactors{Użytkownik}
    \ucdesc{Użytkownik korzysta z modułu wyszukiwania spotów z filtrami i sortowaniem.}
    \ucpre{Użytkownik ma dostęp do modułu wyszukiwarki spotów.}
    \ucpost{Użytkownik otrzymał listę spotów odpowiadających zadanym kryteriom.}
    \ucmain{%
        \begin{enumerate}[nosep,leftmargin=16pt,labelindent=0pt]
            \item Użytkownik otwiera moduł wyszukiwarki spotów.
            \item Użytkownik ustawia kryteria (np. lokalizacja, typ spota, poziom trudności).
            \item Użytkownik uruchamia wyszukiwanie.
            \item System zwraca listę spotów spełniających kryteria wraz z podstawowymi informacjami.
        \end{enumerate}
    }
    \ucalt{%
        \begin{enumerate}[nosep,leftmargin=21pt,labelindent=0pt,label={}]
            \item[4a.] Brak wyników – system informuje o braku dopasowań i proponuje rozluźnienie filtrów.
        \end{enumerate}
    }
}

\usecasecard{tab:pu48-korzystaj-mapa}{Korzystanie z mapy aplikacji}{PU48}{%
    \ucpriority{Wysoki}
    \ucactors{Użytkownik}
    \ucdesc{Użytkownik korzysta z modułu mapy do przeglądania spotów i innych warstw.}
    \ucpre{Użytkownik otworzył moduł mapy.}
    \ucpost{Użytkownik zapoznał się z dostępnymi spotami i warstwami mapy.}
    \ucmain{%
        \begin{enumerate}[nosep,leftmargin=16pt,labelindent=0pt]
            \item Użytkownik przesuwa i przybliża mapę zgodnie z potrzebą.
            \item System dociąga spoty dla bieżącego zakresu.
            \item Użytkownik włącza/wyłącza dostępne warstwy (np. strefy PANSA, typ mapy).
            \item System aktualizuje prezentację mapy zgodnie z ustawieniami.
        \end{enumerate}
    }
    \ucalt{Brak istotnych alternatywnych przepływów.}
}

\usecasecard{tab:pu49-korzystaj-forum}{Korzystanie z forum}{PU49}{%
    \ucpriority{Wysoki}
    \ucactors{Użytkownik}
    \ucdesc{Użytkownik korzysta z forum do wymiany informacji z innymi użytkownikami.}
    \ucpre{Użytkownik ma dostęp do modułu forum.}
    \ucpost{Użytkownik przegląda i tworzy treści na forum zgodnie z uprawnieniami.}
    \ucmain{%
        \begin{enumerate}[nosep,leftmargin=16pt,labelindent=0pt]
            \item Użytkownik wybiera dział forum.
            \item Użytkownik przegląda listę postów i komentarzy.
            \item Użytkownik dodaje nowe posty lub komentarze, jeśli jest zalogowany.
        \end{enumerate}
    }
    \ucalt{%
        \begin{enumerate}[nosep,leftmargin=21pt,labelindent=0pt,label={}]
            \item[3a.] Użytkownik niezalogowany próbuje dodać treść – system przekierowuje na ekran logowania.
        \end{enumerate}
    }
}

\usecasecard{tab:pu50-korzystaj-czat}{Korzystanie z czatu}{PU50}{%
    \ucpriority{Wysoki}
    \ucactors{Użytkownik zalogowany}
    \ucdesc{Użytkownik korzysta z modułu czatu do komunikacji prywatnej i grupowej.}
    \ucpre{Użytkownik jest zalogowany.}
    \ucpost{Użytkownik wysłał i odebrał wiadomości w wybranych czatach.}
    \ucmain{%
        \begin{enumerate}[nosep,leftmargin=16pt,labelindent=0pt]
            \item Użytkownik otwiera listę czatów.
            \item Użytkownik wybiera istniejący czat lub tworzy nowy.
            \item Użytkownik wysyła wiadomości tekstowe, GIF-y lub pliki.
            \item System w czasie rzeczywistym dostarcza wiadomości do pozostałych uczestników.
        \end{enumerate}
    }
    \ucalt{Brak istotnych alternatywnych przepływów.}
}

\usecasecard{tab:pu51-korzystaj-profil}{Korzystanie z profilu użytkownika}{PU51}{%
    \ucpriority{Wysoki}
    \ucactors{Użytkownik zalogowany}
    \ucdesc{Użytkownik zarządza swoim profilem oraz przegląda powiązane z nim treści.}
    \ucpre{Użytkownik jest zalogowany.}
    \ucpost{Profil użytkownika został zaktualizowany lub przejrzany.}
    \ucmain{%
        \begin{enumerate}[nosep,leftmargin=16pt,labelindent=0pt]
            \item Użytkownik otwiera swój profil.
            \item Użytkownik przegląda listę własnych spotów, komentarzy i multimediów.
            \item Użytkownik edytuje dane profilu lub dodaje nowe treści (np. spota).
        \end{enumerate}
    }
    \ucalt{Brak istotnych alternatywnych przepływów.}
}

\usecasecard{tab:pu52-wlasne-spoty}{Przeglądanie dodanych spotów}{PU52}{%
    \ucpriority{Średni}
    \ucactors{Użytkownik zalogowany}
    \ucdesc{Użytkownik wyświetla listę spotów, które sam dodał.}
    \ucpre{Użytkownik jest zalogowany i znajduje się w swoim profilu.}
    \ucpost{Lista własnych spotów została wyświetlona, z możliwością przejścia do szczegółów.}
    \ucmain{%
        \begin{enumerate}[nosep,leftmargin=16pt,labelindent=0pt]
            \item Użytkownik przechodzi do sekcji „Moje spoty”.
            \item System pobiera spoty dodane przez użytkownika.
            \item System wyświetla listę spotów z podstawowymi informacjami i linkami do mapy.
        \end{enumerate}
    }
    \ucalt{%
        \begin{enumerate}[nosep,leftmargin=21pt,labelindent=0pt,label={}]
            \item[2a.] Użytkownik nie dodał jeszcze żadnego spota – system wyświetla komunikat zachęcający do utworzenia pierwszego spota.
        \end{enumerate}
    }
}

\usecasecard{tab:pu53-wlasne-komentarze-spot}{Przeglądanie dodanych komentarzy do spotów}{PU53}{%
    \ucpriority{Średni}
    \ucactors{Użytkownik zalogowany}
    \ucdesc{Użytkownik przegląda komentarze, które dodał do różnych spotów.}
    \ucpre{Użytkownik jest zalogowany i ma dostęp do swojego profilu.}
    \ucpost{Lista komentarzy dodanych przez użytkownika została wyświetlona.}
    \ucmain{%
        \begin{enumerate}[nosep,leftmargin=16pt,labelindent=0pt]
            \item Użytkownik otwiera sekcję „Moje komentarze do spotów”.
            \item System pobiera komentarze powiązane z użytkownikiem.
            \item System wyświetla komentarze z linkami do odpowiednich spotów.
        \end{enumerate}
    }
    \ucalt{Brak istotnych alternatywnych przepływów.}
}

\usecasecard{tab:pu54-wlasne-zdjecia-spot}{Przeglądanie dodanych zdjęć do spotów}{PU54}{%
    \ucpriority{Średni}
    \ucactors{Użytkownik zalogowany, Usługa do przechowywania plików w chmurze}
    \ucdesc{Użytkownik przegląda zdjęcia przypisane do dodanych spotów.}
    \ucpre{Użytkownik jest zalogowany i posiada dodane zdjęcia.}
    \ucpost{Wyświetlono galerię zdjęć powiązanych ze spotami użytkownika.}
    \ucmain{%
        \begin{enumerate}[nosep,leftmargin=16pt,labelindent=0pt]
            \item Użytkownik przechodzi do zakładki „Moje zdjęcia”.
            \item System pobiera metadane zdjęć z chmury.
            \item System wyświetla miniatury zdjęć z możliwością ich powiększenia.
        \end{enumerate}
    }
    \ucalt{Brak istotnych alternatywnych przepływów.}
}

\usecasecard{tab:pu55-wlasne-filmy-spot}{Przeglądanie dodanych filmów do spotów}{PU55}{%
    \ucpriority{Średni}
    \ucactors{Użytkownik zalogowany, Usługa do przechowywania plików w chmurze}
    \ucdesc{Użytkownik przegląda filmy powiązane z jego spotami.}
    \ucpre{Użytkownik jest zalogowany i dodał co najmniej jeden film do spota.}
    \ucpost{Wyświetlono listę lub galerię filmów ze skrótowym podglądem.}
    \ucmain{%
        \begin{enumerate}[nosep,leftmargin=16pt,labelindent=0pt]
            \item Użytkownik otwiera zakładkę „Moje filmy”.
            \item System pobiera metadane filmów.
            \item System wyświetla listę miniatur filmów z możliwością odtworzenia.
        \end{enumerate}
    }
    \ucalt{Brak istotnych alternatywnych przepływów.}
}

\usecasecard{tab:pu56-strefa-czasowa-spota}{Określenie strefy czasowej spota}{PU56}{%
    \ucpriority{Średni}
    \ucactors{Użytkownik zalogowany, Usługa do określania strefy czasowej}
    \ucdesc{Podczas dodawania spota system ustala odpowiednią strefę czasową lokalizacji.}
    \ucpre{Użytkownik wypełnia formularz dodawania spota z lokalizacją.}
    \ucpost{Spot ma przypisaną poprawną strefę czasową.}
    \ucmain{%
        \begin{enumerate}[nosep,leftmargin=16pt,labelindent=0pt]
            \item Użytkownik wskazuje lokalizację spota na mapie.
            \item System wysyła zapytanie do usługi stref czasowych.
            \item System otrzymuje informację o strefie i zapisuje ją przy spocie.
        \end{enumerate}
    }
    \ucalt{%
        \begin{enumerate}[nosep,leftmargin=21pt,labelindent=0pt,label={}]
            \item[2a.] Usługa stref czasowych jest niedostępna – system zapisuje spota z domyślną strefą i oznacza brak pewności.
        \end{enumerate}
    }
}

\usecasecard{tab:pu57-korekta-czasu}{Korekta czasu zdarzeń do lokalnej strefy użytkownika}{PU57}{%
    \ucpriority{Średni}
    \ucactors{Użytkownik}
    \ucdesc{System prezentuje godziny zdarzeń (postów, spotów, wiadomości) w lokalnej strefie czasowej użytkownika.}
    \ucpre{Użytkownik korzysta z aplikacji i ma określoną strefę czasową.}
    \ucpost{Wszystkie prezentowane daty są przeliczone do strefy użytkownika.}
    \ucmain{%
        \begin{enumerate}[nosep,leftmargin=16pt,labelindent=0pt]
            \item System ustala strefę czasową użytkownika (np. z przeglądarki lub profilu).
            \item System przelicza daty i godziny zdarzeń do lokalnego czasu.
            \item Użytkownik ogląda zdarzenia z poprawnie przeliczonym czasem.
        \end{enumerate}
    }
    \ucalt{Brak istotnych alternatywnych przepływów.}
}

\usecasecard{tab:pu58-login-oauth}{Logowanie przy użyciu zewnętrznego dostawcy OAuth}{PU58}{%
    \ucpriority{Wysoki}
    \ucactors{Użytkownik niezalogowany, Usługa OAuth}
    \ucdesc{Użytkownik loguje się do systemu, korzystając z zewnętrznego dostawcy (np. Google).}
    \ucpre{Użytkownik jest na stronie logowania.}
    \ucpost{Użytkownik jest zalogowany przy użyciu konta zewnętrznego.}
    \ucmain{%
        \begin{enumerate}[nosep,leftmargin=16pt,labelindent=0pt]
            \item Użytkownik wybiera opcję logowania przez zewnętrznego dostawcę.
            \item System przekierowuje użytkownika do strony logowania dostawcy.
            \item Użytkownik autoryzuje dostęp aplikacji.
            \item Dostawca zwraca token autoryzacyjny.
            \item System loguje użytkownika na podstawie danych z dostawcy.
        \end{enumerate}
    }
    \ucalt{%
        \begin{enumerate}[nosep,leftmargin=21pt,labelindent=0pt,label={}]
            \item[3a.] Użytkownik anuluje autoryzację – system wraca na ekran logowania i nie loguje użytkownika.
        \end{enumerate}
    }
}

\usecasecard{tab:pu59-polacz-oauth}{Powiązanie konta z dostawcą OAuth}{PU59}{%
    \ucpriority{Średni}
    \ucactors{Użytkownik zalogowany, Usługa OAuth}
    \ucdesc{Użytkownik dodaje do swojego profilu powiązanie z kontem zewnętrznego dostawcy.}
    \ucpre{Użytkownik jest zalogowany i znajduje się w ustawieniach konta.}
    \ucpost{Konto użytkownika jest powiązane z wybranym dostawcą OAuth.}
    \ucmain{%
        \begin{enumerate}[nosep,leftmargin=16pt,labelindent=0pt]
            \item Użytkownik wybiera opcję „Połącz z kontem zewnętrznym”.
            \item System przekierowuje użytkownika do autoryzacji u dostawcy.
            \item Po akceptacji system zapisuje powiązanie z kontem użytkownika.
        \end{enumerate}
    }
    \ucalt{%
        \begin{enumerate}[nosep,leftmargin=21pt,labelindent=0pt,label={}]
            \item[2a.] Użytkownik odrzuca autoryzację – system wraca do ustawień konta bez zmian.
        \end{enumerate}
    }
}

\usecasecard{tab:pu60-potwierdz-email}{Potwierdzenie adresu e-mail po rejestracji}{PU60}{%
    \ucpriority{Wysoki}
    \ucactors{Użytkownik, Usługa SMTP}
    \ucdesc{Użytkownik aktywuje konto klikając link przesłany na e-mail.}
    \ucpre{Użytkownik przeszedł proces rejestracji, a system wysłał wiadomość aktywacyjną.}
    \ucpost{Konto użytkownika jest oznaczone jako aktywne.}
    \ucmain{%
        \begin{enumerate}[nosep,leftmargin=16pt,labelindent=0pt]
            \item Użytkownik otwiera wiadomość e-mail z linkiem aktywacyjnym.
            \item Użytkownik klika link aktywacyjny.
            \item System weryfikuje token aktywacyjny.
            \item System oznacza konto jako aktywne i loguje użytkownika lub przekierowuje na ekran logowania.
        \end{enumerate}
    }
    \ucalt{%
        \begin{enumerate}[nosep,leftmargin=21pt,labelindent=0pt,label={}]
            \item[3a.] Token jest nieprawidłowy lub wygasł – system informuje o błędzie i proponuje ponowne wysłanie linku.
        \end{enumerate}
    }
}

\usecasecard{tab:pu61-historia-platnosci}{Przeglądanie historii płatności i subskrypcji}{PU61}{%
    \ucpriority{Średni}
    \ucactors{Użytkownik zalogowany, System finansowo-księgowy}
    \ucdesc{Użytkownik przegląda historię opłat i status subskrypcji premium.}
    \ucpre{Użytkownik jest zalogowany i posiada historię transakcji.}
    \ucpost{Użytkownik otrzymał listę swoich płatności oraz aktualny status subskrypcji.}
    \ucmain{%
        \begin{enumerate}[nosep,leftmargin=16pt,labelindent=0pt]
            \item Użytkownik przechodzi do sekcji „Płatności i subskrypcje”.
            \item System pobiera dane o transakcjach z systemu finansowo-księgowego.
            \item System wyświetla listę transakcji wraz ze statusem subskrypcji.
        \end{enumerate}
    }
    \ucalt{%
        \begin{enumerate}[nosep,leftmargin=21pt,labelindent=0pt,label={}]
            \item[2a.] Dane płatności są chwilowo niedostępne – system wyświetla stosowny komunikat.
        \end{enumerate}
    }
}

\usecasecard{tab:pu62-anuluj-subskrypcje}{Anulowanie subskrypcji premium}{PU62}{%
    \ucpriority{Wysoki}
    \ucactors{Użytkownik zalogowany, System finansowo-księgowy}
    \ucdesc{Użytkownik rezygnuje z dalszego odnawiania subskrypcji premium.}
    \ucpre{Subskrypcja premium jest aktywna.}
    \ucpost{Subskrypcja została oznaczona jako nieodnawialna po zakończeniu bieżącego okresu.}
    \ucmain{%
        \begin{enumerate}[nosep,leftmargin=16pt,labelindent=0pt]
            \item Użytkownik wchodzi do ustawień subskrypcji.
            \item Użytkownik wybiera opcję anulowania subskrypcji.
            \item System przekazuje informację o anulowaniu do systemu finansowo-księgowego.
            \item System aktualizuje status subskrypcji użytkownika.
        \end{enumerate}
    }
    \ucalt{%
        \begin{enumerate}[nosep,leftmargin=21pt,labelindent=0pt,label={}]
            \item[3a.] System finansowo-księgowy zwraca błąd – system informuje użytkownika i pozostawia subskrypcję aktywną.
        \end{enumerate}
    }
}

\usecasecard{tab:pu63-autoodnowienie-subskrypcji}{Zarządzanie automatycznym odnowieniem subskrypcji}{PU63}{%
    \ucpriority{Średni}
    \ucactors{Użytkownik zalogowany, System finansowo-księgowy}
    \ucdesc{Użytkownik włącza lub wyłącza automatyczne odnawianie subskrypcji premium.}
    \ucpre{Użytkownik posiada aktywną lub niedawno wygasłą subskrypcję.}
    \ucpost{Ustawienia automatycznego odnowienia zostały zapisane.}
    \ucmain{%
        \begin{enumerate}[nosep,leftmargin=16pt,labelindent=0pt]
            \item Użytkownik otwiera ustawienia subskrypcji.
            \item Użytkownik przełącza opcję automatycznego odnowienia.
            \item System zapisuje zmianę i przekazuje ją do systemu finansowo-księgowego.
        \end{enumerate}
    }
    \ucalt{Brak istotnych alternatywnych przepływów.}
}

\usecasecard{tab:pu64-eksport-danych}{Eksport danych profilu użytkownika}{PU64}{%
    \ucpriority{Średni}
    \ucactors{Użytkownik zalogowany}
    \ucdesc{Użytkownik generuje paczkę z danymi swojego konta (np. na potrzeby RODO).}
    \ucpre{Użytkownik jest zalogowany.}
    \ucpost{System przygotował plik z danymi użytkownika i udostępnił go do pobrania.}
    \ucmain{%
        \begin{enumerate}[nosep,leftmargin=16pt,labelindent=0pt]
            \item Użytkownik otwiera sekcję prywatności w ustawieniach konta.
            \item Użytkownik wybiera opcję eksportu danych.
            \item System przygotowuje paczkę danych (spoty, posty, komentarze, profil).
            \item System udostępnia plik do pobrania użytkownikowi.
        \end{enumerate}
    }
    \ucalt{%
        \begin{enumerate}[nosep,leftmargin=21pt,labelindent=0pt,label={}]
            \item[3a.] Proces generowania paczki nie powiódł się – system informuje o błędzie i proponuje ponowną próbę.
        \end{enumerate}
    }
}

\usecasecard{tab:pu65-usuniecie-konta}{Usunięcie konta użytkownika}{PU65}{%
    \ucpriority{Wysoki}
    \ucactors{Użytkownik zalogowany}
    \ucdesc{Użytkownik trwale usuwa swoje konto wraz z powiązanymi danymi (zgodnie z polityką systemu).}
    \ucpre{Użytkownik jest zalogowany i potwierdził tożsamość (np. hasłem).}
    \ucpost{Konto użytkownika jest usunięte lub oznaczone jako zanonimizowane.}
    \ucmain{%
        \begin{enumerate}[nosep,leftmargin=16pt,labelindent=0pt]
            \item Użytkownik przechodzi do sekcji usuwania konta.
            \item Użytkownik zapoznaje się z konsekwencjami usunięcia konta.
            \item Użytkownik potwierdza operację (np. wpisując hasło lub kod).
            \item System usuwa lub anonimizuje dane użytkownika zgodnie z polityką.
        \end{enumerate}
    }
    \ucalt{%
        \begin{enumerate}[nosep,leftmargin=21pt,labelindent=0pt,label={}]
            \item[3a.] Użytkownik przerywa proces – system nie wykonuje żadnych zmian.
        \end{enumerate}
    }
}

\usecasecard{tab:pu66-dezaktywuj-spota}{Dezaktywacja spota przez właściciela}{PU66}{%
    \ucpriority{Średni}
    \ucactors{Użytkownik zalogowany}
    \ucdesc{Właściciel spota dezaktywuje go, aby nie był dłużej widoczny na mapie i w wyszukiwarce.}
    \ucpre{Użytkownik jest zalogowany i jest właścicielem spota.}
    \ucpost{Spot jest oznaczony jako nieaktywny i ukryty dla innych użytkowników.}
    \ucmain{%
        \begin{enumerate}[nosep,leftmargin=16pt,labelindent=0pt]
            \item Użytkownik otwiera szczegóły własnego spota.
            \item Użytkownik wybiera opcję dezaktywacji spota.
            \item System prosi o potwierdzenie operacji.
            \item Po potwierdzeniu system oznacza spota jako nieaktywny i ukrywa go z mapy oraz wyszukiwarki.
        \end{enumerate}
    }
    \ucalt{%
        \begin{enumerate}[nosep,leftmargin=21pt,labelindent=0pt,label={}]
            \item[3a.] Użytkownik rezygnuje z dezaktywacji – system pozostawia spota aktywnym.
        \end{enumerate}
    }
}
