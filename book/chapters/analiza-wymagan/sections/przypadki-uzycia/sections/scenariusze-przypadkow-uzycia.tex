%%! Author = Adam
%%! Date = 22/11/2025

\subsection{Scenariusze przypadków użycia}
\label{subsec:scenariusz-przypadkow-uzycia}

% --- Licznik dla scenariuszy przypadków użycia ---
\newcounter{usecase}[chapter]
\renewcommand{\theusecase}{\thechapter.\arabic{usecase}}

% --- Pomocnicze makra na pola w tabeli scenariusza ---
\newcommand{\ucpriority}[1]{\textbf{Priorytet} & #1 \\ \hline}
\newcommand{\ucactors}[1]{\textbf{Aktorzy} & #1 \\ \hline}
\newcommand{\ucdesc}[1]{\textbf{Opis} & #1 \\ \hline}
\newcommand{\ucpre}[1]{\textbf{Warunki wstępne} & #1 \\ \hline}
\newcommand{\ucpost}[1]{\textbf{Warunki końcowe} & #1 \\ \hline}
\newcommand{\ucmain}[1]{\textbf{Główny przepływ zdarzeń} & #1 \\ \hline}
\newcommand{\ucalt}[1]{\textbf{Alternatywne przepływy zdarzeń} & #1 \\ \hline}

\newcommand{\usecasecard}[4]{%
    \refstepcounter{usecase}%
    \begin{center}
    \renewcommand{\arraystretch}{1.15}%
    \begin{tabularx}{\textwidth}{|p{0.19\textwidth}|X|}
    \hline
    \textbf{Nazwa} & #2 \\ \hline
    \textbf{Numer} & #3 \\ \hline
    #4
    \end{tabularx}
    \vspace{3pt}
    \textbf{Tabela \theusecase:} Scenariusz przypadku użycia: #2\label{#1}
    \end{center}%
    \addcontentsline{lot}{table}{Tabela \theusecase: Scenariusz przypadku użycia: #2}%
}

% Scenariusze PU:
%TODO: na sam koniec zrob aby wygladem rpzypominaly te z uczelni
%TODO: do porpawyts a karty udzialowcow

\usecasecard{tab:pu1-rejestracja}{Rejestracja użytkownika}{PU1}{%
    \ucpriority{Wysoki}
    \ucactors{Użytkownik niezalogowany}
    \ucdesc{Użytkownik zakłada konto poprzez formularz rejestracji.}
    \ucpre{Użytkownik znajduje się na stronie z formularzem rejestracji.}
    \ucpost{Użytkownik posiada konto w systemie.}
    \ucmain{%
        \begin{enumerate}[nosep,leftmargin=16pt,labelindent=0pt]
            \item Użytkownik wypełnia formularz rejestracyjny.
            \item Użytkownik naciska przycisk rejestracji.
            \item System tworzy konto użytkownika.
            \item System loguje użytkownika i przenosi go na stronę główną aplikacji.
        \end{enumerate}
    }
    \ucalt{%
        \begin{enumerate}[nosep,leftmargin=21pt,labelindent=0pt,label={}]
            \item[1a.] Formularz zawiera niepoprawne dane – system wyświetla
            komunikat o błędzie oraz podświetla pola wymagające poprawy.
            \item[2a.] Nazwa użytkownika jest już zajęta – system wyświetla
            komunikat o błędzie.
        \end{enumerate}
    }

}
