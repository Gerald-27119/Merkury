%! Author = Mateusz
%! Date = 21/12/2025

\subsubsection{Wymagania pozafunkcjonalne dla panelu użytkownika}
\label{subsubsec:wymagania-pozafunkcjonalne-dla-panelu-uzytkownika}


% --- Szerokości ---

\newlength{\wppanelLabelWidth}
\settowidth{\wppanelLabelWidth}{\textbf{Wymagania powiązane:}}
\addtolength{\wppanelLabelWidth}{-20pt}

% Szerokość części z treścią (3 prawe kolumny zlane w jedną)
\newlength{\wppanelContentWidth}
\setlength{\wppanelContentWidth}{0.7\textwidth}
\addtolength{\wppanelContentWidth}{20pt}

\newlength{\wppanelHeaderHeight}
\setlength{\wppanelHeaderHeight}{12mm}
% --------- Pola karty (wiersze) ---------

\newcommand{\wppanelHeaderRow}[1]{%
    \rowcolor{lightgray}%
    \multicolumn{4}{|c|}{%
        \parbox[c][\wppanelHeaderHeight][c]{\linewidth}{%
            \centering\bfseries
            \vspace{1.2ex}%
            #1%
            \vspace{1.2ex}%
        }%
    }\\ \hline
}

% Id + priorytet – 4 kolumny
\newcommand{\wppanelpriority}[2]{%
    \textbf{Identyfikator:} & WPPANEL-#1 &
    \textbf{Priorytet:}     & #2 \\ \hline
}

% Typ – normalny wiersz na 3 kolumny
\newcommand{\wppaneltype}[1]{%
    \textbf{Typ:} &
    \multicolumn{3}{|>{\raggedright\arraybackslash}p{\wppanelContentWidth}|}{#1} \\ \hline
}

\newcommand{\wppanelname}[1]{%
    \textbf{Nazwa:} &
    \multicolumn{3}{|>{\raggedright\arraybackslash}p{\wppanelContentWidth}|}{#1} \\ \hline
}

\newcommand{\wppaneldesc}[1]{%
    \textbf{Opis:} &
    \multicolumn{3}{|>{\raggedright\arraybackslash}p{\wppanelContentWidth}|}{#1} \\ \hline
}

% (2) Kryteria akceptacji – ustawienia list jak w wymaganiach funkcjonalnych
\newcommand{\wppanelaccept}[1]{%
    \textbf{Kryteria akceptacji:} &
    \multicolumn{3}{|>{\raggedright\arraybackslash}p{\wppanelContentWidth}|}{%
        \begingroup
        \setlength{\leftmargini}{1.2em}%
        \setlength{\topsep}{0pt}%
        \setlength{\partopsep}{0pt}%
        \setlength{\itemsep}{0.2ex}%
        \setlength{\parsep}{0pt}%
        \vspace*{-1.8ex}%
        #1%
        \vspace*{-1.4ex}%
        \endgroup
    } \\ \hline
}

\newcommand{\wppanelstakeholder}[1]{%
    \textbf{Udziałowiec:} &
    \multicolumn{3}{|>{\raggedright\arraybackslash}p{\wppanelContentWidth}|}{#1} \\ \hline
}

\newcommand{\wppanelresponsible}[1]{%
    \textbf{Realizator:} &
    \multicolumn{3}{|>{\raggedright\arraybackslash}p{\wppanelContentWidth}|}{#1} \\ \hline
}

\newcommand{\wppanelrelated}[1]{%
    \textbf{Wymagania powiązane:} &
    \multicolumn{3}{|>{\raggedright\arraybackslash}p{\wppanelContentWidth}|}{#1} \\ \hline
}

\newcommand{\wppanelatcard}[5]{%
    \refstepcounter{awc}%
    {%
        \centering
        \begin{longtable}{|
                >{\columncolor{lightgray}\raggedright\arraybackslash}p{\wppanelLabelWidth}|
            l|
                >{\columncolor{lightgray}\raggedright\arraybackslash}l|
            p{0.15\textwidth}|}
        \hline
        \wppanelHeaderRow{\shortstack{KARTA WYMAGANIA POZAFUNKCJONALNEGO DLA \\ PANELU UŻYTKOWNIKA}}
        \endfirsthead
        \hline
        \wppanelHeaderRow{\shortstack{KARTA WYMAGANIA POZAFUNKCJONALNEGO DLA \\ PANELU UŻYTKOWNIKA (cd.)}}
        \endhead
        \wppanelpriority{#3}{#4}
        \wppanelname{#2}
        #5
        \end{longtable}
        \par
    }%
    \vspace{3pt}%
    \textbf{Tabela \theawc:} Wymaganie pozafunkcjonalne dla panelu użytkownika: #2\label{#1}%
    \addcontentsline{lot}{table}{Tabela \theawc: Wymaganie pozafunkcjonalne dla panelu użytkownika: #2}%
}

% =========================================================
% WPPANEL-01 – Autoryzacja: listy spotów (polubione/odwiedzone/planowane)
% Typ: bezpieczeństwo
% =========================================================

\wppanelatcard
{wppanel:spots-lists-authorization}
{Dostęp tylko do własnych list spotów (autoryzacja)}
{01}
{M}
{
    \wppaneltype{Bezpieczeństwo}

    \wppaneldesc{Wymaganie dotyczy \textbf{autoryzacji w panelu użytkownika}.
    System zapewnia, że użytkownik ma dostęp wyłącznie do własnych list spotów
        (polubione, odwiedzone i ocenione pozytywnie, odwiedzone i ocenione negatywnie, planowane)
        oraz nie może odczytać ani modyfikować list
        należących do innych użytkowników, nawet jeśli zna ich identyfikatory.}

    \wppanelaccept{%
        \begin{itemize}
            \item Widok list spotów prezentuje dane tylko dla konta aktualnie zalogowanego użytkownika.
            \item Zapytania do \gls{api} o listy spotów innego użytkownika są odrzucane (403) i nie zwracają danych.
            \item Operacje modyfikacji list (usunięcie spota z listy) są możliwe wyłącznie dla własnych list.
        \end{itemize}
    }

    \wppanelstakeholder{U3}
    \wppanelresponsible{Mateusz Redosz}
    \wppanelrelated{%
        \hyperref[wopanel:spots-lists]{WOPANEL-02},
        \hyperref[wfpanel:switch-spots-list-type]{WFPANEL-07},
        \hyperref[wfpanel:remove-spot-from-list]{WFPANEL-08},
        \hyperref[wfpanel:show-spot-on-map]{WFPANEL-09}.%
    }
}

% =========================================================
% WPPANEL-02 – Autoryzacja: lista dodanych spotów
% Typ: bezpieczeństwo
% =========================================================

\wppanelatcard
{wppanel:added-spots-authorization}
{Dostęp tylko do własnej listy z dodanymi spotami (autoryzacja)}
{02}
{M}
{
    \wppaneltype{Bezpieczeństwo}

    \wppaneldesc{System zapewnia, że użytkownik może przeglądać wyłącznie listę spotów
    dodanych przez siebie, a dane innych użytkowników nie są dostępne ani w interfejsie,
        ani poprzez \gls{api}.}

    \wppanelaccept{%
        \begin{itemize}
            \item Lista „Dodane spoty” zawiera wyłącznie spoty dodane przez zalogowanego użytkownika.
            \item Zapytania do \gls{api} o listę dodanych spotów innego użytkownika są odrzucane (403).
        \end{itemize}
    }

    \wppanelstakeholder{U3}
    \wppanelresponsible{Mateusz Redosz}
    \wppanelrelated{%
        \hyperref[wopanel:add-spot]{WOPANEL-03},
        \hyperref[wfpanel:added-spots-list]{WFPANEL-22},
        \hyperref[wfpanel:add-spot-form]{WFPANEL-23}.%
    }
}

% =========================================================
% WPPANEL-03 – Autoryzacja: lista filmów
% Typ: bezpieczeństwo
% =========================================================

\wppanelatcard
{wppanel:videos-authorization}
{Dostęp tylko do własnej listy filmów (autoryzacja)}
{03}
{M}
{
    \wppaneltype{Bezpieczeństwo}

    \wppaneldesc{System zapewnia, że użytkownik ma dostęp wyłącznie do własnej listy filmów
    powiązanych z kontem i nie może odczytać filmów innych użytkowników poprzez panel ani \gls{api}.}

    \wppanelaccept{%
        \begin{itemize}
            \item Lista filmów prezentuje wyłącznie materiały dodane przez zalogowanego użytkownika.
            \item Zapytania do \gls{api} o listę filmów innego użytkownika są odrzucane (403).
        \end{itemize}
    }

    \wppanelstakeholder{U3}
    \wppanelresponsible{Mateusz Redosz}
    \wppanelrelated{%
        \hyperref[wopanel:videos]{WOPANEL-05},
        \hyperref[wfpanel:videos-sort-by-date]{WFPANEL-13},
        \hyperref[wfpanel:videos-filter-by-date]{WFPANEL-14},
        \hyperref[wfpanel:videos-grouped-with-metrics]{WFPANEL-15}.%
    }
}

% =========================================================
% WPPANEL-04 – Autoryzacja: lista komentarzy
% Typ: bezpieczeństwo
% =========================================================

\wppanelatcard
{wppanel:comments-authorization}
{Dostęp tylko do własnej listy z dodanymi komentarzami (autoryzacja)}
{04}
{M}
{
    \wppaneltype{Bezpieczeństwo}

    \wppaneldesc{System zapewnia, że użytkownik ma dostęp wyłącznie do listy komentarzy
    dodanych przez siebie, bez możliwości odczytu komentarzy innych użytkowników z poziomu panelu konta.}

    \wppanelaccept{%
        \begin{itemize}
            \item Lista komentarzy prezentuje wyłącznie komentarze zalogowanego użytkownika.
            \item Zapytania do \gls{api} o komentarze innego użytkownika są odrzucane (403).
        \end{itemize}
    }

    \wppanelstakeholder{U3}
    \wppanelresponsible{Mateusz Redosz}
    \wppanelrelated{%
        \hyperref[wopanel:comments]{WOPANEL-07},
        \hyperref[wfpanel:comments-sort-by-date]{WFPANEL-24},
        \hyperref[wfpanel:comments-filter-by-date]{WFPANEL-25},
        \hyperref[wfpanel:comments-grouped-by-date-and-spot]{WFPANEL-26}.%
    }
}

% =========================================================
% WPPANEL-05 – Autoryzacja: znajomi/obserwacje
% Typ: bezpieczeństwo
% =========================================================

\wppanelatcard
{wppanel:social-authorization}
{Dostęp tylko do zarządzania własnymi znajomymi i obserwowanymi (autoryzacja)}
{05}
{M}
{
    \wppaneltype{Bezpieczeństwo}

    \wppaneldesc{System zapewnia, że użytkownik może zarządzać wyłącznie relacjami społecznościowymi
    w obrębie własnego konta (lista znajomych, obserwowani) i nie ma możliwości
    modyfikowania relacji innego użytkownika poprzez bezpośrednie wywołania \gls{api}.}

    \wppanelaccept{%
        \begin{itemize}
            \item Widoki społeczności prezentują dane tylko dla zalogowanego użytkownika.
            \item Próby modyfikacji relacji w imieniu innego użytkownika są odrzucane (403).
        \end{itemize}
    }

    \wppanelstakeholder{U3}
    \wppanelresponsible{Mateusz Redosz}
    \wppanelrelated{%
        \hyperref[wopanel:community]{WOPANEL-06},
        \hyperref[wfpanel:friend-follow-buttons]{WFPANEL-06},
        \hyperref[wfpanel:friends-list]{WFPANEL-16},
        \hyperref[wfpanel:following-list]{WFPANEL-17},
        \hyperref[wfpanel:followers-list]{WFPANEL-18},
        \hyperref[wfpanel:search-friends-modal]{WFPANEL-20},
        \hyperref[wfpanel:friend-actions]{WFPANEL-21}.%
    }
}

% =========================================================
% WPPANEL-06 – Autoryzacja: zaproszenia do znajomych
% Typ: bezpieczeństwo
% =========================================================

\wppanelatcard
{wppanel:invitations-authorization}
{Dostęp tylko do własnych zaproszeń do znajomych (autoryzacja)}
{06}
{M}
{
    \wppaneltype{Bezpieczeństwo}

    \wppaneldesc{System zapewnia, że użytkownik ma dostęp wyłącznie do własnych zaproszeń do znajomych
        (otrzymanych) oraz nie może podejrzeć zaproszeń innych użytkowników.}

    \wppanelaccept{%
        \begin{itemize}
            \item Lista zaproszeń prezentuje wyłącznie zaproszenia związane z kontem zalogowanego użytkownika.
            \item Zapytania do \gls{api} o zaproszenia innego użytkownika są odrzucane (403).
        \end{itemize}
    }

    \wppanelstakeholder{U3}
    \wppanelresponsible{Mateusz Redosz}
    \wppanelrelated{%
        \hyperref[wopanel:community]{WOPANEL-06},
        \hyperref[wfpanel:invitations-list]{WFPANEL-19}.%
    }
}

% =========================================================
% WPPANEL-07 – Uwierzytelnienie: panel wymaga zalogowania
% Typ: bezpieczeństwo
% =========================================================

\wppanelatcard
{wppanel:login-required}
{Korzystanie z panelu użytkownika wymaga zalogowania (uwierzytelnienie)}
{07}
{M}
{
    \wppaneltype{Bezpieczeństwo}

    \wppaneldesc{Jakakolwiek próba korzystania z panelu użytkownika (podgląd profilu, listy, ustawienia)
        wymaga wcześniejszego zalogowania się do systemu. Użytkownik niezalogowany nie ma dostępu
        do danych konta.}

    \wppanelaccept{%
        \begin{itemize}
            \item Wejście na podstrony panelu przez użytkownika niezalogowanego powoduje przekierowanie na stronę główną.
            \item Zapytania do \gls{api} wykonywane bez ważnych danych sesyjnych \gls{jwt} są odrzucane kodem 403.
        \end{itemize}
    }

    \wppanelstakeholder{U3}
    \wppanelresponsible{Mateusz Redosz}
    \wppanelrelated{%
        \hyperref[wopanel:profile]{WOPANEL-01},
        \hyperref[wopanel:spots-lists]{WOPANEL-02},
        \hyperref[wopanel:add-spot]{WOPANEL-03},
        \hyperref[wopanel:photos]{WOPANEL-04},
        \hyperref[wopanel:videos]{WOPANEL-05},
        \hyperref[wopanel:community]{WOPANEL-06},
        \hyperref[wopanel:comments]{WOPANEL-07},
        \hyperref[wopanel:settings]{WOPANEL-08}.%
    }
}

% =========================================================
% WPPANEL-08 – Użyteczność: grupowanie zdjęć po dacie
% =========================================================

\wppanelatcard
{wppanel:photos-group-by-date}
{Grupowanie zdjęć po dacie dodania}
{08}
{M}
{
    \wppaneltype{Użyteczność}

    \wppaneldesc{Zdjęcia w panelu użytkownika są prezentowane w logicznych grupach odpowiadających
    datom ich dodania, co ułatwia orientację w historii aktywności.}

    \wppanelaccept{%
        \begin{itemize}
            \item W widoku zdjęć widoczne są nagłówki grup (dzień/miesiąc/rok).
            \item Każde zdjęcie jest przypisane do poprawnej grupy daty dodania.
            \item Zmiana sortowania/filtrowania nie powoduje utraty poprawnego grupowania.
        \end{itemize}
    }

    \wppanelstakeholder{U3}
    \wppanelresponsible{Mateusz Redosz}
    \wppanelrelated{%
        \hyperref[wopanel:photos]{WOPANEL-04},
        \hyperref[wfpanel:photos-sort-by-date]{WFPANEL-10},
        \hyperref[wfpanel:photos-filter-by-date]{WFPANEL-11},
        \hyperref[wfpanel:photos-grouped-with-metrics]{WFPANEL-12}.%
    }
}

% =========================================================
% WPPANEL-09 – Użyteczność: grupowanie filmów po dacie
% =========================================================

\wppanelatcard
{wppanel:videos-group-by-date}
{Grupowanie filmów po dacie dodania}
{09}
{M}
{
    \wppaneltype{Użyteczność}

    \wppaneldesc{Filmy w panelu użytkownika są prezentowane w grupach odpowiadających datom ich dodania,
        co ułatwia przeglądanie materiałów.}

    \wppanelaccept{%
        \begin{itemize}
            \item W widoku filmów widoczne są nagłówki grup odpowiadające dacie dodania.
            \item Każdy film jest przypisany do poprawnej grupy daty dodania.
        \end{itemize}
    }

    \wppanelstakeholder{U3}
    \wppanelresponsible{Mateusz Redosz}
    \wppanelrelated{%
        \hyperref[wopanel:videos]{WOPANEL-05},
        \hyperref[wfpanel:videos-sort-by-date]{WFPANEL-13},
        \hyperref[wfpanel:videos-filter-by-date]{WFPANEL-14},
        \hyperref[wfpanel:videos-grouped-with-metrics]{WFPANEL-15}.%
    }
}

% =========================================================
% WPPANEL-10 – Użyteczność: grupowanie komentarzy po dacie i spocie
% =========================================================

\wppanelatcard
{wppanel:comments-group-by-date-spot}
{Grupowanie komentarzy po dacie dodania oraz spocie}
{10}
{M}
{
    \wppaneltype{Użyteczność}

    \wppaneldesc{Komentarze w panelu użytkownika są prezentowane w logicznych grupach
    według daty dodania oraz powiązanego spota, co ułatwia odszukanie kontekstu wypowiedzi.}

    \wppanelaccept{%
        \begin{itemize}
            \item Komentarze są pogrupowane po dacie dodania.
            \item W ramach danej daty komentarze są przypisane do nazw spotów.
            \item Zmiana zakresu danych (filtrowanie/sortowanie) zachowuje poprawne grupowanie.
        \end{itemize}
    }

    \wppanelstakeholder{U3}
    \wppanelresponsible{Mateusz Redosz}
    \wppanelrelated{%
        \hyperref[wopanel:comments]{WOPANEL-07},
        \hyperref[wfpanel:comments-sort-by-date]{WFPANEL-24},
        \hyperref[wfpanel:comments-filter-by-date]{WFPANEL-25},
        \hyperref[wfpanel:comments-grouped-by-date-and-spot]{WFPANEL-26}.%
    }
}

% =========================================================
% WPPANEL-11 – Wydajność: czas ładowania danych < 10 s
% =========================================================

\wppanelatcard
{wppanel:load-under-10s}
{Czas załadowania danych poniżej 10 sekund}
{11}
{S}
{
    \wppaneltype{Wydajność}

    \wppaneldesc{Czas załadowania danych w panelu użytkownika (pobieranie list i profilu)
        powinien mieścić się poniżej 10 sekund w typowych warunkach sieciowych.}

    \wppanelaccept{%
        \begin{itemize}
            \item W co najmniej 95\% pomiarów w warunkach deweloperskich czas odpowiedzi mieści się w przedziale 0--10 s.
            \item Interfejs sygnalizuje trwające ładowanie.
            \item Podczas pobierania danych nie występują zauważalne „zawieszenia” interfejsu.
        \end{itemize}
    }

    \wppanelstakeholder{U3}
    \wppanelresponsible{Mateusz Redosz}
    \wppanelrelated{%
        \hyperref[wopanel:profile]{WOPANEL-01},
        \hyperref[wopanel:spots-lists]{WOPANEL-02},
        \hyperref[wopanel:add-spot]{WOPANEL-03},
        \hyperref[wopanel:photos]{WOPANEL-04},
        \hyperref[wopanel:videos]{WOPANEL-05},
        \hyperref[wopanel:community]{WOPANEL-06},
        \hyperref[wopanel:comments]{WOPANEL-07},
        \hyperref[wopanel:settings]{WOPANEL-08}.%
    }
}

% =========================================================
% WPPANEL-12 – Wydajność: natychmiastowa aktualizacja po dodaniu spota
% =========================================================

\wppanelatcard
{wppanel:add-spot-immediate-update}
{Natychmiastowe wysłanie danych po dodaniu nowego spota}
{12}
{S}
{
    \wppaneltype{Wydajność}

    \wppaneldesc{Po dodaniu nowego spota użytkownik powinien otrzymać informację zwrotną o powodzeniu
    operacji w czasie subiektywnie natychmiastowym, a dane w panelu powinny zostać zaktualizowane
    bez konieczności ręcznego odświeżania strony.}

    \wppanelaccept{%
        \begin{itemize}
            \item Po wysłaniu formularza dodania spota użytkownik otrzymuje potwierdzenie powodzenia w czasie poniżej 1 s (w typowych warunkach sieciowych).
            \item Nowo dodany spot pojawia się na liście własnych spotów bez ręcznego odświeżania widoku (przez odświeżenie danych).
            \item W przypadku błędu API użytkownik otrzymuje czytelny komunikat o niepowodzeniu operacji.
        \end{itemize}
    }

    \wppanelstakeholder{U3}
    \wppanelresponsible{Mateusz Redosz}
    \wppanelrelated{%
        \hyperref[wopanel:add-spot]{WOPANEL-03},
        \hyperref[wfpanel:add-spot-form]{WFPANEL-23},
        \hyperref[wfpanel:added-spots-list]{WFPANEL-22}.%
    }
}
