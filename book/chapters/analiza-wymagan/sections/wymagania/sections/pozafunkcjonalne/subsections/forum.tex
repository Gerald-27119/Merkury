%! Author = kacper
%! Date = 01/01/2026

\subsubsection{Wymagania pozafunkcjonalne dla forum}
\label{subsubsec:wymagania-pozafunkcjonalne-dla-forum}

\newlength{\wpfLabelWidth}
\settowidth{\wpfLabelWidth}{\textbf{Wymagania powiązane:}}
\addtolength{\wpfLabelWidth}{-20pt}

\newlength{\wpfContentWidth}
\setlength{\wpfContentWidth}{0.7\textwidth}

\newcommand{\wpfpriority}[2]{
    \textbf{Identyfikator:} & WPFORUM-#1 &
    \textbf{Priorytet:}     & #2 \\ \hline
}

\newcommand{\wpftype}[1]{
    \textbf{Typ:} &
    \multicolumn{3}{|>{\raggedright\arraybackslash}p{\wpfContentWidth}|}{#1} \\ \hline
}

\newcommand{\wpfname}[1]{
    \textbf{Nazwa:} &
    \multicolumn{3}{|>{\raggedright\arraybackslash}p{\wpfContentWidth}|}{#1} \\ \hline
}

\newcommand{\wpfdesc}[1]{
    \textbf{Opis:} &
    \multicolumn{3}{|>{\raggedright\arraybackslash}p{\wpfContentWidth}|}{#1} \\ \hline
}

\newcommand{\wpfaccept}[1]{
    \textbf{Kryteria akceptacji:} &
    \multicolumn{3}{|>{\raggedright\arraybackslash}p{\wpfContentWidth}|}{%
        \begingroup
        \setlength{\leftmargini}{1.2em}
        \setlength{\topsep}{0pt}
        \setlength{\partopsep}{0pt}
        \setlength{\itemsep}{0.2ex}
        \setlength{\parsep}{0pt}
        \vspace*{-1.8ex}
        #1
        \vspace*{-1.4ex}
        \endgroup
    } \\ \hline
}

\newcommand{\wpfstakeholder}[1]{
    \textbf{Udziałowiec:} &
    \multicolumn{3}{|>{\raggedright\arraybackslash}p{\wpfContentWidth}|}{#1} \\ \hline
}

\newcommand{\wpfresponsible}[1]{
    \textbf{Realizator:} &
    \multicolumn{3}{|>{\raggedright\arraybackslash}p{\wpfContentWidth}|}{#1} \\ \hline
}

\newcommand{\wpfrelated}[1]{
    \textbf{Wymagania powiązane:} &
    \multicolumn{3}{|>{\raggedright\arraybackslash}p{\wpfContentWidth}|}{#1} \\ \hline
}

\newcommand{\wpforumcard}[5]{
    \refstepcounter{awc}
    {
        \centering
        \begin{longtable}{|
                >{\columncolor{lightgray}\raggedright\arraybackslash}p{\wpfLabelWidth}|
            l|
                >{\columncolor{lightgray}\raggedright\arraybackslash}l|
            p{0.15\textwidth}|}
        \hline
        \rowcolor{lightgray}\multicolumn{4}{|c|}{\textbf{KARTA WYMAGANIA POZAFUNKCJONALNEGO DLA FORUM}} \\ \hline
        \endfirsthead
        \hline
        \rowcolor{lightgray}\multicolumn{4}{|c|}{\textbf{KARTA WYMAGANIA POZAFUNKCJONALNEGO DLA FORUM (cd.)}} \\ \hline
        \endhead
        \wpfpriority{#3}{#4}
        \wpfname{#2}
        #5
        \end{longtable}
        \par
    }
    \vspace{3pt}
    \textbf{Tabela \theawc:} Wymaganie pozafunkcjonalne dla forum: #2\label{#1}
    \addcontentsline{lot}{table}{Tabela \theawc: Wymaganie pozafunkcjonalne dla forum: #2}
}

\wpforumcard
{wpforum:search-efficiency}
{Wydajność wyszukiwania}
{01}
{S}
{
    \wpftype{Wydajność}

    \wpfdesc{System powinien zapewniać szybkie wyszukiwanie i filtrowanie postów.}

    \wpfaccept{
        \begin{itemize}
            \item Wyniki wyszukiwania wyświetlane są w czasie mniejszym bądź równym 2 sekundy.
        \end{itemize}
    }

    \wpfstakeholder{U3}

    \wpfresponsible{Kacper Badek}

    \wpfrelated{
        \hyperref[wfforum:search-posts]{WFFORUM-08}
    }
}

\wpforumcard
{wpforum:read-only-mode}
{Tryb tylko do odczytu}
{02}
{S}
{
    \wpftype{Bezpieczeństwo}

    \wpfdesc{Niezalogowani użytkownicy powinni mieć dostęp do forum wyłącznie w trybie tylko do odczytu.}

    \wpfaccept{
        \begin{itemize}
            \item Niezalogowany użytkownik nie ma dostępu do operacji modyfikujących treści.
            \item Funkcje wymagające logowania są odpowiednio zablokowane lub ukryte.
            \item Próba dostępu do niedozwolonej operacji wyświetli komunikat informujący użytkownika, że musi się zalogować.
        \end{itemize}
    }

    \wpfstakeholder{U3}

    \wpfresponsible{Kacper Badek}

    \wpfrelated{
        \hyperref[wfforum:voting]{WFFORUM-09},
        \hyperref[wfforum:follow-posts]{WFFORUM-10},
        \hyperref[wfforum:view-followed]{WFFORUM-11},
        \hyperref[wfforum:manage-posts]{WFFORUM-12},
        \hyperref[wfforum:manage-comments]{WFFORUM-13}
    }
}

\wpforumcard
{wpforum:user-operation-security}
{Bezpieczeństwo operacji użytkownika}
{03}
{S}
{
    \wpftype{Bezpieczeństwo}

    \wpfdesc{System powinien zabezpieczać operacje użytkownika przed nieautoryzowanym dostępem.}

    \wpfaccept{
        \begin{itemize}
            \item Użytkownik może edytować i usuwać wyłącznie własne posty i komentarze.
            \item Próba nieautoryzowanej operacji jest blokowana przez system.
        \end{itemize}
    }

    \wpfstakeholder{U3}

    \wpfresponsible{Kacper Badek}

    \wpfrelated{
        \hyperref[wfforum:manage-posts]{WFFORUM-12},
        \hyperref[wfforum:manage-comments]{WFFORUM-13}
    }
}