%! Author = Adam
%! Date = 30/11/2025

\subsubsection{Wymagania pozafunkcjonalne dla czatu}
\label{subsubsec:wymagania-pozafunkcjonalne-dla-czatu}

% ============================
% WYMAGANIA POZAFUNKCJONALNE DLA CZATU
% Identyfikatory: WPCZAT-XX
% ============================

\newcounter{wpczat}[chapter]
\renewcommand{\thewpczat}{\thechapter.\arabic{wpczat}}

% Szerokość części z treścią (3 prawe kolumny zlane w jedną)
\newlength{\wpczContentWidth}
\setlength{\wpczContentWidth}{0.7\textwidth}

% --------- Pola karty (wiersze) ---------

% Id + priorytet – 4 kolumny
\newcommand{\wpczpriority}[2]{%
    \textbf{Identyfikator:} & WPCZAT-#1 &
    \textbf{Priorytet:}     & #2 \\ \hline
}

% Typ – normalny wiersz na 3 kolumny
\newcommand{\wpcztype}[1]{\textbf{Typ:} &
\multicolumn{3}{|p{\wpczContentWidth}|}{#1} \\ \hline}

% Wiersze z etykietą + treścią na 3 kolumny
\newcommand{\wpczname}[1]{\textbf{Nazwa:}              &
\multicolumn{3}{|p{\wpczContentWidth}|}{#1} \\ \hline}
\newcommand{\wpczdesc}[1]{\textbf{Opis:}               &
\multicolumn{3}{|p{\wpczContentWidth}|}{#1} \\ \hline}
\newcommand{\wpczaccept}[1]{\textbf{Kryteria akceptacji:} &
\multicolumn{3}{|p{\wpczContentWidth}|}{#1} \\ \hline}
\newcommand{\wpczinput}[1]{\textbf{Dane wejściowe:}    &
\multicolumn{3}{|p{\wpczContentWidth}|}{#1} \\ \hline}
\newcommand{\wpczpre}[1]{\textbf{Warunki początkowe:}  &
\multicolumn{3}{|p{\wpczContentWidth}|}{#1} \\ \hline}
\newcommand{\wpczpost}[1]{\textbf{Warunki końcowe:}    &
\multicolumn{3}{|p{\wpczContentWidth}|}{#1} \\ \hline}
\newcommand{\wpczexceptions}[1]{\textbf{Sytuacje wyjątkowe:} &
\multicolumn{3}{|p{\wpczContentWidth}|}{#1} \\ \hline}
\newcommand{\wpczimpl}[1]{\textbf{Szczegóły implementacji:} &
\multicolumn{3}{|p{\wpczContentWidth}|}{#1} \\ \hline}
\newcommand{\wpczstakeholder}[1]{\textbf{Udziałowiec:} &
\multicolumn{3}{|p{\wpczContentWidth}|}{#1} \\ \hline}
\newcommand{\wpczresponsible}[1]{\textbf{Realizator:}  &
\multicolumn{3}{|p{\wpczContentWidth}|}{#1} \\ \hline}
\newcommand{\wpczstatus}[1]{\textbf{Status:}           &
\multicolumn{3}{|p{\wpczContentWidth}|}{#1} \\ \hline}

% (opcjonalnie, gdyby kiedyś były potrzebne)
\newcommand{\wpcznote}[1]{\textbf{Notatka:}            &
\multicolumn{3}{|p{\wpczContentWidth}|}{#1} \\ \hline}

% Karta wymagania pozafunkcjonalnego dla czatu
% #1 – label
% #2 – nazwa
% #3 – numer / sufiks kodu (np. 01 → WPCZAT-01)
% #4 – priorytet (MoSCoW: M, S, C, W)
% #5 – reszta pól: \wpcztype, \wpczdesc, \wpczaccept, ...
\newcommand{\wpczatcard}[5]{%
    \refstepcounter{wpczat}%
    {%
        \centering
        \begin{longtable}{|
                >{\columncolor{lightgray}}l|
            l|
                >{\columncolor{lightgray}}l|
            p{0.15\textwidth}|}
        \hline
        \rowcolor{lightgray}\multicolumn{4}{|c|}{\textbf{KARTA WYMAGANIA POZAFUNKCJONALNEGO DLA CZATU}} \\ \hline
        \endfirsthead
        \hline
        \rowcolor{lightgray}\multicolumn{4}{|c|}{\textbf{KARTA WYMAGANIA POZAFUNKCJONALNEGO DLA CZATU (cd.)}} \\ \hline
        \endhead
        \wpczpriority{#3}{#4}
        \wpczname{#2}
        #5
        \end{longtable}
        \par
    }%
    \vspace{3pt}%
    \textbf{Tabela \thewpczat:} Wymaganie pozafunkcjonalne dla czatu: #2\label{#1}%
    \addcontentsline{lot}{table}{Tabela \thewpczat: Wymaganie pozafunkcjonalne dla czatu: #2}%
}

% =========================================================
% WPCZAT-01 – Użytkownik widzi tylko własne czaty i wiadomości
% Typ: bezpieczeństwo
% =========================================================

\wpczatcard
{wpczat:visibility-members}
{Ograniczenie widoczności czatów do członków}
{01}
{M}
{
    \wpcztype{Bezpieczeństwo}

    \wpczdesc{System zapewnia, że użytkownik widzi wyłącznie listę czatów
    oraz wiadomości z czatów, których jest członkiem. Informacje o innych
    czatach nie są prezentowane w interfejsie ani dostępne poprzez API.}

    \wpczaccept{%
        \begin{itemize}
            \item Lista czatów dla zalogowanego użytkownika zawiera wyłącznie czaty, w których jest on uczestnikiem.
            \item Próba otwarcia czatu, którego użytkownik nie jest członkiem, kończy się czytelnym błędem (np.\ \emph{brak uprawnień} lub \emph{nie znaleziono}).
            \item Wiadomości z czatów, do których użytkownik nie ma dostępu, nie są zwracane przez API ani widoczne w logach klienta.
        \end{itemize}
    }

    \wpczinput{Kontekst zalogowanego użytkownika, identyfikator użytkownika,
        identyfikator czatu (dla widoku szczegółowego).}

    \wpczpre{Użytkownik jest poprawnie uwierzytelniony; w systemie istnieją
    zarejestrowane czaty i przypisania użytkownik--czat.}

    \wpczpost{Użytkownik ma dostęp wyłącznie do własnych czatów i ich wiadomości,
        zgodnie z zapisanymi członkostwami.}

    \wpczexceptions{Błędna konfiguracja uprawnień, niespójne dane o członkostwach
    w bazie, awaria usługi autoryzacji.}

    \wpczimpl{Filtrowanie danych po stronie backendu na podstawie identyfikatora
    użytkownika; wymuszenie sprawdzania członkostwa przy każdym zapytaniu o czat
    lub wiadomości; testy integracyjne dla scenariuszy \emph{brak uprawnień}.}

    \wpczstakeholder{Użytkownik zalogowany, administrator systemu.}

    \wpczresponsible{Adam Langmesser}

    \wpczstatus{Zrealizowano}
}

% =========================================================
% WPCZAT-02 – Korzystanie z czatu wymaga zalogowania
% Typ: bezpieczeństwo
% =========================================================

\wpczatcard
{wpczat:login-required}
{Wymóg zalogowania do korzystania z czatu}
{02}
{M}
{
    \wpcztype{Bezpieczeństwo}

    \wpczdesc{Dostęp do funkcji czatu (lista czatów, wysyłanie i odbieranie
    wiadomości, tworzenie czatów) wymaga wcześniejszego zalogowania się do
    systemu. Użytkownik niezalogowany nie może przeglądać ani modyfikować
    danych czatu.}

    \wpczaccept{%
        \begin{itemize}
            \item Wejście na widok czatu przez użytkownika niezalogowanego powoduje przekierowanie na stronę logowania lub komunikat o braku uprawnień.
            \item Zapytania do API czatu bez ważnego tokena/autoryzacji są odrzucane.
            \item Po poprawnym zalogowaniu użytkownik uzyskuje pełny dostęp do swoich czatów bez konieczności ponownego logowania w tej sesji.
        \end{itemize}
    }

    \wpczinput{Dane logowania użytkownika, token sesji lub inny mechanizm
    uwierzytelniania.}

    \wpczpre{Użytkownik posiada aktywne konto w systemie.}

    \wpczpost{Tylko użytkownicy zalogowani mogą korzystać z funkcji czatu;
    użytkownicy niezalogowani widzą jedynie ekran logowania/rejestracji.}

    \wpczexceptions{Wygaśnięcie sesji, utrata tokena, błąd integracji z modułem
    logowania.}

    \wpczimpl{Middleware/autoryzacja na poziomie backendu wymuszająca
    uwierzytelnienie dla wszystkich endpointów czatu; przechowywanie danych
    sesji w bezpieczny sposób (np.\ token JWT).}

    \wpczstakeholder{Użytkownik zalogowany, administrator systemu.}

    \wpczresponsible{Adam Langmesser}

    \wpczstatus{Zrealizowano}
}

% =========================================================
% WPCZAT-03 – Grupowanie wiadomości według daty wysłania
% Typ: użyteczność
% =========================================================

\wpczatcard
{wpczat:group-by-date}
{Grupowanie wiadomości według daty wysłania}
{03}
{M}
{
    \wpcztype{Użyteczność}

    \wpczdesc{Wiadomości na czacie są prezentowane w logicznych grupach
    odpowiadających datom ich wysłania (np.\ separatory dzienne), co ułatwia
    użytkownikom orientację w historii rozmowy.}

    \wpczaccept{%
        \begin{itemize}
            \item W widoku czatu pojawiają się wizualne znaczniki dat (np.\ „Dzisiaj”, „Wczoraj”, konkretna data).
            \item Wiadomości są zawsze przypisane do poprawnej grupy daty wysłania niezależnie od strefy czasowej klienta.
            \item Zmiana zakresu historii (scrollowanie, przeładowanie) zachowuje poprawne grupowanie dat.
        \end{itemize}
    }

    \wpczinput{Wiadomości danego czatu wraz z ich znacznikami czasu wysłania.}

    \wpczpre{Wiadomości posiadają poprawnie zapisany czas wysłania
    w spójnym formacie.}

    \wpczpost{Użytkownik widzi wiadomości zgrupowane według dat,
        co poprawia czytelność dłuższych rozmów.}

    \wpczexceptions{Błędne strefy czasowe lub niepoprawne wartości znaczników
    czasu w bazie danych.}

    \wpczimpl{Normalizacja dat i czasów do strefy referencyjnej (np.\ UTC)
        po stronie backendu oraz odpowiednie formatowanie i grupowanie po stronie
        klienta.}

    \wpczstakeholder{Użytkownik zalogowany.}

    \wpczresponsible{Adam Langmesser}

    \wpczstatus{Zrealizowano}
}

% =========================================================
% WPCZAT-04 – Wyraźne oznaczenie nadawcy i czasu wysłania
% Typ: użyteczność
% =========================================================

\wpczatcard
{wpczat:sender-and-time}
{Wyraźne oznaczenie nadawcy i czasu wysłania}
{04}
{M}
{
    \wpcztype{Użyteczność}

    \wpczdesc{Każda wiadomość na czacie jest opatrzona wyraźną informacją,
        kto jest jej nadawcą oraz kiedy została wysłana (data i czas). Informacje
        te są łatwo zauważalne i spójne wizualnie.}

    \wpczaccept{%
        \begin{itemize}
            \item Przy każdej wiadomości widoczna jest nazwa lub alias nadawcy.
            \item Czas wysłania jest prezentowany w czytelnej formie (np.\ „12:34”, „wczoraj 21:10”) z uwzględnieniem lokalnej strefy czasowej użytkownika.
            \item Przejście kursorem lub tapnięcie umożliwia wyświetlenie pełnej daty i czasu w bardziej szczegółowym formacie.
        \end{itemize}
    }

    \wpczinput{Dane użytkownika--nadawcy oraz znacznik czasu wysłania
    każdej wiadomości.}

    \wpczpre{Dane użytkowników i wiadomości są spójne; każda wiadomość ma
    powiązanego nadawcę i poprawny czas wysłania.}

    \wpczpost{Użytkownik może jednoznacznie zidentyfikować nadawcę i czas
    wysłania każdej wiadomości.}

    \wpczexceptions{Brak danych o nadawcy lub czasie (stare wiadomości,
        dane testowe), błędy migracji danych.}

    \wpczimpl{Powiązanie wiadomości z tabelą użytkowników po kluczu obcym;
    formatowanie czasu po stronie klienta; spójny komponent UI dla nagłówka
    wiadomości.}

    \wpczstakeholder{Użytkownik zalogowany, moderatorzy czatu.}

    \wpczresponsible{Adam Langmesser}

    \wpczstatus{Zrealizowano}
}

% =========================================================
% WPCZAT-05 – Załadowanie starszych wiadomości < 3 s
% Typ: wydajność
% =========================================================

\wpczatcard
{wpczat:load-older-under-3s}
{Czas załadowania starszych wiadomości poniżej 3 sekund}
{05}
{M}
{
    \wpcztype{Wydajność}

    \wpczdesc{Podczas przewijania historii czatu załadowanie kolejnej partii
    starszych wiadomości powinno trwać krócej niż 3 sekundy w typowych
    warunkach sieciowych.}

    \wpczaccept{%
        \begin{itemize}
            \item W co najmniej 95\% pomiarów laboratoryjnych i testów akceptacyjnych czas pobrania starszych wiadomości mieści się w przedziale 0--3 s.
            \item Interfejs wyraźnie sygnalizuje trwające ładowanie (np.\ animacja „ładowanie”), a po zakończeniu przewijanie jest płynne.
            \item Brak zauważalnych „zawieszeń” interfejsu podczas dociągania danych.
        \end{itemize}
    }

    \wpczinput{Identyfikator czatu, parametry paginacji (np.\ kursor, znacznik czasu),
        liczba wiadomości do pobrania.}

    \wpczpre{W bazie istnieje historia rozmowy; serwer i baza danych działają
    poprawnie.}

    \wpczpost{Starsze wiadomości są dołączone do historii czatu użytkownika
    w czasie krótszym niż 3 s w typowych warunkach.}

    \wpczexceptions{Bardzo słabe łącze użytkownika, awaria sieci,
        wysoka chwilowa niedostępność bazy danych.}

    \wpczimpl{Indeksy po znaczniku czasu w bazie danych, paginacja kursorowa,
        cache po stronie serwera i klienta; ograniczenie liczby pobieranych
        rekordów w jednym żądaniu.}

    \wpczstakeholder{Użytkownik zalogowany.}

    \wpczresponsible{Adam Langmesser}

    \wpczstatus{Zrealizowano}
}

% =========================================================
% WPCZAT-06 – Wysłanie wiadomości następuje natychmiastowo
% Typ: wydajność
% =========================================================

\wpczatcard
{wpczat:send-immediately}
{Natychmiastowe wysyłanie wiadomości}
{06}
{M}
{
    \wpcztype{Wydajność}

    \wpczdesc{Po wysłaniu wiadomości przez użytkownika powinna ona pojawić się
    w widoku czatu w czasie subiektywnie natychmiastowym (rzędu setek
    milisekund), a pozostali uczestnicy powinni ją zobaczyć w czasie
    zbliżonym do rzeczywistego.}

    \wpczaccept{%
        \begin{itemize}
            \item W typowych warunkach sieciowych użytkownik widzi swoją nową wiadomość w czacie w czasie poniżej 1 s od wysłania.
            \item Pozostali uczestnicy czatu otrzymują wiadomość bez konieczności ręcznego odświeżania (np.\ przez WebSocket).
            \item W przypadku chwilowego opóźnienia interfejs sygnalizuje status wysyłania (np.\ ikona „wysyłanie…”, „niedostarczona”).
        \end{itemize}
    }

    \wpczinput{Identyfikator czatu, identyfikator nadawcy, treść wiadomości
    oraz ewentualne załączniki.}

    \wpczpre{Użytkownik jest zalogowany, posiada dostęp do czatu,
        serwer jest dostępny.}

    \wpczpost{Wiadomość jest zapisana w bazie danych oraz dostarczona
    wszystkim uprawnionym uczestnikom czatu.}

    \wpczexceptions{Zerwanie połączenia sieciowego, chwilowa niedostępność
    serwera, przeciążenie systemu.}

    \wpczimpl{Wykorzystanie połączeń WebSocket lub innego kanału push; kolejka
    wiadomości po stronie serwera; asynchroniczne zapisy do bazy danych.}

    \wpczstakeholder{Użytkownik zalogowany.}

    \wpczresponsible{Adam Langmesser}

    \wpczstatus{Zrealizowano}
}

% =========================================================
% WPCZAT-07 – Zachowanie wiadomości przy chwilowej utracie połączenia
% Typ: niezawodność
% =========================================================

\wpczatcard
{wpczat:keep-messages-on-disconnect}
{Zachowanie wiadomości przy chwilowej utracie połączenia}
{07}
{M}
{
    \wpcztype{Niezawodność}

    \wpczdesc{W przypadku krótkotrwałej utraty połączenia sieciowego
    wiadomości wysłane przez użytkownika nie powinny zostać utracone:
    zostaną ponownie wysłane po odzyskaniu łączności lub jednoznacznie
    oznaczone jako niedostarczone.}

    \wpczaccept{%
        \begin{itemize}
            \item Wiadomości wysłane w momencie utraty połączenia są buforowane lokalnie do czasu ponownej próby wysłania.
            \item Po odzyskaniu połączenia następuje automatyczna próba ponownego dostarczenia buforowanych wiadomości.
            \item W przypadku braku możliwości dostarczenia użytkownik otrzymuje czytelną informację (status „niedostarczona”) i może spróbować ponownie.
        \end{itemize}
    }

    \wpczinput{Bufor lokalny wiadomości po stronie klienta, status połączenia
    sieciowego, identyfikator czatu i nadawcy.}

    \wpczpre{Użytkownik jest zalogowany; przed utratą połączenia czat działał
    poprawnie.}

    \wpczpost{Wiadomości wysłane w okresie chwilowej utraty połączenia są
    ostatecznie dostarczone lub jasno oznaczone jako niedostarczone,
        bez „cichej” utraty danych.}

    \wpczexceptions{Długotrwały brak połączenia, ręczne zamknięcie aplikacji
    przed synchronizacją bufora, usunięcie danych lokalnych.}

    \wpczimpl{Bufor wysyłanych wiadomości w pamięci lub local storage
    po stronie klienta; mechanizm ponawiania wysyłki; idempotentne mutacje
    po stronie backendu.}

    \wpczstakeholder{Użytkownik zalogowany.}

    \wpczresponsible{Adam Langmesser}

    \wpczstatus{Zrealizowano}
}

% =========================================================
% WPCZAT-08 – Limit wysłanych wiadomości w jednostce czasu
% Typ: niezawodność
% =========================================================

\wpczatcard
{wpczat:rate-limit}
{Limit wysyłanych wiadomości w jednostce czasu}
{08}
{M}
{
    \wpcztype{Niezawodność}

    \wpczdesc{System ogranicza maksymalną liczbę wiadomości wysyłanych przez
    pojedynczego użytkownika w określonej jednostce czasu, aby chronić czat
    przed spamem oraz przeciążeniem.}

    \wpczaccept{%
        \begin{itemize}
            \item Przy zbyt częstym wysyłaniu wiadomości użytkownik otrzymuje informację o osiągnięciu limitu, a kolejne wiadomości są tymczasowo blokowane.
            \item Normalna praca czatu nie jest utrudniona dla typowych wzorców użycia (pisanie „normalnym tempem” nie wyczerpuje limitu).
            \item Limity mogą być konfigurowane (np.\ inne dla zwykłych użytkowników i administratorów).
        \end{itemize}
    }

    \wpczinput{Identyfikator użytkownika, znaczniki czasu wysłanych wiadomości
    w ostatnim przedziale czasowym.}

    \wpczpre{Użytkownik jest zalogowany; moduł czatu rejestruje czas
    wysłania każdej wiadomości.}

    \wpczpost{Nadmierne tempo wysyłania wiadomości przez jednego użytkownika
    jest ograniczane, co zmniejsza ryzyko przeciążenia systemu i spamu.}

    \wpczexceptions{Błędy konfiguracji limitów, ataki rozproszone z wielu kont,
        duże wahania ruchu w krótkim czasie.}

    \wpczimpl{Mechanizm \emph{rate limiting} na poziomie backendu (np.\ licznik
    w pamięci lub magazynie typu Redis) powiązany z identyfikatorem użytkownika
    lub adresu IP; zwracanie odpowiedniego kodu błędu przy przekroczeniu limitu.}

    \wpczstakeholder{Użytkownik zalogowany, administrator systemu.}

    \wpczresponsible{Adam Langmesser}

    \wpczstatus{Zrealizowano}
}
