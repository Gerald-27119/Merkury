%! Author = Adam
%! Date = 30/11/2025

\subsubsection{Wymagania pozafunkcjonalne dla czatu}
\label{subsubsec:wymagania-pozafunkcjonalne-dla-czatu}

% ============================
% WYMAGANIA POZAFUNKCJONALNE DLA CZATU
% Identyfikatory: WPCZAT-XX
% ============================

\newcounter{wpczat}[chapter]
\renewcommand{\thewpczat}{\thechapter.\arabic{wpczat}}

% Szerokość części z treścią (3 prawe kolumny zlane w jedną)
\newlength{\wpczContentWidth}
\setlength{\wpczContentWidth}{0.8\textwidth}

% --------- Pola karty (wiersze) ---------

% Id + priorytet – 4 kolumny
\newcommand{\wpczpriority}[2]{%
    \textbf{Identyfikator:} & WPCZAT-#1 &
    \textbf{Priorytet:}     & #2 \\ \hline
}

% Typ – normalny wiersz na 3 kolumny
\newcommand{\wpcztype}[1]{\textbf{Typ:} &
\multicolumn{3}{|p{\wpczContentWidth}|}{#1} \\ \hline}

\newcommand{\wpczname}[1]{\textbf{Nazwa:}              &
\multicolumn{3}{|p{\wpczContentWidth}|}{#1} \\ \hline}
\newcommand{\wpczdesc}[1]{\textbf{Opis:}               &
\multicolumn{3}{|p{\wpczContentWidth}|}{#1} \\ \hline}
\newcommand{\wpczaccept}[1]{\textbf{Kryteria akceptacji:} &
\multicolumn{3}{|p{\wpczContentWidth}|}{#1} \\ \hline}
\newcommand{\wpczstakeholder}[1]{\textbf{Udziałowiec:} &
\multicolumn{3}{|p{\wpczContentWidth}|}{#1} \\ \hline}
\newcommand{\wpczresponsible}[1]{\textbf{Realizator:}  &
\multicolumn{3}{|p{\wpczContentWidth}|}{#1} \\ \hline}
\newcommand{\wpczstatus}[1]{\textbf{Status:}           &
\multicolumn{3}{|p{\wpczContentWidth}|}{#1} \\ \hline}

\newcommand{\wpcznote}[1]{\textbf{Notatka:}            &
\multicolumn{3}{|p{\wpczContentWidth}|}{#1} \\ \hline}

\newcommand{\wpczrelated}[1]{\textbf{Wymagania powiązane:} &
\multicolumn{3}{|p{\wpczContentWidth}|}{#1} \\ \hline}

\newcommand{\wpczatcard}[5]{%
    \refstepcounter{wpczat}%
    {%
        \centering
        \begin{longtable}{|
                >{\columncolor{lightgray}}l|
            l|
                >{\columncolor{lightgray}}l|
            p{0.15\textwidth}|}
        \hline
        \rowcolor{lightgray}\multicolumn{4}{|c|}{\textbf{KARTA WYMAGANIA POZAFUNKCJONALNEGO DLA CZATU}} \\ \hline
        \endfirsthead
        \hline
        \rowcolor{lightgray}\multicolumn{4}{|c|}{\textbf{KARTA WYMAGANIA POZAFUNKCJONALNEGO DLA CZATU (cd.)}} \\ \hline
        \endhead
        \wpczpriority{#3}{#4}
        \wpczname{#2}
        #5
        \end{longtable}
        \par
    }%
    \vspace{3pt}%
    \textbf{Tabela \thewpczat:} Wymaganie pozafunkcjonalne dla czatu: #2\label{#1}%
    \addcontentsline{lot}{table}{Tabela \thewpczat: Wymaganie pozafunkcjonalne dla czatu: #2}%
}

% =========================================================
% WPCZAT-01 – Kontrola dostępu do konkretnych czatów (autoryzacja)
% Typ: bezpieczeństwo
% =========================================================

\wpczatcard
{wpczat:visibility-members}
{Dostęp do czatów ograniczony do uczestników (autoryzacja)}
{01}
{S}
{
    \wpcztype{Bezpieczeństwo}

    \wpczdesc{Wymaganie dotyczy \textbf{autoryzacji na poziomie pojedynczych czatów}.
    System zapewnia, że zalogowany użytkownik widzi wyłącznie listę czatów
    oraz wiadomości z czatów, których jest uczestnikiem. Informacje o innych
    czatach nie są prezentowane w interfejsie ani dostępne poprzez API,
        nawet jeśli użytkownik zna ich identyfikatory.}

    \wpczaccept{%
        \begin{itemize}
            \item Dla zalogowanego użytkownika lista czatów zawiera wyłącznie czaty, w których jest on uczestnikiem.
            \item Zapytania do API odwołujące się do czatu, którego użytkownik nie jest członkiem, są odrzucane (np.\ kodem 403) i nie zwracają żadnych danych o tym czacie ani jego wiadomościach.
        \end{itemize}
    }

    \wpczstakeholder{U3}

    \wpczresponsible{Adam Langmesser}

    \wpczstatus{Zrealizowano}

    \wpczrelated{Wymagania ogólne dla czatu:
    WOCZAT-01 (\ref{woczat:send-message}),
        WOCZAT-03 (\ref{woczat:browse-history}).}
}

% =========================================================
% WPCZAT-02 – Korzystanie z czatu wymaga zalogowania (uwierzytelnienie)
% Typ: bezpieczeństwo
% =========================================================

\wpczatcard
{wpczat:login-required}
{Korzystanie z czatu wymaga zalogowania (uwierzytelnienie)}
{02}
{S}
{
    \wpcztype{Bezpieczeństwo}

    \wpczdesc{Wymaganie dotyczy \textbf{uwierzytelnienia}. Jakakolwiek próba
    skorzystania z modułu czatu (wejście na widok czatu, pobieranie listy czatów,
        wysyłanie lub odbieranie wiadomości, tworzenie czatów) wymaga wcześniejszego
        zalogowania się do systemu. Użytkownik niezalogowany w ogóle nie może
        uzyskać dostępu do danych czatu.}

    \wpczaccept{%
        \begin{itemize}
            \item Wejście na widok czatu przez użytkownika niezalogowanego powoduje przekierowanie na ekran logowania lub wyświetlenie komunikatu o braku uprawnień.
            \item Zapytania do API czatu wykonane bez ważnego JWT są odrzucane kodem 401 i nie zwracają żadnych danych.
        \end{itemize}
    }

    \wpczstakeholder{U3}

    \wpczresponsible{Adam Langmesser}

    \wpczstatus{Zrealizowano}

    \wpczrelated{Wymagania ogólne dla czatu:
    WOCZAT-01 (\ref{woczat:send-message}),
        WOCZAT-02 (\ref{woczat:edit-chat}),
        WOCZAT-03 (\ref{woczat:browse-history}),
        WOCZAT-04 (\ref{woczat:create-chat}).}
}

% =========================================================
% WPCZAT-03 – Grupowanie wiadomości według daty wysłania
% Typ: użyteczność
% =========================================================

\wpczatcard
{wpczat:group-by-date}
{Grupowanie wiadomości według daty wysłania}
{03}
{S}
{
    \wpcztype{Użyteczność}

    \wpczdesc{Wiadomości na czacie są prezentowane w logicznych grupach
    odpowiadających datom ich wysłania, co ułatwia
    użytkownikom orientację w historii rozmowy.}

    \wpczaccept{%
        \begin{itemize}
            \item W widoku czatu pojawiają się wizualne znaczniki dat.
            \item Wiadomości są zawsze przypisane do poprawnej grupy daty wysłania niezależnie od strefy czasowej klienta.
            \item Zmiana zakresu historii (scrollowanie, przeładowanie) zachowuje poprawne grupowanie dat.
        \end{itemize}
    }

    \wpczstakeholder{U3}

    \wpczresponsible{Adam Langmesser}

    \wpczstatus{Zrealizowano}

    \wpczrelated{Wymagania ogólne dla czatu:
    WOCZAT-01 (\ref{woczat:send-message}),
        WOCZAT-03 (\ref{woczat:browse-history}).}
}

% =========================================================
% WPCZAT-04 – Wyraźne oznaczenie nadawcy i czasu wysłania
% Typ: użyteczność
% =========================================================

\wpczatcard
{wpczat:sender-and-time}
{Wyraźne oznaczenie nadawcy i czasu wysłania}
{04}
{S}
{
    \wpcztype{Użyteczność}

    \wpczdesc{Każda wiadomość na czacie jest opatrzona wyraźną informacją,
        kto jest jej nadawcą oraz kiedy została wysłan. Informacje
        te są łatwo zauważalne i spójne wizualnie.}

    \wpczaccept{%
        \begin{itemize}
            \item Przy każdej wiadomości widoczna jest nazwa lub alias nadawcy.
            \item Po najechaniu kursorem na daną wiadomość widoczna jest godzina jej wysłania.
        \end{itemize}
    }

    \wpczstakeholder{U3}

    \wpczresponsible{Adam Langmesser}

    \wpczstatus{Zrealizowano}

    \wpczrelated{Wymagania ogólne dla czatu:
    WOCZAT-01 (\ref{woczat:send-message}),
        WOCZAT-03 (\ref{woczat:browse-history}).}
}

% =========================================================
% WPCZAT-05 – Załadowanie starszych wiadomości < 10 s
% Typ: wydajność
% =========================================================

\wpczatcard
{wpczat:load-older-under-10s}
{Czas załadowania starszych wiadomości poniżej 10 sekund}
{05}
{S}
{
    \wpcztype{Wydajność}

    \wpczdesc{Podczas przewijania historii czatu załadowanie kolejnej partii
    starszych wiadomości powinno trwać krócej niż 10 sekundy w typowych
    warunkach sieciowych.}

    \wpczaccept{%
        \begin{itemize}
            \item W co najmniej 95\% pomiarów w warunkach deweloperskich czas pobrania starszych wiadomości mieści się w przedziale 0--10 s.
            \item Interfejs wyraźnie sygnalizuje trwające ładowanie.
            \item Brak zauważalnych „zawieszeń” interfejsu podczas operacji pobierania danych.
        \end{itemize}
    }

    \wpczstakeholder{U3}

    \wpczresponsible{Adam Langmesser}

    \wpczstatus{Zrealizowano}

    \wpczrelated{Wymagania ogólne dla czatu:
    WOCZAT-03 (\ref{woczat:browse-history}).}
}

% =========================================================
% WPCZAT-06 – Wysłanie wiadomości następuje natychmiastowo
% Typ: wydajność
% =========================================================

\wpczatcard
{wpczat:send-immediately}
{Natychmiastowe wysyłanie wiadomości}
{06}
{S}
{
    \wpcztype{Wydajność}

    \wpczdesc{Po wysłaniu wiadomości przez użytkownika powinna ona pojawić się
    w widoku czatu w czasie subiektywnie natychmiastowym (rzędu setek
    milisekund), a pozostali uczestnicy powinni ją zobaczyć w czasie
    zbliżonym do rzeczywistego.}

    \wpczaccept{%
        \begin{itemize}
            \item W typowych warunkach sieciowych użytkownik widzi swoją nową wiadomość w czasie poniżej 1 s od wysłania.
            \item Pozostali uczestnicy czatu otrzymują wiadomość bez konieczności ręcznego odświeżania.
        \end{itemize}
    }

    \wpczstakeholder{U3}

    \wpczresponsible{Adam Langmesser}

    \wpczstatus{Zrealizowano}

    \wpczrelated{Wymagania ogólne dla czatu:
    WOCZAT-01 (\ref{woczat:send-message}).}
}
