%! Author = Stanisław Oziemczuk
%! Date = 20.12.2025

\subsubsection{Wymagania pozafunkcjonalne dla mapy}
\label{subsubsec:wymagania-pozafunkcjonalne-dla-mapy}

% --- Szerokości ---
\newlength{\wpmapLabelWidth}
\settowidth{\wpmapLabelWidth}{\textbf{Wymagania powiązane:}}
\addtolength{\wpmapLabelWidth}{-20pt}

\newlength{\wpmapColTwoWidth}
\newlength{\wpmapColThreeWidth}
\newlength{\wpmapColFourWidth}
\setlength{\wpmapColTwoWidth}{0.31\textwidth}
\setlength{\wpmapColThreeWidth}{0.18\textwidth}
\setlength{\wpmapColFourWidth}{0.15\textwidth}

\newlength{\wpmapContentWidth}
\setlength{\wpmapContentWidth}{\dimexpr
\wpmapColTwoWidth+\wpmapColThreeWidth+\wpmapColFourWidth\relax}

\newlength{\wpmapHeaderHeight}
\setlength{\wpmapHeaderHeight}{12mm}
% --------- Pola karty (wiersze) ---------

\newcommand{\wpmapHeaderRow}[1]{%
    \rowcolor{lightgray}%
    \multicolumn{4}{|c|}{%
        \parbox[c][\wpmapHeaderHeight][c]{\linewidth}{%
            \centering\bfseries
            \vspace{1.2ex}%
            #1%
            \vspace{1.2ex}%
        }%
    }\\ \hline
}

% --------- Pola karty (wiersze) ---------

% Id + priorytet – 4 kolumny
\newcommand{\wpmappriority}[2]{%
    \textbf{Identyfikator:} & WPMAPA-#1 &
    \textbf{Priorytet:}     & #2 \\ \hline
}

% Typ – normalny wiersz na 3 kolumny
\newcommand{\wpmaptype}[1]{%
    \textbf{Typ:} &
    \multicolumn{3}{|>{\raggedright\arraybackslash}p{\wpmapContentWidth}|}{#1} \\ \hline
}

\newcommand{\wpmapname}[1]{%
    \textbf{Nazwa:} &
    \multicolumn{3}{|>{\raggedright\arraybackslash}p{\wpmapContentWidth}|}{#1} \\ \hline
}

\newcommand{\wpmapdesc}[1]{%
    \textbf{Opis:} &
    \multicolumn{3}{|>{\raggedright\arraybackslash}p{\wpmapContentWidth}|}{#1} \\ \hline
}

% (2) Kryteria akceptacji – ustawienia list jak w wymaganiach funkcjonalnych
\newcommand{\wpmapaccept}[1]{%
    \textbf{Kryteria akceptacji:} &
    \multicolumn{3}{|>{\raggedright\arraybackslash}p{\wpmapContentWidth}|}{%
        \begingroup
        \setlength{\leftmargini}{1.2em}%
        \setlength{\topsep}{0pt}%
        \setlength{\partopsep}{0pt}%
        \setlength{\itemsep}{0.2ex}%
        \setlength{\parsep}{0pt}%
        \vspace*{-1.8ex}%
        #1%
        \vspace*{-1.4ex}%
        \endgroup
    } \\ \hline
}

\newcommand{\wpmapstakeholder}[1]{%
    \textbf{Udziałowiec:} &
    \multicolumn{3}{|>{\raggedright\arraybackslash}p{\wpmapContentWidth}|}{#1} \\ \hline
}

\newcommand{\wpmapresponsible}[1]{%
    \textbf{Realizator:} &
    \multicolumn{3}{|>{\raggedright\arraybackslash}p{\wpmapContentWidth}|}{#1} \\ \hline
}

\newcommand{\wpmaprelated}[1]{%
    \textbf{Wymagania powiązane:} &
    \multicolumn{3}{|>{\raggedright\arraybackslash}p{\wpmapContentWidth}|}{#1} \\ \hline
}

\newcommand{\wpmapatcard}[5]{%
    \refstepcounter{awc}%
    {%
        \centering
        \begin{longtable}{|
                >{\columncolor{lightgray}\raggedright\arraybackslash}p{\wpmapLabelWidth}|
            p{\wpmapColTwoWidth}|
                >{\columncolor{lightgray}\raggedright\arraybackslash}p{\wpmapColThreeWidth}|
            p{\wpmapColFourWidth}|}
        \hline
        \wpmapHeaderRow{\textbf{KARTA WYMAGANIA POZAFUNKCJONALNEGO DLA \\ MAPY}}
        \endfirsthead
        \hline
        \wpmapHeaderRow{\textbf{KARTA WYMAGANIA POZAFUNKCJONALNEGO DLA \\ MAPY (cd.)}}
        \endhead
        \wpmappriority{#3}{#4}
        \wpmapname{#2}
        #5
        \end{longtable}
        \par
    }%
    \vspace{3pt}%
    \textbf{Tabela \theawc:} Wymaganie pozafunkcjonalne dla mapy: #2\label{#1}%
    \addcontentsline{lot}{table}{Tabela \theawc: Wymaganie pozafunkcjonalne dla mapy: #2}%
}

\wpmapatcard{wpmap:map-server}
{Hostowanie mapy na niezależnym serwerze}
{01}
{S}
{
    \wpmaptype{Niezawodność}
    \wpmapdesc{Mapa jest hostowana na serwerze zarządzanym przez zespół projektowy.
    Korzystanie z własnego hosta zamiast publicznej instancji dostarczanej przez producenta mapy pozwoli
    uniknąć jego nadmiernego obciążenia, powodującego zwiększenie czasu potrzebnego na pobranie kafelków
    mapy.}
    \wpmapaccept{
        \begin{itemize}
            \item Mapa jest hostowana na niezależnym serwerze.
            \item Serwer jest zarządzany przez zespół projektowy.
            \item Czas pobierania kafelków mapy z serwera jest nie większy niż z serwera dostarczanego przez
            producenta mapy.
        \end{itemize}
    }
    \wpmapstakeholder{U2}
    \wpmapresponsible{Stanisław Oziemczuk}
    \wpmaprelated{\hyperref[womap:display-spots]{WOMAPA-01}}
}

\wpmapatcard{wpmap:map-loading}
{Czas ładowania mapy poniżej 10 sekund}
{02}
{S}
{
    \wpmaptype{Wydajność}
    \wpmapdesc{Ładowanie kafelków mapy podczas pierwszego wejścia na stronę zajmuje maksymalnie 10 sekund przy
    standardowym połączeniu sieciowym.}
    \wpmapaccept{
        \begin{itemize}
            \item W przynajmniej 95\% przypadków załadowanie kafelek mapy w
            widocznym jej obszarze trwa mniej niż 10 sekund.
        \end{itemize}
    }
    \wpmapstakeholder{U3}
    \wpmapresponsible{Stanisław Oziemczuk}
    \wpmaprelated{\hyperref[womap:display-spots]{WOMAPA-01}}
}

\wpmapatcard{wpmap:spot-loading}
{Czas pobierania \glslink{spot}{spotów} poniżej 10 sekund}
{03}
{S}
{
    \wpmaptype{Wydajność}
    \wpmapdesc{Pobieranie \glslink{spot}{spotów} z \glslink{backend}{backendu}
    i wyświetlenie ich na mapie trwa maksymalnie 10 sekund przy
    standardowym połączeniu sieciowym.}
    \wpmapaccept{
        \begin{itemize}
            \item W przynajmniej 95\% przypadków pobranie i wyświetlenie \glslink{spot}{spotów} zajmuje mniej niż
            10 sekund.
        \end{itemize}
    }
    \wpmapstakeholder{U3}
    \wpmapresponsible{Stanisław Oziemczuk}
    \wpmaprelated{\hyperref[womap:display-spots]{WOMAPA-01}}
}
