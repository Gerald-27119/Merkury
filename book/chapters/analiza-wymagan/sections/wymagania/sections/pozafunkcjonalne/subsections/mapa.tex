%! Author = Stanisław Oziemczuk
%! Date = 20.12.2025

\subsubsection{Wymagania pozafunkcjonalne dla mapy}
\label{subsubsec:wymagania-pozafunkcjonalne-dla-mapy}

\newlength{\wpczLabelWidth}
\settowidth{\wpczLabelWidth}{\textbf{Wymagania powiązane:}}
\addtolength{\wpczLabelWidth}{-20pt}

% Szerokość części z treścią (3 prawe kolumny zlane w jedną)
\newlength{\wpczContentWidth}
\setlength{\wpczContentWidth}{0.7\textwidth}
\addtolength{\wpczContentWidth}{20pt}

% --------- Pola karty (wiersze) ---------

% Id + priorytet – 4 kolumny
\newcommand{\wpmappriority}[2]{%
    \textbf{Identyfikator:} & WPMAPA-#1 &
    \textbf{Priorytet:}     & #2 \\ \hline
}

% Typ – normalny wiersz na 3 kolumny
\newcommand{\wpmaptype}[1]{%
    \textbf{Typ:} &
    \multicolumn{3}{|>{\raggedright\arraybackslash}p{\wpmapContentWidth}|}{#1} \\ \hline
}

\newcommand{\wpmapname}[1]{%
    \textbf{Nazwa:} &
    \multicolumn{3}{|>{\raggedright\arraybackslash}p{\wpmapContentWidth}|}{#1} \\ \hline
}

\newcommand{\wpmapdesc}[1]{%
    \textbf{Opis:} &
    \multicolumn{3}{|>{\raggedright\arraybackslash}p{\wpmapContentWidth}|}{#1} \\ \hline
}

% (2) Kryteria akceptacji – ustawienia list jak w wymaganiach funkcjonalnych
\newcommand{\wpmapaccept}[1]{%
    \textbf{Kryteria akceptacji:} &
    \multicolumn{3}{|>{\raggedright\arraybackslash}p{\wpmapContentWidth}|}{%
        \begingroup
        \setlength{\leftmargini}{1.2em}%
        \setlength{\topsep}{0pt}%
        \setlength{\partopsep}{0pt}%
        \setlength{\itemsep}{0.2ex}%
        \setlength{\parsep}{0pt}%
        \vspace*{-1.8ex}%
        #1%
        \vspace*{-1.4ex}%
        \endgroup
    } \\ \hline
}

\newcommand{\wpmapstakeholder}[1]{%
    \textbf{Udziałowiec:} &
    \multicolumn{3}{|>{\raggedright\arraybackslash}p{\wpmapContentWidth}|}{#1} \\ \hline
}

\newcommand{\wpmapresponsible}[1]{%
    \textbf{Realizator:} &
    \multicolumn{3}{|>{\raggedright\arraybackslash}p{\wpmapContentWidth}|}{#1} \\ \hline
}

\newcommand{\wpmaprelated}[1]{%
    \textbf{Wymagania powiązane:} &
    \multicolumn{3}{|>{\raggedright\arraybackslash}p{\wpmapContentWidth}|}{#1} \\ \hline
}

\newcommand{\wpmapatcard}[5]{%
    \refstepcounter{awc}%
    {%
        \centering
        \begin{longtable}{|
                >{\columncolor{lightgray}\raggedright\arraybackslash}p{\wpmapLabelWidth}|
            l|
                >{\columncolor{lightgray}\raggedright\arraybackslash}l|
            p{0.15\textwidth}|}
        \hline
        \rowcolor{lightgray}\multicolumn{4}{|c|}{\textbf{KARTA WYMAGANIA POZAFUNKCJONALNEGO DLA MAPY}} \\ \hline
        \endfirsthead
        \hline
        \rowcolor{lightgray}\multicolumn{4}{|c|}{\textbf{KARTA WYMAGANIA POZAFUNKCJONALNEGO DLA MAPY (cd.)}} \\ \hline
        \endhead
        \wpmappriority{#3}{#4}
        \wpmapname{#2}
        #5
        \end{longtable}
        \par
    }%
    \vspace{3pt}%
    \textbf{Tabela \theawc:} Wymaganie pozafunkcjonalne dla mapy: #2\label{#1}%
    \addcontentsline{lot}{table}{Tabela \theawc: Wymaganie pozafunkcjonalne dla mapy: #2}%
}