%! Author = Mateusz
%! Date = 22/12/2025

\subsubsection{Wymagania pozafunkcjonalne dla logowania i rejestracji}
\label{subsubsec:wymagania-pozafunkcjonalne-dla-wyszukiwarki-spotow}


% --- Szerokości ---

\newlength{\wplogLabelWidth}
\settowidth{\wplogLabelWidth}{\textbf{Wymagania powiązane:}}
\addtolength{\wplogLabelWidth}{-20pt}

% Szerokość części z treścią (3 prawe kolumny zlane w jedną)
\newlength{\wplogContentWidth}
\setlength{\wplogContentWidth}{0.7\textwidth}
\addtolength{\wplogContentWidth}{20pt}

\newlength{\wplogHeaderHeight}
\setlength{\wplogHeaderHeight}{12mm}
% --------- Pola karty (wiersze) ---------

\newcommand{\wplogHeaderRow}[1]{%
    \rowcolor{lightgray}%
    \multicolumn{4}{|c|}{%
        \parbox[c][\wplogHeaderHeight][c]{\linewidth}{%
            \centering\bfseries
            \vspace{1.2ex}%
            #1%
            \vspace{1.2ex}%
        }%
    }\\ \hline
}

% Id + priorytet – 4 kolumny
\newcommand{\wplogpriority}[2]{%
    \textbf{Identyfikator:} & WPLOG-#1 &
    \textbf{Priorytet:}     & #2 \\ \hline
}

% Typ – normalny wiersz na 3 kolumny
\newcommand{\wplogtype}[1]{%
    \textbf{Typ:} &
    \multicolumn{3}{|>{\raggedright\arraybackslash}p{\wplogContentWidth}|}{#1} \\ \hline
}

\newcommand{\wplogname}[1]{%
    \textbf{Nazwa:} &
    \multicolumn{3}{|>{\raggedright\arraybackslash}p{\wplogContentWidth}|}{#1} \\ \hline
}

\newcommand{\wplogdesc}[1]{%
    \textbf{Opis:} &
    \multicolumn{3}{|>{\raggedright\arraybackslash}p{\wplogContentWidth}|}{#1} \\ \hline
}

% (2) Kryteria akceptacji – ustawienia list jak w wymaganiach funkcjonalnych
\newcommand{\wplogaccept}[1]{%
    \textbf{Kryteria akceptacji:} &
    \multicolumn{3}{|>{\raggedright\arraybackslash}p{\wplogContentWidth}|}{%
        \begingroup
        \setlength{\leftmargini}{1.2em}%
        \setlength{\topsep}{0pt}%
        \setlength{\partopsep}{0pt}%
        \setlength{\itemsep}{0.2ex}%
        \setlength{\parsep}{0pt}%
        \vspace*{-1.8ex}%
        #1%
        \vspace*{-1.4ex}%
        \endgroup
    } \\ \hline
}

\newcommand{\wplogstakeholder}[1]{%
    \textbf{Udziałowiec:} &
    \multicolumn{3}{|>{\raggedright\arraybackslash}p{\wplogContentWidth}|}{#1} \\ \hline
}

\newcommand{\wplogresponsible}[1]{%
    \textbf{Realizator:} &
    \multicolumn{3}{|>{\raggedright\arraybackslash}p{\wplogContentWidth}|}{#1} \\ \hline
}

\newcommand{\wplogrelated}[1]{%
    \textbf{Wymagania powiązane:} &
    \multicolumn{3}{|>{\raggedright\arraybackslash}p{\wplogContentWidth}|}{#1} \\ \hline
}

\newcommand{\wplogatcard}[5]{%
    \refstepcounter{awc}%
    {%
        \centering
        \begin{longtable}{|
                >{\columncolor{lightgray}\raggedright\arraybackslash}p{\wplogLabelWidth}|
            l|
                >{\columncolor{lightgray}\raggedright\arraybackslash}l|
            p{0.15\textwidth}|}
        \hline
        \wplogHeaderRow{\shortstack{KARTA WYMAGANIA POZAFUNKCJONALNEGO DLA \\ LOGOWANIA I REJESTRACJI}}
        \endfirsthead
        \hline
        \wplogHeaderRow{\shortstack{KARTA WYMAGANIA POZAFUNKCJONALNEGO DLA \\ LOGOWANIA I REJESTRACJI (cd.)}}
        \endhead
        \wplogpriority{#3}{#4}
        \wplogname{#2}
        #5
        \end{longtable}
        \par
    }%
    \vspace{3pt}%
    \textbf{Tabela \theawc:} Wymaganie pozafunkcjonalne dla logowania i rejestracji: #2\label{#1}%
    \addcontentsline{lot}{table}{Tabela \theawc: Wymaganie pozafunkcjonalne dla logowania i rejestracji: #2}%
}

% =========================================================
% WPLOG-01 – Wydajność: czas odpowiedzi logowania/rejestracji < 10 s
% Typ: wydajność
% =========================================================

\wplogatcard
{wplog:auth-response-under-3s}
{Czas odpowiedzi logowania i rejestracji nie przekracza 10 sekund}
{01}
{S}
{
    \wplogtype{Wydajność}

    \wplogdesc{Operacje logowania oraz rejestracji powinny być realizowane w czasie nie dłuższym niż 10 sekundy
    w typowych warunkach sieciowych, aby użytkownik nie odczuwał opóźnień podczas uwierzytelniania.}

    \wplogaccept{%
        \begin{itemize}
            \item W co najmniej 95\% pomiarów czas odpowiedzi \gls{api} dla logowania i rejestracji mieści się w przedziale 0--10 s.
            \item Podczas oczekiwania system prezentuje stan ładowania.
            \item Po zakończeniu operacji użytkownik otrzymuje jednoznaczną informację o sukcesie lub błędzie.
        \end{itemize}
    }

    \wplogstakeholder{U3}
    \wplogresponsible{Mateusz Redosz, Kacper Badek, Stanisław Oziemczuk}
    \wplogrelated{%
        \hyperref[wolog:login]{WOLOG-01},
        \hyperref[wolog:register]{WOLOG-02},
        \hyperref[wolog:password-reset]{WOLOG-03},
        \hyperref[wflog:login-email-password]{WFLOG-01},
        \hyperref[wflog:login-sso]{WFLOG-02},
        \hyperref[wflog:register-form]{WFLOG-03},
        \hyperref[wflog:register-sso]{WFLOG-04},
        \hyperref[wflog:password-reset-email]{WFLOG-05},
        \hyperref[wflog:password-reset-confirm]{WFLOG-06}.%
    }
}

% =========================================================
% WPLOG-02 – Bezpieczeństwo: ochrona transmisji i danych sesyjnych
% Typ: bezpieczeństwo
% =========================================================

\wplogatcard
{wplog:secure-session-cookie}
{Bezpieczne utrzymanie sesji użytkownika}
{02}
{S}
{
    \wplogtype{Bezpieczeństwo}

    \wplogdesc{Mechanizm uwierzytelniania powinien chronić dane sesyjne użytkownika przed przejęciem,
        w szczególności poprzez bezpieczne przechowywanie i przesyłanie danych sesyjnych.}

    \wplogaccept{%
        \begin{itemize}
            \item Identyfikacja sesji nie jest dostępna dla kodu JavaScript w przeglądarce (\glslink{http-only-cookie}{httpOnly cookie}).
            \item Po wylogowaniu sesja zostaje unieważniona, a użytkownik traci dostęp do zasobów chronionych.
        \end{itemize}
    }

    \wplogstakeholder{U3}
    \wplogresponsible{Mateusz Redosz}
    \wplogrelated{%
        \hyperref[wolog:login]{WOLOG-01},
        \hyperref[wolog:register]{WOLOG-02},
        \hyperref[wflog:login-email-password]{WFLOG-01},
        \hyperref[wflog:login-sso]{WFLOG-02},
        \hyperref[wflog:register-sso]{WFLOG-04},
        \hyperref[wflog:logout]{WFLOG-07},
        \hyperref[wflog:auth-error-handling]{WFLOG-09}.%
    }
}

% =========================================================
% WPLOG-03 – Bezpieczeństwo: komunikaty błędów nie ujawniają informacji
% Typ: bezpieczeństwo
% =========================================================

\wplogatcard
{wplog:no-sensitive-error-details}
{Komunikaty błędów uwierzytelniania nie ujawniają wrażliwych informacji}
{03}
{S}
{
    \wplogtype{Bezpieczeństwo}

    \wplogdesc{System nie powinien ujawniać w komunikatach błędów informacji, które mogłyby ułatwić ataki.}

    \wplogaccept{%
        \begin{itemize}
            \item W przypadku błędnych danych logowania system zwraca ogólny komunikat (bez wskazania, czy błąd dotyczy e-maila czy hasła).
            \item W przypadku błędu serwera użytkownik widzi komunikat ogólny (bez stack trace i szczegółów technicznych).
            \item Komunikaty są czytelne i spójne w całym module logowania/rejestracji.
        \end{itemize}
    }

    \wplogstakeholder{U3}
    \wplogresponsible{Mateusz Redosz, Kacper Badek, Stanisław Oziemczuk}
    \wplogrelated{%
        \hyperref[wolog:login]{WOLOG-01},
        \hyperref[wolog:register]{WOLOG-02},
        \hyperref[wolog:password-reset]{WOLOG-03},
        \hyperref[wflog:login-email-password]{WFLOG-01},
        \hyperref[wflog:login-sso]{WFLOG-02},
        \hyperref[wflog:register-form]{WFLOG-03},
        \hyperref[wflog:register-sso]{WFLOG-04},
        \hyperref[wflog:password-reset-email]{WFLOG-05},
        \hyperref[wflog:password-reset-confirm]{WFLOG-06},
        \hyperref[wflog:auth-error-handling]{WFLOG-09}.%
    }
}

% =========================================================
% WPLOG-04 – Użyteczność: czytelna walidacja formularzy
% Typ: użyteczność
% =========================================================

\wplogatcard
{wplog:form-validation-ux}
{Czytelna walidacja formularzy logowania i rejestracji}
{04}
{S}
{
    \wplogtype{Użyteczność}

    \wplogdesc{Formularze logowania, rejestracji i resetu hasła powinny zapewniać czytelną walidację
    danych wejściowych oraz jednoznaczne komunikaty o błędach, aby ograniczyć liczbę nieudanych prób.}

    \wplogaccept{%
        \begin{itemize}
            \item System blokuje wysłanie formularza z brakującymi lub niepoprawnymi danymi wymaganymi.
            \item Komunikaty walidacyjne są wyświetlane przy odpowiednich polach.
            \item Komunikaty są zrozumiałe i wskazują, jak poprawić dane.
        \end{itemize}
    }

    \wplogstakeholder{U3}
    \wplogresponsible{Mateusz Redosz, Kacper Badek, Stanisław Oziemczuk}
    \wplogrelated{%
        \hyperref[wolog:login]{WOLOG-01},
        \hyperref[wolog:register]{WOLOG-02},
        \hyperref[wolog:password-reset]{WOLOG-03},
        \hyperref[wflog:login-email-password]{WFLOG-01},
        \hyperref[wflog:register-form]{WFLOG-03},
        \hyperref[wflog:password-reset-email]{WFLOG-05},
        \hyperref[wflog:password-reset-confirm]{WFLOG-06},
        \hyperref[wflog:forms-validation]{WFLOG-08},
        \hyperref[wflog:switch-auth-views]{WFLOG-10}.%
    }
}

% =========================================================
% WPLOG-05 – Niezawodność: odporność na błędy sieciowe
% Typ: niezawodność
% =========================================================

\wplogatcard
{wplog:network-failure-handling}
{Odporność na błędy połączenia w procesie logowania}
{05}
{S}
{
    \wplogtype{Niezawodność}

    \wplogdesc{W przypadku problemów sieciowych system powinien zachować przewidywalne działanie,
        informując użytkownika o niepowodzeniu operacji i umożliwiając ponowienie próby.}

    \wplogaccept{%
        \begin{itemize}
            \item Przy braku połączenia lub timeout użytkownik otrzymuje czytelny komunikat o problemie.
            \item Formularz pozostaje w stanie umożliwiającym ponowienie próby (bez utraty wpisanych danych, o ile to możliwe).
            \item System nie przechodzi do stanu „zalogowany” bez potwierdzenia sukcesu z \gls{api}.
        \end{itemize}
    }

    \wplogstakeholder{U3}
    \wplogresponsible{Kacper Badek, Stanisław Oziemczuk}
    \wplogrelated{%
        \hyperref[wolog:login]{WOLOG-01},
        \hyperref[wolog:register]{WOLOG-02},
        \hyperref[wolog:password-reset]{WOLOG-03},
        \hyperref[wflog:login-email-password]{WFLOG-01},
        \hyperref[wflog:login-sso]{WFLOG-02},
        \hyperref[wflog:register-form]{WFLOG-03},
        \hyperref[wflog:register-sso]{WFLOG-04},
        \hyperref[wflog:password-reset-email]{WFLOG-05},
        \hyperref[wflog:password-reset-confirm]{WFLOG-06},
        \hyperref[wflog:auth-error-handling]{WFLOG-09}.%
    }
}

% =========================================================
% WPLOG-06 – Dostępność: obsługa klawiatury i czytelność elementów
% Typ: użyteczność
% =========================================================

\wplogatcard
{wplog:accessibility-keyboard}
{Dostępność formularzy logowania i rejestracji}
{06}
{S}
{
    \wplogtype{Użyteczność}

    \wplogdesc{Formularze powinny być dostępne dla użytkowników korzystających z klawiatury oraz zapewniać
    czytelny układ i fokus elementów, aby proces logowania był możliwy dla szerokiego grona odbiorców.}

    \wplogaccept{%
        \begin{itemize}
            \item Wszystkie pola i przyciski są osiągalne klawiszem Tab w logicznej kolejności.
            \item Fokus elementów jest widoczny podczas nawigacji klawiaturą.
            \item Etykiety pól są jednoznaczne (e-mail, hasło, nazwa użytkownika).
        \end{itemize}
    }

    \wplogstakeholder{U3}
    \wplogresponsible{Mateusz Redosz, Kacper Badek, Stanisław Oziemczuk}
    \wplogrelated{%
        \hyperref[wolog:login]{WOLOG-01},
        \hyperref[wolog:register]{WOLOG-02},
        \hyperref[wolog:password-reset]{WOLOG-03},
        \hyperref[wflog:login-email-password]{WFLOG-01},
        \hyperref[wflog:register-form]{WFLOG-03},
        \hyperref[wflog:password-reset-email]{WFLOG-05},
        \hyperref[wflog:password-reset-confirm]{WFLOG-06},
        \hyperref[wflog:switch-auth-views]{WFLOG-10}.%
    }
}
