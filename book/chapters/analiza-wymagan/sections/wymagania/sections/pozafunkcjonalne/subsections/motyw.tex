%! Author = Mateusz
%! Date = 22/12/2025


\subsubsection{Wymagania pozafunkcjonalne dla motywu}
\label{subsubsec:wymagania-pozafunkcjonalne-dla-motywu}


% --- Szerokości ---

% --- Szerokości ---
\newlength{\wpmotywLabelWidth}
\settowidth{\wpmotywLabelWidth}{\textbf{Wymagania powiązane:}}
\addtolength{\wpmotywLabelWidth}{-20pt}

\newlength{\wpmotywColTwoWidth}
\newlength{\wpmotywColThreeWidth}
\newlength{\wpmotywColFourWidth}
\setlength{\wpmotywColTwoWidth}{0.31\textwidth}
\setlength{\wpmotywColThreeWidth}{0.18\textwidth}
\setlength{\wpmotywColFourWidth}{0.15\textwidth}

\newlength{\wpmotywContentWidth}
\setlength{\wpmotywContentWidth}{\dimexpr
\wpmotywColTwoWidth+\wpmotywColThreeWidth+\wpmotywColFourWidth\relax}

\newlength{\wpmotywHeaderHeight}
\setlength{\wpmotywHeaderHeight}{12mm}
% --------- Pola karty (wiersze) ---------

\newcommand{\wpmotywHeaderRow}[1]{%
    \rowcolor{lightgray}%
    \multicolumn{4}{|c|}{%
        \parbox[c][\wpmotywHeaderHeight][c]{\linewidth}{%
            \centering\bfseries
            \vspace{1.2ex}%
            #1%
            \vspace{1.2ex}%
        }%
    }\\ \hline
}

% Id + priorytet – 4 kolumny
\newcommand{\wpmotywpriority}[2]{%
    \textbf{Identyfikator:} & WPMOTYW-#1 &
    \textbf{Priorytet:}     & #2 \\ \hline
}

% Typ – normalny wiersz na 3 kolumny
\newcommand{\wpmotywtype}[1]{%
    \textbf{Typ:} &
    \multicolumn{3}{|>{\raggedright\arraybackslash}p{\wpmotywContentWidth}|}{#1} \\ \hline
}

\newcommand{\wpmotywname}[1]{%
    \textbf{Nazwa:} &
    \multicolumn{3}{|>{\raggedright\arraybackslash}p{\wpmotywContentWidth}|}{#1} \\ \hline
}

\newcommand{\wpmotywdesc}[1]{%
    \textbf{Opis:} &
    \multicolumn{3}{|>{\raggedright\arraybackslash}p{\wpmotywContentWidth}|}{#1} \\ \hline
}

% (2) Kryteria akceptacji – ustawienia list jak w wymaganiach funkcjonalnych
\newcommand{\wpmotywaccept}[1]{%
    \textbf{Kryteria akceptacji:} &
    \multicolumn{3}{|>{\raggedright\arraybackslash}p{\wpmotywContentWidth}|}{%
        \begingroup
        \setlength{\leftmargini}{1.2em}%
        \setlength{\topsep}{0pt}%
        \setlength{\partopsep}{0pt}%
        \setlength{\itemsep}{0.2ex}%
        \setlength{\parsep}{0pt}%
        \vspace*{-1.8ex}%
        #1%
        \vspace*{-1.4ex}%
        \endgroup
    } \\ \hline
}

\newcommand{\wpmotywstakeholder}[1]{%
    \textbf{Udziałowiec:} &
    \multicolumn{3}{|>{\raggedright\arraybackslash}p{\wpmotywContentWidth}|}{#1} \\ \hline
}

\newcommand{\wpmotywresponsible}[1]{%
    \textbf{Realizator:} &
    \multicolumn{3}{|>{\raggedright\arraybackslash}p{\wpmotywContentWidth}|}{#1} \\ \hline
}

\newcommand{\wpmotywrelated}[1]{%
    \textbf{Wymagania powiązane:} &
    \multicolumn{3}{|>{\raggedright\arraybackslash}p{\wpmotywContentWidth}|}{#1} \\ \hline
}

\newcommand{\wpmotywatcard}[5]{%
    \refstepcounter{awc}%
    {%
        \centering
        \begin{longtable}{|
                >{\columncolor{lightgray}\raggedright\arraybackslash}p{\wpmotywLabelWidth}|
            p{\wpmotywColTwoWidth}|
                >{\columncolor{lightgray}\raggedright\arraybackslash}p{\wpmotywColThreeWidth}|
            p{\wpmotywColFourWidth}|}
        \hline
        \wpmotywHeaderRow{\shortstack{KARTA WYMAGANIA POZAFUNKCJONALNEGO DLA \\ MOTYWU}}
        \endfirsthead
        \hline
        \wpmotywHeaderRow{\shortstack{KARTA WYMAGANIA POZAFUNKCJONALNEGO DLA \\ MOTYWU (cd.)}}
        \endhead
        \wpmotywpriority{#3}{#4}
        \wpmotywname{#2}
        #5
        \end{longtable}
        \par
    }%
    \vspace{3pt}%
    \textbf{Tabela \theawc:} Wymaganie pozafunkcjonalne dla motywu: #2\label{#1}%
    \addcontentsline{lot}{table}{Tabela \theawc: Wymaganie pozafunkcjonalne dla motywu: #2}%
}

% =========================================================
% WPMOTYW-01 – Wydajność: zmiana motywu bez opóźnień
% Typ: wydajność
% =========================================================

\wpmotywatcard
{wpmotyw:theme-switch-under-200ms}
{Zmiana motywu jest natychmiastowa}
{01}
{S}
{
    \wpmotywtype{Wydajność}

    \wpmotywdesc{Przełączenie motywu (jasny/ciemny) powinno następować w czasie subiektywnie natychmiastowym,
        bez zauważalnych opóźnień oraz bez przeładowania strony, aby nie pogarszać komfortu korzystania z aplikacji.}

    \wpmotywaccept{%
        \begin{itemize}
            \item Zmiana motywu następuje bez przeładowania strony.
            \item W co najmniej 95\% prób przełączenie motywu jest widoczne w czasie krótszym niż 200 ms.
            \item Podczas przełączania nie występują zauważalne „zawieszenia” interfejsu.
        \end{itemize}
    }

    \wpmotywstakeholder{U3}
    \wpmotywresponsible{Mateusz Redosz}
    \wpmotywrelated{%
        \hyperref[womotyw:change-theme]{WOMOTYW-01},
        \hyperref[wfmotyw:change-theme]{WFMOTYW-01}.%
    }
}

% =========================================================
% WPMOTYW-02 – Użyteczność: spójność motywu w całej aplikacji
% Typ: użyteczność
% =========================================================

\wpmotywatcard
{wpmotyw:theme-consistency}
{Spójność motywu we wszystkich widokach}
{02}
{M}
{
    \wpmotywtype{Użyteczność}

    \wpmotywdesc{Wybrany motyw powinien być spójnie stosowany we wszystkich widokach aplikacji,
        aby użytkownik nie doświadczał niespójnych kolorów, ikon lub elementów interfejsu.}

    \wpmotywaccept{%
        \begin{itemize}
            \item Po zmianie motywu wszystkie główne widoki aplikacji są prezentowane zgodnie z wybranym motywem.
            \item Nie występują elementy z nieczytelnym kontrastem (np. ciemny tekst na ciemnym tle).
        \end{itemize}
    }

    \wpmotywstakeholder{U3}
    \wpmotywresponsible{Mateusz Redosz}
    \wpmotywrelated{%
        \hyperref[womotyw:change-theme]{WOMOTYW-01},
        \hyperref[wfmotyw:change-theme]{WFMOTYW-01}.%
    }
}

% =========================================================
% WPMOTYW-03 – Użyteczność: brak „mignięcia” stylów przy starcie
% Typ: użyteczność
% =========================================================

\wpmotywatcard
{wpmotyw:no-flash-on-load}
{Brak mignięcia motywu przy uruchomieniu}
{03}
{M}
{
    \wpmotywtype{Użyteczność}

    \wpmotywdesc{Przy wczytywaniu aplikacji motyw powinien zostać ustawiony przed wyrenderowaniem interfejsu,
        aby uniknąć „mignięcia” motywu domyślnego i nagłej zmiany wyglądu.}

    \wpmotywaccept{%
        \begin{itemize}
            \item Po odświeżeniu strony użytkownik nie obserwuje krótkiego wyświetlenia przeciwnego motywu.
            \item Ustawienie motywu następuje przed renderem głównych komponentów aplikacji.
            \item Preferencja motywu jest stosowana już od pierwszej klatki renderowania \gls{ui}.
        \end{itemize}
    }

    \wpmotywstakeholder{U3}
    \wpmotywresponsible{Mateusz Redosz}
    \wpmotywrelated{%
        \hyperref[womotyw:change-theme]{WOMOTYW-01},
        \hyperref[wfmotyw:remember-theme]{WFMOTYW-02}.%
    }
}
