%! Author = Stanisław Oziemczuk
%! Date = 19.12.2025

\subsubsection{Wymagania ogólne dla mapy}
\label{subsubsec:wymagania-ogolne-dla-mapy}


\newlength{\womapContentWidth}
\setlength{\womapContentWidth}{0.74\textwidth}

\newlength{\womapLabelWidth}
\setlength{\womapLabelWidth}{0.2\textwidth}

\newcommand{\womappriority}[2]{%
    \textbf{Identyfikator:} & WOMAP-#1 &
    \textbf{Priorytet:} & #2 \\ \hline
}

\newcommand{\womapname}[1]{\textbf{Nazwa:}              &
\multicolumn{3}{|p{\womapContentWidth}|}{#1} \\ \hline}
\newcommand{\womapdesc}[1]{\textbf{Opis:}               &
\multicolumn{3}{|p{\womapContentWidth}|}{#1} \\ \hline}
\newcommand{\womapstakeholder}[1]{\textbf{Udziałowiec:} &
\multicolumn{3}{|p{\womapContentWidth}|}{#1} \\ \hline}
\newcommand{\womaprelated}[1]{\textbf{Wymagania powiązane:} &
\multicolumn{3}{|p{\womapContentWidth}|}{#1} \\ \hline}

\newcommand{\womapcard}[5]{%
    \refstepcounter{awc}%
    {\centering
    \begin{longtable}{|
            >{\columncolor{lightgray}}p{\womapLabelWidth}|
        p{0.22\textwidth}|
            >{\columncolor{lightgray}}p{0.18\textwidth}|
        p{0.24\textwidth}|}

    \hline
    \rowcolor{lightgray}\multicolumn{4}{|c|}
    {\textbf{KARTA WYMAGANIA OGÓLNEGO DLA MAPY}} \\ \hline
    \endfirsthead
    \hline
    \rowcolor{lightgray}\multicolumn{4}{|c|}
    {\textbf{KARTA WYMAGANIA OGÓLNEGO DLA MAPY (cd.)}} \\ \hline
    \endhead

    % właściwa treść karty
    \womappriority{#3}{#4}
    \womapname{#2}
    #5

    \end{longtable}
    \par} % koniec centrowania

    \vspace{3pt}
    \textbf{Tabela \theawc:}
    Karta wymagania ogólnego dla mapy: #2\label{#1}

    \addcontentsline{lot}{table}
    {Tabela \theawc: Karta wymagania ogólnego dla mapy: #2}%
}

\womapcard{womap:display-spots}
{Wyświetlanie \glslink{spot}{spotów} na mapie}
{01}
{S}
{
    \womapdesc{System wyświetla mapę, na której zaznaczone są \glslink{spot}{spoty} w formie wielokątów lub markerów, tak aby były widoczne dla użytkownika.}
    \womapstakeholder{U3}
    \womaprelated{brak}
}

\womapcard{womap:spot-details}
{Wyświetlanie szczegółów \glslink{spot}{spota}}
{02}
{S}
{
    \womapdesc{Po kliknięciu na \glslink{spot}{spota}, system wyświetla użytkownikowi informacje o danym \glslink{spot}{spocie}, takich
    jak nazwa, lokalizacja, opis, zdjęcia czy komentarze.}
    \womapstakeholder{U3}
    \womaprelated{brak}
}

\womapcard{womap:spot-weather}
{Wyświetlanie informacji pogodowych \glslink{spot}{spota}}
{03}
{S}
{
    \womapdesc{Po kliknięciu na \glslink{spot}{spota}, system wyświetla użytkownikowi informacje o zarówno bieżącej, jak i prognozowanej pogodzie
    tak, aby użytkownik mógł dokładnie zapoznać się z warunkami panującymi w danym \glslink{spot}{spocie}.}
    \womapstakeholder{U3}
    \womaprelated{brak}
}

\womapcard{womap:spot-search}
{Wyszukiwanie \glslink{spot}{spotów}}
{04}
{S}
{
    \womapdesc{System umożliwia użytkownikowi wyszukiwanie \glslink{spot}{spotów} na mapie oraz filtrowanie ich listy.}
    \womapstakeholder{U3}
    \womaprelated{brak}
}

\womapcard{womap:spot-comment}
{Komentowanie \glslink{spot}{spotów}}
{05}
{S}
{
    \womapdesc{Użytkownik może dodać komentarz do wybranego \glslink{spot}{spota}, aby podzielić się swoją opinią i wrażeniami z innymi użytkownikami.}
    \womapstakeholder{U3}
    \womaprelated{brak}
}

\womapcard{womap:spot-media}
{Dodawanie media do \glslink{spot}{spotów}}
{06}
{S}
{
    \womapdesc{System umożliwia użytkownikowi dodanie do \glslink{spot}{spota} różne media, takie jak zdjęcia i filmy.}
    \womapstakeholder{U3}
    \womaprelated{brak}
}

\womapcard{womap:spot-user-location}
{Wyświetlenie lokalizacji użytkownika na mapie}
{07}
{S}
{
    \womapdesc{System wyświetla na mapie obecną lokalizację użytkownika, aby mógł sprawdzić jakie \glslink{spot}{spoty} znajdują się w pobliżu.}
    \womapstakeholder{U3}
    \womaprelated{brak}
}
