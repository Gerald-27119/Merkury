%! Author = Adam
%! Date = 24/11/2025

\subsubsection{Wymagania ogólne dla czatu}
\label{subsubsec:wymagania-ogolne-dla-chatu}

% Licznik tabel wymagań ogólnych dla czatu (resetowany na rozdział)
\newcounter{woczat}[chapter]
\renewcommand{\thewoczat}{\thechapter.\arabic{woczat}}

% Szerokość części z treścią (3 prawe kolumny zlane w jedną)
\newlength{\woczContentWidth}
\setlength{\woczContentWidth}{0.7\textwidth}

% --------- Pola karty (wiersze) ---------

% Id + priorytet – 4 kolumny
\newcommand{\woczpriority}[2]{%
    \textbf{Identyfikator:} & WOCZAT-#1 &
    \textbf{Priorytet:} & #2 \\ \hline
}

% Wiersze z etykietą + treścią na 3 kolumny
\newcommand{\woczname}[1]{\textbf{Nazwa:}              &
\multicolumn{3}{|p{\woczContentWidth}|}{#1} \\ \hline}
\newcommand{\woczdesc}[1]{\textbf{Opis:}               &
\multicolumn{3}{|p{\woczContentWidth}|}{#1} \\ \hline}
\newcommand{\woczstakeholder}[1]{\textbf{Udziałowiec:} &
\multicolumn{3}{|p{\woczContentWidth}|}{#1} \\ \hline}
\newcommand{\woczrelated}[1]{\textbf{Wymagania powiązane:} &
\multicolumn{3}{|p{\woczContentWidth}|}{#1} \\ \hline}

% (Pozostałe stare makra możesz skasować albo zostawić – i tak nie są używane)

\newcommand{\woczatcard}[5]{%
    \refstepcounter{woczat}%
    {\centering
    \begin{longtable}{|
            >{\columncolor{lightgray}}l|
        l|
            >{\columncolor{lightgray}}l|
        p{0.15\textwidth}|}
    \hline
    \rowcolor{lightgray}\multicolumn{4}{|c|}
    {\textbf{KARTA WYMAGANIA OGÓLNEGO DLA CZATU}} \\ \hline
    \endfirsthead
    \hline
    \rowcolor{lightgray}\multicolumn{4}{|c|}
    {\textbf{KARTA WYMAGANIA OGÓLNEGO DLA CZATU (cd.)}} \\ \hline
    \endhead

    % właściwa treść karty
    \woczpriority{#3}{#4}
    \woczname{#2}
    #5

    \end{longtable}
    \par} % koniec centrowania

    \vspace{3pt}
    \textbf{Tabela \thewoczat:}
    Karta wymagania ogólnego dla czatu: #2\label{#1}

    \addcontentsline{lot}{table}
    {Tabela \thewoczat: Karta wymagania ogólnego dla czatu: #2}%
}

% --------- KARTY ---------

\woczatcard
{woczat:send-message}
{Wysyłanie wiadomości na czacie}
{01}
{S}
{
    \woczdesc{System umożliwia użytkownikowi wysyłanie wiadomości w ramach wybranego
    czatu, tak aby uczestnicy mogli przekazywać sobie wiedzę na temat dronów lub umawiać się na wspólne spotkania w danym spocie.}

    \woczstakeholder{UO3}

    \woczrelated{\textemdash\ brak}
}

\woczatcard
{woczat:edit-chat}
{Edycja czatu}
{02}
{S}
{
    \woczdesc{System umożliwia użytkownikowi z odpowiednimi uprawnieniami wprowadzanie
    podstawowych zmian w konfiguracji czatu, takich jak nazwa czatu czy skład uczestników.}

    \woczstakeholder{UO3}

    \woczrelated{\textemdash\ brak}
}

\woczatcard
{woczat:browse-history}
{Przeglądanie historii czatu}
{03}
{S}
{
    \woczdesc{System udostępnia użytkownikowi możliwość przeglądania wcześniejszych
    wiadomości na czacie, tak aby mógł wracać do poprzednich rozmów i ustaleń.}

    \woczstakeholder{UO3}

    \woczrelated{\textemdash\ brak}
}

\woczatcard
{woczat:create-chat}
{Tworzenie czatu}
{04}
{S}
{
    \woczdesc{System umożliwia użytkownikowi inicjowanie nowych rozmów poprzez
    tworzenie czatów prywatnych (1:1) lub grupowych z wybranymi uczestnikami.}

    \woczstakeholder{UO3}

    \woczrelated{\textemdash\ brak}
}
