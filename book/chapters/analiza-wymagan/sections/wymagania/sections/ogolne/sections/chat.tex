%! Author = Adam
%! Date = 24/11/2025

\subsubsection{Wymagania ogólne dla czatu}
\label{subsubsec:wymagania-ogolne-dla-chatu}

% Licznik tabel wymagań ogólnych dla czatu (resetowany na rozdział)
\newcounter{woczat}[chapter]
\renewcommand{\thewoczat}{\thechapter.\arabic{woczat}}

\newlength{\woczContentWidth}
\setlength{\woczContentWidth}{0.74\textwidth}

\newlength{\woczLabelWidth}
\setlength{\woczLabelWidth}{0.2\textwidth}

% --------- Pola karty (wiersze) ---------

% Id + priorytet – 4 kolumny
\newcommand{\woczpriority}[2]{%
    \textbf{Identyfikator:} & WOCZAT-#1 &
    \textbf{Priorytet:} & #2 \\ \hline
}

% Wiersze z etykietą + treścią na 3 kolumny
\newcommand{\woczname}[1]{\textbf{Nazwa:}              &
\multicolumn{3}{|p{\woczContentWidth}|}{#1} \\ \hline}
\newcommand{\woczdesc}[1]{\textbf{Opis:}               &
\multicolumn{3}{|p{\woczContentWidth}|}{#1} \\ \hline}
\newcommand{\woczstakeholder}[1]{\textbf{Udziałowiec:} &
\multicolumn{3}{|p{\woczContentWidth}|}{#1} \\ \hline}
\newcommand{\woczrelated}[1]{\textbf{Wymagania powiązane:} &
\multicolumn{3}{|p{\woczContentWidth}|}{#1} \\ \hline}

\newcommand{\woczatcard}[5]{%
    \refstepcounter{woczat}%
    {\centering
    \begin{longtable}{|
            >{\columncolor{lightgray}}p{\woczLabelWidth}|
        p{0.22\textwidth}|
            >{\columncolor{lightgray}}p{0.18\textwidth}|
        p{0.24\textwidth}|}

    \hline
    \rowcolor{lightgray}\multicolumn{4}{|c|}
    {\textbf{KARTA WYMAGANIA OGÓLNEGO DLA CZATU}} \\ \hline
    \endfirsthead
    \hline
    \rowcolor{lightgray}\multicolumn{4}{|c|}
    {\textbf{KARTA WYMAGANIA OGÓLNEGO DLA CZATU (cd.)}} \\ \hline
    \endhead

    % właściwa treść karty
    \woczpriority{#3}{#4}
    \woczname{#2}
    #5

    \end{longtable}
    \par} % koniec centrowania

    \vspace{3pt}
    \textbf{Tabela \thewoczat:}
    Karta wymagania ogólnego dla czatu: #2\label{#1}

    \addcontentsline{lot}{table}
    {Tabela \thewoczat: Karta wymagania ogólnego dla czatu: #2}%
}


\woczatcard
{woczat:send-message}
{Wysyłanie wiadomości na czacie}
{01}
{S}
{
    \woczdesc{System umożliwia użytkownikowi wysyłanie wiadomości w ramach wybranego
    czatu, tak aby uczestnicy mogli przekazywać sobie wiedzę na temat dronów lub umawiać się na wspólne spotkania w danym \glslink{spot}{spocie}.}

    \woczstakeholder{U3}

    \woczrelated{%
        \hyperref[wfczat:send-gif]{WFCZAT-01},
        \hyperref[wfczat:send-files]{WFCZAT-02},
        \hyperref[wfczat:private-messages]{WFCZAT-03},
        \hyperref[wfczat:group-messages]{WFCZAT-04},
        \hyperref[wfczat:emoticons]{WFCZAT-06},
        \hyperref[wfczat:edit-message]{WFCZAT-10},
        \hyperref[wpczat:visibility-members]{WPCZAT-01},
        \hyperref[wpczat:login-required]{WPCZAT-02},
        \hyperref[wpczat:sender-and-time]{WPCZAT-04},
        \hyperref[wpczat:send-immediately]{WPCZAT-06}%
    }
}

\woczatcard
{woczat:edit-chat}
{Edycja czatu}
{02}
{S}
{
    \woczdesc{System umożliwia użytkownikowi wprowadzanie
    podstawowych zmian w konfiguracji czatu, takich jak nazwa czatu czy skład uczestników.}

    \woczstakeholder{U3}

    \woczrelated{%
        \hyperref[wfczat:edit-group-name]{WFCZAT-08},
        \hyperref[wfczat:edit-group-avatar]{WFCZAT-09},
        \hyperref[wfczat:add-users-existing-chat]{WFCZAT-11},
        \hyperref[wpczat:visibility-members]{WPCZAT-01},
        \hyperref[wpczat:login-required]{WPCZAT-02}%
    }
}

\woczatcard
{woczat:browse-history}
{Przeglądanie historii czatu}
{03}
{S}
{
    \woczdesc{System udostępnia użytkownikowi możliwość przeglądania wcześniejszych
    wiadomości na czacie, tak aby mógł wracać do poprzednich rozmów.}

    \woczstakeholder{U3}

    \woczrelated{%
        \hyperref[wfczat:chat-availability]{WFCZAT-07},
        \hyperref[wfczat:load-older-messages]{WFCZAT-12},
        \hyperref[wpczat:visibility-members]{WPCZAT-01},
        \hyperref[wpczat:login-required]{WPCZAT-02},
        \hyperref[wpczat:group-by-date]{WPCZAT-03},
        \hyperref[wpczat:load-older-under-10s]{WPCZAT-05}%
    }
}

\woczatcard
{woczat:create-chat}
{Tworzenie czatu}
{04}
{S}
{
    \woczdesc{System umożliwia użytkownikowi inicjowanie nowych rozmów poprzez
    tworzenie czatów prywatnych (1:1) lub grupowych z wybranymi uczestnikami.}

    \woczstakeholder{U3}

    \woczrelated{%
        \hyperref[wfczat:create-chat]{WFCZAT-05},
        \hyperref[wpczat:visibility-members]{WPCZAT-01},
        \hyperref[wpczat:login-required]{WPCZAT-02}%
    }
}

