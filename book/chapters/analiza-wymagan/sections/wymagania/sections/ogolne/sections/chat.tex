%! Author = Adam
%! Date = 24/11/2025

\subsubsection{Wymagania ogólne dla czatu}
\label{subsubsec:wymagania-ogolne-dla-chatu}

% ============================
% WYMAGANIA OGÓLNE DLA CZATU
% Identyfikatory: WOCZAT-XX
% ============================

% Licznik tabel wymagań ogólnych dla czatu (resetowany na rozdział)
\newcounter{woczat}[chapter]
\renewcommand{\thewoczat}{\thechapter.\arabic{woczat}}

% Szerokość części z treścią (3 prawe kolumny zlane w jedną)
\newlength{\woczContentWidth}
\setlength{\woczContentWidth}{0.7\textwidth}

% --------- Pola karty (wiersze) ---------

% Id + priorytet – 4 kolumny
\newcommand{\woczpriority}[2]{%
    \textbf{Identyfikator:} & WOCZAT-#1 &
    \textbf{Priorytet:} & #2 \\ \hline
}

% Wiersze z etykietą + treścią na 3 kolumny
\newcommand{\woczname}[1]{\textbf{Nazwa:}              &
\multicolumn{3}{|p{\woczContentWidth}|}{#1} \\ \hline}
\newcommand{\woczdesc}[1]{\textbf{Opis:}               &
\multicolumn{3}{|p{\woczContentWidth}|}{#1} \\ \hline}
\newcommand{\woczaccept}[1]{\textbf{Kryteria akceptacji:} &
\multicolumn{3}{|p{\woczContentWidth}|}{#1} \\ \hline}
\newcommand{\woczinput}[1]{\textbf{Dane wejściowe:}    &
\multicolumn{3}{|p{\woczContentWidth}|}{#1} \\ \hline}
\newcommand{\woczpre}[1]{\textbf{Warunki początkowe:}  &
\multicolumn{3}{|p{\woczContentWidth}|}{#1} \\ \hline}
\newcommand{\woczpost}[1]{\textbf{Warunki końcowe:}    &
\multicolumn{3}{|p{\woczContentWidth}|}{#1} \\ \hline}
\newcommand{\woczexceptions}[1]{\textbf{Sytuacje wyjątkowe:} &
\multicolumn{3}{|p{\woczContentWidth}|}{#1} \\ \hline}
\newcommand{\woczimpl}[1]{\textbf{Szczegóły implementacji:} &
\multicolumn{3}{|p{\woczContentWidth}|}{#1} \\ \hline}
\newcommand{\woczstakeholder}[1]{\textbf{Udziałowiec:} &
\multicolumn{3}{|p{\woczContentWidth}|}{#1} \\ \hline}
\newcommand{\woczresponsible}[1]{\textbf{Realizator:}  &
\multicolumn{3}{|p{\woczContentWidth}|}{#1} \\ \hline}
\newcommand{\woczstatus}[1]{\textbf{Status:}           &
\multicolumn{3}{|p{\woczContentWidth}|}{#1} \\ \hline}
\newcommand{\wocznote}[1]{\textbf{Notatka:}            &
\multicolumn{3}{|p{\woczContentWidth}|}{#1} \\ \hline}


\newcommand{\woczatcard}[5]{%
    \refstepcounter{woczat}%
    {\centering
    \begin{longtable}{|
            >{\columncolor{lightgray}}l|
        l|
            >{\columncolor{lightgray}}l|
        p{0.15\textwidth}|}
    \hline
    \rowcolor{lightgray}\multicolumn{4}{|c|}
    {\textbf{KARTA WYMAGANIA FUNKCJONALNEGO DLA CZATU}} \\ \hline
    \endfirsthead

    \hline
    \rowcolor{lightgray}\multicolumn{4}{|c|}
    {\textbf{KARTA WYMAGANIA FUNKCJONALNEGO DLA CZATU (cd.)}} \\ \hline
    \endhead

    % właściwa treść karty
    \woczpriority{#3}{#4}
    \woczname{#2}
    #5

    \end{longtable}
    \par} % koniec centrowania

    \vspace{3pt}
    \textbf{Tabela \thewoczat:}
    Karta wymagania funkcjonalnego dla czatu: #2\label{#1}

    \addcontentsline{lot}{table}
    {Tabela \thewoczat: Karta wymagania funkcjonalnego dla czatu: #2}%
}

\woczatcard
{woczat:send-message}
{Wysyłanie wiadomości na czacie}
{01}
{M}
{
    \woczdesc{System umożliwia użytkownikowi wysyłanie wiadomości w ramach wybranego
    czatu, tak aby uczestnicy mogli przekazywać sobie informacje w kontekście platformy Merkury.}

    \woczaccept{Wymaganie uznaje się za spełnione, jeżeli zalogowany użytkownik może
    wybrać czat, wprowadzić treść wiadomości i zobaczyć ją jako kolejną pozycję w rozmowie,
        a pozostali uczestnicy czatu widzą tę wiadomość po swojej stronie.}

    \woczinput{Zalogowany użytkownik, identyfikator wybranego czatu, treść wiadomości.}

    \woczpre{Użytkownik jest poprawnie zalogowany i posiada dostęp do danego czatu.}

    \woczpost{Nowo wysłana wiadomość staje się częścią historii danego czatu i jest
    dostępna dla wszystkich uprawnionych uczestników rozmowy.}

    \woczexceptions{Brak dostępu do czatu, problemy z połączeniem sieciowym,
        niepoprawne lub puste dane wejściowe.}

    \woczimpl{Ogólna obsługa wysyłania wiadomości w module czatu; szczegółowe rozwiązania
        (np. obsługa załączników, rodzaje wiadomości) zostaną doprecyzowane w wymaganiach
        szczegółowych.}

    \woczstakeholder{UO3}

    \woczresponsible{Adam Langmesser}

    \woczstatus{Zrealizowano}

    \wocznote{\textemdash\ brak}
}

\woczatcard
{woczat:edit-chat}
{Edycja czatu}
{02}
{M}
{
    \woczdesc{System umożliwia użytkownikowi z odpowiednimi uprawnieniami wprowadzanie
    podstawowych zmian w konfiguracji czatu, takich jak nazwa czatu czy skład uczestników.}

    \woczaccept{Wymaganie uznaje się za spełnione, jeżeli uprawniony użytkownik może
    zmienić kluczowe parametry czatu, a zaktualizowane informacje są widoczne dla
    pozostałych uczestników podczas korzystania z modułu czatu.}

    \woczinput{Zalogowany użytkownik, identyfikator czatu, zestaw nowych wartości
    konfiguracyjnych (np. nowa nazwa, lista uczestników).}

    \woczpre{Użytkownik jest zalogowany i posiada uprawnienia do zarządzania danym czatem.}

    \woczpost{Parametry czatu są zapisane w systemie w nowej postaci, a użytkownicy
    korzystają z odświeżonych informacji podczas dalszej komunikacji.}

    \woczexceptions{Próba edycji przez użytkownika bez wymaganych uprawnień,
        niepoprawne dane wejściowe, konflikt zmian, problemy z zapisem danych.}

    \woczimpl{Ogólna obsługa edycji ustawień czatu w module czatu; konkretne operacje
        (np. edycja nazwy, avatara, listy uczestników) będą doprecyzowane w wymaganiach
        szczegółowych.}

    \woczstakeholder{UO3}

    \woczresponsible{Adam Langmesser}

    \woczstatus{Zrealizowano}

    \wocznote{\textemdash\ brak}
}

\woczatcard
{woczat:browse-history}
{Przeglądanie historii czatu}
{03}
{M}
{
    \woczdesc{System udostępnia użytkownikowi możliwość przeglądania wcześniejszych
    wiadomości na czacie, tak aby mógł wracać do poprzednich rozmów i ustaleń.}

    \woczaccept{Wymaganie uznaje się za spełnione, jeżeli użytkownik po otwarciu
    czatu widzi fragment historii rozmowy i może w prosty sposób docierać do starszych
    wiadomości zgodnie z przyjętym sposobem przeglądania (np. przewijanie).}

    \woczinput{Zalogowany użytkownik, identyfikator czatu, ewentualne ogólne parametry
    zakresu historii (np. kierunek przeglądania).}

    \woczpre{Użytkownik jest zalogowany i posiada dostęp do danego czatu, a w systemie
    istnieją zapisane wiadomości dla tego czatu lub ich brak.}

    \woczpost{Użytkownik ma dostęp do historii rozmowy w takim zakresie, w jakim
    przewidziano to dla danego typu konta i konfiguracji systemu.}

    \woczexceptions{Brak uprawnień do danego czatu, tymczasowy problem z odczytem danych,
        brak jakiejkolwiek historii dla wskazanego czatu.}

    \woczimpl{Ogólny mechanizm udostępniania historii konwersacji w module czatu;
    szczegóły dotyczące paginacji, filtrów i limitów zostaną określone w wymaganiach
    szczegółowych.}

    \woczstakeholder{UO3}

    \woczresponsible{Adam Langmesser}

    \woczstatus{Zrealizowano}

    \wocznote{\textemdash\ brak}
}


\woczatcard
{woczat:create-chat}
{Tworzenie czatu}
{04}
{M}
{
    \woczdesc{System umożliwia użytkownikowi inicjowanie nowych rozmów poprzez
    tworzenie czatów prywatnych (1:1) lub grupowych z wybranymi uczestnikami.}

    \woczaccept{Wymaganie uznaje się za spełnione, jeżeli zalogowany użytkownik może
    zainicjować nowy czat, wskazać podstawowe informacje oraz uczestników, a nowy czat
    pojawia się na liście czatów i jest gotowy do użycia.}

    \woczinput{Zalogowany użytkownik, typ tworzonego czatu (np. 1:1, grupowy),
        podstawowe informacje opisujące czat oraz lista uczestników.}

    \woczpre{Użytkownik jest zalogowany i posiada uprawnienia do inicjowania nowych
    czatów w systemie.}

    \woczpost{Nowy czat jest dodany do systemu i widoczny na listach czatów
    odpowiednich użytkowników, a dalsza komunikacja może odbywać się w jego ramach.}

    \woczexceptions{Próba utworzenia czatu z nieprawidłowymi danymi (np. nieistniejący
    użytkownik), przekroczenie ogólnych limitów systemowych, problem z zapisem nowego
    czatu.}

    \woczimpl{Ogólna obsługa zakładania nowych czatów w module czatu; szczegóły
    dotyczące rodzajów czatów, limitów i dodatkowych ustawień zostaną doprecyzowane
    w wymaganiach szczegółowych.}

    \woczstakeholder{UO3}

    \woczresponsible{Adam Langmesser}

    \woczstatus{Zrealizowano}

    \wocznote{\textemdash\ brak}
}
