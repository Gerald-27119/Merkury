%! Author = mateusz
%! Date = 17/10/2025

\section{Wymagania funkcjonalne}
\label{sec:wymagania-funkcjonalne}

Niniejszy rozdział zawiera wymagania funkcjonalne postawione systemowi.
Został on podzielony tematycznie.

% --- Licznik kart wymagań funkcjonalnych ---
\newcounter{funcreq}[chapter]
\renewcommand{\thefuncreq}{\thechapter.\arabic{funcreq}}

% --- Pola karty wymagania funkcjonalnego (lewa kolumna) ---
\newcommand{\fridpriority}[2]{%
    \textbf{Identyfikator:} &
    #1\hfill
    \begingroup
    \setlength{\fboxsep}{1pt}%
    \colorbox{lightgray}{\strut\textbf{Priorytet:}}~#2%
    \endgroup
    \\ \hline
}
\newcommand{\frname}[1]{\textbf{Nazwa:} & #1 \\ \hline}
\newcommand{\frdesc}[1]{\textbf{Opis:} & #1 \\ \hline}
\newcommand{\fraccept}[1]{\textbf{Kryteria akceptacji:} & #1 \\ \hline}
\newcommand{\frinput}[1]{\textbf{Dane wejściowe:} & #1 \\ \hline}
\newcommand{\frpre}[1]{\textbf{Warunki początkowe:} & #1 \\ \hline}
\newcommand{\frpost}[1]{\textbf{Warunki końcowe:} & #1 \\ \hline}
\newcommand{\frexceptions}[1]{\textbf{Sytuacje wyjątkowe:} & #1 \\ \hline}
\newcommand{\frimpl}[1]{\textbf{Szczegóły implementacji:} & #1 \\ \hline}
\newcommand{\frstakeholders}[1]{\textbf{Udziałowiec:} & #1 \\ \hline}
\newcommand{\frrelated}[1]{\textbf{Wymagania powiązane:} & #1 \\ \hline}

% --- Karta wymagania funkcjonalnego ---
% #1 = label, #2 = Nazwa, #3 = Identyfikator, #4 = Priorytet, #5 = reszta pól (makra fr*)
\newcommand{\funcreqcard}[5]{%
    \refstepcounter{funcreq}%
    \par\begin{center}
    \renewcommand{\arraystretch}{1.15}%
    \begin{tabularx}{\textwidth}{|>{\columncolor{lightgray}\raggedright\arraybackslash}p{0.19\textwidth}|X|}
    \rowcolor{lightgray}
    \multicolumn{2}{|c|}{\textbf{KARTA WYMAGANIA FUNKCJONALNEGO}} \\ \hline
    \fridpriority{#3}{#4}
    \frname{#2}
    #5
    \end{tabularx}
    \vspace{3pt}
    \textbf{Tabela \thefuncreq:} Wymaganie funkcjonalne: #2\label{#1}
    \end{center}%
    \addcontentsline{lot}{table}{Tabela \thefuncreq: Wymaganie funkcjonalne: #2}%
}


\subimport{subimports/}{funkcjonalnosci-dla-mapy.tex}
\subimport{subimports/}{funkcjonalnosci-dla-chatu.tex}
\subimport{subimports/}{funkcjonalnosci-dla-forum.tex}
\subimport{subimports/}{funkcjonalnosci-dla-konta-uzytkownika.tex}
\subimport{subimports/}{funkcjonalnosci-dla-logowania-i-rejestracji.tex}
\subimport{subimports/}{funkcjonalosci-dla-wyszukiwarki-spotow.tex}
\subimport{subimports/}{funkcjonalnoci-dla-motywu.tex}
