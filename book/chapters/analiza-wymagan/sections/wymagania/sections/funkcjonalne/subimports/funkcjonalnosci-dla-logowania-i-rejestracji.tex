%! Author = Mateusz
%! Date = 22/12/2025

\subsubsection{Wymagania funkcjonalne dla logowania i rejestracji}
\label{subsubsec:wymagania-funkcjonalne-dla-logowania-rejestracji}

\newlength{\wflogLabelWidth}
\setlength{\wflogLabelWidth}{0.19\textwidth}

\newlength{\wflogColTwoWidth}
\setlength{\wflogColTwoWidth}{0.21\textwidth}

\newlength{\wflogColThreeWidth}
\setlength{\wflogColThreeWidth}{0.13\textwidth}

\newlength{\wflogColFourWidth}
\setlength{\wflogColFourWidth}{0.28\textwidth}

\newlength{\wflogContentWidth}
\setlength{\wflogContentWidth}{0.60\textwidth}

\newlength{\wflogHeaderHeight}
\setlength{\wflogHeaderHeight}{12mm}

\newcommand{\wflogthreecolcell}[1]{%
    \multicolumn{3}{|>{\raggedright\arraybackslash}p{\wflogContentWidth}|}{#1}%
}

\newcommand{\wflogthreecolcellpadded}[1]{%
    \multicolumn{3}{|>{\raggedright\arraybackslash}p{\wflogContentWidth}|}{%
        \vspace{0.4ex}%
        #1\par\vspace{0.4ex}%
    }%
}

\newcommand{\wflogHeaderRow}[1]{%
    \rowcolor{lightgray}%
    \multicolumn{4}{|c|}{%
        \parbox[c][\wflogHeaderHeight][c]{\linewidth}{%
            \centering\bfseries
            \vspace{1.2ex}%
            #1%
            \vspace{1.2ex}%
        }%
    }\\ \hline
}

% --- pola karty ---

\newcommand{\wflogpriority}[2]{%
    \textbf{Identyfikator:} & WFLOG-#1 &
    \textbf{Priorytet:}     & #2 \\ \hline
}

\newcommand{\wflogname}[1]{%
    \textbf{Nazwa:} &
    \wflogthreecolcell{#1} \\ \hline
}

\newcommand{\wflogdesc}[1]{%
    \textbf{Opis:} &
    \wflogthreecolcell{#1} \\ \hline
}

\newcommand{\wflogaccept}[1]{%
    \textbf{Kryteria akceptacji:} &
    \multicolumn{3}{|>{\raggedright\arraybackslash}p{\wflogContentWidth}|}{%
        \begingroup
        \setlength{\leftmargini}{1.2em}%
        \setlength{\topsep}{0pt}%
        \setlength{\partopsep}{0pt}%
        \setlength{\itemsep}{0.2ex}%
        \setlength{\parsep}{0pt}%
        \vspace*{-1.8ex}
        #1%
        \vspace*{-1.4ex}
        \endgroup
    }\\ \hline
}

\newcommand{\wflogstakeholder}[1]{%
    \textbf{Udziałowiec:} &
    \wflogthreecolcell{#1} \\ \hline
}

\newcommand{\wflogresponsible}[1]{%
    \textbf{Realizator:} &
    \wflogthreecolcell{#1} \\ \hline
}

\newcommand{\wflogrelated}[1]{%
    \textbf{Wymagania powiązane:} &
    \wflogthreecolcell{#1} \\ \hline
}

% --- szablon karty wymagania funkcjonalnego ---

\newcommand{\wflogatcard}[5]{%
    \refstepcounter{awc}%
    {%
        \centering
        \begin{longtable}{|
                >{\columncolor{lightgray}\raggedright\arraybackslash}p{\wflogLabelWidth}|
            p{\wflogColTwoWidth}|
                >{\columncolor{lightgray}\raggedright\arraybackslash}p{\wflogColThreeWidth}|
            p{\wflogColFourWidth}|}
        \hline
        \wflogHeaderRow{\shortstack{KARTA WYMAGANIA FUNKCJONALNEGO DLA \\ LOGOWANIA I REJESTRACJI}}
        \endfirsthead
        \hline
        \wflogHeaderRow{\shortstack{KARTA WYMAGANIA FUNKCJONALNEGO DLA \\ LOGOWANIA I REJESTRACJI (cd.)}}
        \endhead
        \wflogpriority{#3}{#4}
        \wflogname{#2}
        #5
        \end{longtable}
        \par
    }%
    \vspace{3pt}%
    \textbf{Tabela \theawc:} Wymaganie funkcjonalne dla logowania i rejestracji: #2\label{#1}%
    \addcontentsline{lot}{table}{Tabela \theawc: Wymaganie funkcjonalne dla logowania i rejestracji: #2}%
}


% =========================================
% WFLOG-01 – Logowanie e-mail + hasło
% =========================================
\wflogatcard
{wflog:login-email-password}
{Możliwość zalogowania się za pomocą adresu e-mail i hasła}
{01}
{S}
{
    \wflogdesc{System umożliwia użytkownikowi zalogowanie się do aplikacji przy użyciu formularza
        logowania z adresem e-mail i hasłem, aby uzyskać dostęp do funkcji dostępnych dla użytkownika zalogowanego.}

    \wflogaccept{%
        \begin{itemize}
            \item Użytkownik może wprowadzić adres e-mail oraz hasło w formularzu logowania.
            \item Po wysłaniu formularza i poprawnej weryfikacji danych użytkownik zostaje zalogowany.
            \item W przypadku błędnych danych logowania system wyświetla czytelny komunikat o błędzie.
        \end{itemize}
    }

    \wflogstakeholder{U3}
    \wflogresponsible{Kacper Badek}
    \wflogrelated{%
        \hyperref[wolog:login]{WOLOG-01},
        \hyperref[wflog:forms-validation]{WFLOG-08},
        \hyperref[wflog:auth-error-handling]{WFLOG-09},
        \hyperref[wplog:auth-response-under-3s]{WPLOG-01},
        \hyperref[wplog:secure-session-cookie]{WPLOG-02},
        \hyperref[wplog:no-sensitive-error-details]{WPLOG-03},
        \hyperref[wplog:form-validation-ux]{WPLOG-04},
        \hyperref[wplog:network-failure-handling]{WPLOG-05},
        \hyperref[wplog:accessibility-keyboard]{WPLOG-06}.%
    }
}

% =========================================
% WFLOG-02 – Logowanie przez Google/GitHub (SSO)
% =========================================
\wflogatcard
{wflog:login-sso}
{Możliwość zalogowania się za pomocą konta Google lub GitHub}
{02}
{S}
{
    \wflogdesc{System umożliwia użytkownikowi zalogowanie się do aplikacji przy użyciu konta Google lub GitHub
        (logowanie zewnętrzne), aby uprościć proces uwierzytelniania i skrócić czas dostępu do aplikacji.}

    \wflogaccept{%
        \begin{itemize}
            \item Użytkownik ma dostęp do przycisków logowania przez Google oraz GitHub.
            \item Po poprawnej autoryzacji u dostawcy (Google/GitHub) użytkownik zostaje zalogowany w aplikacji.
            \item W przypadku nieudanej lub przerwanej autoryzacji system wyświetla informację o błędzie i pozostawia użytkownika niezalogowanego.
        \end{itemize}
    }

    \wflogstakeholder{U3}
    \wflogresponsible{Stanisław Oziemczuk}
    \wflogrelated{%
        \hyperref[wolog:login]{WOLOG-01},
        \hyperref[wflog:auth-error-handling]{WFLOG-09},
        \hyperref[wplog:auth-response-under-3s]{WPLOG-01},
        \hyperref[wplog:secure-session-cookie]{WPLOG-02},
        \hyperref[wplog:no-sensitive-error-details]{WPLOG-03},
        \hyperref[wplog:network-failure-handling]{WPLOG-05},
        \hyperref[wplog:accessibility-keyboard]{WPLOG-06}.%
    }
}

% =========================================
% WFLOG-03 – Rejestracja formularzem
% =========================================
\wflogatcard
{wflog:register-form}
{Możliwość zarejestrowania się za pomocą formularza}
{03}
{S}
{
    \wflogdesc{System umożliwia użytkownikowi rejestrację konta poprzez formularz rejestracji,
        w którym podaje adres e-mail, nazwę użytkownika oraz hasło, aby utworzyć nowe konto w aplikacji.}

    \wflogaccept{%
        \begin{itemize}
            \item Użytkownik może wypełnić formularz rejestracji: adres e-mail, nazwa użytkownika, hasło, powtórszone hasło.
            \item Po wysłaniu formularza i poprawnej walidacji danych konto zostaje utworzone.
            \item W przypadku błędnych danych lub błędu po stronie API system wyświetla komunikat o błędzie.
        \end{itemize}
    }

    \wflogstakeholder{U3}
    \wflogresponsible{Mateusz Redosz}
    \wflogrelated{%
        \hyperref[wolog:register]{WOLOG-02},
        \hyperref[wflog:forms-validation]{WFLOG-08},
        \hyperref[wflog:auth-error-handling]{WFLOG-09},
        \hyperref[wplog:auth-response-under-3s]{WPLOG-01},
        \hyperref[wplog:no-sensitive-error-details]{WPLOG-03},
        \hyperref[wplog:form-validation-ux]{WPLOG-04},
        \hyperref[wplog:network-failure-handling]{WPLOG-05},
        \hyperref[wplog:accessibility-keyboard]{WPLOG-06}.%
    }
}

% =========================================
% WFLOG-04 – Rejestracja przez Google/GitHub (SSO)
% =========================================
\wflogatcard
{wflog:register-sso}
{Możliwość zarejestrowania się za pomocą konta Google lub GitHub}
{04}
{S}
{
    \wflogdesc{System umożliwia użytkownikowi utworzenie konta w aplikacji przy użyciu konta Google lub GitHub
        (rejestracja zewnętrzna), aby uprościć proces zakładania konta.}

    \wflogaccept{%
        \begin{itemize}
            \item Użytkownik ma dostęp do przycisków rejestracji przez Google oraz GitHub.
            \item Po poprawnej autoryzacji u dostawcy (Google/GitHub) konto w aplikacji zostaje utworzone, a użytkownik zostaje zalogowany.
            \item W przypadku nieudanej lub przerwanej autoryzacji system wyświetla komunikat i nie tworzy konta.
        \end{itemize}
    }

    \wflogstakeholder{U3}
    \wflogresponsible{Stanisław Oziemczuk}
    \wflogrelated{%
        \hyperref[wolog:register]{WOLOG-02},
        \hyperref[wflog:auth-error-handling]{WFLOG-09},
        \hyperref[wplog:auth-response-under-3s]{WPLOG-01},
        \hyperref[wplog:secure-session-cookie]{WPLOG-02},
        \hyperref[wplog:no-sensitive-error-details]{WPLOG-03},
        \hyperref[wplog:network-failure-handling]{WPLOG-05},
        \hyperref[wplog:accessibility-keyboard]{WPLOG-06}.%
    }
}

% =========================================
% WFLOG-05 – Wysłanie e-maila do resetu hasła
% =========================================
\wflogatcard
{wflog:password-reset-email}
{Możliwość wysłania e-maila do resetu hasła}
{05}
{S}
{
    \wflogdesc{System umożliwia użytkownikowi zainicjowanie procedury resetu hasła poprzez wysłanie wiadomości e-mail,
        aby użytkownik mógł odzyskać dostęp do konta w przypadku zapomnienia hasła.}

    \wflogaccept{%
        \begin{itemize}
            \item Użytkownik może podać adres e-mail w formularzu resetu hasła.
            \item Po wysłaniu formularza system wysyła wiadomość e-mail z linkiem do ustawienia nowego hasła.
            \item W przypadku błędu system wyświetla czytelny komunikat o niepowodzeniu operacji.
        \end{itemize}
    }

    \wflogstakeholder{U3}
    \wflogresponsible{Kacper Badek}
    \wflogrelated{%
        \hyperref[wolog:password-reset]{WOLOG-03},
        \hyperref[wflog:forms-validation]{WFLOG-08},
        \hyperref[wflog:auth-error-handling]{WFLOG-09},
        \hyperref[wflog:password-reset-confirm]{WFLOG-06},
        \hyperref[wplog:auth-response-under-3s]{WPLOG-01},
        \hyperref[wplog:no-sensitive-error-details]{WPLOG-03},
        \hyperref[wplog:form-validation-ux]{WPLOG-04},
        \hyperref[wplog:network-failure-handling]{WPLOG-05},
        \hyperref[wplog:accessibility-keyboard]{WPLOG-06}.%
    }
}

% =========================================
% WFLOG-06 – Ustawienie nowego hasła z linku
% =========================================
\wflogatcard
{wflog:password-reset-confirm}
{Możliwość ustawienia nowego hasła po resecie}
{06}
{S}
{
    \wflogdesc{System umożliwia użytkownikowi ustawienie nowego hasła po przejściu z linku otrzymanego w wiadomości e-mail,
        aby zakończyć proces resetu hasła i odzyskać dostęp do konta.}

    \wflogaccept{%
        \begin{itemize}
            \item Użytkownik może otworzyć stronę ustawienia nowego hasła z linku w wiadomości e-mail.
            \item Użytkownik może wprowadzić nowe hasło i potwierdzić je.
            \item Po zapisaniu nowego hasła użytkownik otrzymuje potwierdzenie powodzenia operacji.
            \item W przypadku niepoprawnego lub wygasłego tokenu system wyświetla komunikat o błędzie i nie zmienia hasła.
        \end{itemize}
    }

    \wflogstakeholder{U3}
    \wflogresponsible{Kacper Badek}
    \wflogrelated{%
        \hyperref[wolog:password-reset]{WOLOG-03},
        \hyperref[wflog:forms-validation]{WFLOG-08},
        \hyperref[wflog:auth-error-handling]{WFLOG-09},
        \hyperref[wflog:password-reset-email]{WFLOG-05},
        \hyperref[wplog:auth-response-under-3s]{WPLOG-01},
        \hyperref[wplog:no-sensitive-error-details]{WPLOG-03},
        \hyperref[wplog:form-validation-ux]{WPLOG-04},
        \hyperref[wplog:network-failure-handling]{WPLOG-05},
        \hyperref[wplog:accessibility-keyboard]{WPLOG-06}.%
    }
}

% =========================================
% WFLOG-07 – Wylogowanie użytkownika
% =========================================
\wflogatcard
{wflog:logout}
{Możliwość wylogowania się z aplikacji}
{07}
{M}
{
    \wflogdesc{System umożliwia użytkownikowi wylogowanie się z aplikacji, aby zakończyć sesję i uniemożliwić dostęp
    do konta osobom trzecim na współdzielonym urządzeniu.}

    \wflogaccept{%
        \begin{itemize}
            \item Użytkownik ma dostęp do akcji wylogowania w interfejsie.
            \item Po wylogowaniu użytkownik traci dostęp do widoków wymagających zalogowania.
            \item Po wylogowaniu aplikacja prezentuje stan użytkownika niezalogowanego.
        \end{itemize}
    }

    \wflogstakeholder{U3}
    \wflogresponsible{Mateusz Redosz, Kacper Badek, Stanisław Oziemczuk, Adam Langmesser}
    \wflogrelated{%
        \hyperref[wolog:login]{WOLOG-01},
        \hyperref[wplog:secure-session-cookie]{WPLOG-02},
        \hyperref[wplog:no-sensitive-error-details]{WPLOG-03},
        \hyperref[wplog:network-failure-handling]{WPLOG-05}.%
    }
}

% =========================================
% WFLOG-08 – Walidacja pól formularzy logowania i rejestracji
% =========================================
\wflogatcard
{wflog:forms-validation}
{Walidacja pól formularzy logowania i rejestracji}
{08}
{M}
{
    \wflogdesc{System waliduje dane wprowadzane w formularzach logowania i rejestracji, aby ograniczyć liczbę błędów
    oraz zapewnić poprawność danych wysyłanych do API.}

    \wflogaccept{%
        \begin{itemize}
            \item Formularze nie pozwalają na wysłanie pustych pól wymaganych.
            \item Każde z pól zawiera maksumalną oraz minimalą ilość znaków.
            \item Dla hasła sprawdzana jest odwpowiednio sila kombinacja.
            \item Dla adresu e-mail weryfikowany jest poprawny format.
            \item W przypadku niepoprawnych danych użytkownik widzi czytelne komunikaty walidacyjne przy polach.
        \end{itemize}
    }

    \wflogstakeholder{U3}
    \wflogresponsible{Mateusz Redosz, Kacper Badek, Stanisław Oziemczuk}
    \wflogrelated{%
        \hyperref[wolog:login]{WOLOG-01},
        \hyperref[wolog:register]{WOLOG-02},
        \hyperref[wolog:password-reset]{WOLOG-03},
        \hyperref[wplog:form-validation-ux]{WPLOG-04},
        \hyperref[wplog:accessibility-keyboard]{WPLOG-06}.%
    }
}

% =========================================
% WFLOG-09 – Obsługa błędów logowania i rejestracji
% =========================================
\wflogatcard
{wflog:auth-error-handling}
{Obsługa błędów podczas logowania i rejestracji}
{09}
{M}
{
    \wflogdesc{System zapewnia spójną obsługę błędów podczas logowania i rejestracji, aby użytkownik rozumiał
    przyczynę niepowodzenia i mógł podjąć dalsze działania.}

    \wflogaccept{%
        \begin{itemize}
            \item W przypadku błędnych danych logowania użytkownik widzi komunikat o błędzie.
            \item W przypadku błędu połączenia z API użytkownik widzi komunikat o problemie technicznym.
            \item Komunikaty błędów nie ujawniają wrażliwych informacji (szczegółów wewnętrznych serwera).
        \end{itemize}
    }

    \wflogstakeholder{U3}
    \wflogresponsible{Mateusz Redosz, Kacper Badek, Stanisław Oziemczuk}
    \wflogrelated{%
        \hyperref[wolog:login]{WOLOG-01},
        \hyperref[wolog:register]{WOLOG-02},
        \hyperref[wolog:password-reset]{WOLOG-03},
        \hyperref[wplog:no-sensitive-error-details]{WPLOG-03},
        \hyperref[wplog:network-failure-handling]{WPLOG-05}.%
    }
}

% =========================================
% WFLOG-10 – Przełączanie widoków: logowanie / rejestracja / reset
% =========================================
\wflogatcard
{wflog:switch-auth-views}
{Możliwość przełączania widoków logowania, rejestracji i resetu hasła}
{10}
{M}
{
    \wflogdesc{System umożliwia użytkownikowi przechodzenie pomiędzy formularzami logowania, rejestracji oraz resetu hasła,
        aby użytkownik mógł szybko wybrać właściwą akcję bez opuszczania modułu uwierzytelniania.}

    \wflogaccept{%
        \begin{itemize}
            \item Użytkownik może przejść z logowania do rejestracji i odwrotnie.
            \item Użytkownik może przejść do widoku resetu hasła z poziomu logowania.
            \item Zmiana widoku nie wymaga przeładowania strony.
        \end{itemize}
    }

    \wflogstakeholder{U3}
    \wflogresponsible{Mateusz Redosz, Kacper Badek, Stanisław Oziemczuk}
    \wflogrelated{%
        \hyperref[wolog:login]{WOLOG-01},
        \hyperref[wolog:register]{WOLOG-02},
        \hyperref[wolog:password-reset]{WOLOG-03},
        \hyperref[wplog:form-validation-ux]{WPLOG-04},
        \hyperref[wplog:accessibility-keyboard]{WPLOG-06}.%
    }
}
