%! Author = Stanisław Oziemczuk
%! Date = 19/12/2025

\subsubsection{Wymagania funkcjonalne dla mapy}
\label{subsubsec:wymagania-funkcjonalne-dla-mapy}

\newlength{\wfmapLabelWidth}
\setlength{\wfmapLabelWidth}{0.19\textwidth}

\newlength{\wfmapColTwoWidth}
\setlength{\wfmapColTwoWidth}{0.21\textwidth}

\newlength{\wfmapColThreeWidth}
\setlength{\wfmapColThreeWidth}{0.13\textwidth}

\newlength{\wfmapColFourWidth}
\setlength{\wfmapColFourWidth}{0.28\textwidth}

\newlength{\wfmapContentWidth}
\setlength{\wfmapContentWidth}{0.60\textwidth}

\newcommand{\wfmapthreecolcell}[1]{%
    \multicolumn{3}{|>{\raggedright\arraybackslash}p{\wfmapContentWidth}|}{#1}%
}

\newcommand{\wfmapthreecolcellpadded}[1]{%
    \multicolumn{3}{|>{\raggedright\arraybackslash}p{\wfmapContentWidth}|}{%
        \vspace{0.4ex}%
        #1\par\vspace{0.4ex}%
    }%
}

% --- pola karty ---

\newcommand{\wfmappriority}[2]{%
    \textbf{Identyfikator:} & WFMAPA-#1 &
    \textbf{Priorytet:}     & #2 \\ \hline
}

\newcommand{\wfmapname}[1]{%
    \textbf{Nazwa:} &
    \wfmapthreecolcell{#1} \\ \hline
}

\newcommand{\wfmapdesc}[1]{%
    \textbf{Opis:} &
    \wfmapthreecolcell{#1} \\ \hline
}

\newcommand{\wfmapaccept}[1]{%
    \textbf{Kryteria akceptacji:} &
    \multicolumn{3}{|>{\raggedright\arraybackslash}p{\wfmapContentWidth}|}{%
        \begingroup
        \setlength{\leftmargini}{1.2em}%
        \setlength{\topsep}{0pt}%
        \setlength{\partopsep}{0pt}%
        \setlength{\itemsep}{0.2ex}%
        \setlength{\parsep}{0pt}%
        \vspace*{-1.8ex}
        #1%
        \vspace*{-1.4ex}
        \endgroup
    }\\ \hline
}

\newcommand{\wfmapstakeholder}[1]{%
    \textbf{Udziałowiec:} &
    \wfmapthreecolcell{#1} \\ \hline
}

\newcommand{\wfmapresponsible}[1]{%
    \textbf{Realizator:} &
    \wfmapthreecolcell{#1} \\ \hline
}

\newcommand{\wfmaprelated}[1]{%
    \textbf{Wymagania powiązane:} &
    \wfmapthreecolcell{#1} \\ \hline
}

% --- szablon karty wymagania funkcjonalnego ---

\newcommand{\wfmapcard}[5]{%
    \refstepcounter{awc}%
    {%
        \centering
        \begin{longtable}{|
                >{\columncolor{lightgray}\raggedright\arraybackslash}p{\wfmapLabelWidth}|
            p{\wfmapColTwoWidth}|
                >{\columncolor{lightgray}\raggedright\arraybackslash}p{\wfmapColThreeWidth}|
            p{\wfmapColFourWidth}|}
        \hline
        \rowcolor{lightgray}\multicolumn{4}{|c|}{\textbf{KARTA WYMAGANIA FUNKCJONALNEGO DLA MAPY}} \\ \hline
        \endfirsthead
        \hline
        \rowcolor{lightgray}\multicolumn{4}{|c|}{\textbf{KARTA WYMAGANIA FUNKCJONALNEGO DLA MAPY (cd.)}} \\ \hline
        \endhead
        \wfmappriority{#3}{#4}
        \wfmapname{#2}
        #5
        \end{longtable}
        \par
    }%
    \vspace{3pt}%
    \textbf{Tabela \theawc:} Wymaganie funkcjonalne dla mapy: #2\label{#1}%
    \addcontentsline{lot}{table}{Tabela \theawc: Wymaganie funkcjonalne dla mapy: #2}%
}

\wfmapcard{wfmap:display-spots}
{Wyświetlanie \glslink{spot}{spotów} na mapie}
{01}
{S}
{
    \wfmapdesc{System wyświetla na mapie \glslink{spot}{spoty} w formie wielokątów, a gdy widok mapy zostanie oddalony zamienia je na pinezki.}
    \wfmapaccept{
        \begin{itemize}
            \item Użytkownik widzi na mapie zaznaczone \glslink{spot}{spoty} jako wielokąty.
            \item Po oddaleniu widoku mapy, wielokąty są zamieniane na pinezki znajdujące się w tych samych miejscach.
        \end{itemize}
    }
    \wfmapstakeholder{U3}
    \wfmapresponsible{Stanisław Oziemczuk}
    \wfmaprelated{brak}
}

\wfmapcard{wfmap:spot-details}
{Wyświetlanie szczegółów \glslink{spot}{spota}}
{02}
{S}
{
    \wfmapdesc{Po kliknięciu na wybranego \glslink{spot}{spota}, system wyświetla o nim informacje.}
    \wfmapaccept{
        \begin{itemize}
            \item Użytkownik może kliknąć na mapie dowolny \glslink{spot}{spot}.
            \item O wybranym \glslink{spot}{spocie} wyświetlane są następujące informacje:
            nazwa, lokalizacja, ocena w gwiazdkach, liczba wyświetleń, opis, galeria z mediami (zdjęcia i filmy),
            lista komentarzy.
        \end{itemize}
    }
    \wfmapstakeholder{U3}
    \wfmapresponsible{Stanisław Oziemczuk}
    \wfmaprelated{brak}
}

\wfmapcard{wfmap:spot-media-gallery}
{Wyświetlanie interaktywnej galerii mediów \glslink{spot}{spota}}
{03}
{S}
{
    \wfmapdesc{W panelu ze szczegółami \glslink{spot}{spota} wyświetlana jest galeria mediów (zdjęć i filmów), w której użytkownik może przeglądać
    wszystkie media dodane do wybranego \glslink{spot}{spota}.}
    \wfmapaccept{
        \begin{itemize}
            \item Użytkownikowi wyświetlana jest galeria zawierająca zdjęcia oraz filmy dodane do wybranego \glslink{spot}{spota}.
            \item Galeria umożliwia przechodzenie w sposób zapętlony między mediami za pomocą strzałek.
            \item Pod mediami wyświetlane są znaczniki informujące o ilości elementów w galerii.
        \end{itemize}
    }
    \wfmapstakeholder{U3}
    \wfmapresponsible{Stanisław Oziemczuk}
    \wfmaprelated{brak}
}

\wfmapcard{wfmap:spot-comments-list}
{Wyświetlanie przewijalnej listy komentarzy \glslink{spot}{spota}}
{04}
{S}
{
    \wfmapdesc{W panelu ze szczegółami \glslink{spot}{spota} wyświetlana jest lista wszystkich komentarzy.
    Komentarz zawiera następujące elementy: treść, autor, data dodania, ocena w gwiazdkach, media (opcjonalnie),
        liczbę polubień, liczbę niepolubień.}
    \wfmapaccept{
        \begin{itemize}
            \item Użytkownikowi wyświetlana jest lista wszystkich komentarzy dodanych do wybranego \glslink{spot}{spota}.
            \item Lista komentarzy jest przewijana za pomocą \glslink{infinite-scroll}{nieskończonego przewijania}.
            \item Każdy komentarz zawiera treść, autora, ocenę w gwiazdkach, datę dodania, liczbę polubień, liczbę niepolubień.
            \item Jeśli do komentarze są dodane media, są one wyświetlane.
        \end{itemize}
    }
    \wfmapstakeholder{U3}
    \wfmapresponsible{Stanisław Oziemczuk}
    \wfmaprelated{brak}
}

\wfmapcard{wfmap:spot-comments-vote}
{Ocena komentarza \glslink{spot}{spota}}
{05}
{S}
{
    \wfmapdesc{System umożliwia użytkownikowi ocenienie komentarza \glslink{spot}{spota}.}
    \wfmapaccept{
        \begin{itemize}
            \item Użytkownikowi wyświetlany jest komentarz dodany do wybranego \glslink{spot}{spota}.
            \item Komentarz zawiera przyciski do wystawienia zarówno pozytywnej, jak i negatywnej oceny.
            \item Po kliknięciu przycisku użytkownik natychmiast widzi zmianę w liczbie wystawionych opinii
            wybranego rodzaju oraz oznaczenie, którą opcję wybrał.
            \item Ponownie kliknięcie tego samego przycisku powoduje cofnięcie wystawienia oceny.
            \item Kliknięcie przycisku przeciwnej operacji powoduje wykonanie jej i cofnięcie poprzedniej.
        \end{itemize}
    }
    \wfmapstakeholder{U3}
    \wfmapresponsible{Stanisław Oziemczuk}
    \wfmaprelated{brak}
}

\wfmapcard{wfmap:spot-comment-add}
{Dodanie komentarza do \glslink{spot}{spota}}
{06}
{S}
{
    \wfmapdesc{System umożliwia użytkownikowi dodanie komentarza do \glslink{spot}{spota}.}
    \wfmapaccept{
        \begin{itemize}
            \item Użytkownikowi wyświetlany jest przycisk otwierający formularz umożliwiający dodanie nowego
            komentarza do wybranego \glslink{spot}{spota}.
            \item Formularz zawiera następujące pola: ocena spota w gwiazdkach od 0 do 5,
            treść komentarza, dodanie filmów i/lub zdjęć (od 0 do 20).
            \item Po kliknięciu przycisku do publikacji komentarza, użytkownik widzi go na liście wszystkich komentarzy.
            \item Zamknięcie formularza nie powoduje zapisu danych.
            \item Brak zalogowania uniemożliwia otwarcie formularza, użytkownik jest o tym odpowiednio informowany.
            \item Wpisanie niepoprawnych danych do formularza blokuje możliwość publikacji komentarza, a użytkownik
            jest informowany o błędach.
        \end{itemize}
    }
    \wfmapstakeholder{U3}
    \wfmapresponsible{Stanisław Oziemczuk}
    \wfmaprelated{brak}
}
