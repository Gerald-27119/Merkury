%! Author = Stanisław Oziemczuk
%! Date = 19/12/2025

\subsubsection{Wymagania funkcjonalne dla mapy}
\label{subsubsec:wymagania-funkcjonalne-dla-mapy}

\newlength{\wfmapLabelWidth}
\setlength{\wfmapLabelWidth}{0.19\textwidth}

\newlength{\wfmapColTwoWidth}
\setlength{\wfmapColTwoWidth}{0.21\textwidth}

\newlength{\wfmapColThreeWidth}
\setlength{\wfmapColThreeWidth}{0.13\textwidth}

\newlength{\wfmapColFourWidth}
\setlength{\wfmapColFourWidth}{0.28\textwidth}

\newlength{\wfmapContentWidth}
\setlength{\wfmapContentWidth}{0.60\textwidth}

\newcommand{\wfmapthreecolcell}[1]{%
    \multicolumn{3}{|>{\raggedright\arraybackslash}p{\wfmapContentWidth}|}{#1}%
}

\newcommand{\wfmapthreecolcellpadded}[1]{%
    \multicolumn{3}{|>{\raggedright\arraybackslash}p{\wfmapContentWidth}|}{%
        \vspace{0.4ex}%
        #1\par\vspace{0.4ex}%
    }%
}

% --- pola karty ---

\newcommand{\wfmappriority}[2]{%
    \textbf{Identyfikator:} & WFMAPA-#1 &
    \textbf{Priorytet:}     & #2 \\ \hline
}

\newcommand{\wfmapname}[1]{%
    \textbf{Nazwa:} &
    \wfmapthreecolcell{#1} \\ \hline
}

\newcommand{\wfmapdesc}[1]{%
    \textbf{Opis:} &
    \wfmapthreecolcell{#1} \\ \hline
}

\newcommand{\wfmapaccept}[1]{%
    \textbf{Kryteria akceptacji:} &
    \multicolumn{3}{|>{\raggedright\arraybackslash}p{\wfmapContentWidth}|}{%
        \begingroup
        \setlength{\leftmargini}{1.2em}%
        \setlength{\topsep}{0pt}%
        \setlength{\partopsep}{0pt}%
        \setlength{\itemsep}{0.2ex}%
        \setlength{\parsep}{0pt}%
        \vspace*{-1.8ex}
        #1%
        \vspace*{-1.4ex}
        \endgroup
    }\\ \hline
}

\newcommand{\wfmapstakeholder}[1]{%
    \textbf{Udziałowiec:} &
    \wfmapthreecolcell{#1} \\ \hline
}

\newcommand{\wfmapresponsible}[1]{%
    \textbf{Realizator:} &
    \wfmapthreecolcell{#1} \\ \hline
}

\newcommand{\wfmaprelated}[1]{%
    \textbf{Wymagania powiązane:} &
    \wfmapthreecolcell{#1} \\ \hline
}

% --- szablon karty wymagania funkcjonalnego ---

\newcommand{\wfmapcard}[5]{%
    \refstepcounter{awc}%
    {%
        \centering
        \begin{longtable}{|
                >{\columncolor{lightgray}\raggedright\arraybackslash}p{\wfmapLabelWidth}|
            p{\wfmapColTwoWidth}|
                >{\columncolor{lightgray}\raggedright\arraybackslash}p{\wfmapColThreeWidth}|
            p{\wfmapColFourWidth}|}
        \hline
        \rowcolor{lightgray}\multicolumn{4}{|c|}{\textbf{KARTA WYMAGANIA FUNKCJONALNEGO DLA MAPY}} \\ \hline
        \endfirsthead
        \hline
        \rowcolor{lightgray}\multicolumn{4}{|c|}{\textbf{KARTA WYMAGANIA FUNKCJONALNEGO DLA MAPY (cd.)}} \\ \hline
        \endhead
        \wfmappriority{#3}{#4}
        \wfmapname{#2}
        #5
        \end{longtable}
        \par
    }%
    \vspace{3pt}%
    \textbf{Tabela \theawc:} Wymaganie funkcjonalne dla mapy: #2\label{#1}%
    \addcontentsline{lot}{table}{Tabela \theawc: Wymaganie funkcjonalne dla mapy: #2}%
}

\wfmapcard{wfmap:display-spots}
{Wyświetlanie \glslink{spot}{spotów} na mapie}
{01}
{S}
{
    \wfmapdesc{System wyświetla na mapie \glslink{spot}{spoty} w formie wielokątów, a gdy widok mapy zostanie oddalony zamienia je na pinezki.}
    \wfmapaccept{
        \begin{itemize}
            \item Użytkownik widzi na mapie zaznaczone \glslink{spot}{spoty} jako wielokąty.
            \item Po oddaleniu widoku mapy, wielokąty są zamieniane na pinezki znajdujące się w tych samych miejscach.
        \end{itemize}
    }
    \wfmapstakeholder{U3}
    \wfmapresponsible{Stanisław Oziemczuk}
    \wfmaprelated{brak}
}
