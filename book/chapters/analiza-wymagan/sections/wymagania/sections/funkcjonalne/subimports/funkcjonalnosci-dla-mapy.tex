%! Author = Stanisław Oziemczuk
%! Date = 19/12/2025

\subsubsection{Wymagania funkcjonalne dla mapy}
\label{subsubsec:wymagania-funkcjonalne-dla-mapy}

\newlength{\wfmapLabelWidth}
\setlength{\wfmapLabelWidth}{0.19\textwidth}

\newlength{\wfmapColTwoWidth}
\setlength{\wfmapColTwoWidth}{0.21\textwidth}

\newlength{\wfmapColThreeWidth}
\setlength{\wfmapColThreeWidth}{0.13\textwidth}

\newlength{\wfmapColFourWidth}
\setlength{\wfmapColFourWidth}{0.28\textwidth}

\newlength{\wfmapContentWidth}
\setlength{\wfmapContentWidth}{0.60\textwidth}

\newcommand{\wfmapthreecolcell}[1]{%
    \multicolumn{3}{|>{\raggedright\arraybackslash}p{\wfmapContentWidth}|}{#1}%
}

\newcommand{\wfmapthreecolcellpadded}[1]{%
    \multicolumn{3}{|>{\raggedright\arraybackslash}p{\wfmapContentWidth}|}{%
        \vspace{0.4ex}%
        #1\par\vspace{0.4ex}%
    }%
}

% --- pola karty ---

\newcommand{\wfmappriority}[2]{%
    \textbf{Identyfikator:} & WFMAPA-#1 &
    \textbf{Priorytet:}     & #2 \\ \hline
}

\newcommand{\wfmapname}[1]{%
    \textbf{Nazwa:} &
    \wfmapthreecolcell{#1} \\ \hline
}

\newcommand{\wfmapdesc}[1]{%
    \textbf{Opis:} &
    \wfmapthreecolcell{#1} \\ \hline
}

\newcommand{\wfmapaccept}[1]{%
    \textbf{Kryteria akceptacji:} &
    \multicolumn{3}{|>{\raggedright\arraybackslash}p{\wfmapContentWidth}|}{%
        \begingroup
        \setlength{\leftmargini}{1.2em}%
        \setlength{\topsep}{0pt}%
        \setlength{\partopsep}{0pt}%
        \setlength{\itemsep}{0.1ex}%
        \setlength{\parsep}{0pt}%
        \vspace*{-1.8ex}
        #1%
        \vspace*{-1.4ex}
        \endgroup
    }\\ \hline
}

\newcommand{\wfmapstakeholder}[1]{%
    \textbf{Udziałowiec:} &
    \wfmapthreecolcell{#1} \\ \hline
}

\newcommand{\wfmapresponsible}[1]{%
    \textbf{Realizator:} &
    \wfmapthreecolcell{#1} \\ \hline
}

\newcommand{\wfmaprelated}[1]{%
    \textbf{Wymagania powiązane:} &
    \wfmapthreecolcell{#1} \\ \hline
}

% --- szablon karty wymagania funkcjonalnego ---

\newcommand{\wfmapcard}[5]{%
    \refstepcounter{awc}%
    {%
        \centering
        \begin{longtable}{|
                >{\columncolor{lightgray}\raggedright\arraybackslash}p{\wfmapLabelWidth}|
            p{\wfmapColTwoWidth}|
                >{\columncolor{lightgray}\raggedright\arraybackslash}p{\wfmapColThreeWidth}|
            p{\wfmapColFourWidth}|}
        \hline
        \rowcolor{lightgray}\multicolumn{4}{|c|}{\textbf{KARTA WYMAGANIA FUNKCJONALNEGO DLA MAPY}} \\ \hline
        \endfirsthead
        \hline
        \rowcolor{lightgray}\multicolumn{4}{|c|}{\textbf{KARTA WYMAGANIA FUNKCJONALNEGO DLA MAPY (cd.)}} \\ \hline
        \endhead
        \wfmappriority{#3}{#4}
        \wfmapname{#2}
        #5
        \end{longtable}
        \par
    }%
    \vspace{3pt}%
    \textbf{Tabela \theawc:} Wymaganie funkcjonalne dla mapy: #2\label{#1}%
    \addcontentsline{lot}{table}{Tabela \theawc: Wymaganie funkcjonalne dla mapy: #2}%
}

\wfmapcard{wfmap:display-spots}
{Wyświetlanie \glslink{spot}{spotów} na mapie}
{01}
{S}
{
    \wfmapdesc{System wyświetla na mapie \glslink{spot}{spoty} w formie wielokątów, a gdy widok mapy zostanie oddalony zamienia je na pinezki.}
    \wfmapaccept{
        \begin{itemize}
            \item Użytkownik widzi na mapie zaznaczone \glslink{spot}{spoty} jako wielokąty.
            \item Po oddaleniu widoku mapy, wielokąty są zamieniane na pinezki znajdujące się w tych samych miejscach.
        \end{itemize}
    }
    \wfmapstakeholder{U3}
    \wfmapresponsible{Stanisław Oziemczuk}
    \wfmaprelated{\hyperref[womap:display-spots]{WOMAPA-01}}
}

\wfmapcard{wfmap:spot-details}
{Wyświetlanie szczegółów \glslink{spot}{spota}}
{02}
{S}
{
    \wfmapdesc{Po kliknięciu na wybranego \glslink{spot}{spota}, system wyświetla o nim informacje.}
    \wfmapaccept{
        \begin{itemize}
            \item Użytkownik może kliknąć na mapie dowolny \glslink{spot}{spot}.
            \item O wybranym \glslink{spot}{spocie} wyświetlane są następujące informacje:
            nazwa, lokalizacja, ocena w gwiazdkach, liczba wyświetleń, opis, galeria z mediami (zdjęcia i filmy),
            lista komentarzy.
        \end{itemize}
    }
    \wfmapstakeholder{U3}
    \wfmapresponsible{Stanisław Oziemczuk}
    \wfmaprelated{
        \hyperref[womap:display-spots]{WOMAPA-01},
        \hyperref[womap:spot-details]{WOMAPA-02}
    }
}

\wfmapcard{wfmap:spot-media-gallery}
{Wyświetlanie interaktywnej galerii mediów \glslink{spot}{spota}}
{03}
{S}
{
    \wfmapdesc{W panelu ze szczegółami \glslink{spot}{spota} wyświetlana jest galeria mediów (zdjęć i filmów), w której użytkownik może przeglądać
    wszystkie media dodane do wybranego \glslink{spot}{spota}.}
    \wfmapaccept{
        \begin{itemize}
            \item Użytkownikowi wyświetlana jest galeria zawierająca zdjęcia oraz filmy dodane do wybranego \glslink{spot}{spota}.
            \item Galeria umożliwia przechodzenie w sposób zapętlony między mediami za pomocą strzałek.
            \item Pod mediami wyświetlane są znaczniki informujące o ilości elementów w galerii.
        \end{itemize}
    }
    \wfmapstakeholder{U3}
    \wfmapresponsible{Stanisław Oziemczuk}
    \wfmaprelated{
        \hyperref[womap:spot-details]{WOMAPA-02},
        \hyperref[womap:spot-media]{WOMAPA-06}
    }
}

\wfmapcard{wfmap:spot-comments-list}
{Wyświetlanie przewijalnej listy komentarzy \glslink{spot}{spota}}
{04}
{S}
{
    \wfmapdesc{W panelu ze szczegółami \glslink{spot}{spota} wyświetlana jest lista wszystkich komentarzy.
    Komentarz zawiera następujące elementy: treść, autor, data dodania, ocena w gwiazdkach, media (opcjonalnie),
        liczbę polubień, liczbę niepolubień.}
    \wfmapaccept{
        \begin{itemize}
            \item Użytkownikowi wyświetlana jest lista wszystkich komentarzy dodanych do wybranego \glslink{spot}{spota}.
            \item Lista komentarzy jest przewijana za pomocą \glslink{infinite-scroll}{nieskończonego przewijania}.
            \item Każdy komentarz zawiera treść, autora, ocenę w gwiazdkach, datę dodania, liczbę polubień, liczbę niepolubień.
            \item Jeśli do komentarze są dodane media, są one wyświetlane.
        \end{itemize}
    }
    \wfmapstakeholder{U3}
    \wfmapresponsible{Stanisław Oziemczuk}
    \wfmaprelated{
        \hyperref[womap:spot-details]{WOMAPA-02},
        \hyperref[womap:spot-comment]{WOMAPA-05}
    }
}

\wfmapcard{wfmap:spot-comments-vote}
{Ocena komentarza \glslink{spot}{spota}}
{05}
{S}
{
    \wfmapdesc{System umożliwia użytkownikowi ocenienie komentarza \glslink{spot}{spota}.}
    \wfmapaccept{
        \begin{itemize}
            \item Użytkownikowi wyświetlany jest komentarz dodany do wybranego \glslink{spot}{spota}.
            \item Komentarz zawiera przyciski do wystawienia zarówno pozytywnej, jak i negatywnej oceny.
            \item Po kliknięciu przycisku użytkownik natychmiast widzi zmianę w liczbie wystawionych opinii
            wybranego rodzaju oraz oznaczenie, którą opcję wybrał.
            \item Ponownie kliknięcie tego samego przycisku powoduje cofnięcie wystawienia oceny.
            \item Kliknięcie przycisku przeciwnej operacji powoduje wykonanie jej i cofnięcie poprzedniej.
            \item Jeżeli użytkownik nie jest zalogowany, wyświetlany jest odpowiedni komunikat.
        \end{itemize}
    }
    \wfmapstakeholder{U3}
    \wfmapresponsible{Stanisław Oziemczuk}
    \wfmaprelated{\hyperref[womap:spot-comment]{WOMAPA-05}}
}

\wfmapcard{wfmap:spot-comment-add}
{Dodanie komentarza do \glslink{spot}{spota}}
{06}
{S}
{
    \wfmapdesc{System umożliwia użytkownikowi dodanie komentarza do \glslink{spot}{spota}.}
    \wfmapaccept{
        \begin{itemize}
            \item Użytkownikowi wyświetlany jest przycisk otwierający formularz umożliwiający dodanie nowego
            komentarza do wybranego \glslink{spot}{spota}.
            \item Formularz zawiera następujące pola: ocena spota w gwiazdkach od 0 do 5,
            treść komentarza, dodanie filmów i/lub zdjęć (od 0 do 20).
            \item Po kliknięciu przycisku do publikacji komentarza, użytkownik widzi go na liście wszystkich komentarzy.
            \item Zamknięcie formularza nie powoduje zapisu danych.
            \item Brak zalogowania uniemożliwia otwarcie formularza, użytkownik jest o tym odpowiednio informowany.
            \item Wpisanie niepoprawnych danych do formularza blokuje możliwość publikacji komentarza, a użytkownik
            jest informowany o błędach.
        \end{itemize}
    }
    \wfmapstakeholder{U3}
    \wfmapresponsible{Stanisław Oziemczuk}
    \wfmaprelated{
        \hyperref[womap:spot-details]{WOMAPA-02},
        \hyperref[womap:spot-comment]{WOMAPA-05}
    }
}

\wfmapcard{wfmap:spot-media-add}
{Dodanie media do \glslink{spot}{spota}}
{07}
{S}
{
    \wfmapdesc{System umożliwia użytkownikowi dodanie media (zdjęć lub filmów) do \glslink{spot}{spota}.}
    \wfmapaccept{
        \begin{itemize}
            \item Użytkownikowi wyświetlany jest przycisk otwierający formularz umożliwiający dodanie nowego
            media do wybranego \glslink{spot}{spota}.
            \item Formularz zawiera przycisk do wyboru zdjęć i filmów z urządzenia.
            \item Wybrane pliki są wyświetlane w podglądzie wraz z możliwością ich usunięcia.
            \item Dodanie medii powoduje wyświetlenie ich w galerii mediów \glslink{spot}{spota} oraz w odpowiedniej
            zakładce profilu użytkownika.
            \item Zamknięcie formularza nie powoduje zapisu danych.
            \item Brak zalogowania uniemożliwia otwarcie formularza, użytkownik jest o tym odpowiednio informowany.
            \item Wybranie niepoprawnych formatów plików blokuje możliwość dodania medii, użytkownik
            jest informowany o błędach.
        \end{itemize}
    }
    \wfmapstakeholder{U3}
    \wfmapresponsible{Stanisław Oziemczuk}
    \wfmaprelated{
        \hyperref[womap:spot-details]{WOMAPA-02},
        \hyperref[womap:spot-media]{WOMAPA-06}
    }
}

\wfmapcard{wfmap:spot-favourites-add}
{Dodanie \glslink{spot}{spota} do listy ulubionych}
{08}
{S}
{
    \wfmapdesc{System umożliwia użytkownikowi dodanie \glslink{spot}{spota} do listy ulubionych.}
    \wfmapaccept{
        \begin{itemize}
            \item Użytkownikowi wyświetlany jest przycisk umożliwiający dodanie wybranego \glslink{spot}{spota}
            do listy ulubionych.
            \item Po kliknięciu przycisku użytkownik widzi \glslink{spot}{spota} w swojej liście
            ulubionych, przycisk zmienia wygląd oznaczający, że \glslink{spot}{spot} znajduje się na liście.
            \item Gdy \glslink{spot}{spot} znajduje się na liście ulubionych, kliknięcie przycisku powoduje
            usunięcie go z tej listy.
            \item Brak zalogowania uniemożliwia dodanie \glslink{spot}{spota} do listy ulubionych,
            użytkownik jest o tym odpowiednio informowany.
        \end{itemize}
    }
    \wfmapstakeholder{U3}
    \wfmapresponsible{Stanisław Oziemczuk}
    \wfmaprelated{\hyperref[womap:spot-details]{WOMAPA-02}}
}

\wfmapcard{wfmap:spot-share}
{Udostępnienie \glslink{spot}{spota}}
{09}
{S}
{
    \wfmapdesc{System umożliwia użytkownikowi udostępnienie \glslink{spot}{spota}.}
    \wfmapaccept{
        \begin{itemize}
            \item Użytkownikowi wyświetlany jest przycisk umożliwiający udostępnienie wybranego \glslink{spot}{spota}.
            \item Po kliknięciu przycisku do schowka kopiowany jest link do danego \glslink{spot}{spota}.
            \item Wklejenie linku do przeglądarki powoduje otworzenie panelu ze szczegółami \glslink{spot}{spota} oraz
            przybliżenie jego lokalizacji na mapie.
            \item Użytkownik jest informowany o błędach, które wystąpiły w trakcie udostępniania \glslink{spot}{spota}.
        \end{itemize}
    }
    \wfmapstakeholder{U3}
    \wfmapresponsible{Stanisław Oziemczuk}
    \wfmaprelated{\hyperref[womap:spot-details]{WOMAPA-02}}
}

\wfmapcard{wfmap:spot-naviagte}
{Nawigowanie do \glslink{spot}{spota}}
{10}
{S}
{
    \wfmapdesc{System pokazuje użytkownikowi trasę od jego lokalizacji do wybranego \glslink{spot}{spota}.}
    \wfmapaccept{
        \begin{itemize}
            \item Użytkownikowi wyświetlany jest przycisk umożliwiający pokazanie trasy do wybranego \glslink{spot}{spota}.
            \item Po kliknięciu przycisku pobierana jest aktualna lokalizacja użytkownika,
            a w przeglądarce otwierana jest nowa karta z Google Maps z trasą od pozycji użytkownika do danego
            \glslink{spot}{spota}.
            \item Użytkownik jest informowany o błędach, które wystąpiły w trakcie operacji.
        \end{itemize}
    }
    \wfmapstakeholder{U3}
    \wfmapresponsible{Stanisław Oziemczuk}
    \wfmaprelated{
        \hyperref[womap:spot-details]{WOMAPA-02},
        \hyperref[womap:spot-user-location]{WOMAPA-07}
    }
}

\wfmapcard{wfmap:spot-expanded-media-gallery}
{Duża galeria mediów \glslink{spot}{spota}}
{11}
{S}
{
    \wfmapdesc{System umożliwia użytkownikowi włączenie dużej galerii mediów wybranego \glslink{spot}{spota}.}
    \wfmapaccept{
        \begin{itemize}
            \item Po kliknięciu na zdjęcie w galerii mediów \glslink{spot}{spota}, użytkownikowi wyświetlana jest duża
            galeria mediów.
            Taka sama akcja jest wykonywana po kliknięciu zdjęcie lub film znajdujący się w komentarzu \glslink{spot}{spota}.
            \item W dużej galerii mediów wyświetlana jest lista wszystkich plików wybranego typu, przewijana przy
            użyciu \glslink{infinite-scroll}{nieskończonego przewijania}.
            \item Wyświetlalne jest powiększone obecnie wybrane media.
            \item Użytkownik jest informowany o błędach, które wystąpiły w trakcie operacji.
        \end{itemize}
    }
    \wfmapstakeholder{U3}
    \wfmapresponsible{Stanisław Oziemczuk}
    \wfmaprelated{
        \hyperref[womap:spot-details]{WOMAPA-02},
        \hyperref[womap:spot-media]{WOMAPA-06}
    }
}

\wfmapcard{wfmap:spot-expanded-media-gallery-sidebar}
{Zarządzanie listą medii w dużej galerii mediów \glslink{spot}{spota}}
{12}
{S}
{
    \wfmapdesc{System umożliwia użytkownikowi sortowanie oraz filtrowanie medii w dużej galerii wybranego \glslink{spot}{spota}.}
    \wfmapaccept{
        \begin{itemize}
            \item Użytkownikowi wyświetlane są przyciski do filtrowania listy po typie medii (zdjęcia lub filmy) oraz
            umożliwiające sortowanie po liczbie polubień czy dacie dodania.
            \item Po kliknięciu przycisku lista jest aktualizowana, a obecne wybrane media jest ustawiane na
            pierwszy element listy.
            \item Zaznaczone opcje filtrowania oraz sortowania są oznaczone jako wybrane.
            \item Domyślną opcją filtrowania po typie media jest ta, którą użytkownik kliknął włącząjąc dużą galerię mediów.
            \item Domyślnie sortowanie jest po dacie dodania media, od najnowszego.
            \item Użytkownik jest informowany o błędach, które wystąpiły w trakcie operacji.
        \end{itemize}
    }
    \wfmapstakeholder{U3}
    \wfmapresponsible{Stanisław Oziemczuk}
    \wfmaprelated{\hyperref[womap:spot-media]{WOMAPA-06}}
}

\wfmapcard{wfmap:spot-expanded-media-gallery-expanded-media}
{Wyświetlanie powiększonego obecnie wybranego media w dużej galerii mediów \glslink{spot}{spota}}
{13}
{S}
{
    \wfmapdesc{System wyświetla użytkownikowi powiększone obecnie wybrane media w dużej galerii wybranego \glslink{spot}{spota}.}
    \wfmapaccept{
        \begin{itemize}
            \item Użytkownikowi wyświetlane jest obecnie wybrane media, które jest powiększone.
            \item Powiększone media zawiera: informacje o autorze, przyciski do udostępnienia, pobrania, polubienia,
            powiększenia media na cały ekran oraz liczbę polubień.
            \item Użytkownik jest informowany o błędach, które wystąpiły w trakcie operacji.
        \end{itemize}
    }
    \wfmapstakeholder{U3}
    \wfmapresponsible{Stanisław Oziemczuk}
    \wfmaprelated{\hyperref[womap:spot-media]{WOMAPA-06}}
}

\wfmapcard{wfmap:spot-expanded-media-gallery-share-media}
{Udostępnienie obecnie wybranego media w dużej galerii mediów \glslink{spot}{spota}}
{14}
{S}
{
    \wfmapdesc{System umożliwia użytkownikowi udostępnienie obecnie wybranego media w dużej galerii wybranego \glslink{spot}{spota}.}
    \wfmapaccept{
        \begin{itemize}
            \item Użytkownikowi wyświetlany jest przycisk umożliwiający udostępnienie obecnie wybranego media.
            \item Po kliknięciu przycisku do schowka kopiowany jest link do wybranego media.
            \item Wklejenie linku do przeglądarki powoduje pobranie media.
            \item Użytkownik jest informowany o błędach, które wystąpiły w trakcie operacji.
        \end{itemize}
    }
    \wfmapstakeholder{U3}
    \wfmapresponsible{Stanisław Oziemczuk}
    \wfmaprelated{\hyperref[womap:spot-media]{WOMAPA-06}}
}

\wfmapcard{wfmap:spot-expanded-media-gallery-like-media}
{Polubienie obecnie wybranego media w dużej galerii mediów \glslink{spot}{spota}}
{15}
{S}
{
    \wfmapdesc{System umożliwia użytkownikowi polubienie obecnie wybranego media w dużej galerii wybranego \glslink{spot}{spota}.}
    \wfmapaccept{
        \begin{itemize}
            \item Użytkownikowi wyświetlany jest przycisk umożliwiający polubienie obecnie wybranego media.
            \item Po kliknięciu przycisku użytkownik natychmiast widzi aktualizację liczby polubień, a przycisk
            zmienia wygląd oznaczający wykonanie operacji.
            \item Gdy media jest polubione, kliknięcie przycisku powoduje odwrotną operację.
            \item Brak zalogowania uniemożliwia wykonanie operacji.
            \item Użytkownik jest informowany o błędach, które wystąpiły w trakcie operacji.
        \end{itemize}
    }
    \wfmapstakeholder{U3}
    \wfmapresponsible{Stanisław Oziemczuk}
    \wfmaprelated{\hyperref[womap:spot-media]{WOMAPA-06}}
}

\wfmapcard{wfmap:spot-expanded-media-gallery-expand-media}
{Powiększenie na cały ekran obecnie wybranego media w dużej galerii mediów \glslink{spot}{spota}}
{16}
{S}
{
    \wfmapdesc{System umożliwia użytkownikowi powiększenie na cały ekran obecnie wybranego media w dużej galerii wybranego \glslink{spot}{spota}.}
    \wfmapaccept{
        \begin{itemize}
            \item Użytkownikowi wyświetlany jest przycisk umożliwiający powiększenie na cały ekran obecnie wybranego media.
            \item Po kliknięciu przycisku wybrane media jest powiększone na całą szerokość i wysokość ekranu.
            \item Powiększenie elementu nie powoduje zaburzenia jego proporcji.
        \end{itemize}
    }
    \wfmapstakeholder{U3}
    \wfmapresponsible{Stanisław Oziemczuk}
    \wfmaprelated{\hyperref[womap:spot-media]{WOMAPA-06}}
}

\wfmapcard{wfmap:spot-expanded-media-gallery-download-media}
{Pobranie obecnie wybranego media w dużej galerii mediów \glslink{spot}{spota}}
{17}
{S}
{
    \wfmapdesc{System umożliwia użytkownikowi pobranie obecnie wybranego media w dużej galerii wybranego \glslink{spot}{spota}.}
    \wfmapaccept{
        \begin{itemize}
            \item Użytkownikowi wyświetlany jest przycisk umożliwiający pobranie obecnie wybranego media.
            \item Po kliknięciu przycisku plik zostaje pobrany na urządzenie użytkownika.
            \item Użytkownik jest informowany o błędach, które wystąpiły w trakcie operacji.
        \end{itemize}
    }
    \wfmapstakeholder{U3}
    \wfmapresponsible{Stanisław Oziemczuk}
    \wfmaprelated{\hyperref[womap:spot-media]{WOMAPA-06}}
}

\wfmapcard{wfmap:spot-weather}
{Wyświetlanie informacji pogodowych \glslink{spot}{spota}}
{18}
{S}
{
    \wfmapdesc{System wyświetla użytkownikowi obecne dane pogodowe wybranego \glslink{spot}{spota}.}
    \wfmapaccept{
        \begin{itemize}
            \item Po kliknięciu \glslink{spot}{spota} użytkownikowi wyświetlane są obecne dane pogodowe zawierające:
            temperaturę, prędkość wiatru, ikonę symbolizującą stan pogody (np. zachmurzenie) oraz przycisk do
            otworzenia panelu ze szczegółowymi danymi.
            \item Użytkownik jest informowany o błędach, które wystąpiły w podczas pobierania danych.
        \end{itemize}
    }
    \wfmapstakeholder{U3}
    \wfmapresponsible{Stanisław Oziemczuk}
    \wfmaprelated{
        \hyperref[womap:display-spots]{WOMAPA-01},
        \hyperref[womap:spot-weather]{WOMAPA-03}
    }
}

\wfmapcard{wfmap:spot-detailed-weather}
{Wyświetlanie szczegółowych informacji pogodowych \glslink{spot}{spota}}
{19}
{S}
{
    \wfmapdesc{System wyświetla użytkownikowi szczegółowe dane pogodowe wybranego \glslink{spot}{spota}.}
    \wfmapaccept{
        \begin{itemize}
            \item Użytkownikowi wyświetlane są szczegółowe informacje pogodowe o wybranym \glslink{spot}{spocie}.
            \item Dane zawierają:
            \begin{itemize}
                \item obecną godzinę i temperaturę
                \item ikonę symbolizującą stan pogody wraz z opisem
                \item wskaźnik UV
                \item punkt rosy
                \item wilgotność
                \item prawdopodobieństwo opadów
                \item prędkości wiatrów na różnych wysokościach do wyboru z możliwością określenia jednostki km/h lub m/s
                \item wykres zawierający prognozę pogody na 3 kolejne dni
            \end{itemize}
            \item Użytkownik jest informowany o błędach, które wystąpiły podczas pobierania danych.
        \end{itemize}
    }
    \wfmapstakeholder{U3}
    \wfmapresponsible{Stanisław Oziemczuk}
    \wfmaprelated{\hyperref[womap:spot-weather]{WOMAPA-03}}
}

\wfmapcard{wfmap:spot-name-search}
{Wyszukiwanie \glslink{spot}{spotów} po nazwie}
{20}
{S}
{
    \wfmapdesc{System umożliwia użytkownikowi wyszukiwanie \glslink{spot}{spotów} po nazwie.}
    \wfmapaccept{
        \begin{itemize}
            \item Wyświetlane jest pole, w które użytkownik może wpisać frazę.
            \item Gdy dane są wpisywane, wyświetlana jest lista nazw \glslink{spot}{spotów}, które zawierają w sobie
            podany fragment.
            \item Użytkownik może wybrać pozycję z listy, powoduje to wstawienie jej w pole wyszukiwania i pokazanie wyników.
            \item Użytkownik może nie wybrać pozycji z listy i kliknąć przycisk do pokazania wyników, wyświetlana jest lista
            wszystkich \glslink{spot}{spotów}, których nazwy zawierają w sobie podaną frazę.
            \item Lista z wynikami wyszukiwania jest przewijana za pomocą \glslink{infinite-scroll}{nieskończonego przewijania}.
            \item W przypadku braku pasujących nazw \glslink{spot}{spotów} wyświetlany jest odpowiedni komunikat.
            \item Użytkownik jest informowany o błędach, które wystąpiły w podczas pobierania danych.
            \item Użytkownik ma możliwość wyczyszczenia pola do wyszukiwania i zamknięcia listy z wynikami.
        \end{itemize}
    }
    \wfmapstakeholder{U3}
    \wfmapresponsible{Stanisław Oziemczuk}
    \wfmaprelated{
        \hyperref[womap:display-spots]{WOMAPA-01},
        \hyperref[womap:spot-search]{WOMAPA-04}
    }
}

\wfmapcard{wfmap:spot-name-search-list}
{Sortowanie wyników wyszukiwania \glslink{spot}{spotów} po nazwie}
{21}
{S}
{
    \wfmapdesc{System umożliwia użytkownikowi sortowanie wyników zawierającą wyszukiwania
    \glslink{spot}{spotów} po nazwie.}
    \wfmapaccept{
        \begin{itemize}
            \item Użytkownikowi wyświetlane są przyciski umożliwiające ustawienie sortowania wyników po:
            \begin{itemize}
                \item ocenach rosnąco
                \item ocenach malejąco
                \item liczbie ocen rosnąco
                \item liczbie ocen malejąco
            \end{itemize}.
            \item Po wybraniu opcji sortowania, lista jest natychmiast aktualizowana.
            \item Użytkownik jest informowany o błędach, które wystąpiły w podczas pobierania danych.
        \end{itemize}
    }
    \wfmapstakeholder{U3}
    \wfmapresponsible{Stanisław Oziemczuk}
    \wfmaprelated{\hyperref[womap:spot-search]{WOMAPA-04}}
}

\wfmapcard{wfmap:spot-current-view}
{Sortowanie wyników wyszukiwania \glslink{spot}{spotów} po nazwie}
{22}
{S}
{
    \wfmapdesc{System umożliwia użytkownikowi wyświetlenie listy \glslink{spot}{spotów} znajdujących się w widocznym
    obszarze mapy.}
    \wfmapaccept{
        \begin{itemize}
            \item Użytkownikowi wyświetlany jest przycisk powodujący wyświetlenie listy wszystkich \glslink{spot}{spotów}
            znajdujących się w widocznym obszarze mapy.
            \item Po kliknięciu przycisku wyświetlana lista jest przewijalna przy użyciu \glslink{infinite-scroll}{nieskończonego przewijania}.
            \item Zmienienie obszaru widocznego bez ponownego kliknięcia przycisku nie powoduje pobrania nowych danych.
            \item Ponowne kliknięcie przycisku powoduje pobranie nowych danych i zastąpienie nimi poprzednich wyników.
            \item Gdy w widocznym obszarze mapy nie ma \glslink{spot}{spotów}, wyświetlany jest odpowiedni komunikat.
            \item Użytkownik jest informowany o błędach, które wystąpiły w podczas pobierania danych.
        \end{itemize}
    }
    \wfmapstakeholder{U3}
    \wfmapresponsible{Stanisław Oziemczuk}
    \wfmaprelated{
        \hyperref[womap:display-spots]{WOMAPA-01},
        \hyperref[womap:spot-search]{WOMAPA-04}
    }
}

\wfmapcard{wfmap:spot-name-current-view-list}
{Zarządzanie listą wyników wyszukiwania \glslink{spot}{spotów} w widocznym obszarze mapy}
{23}
{S}
{
    \wfmapdesc{System umożliwia użytkownikowi sortowanie oraz filtrowanie wyników zawierającą \glslink{spot}{spoty}
    znajdujące się w widocznym obszarze mapy.}
    \wfmapaccept{
        \begin{itemize}
            \item Użytkownikowi wyświetlane są przyciski umożliwiające ustawienie sortowania wyników po:
            \begin{itemize}
                \item ocenach rosnąco
                \item ocenach malejąco
                \item liczbie ocen rosnąco
                \item liczbie ocen malejąco
            \end{itemize}.
            \item Po wybraniu opcji sortowania, lista jest natychmiast aktualizowana.
            \item Użytkownik może filtrować \glslink{spot}{spoty} ustawiająć minimalną ocenę.
            \item Użytkownik może filtrować wyniki poprzez wpisanie frazy.
            Podczas wpisywania wyświetlane są podpowiedzi w postaci listy nazw spotów zawierających wpisywany tekst.
            Wynikiem filtrowania są \glslink{spot}{spoty}, których nazwy zawierają w sobie wpisaną frazę.
            \item Po ustawieniu filtrowania lista jest aktualizowana, brane są pod uwagę wszystkie filtry i sortowanie.
            \item W przypadku braku pasujących wyników wyświetlany jest odpowiedni komunikat.
            \item Użytkownik jest informowany o błędach, które wystąpiły w podczas pobierania danych.
        \end{itemize}
    }
    \wfmapstakeholder{U3}
    \wfmapresponsible{Stanisław Oziemczuk}
    \wfmaprelated{\hyperref[womap:spot-search]{WOMAPA-04}}
}

\wfmapcard{wfmap:spot-lists-map-view}
{Przybliżanie mapy do lokalizacji \glslink{spot}{spotów} z listy z wynikami}
{24}
{S}
{
    \wfmapdesc{System umożliwia użytkownikowi przybliżenie widoku mapy do lokalizacji \glslink{spot}{spota},
        będące na liście wyników wyszukiwania.}
    \wfmapaccept{
        \begin{itemize}
            \item Po kliknięciu \glslink{spot}{spota} na liście wyników wyszukiwania, widok mapy
            przenoszony jest do jego lokalizacji.
        \end{itemize}
    }
    \wfmapstakeholder{U3}
    \wfmapresponsible{Stanisław Oziemczuk}
    \wfmaprelated{
        \hyperref[womap:display-spots]{WOMAPA-01},
        \hyperref[womap:spot-search]{WOMAPA-04}
    }
}

\wfmapcard{wfmap:user-location}
{Wyświetlanie na mapie lokalizacji użytkownika}
{25}
{S}
{
    \wfmapdesc{System umożliwia użytkownikowi wyświetlenie na mapie jego obecnej lokalizacji.}
    \wfmapaccept{
        \begin{itemize}
            \item Użytkownikowi wyświetlany jest przycisk umożliwiający zaznaczenie na mapie jego obecnej lokalizacji.
            \item Po kliknięciu przycisku pobierane są dane o lokalizacji użytkownika i jego pozycja jest zaznaczona na mapie.
            \item Widok mapy jest przybliżany na wyznaczoną lokalizację.
            \item Użytkownik jest informowany o błędach, które wystąpiły w podczas operacji.
        \end{itemize}
    }
    \wfmapstakeholder{U3}
    \wfmapresponsible{Stanisław Oziemczuk}
    \wfmaprelated{
        \hyperref[womap:display-spots]{WOMAPA-01},
        \hyperref[womap:spot-user-location]{WOMAPA-07}
    }
}

\wfmapcard{wfmap:map-zoom}
{Ustawianie przybliżenia mapy}
{26}
{S}
{
    \wfmapdesc{System umożliwia użytkownikowi zmianę przybliżenia widoku mapy.}
    \wfmapaccept{
        \begin{itemize}
            \item Użytkownikowi wyświetlane są przyciski do przybliżania i oddalania widoku mapy.
            \item Po kliknięciu jednego z przycisków wykonywana jest odpowiednia akcja.
            \item Przejście między przybliżeniami jest płynne i w razie potrzeby ładowane są kolejne kafelki mapy.
        \end{itemize}
    }
    \wfmapstakeholder{U3}
    \wfmapresponsible{Stanisław Oziemczuk}
    \wfmaprelated{\hyperref[womap:display-spots]{WOMAPA-01}}
}
