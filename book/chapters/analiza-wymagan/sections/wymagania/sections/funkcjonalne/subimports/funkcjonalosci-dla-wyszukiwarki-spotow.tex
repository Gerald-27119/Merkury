%! Author = Mateusz
%! Date = 03/11/2025

\subsubsection{Wymagania funkcjonalne dla wyszukiwarki spotów}
\label{subsubsec:wymagania-funkcjonalne-dla-wyszukiwarki-spotow}

\newlength{\wfwyszLabelWidth}
\setlength{\wfwyszLabelWidth}{0.19\textwidth}

\newlength{\wfwyszColTwoWidth}
\setlength{\wfwyszColTwoWidth}{0.21\textwidth}

\newlength{\wfwyszColThreeWidth}
\setlength{\wfwyszColThreeWidth}{0.13\textwidth}

\newlength{\wfwyszColFourWidth}
\setlength{\wfwyszColFourWidth}{0.28\textwidth}

\newlength{\wfwyszContentWidth}
\setlength{\wfwyszContentWidth}{0.60\textwidth}

\newlength{\wfwyszHeaderHeight}
\setlength{\wfwyszHeaderHeight}{12mm}

\newcommand{\wfwyszthreecolcell}[1]{%
    \multicolumn{3}{|>{\raggedright\arraybackslash}p{\wfwyszContentWidth}|}{#1}%
}

\newcommand{\wfwyszthreecolcellpadded}[1]{%
    \multicolumn{3}{|>{\raggedright\arraybackslash}p{\wfwyszContentWidth}|}{%
        \vspace{0.4ex}%
        #1\par\vspace{0.4ex}%
    }%
}

\newcommand{\wfwyszHeaderRow}[1]{%
    \rowcolor{lightgray}%
    \multicolumn{4}{|c|}{%
        \parbox[c][\wfwyszHeaderHeight][c]{\linewidth}{%
            \centering\bfseries
            \vspace{1.2ex}%
            #1%
            \vspace{1.2ex}%
        }%
    }\\ \hline
}

% --- pola karty ---

\newcommand{\wfwyszpriority}[2]{%
    \textbf{Identyfikator:} & WFWYSZ-#1 &
    \textbf{Priorytet:}     & #2 \\ \hline
}

\newcommand{\wfwyszname}[1]{%
    \textbf{Nazwa:} &
    \wfwyszthreecolcell{#1} \\ \hline
}

\newcommand{\wfwyszdesc}[1]{%
    \textbf{Opis:} &
    \wfwyszthreecolcell{#1} \\ \hline
}

\newcommand{\wfwyszaccept}[1]{%
    \textbf{Kryteria akceptacji:} &
    \multicolumn{3}{|>{\raggedright\arraybackslash}p{\wfwyszContentWidth}|}{%
        \begingroup
        \setlength{\leftmargini}{1.2em}%
        \setlength{\topsep}{0pt}%
        \setlength{\partopsep}{0pt}%
        \setlength{\itemsep}{0.2ex}%
        \setlength{\parsep}{0pt}%
        \vspace*{-1.8ex}
        #1%
        \vspace*{-1.4ex}
        \endgroup
    }\\ \hline
}

\newcommand{\wfwyszstakeholder}[1]{%
    \textbf{Udziałowiec:} &
    \wfwyszthreecolcell{#1} \\ \hline
}

\newcommand{\wfwyszresponsible}[1]{%
    \textbf{Realizator:} &
    \wfwyszthreecolcell{#1} \\ \hline
}

\newcommand{\wfwyszrelated}[1]{%
    \textbf{Wymagania powiązane:} &
    \wfwyszthreecolcell{#1} \\ \hline
}

% --- szablon karty wymagania funkcjonalnego ---

\newcommand{\wfwyszatcard}[5]{%
    \refstepcounter{awc}%
    {%
        \centering
        \begin{longtable}{|
                >{\columncolor{lightgray}\raggedright\arraybackslash}p{\wfwyszLabelWidth}|
            p{\wfwyszColTwoWidth}|
                >{\columncolor{lightgray}\raggedright\arraybackslash}p{\wfwyszColThreeWidth}|
            p{\wfwyszColFourWidth}|}
        \hline
        \wfwyszHeaderRow{\shortstack{KARTA WYMAGANIA FUNKCJONALNEGO DLA \\ WYSZUKIWARKI SPOTÓW}}
        \endfirsthead
        \hline
        \wfwyszHeaderRow{\shortstack{KARTA WYMAGANIA FUNKCJONALNEGO DLA \\ WYSZUKIWARKI SPOTÓW (cd.)}}
        \endhead
        \wfwyszpriority{#3}{#4}
        \wfwyszname{#2}
        #5
        \end{longtable}
        \par
    }%
    \vspace{3pt}%
    \textbf{Tabela \theawc:} Wymaganie funkcjonalne dla wyszukiwarki spotów: #2\label{#1}%
    \addcontentsline{lot}{table}{Tabela \theawc: Wymaganie funkcjonalne dla wyszukiwarki spotów: #2}%
}


% =========================================
% WFWYSZ-01 – Najpopularniejsze spoty w karuzeli
% =========================================
\wfwyszatcard
{wfwysz:top-spots-carousel}
{Wyświetlenie najpopularniejszych spotów w karuzeli}
{01}
{M}
{
    \wfwyszdesc{System prezentuje na stronie głównej karuzelę z najpopularniejszymi spotami,
        aby użytkownik mógł szybko przeglądać rekomendowane miejsca dla wybranej lokalizacji.}

    \wfwyszaccept{%
        \begin{itemize}
            \item Karuzela wyświetla nazwę spota oraz miasto.
            \item Każda pozycja w karuzeli zawiera zdjęcie (miniaturę) spota.
            \item Karuzela jest widoczna po wejściu na stronę główną bez konieczności wykonywania dodatkowych akcji.
        \end{itemize}
    }

    \wfwyszstakeholder{U3}
    \wfwyszresponsible{Mateusz Redosz}
    \wfwyszrelated{%
        \hyperref[wowysz:basic-search]{WOWYSZ-01},
        \hyperref[wpwysz:load-under-10s]{WPWYSZ-01}.%
    }
}

% =========================================
% WFWYSZ-02 – Proste wyszukiwanie po lokalizacji
% =========================================
\wfwyszatcard
{wfwysz:basic-location-search}
{Wyszukiwanie za pomocą państwa, regionu oraz miasta}
{02}
{M}
{
    \wfwyszdesc{System umożliwia użytkownikowi filtrowanie listy spotów według lokalizacji,
        poprzez wybór państwa, regionu oraz miasta. Zastosowanie filtrów następuje po wybraniu
        akcji wyszukiwania.}

    \wfwyszaccept{%
        \begin{itemize}
            \item Użytkownik może wskazać państwo, region oraz miasto jako filtry wyszukiwania.
            \item Zmiana wartości filtrów nie powoduje automatycznego odświeżenia wyników.
            \item Po kliknięciu przycisku wyszukiwania lista spotów jest aktualizowana zgodnie z wybraną lokalizacją.
        \end{itemize}
    }

    \wfwyszstakeholder{U3}
    \wfwyszresponsible{Mateusz Redosz}
    \wfwyszrelated{%
        \hyperref[wowysz:basic-search]{WOWYSZ-01},
        \hyperref[wfwysz:location-autocomplete]{WFWYSZ-04},
        \hyperref[wfwysz:display-search-results]{WFWYSZ-03},
        \hyperref[wpwysz:load-under-10s]{WPWYSZ-01}.%
    }
}

% =========================================
% WFWYSZ-03 – Wyświetlenie wyników wyszukiwania
% =========================================
\wfwyszatcard
{wfwysz:display-search-results}
{Wyświetlenie wyszukanych spotów}
{03}
{M}
{
    \wfwyszdesc{System wyświetla listę spotów spełniających kryteria wyszukiwania,
        aby użytkownik mógł przeglądać dostępne wyniki.}

    \wfwyszaccept{%
        \begin{itemize}
            \item Wyniki wyszukiwania są prezentowane w formie listy (kart spotów).
            \item Każda karta spota zawiera:
            \begin{itemize}
                \item nazwę spota,
                \item zdjęcie,
                \item tagi,
                \item średnią ocenę oraz liczbę ocen,
                \item informacje pogodowe dla lokalizacji spota,
                \item lokalizację (miasto).
            \end{itemize}
        \end{itemize}
    }

    \wfwyszstakeholder{U3}
    \wfwyszresponsible{Mateusz Redosz}
    \wfwyszrelated{%
        \hyperref[wowysz:basic-search]{WOWYSZ-01},
        \hyperref[wowysz:advanced-search]{WOWYSZ-02},
        \hyperref[wfwysz:basic-location-search]{WFWYSZ-02},
        \hyperref[wfwysz:advanced-city-tags]{WFWYSZ-06},
        \hyperref[wfwysz:spot-actions-map-details]{WFWYSZ-05},
        \hyperref[wpwysz:load-under-10s]{WPWYSZ-01}.%
    }
}

% =========================================
% WFWYSZ-04 – Podpowiedzi wartości w polach lokalizacji (autocomplete)
% =========================================
\wfwyszatcard
{wfwysz:location-autocomplete}
{Podpowiedzi wartości w polach lokalizacji (autocomplete)}
{04}
{M}
{
    \wfwyszdesc{System wspiera użytkownika podczas uzupełniania filtrów lokalizacji,
        wyświetlając listę podpowiedzi dla państwa, regionu oraz miasta na podstawie wpisywanej frazy.}

    \wfwyszaccept{%
        \begin{itemize}
            \item Podczas wpisywania w pole państwa, regionu lub miasta system wyświetla listę podpowiedzi pasujących do wpisanej frazy.
            \item Lista podpowiedzi zawęża się wraz z dopisywaniem kolejnych znaków.
            \item Użytkownik może wybrać wartość z listy podpowiedzi (kliknięciem), co uzupełnia pole wyszukiwania.
        \end{itemize}
    }

    \wfwyszstakeholder{U3}
    \wfwyszresponsible{Mateusz Redosz}
    \wfwyszrelated{%
        \hyperref[wowysz:basic-search]{WOWYSZ-01},
        \hyperref[wowysz:advanced-search]{WOWYSZ-02},
        \hyperref[wfwysz:basic-location-search]{WFWYSZ-02},
        \hyperref[wpwysz:autocomplete-location]{WPWYSZ-02}.%
    }
}


% =========================================
% WFWYSZ-05 – Akcje na spocie z listy wyników
% =========================================
\wfwyszatcard
{wfwysz:spot-actions-map-details}
{Przycisk do pokazania spota na mapie oraz zobaczenia jego szczegółów}
{05}
{M}
{
    \wfwyszdesc{System udostępnia w wynikach wyszukiwania akcje umożliwiające użytkownikowi
    przejście do szczegółów spota oraz wyświetlenie go na mapie.}

    \wfwyszaccept{%
        \begin{itemize}
            \item Dla każdego spota na liście wyników dostępna jest akcja „Details”.
            \item Dla każdego spota na liście wyników dostępna jest akcja „See on map”.
            \item Kliknięcie akcji „See on map” przenosi użytkownika do widoku mapy z zaznaczonym spotem.
            \item Kliknięcie akcji „Details” przenosi użytkownika do widoku mapy i otwiera szczegóły spota.
        \end{itemize}
    }

    \wfwyszstakeholder{U3}
    \wfwyszresponsible{Mateusz Redosz}
    \wfwyszrelated{%
        \hyperref[wowysz:basic-search]{WOWYSZ-01},
        \hyperref[wowysz:advanced-search]{WOWYSZ-02},
        \hyperref[wfwysz:display-search-results]{WFWYSZ-03}.%
    }
}

% =========================================
% WFWYSZ-06 – Zaawansowane wyszukiwanie: miasto + tagi
% =========================================
\wfwyszatcard
{wfwysz:advanced-city-tags}
{Wyszukiwanie za pomocą miasta oraz tagów}
{06}
{M}
{
    \wfwyszdesc{System umożliwia użytkownikowi zawężenie wyników wyszukiwania poprzez
    wybór miasta oraz tagów opisujących spot. Zastosowanie filtrów następuje po ponownym
    uruchomieniu wyszukiwania.}

    \wfwyszaccept{%
        \begin{itemize}
            \item Użytkownik może wskazać miasto jako kryterium wyszukiwania.
            \item Użytkownik może wybrać jeden lub wiele tagów jako filtr.
            \item Zmiana miasta nie powoduje automatycznego odświeżenia wyników.
            \item Zmiana tagów powoduje automatyczne odświeżenie wyników.
            \item Po ponownym uruchomieniu wyszukiwania (kliknięciu przycisku wyszukiwania) lista wyników jest aktualizowana zgodnie z wybranymi filtrami.
        \end{itemize}
    }

    \wfwyszstakeholder{U3}
    \wfwyszresponsible{Mateusz Redosz}
    \wfwyszrelated{%
        \hyperref[wowysz:advanced-search]{WOWYSZ-02},
        \hyperref[wfwysz:display-search-results]{WFWYSZ-03},
        \hyperref[wfwysz:location-autocomplete]{WFWYSZ-04},
        \hyperref[wfwysz:filter-rating]{WFWYSZ-08},
        \hyperref[wfwysz:filter-polarity-rating]{WFWYSZ-07},
        \hyperref[wpwysz:load-under-10s]{WPWYSZ-01},
        \hyperref[wpwysz:autocomplete-location]{WPWYSZ-02}.%
    }
}

% =========================================
% WFWYSZ-07 – Zaawansowane filtrowanie po polarności i ocenie
% =========================================
\wfwyszatcard
{wfwysz:filter-polarity-rating}
{Filtrowanie po polarności oraz ocenie}
{07}
{M}
{
    \wfwyszdesc{System umożliwia użytkownikowi filtrowanie listy spotów według polaryzacji
        (popularność / kryterium sortowania lub typ prezentacji) oraz minimalnej oceny spota.}

    \wfwyszaccept{%
        \begin{itemize}
            \item Użytkownik może ustawić kryterium związane z polaryzacją (zgodnie z dostępnymi opcjami w interfejsie).
            \item Użytkownik może wskazać minimalną ocenę spota jako filtr.
            \item Po zastosowaniu filtrów lista wyników jest aktualizowana zgodnie z ustawieniami.
        \end{itemize}
    }

    \wfwyszstakeholder{U3}
    \wfwyszresponsible{Mateusz Redosz}
    \wfwyszrelated{%
        \hyperref[wowysz:advanced-search]{WOWYSZ-02},
        \hyperref[wfwysz:display-search-results]{WFWYSZ-03},
        \hyperref[wfwysz:filter-rating]{WFWYSZ-08},
        \hyperref[wpwysz:load-under-10s]{WPWYSZ-01}.%
    }
}

% =========================================
% WFWYSZ-08 – Filtrowanie po ocenie
% =========================================
\wfwyszatcard
{wfwysz:filter-rating}
{Filtrowanie po ocenie}
{08}
{M}
{
    \wfwyszdesc{System umożliwia użytkownikowi filtrowanie wyników wyszukiwania według minimalnej oceny spota,
        aby szybciej odnajdować miejsca o wysokiej jakości.}

    \wfwyszaccept{%
        \begin{itemize}
            \item Użytkownik może ustawić minimalną ocenę (wybór wartości).
            \item Lista wyników po ustawieniu filtra prezentuje wyłącznie spoty spełniające warunek minimalnej oceny.
        \end{itemize}
    }

    \wfwyszstakeholder{U3}
    \wfwyszresponsible{Mateusz Redosz}
    \wfwyszrelated{%
        \hyperref[wowysz:advanced-search]{WOWYSZ-02},
        \hyperref[wfwysz:display-search-results]{WFWYSZ-03},
        \hyperref[wpwysz:load-under-10s]{WPWYSZ-01}.%
    }
}
