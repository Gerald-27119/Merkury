%! Author = Mateusz
%! Date = 01/11/2025

\subsubsection{Wymagania funkcjonalne dla panelu użytkownika}
\label{subsubsec:wymagania-funkcjonalne-dla-panelu-uzytkownika}

\newlength{\wfpanelLabelWidth}
\setlength{\wfpanelLabelWidth}{0.19\textwidth}

\newlength{\wfpanelColTwoWidth}
\setlength{\wfpanelColTwoWidth}{0.21\textwidth}

\newlength{\wfpanelColThreeWidth}
\setlength{\wfpanelColThreeWidth}{0.13\textwidth}

\newlength{\wfpanelColFourWidth}
\setlength{\wfpanelColFourWidth}{0.28\textwidth}

\newlength{\wfpanelContentWidth}
\setlength{\wfpanelContentWidth}{0.60\textwidth}

\newlength{\wfpanelHeaderHeight}
\setlength{\wfpanelHeaderHeight}{12mm}

\newcommand{\wfpanelthreecolcell}[1]{%
    \multicolumn{3}{|>{\raggedright\arraybackslash}p{\wfpanelContentWidth}|}{#1}%
}

\newcommand{\wfpanelthreecolcellpadded}[1]{%
    \multicolumn{3}{|>{\raggedright\arraybackslash}p{\wfpanelContentWidth}|}{%
        \vspace{0.4ex}%
        #1\par\vspace{0.4ex}%
    }%
}

\newcommand{\wfpanelHeaderRow}[1]{%
    \rowcolor{lightgray}%
    \multicolumn{4}{|c|}{%
        \parbox[c][\wfpanelHeaderHeight][c]{\linewidth}{%
            \centering\bfseries
            \vspace{1.2ex}%
            #1%
            \vspace{1.2ex}%
        }%
    }\\ \hline
}

% --- pola karty ---

\newcommand{\wfpanelpriority}[2]{%
    \textbf{Identyfikator:} & WFPANEL-#1 &
    \textbf{Priorytet:}     & #2 \\ \hline
}

\newcommand{\wfpanelname}[1]{%
    \textbf{Nazwa:} &
    \wfpanelthreecolcell{#1} \\ \hline
}

\newcommand{\wfpaneldesc}[1]{%
    \textbf{Opis:} &
    \wfpanelthreecolcell{#1} \\ \hline
}

\newcommand{\wfpanelaccept}[1]{%
    \textbf{Kryteria akceptacji:} &
    \multicolumn{3}{|>{\raggedright\arraybackslash}p{\wfpanelContentWidth}|}{%
        \begingroup
        \setlength{\leftmargini}{1.2em}%
        \setlength{\topsep}{0pt}%
        \setlength{\partopsep}{0pt}%
        \setlength{\itemsep}{0.2ex}%
        \setlength{\parsep}{0pt}%
        \vspace*{-1.8ex}
        #1%
        \vspace*{-1.4ex}
        \endgroup
    }\\ \hline
}

\newcommand{\wfpanelstakeholder}[1]{%
    \textbf{Udziałowiec:} &
    \wfpanelthreecolcell{#1} \\ \hline
}

\newcommand{\wfpanelresponsible}[1]{%
    \textbf{Realizator:} &
    \wfpanelthreecolcell{#1} \\ \hline
}

\newcommand{\wfpanelrelated}[1]{%
    \textbf{Wymagania powiązane:} &
    \wfpanelthreecolcell{#1} \\ \hline
}

% --- szablon karty wymagania funkcjonalnego ---

\newcommand{\wfpanelatcard}[5]{%
    \refstepcounter{awc}%
    {%
        \centering
        \begin{longtable}{|
                >{\columncolor{lightgray}\raggedright\arraybackslash}p{\wfpanelLabelWidth}|
            p{\wfpanelColTwoWidth}|
                >{\columncolor{lightgray}\raggedright\arraybackslash}p{\wfpanelColThreeWidth}|
            p{\wfpanelColFourWidth}|}
        \hline
        \wfpanelHeaderRow{\shortstack{KARTA WYMAGANIA FUNKCJONALNEGO DLA \\ PANELU UŻYTKOWNIKA}}
        \endfirsthead
        \hline
        \wfpanelHeaderRow{\shortstack{KARTA WYMAGANIA FUNKCJONALNEGO DLA \\ PANELU UŻYTKOWNIKA (cd.)}}
        \endhead
        \wfpanelpriority{#3}{#4}
        \wfpanelname{#2}
        #5
        \end{longtable}
        \par
    }%
    \vspace{3pt}%
    \textbf{Tabela \theawc:} Wymaganie funkcjonalne dla panelu użytkownika: #2\label{#1}%
    \addcontentsline{lot}{table}{Tabela \theawc: Wymaganie funkcjonalne dla panelu użytkownika: #2}%
}


% =========================================
% WFPANEL-01 – Wyświetlanie podstawowych danych profilu
% =========================================
\wfpanelatcard
{wfpanel:display-basic-profile}
{Wyświetlanie podstawowych danych o użytkowniku}
{01}
{M}
{
    \wfpaneldesc{System umożliwia użytkownikowi wyświetlenie podstawowych danych profilu w panelu konta.}

    \wfpanelaccept{%
        \begin{itemize}
            \item Użytkownik widzi podstawowe informacje o koncie (nazwa użytkownika oraz avatar).
            \item Dane są pobierane z \glslink{api}{API} i prezentowane po wejściu na podstronę profilu.
        \end{itemize}
    }

    \wfpanelstakeholder{U3}
    \wfpanelresponsible{Mateusz Redosz}
    \wfpanelrelated{%
        \hyperref[wopanu:profile]{WOPANEL-01},
        \hyperref[wppanel:login-required]{WPPANEL-07},
        \hyperref[wppanel:load-under-10s]{WPPANEL-11}.%
    }
}

% =========================================
% WFPANEL-02 – Zmiana zdjęcia profilowego
% =========================================
\wfpanelatcard
{wfpanel:change-avatar}
{Możliwość zmiany zdjęcia profilowego}
{02}
{S}
{
    \wfpaneldesc{System umożliwia użytkownikowi zmianę zdjęcia profilowego w panelu konta.}

    \wfpanelaccept{%
        \begin{itemize}
            \item Użytkownik może wybrać zdjęcie i zapisać jako nowy avatar.
            \item Po zapisaniu nowy avatar jest widoczny na profilu.
            \item W przypadku błędu zapisu lub niepoprawnego pliku system wyświetla odpowiedni komunikat.
        \end{itemize}
    }

    \wfpanelstakeholder{U3}
    \wfpanelresponsible{Mateusz Redosz}
    \wfpanelrelated{%
        \hyperref[wopanu:profile]{WOPANEL-01},
        \hyperref[wppanel:login-required]{WPPANEL-07},
        \hyperref[wppanel:load-under-10s]{WPPANEL-11}.%
    }
}

% =========================================
% WFPANEL-03 – Wyświetlanie statystyk społeczności
% =========================================
\wfpanelatcard
{wfpanel:display-social-stats}
{Wyświetlanie statystyk społeczności użytkownika}
{03}
{M}
{
    \wfpaneldesc{System umożliwia użytkownikowi podgląd statystyk w profilu (znajomi, obserwowani, obserwujący, zdjęcia).}

    \wfpanelaccept{%
        \begin{itemize}
            \item Użytkownik widzi liczbę znajomych.
            \item Użytkownik widzi liczbę obserwowanych.
            \item Użytkownik widzi liczbę obserwujących.
            \item Użytkownik widzi liczbę zdjęć.
        \end{itemize}
    }

    \wfpanelstakeholder{U3}
    \wfpanelresponsible{Mateusz Redosz}
    \wfpanelrelated{%
        \hyperref[wopanu:profile]{WOPANEL-01},
        \hyperref[wppanel:login-required]{WPPANEL-07}.%
    }
}

% =========================================
% WFPANEL-04 – Nawigacja po kliknięciu w statystyki
% =========================================
\wfpanelatcard
{wfpanel:navigate-from-stats}
{Możliwość przejścia do odpowiedniej podstrony po kliknięciu w statystykę}
{04}
{M}
{
    \wfpaneldesc{System umożliwia przejście do odpowiedniej podstrony panelu po kliknięciu w statystyki na profilu.}

    \wfpanelaccept{%
        \begin{itemize}
            \item Kliknięcie w liczbę znajomych przenosi do listy znajomych.
            \item Kliknięcie w liczbę obserwowanych przenosi do listy obserwowanych.
            \item Kliknięcie w liczbę obserwujących przenosi do listy obserwujących.
            \item Kliknięcie w liczbę zdjęć przenosi do listy zdjęć.
        \end{itemize}
    }

    \wfpanelstakeholder{U3}
    \wfpanelresponsible{Mateusz Redosz}
    \wfpanelrelated{%
        \hyperref[wopanu:profile]{WOPANEL-01},
        \hyperref[wopanu:community]{WOPANEL-06},
        \hyperref[wopanu:photos]{WOPANEL-04},
        \hyperref[wfpanel:friends-list]{WFPANEL-16},
        \hyperref[wfpanel:following-list]{WFPANEL-17},
        \hyperref[wfpanel:followers-list]{WFPANEL-18},
        \hyperref[wfpanel:photos-grouped-with-metrics]{WFPANEL-12}.%
    }
}

% =========================================
% WFPANEL-05 – Najpopularniejsze zdjęcia (top 4)
% =========================================
\wfpanelatcard
{wfpanel:top-photos}
{Wyświetlenie listy czterech najpopularniejszych zdjęć}
{05}
{M}
{
    \wfpaneldesc{System umożliwia wyświetlenie w profilu użytkownika listy czterech najpopularniejszych zdjęć.}

    \wfpanelaccept{%
        \begin{itemize}
            \item Na profilu wyświetlane są miniatury czterech najpopularniejszych zdjęć użytkownika.
            \item W przypadku braku zdjęć system wyświetla odpowiedni komunikat.
        \end{itemize}
    }

    \wfpanelstakeholder{U3}
    \wfpanelresponsible{Mateusz Redosz}
    \wfpanelrelated{%
        \hyperref[wopanu:profile]{WOPANEL-01},
        \hyperref[wopanu:photos]{WOPANEL-04}.%
    }
}

% =========================================
% WFPANEL-06 – Przyciski „dodaj do znajomych” i „obserwuj”
% =========================================
\wfpanelatcard
{wfpanel:friend-follow-buttons}
{Wyświetlenie przycisków do dodania do znajomych oraz do obserwacji}
{06}
{M}
{
    \wfpaneldesc{System umożliwia wykonanie akcji społecznościowych z poziomu profilu innego użytkownika
    poprzez przyciski dodania do znajomych oraz obserwowania.}

    \wfpanelaccept{%
        \begin{itemize}
            \item Na profilu innego użytkownika widoczne są przyciski: dodaj do znajomych oraz obserwuj.
            \item Po kliknięciu stan relacji jest aktualizowany w interfejsie („Waiting For Confirmation”, „Unfollow”).
        \end{itemize}
    }

    \wfpanelstakeholder{U3}
    \wfpanelresponsible{Mateusz Redosz}
    \wfpanelrelated{%
        \hyperref[wopanu:profile]{WOPANEL-01},
        \hyperref[wopanu:community]{WOPANEL-06},
        \hyperref[wppanel:social-authorization]{WPPANEL-05},
        \hyperref[wppanel:login-required]{WPPANEL-07}.%
    }
}

% =========================================
% WFPANEL-07 – Przełączanie typu listy spotów
% =========================================
\wfpanelatcard
{wfpanel:switch-spots-list-type}
{Możliwość zmiany typu listy w liście spotów}
{07}
{M}
{
    \wfpaneldesc{System umożliwia użytkownikowi przełączanie typu listy \glslink{spot}{spotów} w panelu konta
    (polubione, odwiedzone i ocenione pozytywnie, odwiedzone i ocenione negatywnie, planowane).}

    \wfpanelaccept{%
        \begin{itemize}
            \item Użytkownik może wybrać kategorię listy \glslink{spot}{spotów}.
            \item Po zmianie kategorii system wyświetla odpowiednie dane dla wybranej listy.
            \item W przypadku braku \glslink{spot}{spotów} system wyświetla odpowiedni komunikat.
        \end{itemize}
    }

    \wfpanelstakeholder{U3}
    \wfpanelresponsible{Mateusz Redosz}
    \wfpanelrelated{%
        \hyperref[wopanu:spots-lists]{WOPANEL-02},
        \hyperref[wppanel:spots-lists-authorization]{WPPANEL-01}.%
    }
}

% =========================================
% WFPANEL-08 – Usuwanie spota z listy
% =========================================
\wfpanelatcard
{wfpanel:remove-spot-from-list}
{Możliwość usunięcia spota z listy}
{08}
{M}
{
    \wfpaneldesc{System umożliwia użytkownikowi usunięcie \glslink{spot}{spota} z aktualnie wyświetlanej listy w panelu konta.}

    \wfpanelaccept{%
        \begin{itemize}
            \item Użytkownik może usunąć wybrany \glslink{spot}{spot} z listy.
            \item Po usunięciu \glslink{spot}{spot} znika z listy bez konieczności przeładowania całej strony.
        \end{itemize}
    }

    \wfpanelstakeholder{U3}
    \wfpanelresponsible{Mateusz Redosz}
    \wfpanelrelated{%
        \hyperref[wopanu:spots-lists]{WOPANEL-02},
        \hyperref[wppanel:spots-lists-authorization]{WPPANEL-01}.%
    }
}

% =========================================
% WFPANEL-09 – Wyświetlenie spota na mapie
% =========================================
\wfpanelatcard
{wfpanel:show-spot-on-map}
{Przycisk do wyświetlenia spota na mapie}
{09}
{M}
{
    \wfpaneldesc{System udostępnia przycisk umożliwiający wyświetlenie lokalizacji wybranego \glslink{spot}{spota} na mapie.}

    \wfpanelaccept{%
        \begin{itemize}
            \item Dla \glslink{spot}{spota} na liście dostępny jest przycisk „See on map”.
            \item Po kliknięciu użytkownik jest przenoszony do widoku mapy z zaznaczonym spotem.
        \end{itemize}
    }

    \wfpanelstakeholder{U3}
    \wfpanelresponsible{Mateusz Redosz}
    \wfpanelrelated{%
        \hyperref[wopanu:spots-lists]{WOPANEL-02}.%
    }
}

% =========================================
% WFPANEL-10 – Sortowanie zdjęć po dacie
% =========================================
\wfpanelatcard
{wfpanel:photos-sort-by-date}
{Możliwość sortowania dodanych zdjęć po dacie}
{10}
{M}
{
    \wfpaneldesc{System umożliwia użytkownikowi sortowanie listy zdjęć po dacie dodania.}

    \wfpanelaccept{%
        \begin{itemize}
            \item Użytkownik może wybrać sortowanie rosnące lub malejące po dacie.
            \item Po zmianie sortowania kolejność elementów aktualizuje się w liście.
        \end{itemize}
    }

    \wfpanelstakeholder{U3}
    \wfpanelresponsible{Mateusz Redosz}
    \wfpanelrelated{%
        \hyperref[wopanu:photos]{WOPANEL-04},
        \hyperref[wppanel:photos-group-by-date]{WPPANEL-08}.
        \hyperref[wppanel:photos-authorization]{WPPANEL-13}.%
    }
}

% =========================================
% WFPANEL-11 – Filtrowanie zdjęć po dacie
% =========================================
\wfpanelatcard
{wfpanel:photos-filter-by-date}
{Możliwość filtrowania dodanych zdjęć po dacie}
{11}
{M}
{
    \wfpaneldesc{System umożliwia użytkownikowi filtrowanie listy zdjęć według daty dodania.}

    \wfpanelaccept{%
        \begin{itemize}
            \item Użytkownik może wskazać zakres dat do filtrowania.
            \item Lista prezentuje tylko zdjęcia spełniające warunki filtra.
        \end{itemize}
    }

    \wfpanelstakeholder{U3}
    \wfpanelresponsible{Mateusz Redosz}
    \wfpanelrelated{%
        \hyperref[wopanu:photos]{WOPANEL-04}.
        \hyperref[wppanel:photos-authorization]{WPPANEL-13}.%
    }
}

% =========================================
% WFPANEL-12 – Grupowanie zdjęć po dacie + metryki
% =========================================
\wfpanelatcard
{wfpanel:photos-grouped-with-metrics}
{Wyświetlenie listy zdjęć zgrupowanych po dacie wraz z metrykami}
{12}
{M}
{
    \wfpaneldesc{System umożliwia wyświetlenie listy zdjęć zgrupowanej po dacie dodania,
    przy czym każde zdjęcie prezentuje informacje o polubieniach oraz wyświetleniach.}

    \wfpanelaccept{%
        \begin{itemize}
            \item Zdjęcia są pogrupowane w sekcje odpowiadające dacie dodania.
            \item Każde zdjęcie zawiera liczbę polubień oraz wyświetleń.
            \item W przypadku braku zdjęć system wyświetla odpowiedni komunikat.
        \end{itemize}
    }

    \wfpanelstakeholder{U3}
    \wfpanelresponsible{Mateusz Redosz}
    \wfpanelrelated{%
        \hyperref[wopanu:photos]{WOPANEL-04},
        \hyperref[wppanel:photos-group-by-date]{WPPANEL-08}.
        \hyperref[wppanel:photos-authorization]{WPPANEL-13}.%
    }
}

% =========================================
% WFPANEL-13 – Sortowanie filmów po dacie
% =========================================
\wfpanelatcard
{wfpanel:videos-sort-by-date}
{Możliwość sortowania dodanych filmów po dacie}
{13}
{M}
{
    \wfpaneldesc{System umożliwia użytkownikowi sortowanie listy filmów po dacie dodania.}

    \wfpanelaccept{%
        \begin{itemize}
            \item Użytkownik może wybrać sortowanie rosnące lub malejące po dacie.
            \item Po zmianie sortowania kolejność filmów aktualizuje się w liście.
        \end{itemize}
    }

    \wfpanelstakeholder{U3}
    \wfpanelresponsible{Mateusz Redosz}
    \wfpanelrelated{%
        \hyperref[wopanu:videos]{WOPANEL-05},
        \hyperref[wppanel:videos-group-by-date]{WPPANEL-09}.%
    }
}

% =========================================
% WFPANEL-14 – Filtrowanie filmów po dacie
% =========================================
\wfpanelatcard
{wfpanel:videos-filter-by-date}
{Możliwość filtrowania dodanych filmów po dacie}
{14}
{M}
{
    \wfpaneldesc{System umożliwia użytkownikowi filtrowanie listy filmów według daty dodania.}

    \wfpanelaccept{%
        \begin{itemize}
            \item Użytkownik może wskazać zakres dat do filtrowania.
            \item Lista prezentuje tylko filmy spełniające warunki filtra.
        \end{itemize}
    }

    \wfpanelstakeholder{U3}
    \wfpanelresponsible{Mateusz Redosz}
    \wfpanelrelated{%
        \hyperref[wopanu:videos]{WOPANEL-05}.%
    }
}

% =========================================
% WFPANEL-15 – Grupowanie filmów po dacie + metryki
% =========================================
\wfpanelatcard
{wfpanel:videos-grouped-with-metrics}
{Wyświetlenie listy filmów zgrupowanych po dacie wraz z metrykami}
{15}
{M}
{
    \wfpaneldesc{System umożliwia wyświetlenie listy filmów zgrupowanej po dacie dodania,
    przy czym każdy film prezentuje informacje o polubieniach oraz wyświetleniach.}

    \wfpanelaccept{%
        \begin{itemize}
            \item Filmy są pogrupowane w sekcje odpowiadające dacie dodania.
            \item Każdy film zawiera liczbę polubień oraz wyświetleń.
            \item W przypadku braku filmów system wyświetla odpowiedni komunikat.
        \end{itemize}
    }

    \wfpanelstakeholder{U3}
    \wfpanelresponsible{Mateusz Redosz}
    \wfpanelrelated{%
        \hyperref[wopanu:videos]{WOPANEL-05},
        \hyperref[wppanel:videos-group-by-date]{WPPANEL-09}.%
    }
}

% =========================================
% WFPANEL-16 – Wyświetlenie listy znajomych
% =========================================
\wfpanelatcard
{wfpanel:friends-list}
{Wyświetlenie listy znajomych}
{16}
{M}
{
    \wfpaneldesc{System umożliwia użytkownikowi wyświetlenie listy znajomych w panelu konta.}

    \wfpanelaccept{%
        \begin{itemize}
            \item Użytkownik widzi listę znajomych.
            \item Lista jest pobierana z \glslink{api}{API} i prezentowana w sekcji społeczności.
            \item W przypadku braku znajomych system wyświetla odpowiedni komunikat.
        \end{itemize}
    }

    \wfpanelstakeholder{U3}
    \wfpanelresponsible{Mateusz Redosz}
    \wfpanelrelated{%
        \hyperref[wopanu:community]{WOPANEL-06},
        \hyperref[wppanel:social-authorization]{WPPANEL-05}.%
    }
}

% =========================================
% WFPANEL-17 – Wyświetlenie listy obserwowanych
% =========================================
\wfpanelatcard
{wfpanel:following-list}
{Wyświetlenie listy obserwowanych}
{17}
{M}
{
    \wfpaneldesc{System umożliwia użytkownikowi wyświetlenie listy obserwowanych w panelu konta.}

    \wfpanelaccept{%
        \begin{itemize}
            \item Użytkownik widzi listę obserwowanych.
            \item W przypadku braku obserwowanych system wyświetla odpowiedni komunikat.
        \end{itemize}
    }

    \wfpanelstakeholder{U3}
    \wfpanelresponsible{Mateusz Redosz}
    \wfpanelrelated{%
        \hyperref[wopanu:community]{WOPANEL-06},
        \hyperref[wppanel:social-authorization]{WPPANEL-05}.%
    }
}

% =========================================
% WFPANEL-18 – Wyświetlenie listy obserwujących
% =========================================
\wfpanelatcard
{wfpanel:followers-list}
{Wyświetlenie listy obserwujących}
{18}
{M}
{
    \wfpaneldesc{System umożliwia użytkownikowi wyświetlenie listy obserwujących w panelu konta.}

    \wfpanelaccept{%
        \begin{itemize}
            \item Użytkownik widzi listę obserwujących.
            \item W przypadku braku obserwujących system wyświetla odpowiedni komunikat.
        \end{itemize}
    }

    \wfpanelstakeholder{U3}
    \wfpanelresponsible{Mateusz Redosz}
    \wfpanelrelated{%
        \hyperref[wopanu:community]{WOPANEL-06},
        \hyperref[wppanel:social-authorization]{WPPANEL-05}.%
    }
}

% =========================================
% WFPANEL-19 – Wyświetlenie listy zaproszeń
% =========================================
\wfpanelatcard
{wfpanel:invitations-list}
{Wyświetlenie listy zaproszeń po kliknięciu przycisku}
{19}
{M}
{
    \wfpaneldesc{System umożliwia użytkownikowi wyświetlenie listy zaproszeń do znajomych po wybraniu odpowiedniej akcji w sekcji społeczności.}

    \wfpanelaccept{%
        \begin{itemize}
            \item Użytkownik może otworzyć listę zaproszeń poprzez przycisk w sekcji społeczności.
            \item Lista prezentuje oczekujące zaproszenia.
            \item W przypadku braku zaproszeń system wyświetla odpowiedni komunikat.
        \end{itemize}
    }

    \wfpanelstakeholder{U3}
    \wfpanelresponsible{Mateusz Redosz}
    \wfpanelrelated{%
        \hyperref[wopanu:community]{WOPANEL-06},
        \hyperref[wppanel:invitations-authorization]{WPPANEL-06}.%
    }
}

% =========================================
% WFPANEL-20 – Wyszukiwanie nowych znajomych (okno)
% =========================================
\wfpanelatcard
{wfpanel:search-friends-modal}
{Okno do wyszukiwania nowych znajomych}
{20}
{M}
{
    \wfpaneldesc{System umożliwia użytkownikowi otwarcie okna wyszukiwania użytkowników w celu dodania do znajomych.}

    \wfpanelaccept{%
        \begin{itemize}
            \item Użytkownik może otworzyć okno wyszukiwania.
            \item Użytkownik może wyszukać użytkownika po wprowadzonej frazie.
            \item W przypadku braku potencjalnych znajomych system wyświetla odpowiedni komunikat.
        \end{itemize}
    }

    \wfpanelstakeholder{U3}
    \wfpanelresponsible{Mateusz Redosz}
    \wfpanelrelated{%
        \hyperref[wopanu:community]{WOPANEL-06},
        \hyperref[wppanel:social-authorization]{WPPANEL-05}.%
    }
}

% =========================================
% WFPANEL-21 – Akcje na znajomym (usuń / profil / wiadomość)
% =========================================
\wfpanelatcard
{wfpanel:friend-actions}
{Możliwość usunięcia z listy, wejścia w profil oraz napisania wiadomości do znajomego}
{21}
{M}
{
    \wfpaneldesc{System umożliwia użytkownikowi wykonanie podstawowych akcji na użytkowniku z listy znajomych.}

    \wfpanelaccept{%
        \begin{itemize}
            \item Dla znajomego, obserwowanego oraz obserwującego dostępna jest akcja przejścia do profilu.
            \item Dla znajomego oraz obserwowanego dostępna jest akcja usunięcia z listy.
            \item Dla znajomego dostępna jest akcja napisania wiadomości (przejście do czatu 1:1 lub utworzenie go).
        \end{itemize}
    }

    \wfpanelstakeholder{U3}
    \wfpanelresponsible{Mateusz Redosz}
    \wfpanelrelated{%
        \hyperref[wopanu:community]{WOPANEL-06},
        \hyperref[wfpanel:search-friends-modal]{WFPANEL-20},
        \hyperref[wfpanel:invitations-list]{WFPANEL-19},
        \hyperref[wppanel:social-authorization]{WPPANEL-05}.%
    }
}

% =========================================
% WFPANEL-22 – Lista dodanych spotów
% =========================================
\wfpanelatcard
{wfpanel:added-spots-list}
{Wyświetlenie listy dodanych spotów}
{22}
{M}
{
    \wfpaneldesc{System umożliwia użytkownikowi wyświetlenie listy \glslink{spot}{spotów} dodanych przez niego w panelu konta.}

    \wfpanelaccept{%
        \begin{itemize}
            \item Użytkownik widzi listę własnych dodanych \glslink{spot}{spotów}.
            \item W przypadku braku \glslink{spot}{spotów} system wyświetla odpowiedni komunikat.
        \end{itemize}
    }

    \wfpanelstakeholder{U3}
    \wfpanelresponsible{Mateusz Redosz}
    \wfpanelrelated{%
        \hyperref[wopanu:add-spot]{WOPANEL-03},
        \hyperref[wppanel:added-spots-authorization]{WPPANEL-02}.%
    }
}

% =========================================
% WFPANEL-23 – Dodawanie spota formularzem
% =========================================
\wfpanelatcard
{wfpanel:add-spot-form}
{Możliwość dodania spota za pomocą formularza}
{23}
{M}
{
    \wfpaneldesc{System umożliwia użytkownikowi dodanie nowego \glslink{spot}{spota} poprzez formularz w panelu konta.}

    \wfpanelaccept{%
        \begin{itemize}
            \item Użytkownik ma dostęp do formularza dodawania \glslink{spot}{spota}.
            \item Użytkownik może wysłać poprawnie wypełniony formularz.
            \item W przypadku błędu walidacji lub błędu \glslink{api}{API} system wyświetla odpowiedni komunikat.
        \end{itemize}
    }

    \wfpanelstakeholder{U3}
    \wfpanelresponsible{Mateusz Redosz}
    \wfpanelrelated{%
        \hyperref[wopanu:add-spot]{WOPANEL-03},
        \hyperref[wppanel:add-spot-immediate-update]{WPPANEL-12},
        \hyperref[wppanel:login-required]{WPPANEL-07}.%
    }
}

% =========================================
% WFPANEL-24 – Sortowanie komentarzy po dacie
% =========================================
\wfpanelatcard
{wfpanel:comments-sort-by-date}
{Możliwość sortowania dodanych komentarzy po dacie}
{24}
{M}
{
    \wfpaneldesc{System umożliwia użytkownikowi sortowanie listy komentarzy po dacie dodania.}

    \wfpanelaccept{%
        \begin{itemize}
            \item Użytkownik może wybrać sortowanie rosnące lub malejące po dacie.
            \item Po zmianie sortowania kolejność komentarzy aktualizuje się w liście.
        \end{itemize}
    }

    \wfpanelstakeholder{U3}
    \wfpanelresponsible{Mateusz Redosz}
    \wfpanelrelated{%
        \hyperref[wopanu:comments]{WOPANEL-07},
        \hyperref[wppanel:comments-authorization]{WPPANEL-04}.%
    }
}

% =========================================
% WFPANEL-25 – Filtrowanie komentarzy po dacie
% =========================================
\wfpanelatcard
{wfpanel:comments-filter-by-date}
{Możliwość filtrowania dodanych komentarzy po dacie}
{25}
{M}
{
    \wfpaneldesc{System umożliwia użytkownikowi filtrowanie listy komentarzy według daty dodania.}

    \wfpanelaccept{%
        \begin{itemize}
            \item Użytkownik może wskazać zakres dat do filtrowania.
            \item Lista prezentuje tylko komentarze spełniające warunki filtra.
        \end{itemize}
    }

    \wfpanelstakeholder{U3}
    \wfpanelresponsible{Mateusz Redosz}
    \wfpanelrelated{%
        \hyperref[wopanu:comments]{WOPANEL-07},
        \hyperref[wppanel:comments-authorization]{WPPANEL-04}.%
    }
}

% =========================================
% WFPANEL-26 – Grupowanie komentarzy po dacie i spocie
% =========================================
\wfpanelatcard
{wfpanel:comments-grouped-by-date-and-spot}
{Wyświetlenie listy komentarzy zgrupowanych po dacie oraz nazwie spota}
{26}
{M}
{
    \wfpaneldesc{System umożliwia wyświetlenie listy komentarzy użytkownika zgrupowanej po dacie
    oraz nazwie \glslink{spot}{spota}.}

    \wfpanelaccept{%
        \begin{itemize}
            \item Komentarze są pogrupowane po dacie dodania.
            \item W ramach grupy daty komentarze są przypisane do nazw \glslink{spot}{spota}.
            \item W przypadku braku komentarzy system wyświetla odpowiedni komunikat.
        \end{itemize}
    }

    \wfpanelstakeholder{U3}
    \wfpanelresponsible{Mateusz Redosz}
    \wfpanelrelated{%
        \hyperref[wopanu:comments]{WOPANEL-07},
        \hyperref[wppanel:comments-group-by-date-spot]{WPPANEL-10}.%
    }
}

% =========================================
% WFPANEL-27 – Edycja nazwy użytkownika
% =========================================
\wfpanelatcard
{wfpanel:edit-username}
{Możliwość edycji nazwy użytkownika}
{27}
{M}
{
    \wfpaneldesc{System umożliwia użytkownikowi edycję nazwy użytkownika w ustawieniach konta.}

    \wfpanelaccept{%
        \begin{itemize}
            \item Użytkownik może zmienić nazwę użytkownika i zapisać zmiany.
            \item Po zapisie nowa nazwa jest widoczna w panelu konta.
        \end{itemize}
    }

    \wfpanelstakeholder{U3}
    \wfpanelresponsible{Mateusz Redosz}
    \wfpanelrelated{%
        \hyperref[wopanu:settings]{WOPANEL-08},
        \hyperref[wppanel:login-required]{WPPANEL-07}.%
    }
}

% =========================================
% WFPANEL-28 – Edycja adresu e-mail
% =========================================
\wfpanelatcard
{wfpanel:edit-email}
{Możliwość edycji adresu e-mail}
{28}
{M}
{
    \wfpaneldesc{System umożliwia użytkownikowi edycję adresu e-mail w ustawieniach konta.}

    \wfpanelaccept{%
        \begin{itemize}
            \item Użytkownik może zmienić adres e-mail i zapisać zmiany.
            \item W przypadku błędnych danych lub błędu \glslink{api}{API} system wyświetla odpowiedni komunikat.
        \end{itemize}
    }

    \wfpanelstakeholder{U3}
    \wfpanelresponsible{Mateusz Redosz}
    \wfpanelrelated{%
        \hyperref[wopanu:settings]{WOPANEL-08},
        \hyperref[wppanel:login-required]{WPPANEL-07}.%
    }
}

% =========================================
% WFPANEL-29 – Zmiana hasła
% =========================================
\wfpanelatcard
{wfpanel:change-password}
{Możliwość zmiany hasła}
{29}
{M}
{
    \wfpaneldesc{System umożliwia użytkownikowi zmianę hasła w ustawieniach konta.}

    \wfpanelaccept{%
        \begin{itemize}
            \item Użytkownik może ustawić nowe hasło zgodnie z wymaganiami walidacyjnymi.
            \item W przypadku błędu system wyświetla odpowiedni komunikat.
        \end{itemize}
    }

    \wfpanelstakeholder{U3}
    \wfpanelresponsible{Mateusz Redosz}
    \wfpanelrelated{%
        \hyperref[wopanu:settings]{WOPANEL-08},
        \hyperref[wppanel:login-required]{WPPANEL-07}.%
    }
}
