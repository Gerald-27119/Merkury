%! Author = Adam
%! Date = 23/11/2025

\subsection{Funkcjonalności dla chatu}
\label{subsec:funkcjonalnosci-dla-chatu}

% ============================
% WYMAGANIA FUNKCJONALNE DLA CZATU
% Identyfikatory: WFCZAT-XX
% ============================

\newcounter{wfczat}[chapter]
\renewcommand{\thewfczat}{\thechapter.\arabic{wfczat}}

% Pola karty (lewa kolumna)
\newcommand{\wfczpriority}[2]{%
    \textbf{Identyfikator:} &
    WFCZAT-#1\hfill
    \begingroup
    \setlength{\fboxsep}{1pt}%
    \colorbox{lightgray}{\strut\textbf{Priorytet:}}~#2%
    \endgroup
    \\ \hline
}
\newcommand{\wfczname}[1]{\textbf{Nazwa:} & #1 \\ \hline}
\newcommand{\wfczdesc}[1]{\textbf{Opis:} & #1 \\ \hline}
\newcommand{\wfczaccept}[1]{\textbf{Kryteria akceptacji:} & #1 \\ \hline}
\newcommand{\wfczinput}[1]{\textbf{Dane wejściowe:} & #1 \\ \hline}
\newcommand{\wfczpre}[1]{\textbf{Warunki początkowe:} & #1 \\ \hline}
\newcommand{\wfczpost}[1]{\textbf{Warunki końcowe:} & #1 \\ \hline}
\newcommand{\wfczexceptions}[1]{\textbf{Sytuacje wyjątkowe:} & #1 \\ \hline}
\newcommand{\wfczimpl}[1]{\textbf{Szczegóły implementacji:} & #1 \\ \hline}
\newcommand{\wfczstakeholders}[1]{\textbf{Udziałowiec:} & #1 \\ \hline}
\newcommand{\wfczrelated}[1]{\textbf{Wymagania powiązane:} & #1 \\ \hline}
\newcommand{\wfczstatus}[1]{\textbf{Status:} & #1 \\ \hline}
\newcommand{\wfczresponsible}[1]{\textbf{Osoba odpowiedzialna za realizację:} & #1 \\ \hline}
\newcommand{\wfcznote}[1]{\textbf{Notatka:} & #1 \\ \hline}

% Karta wymagania funkcjonalnego dla czatu
% #1 – label
% #2 – nazwa
% #3 – numer / sufiks kodu (np. 01 → WFCZAT-01)
% #4 – priorytet
% #5 – reszta pól: \wfczdesc, \wfczaccept, \wfczinput, ...
\newcommand{\wfczatcard}[5]{%
    \refstepcounter{wfczat}%
    \par\begin{center}
    \renewcommand{\arraystretch}{1.15}%
    \begin{tabularx}{\textwidth}{|>{\columncolor{lightgray}\raggedright\arraybackslash}p{0.19\textwidth}|X|}
    \rowcolor{lightgray}
    \multicolumn{2}{|c|}{\textbf{KARTA WYMAGANIA FUNKCJONALNEGO DLA CZATU}} \\ \hline
    \wfczpriority{#3}{#4}
    \wfczname{#2}
    #5
    \end{tabularx}
    \vspace{3pt}
    \textbf{Tabela \thewfczat:} Wymaganie funkcjonalne dla czatu: #2\label{#1}
    \end{center}%
    \addcontentsline{lot}{table}{Tabela \thewfczat: Wymaganie funkcjonalne dla czatu: #2}%
}

% -------------------------------------------------
% WFCZAT-01 – Wysyłanie GIF-ów
% -------------------------------------------------

\wfczatcard
{wfczat:send-gif}
{Wysyłanie GIF-ów}
{01}
{wysoki}
{
    \wfczdesc{System umożliwia wysyłanie w wiadomościach animowanych obrazów GIF
    w ramach wybranego czatu (1:1 lub grupowego).}

    \wfczaccept{%
        \begin{itemize}
            \item Użytkownik może wybrać GIF z listy/picker'a lub z dysku i dołączyć go do wiadomości.
            \item Po wysłaniu GIF wyświetla się poprawnie w treści czatu u wszystkich uczestników.
            \item Błędne lub zbyt duże pliki GIF są odrzucane z czytelnym komunikatem.
        \end{itemize}
    }

    \wfczinput{Identyfikator czatu, identyfikator nadawcy, wybrany GIF
        (plik lub identyfikator z zewnętrznej usługi).}

    \wfczpre{Użytkownik jest zalogowany, należy do danego czatu,
        połączenie z serwerem jest aktywne.}

    \wfczpost{Wiadomość z GIF-em jest zapisana w bazie danych i widoczna
    dla wszystkich uczestników czatu.}

    \wfczexceptions{Brak połączenia z serwerem, przekroczony limit rozmiaru pliku,
        nieobsługiwany format, brak uprawnień do czatu.}

    \wfczimpl{Rozszerzenie mutacji GraphQL \texttt{createChatMessage} o typ wiadomości \emph{GIF};
    przechowywanie pliku w usłudze składowania plików (np.\ Azure Blob Storage) lub
    przechowywanie identyfikatora GIF-a z usługi zewnętrznej.}

    \wfczstakeholders{Użytkownik zalogowany, Użytkownik premium.}

    \wfczrelated{WFCZAT-02 -- Wysyłanie plików; WFCZAT-06 -- Wysyłanie emotikonów.}

    \wfczstatus{W trakcie implementacji.}

    \wfczresponsible{Zespół deweloperski Merkury.}

    \wfcznote{Docelowo możliwe podpięcie zewnętrznej biblioteki GIF-ów (np.\ GIPHY/Tenor).}
}

% -------------------------------------------------
% WFCZAT-02 – Wysyłanie plików
% -------------------------------------------------

\wfczatcard
{wfczat:send-files}
{Wysyłanie plików}
{02}
{wysoki}
{
    \wfczdesc{System umożliwia dołączanie i wysyłanie plików (np.\ PDF, zdjęcia,
        dokumenty) jako załączników do wiadomości na czacie.}

    \wfczaccept{%
        \begin{itemize}
            \item Użytkownik może wybrać plik z dysku i dołączyć go do wiadomości.
            \item Odbiorca widzi w czacie czytelny element reprezentujący plik (nazwa, rozmiar, ikona typu).
            \item Kliknięcie w załącznik umożliwia jego pobranie lub otwarcie w nowej karcie/przeglądarce.
        \end{itemize}
    }

    \wfczinput{Identyfikator czatu, identyfikator nadawcy, co najmniej jeden
    plik do wysłania.}

    \wfczpre{Użytkownik jest zalogowany, należy do danego czatu; plik nie przekracza
    ustalonego limitu rozmiaru i jest dozwolonego typu.}

    \wfczpost{Wiadomość z załącznikiem jest zapisana w systemie; plik jest
    dostępny do pobrania dla uczestników czatu.}

    \wfczexceptions{Przekroczony limit rozmiaru, nieobsługiwany typ pliku,
        błąd przesyłania, brak miejsca w magazynie plików.}

    \wfczimpl{Przesyłanie plików przez endpoint uploadu lub część mutacji
    \texttt{createChatMessage}; składowanie w zewnętrznej usłudze plików
    z zapisem tylko metadanych w bazie.}

    \wfczstakeholders{Użytkownik zalogowany, Użytkownik premium.}

    \wfczrelated{WFCZAT-01 -- Wysyłanie GIF-ów; WFCZAT-03 -- Wysyłanie wiadomości prywatnych.}

    \wfczstatus{W trakcie implementacji.}

    \wfczresponsible{Zespół deweloperski Merkury.}

    \wfcznote{W przyszłości możliwe ograniczenie typów plików dla użytkowników
    niepremium.}
}

% -------------------------------------------------
% WFCZAT-03 – Wysyłanie wiadomości prywatnych
% -------------------------------------------------

\wfczatcard
{wfczat:private-messages}
{Wysyłanie wiadomości prywatnych}
{03}
{wysoki}
{
    \wfczdesc{System umożliwia prowadzenie prywatnych rozmów 1:1
    pomiędzy dwoma użytkownikami.}

    \wfczaccept{%
        \begin{itemize}
            \item Użytkownik może rozpocząć lub kontynuować konwersację 1:1
            z wybranym użytkownikiem.
            \item Wiadomości z prywatnego czatu są widoczne wyłącznie dla tych dwóch użytkowników.
            \item Nowe wiadomości pojawiają się w czasie zbliżonym do rzeczywistego bez przeładowania strony.
        \end{itemize}
    }

    \wfczinput{Identyfikator nadawcy, identyfikator odbiorcy (użytkownika),
        treść wiadomości oraz ewentualne załączniki.}

    \wfczpre{Obaj użytkownicy posiadają aktywne konta i nie zablokowali się nawzajem.}

    \wfczpost{Wiadomości prywatne są zapisane w historii czatu 1:1 i dostępne
    po ponownym otwarciu rozmowy.}

    \wfczexceptions{Brak uprawnień (np.\ zablokowany użytkownik),
        brak połączenia z serwerem, błąd walidacji treści.}

    \wfczimpl{Wydzielony typ czatu \emph{PRIVATE} w modelu domenowym;
    mutacja \texttt{createChat} dla utworzenia czatu 1:1 oraz
    \texttt{createChatMessage} dla kolejnych wiadomości.}

    \wfczstakeholders{Użytkownik zalogowany.}

    \wfczrelated{WFCZAT-05 -- Rozpoczynanie nowego czatu; WFCZAT-13 -- Wyświetlanie starszych wiadomości.}

    \wfczstatus{Zaimplementowane i testowane integracyjnie.}

    \wfczresponsible{Zespół deweloperski Merkury.}

    \wfcznote{Możliwa rozbudowa o oznaczanie czatu prywatnego jako \emph{ulubiony}.}
}

% -------------------------------------------------
% WFCZAT-04 – Wysyłanie wiadomości do wielu osób jednocześnie
% -------------------------------------------------

\wfczatcard
{wfczat:group-messages}
{Wysyłanie wiadomości do wielu osób jednocześnie}
{04}
{wysoki}
{
    \wfczdesc{System umożliwia wysyłanie jednej wiadomości do wielu użytkowników
    w ramach czatu grupowego.}

    \wfczaccept{%
        \begin{itemize}
            \item Wiadomość wysłana w czacie grupowym jest dostarczana wszystkim jego członkom.
            \item Interfejs wyraźnie wskazuje, że rozmowa to czat grupowy (np.\ nazwa, avatar, liczba uczestników).
            \item Nowi uczestnicy dołączeni do czatu widzą historię rozmowy zgodnie z przyjętą polityką prywatności.
        \end{itemize}
    }

    \wfczinput{Identyfikator czatu grupowego, identyfikator nadawcy, treść wiadomości
    oraz ewentualne załączniki.}

    \wfczpre{Czat typu grupowego istnieje, a użytkownik wysyłający wiadomość jest jego członkiem.}

    \wfczpost{Wiadomość jest zapisana w historii czatu i widoczna dla wszystkich aktualnych uczestników.}

    \wfczexceptions{Brak członkostwa nadawcy w czacie, błąd autoryzacji,
        przekroczenie limitu użytkowników czatu.}

    \wfczimpl{Typ czatu \emph{GROUP} w modelu; wiadomości powiązane z jednym czatem
    i wieloma użytkownikami w tabeli relacyjnej; dystrybucja po WebSocketach.}

    \wfczstakeholders{Użytkownik zalogowany, Użytkownik premium.}

    \wfczrelated{WFCZAT-05 -- Rozpoczynanie nowego czatu; WFCZAT-12 -- Dodawanie użytkowników do istniejącego czatu.}

    \wfczstatus{W trakcie implementacji.}

    \wfczresponsible{Zespół deweloperski Merkury.}

    \wfcznote{Możliwa przyszła obsługa ról (właściciel, moderator) w czatach grupowych.}
}

% -------------------------------------------------
% WFCZAT-05 – Rozpoczynanie nowego czatu (Dodawanie do czatu)
% -------------------------------------------------

\wfczatcard
{wfczat:create-chat}
{Rozpoczynanie nowego czatu}
{05}
{wysoki}
{
    \wfczdesc{System umożliwia użytkownikowi utworzenie nowego czatu prywatnego
    lub grupowego oraz dodanie do niego wskazanych uczestników.}

    \wfczaccept{%
        \begin{itemize}
            \item Użytkownik może zainicjować nowy czat z widoku listy czatów
            lub profilu innego użytkownika.
            \item Po utworzeniu czat pojawia się na liście czatów wszystkich jego członków.
            \item Uczestnicy mogą natychmiast rozpocząć wymianę wiadomości.
        \end{itemize}
    }

    \wfczinput{Typ czatu (prywatny/grupowy), nazwa czatu (dla grup),
        lista identyfikatorów uczestników, identyfikator tworzącego użytkownika.}

    \wfczpre{Użytkownik jest zalogowany i ma uprawnienia do zakładania nowych czatów.}

    \wfczpost{Nowy czat jest zapisany w bazie danych, powiązany z uczestnikami
    i widoczny w ich listach czatów.}

    \wfczexceptions{Próba utworzenia czatu z nieistniejącym użytkownikiem,
        przekroczenie maksymalnej liczby uczestników, błąd zapisu w bazie.}

    \wfczimpl{Mutacja GraphQL \texttt{createChat}; walidacja listy uczestników
    oraz unikalności czatów prywatnych (1:1).}

    \wfczstakeholders{Użytkownik zalogowany, Użytkownik premium.}

    \wfczrelated{WFCZAT-03 -- Wysyłanie wiadomości prywatnych;
    WFCZAT-04 -- Wysyłanie wiadomości do wielu osób jednocześnie.}

    \wfczstatus{Zaimplementowane na backendzie, integracja z frontendem w toku.}

    \wfczresponsible{Zespół deweloperski Merkury.}

    \wfcznote{Funkcjonalność odpowiada pozycji „Dodawanie do czatu” z diagramu funkcjonalności.}
}

% -------------------------------------------------
% WFCZAT-06 – Wysyłanie emotikonów
% -------------------------------------------------

\wfczatcard
{wfczat:emoticons}
{Wysyłanie emotikonów}
{06}
{wysoki}
{
    \wfczdesc{System umożliwia wysyłanie emotikonów (emoji) w treści wiadomości
    na czacie.}

    \wfczaccept{%
        \begin{itemize}
            \item Użytkownik może wstawić emotikony z wbudowanego pickera emoji.
            \item Emotikony są poprawnie renderowane na różnych urządzeniach
            (mobile/desktop).
            \item Kopiowanie/edycja wiadomości zachowuje emotikony.
        \end{itemize}
    }

    \wfczinput{Identyfikator czatu, identyfikator nadawcy, treść wiadomości
    zawierająca emoji.}

    \wfczpre{Użytkownik jest zalogowany i ma dostęp do czatu.}

    \wfczpost{Wiadomość z emotikonami jest zapisana w historii czatu i
    wyświetlana u wszystkich uczestników.}

    \wfczexceptions{Niepoprawne kodowanie znaków, brak wsparcia dla części emoji
    w przeglądarce.}

    \wfczimpl{Wykorzystanie Unicode dla emoji; opcjonalne podpięcie biblioteki
    komponentu pickera (np.\ \texttt{emoji-mart}).}

    \wfczstakeholders{Użytkownik zalogowany.}

    \wfczrelated{WFCZAT-01 -- Wysyłanie GIF-ów; WFCZAT-02 -- Wysyłanie plików.}

    \wfczstatus{Zaimplementowane.}

    \wfczresponsible{Zespół deweloperski Merkury.}

    \wfcznote{Możliwa rozbudowa o reakcje na wiadomość w formie emoji.}
}

% -------------------------------------------------
% WFCZAT-07 – Dostępność czatu po utworzeniu
% -------------------------------------------------

\wfczatcard
{wfczat:chat-availability}
{Dostępność czatu po utworzeniu}
{07}
{wysoki}
{
    \wfczdesc{Czat po utworzeniu pozostaje dostępny dla jego członków;
    użytkownik może później odnaleźć czat i wrócić do wcześniejszej konwersacji.}

    \wfczaccept{%
        \begin{itemize}
            \item Utworzone czaty pojawiają się na liście czatów użytkownika.
            \item Po ponownym zalogowaniu użytkownik może otworzyć istniejący czat
            i zobaczyć jego historię.
            \item Usunięcie użytkownika z czatu powoduje usunięcie go z jego listy czatów.
        \end{itemize}
    }

    \wfczinput{Kontekst zalogowanego użytkownika; ewentualne parametry filtrów
    listy czatów.}

    \wfczpre{Czat został wcześniej utworzony, a użytkownik jest jego członkiem.}

    \wfczpost{Lista czatów użytkownika prezentuje wszystkie czaty, do których
    ma on dostęp.}

    \wfczexceptions{Błąd autoryzacji (sesja wygasła), błąd bazy danych podczas
    pobierania listy czatów.}

    \wfczimpl{Zapytanie GraphQL/REST \texttt{chatsByCurrentUser}, używane w panelu
    bocznym listy czatów.}

    \wfczstakeholders{Użytkownik zalogowany, Użytkownik premium.}

    \wfczrelated{WFCZAT-05 -- Rozpoczynanie nowego czatu; WFCZAT-13 -- Wyświetlanie starszych wiadomości.}

    \wfczstatus{Zaimplementowane.}

    \wfczresponsible{Zespół deweloperski Merkury.}

    \wfcznote{Możliwość sortowania listy czatów po dacie ostatniej wiadomości.}
}

% -------------------------------------------------
% WFCZAT-08 – Edytowanie nazwy czatu grupowego
% -------------------------------------------------

\wfczatcard
{wfczat:edit-group-name}
{Edytowanie nazwy czatu grupowego}
{08}
{wysoki}
{
    \wfczdesc{System umożliwia zmianę nazwy istniejącego czatu grupowego
    przez jego właściciela lub uprawnionego administratora.}

    \wfczaccept{%
        \begin{itemize}
            \item Właściciel czatu może edytować nazwę z poziomu ustawień czatu.
            \item Nowa nazwa jest natychmiast widoczna na liście czatów u wszystkich uczestników.
            \item Historia wiadomości pozostaje niezmieniona po zmianie nazwy.
        \end{itemize}
    }

    \wfczinput{Identyfikator czatu grupowego, nowa nazwa czatu, identyfikator użytkownika
    wykonującego operację.}

    \wfczpre{Użytkownik jest zalogowany i posiada uprawnienia właściciela/administratora
    w danym czacie grupowym.}

    \wfczpost{Nazwa czatu jest zaktualizowana w bazie danych i w interfejsie użytkowników.}

    \wfczexceptions{Brak uprawnień do edycji, zbyt długa lub pusta nazwa,
        błąd zapisu w bazie.}

    \wfczimpl{Mutacja GraphQL \texttt{updateChatMetadata} z polem \texttt{name};
    walidacja uprawnień po stronie backendu.}

    \wfczstakeholders{Właściciel czatu grupowego, pozostali uczestnicy czatu.}

    \wfczrelated{WFCZAT-09 -- Edycja zdjęcia czatu grupowego.}

    \wfczstatus{W trakcie implementacji.}

    \wfczresponsible{Zespół deweloperski Merkury.}

    \wfcznote{Możliwa historia zmian nazwy w logach administracyjnych.}
}

% -------------------------------------------------
% WFCZAT-09 – Edycja zdjęcia reprezentującego czat grupowy
% -------------------------------------------------

\wfczatcard
{wfczat:edit-group-avatar}
{Edycja zdjęcia czatu grupowego}
{09}
{wysoki}
{
    \wfczdesc{System umożliwia zmianę obrazu/avatara reprezentującego czat grupowy.}

    \wfczaccept{%
        \begin{itemize}
            \item Właściciel lub administrator może wgrać nowe zdjęcie czatu grupowego.
            \item Zmienione zdjęcie jest widoczne na liście czatów i w nagłówku rozmowy.
            \item Niepoprawne formaty lub zbyt duże pliki są odrzucane z informacją o błędzie.
        \end{itemize}
    }

    \wfczinput{Identyfikator czatu grupowego, nowy plik graficzny,
        identyfikator użytkownika wykonującego operację.}

    \wfczpre{Użytkownik posiada uprawnienia do edycji ustawień czatu grupowego.}

    \wfczpost{Nowe zdjęcie czatu jest zapisane w systemie i używane w interfejsie.}

    \wfczexceptions{Nieobsługiwany format obrazu, przekroczony limit rozmiaru,
        błąd przesyłania lub zapisu w magazynie plików.}

    \wfczimpl{Pole \texttt{avatarUrl} w encji czatu; przesyłanie pliku do usługi
    składowania; aktualizacja adresu URL w bazie.}

    \wfczstakeholders{Właściciel czatu grupowego, uczestnicy czatu.}

    \wfczrelated{WFCZAT-08 -- Edytowanie nazwy czatu grupowego.}

    \wfczstatus{W trakcie implementacji.}

    \wfczresponsible{Zespół deweloperski Merkury.}

    \wfcznote{Możliwość ustawienia domyślnego avatara generowanego z nazwy czatu.}
}

% -------------------------------------------------
% WFCZAT-10 – Edycja wysłanej wiadomości
% -------------------------------------------------

\wfczatcard
{wfczat:edit-message}
{Edycja wysłanej wiadomości}
{10}
{średni}
{
    \wfczdesc{System umożliwia użytkownikowi edycję treści wcześniej wysłanej
    wiadomości na czacie (np.\ poprawa literówki).}

    \wfczaccept{%
        \begin{itemize}
            \item Użytkownik może edytować swoje wiadomości przez ograniczony czas
            od momentu wysłania (konfigurowalny limit).
            \item Zmieniona wiadomość jest oznaczona etykietą „(edytowano)”.
            \item Odbiorcy widzą zaktualizowaną treść bez duplikowania wiadomości.
        \end{itemize}
    }

    \wfczinput{Identyfikator wiadomości, nowa treść wiadomości,
        identyfikator użytkownika wykonującego edycję.}

    \wfczpre{Użytkownik jest autorem wiadomości i mieści się w dozwolonym
    oknie czasowym edycji.}

    \wfczpost{Treść wiadomości w bazie danych zostaje zaktualizowana, a widok
    czatu odświeżony.}

    \wfczexceptions{Próba edycji cudzej wiadomości, przekroczony limit czasu,
        błąd walidacji treści, brak połączenia.}

    \wfczimpl{Mutacja GraphQL \texttt{updateChatMessage}; przechowywanie flagi
    \texttt{editedAt} oraz aktualizacji poprzez WebSocket.}

    \wfczstakeholders{Użytkownik zalogowany, odbiorcy wiadomości.}

    \wfczrelated{WFCZAT-11 -- Usuwanie wysłanej wiadomości.}

    \wfczstatus{Planowane.}

    \wfczresponsible{Zespół deweloperski Merkury.}

    \wfcznote{Możliwa archiwizacja poprzednich wersji wiadomości tylko
    na potrzeby moderacji.}
}

% -------------------------------------------------
% WFCZAT-11 – Usuwanie wysłanej wiadomości
% -------------------------------------------------

\wfczatcard
{wfczat:delete-message}
{Usuwanie wysłanej wiadomości}
{11}
{średni}
{
    \wfczdesc{System umożliwia usunięcie wysłanej wiadomości z czatu
        (dla siebie lub dla wszystkich uczestników, zgodnie z konfiguracją).}

    \wfczaccept{%
        \begin{itemize}
            \item Autor wiadomości może ją usunąć z historii rozmowy
            w dopuszczonym przedziale czasowym.
            \item W miejscu usuniętej wiadomości może pojawić się informacja
            typu „Wiadomość usunięta”.
            \item Usunięcie wiadomości nie narusza integralności pozostałej historii czatu.
        \end{itemize}
    }

    \wfczinput{Identyfikator wiadomości, typ usunięcia (dla siebie / dla wszystkich),
        identyfikator użytkownika.}

    \wfczpre{Użytkownik jest autorem wiadomości lub posiada uprawnienia
    moderacyjne w czacie.}

    \wfczpost{Wiadomość jest oznaczona jako usunięta i nie jest prezentowana
    w pierwotnej formie uczestnikom czatu.}

    \wfczexceptions{Próba usunięcia cudzej wiadomości bez uprawnień,
        przekroczony limit czasu, błąd zapisu w bazie.}

    \wfczimpl{Mutacja \texttt{deleteChatMessage} lub aktualizacja pola
    \texttt{deletedAt}; odświeżenie widoku w kliencie przez WebSocket.}

    \wfczstakeholders{Użytkownik zalogowany, moderatorzy czatu.}

    \wfczrelated{WFCZAT-10 -- Edycja wysłanej wiadomości.}

    \wfczstatus{Planowane.}

    \wfczresponsible{Zespół deweloperski Merkury.}

    \wfcznote{Polityka „dla siebie / dla wszystkich” do ustalenia na etapie projektu UX.}
}

% -------------------------------------------------
% WFCZAT-12 – Dodawanie użytkowników do istniejącego czatu
% -------------------------------------------------

\wfczatcard
{wfczat:add-users-existing-chat}
{Dodawanie użytkowników do istniejącego czatu}
{12}
{wysoki}
{
    \wfczdesc{System umożliwia dodawanie nowych użytkowników do już istniejącego
    czatu grupowego.}

    \wfczaccept{%
        \begin{itemize}
            \item Właściciel lub uprawniony użytkownik może wyszukać i dodać nowe osoby do czatu.
            \item Nowo dodani użytkownicy pojawiają się na liście uczestników
            i mogą od razu brać udział w rozmowie.
            \item Dodanie uczestnika jest widoczne dla pozostałych w formie komunikatu systemowego.
        \end{itemize}
    }

    \wfczinput{Identyfikator czatu, lista identyfikatorów użytkowników do dodania,
        identyfikator wykonującego operację.}

    \wfczpre{Czat typu grupowego istnieje; użytkownik wykonujący operację ma odpowiednie
    uprawnienia.}

    \wfczpost{Lista uczestników czatu jest zaktualizowana; nowe osoby są powiązane
    z czatem w bazie danych.}

    \wfczexceptions{Dodanie użytkownika, który nie ma konta, został zablokowany
    lub jest już członkiem czatu; przekroczenie maksymalnej liczby uczestników.}

    \wfczimpl{Mutacja GraphQL \texttt{updateChatParticipants} z listą identyfikatorów
    uczestników; walidacja po stronie backendu.}

    \wfczstakeholders{Właściciel czatu grupowego, pozostali uczestnicy,
        nowo dodani użytkownicy.}

    \wfczrelated{WFCZAT-04 -- Wysyłanie wiadomości do wielu osób jednocześnie;
    WFCZAT-05 -- Rozpoczynanie nowego czatu.}

    \wfczstatus{W trakcie implementacji.}

    \wfczresponsible{Zespół deweloperski Merkury.}

    \wfcznote{Możliwa rozbudowa o zaproszenia wymagające akceptacji.}
}

% -------------------------------------------------
% WFCZAT-13 – Wyświetlanie starszych wiadomości
% -------------------------------------------------

\wfczatcard
{wfczat:load-older-messages}
{Wyświetlanie starszych wiadomości}
{13}
{wysoki}
{
    \wfczdesc{System powinien domyślnie wyświetlać co najmniej ostatnie 20
    wiadomości w czacie, a starsze wiadomości dociągać na bieżąco podczas
    przewijania historii przez użytkownika.}

    \wfczaccept{%
        \begin{itemize}
            \item Po wejściu na czat użytkownik widzi minimum 20 ostatnich wiadomości.
            \item Przewijanie w górę automatycznie pobiera starsze wiadomości
            (mechanizm infinite scroll).
            \item Dociąganie wiadomości nie powoduje zauważalnych opóźnień interfejsu.
        \end{itemize}
    }

    \wfczinput{Identyfikator czatu, parametry paginacji (np.\ kursor, znacznik czasu).}

    \wfczpre{Użytkownik jest zalogowany i posiada dostęp do czatu; w bazie istnieje
    historia rozmowy.}

    \wfczpost{Użytkownik ma możliwość przeglądania pełnej historii czatu w zakresie
    swoich uprawnień.}

    \wfczexceptions{Brak połączenia z serwerem, błąd paginacji,
        przekroczenie limitu zapytań.}

    \wfczimpl{Zapytania GraphQL z paginacją kursorową; komponent
    \emph{infinite scroll} oparty na TanStack Query; cache po stronie klienta.}

    \wfczstakeholders{Użytkownik zalogowany, Użytkownik premium.}

    \wfczrelated{WFCZAT-03 -- Wysyłanie wiadomości prywatnych;
    WFCZAT-04 -- Wysyłanie wiadomości do wielu osób jednocześnie.}

    \wfczstatus{Zaimplementowane, wymaga dalszej optymalizacji wydajności.}

    \wfczresponsible{Zespół deweloperski Merkury.}

    \wfcznote{Liczbę początkowo ładowanych wiadomości można konfigurować w ustawieniach systemu.}
}
