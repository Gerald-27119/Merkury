%! Author = Adam
%! Date = 30/11/2025

\subsubsection{Wymagania funkcjonalne dla czatu}
\label{subsubsec:wymagania-funkcjonalne-dla-chatu}

\newcounter{wfczat}[chapter]
\renewcommand{\thewfczat}{\thechapter.\arabic{wfczat}}

\newlength{\wfczLabelWidth}
\setlength{\wfczLabelWidth}{0.17\textwidth}

\newlength{\wfczColTwoWidth}
\setlength{\wfczColTwoWidth}{0.21\textwidth}

\newlength{\wfczColThreeWidth}
\setlength{\wfczColThreeWidth}{0.13\textwidth}

\newlength{\wfczColFourWidth}
\setlength{\wfczColFourWidth}{0.29\textwidth}

\newlength{\wfczContentWidth}
\setlength{\wfczContentWidth}{0.78\textwidth}

\newcommand{\wfczthreecolcell}[1]{%
    \multicolumn{3}{|p{\wfczContentWidth}|}{%
        \begin{minipage}[t]{\wfczContentWidth}%
        #1%
        \end{minipage}%
    }%
}

\newcommand{\wfczpriority}[2]{%
    \textbf{Identyfikator:} & WFCZAT-#1 &
    \textbf{Priorytet:}     & #2 \\ \hline
}

\newcommand{\wfczname}[1]{%
    \textbf{Nazwa:} &
    \wfczthreecolcell{#1} \\ \hline
}

\newcommand{\wfczdesc}[1]{%
    \textbf{Opis:} &
    \wfczthreecolcell{#1} \\ \hline
}

\newcommand{\wfczaccept}[1]{%
    \textbf{Kryteria akceptacji:} &
    \wfczthreecolcell{%
        \begingroup
        \setlength{\leftmargini}{1.5em}%
        #1%
        \endgroup
    } \\ \hline
}

\newcommand{\wfczinput}[1]{%
    \textbf{Dane wejściowe:} &
    \wfczthreecolcell{#1} \\ \hline
}

\newcommand{\wfczpre}[1]{%
    \textbf{Warunki początkowe:} &
    \wfczthreecolcell{#1} \\ \hline
}

\newcommand{\wfczpost}[1]{%
    \textbf{Warunki końcowe:} &
    \wfczthreecolcell{#1} \\ \hline
}

\newcommand{\wfczexceptions}[1]{%
    \textbf{Sytuacje wyjątkowe:} &
    \wfczthreecolcell{#1} \\ \hline
}

\newcommand{\wfczstakeholder}[1]{%
    \textbf{Udziałowiec:} &
    \wfczthreecolcell{#1} \\ \hline
}

\newcommand{\wfczresponsible}[1]{%
    \textbf{Realizator:} &
    \wfczthreecolcell{#1} \\ \hline
}

\newcommand{\wfcznote}[1]{%
    \textbf{Notatka:} &
    \wfczthreecolcell{#1} \\ \hline
}

\newcommand{\wfczrelated}[1]{%
    \textbf{Wymagania powiązane:} &
    \wfczthreecolcell{#1} \\ \hline
}

\newcommand{\wfczatcard}[5]{%
    \refstepcounter{wfczat}%
    {%
        \centering
        \begin{longtable}{|
                >{\columncolor{lightgray}\raggedright\arraybackslash}p{\wfczLabelWidth}|
            p{\wfczColTwoWidth}|
                >{\columncolor{lightgray}\raggedright\arraybackslash}p{\wfczColThreeWidth}|
            p{\wfczColFourWidth}|}
        \hline
        \rowcolor{lightgray}\multicolumn{4}{|c|}{\textbf{KARTA WYMAGANIA FUNKCJONALNEGO DLA CZATU}} \\ \hline
        \endfirsthead
        \hline
        \rowcolor{lightgray}\multicolumn{4}{|c|}{\textbf{KARTA WYMAGANIA FUNKCJONALNEGO DLA CZATU (cd.)}} \\ \hline
        \endhead
        \wfczpriority{#3}{#4}
        \wfczname{#2}
        #5
        \end{longtable}
        \par
    }%
    \vspace{3pt}%
    \textbf{Tabela \thewfczat:} Wymaganie funkcjonalne dla czatu: #2\label{#1}%
    \addcontentsline{lot}{table}{Tabela \thewfczat: Wymaganie funkcjonalne dla czatu: #2}%
}


% =========================================
% WFCZAT-01 – Wysyłanie GIF-ów
% =========================================

\wfczatcard
{wfczat:send-gif}
{Wysyłanie GIF-ów}
{01}
{S}
{
    \wfczdesc{System umożliwia wysyłanie w wiadomościach animowanych obrazów \gls{gif}
    w ramach wybranego czatu (1:1 lub grupowego).}

    \wfczaccept{%
        \begin{itemize}
            \item Użytkownik może wybrać animowany obraz \gls{gif} z wbudowanego okna wyboru i dołączyć go do wiadomości.
            \item Po wysłaniu \gls{gif} wyświetla się poprawnie w treści czatu u wszystkich uczestników.
        \end{itemize}
    }

    \wfczinput{Identyfikator czatu, identyfikator nadawcy, wybrany \gls{gif}
    (identyfikator z usługi zewnętrznej).}

    \wfczpre{Użytkownik jest zalogowany i jest członkiem danego czatu.}

    \wfczpost{Wiadomość z \gls{gif}em jest zapisana w historii czatu i widoczna
    dla wszystkich uprawnionych uczestników.}

    \wfczexceptions{Brak połączenia z serwerem, przekroczony limit rozmiaru pliku,
        nieobsługiwany format, błąd usługi zewnętrznej.}

    \wfczstakeholder{U3}

    \wfczresponsible{Adam Langmesser}

    \wfcznote{\textemdash\ brak}

    \wfczrelated{\hyperref[woczat:send-message]{WOCZAT-01}.}
}

% =========================================
% WFCZAT-02 – Wysyłanie plików
% =========================================

\wfczatcard
{wfczat:send-files}
{Wysyłanie plików}
{02}
{S}
{
    \wfczdesc{System umożliwia dołączanie i wysyłanie plików (np.\ dokumentów,
        zdjęć, archiwów) jako załączników do wiadomości na czacie.}

    \wfczaccept{%
        \begin{itemize}
            \item Użytkownik może wybrać jeden lub wiele plików z dysku i dołączyć je do wiadomości.
            \item Odbiorca widzi w czacie element reprezentujący plik.
            \item Kliknięcie w załącznik umożliwia pobranie lub otwarcie pliku.
        \end{itemize}
    }

    \wfczinput{Identyfikator czatu, identyfikator nadawcy, co najmniej jeden
    plik do wysłania.}

    \wfczpre{Użytkownik jest zalogowany, należy do danego czatu; plik nie przekracza
    ustalonego limitu rozmiaru i jest dozwolonego typu.}

    \wfczpost{Wiadomość z załącznikiem jest zapisana w systemie, a plik jest
    dostępny do pobrania dla uczestników czatu.}

    \wfczexceptions{Przekroczony limit rozmiaru, nieobsługiwany typ pliku,
        błąd przesyłania, brak miejsca w magazynie plików.}

    \wfczstakeholder{U3}

    \wfczresponsible{Adam Langmesser}

    \wfcznote{\textemdash\ brak}

    \wfczrelated{\hyperref[woczat:send-message]{WOCZAT-01}.}
}

% =========================================
% WFCZAT-03 – Wysyłanie wiadomości prywatnych
% =========================================

\wfczatcard
{wfczat:private-messages}
{Wysyłanie wiadomości prywatnych}
{03}
{S}
{
    \wfczdesc{System umożliwia prowadzenie prywatnych rozmów 1:1
    pomiędzy dwoma użytkownikami.}

    \wfczaccept{%
        \begin{itemize}
            \item Użytkownik może rozpocząć nowy czat 1:1 z innym użytkownikiem lub kontynuować istniejący.
            \item Wiadomości z prywatnego czatu są widoczne wyłącznie dla tych dwóch użytkowników.
            \item Nowe wiadomości pojawiają się w czasie zbliżonym do rzeczywistego bez konieczności przeładowania strony.
        \end{itemize}
    }

    \wfczinput{Identyfikator nadawcy, identyfikator odbiorcy,
        treść wiadomości oraz ewentualne załączniki.}

    \wfczpre{Obaj użytkownicy posiadają aktywne konta.}

    \wfczpost{Wiadomości prywatne są zapisane w historii czatu 1:1 i dostępne
    po ponownym otwarciu rozmowy.}

    \wfczexceptions{Brak połączenia z serwerem, błąd walidacji treści.}

    \wfczstakeholder{U3}

    \wfczresponsible{Adam Langmesser}

    \wfcznote{\textemdash\ brak}

    \wfczrelated{\hyperref[woczat:send-message]{WOCZAT-01}.}
}

% =========================================
% WFCZAT-04 – Wysyłanie wiadomości do wielu osób jednocześnie
% =========================================

\wfczatcard
{wfczat:group-messages}
{Wysyłanie wiadomości do wielu osób jednocześnie}
{04}
{S}
{
    \wfczdesc{System umożliwia wysyłanie jednej wiadomości do wielu użytkowników
    w ramach czatu grupowego.}

    \wfczaccept{%
        \begin{itemize}
            \item Wiadomość wysłana w czacie grupowym jest dostarczana wszystkim jego członkom.
            \item \gls{ui} wyraźnie wskazuje, że rozmowa to czat grupowy (np.\ nazwa, avatar, liczba uczestników).
            \item Nowi uczestnicy dołączeni do czatu widzą historię rozmowy.
        \end{itemize}
    }

    \wfczinput{Identyfikator czatu grupowego, identyfikator nadawcy, treść wiadomości
    oraz ewentualne załączniki.}

    \wfczpre{Czat typu grupowego istnieje, a użytkownik wysyłający wiadomość jest jego członkiem.}

    \wfczpost{Wiadomość jest zapisana w historii czatu i widoczna dla wszystkich aktualnych
    uczestników.}

    \wfczexceptions{Brak członkostwa nadawcy w czacie, błąd autoryzacji,
        przekroczenie limitu użytkowników w czacie.}

    \wfczstakeholder{U3}

    \wfczresponsible{Adam Langmesser}

    \wfcznote{\textemdash\ brak}

    \wfczrelated{\hyperref[woczat:send-message]{WOCZAT-01}.}
}

% =========================================
% WFCZAT-05 – Rozpoczynanie nowego czatu
% =========================================

\wfczatcard
{wfczat:create-chat}
{Rozpoczynanie nowego czatu}
{05}
{S}
{
    \wfczdesc{System umożliwia użytkownikowi utworzenie nowego czatu prywatnego
    lub grupowego oraz dodanie do niego wskazanych uczestników.}

    \wfczaccept{%
        \begin{itemize}
            \item Użytkownik może zainicjować nowy czat z widoku listy czatów lub profilu innego użytkownika.
            \item Po wysłaniu pierwszej wiadomości przez twórcę, czat pojawia się na liście czatów wszystkich jego członków.
        \end{itemize}
    }

    \wfczinput{Typ czatu (prywatny/grupowy), lista identyfikatorów uczestników, identyfikator tworzącego użytkownika.}

    \wfczpre{Użytkownik jest zalogowany i ma uprawnienia do zakładania nowych czatów.}

    \wfczpost{Nowy czat jest zapisany w \glslink{baza-danych}{bazie danych}, powiązany z uczestnikami
    i widoczny w ich listach czatów.}

    \wfczexceptions{Próba utworzenia czatu z nieistniejącym użytkownikiem,
        przekroczenie maksymalnej liczby uczestników, błąd zapisu w \glslink{baza-danych}{bazie danych}.}

    \wfczstakeholder{U3}

    \wfczresponsible{Adam Langmesser}

    \wfcznote{\textemdash\ brak}

    \wfczrelated{\hyperref[woczat:create-chat]{WOCZAT-04}.}
}

% =========================================
% WFCZAT-06 – Wysyłanie emotikonów
% =========================================

\wfczatcard
{wfczat:emoticons}
{Wysyłanie emotikonów}
{06}
{S}
{
    \wfczdesc{System umożliwia wysyłanie emotikonów (\gls{emoji}) w treści wiadomości
    na czacie.}

    \wfczaccept{%
        \begin{itemize}
            \item Użytkownik może wstawiać emotikony z wbudowanego panelu wyboru \gls{emoji}.
        \end{itemize}
    }

    \wfczinput{Identyfikator czatu, identyfikator nadawcy, treść wiadomości
    zawierająca \gls{emoji}.}

    \wfczpre{Użytkownik jest zalogowany i ma dostęp do czatu.}

    \wfczpost{Wiadomość z emotikonami jest zapisana w historii czatu i
    wyświetlana u wszystkich uczestników.}

    \wfczexceptions{Problemy z kodowaniem znaków, brak wsparcia dla części \gls{emoji}
    w danej przeglądarce lub systemie.}

    \wfczstakeholder{U3}

    \wfczresponsible{Adam Langmesser}

    \wfcznote{\textemdash\ brak}

    \wfczrelated{\hyperref[woczat:send-message]{WOCZAT-01}.}
}

% =========================================
% WFCZAT-07 – Dostępność czatu po utworzeniu
% =========================================

\wfczatcard
{wfczat:chat-availability}
{Dostępność czatu po utworzeniu}
{07}
{S}
{
    \wfczdesc{Czat po utworzeniu pozostaje dostępny dla jego członków;
    użytkownik może później odnaleźć czat i wrócić do wcześniejszej konwersacji.}

    \wfczaccept{%
        \begin{itemize}
            \item Utworzone czaty pojawiają się na liście czatów użytkownika.
            \item Po ponownym zalogowaniu użytkownik może otworzyć istniejący czat
            i zobaczyć jego historię.
        \end{itemize}
    }

    \wfczinput{Kontekst zalogowanego użytkownika oraz parametry filtrów listy czatów.}

    \wfczpre{Czat został wcześniej utworzony, a użytkownik jest jego członkiem.}

    \wfczpost{Lista czatów użytkownika prezentuje wszystkie czaty, do których
    ma on dostęp.}

    \wfczexceptions{Błąd autoryzacji, błąd \glslink{baza-danych}{bazy danych} podczas
    pobierania listy czatów.}

    \wfczstakeholder{U3}

    \wfczresponsible{Adam Langmesser}

    \wfcznote{\textemdash\ brak}

    \wfczrelated{\hyperref[woczat:browse-history]{WOCZAT-03}.}
}

% =========================================
% WFCZAT-08 – Edytowanie nazwy czatu grupowego
% =========================================

\wfczatcard
{wfczat:edit-group-name}
{Edytowanie nazwy czatu grupowego}
{08}
{S}
{
    \wfczdesc{System umożliwia zmianę nazwy istniejącego czatu grupowego
    przez członka czatu.}

    \wfczaccept{%
        \begin{itemize}
            \item Użytkownik może edytować nazwę z poziomu ustawień czatu.
            \item Nowa nazwa jest natychmiast widoczna na liście czatów u wszystkich uczestników.
            \item Historia wiadomości pozostaje niezmieniona po zmianie nazwy czatu.
        \end{itemize}
    }

    \wfczinput{Identyfikator czatu grupowego, nowa nazwa czatu, identyfikator użytkownika
    wykonującego operację.}

    \wfczpre{Użytkownik jest członkiem czatu.}

    \wfczpost{Nazwa czatu jest zaktualizowana w \glslink{baza-danych}{bazie danych} i w \glslink{ui}{interfejsie użytkownika}.}

    \wfczexceptions{Błąd walidacji nowej nazwy,
        błąd zapisu w \glslink{baza-danych}{bazie danych}.}

    \wfczstakeholder{U3}

    \wfczresponsible{Adam Langmesser}

    \wfcznote{\textemdash\ brak}

    \wfczrelated{\hyperref[woczat:edit-chat]{WOCZAT-02}.}
}

% =========================================
% WFCZAT-09 – Edycja zdjęcia czatu grupowego
% =========================================

\wfczatcard
{wfczat:edit-group-avatar}
{Edycja zdjęcia czatu grupowego}
{09}
{S}
{
    \wfczdesc{System umożliwia zmianę obrazu/avatara reprezentującego czat grupowy.}

    \wfczaccept{%
        \begin{itemize}
            \item Członek czatu może wgrać nowe zdjęcie czatu grupowego.
            \item Zmienione zdjęcie jest widoczne na liście czatów i w nagłówku rozmowy.
            \item Niepoprawne formaty lub zbyt duże pliki są odrzucane z informacją o błędzie.
        \end{itemize}
    }

    \wfczinput{Identyfikator czatu grupowego, nowy plik graficzny,
        identyfikator użytkownika wykonującego operację.}

    \wfczpre{Użytkownik posiada uprawnienia do edycji ustawień czatu grupowego.}

    \wfczpost{Nowe zdjęcie czatu jest zapisane w systemie i używane w \glslink{ui}{interfejsie użytkownika}.}

    \wfczexceptions{Nieobsługiwany format obrazu, przekroczony limit rozmiaru,
        błąd przesyłania lub zapisu w magazynie plików.}

    \wfczstakeholder{U3}

    \wfczresponsible{Adam Langmesser}

    \wfcznote{\textemdash\ brak}

    \wfczrelated{\hyperref[woczat:edit-chat]{WOCZAT-02}.}
}

% =========================================
% WFCZAT-10 – Edycja wysłanej wiadomości
% =========================================

\wfczatcard
{wfczat:edit-message}
{Edycja wysłanej wiadomości}
{10}
{W}
{
    \wfczdesc{System umożliwia użytkownikowi edycję treści wcześniej wysłanej
    wiadomości na czacie (np.\ poprawa literówki).}

    \wfczaccept{%
        \begin{itemize}
            \item Użytkownik może edytować swoje wiadomości przez ograniczony czas
            od momentu wysłania (konfigurowalny limit).
            \item Zmieniona wiadomość jest oznaczona etykietą „(edytowano)”.
            \item Odbiorcy widzą zaktualizowaną treść bez duplikowania wiadomości.
        \end{itemize}
    }

    \wfczinput{Identyfikator wiadomości, nowa treść wiadomości,
        identyfikator użytkownika wykonującego edycję.}

    \wfczpre{Użytkownik jest autorem wiadomości i mieści się w dozwolonym
    oknie czasowym edycji.}

    \wfczpost{Treść wiadomości w \glslink{baza-danych}{bazie danych} zostaje zaktualizowana, a widok
    czatu odświeżony u wszystkich uczestników.}

    \wfczexceptions{Próba edycji cudzej wiadomości, przekroczony limit czasu,
        błąd walidacji treści, brak połączenia z serwerem.}

    \wfczstakeholder{U3}

    \wfczresponsible{Adam Langmesser}

    \wfcznote{\textemdash\ Ze względu na konieczność priorytetyzacji pracy zespół zdecydował się zrezygnować z danego wymagania.}

    \wfczrelated{\hyperref[woczat:send-message]{WOCZAT-01}.}
}

% =========================================
% WFCZAT-11 – Dodawanie użytkowników do istniejącego czatu
% =========================================

\wfczatcard
{wfczat:add-users-existing-chat}
{Dodawanie użytkowników do istniejącego czatu}
{11}
{S}
{
    \wfczdesc{System umożliwia dodawanie nowych użytkowników do już istniejącego
    czatu grupowego.}

    \wfczaccept{%
        \begin{itemize}
            \item Właściciel lub uprawniony użytkownik może wyszukać i dodać nowe osoby do czatu.
            \item Nowo dodani użytkownicy pojawiają się na liście uczestników
            i mogą od razu brać udział w rozmowie.
        \end{itemize}
    }

    \wfczinput{Identyfikator czatu grupowego, lista identyfikatorów użytkowników do dodania,
        identyfikator użytkownika wykonującego operację.}

    \wfczpre{Czat istnieje.}

    \wfczpost{Lista uczestników czatu jest zaktualizowana; nowe osoby są powiązane
    z czatem w \glslink{baza-danych}{bazie danych}.}

    \wfczexceptions{Dodanie użytkownika, który jest już członkiem czatu; przekroczenie maksymalnej liczby uczestników.}

    \wfczstakeholder{U3}

    \wfczresponsible{Adam Langmesser}

    \wfcznote{\textemdash\ brak}

    \wfczrelated{\hyperref[woczat:edit-chat]{WOCZAT-02}.}
}

% =========================================
% WFCZAT-12 – Wyświetlanie starszych wiadomości
% =========================================

\wfczatcard
{wfczat:load-older-messages}
{Wyświetlanie starszych wiadomości}
{12}
{S}
{
    \wfczdesc{System powinien domyślnie wyświetlać co najmniej ostatnie 20
    wiadomości w czacie, a starsze wiadomości pobierać na bieżąco podczas
    przewijania historii przez użytkownika.}

    \wfczaccept{%
        \begin{itemize}
            \item Po wejściu na czat użytkownik widzi minimum 20 ostatnich wiadomości.
            \item Przewijanie historii w górę automatycznie pobiera starsze wiadomości
            (mechanizm \gls{infinite-scroll}).
            \item Dociąganie wiadomości nie powoduje zauważalnych opóźnień interfejsu.
        \end{itemize}
    }

    \wfczinput{Identyfikator czatu, parametry \glslink{paginacja}{paginacji}.}

    \wfczpre{Użytkownik jest zalogowany i posiada dostęp do czatu;
    w \glslink{baza-danych}{bazie danych} istnieje historia rozmowy.}

    \wfczpost{Użytkownik ma możliwość przeglądania pełnej historii czatu.}

    \wfczexceptions{Brak połączenia z serwerem, błąd \glslink{paginacja}{paginacji},
        przekroczenie limitu zapytań do \gls{api}.}

    \wfczstakeholder{U3}

    \wfczresponsible{Adam Langmesser}

    \wfcznote{\textemdash\ brak}

    \wfczrelated{\hyperref[woczat:browse-history]{WOCZAT-03}.}
}
