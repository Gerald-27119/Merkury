%! Author = Adam
%! Date = 30/11/2025

\subsubsection{Wymagania funkcjonalne dla czatu}
\label{subsubsec:wymagania-funkcjonalne-dla-chatu}

% ============================
% WYMAGANIA FUNKCJONALNE DLA CZATU
% Identyfikatory: WFCZAT-XX
% ============================

\newcounter{wfczat}[chapter]
\renewcommand{\thewfczat}{\thechapter.\arabic{wfczat}}

% Szerokość części z treścią (3 prawe kolumny zlane w jedną)
\newlength{\wfczContentWidth}
\setlength{\wfczContentWidth}{0.7\textwidth}

% --------- Pola karty (wiersze) ---------

% Id + priorytet – 4 kolumny
\newcommand{\wfczpriority}[2]{%
    \textbf{Identyfikator:} & WFCZAT-#1 &
    \textbf{Priorytet:}     & #2 \\ \hline
}

% Wiersze z etykietą + treścią na 3 kolumny
\newcommand{\wfczname}[1]{\textbf{Nazwa:}              &
\multicolumn{3}{|p{\wfczContentWidth}|}{#1} \\ \hline}
\newcommand{\wfczdesc}[1]{\textbf{Opis:}               &
\multicolumn{3}{|p{\wfczContentWidth}|}{#1} \\ \hline}
\newcommand{\wfczaccept}[1]{\textbf{Kryteria akceptacji:} &
\multicolumn{3}{|p{\wfczContentWidth}|}{#1} \\ \hline}
\newcommand{\wfczinput}[1]{\textbf{Dane wejściowe:}    &
\multicolumn{3}{|p{\wfczContentWidth}|}{#1} \\ \hline}
\newcommand{\wfczpre}[1]{\textbf{Warunki początkowe:}  &
\multicolumn{3}{|p{\wfczContentWidth}|}{#1} \\ \hline}
\newcommand{\wfczpost}[1]{\textbf{Warunki końcowe:}    &
\multicolumn{3}{|p{\wfczContentWidth}|}{#1} \\ \hline}
\newcommand{\wfczexceptions}[1]{\textbf{Sytuacje wyjątkowe:} &
\multicolumn{3}{|p{\wfczContentWidth}|}{#1} \\ \hline}
\newcommand{\wfczimpl}[1]{\textbf{Szczegóły implementacji:} &
\multicolumn{3}{|p{\wfczContentWidth}|}{#1} \\ \hline}
\newcommand{\wfczstakeholder}[1]{\textbf{Udziałowiec:} &
\multicolumn{3}{|p{\wfczContentWidth}|}{#1} \\ \hline}
\newcommand{\wfczresponsible}[1]{\textbf{Realizator:}  &
\multicolumn{3}{|p{\wfczContentWidth}|}{#1} \\ \hline}
\newcommand{\wfczstatus}[1]{\textbf{Status:}           &
\multicolumn{3}{|p{\wfczContentWidth}|}{#1} \\ \hline}

% Karta wymagania funkcjonalnego dla czatu
% #1 – label
% #2 – nazwa
% #3 – numer / sufiks kodu (np. 01 → WFCZAT-01)
% #4 – priorytet (MoSCoW: M, S, C, W)
% #5 – reszta pól: \wfczdesc, \wfczaccept, ...
\newcommand{\wfczatcard}[5]{%
    \refstepcounter{wfczat}%
    {%
        \centering
        \begin{longtable}{|
                >{\columncolor{lightgray}}l|
            l|
                >{\columncolor{lightgray}}l|
            p{0.15\textwidth}|}
        \hline
        \rowcolor{lightgray}\multicolumn{4}{|c|}{\textbf{KARTA WYMAGANIA FUNKCJONALNEGO DLA CZATU}} \\ \hline
        \endfirsthead
        \hline
        \rowcolor{lightgray}\multicolumn{4}{|c|}{\textbf{KARTA WYMAGANIA FUNKCJONALNEGO DLA CZATU (cd.)}} \\ \hline
        \endhead
        \wfczpriority{#3}{#4}
        \wfczname{#2}
        #5
        \end{longtable}
        \par
    }%
    \vspace{3pt}%
    \textbf{Tabela \thewfczat:} Wymaganie funkcjonalne dla czatu: #2\label{#1}%
    \addcontentsline{lot}{table}{Tabela \thewfczat: Wymaganie funkcjonalne dla czatu: #2}%
}

% =========================================
% WFCZAT-01 – Wysyłanie GIF-ów
% =========================================

\wfczatcard
{wfczat:send-gif}
{Wysyłanie GIF-ów}
{01}
{M}
{
    \wfczdesc{System umożliwia wysyłanie w wiadomościach animowanych obrazów GIF
    w ramach wybranego czatu (1:1 lub grupowego).}

    \wfczaccept{%
        \begin{itemize}
            \item Użytkownik może wybrać animowany obraz GIF z dysku lub z wbudowanego pickera i dołączyć go do wiadomości.
            \item Po wysłaniu GIF wyświetla się poprawnie w treści czatu u wszystkich uczestników.
            \item Błędne lub zbyt duże pliki GIF są odrzucane z czytelnym komunikatem o błędzie.
        \end{itemize}
    }

    \wfczinput{Identyfikator czatu, identyfikator nadawcy, wybrany GIF
        (plik lub identyfikator z usługi zewnętrznej).}

    \wfczpre{Użytkownik jest zalogowany i jest członkiem danego czatu;
    połączenie z serwerem jest aktywne.}

    \wfczpost{Wiadomość z GIF-em jest zapisana w historii czatu i widoczna
    dla wszystkich uprawnionych uczestników.}

    \wfczexceptions{Brak połączenia z serwerem, przekroczony limit rozmiaru pliku,
        nieobsługiwany format, brak uprawnień do czatu.}

    \wfczimpl{Rozszerzenie mutacji GraphQL \texttt{createChatMessage} o typ
    wiadomości \emph{GIF}; przechowywanie pliku w usłudze składowania plików
        (np.\ Azure Blob Storage) lub przechowywanie identyfikatora GIF-a z usługi
        zewnętrznej.}

    \wfczstakeholder{Użytkownik zalogowany.}

    \wfczresponsible{Adam Langmesser}

    \wfczstatus{Zrealizowano}
}

% =========================================
% WFCZAT-02 – Wysyłanie plików
% =========================================

\wfczatcard
{wfczat:send-files}
{Wysyłanie plików}
{02}
{M}
{
    \wfczdesc{System umożliwia dołączanie i wysyłanie plików (np.\ dokumentów,
        zdjęć, archiwów) jako załączników do wiadomości na czacie.}

    \wfczaccept{%
        \begin{itemize}
            \item Użytkownik może wybrać jeden lub wiele plików z dysku i dołączyć je do wiadomości.
            \item Odbiorca widzi w czacie element reprezentujący plik (nazwa, rozmiar, ikona typu).
            \item Kliknięcie w załącznik umożliwia pobranie lub otwarcie pliku.
        \end{itemize}
    }

    \wfczinput{Identyfikator czatu, identyfikator nadawcy, co najmniej jeden
    plik do wysłania.}

    \wfczpre{Użytkownik jest zalogowany, należy do danego czatu; plik nie przekracza
    ustalonego limitu rozmiaru i jest dozwolonego typu.}

    \wfczpost{Wiadomość z załącznikiem jest zapisana w systemie, a plik jest
    dostępny do pobrania dla uczestników czatu.}

    \wfczexceptions{Przekroczony limit rozmiaru, nieobsługiwany typ pliku,
        błąd przesyłania, brak miejsca w magazynie plików.}

    \wfczimpl{Przesyłanie plików przez dedykowany endpoint uploadu lub
    część mutacji \texttt{createChatMessage}; składowanie plików w zewnętrznej
    usłudze (np.\ Azure Blob Storage) i przechowywanie w bazie jedynie metadanych.}

    \wfczstakeholder{Użytkownik zalogowany.}

    \wfczresponsible{Adam Langmesser}

    \wfczstatus{Zrealizowano}
}

% =========================================
% WFCZAT-03 – Wysyłanie wiadomości prywatnych
% =========================================

\wfczatcard
{wfczat:private-messages}
{Wysyłanie wiadomości prywatnych}
{03}
{M}
{
    \wfczdesc{System umożliwia prowadzenie prywatnych rozmów 1:1
    pomiędzy dwoma użytkownikami platformy Merkury.}

    \wfczaccept{%
        \begin{itemize}
            \item Użytkownik może rozpocząć nowy czat 1:1 z innym użytkownikiem lub kontynuować istniejący.
            \item Wiadomości z prywatnego czatu są widoczne wyłącznie dla tych dwóch użytkowników.
            \item Nowe wiadomości pojawiają się w czasie zbliżonym do rzeczywistego bez konieczności przeładowania strony.
        \end{itemize}
    }

    \wfczinput{Identyfikator nadawcy, identyfikator odbiorcy,
        treść wiadomości oraz ewentualne załączniki.}

    \wfczpre{Obaj użytkownicy posiadają aktywne konta i nie zablokowali się nawzajem.}

    \wfczpost{Wiadomości prywatne są zapisane w historii czatu 1:1 i dostępne
    po ponownym otwarciu rozmowy.}

    \wfczexceptions{Brak uprawnień (np.\ zablokowany użytkownik),
        brak połączenia z serwerem, błąd walidacji treści.}

    \wfczimpl{Wydzielenie typu czatu \emph{PRIVATE} w modelu domenowym;
    mutacja \texttt{createChat} dla utworzenia czatu 1:1 oraz
    \texttt{createChatMessage} dla wysyłania kolejnych wiadomości.}

    \wfczstakeholder{Użytkownik zalogowany.}

    \wfczresponsible{Adam Langmesser}

    \wfczstatus{Zrealizowano}
}

% =========================================
% WFCZAT-04 – Wysyłanie wiadomości do wielu osób jednocześnie
% =========================================

\wfczatcard
{wfczat:group-messages}
{Wysyłanie wiadomości do wielu osób jednocześnie}
{04}
{M}
{
    \wfczdesc{System umożliwia wysyłanie jednej wiadomości do wielu użytkowników
    w ramach czatu grupowego.}

    \wfczaccept{%
        \begin{itemize}
            \item Wiadomość wysłana w czacie grupowym jest dostarczana wszystkim jego członkom.
            \item Interfejs wyraźnie wskazuje, że rozmowa to czat grupowy (np.\ nazwa, avatar, liczba uczestników).
            \item Nowi uczestnicy dołączeni do czatu widzą historię rozmowy zgodnie z przyjętą polityką prywatności.
        \end{itemize}
    }

    \wfczinput{Identyfikator czatu grupowego, identyfikator nadawcy, treść wiadomości
    oraz ewentualne załączniki.}

    \wfczpre{Czat typu grupowego istnieje, a użytkownik wysyłający wiadomość jest jego członkiem.}

    \wfczpost{Wiadomość jest zapisana w historii czatu i widoczna dla wszystkich aktualnych
    uczestników.}

    \wfczexceptions{Brak członkostwa nadawcy w czacie, błąd autoryzacji,
        przekroczenie limitu użytkowników w czacie.}

    \wfczimpl{Typ czatu \emph{GROUP} w modelu domenowym; wiadomości powiązane z jednym czatem
    i wieloma użytkownikami; dystrybucja powiadomień i aktualizacji po WebSocketach.}

    \wfczstakeholder{Użytkownik zalogowany.}

    \wfczresponsible{Adam Langmesser}

    \wfczstatus{Zrealizowano}
}

% =========================================
% WFCZAT-05 – Rozpoczynanie nowego czatu (Dodawanie do czatu)
% =========================================

\wfczatcard
{wfczat:create-chat}
{Rozpoczynanie nowego czatu}
{05}
{M}
{
    \wfczdesc{System umożliwia użytkownikowi utworzenie nowego czatu prywatnego
    lub grupowego oraz dodanie do niego wskazanych uczestników.}

    \wfczaccept{%
        \begin{itemize}
            \item Użytkownik może zainicjować nowy czat z widoku listy czatów lub profilu innego użytkownika.
            \item Po utworzeniu czat pojawia się na liście czatów wszystkich jego członków.
            \item Uczestnicy mogą natychmiast rozpocząć wymianę wiadomości.
        \end{itemize}
    }

    \wfczinput{Typ czatu (prywatny/grupowy), nazwa czatu (dla grup),
        lista identyfikatorów uczestników, identyfikator tworzącego użytkownika.}

    \wfczpre{Użytkownik jest zalogowany i ma uprawnienia do zakładania nowych czatów.}

    \wfczpost{Nowy czat jest zapisany w bazie danych, powiązany z uczestnikami
    i widoczny w ich listach czatów.}

    \wfczexceptions{Próba utworzenia czatu z nieistniejącym użytkownikiem,
        przekroczenie maksymalnej liczby uczestników, błąd zapisu w bazie.}

    \wfczimpl{Mutacja GraphQL \texttt{createChat}; walidacja listy uczestników
    oraz ewentualnej unikalności czatów 1:1.}

    \wfczstakeholder{Użytkownik zalogowany.}

    \wfczresponsible{Adam Langmesser}

    \wfczstatus{Zrealizowano}
}

% =========================================
% WFCZAT-06 – Wysyłanie emotikonów
% =========================================

\wfczatcard
{wfczat:emoticons}
{Wysyłanie emotikonów}
{06}
{M}
{
    \wfczdesc{System umożliwia wysyłanie emotikonów (emoji) w treści wiadomości
    na czacie.}

    \wfczaccept{%
        \begin{itemize}
            \item Użytkownik może wstawiać emotikony z wbudowanego pickera emoji.
            \item Emotikony są poprawnie renderowane na różnych urządzeniach (desktop, mobile).
            \item Kopiowanie lub edycja wiadomości zachowuje wstawione emotikony.
        \end{itemize}
    }

    \wfczinput{Identyfikator czatu, identyfikator nadawcy, treść wiadomości
    zawierająca emoji.}

    \wfczpre{Użytkownik jest zalogowany i ma dostęp do czatu.}

    \wfczpost{Wiadomość z emotikonami jest zapisana w historii czatu i
    wyświetlana u wszystkich uczestników.}

    \wfczexceptions{Problemy z kodowaniem znaków, brak wsparcia dla części emoji
    w danej przeglądarce lub systemie.}

    \wfczimpl{Wykorzystanie znaków Unicode dla emoji; podpięcie komponentu pickera
        (np.\ \texttt{emoji-mart}) po stronie frontendu.}

    \wfczstakeholder{Użytkownik zalogowany.}

    \wfczresponsible{Adam Langmesser}

    \wfczstatus{Zrealizowano}
}

% =========================================
% WFCZAT-07 – Dostępność czatu po utworzeniu
% =========================================

\wfczatcard
{wfczat:chat-availability}
{Dostępność czatu po utworzeniu}
{07}
{M}
{
    \wfczdesc{Czat po utworzeniu pozostaje dostępny dla jego członków;
    użytkownik może później odnaleźć czat i wrócić do wcześniejszej konwersacji.}

    \wfczaccept{%
        \begin{itemize}
            \item Utworzone czaty pojawiają się na liście czatów użytkownika.
            \item Po ponownym zalogowaniu użytkownik może otworzyć istniejący czat
            i zobaczyć jego historię.
            \item Usunięcie użytkownika z czatu powoduje usunięcie go z jego listy czatów.
        \end{itemize}
    }

    \wfczinput{Kontekst zalogowanego użytkownika oraz parametry filtrów listy czatów.}

    \wfczpre{Czat został wcześniej utworzony, a użytkownik jest jego członkiem.}

    \wfczpost{Lista czatów użytkownika prezentuje wszystkie czaty, do których
    ma on dostęp.}

    \wfczexceptions{Błąd autoryzacji (sesja wygasła), błąd bazy danych podczas
    pobierania listy czatów.}

    \wfczimpl{Zapytanie GraphQL \texttt{chatsByCurrentUser} wykorzystywane
    w panelu bocznym listy czatów; cache po stronie klienta.}

    \wfczstakeholder{Użytkownik zalogowany.}

    \wfczresponsible{Adam Langmesser}

    \wfczstatus{Zrealizowano}
}

% =========================================
% WFCZAT-08 – Edytowanie nazwy czatu grupowego
% =========================================

\wfczatcard
{wfczat:edit-group-name}
{Edytowanie nazwy czatu grupowego}
{08}
{M}
{
    \wfczdesc{System umożliwia zmianę nazwy istniejącego czatu grupowego
    przez jego właściciela lub uprawnionego administratora.}

    \wfczaccept{%
        \begin{itemize}
            \item Uprawniony użytkownik może edytować nazwę z poziomu ustawień czatu.
            \item Nowa nazwa jest natychmiast widoczna na liście czatów u wszystkich uczestników.
            \item Historia wiadomości pozostaje niezmieniona po zmianie nazwy czatu.
        \end{itemize}
    }

    \wfczinput{Identyfikator czatu grupowego, nowa nazwa czatu, identyfikator użytkownika
    wykonującego operację.}

    \wfczpre{Użytkownik posiada uprawnienia właściciela/administratora
    w danym czacie grupowym.}

    \wfczpost{Nazwa czatu jest zaktualizowana w bazie danych i w interfejsie użytkownika.}

    \wfczexceptions{Brak uprawnień do edycji, zbyt długa lub pusta nazwa,
        błąd zapisu w bazie danych.}

    \wfczimpl{Mutacja GraphQL \texttt{updateChatMetadata} z polem \texttt{name};
    walidacja uprawnień po stronie backendu.}

    \wfczstakeholder{Właściciel i uczestnicy czatu grupowego.}

    \wfczresponsible{Adam Langmesser}

    \wfczstatus{Zrealizowano}
}

% =========================================
% WFCZAT-09 – Edycja zdjęcia czatu grupowego
% =========================================

\wfczatcard
{wfczat:edit-group-avatar}
{Edycja zdjęcia czatu grupowego}
{09}
{M}
{
    \wfczdesc{System umożliwia zmianę obrazu/avatara reprezentującego czat grupowy.}

    \wfczaccept{%
        \begin{itemize}
            \item Właściciel lub administrator może wgrać nowe zdjęcie czatu grupowego.
            \item Zmienione zdjęcie jest widoczne na liście czatów i w nagłówku rozmowy.
            \item Niepoprawne formaty lub zbyt duże pliki są odrzucane z informacją o błędzie.
        \end{itemize}
    }

    \wfczinput{Identyfikator czatu grupowego, nowy plik graficzny,
        identyfikator użytkownika wykonującego operację.}

    \wfczpre{Użytkownik posiada uprawnienia do edycji ustawień czatu grupowego.}

    \wfczpost{Nowe zdjęcie czatu jest zapisane w systemie i używane w interfejsie.}

    \wfczexceptions{Nieobsługiwany format obrazu, przekroczony limit rozmiaru,
        błąd przesyłania lub zapisu w magazynie plików.}

    \wfczimpl{Pole \texttt{avatarUrl} w encji czatu; przesyłanie pliku do usługi
    składowania plików; aktualizacja adresu URL w bazie danych.}

    \wfczstakeholder{Właściciel i uczestnicy czatu grupowego.}

    \wfczresponsible{Adam Langmesser}

    \wfczstatus{Zrealizowano}
}

% =========================================
% WFCZAT-10 – Edycja wysłanej wiadomości
% =========================================

\wfczatcard
{wfczat:edit-message}
{Edycja wysłanej wiadomości}
{10}
{M}
{
    \wfczdesc{System umożliwia użytkownikowi edycję treści wcześniej wysłanej
    wiadomości na czacie (np.\ poprawa literówki).}

    \wfczaccept{%
        \begin{itemize}
            \item Użytkownik może edytować swoje wiadomości przez ograniczony czas
            od momentu wysłania (konfigurowalny limit).
            \item Zmieniona wiadomość jest oznaczona etykietą „(edytowano)”.
            \item Odbiorcy widzą zaktualizowaną treść bez duplikowania wiadomości.
        \end{itemize}
    }

    \wfczinput{Identyfikator wiadomości, nowa treść wiadomości,
        identyfikator użytkownika wykonującego edycję.}

    \wfczpre{Użytkownik jest autorem wiadomości i mieści się w dozwolonym
    oknie czasowym edycji.}

    \wfczpost{Treść wiadomości w bazie danych zostaje zaktualizowana, a widok
    czatu odświeżony u wszystkich uczestników.}

    \wfczexceptions{Próba edycji cudzej wiadomości, przekroczony limit czasu,
        błąd walidacji treści, brak połączenia z serwerem.}

    \wfczimpl{Mutacja GraphQL \texttt{updateChatMessage}; przechowywanie znacznika
    \texttt{editedAt} oraz dystrybucja aktualizacji poprzez WebSocket.}

    \wfczstakeholder{Użytkownik zalogowany oraz odbiorcy wiadomości.}

    \wfczresponsible{Adam Langmesser}

    \wfczstatus{Zrealizowano}
}

% =========================================
% WFCZAT-11 – Usunięcie wysłanej wiadomości
% =========================================

\wfczatcard
{wfczat:delete-message}
{Usunięcie wysłanej wiadomości}
{11}
{M}
{
    \wfczdesc{System umożliwia usunięcie wysłanej wiadomości z czatu
        (dla siebie lub dla wszystkich uczestników, zgodnie z konfiguracją).}

    \wfczaccept{%
        \begin{itemize}
            \item Autor wiadomości może ją usunąć w dozwolonym przedziale czasowym.
            \item W miejscu usuniętej wiadomości może pojawić się informacja
            typu „Wiadomość usunięta”.
            \item Usunięcie jednej wiadomości nie narusza integralności pozostałej historii czatu.
        \end{itemize}
    }

    \wfczinput{Identyfikator wiadomości, tryb usunięcia (dla siebie / dla wszystkich),
        identyfikator użytkownika.}

    \wfczpre{Użytkownik jest autorem wiadomości lub posiada uprawnienia
    moderacyjne w czacie.}

    \wfczpost{Wiadomość jest oznaczona jako usunięta i nie jest prezentowana
    w pierwotnej treści uczestnikom czatu.}

    \wfczexceptions{Próba usunięcia cudzej wiadomości bez uprawnień,
        przekroczony limit czasu, błąd zapisu w bazie danych.}

    \wfczimpl{Mutacja \texttt{deleteChatMessage} lub aktualizacja pola
    \texttt{deletedAt}; odświeżenie widoku w kliencie przez WebSocket.}

    \wfczstakeholder{Użytkownik zalogowany oraz moderatorzy czatu.}

    \wfczresponsible{Adam Langmesser}

    \wfczstatus{Zrealizowano}
}

% =========================================
% WFCZAT-12 – Dodawanie użytkowników do istniejącego czatu
% =========================================

\wfczatcard
{wfczat:add-users-existing-chat}
{Dodawanie użytkowników do istniejącego czatu}
{12}
{M}
{
    \wfczdesc{System umożliwia dodawanie nowych użytkowników do już istniejącego
    czatu grupowego.}

    \wfczaccept{%
        \begin{itemize}
            \item Właściciel lub uprawniony użytkownik może wyszukać i dodać nowe osoby do czatu.
            \item Nowo dodani użytkownicy pojawiają się na liście uczestników
            i mogą od razu brać udział w rozmowie.
            \item Dodanie uczestnika jest widoczne dla pozostałych w formie komunikatu systemowego.
        \end{itemize}
    }

    \wfczinput{Identyfikator czatu grupowego, lista identyfikatorów użytkowników do dodania,
        identyfikator użytkownika wykonującego operację.}

    \wfczpre{Czat typu grupowego istnieje; użytkownik wykonujący operację ma odpowiednie
    uprawnienia.}

    \wfczpost{Lista uczestników czatu jest zaktualizowana; nowe osoby są powiązane
    z czatem w bazie danych.}

    \wfczexceptions{Dodanie użytkownika, który nie ma konta, został zablokowany
    lub jest już członkiem czatu; przekroczenie maksymalnej liczby uczestników.}

    \wfczimpl{Mutacja GraphQL \texttt{updateChatParticipants} z listą identyfikatorów
    uczestników; walidacja po stronie backendu.}

    \wfczstakeholder{Właściciel czatu grupowego oraz jego uczestnicy.}

    \wfczresponsible{Adam Langmesser}

    \wfczstatus{Zrealizowano}
}

% =========================================
% WFCZAT-13 – Wyświetlanie starszych wiadomości
% =========================================

\wfczatcard
{wfczat:load-older-messages}
{Wyświetlanie starszych wiadomości}
{13}
{M}
{
    \wfczdesc{System powinien domyślnie wyświetlać co najmniej ostatnie 20
    wiadomości w czacie, a starsze wiadomości dociągać na bieżąco podczas
    przewijania historii przez użytkownika.}

    \wfczaccept{%
        \begin{itemize}
            \item Po wejściu na czat użytkownik widzi minimum 20 ostatnich wiadomości.
            \item Przewijanie historii w górę automatycznie pobiera starsze wiadomości
            (mechanizm \emph{infinite scroll}).
            \item Dociąganie wiadomości nie powoduje zauważalnych opóźnień interfejsu.
        \end{itemize}
    }

    \wfczinput{Identyfikator czatu, parametry paginacji (np.\ kursor, znacznik czasu).}

    \wfczpre{Użytkownik jest zalogowany i posiada dostęp do czatu;
    w bazie istnieje historia rozmowy.}

    \wfczpost{Użytkownik ma możliwość przeglądania pełnej historii czatu w zakresie
    swoich uprawnień.}

    \wfczexceptions{Brak połączenia z serwerem, błąd paginacji,
        przekroczenie limitu zapytań do API.}

    \wfczimpl{Zapytania GraphQL z paginacją kursorową; komponent
    \emph{infinite scroll} oparty na TanStack Query; cache i stronicowanie
    po stronie klienta.}

    \wfczstakeholder{Użytkownik zalogowany.}

    \wfczresponsible{Adam Langmesser}

    \wfczstatus{Zrealizowano}
}
