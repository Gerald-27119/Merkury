%! Author = Adam
%! Date = 23/11/2025

\subsection{Funkcjonalności dla chatu}
\label{subsec:funkcjonalnosci-dla-chatu}

\funcreqcard{tab:req-chat-list}
{Przeglądanie listy czatów użytkownika}
{FCHAT01}
{Wysoki}
{
    \frdesc{Jako użytkownik chcę mieć możliwość przeglądania listy czatów,
        do których należę, aby szybko odnajdywać i otwierać interesujące mnie
        rozmowy.}

    \fraccept{Po zalogowaniu użytkownik ma dostęp do listy czatów, do których
    jest przypisany. Lista wyświetla przynajmniej nazwę czatu oraz datę
    lub fragment ostatniej wiadomości. Dołączenie użytkownika do nowego
    czatu lub usunięcie z istniejącego jest odzwierciedlone na liście bez
    konieczności ponownego logowania. Próba pobrania listy czatów innego
    użytkownika kończy się odmową dostępu.}

    \frinput{Kontekst zalogowanego użytkownika (identyfikator użytkownika
    wynikający z sesji lub tokena). Opcjonalnie parametry filtrowania
    i paginacji (np. fraza wyszukiwania, numer strony).}

    \frpre{Użytkownik jest poprawnie zalogowany do systemu. W bazie danych
    istnieją czaty powiązane z danym użytkownikiem lub brak takich czatów.}

    \frpost{Użytkownik otrzymuje listę czatów, do których należy, lub
    informację o braku czatów. Użytkownik może wybrać czat z listy,
        aby przejść do widoku rozmowy.}

    \frexceptions{Brak czatów powiązanych z użytkownikiem (wyświetlenie
    komunikatu informującego). Błąd autoryzacji (np. utrata sesji) skutkujący
    przekierowaniem do ekranu logowania. Błąd komunikacji z serwerem
        (np. timeout) skutkujący prezentacją komunikatu o błędzie
        i możliwości ponownego odświeżenia listy.}

    \frimpl{Frontend: komponent listy czatów w panelu nawigacyjnym aplikacji
        (React), korzystający z klienta HTTP/GraphQL (np. \texttt{axios} lub
        \texttt{@tanstack/query}) do pobierania listy czatów aktualnie
        zalogowanego użytkownika. Obsługa stanu ładowania, błędu oraz pustej listy.
        Backend: zapytanie (np. GraphQL \texttt{chatsByCurrentUser} lub
        REST \texttt{/api/chats/me}) zwracające stronicowaną listę czatów,
        w której filtr po użytkowniku jest wymuszany po stronie serwera
        na podstawie kontekstu autoryzacji, a nie danych z klienta.}

    \frstakeholders{Zespół projektowy~2.1, promotor~2.2, droniarze~2.3.}

    \frrelated{FOXX -- Logowanie i rejestracja; FCHAT02 -- Wysyłanie
    wiadomości na czacie; FCHAT03 -- Otrzymywanie nowych wiadomości
    w czasie zbliżonym do rzeczywistego.}
}
