%! Author = Mateusz
%! Date = 03/11/2025

\subsubsection{Wymagania funkcjonalne dla motywu}
\label{subsubsec:wymagania-funkcjonalne-dla-motywu}

\newlength{\wfmotywLabelWidth}
\setlength{\wfmotywLabelWidth}{0.19\textwidth}

\newlength{\wfmotywColTwoWidth}
\setlength{\wfmotywColTwoWidth}{0.21\textwidth}

\newlength{\wfmotywColThreeWidth}
\setlength{\wfmotywColThreeWidth}{0.13\textwidth}

\newlength{\wfmotywColFourWidth}
\setlength{\wfmotywColFourWidth}{0.28\textwidth}

\newlength{\wfmotywContentWidth}
\setlength{\wfmotywContentWidth}{0.60\textwidth}

\newlength{\wfmotywHeaderHeight}
\setlength{\wfmotywHeaderHeight}{12mm}

\newcommand{\wfmotywthreecolcell}[1]{%
    \multicolumn{3}{|>{\raggedright\arraybackslash}p{\wfmotywContentWidth}|}{#1}%
}

\newcommand{\wfmotywthreecolcellpadded}[1]{%
    \multicolumn{3}{|>{\raggedright\arraybackslash}p{\wfmotywContentWidth}|}{%
        \vspace{0.4ex}%
        #1\par\vspace{0.4ex}%
    }%
}

\newcommand{\wfmotywHeaderRow}[1]{%
    \rowcolor{lightgray}%
    \multicolumn{4}{|c|}{%
        \parbox[c][\wfmotywHeaderHeight][c]{\linewidth}{%
            \centering\bfseries
            \vspace{1.2ex}%
            #1%
            \vspace{1.2ex}%
        }%
    }\\ \hline
}

% --- pola karty ---

\newcommand{\wfmotywpriority}[2]{%
    \textbf{Identyfikator:} & WFMOTYW-#1 &
    \textbf{Priorytet:}     & #2 \\ \hline
}

\newcommand{\wfmotywname}[1]{%
    \textbf{Nazwa:} &
    \wfmotywthreecolcell{#1} \\ \hline
}

\newcommand{\wfmotywdesc}[1]{%
    \textbf{Opis:} &
    \wfmotywthreecolcell{#1} \\ \hline
}

\newcommand{\wfmotywaccept}[1]{%
    \textbf{Kryteria akceptacji:} &
    \multicolumn{3}{|>{\raggedright\arraybackslash}p{\wfmotywContentWidth}|}{%
        \begingroup
        \setlength{\leftmargini}{1.2em}%
        \setlength{\topsep}{0pt}%
        \setlength{\partopsep}{0pt}%
        \setlength{\itemsep}{0.2ex}%
        \setlength{\parsep}{0pt}%
        \vspace*{-1.8ex}
        #1%
        \vspace*{-1.4ex}
        \endgroup
    }\\ \hline
}

\newcommand{\wfmotywstakeholder}[1]{%
    \textbf{Udziałowiec:} &
    \wfmotywthreecolcell{#1} \\ \hline
}

\newcommand{\wfmotywresponsible}[1]{%
    \textbf{Realizator:} &
    \wfmotywthreecolcell{#1} \\ \hline
}

\newcommand{\wfmotywrelated}[1]{%
    \textbf{Wymagania powiązane:} &
    \wfmotywthreecolcell{#1} \\ \hline
}

% --- szablon karty wymagania funkcjonalnego ---

\newcommand{\wfmotywatcard}[5]{%
    \refstepcounter{awc}%
    {%
        \centering
        \begin{longtable}{|
                >{\columncolor{lightgray}\raggedright\arraybackslash}p{\wfmotywLabelWidth}|
            p{\wfmotywColTwoWidth}|
                >{\columncolor{lightgray}\raggedright\arraybackslash}p{\wfmotywColThreeWidth}|
            p{\wfmotywColFourWidth}|}
        \hline
        \wfmotywHeaderRow{\shortstack{KARTA WYMAGANIA FUNKCJONALNEGO DLA \\ MOTYWU}}
        \endfirsthead
        \hline
        \wfmotywHeaderRow{\shortstack{KARTA WYMAGANIA FUNKCJONALNEGO DLA \\ MOTYWU (cd.)}}
        \endhead
        \wfmotywpriority{#3}{#4}
        \wfmotywname{#2}
        #5
        \end{longtable}
        \par
    }%
    \vspace{3pt}%
    \textbf{Tabela \theawc:} Wymaganie funkcjonalne dla motywu: #2\label{#1}%
    \addcontentsline{lot}{table}{Tabela \theawc: Wymaganie funkcjonalne dla motywu: #2}%
}

% =========================================
% WFMOTYW-01 – Zmiana motywu (jasny/ciemny)
% =========================================
\wfmotywatcard
{wfmotyw:change-theme}
{Możliwość zmiany motywu}
{01}
{S}
{
    \wfmotywdesc{System umożliwia użytkownikowi ręczne przełączenie motywu aplikacji pomiędzy trybem jasnym
        i ciemnym, aby dopasować wygląd interfejsu do preferencji użytkownika. Zmiana motywu następuje bez
        przeładowania strony i jest widoczna we wszystkich widokach aplikacji.}

    \wfmotywaccept{%
        \begin{itemize}
            \item Użytkownik ma dostęp do akcji przełączenia motywu (jasny/ciemny) w interfejsie aplikacji.
            \item Zmiana motywu następuje natychmiast po wykonaniu akcji (bez przeładowania strony).
            \item Po zmianie motywu wszystkie widoki aplikacji są prezentowane zgodnie z wybranym trybem.
        \end{itemize}
    }

    \wfmotywstakeholder{U3}
    \wfmotywresponsible{Mateusz Redosz}
    \wfmotywrelated{%
        \hyperref[womotyw:change-theme]{WOMOTYW-01},
        \hyperref[wpmotyw:theme-switch-under-200ms]{WPMOTYW-01},
        \hyperref[wpmotyw:theme-consistency]{WPMOTYW-02}.%
    }
}

% =========================================
% WFMOTYW-02 – Zapamiętanie preferencji motywu
% =========================================
\wfmotywatcard
{wfmotyw:remember-theme}
{Zapamiętanie ustawień motywu}
{02}
{S}
{
    \wfmotywdesc{System zapamiętuje preferencję motywu użytkownika (jasny/ciemny) w przeglądarce i przywraca ją
        przy kolejnym uruchomieniu aplikacji, aby zachować spójne doświadczenie użytkownika.}

    \wfmotywaccept{%
        \begin{itemize}
            \item Po ustawieniu motywu preferencja jest zapisywana lokalnie w przeglądarce.
            \item Po odświeżeniu strony wybrany wcześniej motyw pozostaje aktywny.
            \item Po ponownym uruchomieniu aplikacji (kolejna wizyta) system automatycznie przywraca ostatnio wybrany motyw.
        \end{itemize}
    }

    \wfmotywstakeholder{U3}
    \wfmotywresponsible{Mateusz Redosz}
    \wfmotywrelated{%
        \hyperref[womotyw:change-theme]{WOMOTYW-01},
        \hyperref[wpmotyw:no-flash-on-load]{WPMOTYW-03}.%
    }
}
