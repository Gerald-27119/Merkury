%! Author = Adam
%! Date = 24/11/2025

\section{Wymagania}
\label{sec:wymagania}

W niniejszym rozdziale przedstawiono wymagania stawiane projektowanemu systemowi.
Celem rozdziału jest zebranie w jednym miejscu oczekiwań interesariuszy oraz
usystematyzowanie ich w postaci jasno zdefiniowanych kategorii wymagań.

W pracy wyróżniono następujące typy wymagań:
\begin{itemize}
    \item \textbf{Wymagania ogólne} -- opisują system na wysokim poziomie,
    z perspektywy celu biznesowego i użytkownika, bez wchodzenia w szczegóły techniczne.
    Określają, jakie główne problemy ma rozwiązywać system i jakie korzyści ma
    dostarczać interesariuszom.

    \item \textbf{Wymagania dziedzinowe} -- wynikają ze specyfiki obszaru,
    w którym wykorzystywany jest system (domena problemu).
    Odzwierciedlają reguły biznesowe, ograniczenia prawne oraz ustaloną praktykę
    w danej dziedzinie i muszą być spełnione niezależnie od przyjętych rozwiązań technicznych.

    \item \textbf{Wymagania funkcjonalne} -- opisują konkretne funkcje systemu widziane
    z perspektywy użytkownika lub innego aktora.
    Definiują, jakie operacje system ma umożliwiać, jakie dane przetwarzać
    oraz jak reagować na określone zdarzenia i scenariusze użycia.

    \item \textbf{Wymagania pozafunkcjonalne} -- określają, \emph{jak} system ma realizować
    swoje funkcje, a nie \emph{co} ma robić.
    Obejmują m.in.\ wymagania dotyczące wydajności, bezpieczeństwa, niezawodności,
    użyteczności, skalowalności oraz utrzymywalności rozwiązania.

    \item \textbf{Wymagania dotyczące interfejsu z otoczeniem} -- opisują sposób komunikacji
    systemu z innymi systemami, urządzeniami lub usługami zewnętrznymi.
    Określają format wymienianych danych, wykorzystywane protokoły, kierunek
    i częstotliwość wymiany informacji oraz wymagania dotyczące integracji.

    \item \textbf{Wymagania dotyczące środowiska docelowego} -- definiują warunki techniczne,
    w jakich system ma być uruchamiany i eksploatowany.
    Dotyczą m.in.\ platformy sprzętowej i programowej, systemów operacyjnych,
    zasobów sieciowych oraz innych elementów infrastruktury niezbędnych
    do poprawnego działania aplikacji. (TODO do usuniecia?)
\end{itemize}

W dalszej części rozdziału wymagania zostały logicznie pogrupowane według
obszarów funkcjonalnych systemu: przedstawiono wymagania ogólne, a następnie
wymagania dotyczące czatu, forum, mapy oraz panelu użytkownika.

Ponadto wymagania mogą mieć jeden z dwóch statusów realizacji:
\begin{description}
    \item[\emph{Zrealizowano}] -- wymaganie zostało zrealizowane.
    \item[\emph{Anulowano}] -- wymaganie zostało anulowane.
\end{description}

\setcounter{secnumdepth}{4}
\setcounter{tocdepth}{4}

\subimport{sections/ogolne/}{glowny.tex}
\subimport{sections/funkcjonalne/}{wymagania-funkcjonalne.tex}
\subimport{sections/pozafunkcjonalne/}{wymagania-pozafunkcjonalne.tex}

\setcounter{secnumdepth}{3}
\setcounter{tocdepth}{3}
