%! Author = Mateusz
%! Date = 03/11/2025

\subsection{Funkcjonalności dla wyszukiwarki spotów}
\label{subsec:funkcjonalnosci-dla-wyszukiwarki-spotow}

\begin{requirementstab}[placement=H, label={tab:requirements:func1},caption={}]
    \id{jednoznaczny symbol np. FO1, FO2 .. }
    \priority{M}
    \name{Strona główna z podstawowymi filtrami}
    \descr{Jako użytkownik chcę mieć dostęp do strony głównej,
        która wyświetla najpopularniejsze spoty i listę spotów które
        można filtrować.}
    \acceptcrit{Użytkownik może zobaczyć listę najpopularniejszych miejsc.
    Lista zawiera zdjęcia, nazwę miejsca, lokalizację, tagi i opinie.
    Użytkownik może filtrować miejsca według lokalizacji
        (kraj, region, miasto).}
    \inputdata{Lokalizacja użytkownika (kraj, region, miasto).
    Ewentualne dane z bazy miejsc turystycznych.}
    \preconditions{Użytkownik nie musi być zalogowany.}
    \postconditions{Użytkownik widzi popularne miejsca z danego miasta
        (np. Gdańsk). Użytkownik może przejść do szczegółów danego miejsca.}
    \exceptions{}
    \implementation{}
    \sholder{}
    \reqrelated{}
\end{requirementstab}

\begin{requirementstab}[placement=H, label={tab:requirements:func1},caption={}]
    \id{jednoznaczny symbol np. FO1, FO2 .. }
    \priority{M}
    \name{Strona główna z zaawansowanymi filtrami}
    \descr{Jako użytkownik chcę mieć dostęp do strony głównej,
        która wyświetla najpopularniejsze spoty i listę spotów które
        można filtrować i sortować.}
    \acceptcrit{Użytkownik może zobaczyć listę najpopularniejszych miejsc.
    Lista zawiera zdjęcia, nazwę miejsca, lokalizację i krótki opis.
    Użytkownik może filtrować miejsca według lokalizacji
        (kraj, region, miasto), po ocenie oraz sortować po ocenie.}
    \inputdata{Lokalizacja użytkownika (kraj, region, miasto),
        sortowanie i filtry. Ewentualne dane z bazy miejsc turystycznych.}
    \preconditions{Użytkownik nie musi być zalogowany.}
    \postconditions{Użytkownik widzi popularne miejsca z danego miasta
        (np. Gdańsk). Użytkownik może przejść do szczegółów danego miejsca.}
    \exceptions{}
    \implementation{}
    \sholder{}
    \reqrelated{}
\end{requirementstab}