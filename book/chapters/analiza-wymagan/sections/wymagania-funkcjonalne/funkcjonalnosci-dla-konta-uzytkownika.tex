%! Author = Mateusz
%! Date = 01/11/2025

\subsection{Funkcjonalności dla konta użytkownika}
\label{subsec:funkcjonalnosci-dla-konta-uzytkownika}

\begin{requirementstab}[label={tab:requirements:func1},caption={}]
    \id{jednoznaczny symbol np. FO1, FO2 .. }
    \priority{M}
    \name{Profil użytkownika}
    \descr{Jako użytkownik chcę mieć dostęp do strony profilu, aby sprawdzić informacje o swoim profilu.}
    \acceptcrit{Użytkownik może zobaczyć informacje o ilości znajomych,
        obserwowanych i obserwujących oraz o najpopularniejszych zdjęciach.}
    \inputdata{Lista zdjęć i liczby znajomych, obserwujących i obserwowanych.}
    \preconditions{Użytkownik jest zalogowany}
    \postconditions{Wyświetlenie się informacji}
    \exceptions{}
    \implementation{}
    \sholder{}
    \reqrelated{}
\end{requirementstab}

\begin{requirementstab}[label={tab:requirements:func1},caption={}]
    \id{jednoznaczny symbol np. FO1, FO2 .. }
    \priority{M}
    \name{Lista dodanych Spotów}
    \descr{Jako użytkownik chcę mieć dostęp do sprawdzenia listy dodanych spotów.}
    \acceptcrit{Użytkownik może zobaczyć listę danych spotów.}
    \inputdata{Lista dodanych spotów.}
    \preconditions{Użytkownik jest zalogowany.}
    \postconditions{Wyświetlenie się informacji.}
    \exceptions{}
    \implementation{}
    \sholder{}
    \reqrelated{}
\end{requirementstab}

\begin{requirementstab}[label={tab:requirements:func1},caption={}]
    \id{jednoznaczny symbol np. FO1, FO2 .. }
    \priority{M}
    \name{Dodanie Spotów}
    \descr{Jako użytkownik chcę mieć dostęp do formularza do dodania spota.}
    \acceptcrit{Użytkownik ma dostęp do formularza do dodania spota.}
    \inputdata{Formularz}
    \preconditions{Użytkownik jest zalogowany.}
    \postconditions{Wyświetlenie się formularza.}
    \exceptions{}
    \implementation{}
    \sholder{}
    \reqrelated{}
\end{requirementstab}

\begin{requirementstab}[label={tab:requirements:func1},caption={}]
    \id{jednoznaczny symbol np. FO1, FO2 .. }
    \priority{M}
    \name{Lista zdjęć}
    \descr{Jako użytkownik chcę mieć dostęp do listy zdjęć,
        które dodałem na forum oraz do komentarzy pod spotem.}
    \acceptcrit{Użytkownik ma dostęp do listy zdjęć.}
    \inputdata{Lista zdjęć.}
    \preconditions{Użytkownik jest zalogowany.}
    \postconditions{Wyświetlenie się informacji.}
    \exceptions{}
    \implementation{}
    \sholder{}
    \reqrelated{}
\end{requirementstab}

\begin{requirementstab}[label={tab:requirements:func1},caption={}]
    \id{jednoznaczny symbol np. FO1, FO2 .. }
    \priority{M}
    \name{Lista filmów}
    \descr{Jako użytkownik chcę mieć dostęp do listy filmów,
        które dodałem na forum oraz do komentarzy pod spotem.}
    \acceptcrit{Użytkownik ma dostęp do listy filmów.}
    \inputdata{Lista filmów.}
    \preconditions{Użytkownik jest zalogowany.}
    \postconditions{Wyświetlenie się informacji.}
    \exceptions{}
    \implementation{}
    \sholder{}
    \reqrelated{}
\end{requirementstab}

\begin{requirementstab}[label={tab:requirements:func1},caption={}]
    \id{jednoznaczny symbol np. FO1, FO2 .. }
    \priority{M}
    \name{Lista znajomych}
    \descr{Jako użytkownik chcę mieć dostęp do listy znajomych.}
    \acceptcrit{Użytkownik ma dostęp do listy znajomych.}
    \inputdata{Lista znajomych}
    \preconditions{Użytkownik jest zalogowany.}
    \postconditions{Wyświetlenie się informacji.}
    \exceptions{}
    \implementation{}
    \sholder{}
    \reqrelated{}
\end{requirementstab}

\begin{requirementstab}[label={tab:requirements:func1},caption={}]
    \id{jednoznaczny symbol np. FO1, FO2 .. }
    \priority{M}
    \name{Lista obserwujących}
    \descr{Jako użytkownik chcę mieć dostęp do listy obserwujących.}
    \acceptcrit{Użytkownik ma dostęp do listy obserwujących.}
    \inputdata{Lista obserwujących}
    \preconditions{Użytkownik jest zalogowany.}
    \postconditions{Wyświetlenie się informacji.}
    \exceptions{}
    \implementation{}
    \sholder{}
    \reqrelated{}
\end{requirementstab}

\begin{requirementstab}[label={tab:requirements:func1},caption={}]
    \id{jednoznaczny symbol np. FO1, FO2 .. }
    \priority{M}
    \name{Lista obserwowanych}
    \descr{Jako użytkownik chcę mieć dostęp do listy obserwowanych.}
    \acceptcrit{Użytkownik ma dostęp do listy obserwowanych.}
    \inputdata{Lista obserwowanych}
    \preconditions{Użytkownik jest zalogowany.}
    \postconditions{Wyświetlenie się informacji.}
    \exceptions{}
    \implementation{}
    \sholder{}
    \reqrelated{}
\end{requirementstab}

\begin{requirementstab}[label={tab:requirements:func1},caption={}]
    \id{jednoznaczny symbol np. FO1, FO2 .. }
    \priority{M}
    \name{Lista spotów}
    \descr{Jako użytkownik chcę mieć dostęp do listy spotów które polubiłem,
        odwiedziłem i planuję odwiedzić.}
    \acceptcrit{Użytkownik ma dostęp do listy spotów.}
    \inputdata{Lista spotów}
    \preconditions{Użytkownik jest zalogowany.}
    \postconditions{Wyświetlenie się informacji.}
    \exceptions{}
    \implementation{}
    \sholder{}
    \reqrelated{}
\end{requirementstab}

\begin{requirementstab}[label={tab:requirements:func1},caption={}]
    \id{jednoznaczny symbol np. FO1, FO2 .. }
    \priority{M}
    \name{Lista komentarzy}
    \descr{Jako użytkownik chcę mieć dostęp do listy komentarzy}
    \acceptcrit{Użytkownik ma dostęp do listy komentarzy.}
    \inputdata{Lista komentarzy}
    \preconditions{Użytkownik jest zalogowany.}
    \postconditions{Wyświetlenie się informacji.}
    \exceptions{}
    \implementation{}
    \sholder{}
    \reqrelated{}
\end{requirementstab}

\begin{requirementstab}[label={tab:requirements:func1},caption={}]
    \id{jednoznaczny symbol np. FO1, FO2 .. }
    \priority{M}
    \name{Ustawienia}
    \descr{Jako użytkownik chcę mieć możliwość zmiany danych.}
    \acceptcrit{Użytkownik ma możliwość zmiany danych.}
    \inputdata{Formularz do zmiany danych.}
    \preconditions{Użytkownik jest zalogowany.}
    \postconditions{Wyświetlenie się formularza do edycji.}
    \exceptions{}
    \implementation{}
    \sholder{}
    \reqrelated{}
\end{requirementstab}

\begin{requirementstab}[label={tab:requirements:func1},caption={}]
    \id{jednoznaczny symbol np. FO1, FO2 .. }
    \priority{M}
    \name{Resetowanie hasła}
    \descr{Jako użytkownik chcę mieć możliwość zresetowania hasła do swojego konta.}
    \acceptcrit{Użytkownik po kliknięciu w odpowiedni link ma możliwość
    zresetowania hasła do swojego konta w aplikacji.}
    \inputdata{Adres email użytkownika do wysłania linku resetującego.}
    \preconditions{Użytkownik podał poprawny email podany podczas rejestracji.}
    \postconditions{Zresetowanie hasła.}
    \exceptions{}
    \implementation{}
    \sholder{}
    \reqrelated{}
\end{requirementstab}

\begin{requirementstab}[label={tab:requirements:func1},caption={}]
    \id{jednoznaczny symbol np. FO1, FO2 .. }
    \priority{M}
    \name{Dodawanie użytkowników do listy znajomych.}
    \descr{Jako użytkownik chcę mieć możliwość dodawania innych użytkowników do listy znajomych.}
    \acceptcrit{Użytkownik może dodać innych użytkowników do swojej listy znajomych.}
    \inputdata{Dane użytkownika, którego chcemy dodać do znajomych.}
    \preconditions{Zalogowany użytkownik.}
    \postconditions{Znajomi są dodani do listy i widoczni w profilu użytkownika.}
    \exceptions{}
    \implementation{}
    \sholder{}
    \reqrelated{}
\end{requirementstab}
