%! Author = Mateusz
%! Date = 01/11/2025

\subsection{Funkcjonalności dla konta użytkownika}
\label{subsec:funkcjonalnosci-dla-konta-uzytkownika}

\begin{table}[H]
    \centering
    \vspace{0pt}
\end{table}

\begin{requirementstab}[placement=H, label={tab:requirements:func1},caption={Pryzkładowa tabela z wymaganiami na interfejs z otoczeniem}]
    \id{jednoznaczny symbol np. FO1, FO2 .. }
    \priority{Ważność}
    \name{krótki opis}
    \descr{opis szczegółowy, należy dążyć do tego, żeby wszystkie znane na ten moment szczegóły wymagania zostały wydobyte i wyspecyfikowane

    Można zastosować opis jak w User Story
        \begin{itemize}
            \item Jako.. (konkretny użytkownik systemu)
            \item chcę... (pożądana cecha lub problem, który trzeba rozwiązać)
            \item bo wtedy/ponieważ… (korzyść płynąca z ukończenia story)
        \end{itemize}
    }
    \acceptcrit{Warunki Satysfakcji (Szczegóły dodane na potrzeby  testów akceptacyjnych)}
    \inputdata{uzupełniane w trakcie sprintu – dane wejściowe, związane z wymaganiem}
    \preconditions{ uzupełniane w trakcie sprintu – warunki, które muszą być prawdziwe przed wywołaniem operacji}
    \postconditions{ uzupełniane w trakcie sprintu – warunki, które muszą być prawdziwe po wywołaniu operacji}
    \exceptions{ uzupełniane w trakcie sprintu – niepożądane sytuacje i sposoby ich obsługi}
    \implementation{ uzupełniane w trakcie sprintu – opis sposobu realizacji}
    \sholder{nazwa udziałowca, który podał wymaganie}
    \reqrelated{wymagania zależne i uszczegóławiające – odesłanie poprzez identyfikator}
\end{requirementstab}