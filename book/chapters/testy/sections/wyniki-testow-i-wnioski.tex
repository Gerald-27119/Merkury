%! Author = Mateusz
%! Date = 13/12/2025

\section{Wyniki testów i wnioski}
\label{sec:wyniki-testow-i-wnioski}

Przeprowadzone testy jednostkowe, integracyjne oraz end-to-end potwierdziły poprawność działania
kluczowych funkcjonalności aplikacji zarówno po stronie \glslink{frontend}{frontendu}, jak i
\glslink{backend}{backendu}.
Uzyskane wyniki wskazują, że zaimplementowana logika biznesowa, warstwa prezentacji oraz
komunikacja z \gls{api} działają zgodnie z założeniami projektowymi.

Testy jednostkowe pozwoliły zweryfikować poprawność działania pojedynczych komponentów i usług,
w tym obsługę przypadków brzegowych oraz walidację danych wejściowych.
Testy integracyjne umożliwiły potwierdzenie współpracy większych fragmentów systemu, w szczególności przepływu danych
pomiędzy komponentami oraz reakcji aplikacji na działania użytkownika (zmianę stanu po interakcji).
Z kolei testy E2E, uruchamiane w środowisku zbliżonym do rzeczywistego, potwierdziły poprawną realizację
pełnych scenariuszy użytkownika, obejmujących logowanie, edycję ustawień konta, przeglądanie treści
oraz dynamiczne doładowywanie danych (\glslink{infinite-scroll}{\textit{infinite scroll}}).

Na podstawie wyników testów sformułowano następujące wnioski:
\begin{itemize}
    \item Zapewniono wysoką stabilność kluczowych modułów aplikacji dzięki szerokiemu pokryciu testami jednostkowymi.
    \item Mechanizmy odpowiedzialne za paginację i doładowywanie danych (\glslink{infinite-scroll}{\textit{infinite scroll}}) działają poprawnie
    w testowanych widokach oraz nie powodują błędów w interfejsie.
    \item Integracja frontendu z backendem została potwierdzona zarówno w testach integracyjnych, jak i E2E,
    co minimalizuje ryzyko regresji w przypadku dalszego rozwoju aplikacji.
    \item Zastosowanie mockowania zapytań w testach ułatwiło deterministyczne odtwarzanie scenariuszy
    oraz testowanie stanów trudnych do uzyskania w środowisku rzeczywistym (puste listy, konkretne warianty sortowania).
\end{itemize}
