%! Author = Mateusz
%! Date = 13/12/2025

\section{Scenariusze testów end-to-end (E2E)}
\label{sec:scenariusze-testow-e2e}


\newcounter{eTOe}[chapter]
\renewcommand{\theeTOe}{\thechapter.\arabic{eTOe}}

\newcommand{\eTOeid}[1]{\textbf{Identyfikator:} & #1 \\ \hline}
\newcommand{\eTOegoal}[1]{\textbf{Cel:} & #1 \\ \hline}
\newcommand{\eTOetype}[1]{\textbf{Typ:} & #1 \\ \hline}
\newcommand{\eTOepre}[1]{\textbf{Warunki wstępne:} & #1 \\ \hline}

\newcommand{\eTOesteps}[1]{%
    \textbf{Kroki:} & \begin{minipage}[t]{\linewidth}#1\end{minipage} \\ \hline}
\newcommand{\eTOeexpected}[1]{\textbf{Oczekiwany rezultat:} & #1 \\ \hline}

\newcommand{\eTOecard}[3]{%
    \refstepcounter{eTOe}%
    \par\begin{center}
    \renewcommand{\arraystretch}{1.15}%
    \begin{tabularx}{\textwidth}{|>{\columncolor{lightgray}\raggedright\arraybackslash}p{0.19\textwidth}|X|}
    \rowcolor{lightgray}
    \multicolumn{2}{|c|}{\textbf{KARTA SCENARIUSZA E2E}} \\ \hline
    #3
    \end{tabularx}
    \vspace{3pt}
    \textbf{Tabela \theeTOe:} Scenariusz E2E: #2\label{#1}
    \end{center}%
    \addcontentsline{lot}{table}{Tabela \theeTOe: Scenariusz E2E: #2}%
}

\newenvironment{tabitemizeeTOe}[1][]{%
    \begin{enumerate}[
        leftmargin=*,
        nosep,
        topsep=0pt,
        partopsep=0pt,
        parsep=0pt,
        itemsep=0pt,
        before=\vspace*{-0.5\baselineskip},
        after=\vspace*{0.3\baselineskip},
        #1
    ]
}{%
    \end{enumerate}
}


Poniżej przedstawiono scenariusze testowe zrealizowane w ramach testów end-to-end (E2E).
W opisach rozróżniono przypadki z użyciem danych mockowanych (poprzez przechwytywanie żądań HTTP)
oraz przypadki wykonywane na Rzeczywistym backendzie.

%------------------------------------------------
\subsection{Account (logowanie i rejestracja)}

\eTOecard{tab:e2e:acc01}{Logowanie użytkownika}{%
    \eTOeid{E2E-ACC-01}
    \eTOegoal{Potwierdzenie możliwości zalogowania użytkownika.}
    \eTOetype{Rzeczywisty backend}
    \eTOepre{brak}
    \eTOesteps{%
        \begin{tabitemizeeTOe}
            \item Otwarcie strony głównej aplikacji.
            \item Przejście do widoku logowania.
            \item Wprowadzenie nazwy użytkownika oraz hasła.
            \item Wysłanie formularza.
        \end{tabitemizeeTOe}
    }
    \eTOeexpected{Zakończenie procesu logowania bez błędów oraz przejście do aplikacji.}
}

\eTOecard{tab:e2e:acc02}{Rejestracja użytkownika}{%
    \eTOeid{E2E-ACC-02}
    \eTOegoal{Potwierdzenie możliwości założenia konta.}
    \eTOetype{Rzeczywisty backend}
    \eTOepre{brak}
    \eTOesteps{%
        \begin{tabitemizeeTOe}
            \item Otwarcie strony głównej aplikacji.
            \item Przejście do widoku logowania, a następnie do widoku rejestracji.
            \item Wprowadzenie danych rejestracyjnych (nazwa użytkownika, e-mail, hasło, potwierdzenie hasła).
            \item Wysłanie formularza.
        \end{tabitemizeeTOe}
    }
    \eTOeexpected{Zakończenie rejestracji bez błędów.}
}

%------------------------------------------------
\subsection{User dashboard -- Add spot}

\eTOecard{tab:e2e:add01}{Wyświetlenie stanu pustego dla dodanych miejsc}{%
    \eTOeid{E2E-ADD-01}
    \eTOegoal{Weryfikacja poprawnej obsługi pustej listy dodanych miejsc.}
    \eTOetype{Mockowane API}
    \eTOepre{Zasymulowanie zalogowania poprzez wpisy w \textit{localStorage}.}
    \eTOesteps{%
        \begin{tabitemizeeTOe}
            \item Otwarcie widoku \textit{Add spot}.
            \item Oczekiwanie na odpowiedź endpointu z listą miejsc (pusta lista).
        \end{tabitemizeeTOe}
    }
    \eTOeexpected{Wyświetlenie komunikatu o braku dodanych miejsc.}
}

\eTOecard{tab:e2e:add02}{Ładowanie kolejnych elementów listy przy przewijaniu (\textit{infinite scroll})}{%
    \eTOeid{E2E-ADD-02}
    \eTOegoal{Weryfikacja działania mechanizmu \textit{infinite scroll}.}
    \eTOetype{Mockowane API}
    \eTOepre{Zasymulowanie \textit{IntersectionObserver} wymuszające dociąganie danych.}
    \eTOesteps{%
        \begin{tabitemizeeTOe}
            \item Otwarcie widoku \textit{Add spot}.
            \item Weryfikacja wyświetlenia pierwszej porcji danych.
            \item Doprowadzenie do uruchomienia mechanizmu \textit{infinite scroll} i pobranie kolejnej strony.
        \end{tabitemizeeTOe}
    }
    \eTOeexpected{Dołączenie kolejnych elementów do listy oraz ich poprawna prezentacja.}
}

\eTOecard{tab:e2e:add03}{Otwarcie modala dodawania miejsca na widoku desktop}{%
    \eTOeid{E2E-ADD-03}
    \eTOegoal{Weryfikacja dostępności modala dodawania miejsca w trybie desktop.}
    \eTOetype{Mockowane API}
    \eTOepre{brak}
    \eTOesteps{%
        \begin{tabitemizeeTOe}
            \item Ustawienie rozdzielczości okna na tryb desktop.
            \item Otwarcie widoku \textit{Add spot}.
            \item Wybranie przycisku \textit{Add spot}.
        \end{tabitemizeeTOe}
    }
    \eTOeexpected{Wyświetlenie modala dodawania miejsca wraz z sekcjami (\textit{Basic Information}, \textit{Upload Media}).}
}

\eTOecard{tab:e2e:add04}{Zablokowanie dodawania miejsca na małym ekranie}{%
    \eTOeid{E2E-ADD-04}
    \eTOegoal{Weryfikacja blokady dodawania miejsca dla małych rozdzielczości.}
    \eTOetype{Mockowane API}
    \eTOepre{brak}
    \eTOesteps{%
        \begin{tabitemizeeTOe}
            \item Ustawienie rozdzielczości okna na tryb mobilny.
            \item Otwarcie widoku \textit{Add spot}.
            \item Próba otwarcia modala dodawania miejsca.
        \end{tabitemizeeTOe}
    }
    \eTOeexpected{Wyświetlenie komunikatu o wymaganym większym ekranie oraz brak wyświetlenia formularza dodawania.}
}

\eTOecard{tab:e2e:add05}{Dodanie nowego miejsca z użyciem rzeczywistego backendu}{%
    \eTOeid{E2E-ADD-05}
    \eTOegoal{Weryfikacja pełnego procesu dodania miejsca (UI $\rightarrow$ backend).}
    \eTOetype{Rzeczywisty backend}
    \eTOepre{Użytkownik posiada konto i możliwość zalogowania.}
    \eTOesteps{%
        \begin{tabitemizeeTOe}
            \item Wykonanie logowania użytkownika.
            \item Przejście do widoku \textit{Add spot}.
            \item Otwarcie modala dodawania miejsca.
            \item Uzupełnienie danych podstawowych.
            \item Dodanie pliku graficznego w sekcji multimediów.
            \item Wyznaczenie wielokąta na mapie oraz zakończenie rysowania.
            \item Zatwierdzenie dodania miejsca.
        \end{tabitemizeeTOe}
    }
    \eTOeexpected{Zapis miejsca, komunikat o powodzeniu oraz obecność nowo dodanego miejsca na liście.}
}

%------------------------------------------------
\subsection{User dashboard -- Comments}

\eTOecard{tab:e2e:com01}{Wyświetlenie stanu pustego dla komentarzy}{%
    \eTOeid{E2E-COM-01}
    \eTOegoal{Weryfikacja obsługi pustej listy komentarzy.}
    \eTOetype{Mockowane API}
    \eTOepre{brak}
    \eTOesteps{%
        \begin{tabitemizeeTOe}
            \item Otwarcie widoku komentarzy.
            \item Oczekiwanie na odpowiedź z pustą listą komentarzy.
        \end{tabitemizeeTOe}
    }
    \eTOeexpected{Wyświetlenie komunikatu o braku komentarzy oraz brak elementów listy.}
}

\eTOecard{tab:e2e:com02}{Sortowanie komentarzy po zmianie typu sortowania}{%
    \eTOeid{E2E-COM-02}
    \eTOegoal{Weryfikacja ponownego pobrania i prezentacji danych po zmianie sortowania.}
    \eTOetype{Mockowane API}
    \eTOepre{brak}
    \eTOesteps{%
        \begin{tabitemizeeTOe}
            \item Otwarcie widoku komentarzy.
            \item Weryfikacja początkowego układu listy.
            \item Zmiana sortowania.
            \item Oczekiwanie na ponowne pobranie danych.
        \end{tabitemizeeTOe}
    }
    \eTOeexpected{Aktualizacja listy zgodnie z wybranym sortowaniem.}
}

\eTOecard{tab:e2e:com03}{Załadowanie widoku komentarzy po Rzeczywistym logowaniu}{%
    \eTOeid{E2E-COM-03}
    \eTOegoal{Weryfikacja poprawnego pobrania danych komentarzy z backendu po logowaniu.}
    \eTOetype{Rzeczywisty backend}
    \eTOepre{Użytkownik wykonuje logowanie w aplikacji.}
    \eTOesteps{%
        \begin{tabitemizeeTOe}
            \item Wykonanie logowania użytkownika.
            \item Przejście do widoku komentarzy.
            \item Weryfikacja poprawnej odpowiedzi API (np. struktura \texttt{items} i \texttt{hasNext}).
        \end{tabitemizeeTOe}
    }
    \eTOeexpected{Poprawne załadowanie widoku komentarzy.}
}

%------------------------------------------------
\subsection{User dashboard -- Favorite spots}

\eTOecard{tab:e2e:fav01}{Wyświetlenie stanu pustego dla list ulubionych miejsc}{%
    \eTOeid{E2E-FAV-01}
    \eTOegoal{Weryfikacja obsługi pustej listy ulubionych miejsc.}
    \eTOetype{Mockowane API}
    \eTOepre{brak}
    \eTOesteps{%
        \begin{tabitemizeeTOe}
            \item Otwarcie widoku ulubionych miejsc.
            \item Pobranie pustej listy.
        \end{tabitemizeeTOe}
    }
    \eTOeexpected{Komunikat o braku miejsc w liście.}
}

\eTOecard{tab:e2e:fav02}{Zmiana typu listy}{%
    \eTOeid{E2E-FAV-02}
    \eTOegoal{Weryfikacja przełączania zakładek i pobrania danych dla wybranego typu listy.}
    \eTOetype{Mockowane API}
    \eTOepre{brak}
    \eTOesteps{%
        \begin{tabitemizeeTOe}
            \item Otwarcie widoku ulubionych miejsc.
            \item Weryfikacja elementów w domyślnej liście.
            \item Przełączenie zakładki listy (na \textit{Favorites}).
        \end{tabitemizeeTOe}
    }
    \eTOeexpected{Pobranie i wyświetlenie danych odpowiadających wybranemu typowi listy.}
}

\eTOecard{tab:e2e:fav03}{Dociąganie kolejnej strony przy przewijaniu}{%
    \eTOeid{E2E-FAV-03}
    \eTOegoal{Weryfikacja infinite scroll w liście ulubionych miejsc.}
    \eTOetype{Mockowane API}
    \eTOepre{brak}
    \eTOesteps{%
        \begin{tabitemizeeTOe}
            \item Przewinięcie listy do dołu w celu pobrania kolejnej strony.
        \end{tabitemizeeTOe}
    }
    \eTOeexpected{Zwiększenie liczby elementów listy o kolejne pozycje.}
}

\eTOecard{tab:e2e:fav04}{Załadowanie danych z rzeczywistego backendu po logowaniu}{%
    \eTOeid{E2E-FAV-04}
    \eTOegoal{Weryfikacja pobrania danych ulubionych miejsc z backendu po logowaniu.}
    \eTOetype{Rzeczywisty backend}
    \eTOepre{Użytkownik wykonuje logowanie w aplikacji.}
    \eTOesteps{%
        \begin{tabitemizeeTOe}
            \item Logowanie użytkownika.
            \item Przejście do widoku ulubionych miejsc.
            \item Weryfikacja poprawnej odpowiedzi API.
        \end{tabitemizeeTOe}
    }
    \eTOeexpected{Poprawne wyświetlenie widoku oraz elementów (jeśli istnieją).}
}

%------------------------------------------------
\subsection{User dashboard -- Movies}

\eTOecard{tab:e2e:mov01}{Stan pusty dla filmów}{%
    \eTOeid{E2E-MOV-01}
    \eTOegoal{Weryfikacja obsługi pustej listy filmów.}
    \eTOetype{Mockowane API}
    \eTOepre{brak}
    \eTOesteps{%
        \begin{tabitemizeeTOe}
            \item Otwarcie widoku filmów.
            \item Pobranie pustej listy.
        \end{tabitemizeeTOe}
    }
    \eTOeexpected{Brak elementów listy oraz poprawne wyświetlenie nagłówka widoku.}
}

\eTOecard{tab:e2e:mov02}{Zmiana sortowania filmów}{%
    \eTOeid{E2E-MOV-02}
    \eTOegoal{Weryfikacja odświeżenia listy po zmianie sortowania.}
    \eTOetype{Mockowane API}
    \eTOepre{brak}
    \eTOesteps{%
        \begin{tabitemizeeTOe}
            \item Przełączenie sortowania.
            \item Oczekiwanie na ponowne pobranie danych.
        \end{tabitemizeeTOe}
    }
    \eTOeexpected{Odświeżenie listy zgodnie z wybranym sortowaniem.}
}

\eTOecard{tab:e2e:mov03}{\textit{Infinite scroll} w widoku filmów}{%
    \eTOeid{E2E-MOV-03}
    \eTOegoal{Weryfikacja dociągania kolejnych stron danych.}
    \eTOetype{Mockowane API}
    \eTOepre{brak}
    \eTOesteps{%
        \begin{tabitemizeeTOe}
            \item Przewinięcie listy do dołu w celu pobrania kolejnej strony.
        \end{tabitemizeeTOe}
    }
    \eTOeexpected{Dołączenie nowych elementów do listy.}
}

\eTOecard{tab:e2e:mov04}{Załadowanie filmów po logowaniu na Rzeczywistym backendzie}{%
    \eTOeid{E2E-MOV-04}
    \eTOegoal{Weryfikacja pobrania filmów z backendu po logowaniu.}
    \eTOetype{Rzeczywisty backend}
    \eTOepre{Użytkownik wykonuje logowanie w aplikacji.}
    \eTOesteps{%
        \begin{tabitemizeeTOe}
            \item Logowanie użytkownika.
            \item Przejście do widoku filmów.
            \item Weryfikacja poprawnej odpowiedzi API.
        \end{tabitemizeeTOe}
    }
    \eTOeexpected{Poprawne wyświetlenie widoku filmów.}
}

%------------------------------------------------
\subsection{User dashboard -- Photos}

\eTOecard{tab:e2e:pho01}{Stan pusty dla zdjęć}{%
    \eTOeid{E2E-PHO-01}
    \eTOegoal{Weryfikacja obsługi pustej listy zdjęć.}
    \eTOetype{Mockowane API}
    \eTOepre{brak}
    \eTOesteps{%
        \begin{tabitemizeeTOe}
            \item Otwarcie widoku zdjęć.
            \item Pobranie pustej listy.
        \end{tabitemizeeTOe}
    }
    \eTOeexpected{Brak elementów listy zdjęć.}
}

\eTOecard{tab:e2e:pho02}{Zmiana sortowania zdjęć}{%
    \eTOeid{E2E-PHO-02}
    \eTOegoal{Weryfikacja odświeżenia listy po zmianie sortowania.}
    \eTOetype{Mockowane API}
    \eTOepre{brak}
    \eTOesteps{%
        \begin{tabitemizeeTOe}
            \item Przełączenie sortowania.
            \item Oczekiwanie na ponowne pobranie danych.
        \end{tabitemizeeTOe}
    }
    \eTOeexpected{Aktualizacja listy zdjęć.}
}

\eTOecard{tab:e2e:pho03}{\textit{Infinite scroll} w widoku zdjęć}{%
    \eTOeid{E2E-PHO-03}
    \eTOegoal{Weryfikacja dociągania kolejnych stron danych w liście zdjęć.}
    \eTOetype{Mockowane API}
    \eTOepre{brak}
    \eTOesteps{%
        \begin{tabitemizeeTOe}
            \item Przewinięcie listy w celu pobrania kolejnej strony.
        \end{tabitemizeeTOe}
    }
    \eTOeexpected{Zwiększenie liczby elementów na liście zdjęć.}
}

\eTOecard{tab:e2e:pho04}{Widok zdjęć po logowaniu na Rzeczywistym backendzie}{%
    \eTOeid{E2E-PHO-04}
    \eTOegoal{Weryfikacja pobrania zdjęć z backendu po logowaniu.}
    \eTOetype{Rzeczywisty backend}
    \eTOepre{Użytkownik wykonuje logowanie w aplikacji.}
    \eTOesteps{%
        \begin{tabitemizeeTOe}
            \item Logowanie użytkownika.
            \item Przejście do widoku zdjęć.
            \item Weryfikacja obecności danych.
        \end{tabitemizeeTOe}
    }
    \eTOeexpected{Poprawne wyświetlenie widoku oraz zdjęć użytkownika.}
}

%------------------------------------------------
\subsection{User dashboard -- Profile}

\eTOecard{tab:e2e:pro01}{Wyświetlenie profilu użytkownika (statystyki i popularne zdjęcia)}{%
    \eTOeid{E2E-PRO-01}
    \eTOegoal{Weryfikacja poprawnej prezentacji profilu użytkownika.}
    \eTOetype{Mockowane API}
    \eTOepre{brak}
    \eTOesteps{%
        \begin{tabitemizeeTOe}
            \item Otwarcie profilu własnego.
            \item Weryfikacja statystyk oraz wyświetlenia najpopularniejszych zdjęć.
        \end{tabitemizeeTOe}
    }
    \eTOeexpected{Obecność zdjęcia profilowego, statystyk oraz sekcji najpopularniejszych zdjęć.}
}

\eTOecard{tab:e2e:pro02}{Komunikat o braku zdjęć w profilu}{%
    \eTOeid{E2E-PRO-02}
    \eTOegoal{Weryfikacja obsługi pustej listy zdjęć w profilu.}
    \eTOetype{Mockowane API}
    \eTOepre{brak}
    \eTOesteps{%
        \begin{tabitemizeeTOe}
            \item Otwarcie profilu własnego z pustą listą zdjęć.
        \end{tabitemizeeTOe}
    }
    \eTOeexpected{Wyświetlenie komunikatu o braku dodanych zdjęć.}
}

\eTOecard{tab:e2e:pro03}{Nawigacja do listy znajomych z poziomu profilu}{%
    \eTOeid{E2E-PRO-03}
    \eTOegoal{Weryfikacja nawigacji do widoku znajomych.}
    \eTOetype{Mockowane API}
    \eTOepre{brak}
    \eTOesteps{%
        \begin{tabitemizeeTOe}
            \item Kliknięcie statystyki \textit{Friends}.
            \item Oczekiwanie na załadowanie widoku znajomych.
        \end{tabitemizeeTOe}
    }
    \eTOeexpected{Przejście do \texttt{/account/friends}.}
}

\eTOecard{tab:e2e:pro04}{Nawigacja do zdjęć z poziomu profilu}{%
    \eTOeid{E2E-PRO-04}
    \eTOegoal{Weryfikacja nawigacji do widoku zdjęć.}
    \eTOetype{Mockowane API}
    \eTOepre{brak}
    \eTOesteps{%
        \begin{tabitemizeeTOe}
            \item Kliknięcie statystyki \textit{Photos}.
            \item Oczekiwanie na załadowanie widoku zdjęć.
        \end{tabitemizeeTOe}
    }
    \eTOeexpected{Przejście do \texttt{/account/photos}.}
}

\eTOecard{tab:e2e:pro05}{Wyświetlenie profilu innego użytkownika (akcje follow/friends)}{%
    \eTOeid{E2E-PRO-05}
    \eTOegoal{Weryfikacja widoczności akcji społecznościowych na profilu innego użytkownika.}
    \eTOetype{Mockowane API}
    \eTOepre{brak}
    \eTOesteps{%
        \begin{tabitemizeeTOe}
            \item Otwarcie profilu innego użytkownika.
            \item Weryfikacja dostępnych akcji.
        \end{tabitemizeeTOe}
    }
    \eTOeexpected{Widoczność przycisków \textit{follow} oraz \textit{add to friends}.}
}

\eTOecard{tab:e2e:pro06}{Wysłanie żądania follow oraz zaproszenia do znajomych}{%
    \eTOeid{E2E-PRO-06}
    \eTOegoal{Weryfikacja poprawnych żądań HTTP dla akcji follow/friends.}
    \eTOetype{Mockowane API}
    \eTOepre{brak}
    \eTOesteps{%
        \begin{tabitemizeeTOe}
            \item Kliknięcie \textit{follow}.
            \item Kliknięcie \textit{add to friends}.
            \item Weryfikacja żądań HTTP.
        \end{tabitemizeeTOe}
    }
    \eTOeexpected{Wysłanie żądań \texttt{PATCH} z poprawnymi parametrami.}
}

\eTOecard{tab:e2e:pro07}{Załadowanie profilu własnego po logowaniu na Rzeczywistym backendzie}{%
    \eTOeid{E2E-PRO-07}
    \eTOegoal{Weryfikacja poprawnego wyświetlenia profilu po logowaniu.}
    \eTOetype{Rzeczywisty backend}
    \eTOepre{Użytkownik wykonuje logowanie w aplikacji.}
    \eTOesteps{%
        \begin{tabitemizeeTOe}
            \item Logowanie.
            \item Przejście do profilu własnego.
        \end{tabitemizeeTOe}
    }
    \eTOeexpected{Poprawne wyświetlenie profilu oraz elementów interfejsu.}
}

%------------------------------------------------
\subsection{User dashboard -- Settings}

\eTOecard{tab:e2e:set01}{Wyświetlenie danych konta i otwarcie formularza zmiany nazwy użytkownika}{%
    \eTOeid{E2E-SET-01}
    \eTOegoal{Weryfikacja obecności danych konta i możliwości edycji nazwy użytkownika.}
    \eTOetype{Mockowane API}
    \eTOepre{brak}
    \eTOesteps{%
        \begin{tabitemizeeTOe}
            \item Otwarcie ustawień.
            \item Weryfikacja pól.
            \item Wybranie opcji \textit{Edit} dla nazwy użytkownika.
        \end{tabitemizeeTOe}
    }
    \eTOeexpected{Wyświetlenie formularza zmiany nazwy użytkownika.}
}

\eTOecard{tab:e2e:set02}{Zmiana nazwy użytkownika}{%
    \eTOeid{E2E-SET-02}
    \eTOegoal{Weryfikacja wysłania poprawnego żądania zmiany nazwy użytkownika.}
    \eTOetype{Mockowane API}
    \eTOepre{brak}
    \eTOesteps{%
        \begin{tabitemizeeTOe}
            \item Wprowadzenie nowej nazwy.
            \item Zapisanie zmian.
        \end{tabitemizeeTOe}
    }
    \eTOeexpected{Wysłanie żądania \texttt{PATCH} oraz komunikat o powodzeniu.}
}

\eTOecard{tab:e2e:set03}{Zmiana adresu e-mail}{%
    \eTOeid{E2E-SET-03}
    \eTOegoal{Weryfikacja wysłania poprawnego żądania zmiany e-mail.}
    \eTOetype{Mockowane API}
    \eTOepre{brak}
    \eTOesteps{%
        \begin{tabitemizeeTOe}
            \item Wprowadzenie nowego adresu e-mail.
            \item Zapisanie zmian.
        \end{tabitemizeeTOe}
    }
    \eTOeexpected{Wysłanie żądania \texttt{PATCH} oraz komunikat o powodzeniu.}
}

\eTOecard{tab:e2e:set04}{Zmiana hasła}{%
    \eTOeid{E2E-SET-04}
    \eTOegoal{Weryfikacja procesu zmiany hasła.}
    \eTOetype{Mockowane API}
    \eTOepre{brak}
    \eTOesteps{%
        \begin{tabitemizeeTOe}
            \item Wprowadzenie starego i nowego hasła (z potwierdzeniem).
            \item Zapis zmian.
        \end{tabitemizeeTOe}
    }
    \eTOeexpected{Wysłanie żądania \texttt{PATCH} oraz komunikat o powodzeniu.}
}

\eTOecard{tab:e2e:set05}{Ograniczone ustawienia dla konta OAuth}{%
    \eTOeid{E2E-SET-05}
    \eTOegoal{Weryfikacja ograniczeń edycji danych dla kont OAuth.}
    \eTOetype{Mockowane API}
    \eTOepre{Użytkownik jest zalogowany kontem z providerem OAuth.}
    \eTOesteps{%
        \begin{tabitemizeeTOe}
            \item Otwarcie ustawień dla użytkownika z providerem OAuth.
        \end{tabitemizeeTOe}
    }
    \eTOeexpected{Brak formularzy edycji danych oraz wyświetlenie informacji o providerze.}
}

\eTOecard{tab:e2e:set06}{Wyświetlenie ustawień po logowaniu na Rzeczywistym backendzie}{%
    \eTOeid{E2E-SET-06}
    \eTOegoal{Weryfikacja poprawnego wyświetlenia ustawień po logowaniu.}
    \eTOetype{Rzeczywisty backend}
    \eTOepre{Użytkownik wykonuje logowanie w aplikacji.}
    \eTOesteps{%
        \begin{tabitemizeeTOe}
            \item Logowanie.
            \item Przejście do ustawień.
        \end{tabitemizeeTOe}
    }
    \eTOeexpected{Poprawne wyświetlenie danych konta.}
}

\eTOecard{tab:e2e:set07}{Zmiana hasła na Rzeczywistym backendzie oraz ponowne logowanie}{%
    \eTOeid{E2E-SET-07}
    \eTOegoal{Weryfikacja zmiany hasła i ponownego logowania.}
    \eTOetype{Rzeczywisty backend}
    \eTOepre{Użytkownik posiada konto lokalne (nie OAuth).}
    \eTOesteps{%
        \begin{tabitemizeeTOe}
            \item Wykonanie logowania użytkownika.
            \item Przejście do ustawień i zmiana hasła.
            \item Wyczyszczenie danych sesyjnych po stronie przeglądarki.
            \item Ponowne logowanie z nowym hasłem.
        \end{tabitemizeeTOe}
    }
    \eTOeexpected{Możliwość zalogowania się nowym hasłem oraz poprawne załadowanie widoku ustawień.}
}

%------------------------------------------------
\subsection{User dashboard -- Social (friends/followed/followers)}

\eTOecard{tab:e2e:soc01}{Stan pusty listy znajomych}{%
    \eTOeid{E2E-SOC-01}
    \eTOegoal{Weryfikacja obsługi pustej listy znajomych.}
    \eTOetype{Mockowane API}
    \eTOepre{brak}
    \eTOesteps{%
        \begin{tabitemizeeTOe}
            \item Otwarcie widoku listy społeczności.
            \item Pobranie pustej listy znajomych.
        \end{tabitemizeeTOe}
    }
    \eTOeexpected{Wyświetlenie komunikatu o braku znajomych.}
}

\eTOecard{tab:e2e:soc02}{Przełączanie zakładek friends/followed/followers}{%
    \eTOeid{E2E-SOC-02}
    \eTOegoal{Weryfikacja przełączania zakładek i poprawnego wyświetlania danych.}
    \eTOetype{Mockowane API}
    \eTOepre{brak}
    \eTOesteps{%
        \begin{tabitemizeeTOe}
            \item Otwarcie widoku społeczności.
            \item Weryfikacja listy znajomych.
            \item Przełączenie zakładki na \textit{followed}, a następnie \textit{followers}.
        \end{tabitemizeeTOe}
    }
    \eTOeexpected{Wyświetlenie list odpowiednich dla wybranych zakładek.}
}

\eTOecard{tab:e2e:soc03}{\textit{Infinite scroll} na liście znajomych}{%
    \eTOeid{E2E-SOC-03}
    \eTOegoal{Weryfikacja dociągania kolejnych stron danych w liście społeczności.}
    \eTOetype{Mockowane API}
    \eTOepre{brak}
    \eTOesteps{%
        \begin{tabitemizeeTOe}
            \item Przewinięcie listy do dołu w celu pobrania kolejnej strony danych.
        \end{tabitemizeeTOe}
    }
    \eTOeexpected{Dołączenie kolejnych kart użytkowników do listy.}
}

\eTOecard{tab:e2e:soc04}{Obsługa zaproszeń do znajomych}{%
    \eTOeid{E2E-SOC-04}
    \eTOegoal{Weryfikacja akceptowania zaproszeń i wysyłania poprawnych żądań.}
    \eTOetype{Mockowane API}
    \eTOepre{brak}
    \eTOesteps{%
        \begin{tabitemizeeTOe}
            \item Otwarcie widoku społeczności.
            \item Przejście do widoku zaproszeń (\textit{See friend invites}).
            \item Akceptacja wybranego zaproszenia.
        \end{tabitemizeeTOe}
    }
    \eTOeexpected{Wysłanie żądania zmiany statusu zaproszenia oraz brak błędów w interfejsie.}
}
