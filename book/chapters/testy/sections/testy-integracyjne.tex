%! Author = Mateusz
%! Date = 13/12/2025

\section{Testy integracyjne}
\label{sec:testy-integracyjne}

Do automatycznego uruchamiania testów integracyjnych wykorzystano \gls{github-actions},
co umożliwiło ich cykliczne wykonywanie w ramach procesu \gls{cicd}
(przy każdym \textit{push} lub \textit{pull request}).
Łącznie przygotowano 102 testy integracyjne, w tym 52 dla warstwy \glslink{frontend}{frontendowej}
oraz 50 dla warstwy \glslink{backend}{backendowej}.
Wszystkie przygotowane testy zakończyły się powodzeniem (rys.~\ref{fig:integration-tests-frontend}
oraz rys.~\ref{fig:integration-tests-backend-suite}--\ref{fig:integration-tests-backend-summary}).

W warstwie \glslink{frontend}{frontendu} utworzono łącznie 52 testy integracyjne
(w 11 plikach testowych).
Testy te pozwalały zweryfikować poprawną współpracę wybranych komponentów w ramach większych
fragmentów interfejsu oraz spójność przepływu danych pomiędzy nimi.
Dodatkowo sprawdzano poprawność zmian stanu aplikacji w odpowiedzi na interakcje użytkownika
(kliknięcia) oraz działanie mechanizmu \glslink{infinite-scroll}{\textit{infinite scroll}},
w tym poprawne dociąganie i prezentację kolejnych elementów listy.

Przykładowy wynik uruchomienia testów integracyjnych \glslink{frontend}{frontendu} przedstawiono
(rys.~\ref{fig:integration-tests-frontend}).

\begin{figure}[H]
    \centering
    \includegraphics[width=1\textwidth]{attachments/testy/integration-frontned}
    \caption{Wynik uruchomienia testów integracyjnych warstwy frontendowej}
    \label{fig:integration-tests-frontend}
\end{figure}

Dla \glslink{backend}{backendu} przygotowano łącznie 50 testów integracyjnych.
Testy te służyły do potwierdzenia poprawnej współpracy kluczowych warstw aplikacji,
w tym poprawnego uruchomienia kontekstu aplikacji.
Zestaw uruchomionych testów \glslink{backend}{backendu} pokazano (rys.~\ref{fig:integration-tests-backend-suite}),
natomiast podsumowanie ich wykonania przedstawiono (rys.~\ref{fig:integration-tests-backend-summary}).

\begin{figure}[H]
    \centering
    \includegraphics[width=1\textwidth]{attachments/testy/integration-backend}
    \caption{Zestaw testów integracyjnych uruchomionych dla warstwy backendowej}
    \label{fig:integration-tests-backend-suite}
\end{figure}

\begin{figure}[H]
    \centering
    \includegraphics[width=1\textwidth]{attachments/testy/integration-backend-liczba}
    \caption{Podsumowanie uruchomienia testów integracyjnych warstwy backendowej}
    \label{fig:integration-tests-backend-summary}
\end{figure}
