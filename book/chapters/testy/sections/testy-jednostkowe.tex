%! Author = Mateusz
%! Date = 13/12/2025

\section{Testy jednostkowe}
\label{sec:testy-jednostkowe}

Do automatycznego uruchamiania testów jednostkowych wykorzystano
\gls{github-actions}, co umożliwiło ich cykliczne wykonywanie w ramach procesu \gls{cicd}
(przy każdym \textit{push} lub \textit{pull request}).
Łącznie przygotowano 273 testy jednostkowe, w tym 211 dla warstwy \glslink{frontend}{frontendowej}
oraz 62 dla warstwy \glslink{backend}{backendowej}.
Wszystkie przygotowane testy zakończyły się powodzeniem (rys.~\ref{fig:unit-tests-frontend}
oraz rys.~\ref{fig:unit-tests-backend-suite}--\ref{fig:unit-tests-backend-summary}).

W warstwie \glslink{frontend}{frontendu} utworzono łącznie 211 testów jednostkowych
(w 23 plikach testowych).
Testy opracowano dla następujących modułów i komponentów:
\begin{itemize}
    \item AddedSpots,
    \item Comments,
    \item FavoriteSpots,
    \item Movies,
    \item Photos (w tym DateChooser oraz SortDropdown),
    \item Profile (ProfileForViewer, UserOwnProfile),
    \item Settings,
    \item Social (SocialCard, SocialForViewer, UserOwnSocial),
    \item Login,
    \item Register,
    \item CurrentViewSpotsList,
    \item SearchedSpotsList,
    \item SearchedSpotsSortingForm,
    \item Sidebar,
    \item SpotDetails,
    \item SpotsNameSearchBar,
    \item UserLocationPanel,
    \item ZoomPanelControl.
\end{itemize}
W testach weryfikowano poprawność renderowania komponentów, obecność i treść kluczowych
elementów interfejsu.
Przykładowy wynik uruchomienia testów jednostkowych frontendu przedstawiono (rys.~\ref{fig:unit-tests-frontend}).

\begin{figure}[H]
    \centering
    \includegraphics[width=1\textwidth]{attachments/testy/unit-frontend}
    \caption{Wynik uruchomienia testów jednostkowych warstwy frontendowej}
    \label{fig:unit-tests-frontend}
\end{figure}

Dla \glslink{backend}{backendu} przygotowano łącznie 62 testy jednostkowe, napisane dla serwisów:
\begin{itemize}
    \item FollowersService,
    \item CommentsService,
    \item MediaService,
    \item AddSpotService,
    \item SettingsService,
    \item ProfileService,
    \item FavoriteSpotService,
    \item FriendsService,
    \item RegisterService.
\end{itemize}
Testy te służyły do potwierdzenia poprawnego działania metod serwisowych, w tym obsługi
przypadków brzegowych oraz walidacji danych wejściowych.
Dodatkowo sprawdzano poprawność współpracy z wybranymi zewnętrznymi interfejsami \gls{api},
przy zachowaniu izolacji logiki aplikacyjnej (poprzez zastępowanie zależności atrapami).
Zestaw uruchomionych testów backendu pokazano (rys.~\ref{fig:unit-tests-backend-suite}),
natomiast podsumowanie ich wykonania przedstawiono (rys.~\ref{fig:unit-tests-backend-summary}).

\begin{figure}[H]
    \centering
    \includegraphics[width=1\textwidth]{attachments/testy/unit-backend}
    \caption{Zestaw testów jednostkowych uruchomionych dla warstwy backendowej}
    \label{fig:unit-tests-backend-suite}
\end{figure}

\begin{figure}[H]
    \centering
    \includegraphics[width=1\textwidth]{attachments/testy/unit-backend-liczba}
    \caption{Podsumowanie uruchomienia testów jednostkowych warstwy backendowej}
    \label{fig:unit-tests-backend-summary}
\end{figure}
