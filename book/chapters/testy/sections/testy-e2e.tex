%! Author = Mateusz
%! Date = 13/12/2025

\section{Testy end-to-end (E2E)}
\label{sec:testy-e2e}

Testy end-to-end (E2E) (ang. \textit{end-to-end tests}) to testy automatyczne weryfikujące działanie
aplikacji jako całości z perspektywy użytkownika.
Obejmują one pełny przepływ realizacji funkcjonalności, od interakcji w interfejsie (nawigacja, kliknięcia,
wypełnianie formularzy) aż po komunikację z warstwą serwerową i przetwarzanie danych, dzięki czemu pozwalają
potwierdzić poprawność działania kluczowych scenariuszy biznesowych w warunkach zbliżonych do rzeczywistego użycia systemu.

W odróżnieniu od testów jednostkowych i integracyjnych, testy E2E uruchamiane są na działającej aplikacji
i wykorzystują rzeczywistą przeglądarkę, co zwiększa wiarygodność weryfikacji, ale jednocześnie zwykle wiąże się
z dłuższym czasem wykonania oraz większą wrażliwością na zmiany w interfejsie.

Testy end-to-end (E2E) zrealizowano z wykorzystaniem narzędzia Cypress.
W przeciwieństwie do testów jednostkowych i integracyjnych, testy E2E nie były uruchamiane w ramach
\gls{github-actions} (procesu \gls{cicd}), lecz wykonywano je lokalnie w środowisku deweloperskim.
Testy uruchamiano na działającej aplikacji, symulując rzeczywiste działania użytkownika w przeglądarce,
co pozwoliło zweryfikować pełny przepływ od interfejsu \glslink{frontend}{frontendowego}
do warstwy \glslink{backend}{backendowej}.

Łącznie przygotowano 40 testów E2E (w 9 plikach), a wszystkie zakończyły się powodzeniem
(rys. \ref{fig:e2e-tests-frontend-summary}).
Zaimplementowane testy stanowią bezpośrednią realizację scenariuszy testowych opisanych w sekcji
\ref{sec:scenariusze-testow-e2e}, obejmujących zarówno przypadki z użyciem danych mockowanych
(poprzez przechwytywanie żądań HTTP), jak i scenariusze wykonywane na rzeczywistym backendzie.

Testy E2E opracowano dla kluczowych obszarów aplikacji:
\begin{itemize}
    \item account (logowanie, rejestracja),
    \item user-dashboard/add-spot (lista, \glslink{infinite-scroll}{\textit{infinite scroll}}, dodawanie miejsca),
    \item user-dashboard/comments (lista, sortowanie),
    \item user-dashboard/favorite-spots (listy, przełączanie typów, \glslink{infinite-scroll}{\textit{infinite scroll}}),
    \item user-dashboard/movies (lista, sortowanie, \glslink{infinite-scroll}{\textit{infinite scroll}}),
    \item user-dashboard/photos (lista, sortowanie, \glslink{infinite-scroll}{\textit{infinite scroll}}),
    \item user-dashboard/profile (widok profilu, nawigacja, akcje społecznościowe),
    \item user-dashboard/settings (edycja danych konta oraz ograniczenia dla kont OAuth),
    \item user-dashboard/social (listy: friends/followed/followers, zaproszenia, \newline \glslink{infinite-scroll}{\textit{infinite scroll}}).
\end{itemize}

W testach weryfikowano poprawność realizacji scenariuszy użytkownika, w tym nawigację pomiędzy widokami,
wykonywanie operacji w interfejsie (kliknięcia i wypełnianie formularzy) oraz poprawną aktualizację
stanu aplikacji po wykonanych akcjach.
Dodatkowo sprawdzano działanie mechanizmu przewijania z dynamicznym doładowywaniem danych
(\glslink{infinite-scroll}{\textit{infinite scroll}}) w warunkach zbliżonych do rzeczywistego użycia aplikacji.

\begin{figure}[H]
    \centering
    \includegraphics[width=1\textwidth]{attachments/testy/e2e}
    \caption{Podsumowanie uruchomienia testów end-to-end (E2E) w Cypress}
    \label{fig:e2e-tests-frontend-summary}
\end{figure}
