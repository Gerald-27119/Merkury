%! Author = Adam
%! Date = 01/01/2025

\subsection{Integracja z dostawcą GIF-ów}
\label{subsec:integracja-z-dostawca-gif-ow}

W module czatu w warstwie \glslink{backend}{backendu} zaimplementowano integrację z zewnętrznym dostawcą
\glslink{gif}{GIF-ów} w celu umożliwienia użytkownikom wysyłania ich na czacie.
Jako źródło danych wykorzystano usługę \glslink{tenor}{Tenor}, a integrację rozdzielono na dwie warstwy:
\glslink{klient}{klienta} dostawcy oraz serwis aplikacyjny.

\subsubsection{Klient dostawcy -- \texttt{TenorGifProviderClient}}

\begin{figure}[H]
    \centering
    \includegraphics[width=\textwidth]{./attachments/implementacja-backendu/czat/gif/provider1}
    \caption{\texttt{TenorGifProviderClient} (1/2)}
    \label{fig:gif:tenor-client-1}
\end{figure}

\begin{figure}[H]
    \centering
    \includegraphics[width=\textwidth]{./attachments/implementacja-backendu/czat/gif/provider2}
    \caption{\texttt{TenorGifProviderClient} (2/2)}
    \label{fig:gif:tenor-client-2}
\end{figure}

Rysunki~\ref{fig:gif:tenor-client-1} oraz~\ref{fig:gif:tenor-client-2} przedstawiają klasę odpowiedzialną
za bezpośrednią komunikację z \glslink{api}{API} \glslink{tenor}{Tenor}. Klient wykorzystuje \glslink{webclient}{WebClient'a},
co umożliwia wykonywanie zapytań w sposób nieblokujący. W ramach integracji udostępniono dwa podstawowe scenariusze:
pobranie listy popularnych kategorii (\texttt{/categories}) oraz wyszukiwanie \glslink{gif}{GIF-ów}
dla podanej frazy (\texttt{/search}).

Na poziomie budowy zapytań zastosowano parametry konfigurujące sposób zwracania wyników, m.in. lokalizację (\texttt{locale})
oraz limit liczby elementów na stronę. Mechanizm stronicowania obsługiwany jest poprzez opcjonalny parametr \texttt{pos}
(przekazywany dalej jako \texttt{next}), dzięki czemu \glslink{klient}{klient} może pobierać kolejne porcje wyników
bez konieczności utrzymywania \glslink{stan}{stanu} po stronie aplikacji.

\subsubsection{Warstwa serwisowa -- \texttt{GifService}}

\begin{figure}[H]
    \centering
    \includegraphics[width=\textwidth]{./attachments/implementacja-backendu/czat/gif/service}
    \caption{\texttt{GifService}}
    \label{fig:gif:service}
\end{figure}

Rysunek~\ref{fig:gif:service} prezentuje serwis aplikacyjny, który stanowi warstwę pośrednią pomiędzy kontrolerem
a klientem dostawcy. Odpowiada on za:
\begin{itemize}
    \item delegowanie wywołań do klienta \glslink{tenor}{Tenor},
    \item mapowanie odpowiedzi z formatu dostawcy na struktury \glslink{dto}{DTO} wykorzystywane w aplikacji,
    \item zastosowanie mechanizmów optymalizacyjnych po stronie serwera.
\end{itemize}

Istotnym elementem jest użycie \glslink{annotation}{adnotacji} \texttt{@Cacheable} dla operacji pobierania popularnych kategorii.
Zastosowanie pamięci podręcznej (\glslink{cache}{cache}) ogranicza liczbę wywołań do zewnętrznej usługi w przypadku danych,
które zmieniają się relatywnie rzadko.
