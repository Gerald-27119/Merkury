%! Author = Adam
%! Date = 01/01/2025

\subsection{Integracja z dostawcą GIF-ów}
\label{subsec:integracja-z-dostawca-gif-ow}

W module czatu zaimplementowano integrację z zewnętrznym dostawcą GIF-ów w celu umożliwienia
użytkownikom wysyłania GIF-ów na czacie. Jako źródło danych wykorzystano usługę Tenor,
a integrację rozdzielono na trzy warstwy: klienta dostawcy (provider), serwis aplikacyjny oraz kontroler REST.

\subsubsection{Klient dostawcy -- \texttt{TenorGifProviderClient}}

\begin{figure}[H]
    \centering
    \includegraphics[width=\textwidth]{./attachments/implementacja-backendu/czat/gif/provider1}
    \caption{Klient dostawcy GIF-ów \texttt{TenorGifProviderClient} -- konfiguracja oraz inicjalizacja komunikacji z API Tenor przy użyciu \texttt{WebClient}.}
    \label{fig:gif:tenor-client-1}
\end{figure}

\begin{figure}[H]
    \centering
    \includegraphics[width=\textwidth]{./attachments/implementacja-backendu/czat/gif/provider2}
    \caption{\texttt{TenorGifProviderClient} -- implementacja scenariuszy pobierania kategorii (\texttt{/categories}) oraz wyszukiwania GIF-ów (\texttt{/search}) wraz z obsługą paginacji.}
    \label{fig:gif:tenor-client-2}
\end{figure}

Rysunki~\ref{fig:gif:tenor-client-1} oraz~\ref{fig:gif:tenor-client-2} przedstawiają klasę odpowiedzialną
za bezpośrednią komunikację z API Tenor. Klient wykorzystuje \texttt{WebClient} (Spring WebFlux),
co umożliwia wykonywanie zapytań w sposób nieblokujący oraz wpisuje się w reaktywny charakter modułu
(zwracanie typów \texttt{Mono}). W ramach integracji udostępniono dwa podstawowe scenariusze:
pobranie listy popularnych kategorii (\texttt{/categories}) oraz wyszukiwanie GIF-ów dla podanej frazy (\texttt{/search}).

Na poziomie budowy zapytań zastosowano parametry konfigurujące sposób zwracania wyników, m.in. lokalizację (\texttt{locale})
oraz limit liczby elementów na stronę. Mechanizm stronicowania obsługiwany jest poprzez opcjonalny parametr \texttt{pos}
(przekazywany dalej jako \texttt{next}), dzięki czemu warstwa kliencka może pobierać kolejne porcje wyników
bez konieczności utrzymywania stanu po stronie aplikacji.

\subsubsection{Warstwa serwisowa -- \texttt{GifService}}

\begin{figure}[H]
    \centering
    \includegraphics[width=\textwidth]{./attachments/implementacja-backendu/czat/gif/service}
    \caption{Warstwa serwisowa \texttt{GifService} -- delegowanie wywołań do klienta Tenor, mapowanie odpowiedzi do \gls{dto} oraz zastosowanie mechanizmu cache dla danych rzadko zmiennych.}
    \label{fig:gif:service}
\end{figure}

Rysunek~\ref{fig:gif:service} prezentuje serwis aplikacyjny, który stanowi warstwę pośrednią pomiędzy kontrolerem
a klientem dostawcy. Odpowiada on za:
\begin{itemize}
    \item delegowanie wywołań do klienta Tenor,
    \item mapowanie odpowiedzi z formatu dostawcy na struktury \gls{dto} wykorzystywane w aplikacji,
    \item zastosowanie mechanizmów optymalizacyjnych po stronie serwera.
\end{itemize}

Istotnym elementem jest użycie adnotacji \texttt{@Cacheable} dla operacji pobierania popularnych kategorii.
Zastosowanie pamięci podręcznej ogranicza liczbę wywołań do zewnętrznej usługi w przypadku danych,
które zmieniają się relatywnie rzadko.

\subsubsection{Warstwa prezentacji -- \texttt{GifController}}

\begin{figure}[H]
    \centering
    \includegraphics[width=\textwidth]{./attachments/implementacja-backendu/czat/gif/controller}
    \caption{Kontroler REST \texttt{GifController} udostępniający endpointy dla frontendu: lista trendujących kategorii oraz wyszukiwanie GIF-ów z parametrami zapytania.}
    \label{fig:gif:controller}
\end{figure}

Rysunek~\ref{fig:gif:controller} przedstawia kontroler REST udostępniający funkcjonalności GIF-ów na potrzeby frontendu.
Udostępniono dwa endpointy:
\begin{itemize}
    \item \texttt{GET /gifs/trending} -- zwraca listę popularnych kategorii,
    \item \texttt{GET /gifs/search} -- zwraca wyniki wyszukiwania dla podanej frazy, ze wsparciem paginacji.
\end{itemize}

