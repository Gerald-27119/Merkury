%! Author = Adam
%! Date = 01/01/2025

\subsection{Moduł czatu}
\label{subsec:modul-czatu}

%TODO: wszedzeie zawsze jasny motyw, w prezentacji systemu tez na innym PRerze!

Chat.java
ChatMessage.java
ChatMessageAttachedFile.java

\subsubsection{Integracja z dostawcą GIF-ów}
\label{subsubsec:integracja-tenor}

W module czatu zaimplementowano integrację z zewnętrznym dostawcą GIF-ów w celu umożliwienia
użytkownikom wysyłania GIF-ów na czacie.
Jako źródło danych wykorzystano usługę Tenor, a integrację rozdzielono na trzy warstwy:
klienta dostawcy (provider), serwis aplikacyjny oraz kontroler REST.

\paragraph{Klient dostawcy --\texttt{TenorGifProviderClient}}
Rysunek~X przedstawia klasę odpowiedzialną za bezpośrednią komunikację z API Tenor.
Klient wykorzystuje \texttt{WebClient} (Spring WebFlux), co umożliwia wykonywanie zapytań w sposób
nieblokujący oraz naturalnie wpisuje się w reaktywny charakter modułu (zwracanie typów \texttt{Mono}).
W ramach integracji udostępniono dwa podstawowe scenariusze:
pobranie listy popularnych kategorii (endpoint \texttt{/categories}) oraz wyszukiwanie GIF-ów
dla podanej frazy (endpoint \texttt{/search}).

Na poziomie budowy zapytań zastosowano parametry konfigurujące sposób zwracania wyników:
lokalizację (\texttt{locale}) oraz limit liczby elementów na stronę.
Mechanizm stronicowania jest obsługiwany poprzez opcjonalny parametr \texttt{pos}
(przekazywany dalej jako \texttt{next}), dzięki czemu warstwa kliencka może pobierać kolejne porcje
wyników bez konieczności utrzymywania stanu po stronie aplikacji.

\paragraph{Warstwa serwisowa -- \texttt{GifService}}
Rysunek~Y prezentuje serwis aplikacyjny, który stanowi pośrednią warstwę pomiędzy kontrolerem
a klientem dostawcy. Odpowiada on za:
\begin{itemize}
    \item delegowanie wywołań do klienta Tenor,
    \item mapowanie odpowiedzi z formatu dostawcy na struktury \gls{dto} wykorzystywane w aplikacji,
    \item zastosowanie mechanizmów optymalizacyjnych po stronie serwera.
\end{itemize}

Istotnym elementem jest użycie adnotacji \texttt{@Cacheable} dla operacji pobierania popularnych kategorii.
Zastosowanie pamięci podręcznej ogranicza liczbę wywołań do zewnętrznej usługi w przypadku danych,
które zmieniają się relatywnie rzadko.

\paragraph{Warstwa prezentacji --\texttt{GifController}}
Rysunek~Z przedstawia kontroler REST udostępniający funkcjonalności GIF-ów na potrzeby frontendu.
Udostępniono dwa endpointy:
\begin{itemize}
    \item \texttt{GET /gifs/trending} -- zwraca listę popularnych kategorii,
    \item \texttt{GET /gifs/search} -- zwraca wyniki wyszukiwania dla podanej frazy, ze wsparciem paginacji.
\end{itemize}

