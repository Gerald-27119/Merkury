%! Author = Mateusz
%! Date = 01/12/2025

\subsection{Konteneryzacja}
\label{subsec:konteneryzacja}
W celu zapewnienia łatwego uruchamiania aplikacji oraz możliwości jej skalowania zastosowano
konteneryzację z wykorzystaniem \glslink{docker}{Dockera}.
Zarówno relacyjna \glslink{baza-danych}{baza danych} PostgreSQL, jak i \glslink{redis}{Redis} zostały uruchomione
w odseparowanych \glslink{kontener}{kontenerach}.
Do włączenia wielu kontenerów jednocześnie wykorzystano narzędzie
\glslink{docker-compose}{Docker Compose} (rys. \ref{img:docker-compose}), dzięki czemu wystarczy uruchomić jeden
plik konfiguracyjny, a wszystkie wymagane usługi są automatycznie uruchamiane z odpowiednimi ustawieniami.

Poniżej opisano skonteneryzowane \glslink{baza-danych}{bazy danych}:
\newline
\textbf{PostgreSQL} – usługa uruchamiana jest na podstawie obrazu \texttt{postgres:latest},
pobieranego ze zdalnego repozytorium \glslink{docker-hub}{Docker Hub}.
\glslink{kontener}{Kontener} odpowiada za działanie relacyjnej \glslink{baza-danych}{bazy danych} systemu.
Parametry połączenia, takie jak nazwa użytkownika, hasło oraz nazwa \glslink{baza-danych}{bazy danych},
określono za pomocą zmiennych środowiskowych zdefiniowanych w pliku \texttt{postgres.env}.
\glslink{baza-danych}{Baza danych} jest udostępniana lokalnie na porcie \texttt{5432}.
\newline
\textbf{Redis} – usługa uruchamiana jest na podstawie obrazu \texttt{redis:latest},
pobieranego ze zdalnego repozytorium \glslink{docker-hub}{Docker Hub}.
\glslink{kontener}{Kontener} odpowiada za działanie bazy typu in-memory wykorzystywanej do krótkoterminowego
przechowywania danych oraz mechanizmów \glslink{cache}{cache’owania}.
Usługa jest udostępniana lokalnie na porcie \texttt{6379}.

Zastosowanie \glslink{docker}{Dockera} i \glslink{docker-compose}{Docker Compose} umożliwia łatwe odtworzenie środowiska na
dowolnej maszynie, ogranicza liczbę ręcznych kroków konfiguracyjnych oraz ułatwia dalsze
skalowanie i automatyzację procesu wdrażania aplikacji.

\begin{figure}[H]
    \centering
    \includegraphics[width=0.78\textwidth]{attachments/implementacja-backendu/docker-compose}
    \caption{Plik konfiguracyjny \glslink{docker-compose}{Docker Compose}}
    \label{img:docker-compose}
\end{figure}
