%! Author = Mateusz
%! Date = 01/12/2025

\subsection{Konteneryzacja}
\label{subsec:konteneryzacja}
W celu zapewnienia łatwego uruchamiania aplikacji oraz możliwości jej skalowania zastosowano
konteneryzację z wykorzystaniem Dockera.
Zarówno relacyjna baza danych PostgreSQL, jak i magazyn klucz–wartość Redis zostały uruchomione
w odseparowanych kontenerach.
Do uruchamiania wielu kontenerów jednocześnie wykorzystano narzędzie
Docker Compose (rys. \ref{img:docker-compose}), dzięki czemu wystarczy uruchomić jeden
plik konfiguracyjny, a wszystkie wymagane usługi są automatycznie uruchamiane z odpowiednimi ustawieniami.

Poniżej opisano skonteneryzowane bazy danych:
\newline
\textbf{PostgreSQL} – usługa uruchamiana jest na podstawie obrazu \texttt{postgres:latest},
pobieranego ze zdalnego repozytorium Docker Hub.
Kontener odpowiada za działanie relacyjnej bazy danych systemu.
Parametry połączenia, takie jak nazwa użytkownika, hasło oraz nazwa bazy danych,
określono za pomocą zmiennych środowiskowych zdefiniowanych w pliku \texttt{postgres.env}.
Baza danych jest udostępniana lokalnie na porcie \texttt{5432}.
\newline
\textbf{Redis} – usługa uruchamiana jest na podstawie obrazu \texttt{redis:latest},
pobieranego ze zdalnego repozytorium Docker Hub.
Kontener odpowiada za działanie bazy typu in-memory wykorzystywanej do krótkoterminowego
przechowywania danych oraz mechanizmów cache’owania.
Usługa jest udostępniana lokalnie na porcie \texttt{6379}.
Zastosowanie Dockera i Docker Compose umożliwia łatwe odtworzenie środowiska na
dowolnej maszynie, ogranicza liczbę ręcznych kroków konfiguracyjnych oraz ułatwia dalsze
skalowanie i automatyzację procesu wdrażania aplikacji.

\begin{figure}[H]
    \centering
    \includegraphics[width=0.78\textwidth]{attachments/implementacja-backendu/docker-compose}
    \caption{Plik konfiguracyjny docker-compose}
    \label{img:docker-compose}
\end{figure}
