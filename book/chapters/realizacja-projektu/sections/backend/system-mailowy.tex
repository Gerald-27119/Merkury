%! Author = kacper
%! Date = 03.01.2026

\subsection{System wysyłania wiadomości e-mail}
\label{subsec:system-mailowy}


W systemie zaimplementowano mechanizm wysyłania wiadomości e-mail, wykorzystywany w procesach obsługi kont użytkowników.
Funkcjonalność została zrealizowana z wykorzystaniem zewnętrznej usługi \gls{mail-trap}.

System wysyłania wiadomości e-mail wykorzystywany jest w dwóch przypadkach:
\begin{itemize}
    \item po pomyślnej rejestracji nowego użytkownika,
    \item w procesie resetowania hasła.
\end{itemize}

Do obsługi wysyłki wiadomości zastosowano bibliotekę JavaMail (Jakarta Mail).
Treść wiadomości definiowana jest w postaci statycznych plików HTML (rys. \ref{img:template}), przechowywanych w zasobach aplikacji, co umożliwia łatwą modyfikację zawartości bez ingerencji w kod źródłowy.

Konfiguracja połączenia SMTP realizowana jest dynamicznie z wykorzystaniem kilku portów komunikacyjnych.(rys. \ref{img:ports})
W przypadku niepowodzenia wysyłki na jednym porcie system podejmuje próbę ponownej wysyłki przy użyciu kolejnych portów. Dane uwierzytelniające do serwera SMTP pobierane są z poziomu zmiennych środowiskowych, co zwiększa poziom bezpieczeństwa i eliminuje konieczność ich przechowywania w repozytorium kodu.

Proces wysyłania wiadomości realizowany jest asynchronicznie oraz wspierany mechanizmem automatycznego ponawiania prób, z maksymalną liczbą trzech podejść.(rys. \ref{img:send-email-function})
W przypadku trwałego niepowodzenia generowany jest wyjątek, a odpowiednie zdarzenie rejestrowane jest w logach systemowych w celu dalszej analizy.(rys. \ref{img:recover})

\begin{figure}[H]
    \centering
    \includegraphics[width=1\textwidth]{attachments/implementacja-backendu/template}
    \caption{Przykładowy szablon wiadomości e-mail w formacie HTML}
    \label{img:template}
\end{figure}

\begin{figure}[H]
    \centering
    \includegraphics[width=0.78\textwidth]{attachments/implementacja-backendu/ports}
    \caption{Użyte porty komunikacyjne}
    \label{img:ports}
\end{figure}

\begin{figure}[H]
    \centering
    \includegraphics[width=1\textwidth]{attachments/implementacja-backendu/send_email_function}
    \caption{Implementacja metody odpowiedzialnej za asynchroniczną wysyłkę wiadomości e-mail}
    \label{img:send-email-function}
\end{figure}

\begin{figure}[H]
    \centering
    \includegraphics[width=1\textwidth]{attachments/implementacja-backendu/recover}
    \caption{Metoda odpowiedzialna za logowanie błędu}
    \label{img:recover}
\end{figure}