%! Author = Mateusz
%! Date = 28/11/2025

\subsection{Struktura projektu}
\label{subsec:struktura-projektu}

\glslink{backend}{Backend} aplikacji został zaimplementowany przy użyciu \glslink{framework}{frameworka} Spring Boot,
co umożliwiło stworzenie spójnej i skalowalnej architektury w prosty sposób.
W projekcie zastosowano rozwiązanie typu \glslink{rest_api}{REST API}, gdyż zespół projektowy dysponuje
największym doświadczeniem w jego wykorzystaniu.
Struktura projektu została zorganizowana zgodnie z podejściem \glslink{folder-by-type}{folder by type},
dzięki czemu każdy plik znajduje się w odpowiadającym mu katalogu.
Takie podejście ułatwia zarówno lokalizację istniejących plików, jak i określenie miejsca tworzenia nowych komponentów.
Poniżej przedstawiono przykładową strukturę katalogów \glslink{backend}{backendu}:

\begin{figure}[H]
\centering
\includegraphics[width=0.78\textwidth]{attachments/implementacja-backendu/struktura-backend1}
\caption{Struktura katalogów (1)}
\label{img:struktura-backend1}
\end{figure}

\begin{figure}[H]
\centering
\includegraphics[width=0.78\textwidth]{attachments/implementacja-backendu/struktura-backend2}
\caption{Struktura katalogów (2)}
\label{img:struktura-backend2}
\end{figure}

Dzięki takiej organizacji kod jest bardziej czytelny i łatwiejszy w utrzymaniu.
Umożliwia również szybkie odnalezienie odpowiednich modułów oraz ułatwia rozbudowę projektu w przyszłości.
