%! Author = Mateusz
%! Date = 26/11/2025

\subsection{Endpointy systemu}
\label{subsec:endpointy-systemu}

\newcounter{endpoint}[chapter]
\renewcommand{\theendpoint}{\thechapter.\arabic{endpoint}}

\newcommand{\epid}[1]{\textbf{Identyfikator:} & #1 \\ \hline}
\newcommand{\epurl}[1]{\textbf{Ścieżka:} & #1 \\ \hline}
\newcommand{\epname}[1]{\textbf{Nazwa:} & #1 \\ \hline}
\newcommand{\epinput}[1]{\textbf{Parametry wejściowe:} & #1 \\ \hline}
\newcommand{\epquery}[1]{\textbf{\glslink{query-params}{Query params:}} & #1 \\ \hline}
\newcommand{\epresponse}[1]{\textbf{Kod(y) statusu odpowiedzi:} & #1 \\ \hline}
\newcommand{\epoutput}[1]{\textbf{Dane zwracane:} & #1 \\ \hline}

\newenvironment{tabitemize}[1][]{%
    \begin{itemize}[
        leftmargin=*,
        nosep,
        topsep=0pt,
        partopsep=0pt,
        parsep=0pt,
        itemsep=0pt,
        before=\vspace*{-0.6\baselineskip},
        #1
    ]
}{%
    \end{itemize}
}

\newcommand{\cardendpoint}[7]{%
    \refstepcounter{endpoint}%
    \par\begin{center}
    \renewcommand{\arraystretch}{1.15}%
    \begin{tabularx}{\textwidth}{|>{\columncolor{lightgray}\raggedright\arraybackslash}p{0.2\textwidth}|X|}
    \rowcolor{lightgray}
    \multicolumn{2}{|c|}{\textbf{KARTA ENDPOINTU API}} \\ \hline
    \epid{#1}
    \epurl{#2}
    \epname{#3}
    \epinput{#4}
    \epquery{#5}
    \epresponse{#6}
    \epoutput{#7}
    \end{tabularx}
    \vspace{3pt}
    \textbf{Tabela \theendpoint:} Karta endpointu: #2\label{#1}
    \end{center}%
    \addcontentsline{lot}{table}{Tabela \theendpoint: Karta endpointu: #2}%
}

\newcommand{\summarytableendpoint}[4]{%
    \refstepcounter{endpoint}%
    \par\begin{center}
    \renewcommand{\arraystretch}{1.15}%
    \begin{tabularx}{\textwidth}{|>{\columncolor{lightgray}\raggedright\arraybackslash}p{0.18\textwidth}|X|}
    \rowcolor{lightgray}
    \multicolumn{2}{|>{\centering\arraybackslash}p{\dimexpr\textwidth-2\tabcolsep\relax}|}{\textbf{#1}} \\ \hline
    \rowcolor{lightgray}
    \textbf{Metoda HTTP} & \textbf{Ścieżka} \\
    \hline
    #4
    \end{tabularx}
    \vspace{3pt}
    \textbf{Tabela \theendpoint:} Zestawienie endpointów: #3\label{#2}
    \end{center}%
    \addcontentsline{lot}{table}{Tabela \theendpoint: Zestawienie endpointów: #3}%
}

Projektowany system udostępnia \glslink{rest_api}{REST-owe API} HTTP, za pomocą którego
klienci komunikują się z serwerem.

Na potrzeby niniejszej pracy przez \emph{endpoint \glslink{rest_api}{REST API}} rozumiany będzie
konkretny punkt dostępu do systemu, zdefiniowany jako para:
\[
    \text{metoda HTTP} + \text{ścieżka URL}
\]
pod którym aplikacja udostępnia określoną funkcjonalność lub zasób.
Przykładowo, endpoint \texttt{GET /public/spot/\{spotId\}} służy do
pobierania informacji o wybranym spocie.

W dalszej części rozdziału przedstawiono zestawienie wszystkich
endpointów HTTP systemu, a następnie szczegółowe karty wybranych
endpointów, opisujące m.in. parametry wejściowe, \glslink{query-params}{Query params}
oraz strukturę odpowiedzi.

% ================= PANEL UŻYTKOWNIKA =================
\summarytableendpoint
{Zestawienie wszystkich endpointów HTTP panelu użytkownika}
{tab:endpoints-user-dashboard}
{panelu użytkownika}
{
    GET   & /user-dashboard/profile \\ \hline
    GET   & /public/user-dashboard/profile/\{targetUsername\} \\ \hline
    PATCH & /user-dashboard/profile \\ \hline
    GET   & /user-dashboard/friends \\ \hline
    GET   & /public/user-dashboard/friends/\{targetUsername\} \\ \hline
    PATCH & /user-dashboard/friends \\ \hline
    PATCH & /user-dashboard/friends/change-status \\ \hline
    GET   & /user-dashboard/followers \\ \hline
    GET   & /public/user-dashboard/followers/\{targetUsername\} \\ \hline
    GET   & /user-dashboard/followed \\ \hline
    GET   & /public/user-dashboard/followed/\{targetUsername\} \\ \hline
    GET   & /user-dashboard/friends/find \\ \hline
    GET   & /user-dashboard/friends/invites \\ \hline
    PATCH & /user-dashboard/followed \\ \hline
    GET   & /user-dashboard/favorite-spots \\ \hline
    PATCH & /user-dashboard/favorite-spots \\ \hline
    POST  & /user-dashboard/add-spot-media \\ \hline
    GET   & /user-dashboard/is-spot-favourite \\ \hline
    GET   & /user-dashboard/photos \\ \hline
    GET   & /user-dashboard/comments \\ \hline
    PATCH & /user-dashboard/settings \\ \hline
    GET   & /user-dashboard/settings \\ \hline
    GET   & /user-dashboard/movies \\ \hline
    GET   & /user-dashboard/photos/\{targetUsername\} \\ \hline
    GET   & /user-dashboard/add-spot \\ \hline
    POST  & /user-dashboard/add-spot \\ \hline
    GET   & /user-dashboard/add-spot/coordinates \\ \hline
}

% ================= SPOTY =================
\summarytableendpoint
{Zestawienie wszystkich endpointów HTTP modułu spotów}
{tab:endpoints-spots}
{modułu spotów}
{
    GET   & /public/spot/gallery \\ \hline
    GET   & /public/spot/gallery-media-position \\ \hline
    GET   & /public/spot/gallery-fullscreen-media \\ \hline
    GET   & /public/spot/current-view \\ \hline
    GET   & /public/spot/current-view/spot-names \\ \hline
    GET   & /public/spot/\{spotId\} \\ \hline
    PATCH & /public/spot/increase-view-count \\ \hline
    GET   & /public/spot/search/map \\ \hline
    GET   & /public/spot/search/list \\ \hline
    GET   & /public/spot/names \\ \hline
    GET   & /public/spot/most-popular \\ \hline
    GET   & /public/spot/search/home-page \\ \hline
    GET   & /public/spot/search/home-page/locations \\ \hline
    GET   & /public/spot/search/home-page/advance \\ \hline
    GET   & /public/spot/get-spot-basic-weather \\ \hline
    GET   & /public/spot/get-spot-detailed-weather \\ \hline
    GET   & /public/spot/get-spot-wind-speeds \\ \hline
    GET   & /public/spot/get-spot-weather-timeline-plot-data \\ \hline
    PATCH & /public/spot/increase-spot-media-views-count \\ \hline
    PATCH & /public/spot/edit-spot-media-likes \\ \hline
    GET   & /spot/check-is-spot-media-liked \\ \hline
    GET   & /public/spot/get-spot-time-zone \\ \hline
}

% ================= KOMENTARZE DO SPOTÓW =================
\summarytableendpoint
{Zestawienie wszystkich endpointów HTTP komentarzy do spotów}
{tab:endpoints-spot-comments}
{komentarzy do spotów}
{
    GET    & /public/spot/\{spotId\}/comments \\ \hline
    GET    & /public/spot/\{spotId\}/comments/\{commentId\} \\ \hline
    POST   & /spot/\{spotId\}/comments \\ \hline
    DELETE & /spot/comments/\{commentId\} \\ \hline
    PATCH  & /spot/comments/\{commentId\} \\ \hline
    PATCH  & /spot/comments/\{commentId\}/vote \\ \hline
    GET    & /spot/comments/vote-type \\ \hline
}

% ================= POSTY FORUM =================
\summarytableendpoint
{Zestawienie wszystkich endpointów HTTP postów forum}
{tab:endpoints-forum-posts}
{postów forum}
{
    GET    & /public/post/\{postId\} \\ \hline
    GET    & /public/post \\ \hline
    POST   & /post \\ \hline
    DELETE & /post/\{postId\} \\ \hline
    PATCH  & /post/\{postId\} \\ \hline
    PATCH  & /post/\{postId\}/vote \\ \hline
    PATCH  & /post/\{postId\}/follow \\ \hline
    PATCH  & /post/\{postId\}/report \\ \hline
    GET    & /public/categories-tags \\ \hline
}

% ================= KOMENTARZE FORUM =================
\summarytableendpoint
{Zestawienie wszystkich endpointów HTTP komentarzy forum}
{tab:endpoints-forum-comments}
{komentarzy forum}
{
    GET    & /public/post/\{postId\}/comments \\ \hline
    GET    & /public/comments/\{commentId\}/replies \\ \hline
    POST   & /post/\{postId\}/comments \\ \hline
    DELETE & /post/comments/\{commentId\} \\ \hline
    PATCH  & /post/comments/\{commentId\} \\ \hline
    PATCH  & /post/comments/\{commentId\}/vote \\ \hline
    PATCH  & /post/comments/\{commentId\}/report \\ \hline
    POST   & /comments/\{commentId\}/replies \\ \hline
}

% ================= KONTO UŻYTKOWNIKA / AUTORYZACJA =================
\summarytableendpoint
{Zestawienie wszystkich endpointów HTTP konta użytkownika i autoryzacji}
{tab:endpoints-account-auth}
{konta użytkownika i autoryzacji}
{
    POST & /public/account/register \\ \hline
    POST & /public/account/login \\ \hline
    GET  & /account/login-success \\ \hline
    POST & /public/account/forgot-password \\ \hline
    POST & /public/account/set-new-password \\ \hline
    GET  & /account/check \\ \hline
}

% ================= GIF-Y (TENOR) =================
\summarytableendpoint
{Zestawienie wszystkich endpointów HTTP integracji GIF-ów (Tenor)}
{tab:endpoints-gifs}
{integracji GIF-ów}
{
    GET & /gifs/trending \\ \hline
    GET & /gifs/search \\ \hline
}

% ================= CZAT =================
\summarytableendpoint
{Zestawienie wszystkich endpointów HTTP modułu czatu}
{tab:endpoints-chat}
{modułu czatu}
{
    GET   & /chats/\{chatId\}/messages \\ \hline
    GET   & /chats/user-chats \\ \hline
    POST  & /chats/get-or-create-private-chat \\ \hline
    POST  & /chats/\{chatId\}/send-files \\ \hline
    POST  & /chats/create/group \\ \hline
    PATCH & /chats/\{chatId\} \\ \hline
    GET   & /chats/group-chat/add/search/\{chatId\} \\ \hline
    PUT   & /chats/add/users/\{chatId\} \\ \hline
}

%! Author = Mateusz
%! Date = 30/11/2025

\noindent\textbf{Panel użytkownika}

\cardendpoint
{EP01}
{/user-dashboard/profile}
{Pobierz własny profil użytkownika}
{Brak}
{Brak}
{200 OK, 404 Not Found, 401 Unauthorized}
{%
    \begin{tabitemize}
        \item \textbf{username}: String
        \item \textbf{profilePhoto}: String (URL)
        \item \textbf{followersCount}: Integer
        \item \textbf{followedCount}: Integer
        \item \textbf{friendsCount}: Integer
        \item \textbf{photosCount}: Integer
        \item \textbf{mostPopularPhotos}: List<ImageDto>
    \end{tabitemize}
}

\cardendpoint {EP02} {/public/user-dashboard/profile/\{targetUsername\}} {Pobierz profil innego użytkownika (widok publiczny)} {%
    \begin{tabitemize}
        \item \textbf{targetUsername}: String (nazwa użytkownika w ścieżce URL)
    \end{tabitemize}
} {Brak} {200 OK, 404 Not Found} {%
    \begin{tabitemize}
        \item \textbf{profile}: UserProfileDto, zawiera m.in.:
        \begin{itemize}
            \item \textbf{username}: String
            \item \textbf{profilePhoto}: String (URL)
            \item \textbf{followersCount}: Integer
            \item \textbf{followedCount}: Integer
            \item \textbf{friendsCount}: Integer
            \item \textbf{photosCount}: Integer
            \item \textbf{mostPopularPhotos}: List<ImageDto>
        \end{itemize}
        \item \textbf{friendStatus}: UserFriendStatus
        \item \textbf{isFollowing}: Boolean
        \item \textbf{isOwnProfile}: Boolean
    \end{tabitemize}
}

\cardendpoint {EP03} {/user-dashboard/profile} {Zmień zdjęcie profilowe użytkownika} {%
    \begin{tabitemize}
        \item \textbf{profilePhoto}: MultipartFile (nowe zdjęcie profilowe)
    \end{tabitemize}
} {Brak} {200 OK, 404 Not Found, 401 Unauthorized} {Brak (pusta odpowiedź)}

\cardendpoint {EP04} {/user-dashboard/friends} {Pobierz listę znajomych zalogowanego użytkownika} {Brak} {%
    \begin{tabitemize}
        \item \textbf{page}: Integer (opcjonalnie, domyślnie 0)
        \item \textbf{size}: Integer (opcjonalnie, domyślnie 20)
    \end{tabitemize}
} {200 OK, 404 Not Found, 401 Unauthorized} {%
    \begin{tabitemize}
        \item \textbf{items}: List<SocialDto>, gdzie każdy element zawiera:
        \begin{itemize}
            \item \textbf{username}: String
            \item \textbf{profilePhoto}: String (URL)
            \item \textbf{commonPrivateChatId}: Long (opcjonalne)
            \item \textbf{isUserFriend}: boolean
            \item \textbf{status}: UserFriendStatus
        \end{itemize}
        \item \textbf{hasNext}: boolean (czy istnieje kolejna strona wyników)
    \end{tabitemize}
}

\cardendpoint {EP05} {/public/user-dashboard/friends/\{targetUsername\}} {Pobierz listę znajomych innego użytkownika (widok publiczny)} {%
    \begin{tabitemize}
        \item \textbf{targetUsername}: String (nazwa użytkownika w ścieżce URL)
    \end{tabitemize}
} {%
    \begin{tabitemize}
        \item \textbf{page}: Integer (opcjonalnie, domyślnie 0)
        \item \textbf{size}: Integer (opcjonalnie, domyślnie 20)
    \end{tabitemize}
} {200 OK, 404 Not Found} {%
    \begin{tabitemize}
        \item \textbf{items}: List<SocialDto>
        \item \textbf{hasNext}: boolean
    \end{tabitemize}
}

\cardendpoint {EP06} {/user-dashboard/friends} {Edytuj relację znajomości z użytkownikiem} {Brak} {%
    \begin{tabitemize}
        \item \textbf{friendUsername}: String (nazwa użytkownika)
        \item \textbf{type}: UserRelationEditType (typ operacji, np. dodanie/wycofanie zaproszenia, usunięcie znajomego)
    \end{tabitemize}
} {200 OK, 400 Bad Request, 404 Not Found, 409 Conflict, 401 Unauthorized} {Brak (pusta odpowiedź)}

\cardendpoint {EP07} {/user-dashboard/friends/change-status} {Zmień status znajomego} {Brak} {%
    \begin{tabitemize}
        \item \textbf{friendUsername}: String (nazwa użytkownika)
        \item \textbf{status}: UserFriendStatus (nowy status relacji)
    \end{tabitemize}
} {200 OK, 400 Bad Request, 404 Not Found, 401 Unauthorized} {Brak (pusta odpowiedź)}

\cardendpoint {EP08} {/user-dashboard/followers} {Pobierz obserwujących zalogowanego użytkownika} {Brak} {%
    \begin{tabitemize}
        \item \textbf{page}: Integer (opcjonalnie, domyślnie 0)
        \item \textbf{size}: Integer (opcjonalnie, domyślnie 20)
    \end{tabitemize}
} {200 OK, 404 Not Found, 401 Unauthorized} {%
    \begin{tabitemize}
        \item \textbf{items}: List<SocialDto>
        \item \textbf{hasNext}: boolean
    \end{tabitemize}
}

\cardendpoint {EP09} {/public/user-dashboard/followers/\{targetUsername\}} {Pobierz obserwujących wybranego użytkownika (widok publiczny)} {%
    \begin{tabitemize}
        \item \textbf{targetUsername}: String
    \end{tabitemize}
} {%
    \begin{tabitemize}
        \item \textbf{page}: Integer (opcjonalnie, domyślnie 0)
        \item \textbf{size}: Integer (opcjonalnie, domyślnie 20)
    \end{tabitemize}
} {200 OK, 404 Not Found} {%
    \begin{tabitemize}
        \item \textbf{items}: List<SocialDto>
        \item \textbf{hasNext}: boolean
    \end{tabitemize}
}

\cardendpoint {EP10} {/user-dashboard/followed} {Pobierz obserwowanych przez zalogowanego użytkownika} {Brak} {%
    \begin{tabitemize}
        \item \textbf{page}: Integer (opcjonalnie, domyślnie 0)
        \item \textbf{size}: Integer (opcjonalnie, domyślnie 20)
    \end{tabitemize}
} {200 OK, 404 Not Found, 401 Unauthorized} {%
    \begin{tabitemize}
        \item \textbf{items}: List<SocialDto>
        \item \textbf{hasNext}: boolean
    \end{tabitemize}
}

\cardendpoint {EP11} {/public/user-dashboard/followed/\{targetUsername\}} {Pobierz obserwowanych wybranego użytkownika (widok publiczny)} {%
    \begin{tabitemize}
        \item \textbf{targetUsername}: String
    \end{tabitemize}
} {%
    \begin{tabitemize}
        \item \textbf{page}: Integer (opcjonalnie, domyślnie 0)
        \item \textbf{size}: Integer (opcjonalnie, domyślnie 20)
    \end{tabitemize}
} {200 OK, 404 Not Found} {%
    \begin{tabitemize}
        \item \textbf{items}: List<SocialDto>
        \item \textbf{hasNext}: boolean
    \end{tabitemize}
}

\cardendpoint {EP12} {/user-dashboard/friends/find} {Wyszukaj użytkowników po nazwie użytkownika} {Brak} {%
    \begin{tabitemize}
        \item \textbf{query}: String (fraza wyszukiwania)
        \item \textbf{page}: Integer (opcjonalnie, domyślnie 0)
        \item \textbf{size}: Integer (opcjonalnie, domyślnie 20)
    \end{tabitemize}
} {200 OK, 404 Not Found, 401 Unauthorized} {%
    \begin{tabitemize}
        \item \textbf{items}: List<SocialDto>
        \item \textbf{hasNext}: boolean
    \end{tabitemize}
}

\cardendpoint {EP13} {/user-dashboard/friends/invites} {Pobierz wszystkie zaproszenia do znajomych} {Brak} {%
    \begin{tabitemize}
        \item \textbf{page}: Integer (opcjonalnie, domyślnie 0)
        \item \textbf{size}: Integer (opcjonalnie, domyślnie 20)
    \end{tabitemize}
} {200 OK, 404 Not Found, 401 Unauthorized} {%
    \begin{tabitemize}
        \item \textbf{items}: List<SocialDto>
        \item \textbf{hasNext}: boolean
    \end{tabitemize}
}

\cardendpoint {EP14} {/user-dashboard/followed} {Edytuj listę obserwowanych użytkowników} {Brak} {%
    \begin{tabitemize}
        \item \textbf{followedUsername}: String (nazwa użytkownika)
        \item \textbf{type}: UserRelationEditType (typ operacji)
    \end{tabitemize}
} {200 OK, 400 Bad Request, 404 Not Found, 409 Conflict, 401 Unauthorized} {Brak (pusta odpowiedź)}

\cardendpoint {EP15} {/user-dashboard/favorite-spots} {Pobierz listę ulubionych spotów użytkownika} {Brak} {%
    \begin{tabitemize}
        \item \textbf{type}: FavoriteSpotsListType (typ listy, np. ulubione/odwiedzone)
        \item \textbf{page}: Integer (opcjonalnie, domyślnie 0)
        \item \textbf{size}: Integer (opcjonalnie, domyślnie 10)
    \end{tabitemize}
} {200 OK, 401 Unauthorized} {%
    \begin{tabitemize}
        \item \textbf{items}: List<FavoriteSpotDto>, gdzie każdy element zawiera:
        \begin{itemize}
            \item \textbf{id}: Long
            \item \textbf{name}: String
            \item \textbf{country}: String
            \item \textbf{city}: String
            \item \textbf{description}: String
            \item \textbf{rating}: Double
            \item \textbf{viewsCount}: Integer
            \item \textbf{imageUrl}: String (URL)
            \item \textbf{type}: FavoriteSpotsListType
            \item \textbf{coords}: SpotCoordinatesDto
            \item \textbf{tags}: Set<SpotTagDto>
        \end{itemize}
        \item \textbf{hasNext}: boolean
    \end{tabitemize}
}

\cardendpoint {EP16} {/user-dashboard/favorite-spots} {Modyfikuj listę ulubionych spotów użytkownika} {Brak} {%
    \begin{tabitemize}
        \item \textbf{type}: FavoriteSpotsListType
        \item \textbf{spotId}: Long (identyfikator spotu)
        \item \textbf{operationType}: FavouriteSpotListOperationType (typ operacji, np. dodanie/usunięcie)
    \end{tabitemize}
} {200 OK, 404 Not Found, 401 Unauthorized} {Brak (pusta odpowiedź)}

\cardendpoint {EP17} {/user-dashboard/add-spot-media} {Dodaj media do istniejącego spotu użytkownika} {%
    \begin{tabitemize}
        \item \textbf{mediaFiles}: List<MultipartFile> (lista plików multimedialnych)
    \end{tabitemize}
} {%
    \begin{tabitemize}
        \item \textbf{spotId}: Long (identyfikator spotu)
    \end{tabitemize}
} {200 OK, 404 Not Found, 401 Unauthorized} {Brak (pusta odpowiedź)}

\cardendpoint {EP18} {/user-dashboard/is-spot-favourite} {Sprawdź, czy spot jest w ulubionych użytkownika} {Brak} {%
    \begin{tabitemize}
        \item \textbf{spotId}: Long (identyfikator spotu)
    \end{tabitemize}
} {200 OK, 401 Unauthorized} {%
    \begin{tabitemize}
        \item \textbf{isFavouriteSpot}: boolean
    \end{tabitemize}
}

\cardendpoint {EP19} {/user-dashboard/photos} {Pobierz posortowane zdjęcia użytkownika} {Brak} {%
    \begin{tabitemize}
        \item \textbf{type}: DateSortType (typ sortowania po dacie)
        \item \textbf{from}: LocalDate (opcjonalnie, data od)
        \item \textbf{to}: LocalDate (opcjonalnie, data do)
        \item \textbf{page}: Integer (opcjonalnie, domyślnie 0)
        \item \textbf{size}: Integer (opcjonalnie, domyślnie 20)
    \end{tabitemize}
} {200 OK, 400 Bad Request, 401 Unauthorized} {%
    \begin{tabitemize}
        \item \textbf{items}: List<DatedMediaGroupDto>, gdzie:
        \begin{itemize}
            \item \textbf{date}: LocalDate (data grupy)
            \item \textbf{media}: List<MediaDto>, każdy element:
            \begin{itemize}
                \item \textbf{src}: String (URL/ścieżka do pliku)
                \item \textbf{heartsCount}: Integer
                \item \textbf{viewsCount}: Integer
                \item \textbf{id}: Long
            \end{itemize}
        \end{itemize}
        \item \textbf{hasNext}: boolean
    \end{tabitemize}
}

\cardendpoint {EP20} {/user-dashboard/comments} {Pobierz posortowane komentarze użytkownika} {Brak} {%
    \begin{tabitemize}
        \item \textbf{type}: DateSortType (typ sortowania po dacie)
        \item \textbf{from}: LocalDate (opcjonalnie, data od)
        \item \textbf{to}: LocalDate (opcjonalnie, data do)
        \item \textbf{page}: Integer (opcjonalnie, domyślnie 0)
        \item \textbf{size}: Integer (opcjonalnie, domyślnie 20)
    \end{tabitemize}
} {200 OK, 400 Bad Request, 401 Unauthorized} {%
    \begin{tabitemize}
        \item \textbf{items}: List<DatedCommentsGroupDto>, gdzie:
        \begin{itemize}
            \item \textbf{date}: LocalDate (data)
            \item \textbf{spotName}: String (nazwa spotu)
            \item \textbf{comments}: List<CommentDto>, każdy element:
            \begin{itemize}
                \item \textbf{addTime}: LocalTime (HH:mm)
                \item \textbf{id}: Long
                \item \textbf{text}: String
                \item \textbf{spotName}: String
            \end{itemize}
        \end{itemize}
        \item \textbf{hasNext}: boolean
    \end{tabitemize}
}

\cardendpoint {EP21} {/user-dashboard/settings} {Edytuj ustawienia konta użytkownika} {%
    \begin{tabitemize}
        \item \textbf{UserEditDataDto}: JSON z danymi do zmiany (np. nazwa użytkownika, e-mail, hasło itd.)
    \end{tabitemize}
} {Brak} {200 OK, 400 Bad Request, 403 Forbidden, 404 Not Found, 409 Conflict, 401 Unauthorized} {Brak (pusta odpowiedź)}

\cardendpoint {EP22} {/user-dashboard/settings} {Pobierz dane konta użytkownika} {Brak} {Brak} {200 OK, 404 Not Found, 401 Unauthorized} {%
    \begin{tabitemize}
        \item \textbf{username}: String
        \item \textbf{email}: String
        \item \textbf{provider}: Provider (np. lokalny, Google)
    \end{tabitemize}
}

\cardendpoint {EP23} {/user-dashboard/movies} {Pobierz posortowane filmy użytkownika} {Brak} {%
    \begin{tabitemize}
        \item \textbf{type}: DateSortType (typ sortowania po dacie)
        \item \textbf{from}: LocalDate (opcjonalnie, data od)
        \item \textbf{to}: LocalDate (opcjonalnie, data do)
        \item \textbf{page}: Integer (opcjonalnie, domyślnie 0)
        \item \textbf{size}: Integer (opcjonalnie, domyślnie 20)
    \end{tabitemize}
} {200 OK, 400 Bad Request, 401 Unauthorized} {%
    \begin{tabitemize}
        \item \textbf{items}: List<DatedMediaGroupDto>
        \item \textbf{hasNext}: boolean
    \end{tabitemize}
}

\cardendpoint {EP24} {/user-dashboard/photos/\{targetUsername\}} {Pobierz wszystkie zdjęcia wybranego użytkownika} {%
    \begin{tabitemize}
        \item \textbf{targetUsername}: String
    \end{tabitemize}
} {%
    \begin{tabitemize}
        \item \textbf{page}: Integer (opcjonalnie, domyślnie 0)
        \item \textbf{size}: Integer (opcjonalnie, domyślnie 20)
    \end{tabitemize}
} {200 OK, 400 Bad Request, 401 Unauthorized} {%
    \begin{tabitemize}
        \item \textbf{items}: List<DatedMediaGroupDto>
        \item \textbf{hasNext}: boolean
    \end{tabitemize}
}

\cardendpoint {EP25} {/user-dashboard/add-spot} {Pobierz listę spotów dodanych przez użytkownika} {Brak} {%
    \begin{tabitemize}
        \item \textbf{page}: Integer (opcjonalnie, domyślnie 0)
        \item \textbf{size}: Integer (opcjonalnie, domyślnie 20)
    \end{tabitemize}
} {200 OK, 401 Unauthorized} {%
    \begin{tabitemize}
        \item \textbf{items}: List<AddSpotDto>, każdy element:
        \begin{itemize}
            \item \textbf{id}: Long
            \item \textbf{name}: String
            \item \textbf{description}: String
            \item \textbf{country}: String
            \item \textbf{region}: String
            \item \textbf{city}: String
            \item \textbf{street}: String
            \item \textbf{borderPoints}: List<BorderPoint> (x, y)
            \item \textbf{firstPhotoUrl}: String (URL)
        \end{itemize}
        \item \textbf{hasNext}: boolean
    \end{tabitemize}
}

\cardendpoint {EP26} {/user-dashboard/add-spot} {Dodaj nowy spot użytkownika} {%
    \begin{tabitemize}
        \item \textbf{spot}: String (część multipart, JSON z danymi nowego spotu)
        \item \textbf{media}: List<MultipartFile> (część multipart, pliki multimedialne spotu)
    \end{tabitemize}
} {Brak} {200 OK, 404 Not Found, 401 Unauthorized} {Brak (pusta odpowiedź)}

\cardendpoint {EP27} {/user-dashboard/add-spot/coordinates} {Wyszukaj koordynaty lokalizacji na podstawie zapytania tekstowego} {Brak} {%
    \begin{tabitemize}
        \item \textbf{query}: String (opis lokalizacji, np. adres, nazwa miejsca)
    \end{tabitemize}
} {200 OK, 401 Unauthorized} {%
    \begin{tabitemize}
        \item \textbf{x}: Double (szerokość/długość geograficzna – składnik 1)
        \item \textbf{y}: Double (szerokość/długość geograficzna – składnik 2)
    \end{tabitemize}
}

%! Author = Adam
%! Date = 22/11/2025

\subsubsection{Scenariusze przypadków użycia dla mapy}

\usecasecard{tab:pu8-mapa}{Przeglądanie mapy spotów}{%
    \ucpriority{Wysoki}
    \ucactors{Użytkownik niezalogowany, Usługa do wyświetlania mapy}
    \ucdesc{Użytkownik przegląda mapę spotów.}
    \ucpre{Użytkownik znajduje się w module mapy.}
    \ucpost{Mapa ze spotami została wyświetlona, a użytkownik może przybliżać, oddalać i przesuwać widok.}
    \ucmain{%
        \begin{enumerate}[nosep,leftmargin=16pt,labelindent=0pt]
            \item System inicjuje widok mapy z domyślnym obszarem.
            \item System pobiera listę spotów.
            \item System rysuje znaczniki spotów na mapie.
        \end{enumerate}
    }
    \ucalt{%
        \begin{enumerate}[nosep,leftmargin=21pt,labelindent=0pt,label={}]
            \item[2a.] Usługa mapy jest niedostępna – system wyświetla komunikat o błędzie.
        \end{enumerate}
    }
}

\usecasecard{tab:pu11-szczegoly-spota}{Otwarcie szczegółów spota}{%
    \ucpriority{Wysoki}
    \ucactors{Użytkownik niezalogowany, Użytkownik zalogowany}
    \ucdesc{Użytkownik otwiera widok szczegółów wybranego spota.}
    \ucpre{Użytkownik widzi mapę spotów.}
    \ucpost{Wyświetlony został widok szczegółów spota z podstawowymi informacjami oraz jego lokalizacją na mapie.}
    \ucmain{%
        \begin{enumerate}[nosep,leftmargin=16pt,labelindent=0pt]
            \item Użytkownik wybiera spota z mapy.
            \item System pobiera dane szczegółowe spota (informacje opisowe, lokalizacja).
            \item System otwiera widok szczegółów spota.
            \item System prezentuje informacje o spocie oraz mapę przybliżoną do jego lokalizacji.
        \end{enumerate}
    }
    \ucalt{%
        \begin{enumerate}[nosep,leftmargin=21pt,labelindent=0pt,label={}]
            \item[2a.] Spot nie istnieje (został usunięty lub ukryty) – system informuje użytkownika i powraca do poprzedniego widoku.
            \item[2b.] Wystąpił błąd podczas pobierania danych spota – system wyświetla komunikat o błędzie i umożliwia ponowną próbę.
        \end{enumerate}
    }
}

\usecasecard{tab:pu12-komentarze-spota}{Przeglądanie komentarzy do spota}{%
    \ucpriority{Średni}
    \ucactors{Użytkownik niezalogowany}
    \ucdesc{Użytkownik czyta komentarze pod wybranym spotem.}
    \ucpre{Wyświetlany jest widok szczegółów spota.}
    \ucpost{Lista komentarzy do spota została wyświetlona.}
    \ucmain{%
        \begin{enumerate}[nosep,leftmargin=16pt,labelindent=0pt]
            \item System pobiera komentarze powiązane ze spotem.
            \item System wyświetla komentarze w kolejności chronologicznej lub według popularności.
            \item Użytkownik przewija listę komentarzy.
        \end{enumerate}
    }
    \ucalt{%
        \begin{enumerate}[nosep,leftmargin=21pt,labelindent=0pt,label={}]
            \item[1a.] Spot nie ma jeszcze komentarzy – system wyświetla odpowiednią informację.
        \end{enumerate}
    }
}

\usecasecard{tab:pu13-pogoda}{Przeglądanie pogody na spocie}{%
    \ucpriority{Średni}
    \ucactors{Użytkownik zalogowany, Usługa danych pogodowych}
    \ucdesc{Użytkownik sprawdza prognozę pogody dla lokalizacji spota.}
    \ucpre{Wyświetlany jest widok szczegółów spota.}
    \ucpost{Prognoza pogody dla spota została wyświetlona.}
    \ucmain{%
        \begin{enumerate}[nosep,leftmargin=16pt,labelindent=0pt]
            \item Użytkownik otwiera zakładkę pogody.
            \item System wysyła zapytanie do usługi pogodowej z lokalizacją spota.
            \item System odbiera prognozę i prezentuje ją (temperatura, prędkość wiatru, opady).
        \end{enumerate}
    }
    \ucalt{%
        \begin{enumerate}[nosep,leftmargin=21pt,labelindent=0pt,label={}]
            \item[2a.] Usługa pogodowa jest niedostępna – system wyświetla komunikat o braku danych pogodowych.
        \end{enumerate}
    }
}

\usecasecard{tab:pu9-szukaj-na-mapie}{Wyszukiwanie spota na mapie}{%
    \ucpriority{Wysoki}
    \ucactors{Użytkownik niezalogowany}
    \ucdesc{Użytkownik wyszukuje spota po nazwie korzystając z pola wyszukiwania na mapie.}
    \ucpre{Użytkownik widzi mapę spotów.}
    \ucpost{Mapa zostaje ustawiona na wybranego spota lub listę dopasowań.}
    \ucmain{%
        \begin{enumerate}[nosep,leftmargin=16pt,labelindent=0pt]
            \item Użytkownik wpisuje frazę w polu wyszukiwania na mapie.
            \item System podpowiada listę pasujących spotów.
            \item Użytkownik wybiera spota z listy.
            \item System przenosi użytkownika na mapie do wybranego spota.
        \end{enumerate}
    }
    \ucalt{%
        \begin{enumerate}[nosep,leftmargin=21pt,labelindent=0pt,label={}]
            \item[2a.] Brak wyników dla podanej frazy – system informuje użytkownika o braku dopasowań.
        \end{enumerate}
    }
}

%! Author = Mateusz
%! Date = 21/12/2025

\subsubsection{Wymagania ogólne dla wyszukiwarki spotów}
\label{subsubsec:wymagania-ogolne-dla-wyszukiwarki-spotow}


\newlength{\wowyszContentWidth}
\setlength{\wowyszContentWidth}{0.74\textwidth}

\newlength{\wowyszLabelWidth}
\setlength{\wowyszLabelWidth}{0.2\textwidth}

\newlength{\wowyszHeaderHeight}
\setlength{\wowyszHeaderHeight}{12mm}

% --------- Pola karty (wiersze) ---------

% Id + priorytet – 4 kolumny
\newcommand{\wowyszpriority}[2]{%
    \textbf{Identyfikator:} & WOWYSZ-#1 &
    \textbf{Priorytet:} & #2 \\ \hline
}

% Wiersze z etykietą + treścią na 3 kolumny
\newcommand{\wowyszname}[1]{\textbf{Nazwa:}              &
\multicolumn{3}{|p{\wowyszContentWidth}|}{#1} \\ \hline}
\newcommand{\wowyszdesc}[1]{\textbf{Opis:}               &
\multicolumn{3}{|p{\wowyszContentWidth}|}{#1} \\ \hline}
\newcommand{\wowyszstakeholder}[1]{\textbf{Udziałowiec:} &
\multicolumn{3}{|p{\wowyszContentWidth}|}{#1} \\ \hline}
\newcommand{\wowyszrelated}[1]{\textbf{Wymagania powiązane:} &
\multicolumn{3}{|p{\wowyszContentWidth}|}{#1} \\ \hline}

\newcommand{\wowyszHeaderRow}[1]{%
    \rowcolor{lightgray}%
    \multicolumn{4}{|c|}{%
        \parbox[c][\wowyszHeaderHeight][c]{\linewidth}{%
            \centering\bfseries
            \vspace{1.2ex}%
            #1%
            \vspace{1.2ex}%
        }%
    }\\ \hline
}

\newcommand{\wowyszatcard}[5]{%
    \refstepcounter{awc}%
    {\centering
    \begin{longtable}{|
            >{\columncolor{lightgray}}p{\wowyszLabelWidth}|
        p{0.22\textwidth}|
            >{\columncolor{lightgray}}p{0.18\textwidth}|
        p{0.24\textwidth}|}

    \hline
    \wowyszHeaderRow{\shortstack{KARTA WYMAGANIA OGÓLNEGO DLA \\ WYSZUKIWARKI SPOTÓW}}
    \endfirsthead
    \hline
    \wowyszHeaderRow{\shortstack{KARTA WYMAGANIA OGÓLNEGO DLA \\ WYSZUKIWARKI SPOTÓW (cd.)}}
    \endhead

    % właściwa treść karty
    \wowyszpriority{#3}{#4}
    \wowyszname{#2}
    #5

    \end{longtable}
    \par} % koniec centrowania

    \vspace{3pt}
    \textbf{Tabela \theawc:}
    Karta wymagania ogólnego dla wyszukiwarki spotów: #2\label{#1}

    \addcontentsline{lot}{table}
    {Tabela \theawc: Karta wymagania ogólnego dla wyszukiwarki spotów: #2}%
}


\wowyszatcard
{wowysz:basic-search}
{Wyszukiwanie na prostej wyszukiwarce}
{01}
{M}
{
    \wowyszdesc{System udostępnia użytkownikowi prostą wyszukiwarkę spotów na stronie głównej,
        umożliwiającą przeglądanie listy spotów oraz wyszukiwanie wyników według lokalizacji
        (kraj, region, miasto). Dodatkowo system prezentuje karuzelę najpopularniejszych spotów
        dla wybranej lokalizacji wraz z podstawowymi informacjami o miejscu.}

    \wowyszstakeholder{U3}

    \wowyszrelated{%
        \hyperref[wfwysz:top-spots-carousel]{WFWYSZ-01},
        \hyperref[wfwysz:basic-location-search]{WFWYSZ-02},
        \hyperref[wfwysz:display-search-results]{WFWYSZ-03},
        \hyperref[wfwysz:location-autocomplete]{WFWYSZ-04},
        \hyperref[wfwysz:spot-actions-map-details]{WFWYSZ-05},
        \hyperref[wpwysz:load-under-10s]{WPWYSZ-01},
        \hyperref[wpwysz:autocomplete-location]{WPWYSZ-02}.%
    }
}

\wowyszatcard
{wowysz:advanced-search}
{Wyszukiwanie na zaawansowanej wyszukiwarce}
{02}
{M}
{
    \wowyszdesc{System udostępnia użytkownikowi zaawansowaną wyszukiwarkę spotów na stronie głównej,
        umożliwiającą filtrowanie listy spotów według miasta, tagów oraz minimalnej oceny, a także
        sortowanie wyników według oceny oraz popularności. Wyniki wyszukiwania są aktualizowane zgodnie
        z ustawionymi filtrami i sortowaniem, a użytkownik może przejść do szczegółów wybranego spota.}

    \wowyszstakeholder{U3}

    \wowyszrelated{%
        \hyperref[wfwysz:display-search-results]{WFWYSZ-03},
        \hyperref[wfwysz:location-autocomplete]{WFWYSZ-04},
        \hyperref[wfwysz:spot-actions-map-details]{WFWYSZ-05},
        \hyperref[wfwysz:advanced-city-tags]{WFWYSZ-06},
        \hyperref[wfwysz:filter-polarity-rating]{WFWYSZ-07},
        \hyperref[wfwysz:filter-rating]{WFWYSZ-08},
        \hyperref[wpwysz:load-under-10s]{WPWYSZ-01},
        \hyperref[wpwysz:autocomplete-location]{WPWYSZ-02}.%
    }
}

%! Author = Mateusz
%! Date = 30/11/2025


\noindent\textbf{Komentarze do spotów}

\cardendpoint
{EP15}
{/public/spot/\{spotId\}/comments}
{Pobierz stronicowaną listę komentarzy dla wskazanego spotu}
{%
    \begin{tabitemize}
        \item \textbf{spotId}: Long (identyfikator spota w ścieżce URL)
    \end{tabitemize}
}
{%
    \begin{tabitemize}
        \item \textbf{page}: Integer (numer strony, domyślnie 0; rozmiar strony = 2)
    \end{tabitemize}
}
{200 OK, 404 Not Found, 401 Unauthorized}
{%
    \begin{tabitemize}
        \item \textbf{Page<SpotCommentDto>}: stronicowana lista komentarzy dla danego spota, każdy element zawiera:
        \begin{itemize}
            \item \textbf{id}: Long (identyfikator komentarza)
            \item \textbf{author}: SpotCommentAuthorDto (dane autora komentarza)
            \item \textbf{text}: String (treść komentarza)
            \item \textbf{rating}: Double (ocena spota wystawiona w komentarzu, 0--5)
            \item \textbf{upvotes}: Integer (liczba głosów pozytywnych na komentarz)
            \item \textbf{downvotes}: Integer (liczba głosów negatywnych na komentarz)
            \item \textbf{publishDate}: LocalDateTime (data i godzina publikacji komentarza)
            \item \textbf{isUpVoted}: Boolean (czy bieżący użytkownik oddał głos w górę na ten komentarz)
            \item \textbf{isDownVoted}: Boolean (czy bieżący użytkownik oddał głos w dół na ten komentarz)
            \item \textbf{numberOfMedia}: Integer (łączna liczba dołączonych plików multimedialnych)
            \item \textbf{mediaList}: List<SpotCommentMediaDto> (lista pierwszych plików komentarza)
        \end{itemize}
    \end{tabitemize}
}

\cardendpoint
{EP16}
{/public/spot/\{spotId\}/comments/\{commentId\}}
{Pobierz pełną listę mediów powiązanych z komentarzem}
{%
    \begin{tabitemize}
        \item \textbf{spotId}: Long (identyfikator spota w ścieżce URL)
        \item \textbf{commentId}: Long (identyfikator komentarza w ścieżce URL)
    \end{tabitemize}
}
{Brak}
{200 OK, 404 Not Found}
{%
    \begin{tabitemize}
        \item \textbf{List<SpotCommentMediaDto>}: pełna lista mediów powiązanych z komentarzem, każdy element:
        \begin{itemize}
            \item \textbf{id}: Long (identyfikator pliku multimedialnego)
            \item \textbf{url}: String (URL pliku, używany do pobrania/wyświetlenia)
            \item \textbf{genericMediaType}: GenericMediaType (typ pliku, \texttt{PHOTO} lub \texttt{VIDEO})
        \end{itemize}
    \end{tabitemize}
}

\cardendpoint
{EP17}
{/spot/\{spotId\}/comments}
{Dodaj nowy komentarz do wskazanego spotu}
{%
    \begin{tabitemize}
        \item \textbf{spotId}: Long (identyfikator spota w ścieżce URL)
        \item \textbf{body}: SpotCommentAddDto (dane nowego komentarza), zawiera:
        \begin{itemize}
            \item \textbf{text}: String (treść komentarza)
            \item \textbf{rating}: Double (ocena spota w komentarzu, zakres 0--5)
            \item \textbf{mediaFiles}: List<MultipartFile> (lista załączonych plików, zdjęcia/filmy)
        \end{itemize}
    \end{tabitemize}
}
{Brak}
{201 Created, 404 Not Found, 401 Unauthorized, 422 Unprocessable Entity}
{Brak (pusta odpowiedź)}

\cardendpoint
{EP18}
{/spot/comments/\{commentId\}/vote}
{Oddaj głos na komentarz (góra/dół)}
{%
    \begin{tabitemize}
        \item \textbf{commentId}: Long (identyfikator komentarza w ścieżce URL)
    \end{tabitemize}
}
{%
    \begin{tabitemize}
        \item \textbf{isUpvote}: boolean (true = głos w górę, false = głos w dół)
    \end{tabitemize}
}
{200 OK, 401 Unauthorized, 404 Not Found, 409 Conflict, 403 Forbidden}
{Brak (pusta odpowiedź)}

\cardendpoint
{EP19}
{/spot/comments/vote-type}
{Pobierz informację, jak bieżący użytkownik zagłosował na komentarz}
{Brak}
{%
    \begin{tabitemize}
        \item \textbf{commentId}: Long (identyfikator komentarza)
    \end{tabitemize}
}
{200 OK, 404 Not Found, 401 Unauthorized}
{%
    \begin{tabitemize}
        \item \textbf{voteInfo}: SpotCommentVoteType (typ oddanego głosu, \texttt{UPVOTE}, \texttt{DOWNVOTE}, \texttt{NONE})
    \end{tabitemize}
}

%! Author = Mateusz
%! Date = 30/11/2025

\subsubsection{Forum – posty}
\label{subsubsec:forum-posty}

\cardendpoint
{EP05}
{/public/post/\{postId\}}
{Pobierz szczegółowe informacje o poście}
{%
    \begin{tabitemize}
        \item \textbf{postId}: Long (identyfikator posta w ścieżce URL)
    \end{tabitemize}
}
{Brak}
{200 OK, 404 Not Found}
{%
    \begin{tabitemize}
        \item \textbf{PostDetailsDto}, zawiera:
        \begin{itemize}
            \item \textbf{id}: Long (identyfikator posta)
            \item \textbf{title}: String (tytuł posta)
            \item \textbf{content}: String (pełna treść posta)
            \item \textbf{category}: ForumCategoryDto (kategoria forum, do której należy post)
            \item \textbf{tags}: List<ForumTagDto> (lista tagów przypisanych do posta)
            \item \textbf{author}: AuthorDto (dane autora posta)
            \item \textbf{isAuthor}: Boolean (czy bieżący użytkownik jest autorem posta)
            \item \textbf{isFollowed}: Boolean (czy bieżący użytkownik obserwuje ten post)
            \item \textbf{publishDate}: LocalDateTime (data i godzina publikacji posta)
            \item \textbf{views}: Integer (liczba wyświetleń posta)
            \item \textbf{upVotes}: Integer (liczba głosów w górę na post)
            \item \textbf{downVotes}: Integer (liczba głosów w dół na post)
            \item \textbf{isUpVoted}: Boolean (czy bieżący użytkownik oddał głos w górę na post)
            \item \textbf{isDownVoted}: Boolean (czy bieżący użytkownik oddał głos w dół na post)
            \item \textbf{commentsCount}: Integer (łączna liczba komentarzy pod postem)
        \end{itemize}
    \end{tabitemize}
}

%! Author = Mateusz
%! Date = 30/11/2025

\subsubsection{Forum – komentarze do postów}
\label{subsubsec:forum-komentarze}

\cardendpoint
{EP25}
{/public/post/\{postId\}/comments}
{Pobierz stronicowaną listę komentarzy posta}
{%
    \begin{tabitemize}
        \item \textbf{postId}: Long (identyfikator posta w ścieżce URL)
    \end{tabitemize}
}
{%
    \begin{tabitemize}
        \item \textbf{page}: Integer (numer strony, domyślnie 0)
        \item \textbf{size}: Integer (liczba komentarzy na stronie, domyślnie 10)
        \item \textbf{sortBy}: PostCommentSortField (pole sortowania, domyślnie \texttt{PUBLISH\_DATE})
        \item \textbf{sortDirection}: SortDirection (kierunek sortowania, domyślnie \texttt{DESC})
    \end{tabitemize}
}
{200 OK, 404 Not Found}
{%
    \begin{tabitemize}
        \item \textbf{Page<PostCommentGeneralDto>}, każdy element zawiera:
        \begin{itemize}
            \item \textbf{id}: Long (identyfikator komentarza)
            \item \textbf{content}: String (treść komentarza)
            \item \textbf{upVotes}: Integer (liczba głosów w górę)
            \item \textbf{downVotes}: Integer (liczba głosów w dół)
            \item \textbf{repliesCount}: Integer (liczba odpowiedzi)
            \item \textbf{publishDate}: LocalDateTime (data publikacji)
            \item \textbf{author}: AuthorDto (dane autora)
            \item \textbf{isAuthor}: Boolean (czy bieżący użytkownik jest autorem)
            \item \textbf{isUpVoted}: Boolean (czy użytkownik zagłosował w górę)
            \item \textbf{isDownVoted}: Boolean (czy użytkownik zagłosował w dół)
            \item \textbf{isReply}: Boolean (czy komentarz jest odpowiedzią)
            \item \textbf{isDeleted}: Boolean (czy komentarz został usunięty logicznie)
        \end{itemize}
    \end{tabitemize}
}

\cardendpoint
{EP26}
{/post/\{postId\}/comments}
{Dodaj nowy komentarz do posta}
{%
    \begin{tabitemize}
        \item \textbf{postId}: Long (identyfikator posta w ścieżce URL)
        \item \textbf{body}: PostCommentDto, zawiera:
        \begin{itemize}
            \item \textbf{content}: String (treść komentarza)
        \end{itemize}
    \end{tabitemize}
}
{Brak}
{201 Created, 400 Bad Request, 404 Not Found, 422 Unprocessable Entity}
{Brak (pusta odpowiedź)}

\cardendpoint
{EP27}
{/post/comments/\{commentId\}}
{Edytuj istniejący komentarz do posta}
{%
    \begin{tabitemize}
        \item \textbf{commentId}: Long (identyfikator komentarza w ścieżce URL)
        \item \textbf{body}: PostCommentDto, zawiera:
        \begin{itemize}
            \item \textbf{content}: String (treść komentarza)
        \end{itemize}
    \end{tabitemize}
}
{Brak}
{200 OK, 400 Bad Request, 401 Unauthorized, 403 Forbidden, 404 Not Found, 422 Unprocessable Entity}
{Brak (pusta odpowiedź)}

\cardendpoint
{EP28}
{/post/comments/\{commentId\}/vote}
{Oddaj głos na komentarz (góra/dół)}
{%
    \begin{tabitemize}
        \item \textbf{commentId}: Long (identyfikator komentarza w ścieżce URL)
    \end{tabitemize}
}
{%
    \begin{tabitemize}
        \item \textbf{isUpvote}: boolean (true = głos w górę, false = głos w dół)
    \end{tabitemize}
}
{200 OK, 403 Forbidden, 404 Not Found}
{Brak (pusta odpowiedź)}

\cardendpoint
{EP29}
{/comments/\{commentId\}/replies}
{Dodaj odpowiedź na komentarz}
{%
    \begin{tabitemize}
        \item \textbf{commentId}: Long (identyfikator komentarza nadrzędnego w ścieżce URL)
        \item \textbf{body}: PostCommentDto, zawiera:
        \begin{itemize}
            \item \textbf{content}: String (treść komentarza)
        \end{itemize}
    \end{tabitemize}
}
{Brak}
{201 Created, 400 Bad Request, 404 Not Found, 409 Conflict, 422 Unprocessable Entity}
{Brak (pusta odpowiedź)}

%! Author = Mateusz
%! Date = 30/11/2025

\noindent\textbf{Konto użytkownika – rejestracja, logowanie, hasło}

\cardendpoint
{EP75}
{/public/account/register}
{Zarejestruj nowego użytkownika}
{%
    \begin{tabitemize}
        \item \textbf{body}: UserRegisterDto (JSON), zawiera:
        \begin{itemize}
            \item \textbf{username}: String (3–16 znaków, litery/cyfry/\_)
            \item \textbf{email}: String (poprawny adres e-mail)
            \item \textbf{password}: String (8–16 znaków, min. jedna cyfra, mała i duża litera oraz znak specjalny)
        \end{itemize}
    \end{tabitemize}
}
{Brak}
{201 Created, 401 Unauthorized, 409 Conflict, 422 Unprocessable Entity, 500 Internal Server Error}
{%
    \begin{tabitemize}
        \item \textbf{body}: String (komunikat, np. \texttt{"User registered successfully"})
        \item JWT tokeny ustawione w ciasteczkach HTTP-only (automatyczne zalogowanie)
    \end{tabitemize}
}

\cardendpoint
{EP76}
{/public/account/login}
{Zaloguj użytkownika}
{%
    \begin{tabitemize}
        \item \textbf{body}: UserLoginDto (JSON), zawiera:
        \begin{itemize}
            \item \textbf{username}: String
            \item \textbf{password}: String
        \end{itemize}
    \end{tabitemize}
}
{Brak}
{200 OK, 401 Unauthorized, 422 Unprocessable Entity}
{%
    \begin{tabitemize}
        \item Brak (pusta odpowiedź w body)
        \item JWT tokeny zwrócone w ciasteczkach HTTP-only
    \end{tabitemize}
}

\cardendpoint
{EP77}
{/account/login-success}
{Obsłuż pomyślne logowanie OAuth2 i przekieruj użytkownika}
{Brak}
{Brak}
{302 Found, 404 Not Found, 409 Conflict, 422 Unprocessable Entity, 500 Internal Server Error}
{%
    \begin{tabitemize}
        \item \textbf{Redirect}: przekierowanie na stronę po zalogowaniu (\texttt{afterLoginPageUrl} z konfiguracji)
    \end{tabitemize}
}

\cardendpoint
{EP78}
{/public/account/forgot-password}
{Rozpocznij procedurę resetu hasła (wyślij link na e-mail)}
{%
    \begin{tabitemize}
        \item \textbf{body}: String (adres e-mail użytkownika w treści żądania)
    \end{tabitemize}
}
{Brak}
{200 OK, 404 Not Found, 422 Unprocessable Entity, 500 Internal Server Error}
{%
    \begin{tabitemize}
        \item \textbf{body}: String (komunikat, np. \texttt{"Password reset link sent to: user@example.com"})
        \item Link resetujący hasło wysłany na podany adres e-mail
    \end{tabitemize}
}

\cardendpoint
{EP79}
{/public/account/set-new-password}
{Ustaw nowe hasło użytkownika na podstawie tokenu resetującego}
{%
    \begin{tabitemize}
        \item \textbf{body}: UserPasswordResetDto (JSON), zawiera:
        \begin{itemize}
            \item \textbf{token}: String (UUID – token resetu hasła)
            \item \textbf{password}: String (8–16 znaków, min. jedna cyfra, mała i duża litera oraz znak specjalny)
        \end{itemize}
    \end{tabitemize}
}
{Brak}
{200 OK, 400 Bad Request, 404 Not Found, 422 Unprocessable Entity}
{%
    \begin{tabitemize}
        \item \textbf{body}: String (komunikat, np. \texttt{"Password set successfully!"})
    \end{tabitemize}
}

\cardendpoint
{EP80}
{/account/check}
{Sprawdź, czy użytkownik jest uwierzytelniony}
{Brak}
{Brak}
{200 OK, 401 Unauthorized, 403 Forbidden}
{%
    \begin{tabitemize}
        \item Brak (pusta odpowiedź; sam status informuje o uwierzytelnieniu)
    \end{tabitemize}
}

%! Author = Adam
%! Date = 30/12/2025

\subsection{Czat}
\label{subsec:chat-frontend}

\newcommand{\chatimplfig}[1]{./attachments/implementacja-frontendu/czat/#1}

W niniejszym rozdziale przedstawiono implementację modułu czatu po stronie \glslink{frontend}{frontendu}. \newline
Czat stanowi jeden z kluczowych elementów aplikacji, zapewniając komunikację w wariancie rozmów prywatnych
oraz grupowych.

Na główne \glslink{react-component}{komponenty} modułu składają się:
\begin{itemize}
    \item \texttt{ChatsPage} -- główny kontener układu,
    \item \texttt{ChatList} oraz \texttt{ListedChat} -- \glslink{react-component}{komponenty} odpowiedzialne za prezentację i obsługę listy rozmów,
    \item \texttt{ChatTopBar}, \texttt{ChatMessagingWindow}, \texttt{ChatBottomBar} -- elementy składające się na okno konwersacji,
    \item \texttt{EmojiGifWindowWrapper} wraz z \texttt{EmojiWindow} i \texttt{GifWindow} -- panel wyboru \glslink{emoji}{emoji}/\glslink{gif}{GIF-ów},
    \item \texttt{GroupChatParticipantsSideBar} -- panel listy uczestników rozmowy grupowej.
\end{itemize}

\subsubsection{ChatsPage -- główny kontener modułu}

\begin{itemize}
    \item \glslink{react-component}{Komponent} \texttt{ChatsPage} jest nadrzędnym kontenerem widoku czatu.
    \item Odpowiada za rozłożenie interfejsu na trzy sekcje:
    \begin{enumerate}
        \item lewy panel z listą rozmów,
        \item centralne okno konwersacji,
        \item opcjonalny panel boczny widoczny wyłącznie dla czatów grupowych.
    \end{enumerate}
\end{itemize}

\glslink{stan}{Stan} wyboru rozmowy oraz widoczność panelu bocznego pobierane są ze \glslink{stan}{stanu} globalnego \glslink{redux}{Redux}.

\subsubsection{Lista czatów}

\paragraph{Komponent ChatList}

\glslink{react-component}{Komponent} \texttt{ChatList} odpowiada za pobieranie oraz prezentację listy rozmów użytkownika.
Dane są ładowane z \glslink{backend}{backendu} przy użyciu \glslink{hook}{hook'a}
\texttt{useInfiniteQuery} z \glslink{tanstack-query}{TanStack Query}.
Zaimplementowano mechanizm \glslink{infinite-scroll}{przewijania nieskończonego},
który doładowuje kolejne strony wyników, gdy \glslink{sentinel}{element-strażnik} pojawi się na ekranie użytkownika.
W tym celu użyto mechanizmu \glslink{intersection-observer}{Intersection Observer}.

Po każdorazowym pobraniu strony dane są mapowane do lokalnego formatu i zapisywane w \glslink{redux}{Redux},
dzięki czemu \glslink{stan}{stan} listy rozmów jest współdzielony z innymi komponentami modułu.
Dodatkowo przy pierwszym renderowaniu, jeżeli użytkownik nie ma wybranego czatu,
ustawiana jest domyślna rozmowa.

W trakcie ładowania wyświetlane są elementy typu \glslink{skeleton-loader}{skeleton loader},
natomiast w przypadku doładowywania kolejnych stron prezentowany jest wskaźnik ładowania.

\paragraph{Komponent ListedChat}

\glslink{react-component}{Komponent} \texttt{ListedChat} reprezentuje pojedynczy element listy rozmów.
Wyświetla podstawowe informacje o czacie: nazwę, awatar, treść ostatniej wiadomości oraz czas jej wysłania.
\glslink{react-component}{Komponent} wizualnie rozróżnia stan:
\begin{itemize}
    \item aktualnie wybranego czatu,
    \item czatu posiadającego nowe, nieodczytane wiadomości,
    \item czatu nieaktywnego.
\end{itemize}

Kliknięcie w element listy aktualizuje identyfikator wybranego czatu w \glslink{redux}{Redux} i
jednocześnie czyści znacznik nowych wiadomości dla danej rozmowy.

\subsubsection{Okno konwersacji}

\paragraph{Komponent ChatContent}

\glslink{react-component}{Komponent} \texttt{ChatContent} pełni rolę kontenera na \glslink{react-component}{komponenty} okna konwersacji.
Jego zadaniem jest pobranie z \glslink{redux}{Redux} identyfikatora aktualnie wybranego czatu, a następnie odczytanie
pełnych danych rozmowy na podstawie selektora \newline \texttt{selectChatById}.

\texttt{ChatContent} składa okno rozmowy z trzech części:
paska nagłówka (\texttt{ChatTopBar}), listy wiadomości (\texttt{ChatMessagingWindow}) oraz paska tworzenia wiadomości
\newline (\texttt{ChatBottomBar}).

\paragraph{Komponent ChatTopBar}

\glslink{react-component}{Komponent} \texttt{ChatTopBar} pełni rolę nagłówka konwersacji.
W zależności od typu rozmowy (prywatna/grupowa) udostępnia różne akcje:
dla czatu prywatnego umożliwia przejście do profilu rozmówcy, natomiast w czacie grupowym
udostępnia przełącznik panelu bocznego oraz operacje zarządzania rozmową.
Część funkcjonalności realizowana jest w \glslink{modal}{oknach modalnych}
(tworzenie nowego czatu grupowego, edycja czatu lub dodawanie uczestników).

\paragraph{Komponent ChatMessagingWindow}

\glslink{react-component}{Komponent} \texttt{ChatMessagingWindow} odpowiada za wyświetlanie wiadomości
w obrębie wybranej konwersacji. Dane pobierane z \glslink{backend}{backendu} przy użyciu
\texttt{useInfiniteQuery}.
Wiadomości są renderowane w układzie „od dołu” (ostatnia wiadomość na końcu),
a doładowywanie starszych fragmentów odbywa się po przewinięciu w górę.

Dodatkowo komponent porządkuje wiadomości w czytelny sposób:
\begin{itemize}
    \item wstawia separatory dat przy zmianie dnia,
    \item grupuje wiadomości tego samego autora wysłane w krótkim odstępie czasu.
\end{itemize}

Okno konwersacji współpracuje również z aktualizacją danych w czasie rzeczywistym.
Po odebraniu nowej wiadomości jest ona natychmiast wyświetlana.

\paragraph{Komponent ChatBottomBar}

\glslink{react-component}{Komponent} \texttt{ChatBottomBar} realizuje wysyłanie wiadomości tekstowych,
załączników oraz interakcję z oknami wyboru \glslink{emoji}{emoji} i \glslink{gif}{GIF-ów}.
Wysyłanie wiadomości w czasie rzeczywistym odbywa się z użyciem \glslink{websocket}{WebSocket}
oraz protokołu \glslink{stomp}{STOMP}.
Obsługa załączników obejmuje wybór wielu plików oraz ich podgląd (jeżeli są zdjęciem).

\subsubsection{Okna emoji i GIF}

\paragraph{Komponent EmojiGifWindowWrapper}

\glslink{react-component}{Komponent} \texttt{EmojiGifWindowWrapper} stanowi warstwę pośrednią,
która w zależności od aktualnego trybu wyświetla jedno z dwóch okien:
\texttt{EmojiWindow} lub \texttt{GifWindow}.

\paragraph{Komponent EmojiWindow}

\glslink{react-component}{Komponent} \texttt{EmojiWindow} integruje zewnętrzny selektor \glslink{emoji}{emoji}
i umożliwia wyszukiwanie oraz wybór emotikonów.
Po kliknięciu \glslink{emoji}{emoji}, \glslink{react-component}{komponent} dopisuje go do aktualnie budowanej treści wiadomości.

\paragraph{Komponent GifWindow}

\glslink{react-component}{Komponent} \texttt{GifWindow} umożliwia wyszukiwanie oraz wysyłanie \glslink{gif}{GIF-ów}
z wykorzystaniem integracji z dostawcą \glslink{tenor}{Tenor} po stronie \glslink{backend}{backendu}.
Widok udostępnia dwa scenariusze:
\begin{itemize}
    \item prezentację popularnych kategorii,
    \item wyszukiwanie \glslink{gif}{GIF-ów} po frazie.
\end{itemize}

Wyszukiwanie realizowane jest przy użyciu \texttt{useInfiniteQuery}.
W przypadku kliknięcia w wybrany \glslink{gif}{GIF} jego adres \glslink{url}{URL} jest przesyłany jako treść wiadomości
przez \glslink{websocket}{WebSocket}/\glslink{stomp}{STOMP}, a okno selektora zostaje zamknięte.

\subsubsection{Lista uczestników czatu grupowego}

\paragraph{Komponent GroupChatParticipantsSideBar}

\glslink{react-component}{Komponent} \texttt{GroupChatParticipantsSideBar} jest panelem bocznym,
który pojawia się wyłącznie dla rozmów grupowych.
Panel prezentuje listę uczestników czatu wraz z awatarem i nazwą użytkownika,
a kliknięcie w wybraną osobę przekierowuje do jej profilu.

