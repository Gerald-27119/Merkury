%! Author = Mateusz
%! Date = 30/11/2025


\noindent\textbf{Komentarze do spotów}

\cardendpoint
{EP15}
{/public/spot/\{spotId\}/comments}
{Pobierz stronicowaną listę komentarzy dla wskazanego spotu}
{%
    \begin{tabitemize}
        \item \textbf{spotId}: Long (ID spotu w ścieżce URL)
    \end{tabitemize}
}
{%
    \begin{tabitemize}
        \item \textbf{page}: Integer (numer strony, domyślnie 0; rozmiar strony = 2)
    \end{tabitemize}
}
{200 OK, 404 Not Found, 401 Unauthorized}
{%
    \begin{tabitemize}
        \item \textbf{Page<SpotCommentDto>}: stronicowana lista komentarzy, każdy element zawiera:
        \begin{itemize}
            \item \textbf{id}: Long (ID komentarza)
            \item \textbf{author}: SpotCommentAuthorDto (dane autora)
            \item \textbf{text}: String (treść komentarza)
            \item \textbf{rating}: Double (ocena komentarza/spotu, 0--5)
            \item \textbf{upvotes}: Integer (liczba głosów pozytywnych)
            \item \textbf{downvotes}: Integer (liczba głosów negatywnych)
            \item \textbf{publishDate}: LocalDateTime (data publikacji)
            \item \textbf{isUpVoted}: Boolean (czy bieżący użytkownik zagłosował w górę)
            \item \textbf{isDownVoted}: Boolean (czy bieżący użytkownik zagłosował w dół)
            \item \textbf{numberOfMedia}: Integer (liczba dołączonych plików)
            \item \textbf{mediaList}: List<SpotCommentMediaDto> (lista pierwszych plików komentarza)
        \end{itemize}
    \end{tabitemize}
}

\cardendpoint
{EP16}
{/public/spot/\{spotId\}/comments/\{commentId\}}
{Pobierz pełną listę mediów powiązanych z komentarzem}
{%
    \begin{tabitemize}
        \item \textbf{spotId}: Long (ID spotu w ścieżce URL)
        \item \textbf{commentId}: Long (ID komentarza w ścieżce URL)
    \end{tabitemize}
}
{Brak}
{200 OK, 404 Not Found}
{%
    \begin{tabitemize}
        \item \textbf{List<SpotCommentMediaDto>}: lista mediów komentarza, każdy element:
        \begin{itemize}
            \item \textbf{id}: Long (ID pliku)
            \item \textbf{url}: String (URL pliku)
            \item \textbf{genericMediaType}: GenericMediaType (typ pliku, \texttt{PHOTO}, \texttt{VIDEO})
        \end{itemize}
    \end{tabitemize}
}

\cardendpoint
{EP17}
{/spot/\{spotId\}/comments}
{Dodaj nowy komentarz do wskazanego spotu}
{%
    \begin{tabitemize}
        \item \textbf{spotId}: Long (ID spotu w ścieżce URL)
        \item \textbf{body}: SpotCommentAddDto (JSON z danymi nowego komentarza: tekst, ocena, media itp.)
    \end{tabitemize}
}
{Brak}
{201 Created, 404 Not Found, 401 Unauthorized, 422 Unprocessable Entity}
{Brak (pusta odpowiedź)}

\cardendpoint
{EP18}
{/spot/comments/\{commentId\}/vote}
{Oddaj głos na komentarz (góra/dół)}
{%
    \begin{tabitemize}
        \item \textbf{commentId}: Long (ID komentarza w ścieżce URL)
    \end{tabitemize}
}
{%
    \begin{tabitemize}
        \item \textbf{isUpvote}: boolean (true = głos w górę, false = głos w dół)
    \end{tabitemize}
}
{200 OK, 401 Unauthorized, 404 Not Found, 409 Conflict, 403 Forbidden}
{Brak (pusta odpowiedź)}

\cardendpoint
{EP19}
{/spot/comments/vote-type}
{Pobierz informację, jak bieżący użytkownik zagłosował na komentarz}
{Brak}
{%
    \begin{tabitemize}
        \item \textbf{commentId}: Long (ID komentarza)
    \end{tabitemize}
}
{200 OK, 404 Not Found, 401 Unauthorized}
{%
    \begin{tabitemize}
        \item \textbf{voteInfo}: SpotCommentVoteType (typ oddanego głosu, \texttt{UPVOTE}, \texttt{DOWNVOTE}, \texttt{NONE})
    \end{tabitemize}
}
