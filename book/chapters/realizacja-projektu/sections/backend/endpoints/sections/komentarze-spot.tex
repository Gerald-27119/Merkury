%! Author = Mateusz
%! Date = 30/11/2025


\noindent\textbf{Komentarze do spotów}

\cardendpoint
{EP15}
{/public/spot/\{spotId\}/comments}
{Pobierz stronicowaną listę komentarzy dla wskazanego spotu}
{%
    \begin{tabitemize}
        \item \textbf{spotId}: Long (identyfikator spota w ścieżce URL)
    \end{tabitemize}
}
{%
    \begin{tabitemize}
        \item \textbf{page}: Integer (numer strony, domyślnie 0; rozmiar strony = 2)
    \end{tabitemize}
}
{200 OK, 404 Not Found, 401 Unauthorized}
{%
    \begin{tabitemize}
        \item \textbf{Page<SpotCommentDto>}: stronicowana lista komentarzy dla danego spota, każdy element zawiera:
        \begin{itemize}
            \item \textbf{id}: Long (identyfikator komentarza)
            \item \textbf{author}: SpotCommentAuthorDto (dane autora komentarza)
            \item \textbf{text}: String (treść komentarza)
            \item \textbf{rating}: Double (ocena spota wystawiona w komentarzu, 0--5)
            \item \textbf{upvotes}: Integer (liczba głosów pozytywnych na komentarz)
            \item \textbf{downvotes}: Integer (liczba głosów negatywnych na komentarz)
            \item \textbf{publishDate}: LocalDateTime (data i godzina publikacji komentarza)
            \item \textbf{isUpVoted}: Boolean (czy bieżący użytkownik oddał głos w górę na ten komentarz)
            \item \textbf{isDownVoted}: Boolean (czy bieżący użytkownik oddał głos w dół na ten komentarz)
            \item \textbf{numberOfMedia}: Integer (łączna liczba dołączonych plików multimedialnych)
            \item \textbf{mediaList}: List<SpotCommentMediaDto> (lista pierwszych plików komentarza)
        \end{itemize}
    \end{tabitemize}
}

\cardendpoint
{EP16}
{/public/spot/\{spotId\}/comments/\{commentId\}}
{Pobierz pełną listę mediów powiązanych z komentarzem}
{%
    \begin{tabitemize}
        \item \textbf{spotId}: Long (identyfikator spota w ścieżce URL)
        \item \textbf{commentId}: Long (identyfikator komentarza w ścieżce URL)
    \end{tabitemize}
}
{Brak}
{200 OK, 404 Not Found}
{%
    \begin{tabitemize}
        \item \textbf{List<SpotCommentMediaDto>}: pełna lista mediów powiązanych z komentarzem, każdy element:
        \begin{itemize}
            \item \textbf{id}: Long (identyfikator pliku multimedialnego)
            \item \textbf{url}: String (URL pliku, używany do pobrania/wyświetlenia)
            \item \textbf{genericMediaType}: GenericMediaType (typ pliku, \texttt{PHOTO} lub \texttt{VIDEO})
        \end{itemize}
    \end{tabitemize}
}

\cardendpoint
{EP17}
{/spot/\{spotId\}/comments}
{Dodaj nowy komentarz do wskazanego spotu}
{%
    \begin{tabitemize}
        \item \textbf{spotId}: Long (identyfikator spota w ścieżce URL)
        \item \textbf{body}: SpotCommentAddDto (dane nowego komentarza), zawiera:
        \begin{itemize}
            \item \textbf{text}: String (treść komentarza)
            \item \textbf{rating}: Double (ocena spota w komentarzu, zakres 0--5)
            \item \textbf{mediaFiles}: List<MultipartFile> (lista załączonych plików, zdjęcia/filmy)
        \end{itemize}
    \end{tabitemize}
}
{Brak}
{201 Created, 404 Not Found, 401 Unauthorized, 422 Unprocessable Entity}
{Brak (pusta odpowiedź)}

\cardendpoint
{EP18}
{/spot/comments/\{commentId\}/vote}
{Oddaj głos na komentarz (góra/dół)}
{%
    \begin{tabitemize}
        \item \textbf{commentId}: Long (identyfikator komentarza w ścieżce URL)
    \end{tabitemize}
}
{%
    \begin{tabitemize}
        \item \textbf{isUpvote}: boolean (true = głos w górę, false = głos w dół)
    \end{tabitemize}
}
{200 OK, 401 Unauthorized, 404 Not Found, 409 Conflict, 403 Forbidden}
{Brak (pusta odpowiedź)}

\cardendpoint
{EP19}
{/spot/comments/vote-type}
{Pobierz informację, jak bieżący użytkownik zagłosował na komentarz}
{Brak}
{%
    \begin{tabitemize}
        \item \textbf{commentId}: Long (identyfikator komentarza)
    \end{tabitemize}
}
{200 OK, 404 Not Found, 401 Unauthorized}
{%
    \begin{tabitemize}
        \item \textbf{voteInfo}: SpotCommentVoteType (typ oddanego głosu, \texttt{UPVOTE}, \texttt{DOWNVOTE}, \texttt{NONE})
    \end{tabitemize}
}
