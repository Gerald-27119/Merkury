%! Author = Mateusz
%! Date = 30/11/2025

\noindent\textbf{GIF-y (Tenor) – integracja czatu}

\cardendpoint
{EP35}
{/gifs/trending}
{Pobierz listę trendujących kategorii GIF-ów z Tenor}
{Brak}
{Brak}
{200 OK, 500 Internal Server Error}
{%
    \begin{tabitemize}
        \item \textbf{items}: List<TenorGifCategoryDto>, każdy element zawiera:
        \begin{itemize}
            \item \textbf{searchTerm}: String (fraza wyszukiwania powiązana z kategorią)
            \item \textbf{path}: String (ścieżka kategorii w Tenor)
            \item \textbf{gifUrl}: String (URL GIF-a)
        \end{itemize}
    \end{tabitemize}
}

\cardendpoint
{EP36}
{/gifs/search}
{Wyszukaj GIF-y po frazie tekstowej}
{Brak}
{%
    \begin{tabitemize}
        \item \textbf{searchPhrase}: String (fraza wyszukiwania)
        \item \textbf{next}: String (token paginacji zwrócony z poprzedniego wywołania; dla pierwszego zapytania może być pusty)
    \end{tabitemize}
}
{200 OK, 500 Internal Server Error}
{%
    \begin{tabitemize}
        \item \textbf{body}: TenorGifSearchWrapperDto (wyniki wyszukiwania GIF-ów), zawiera:
        \begin{itemize}
            \item \textbf{gifs}: List<TenorGifSearchDto> (lista pasujących GIF-ów), każdy element:
            \begin{itemize}
                \item \textbf{url}: String (URL GIF-a)
            \end{itemize}
            \item \textbf{next}: String (token do pobrania kolejnej strony wyników)
        \end{itemize}
    \end{tabitemize}
}

\noindent\textbf{Czat – REST API}

\cardendpoint
{EP37}
{/chats/\{chatId\}/messages}
{Pobierz stronicowane wiadomości dla wybranego czatu}
{%
    \begin{tabitemize}
        \item \textbf{chatId}: Long (identyfikator czatu)
    \end{tabitemize}
}
{%
    \begin{tabitemize}
        \item \textbf{pageParam}: Integer (numer strony wiadomości, domyślnie 1 – pierwsza strona po wstępnym pobraniu)
        \item \textbf{numberOfMessagesPerPage}: Integer (liczba wiadomości na stronę, domyślnie 20)
    \end{tabitemize}
}
{200 OK, 401 Unauthorized, 404 Not Found}
{%
    \begin{tabitemize}
        \item \textbf{body}: ChatMessageDtoSlice, zawiera:
        \begin{itemize}
            \item \textbf{messages}: List<ChatMessageDto>, każdy element:
            \begin{itemize}
                \item \textbf{id}: Long (identyfikator wiadomości)
                \item \textbf{sender}: ChatMessageSenderDto (dane nadawcy wiadomości)
                \item \textbf{sentAt}: LocalDateTime (data i godzina wysłania wiadomości)
                \item \textbf{content}: String (treść wiadomości; dla wiadomości plikowych może być pusty)
                \item \textbf{chatId}: Long (identyfikator czatu, do którego należy wiadomość)
                \item \textbf{attachedFiles}: \newline List<ChatMessageAttachedFileDto> (lista załączonych plików)
            \end{itemize}
            \item \textbf{hasNextSlice}: Boolean (czy istnieje kolejna „strona” / porcja wiadomości)
            \item \textbf{numberOfMessages}: Integer (liczba wiadomości zwróconych w tej odpowiedzi)
            \item \textbf{sliceNumber}: Integer (numer bieżącej porcji wiadomości)
        \end{itemize}
    \end{tabitemize}
}

\cardendpoint
{EP38}
{/chats/get-or-create-private-chat}
{Pobierz istniejący lub utwórz nowy prywatny czat z użytkownikiem}
{Brak}
{%
    \begin{tabitemize}
        \item \textbf{receiverUsername}: String (nazwa użytkownika, z którym chcemy rozpocząć lub kontynuować rozmowę)
        \item \textbf{chatId}: Long (opcjonalnie, identyfikator istniejącego czatu – jeśli jest już znany)
    \end{tabitemize}
}
{200 OK, 401 Unauthorized, 404 Not Found}
{%
    \begin{tabitemize}
        \item \textbf{body}: ChatDto (szczegóły czatu), zawiera:
        \begin{itemize}
            \item \textbf{id}: Long (identyfikator czatu)
            \item \textbf{name}: String (nazwa czatu – nazwa grupy lub nazwa rozmówcy)
            \item \textbf{lastMessage}: ChatMessageDto (ostatnia wiadomość w czacie, jeśli istnieje)
            \item \textbf{imgUrl}: String (URL avatara czatu lub rozmówcy)
            \item \textbf{messages}: List<ChatMessageDto> (lista wiadomości zwróconych razem z czatem)
            \item \textbf{chatType}: ChatType (typ czatu: \texttt{PRIVATE} lub \texttt{GROUP})
            \item \textbf{participants}: List<ChatParticipantDto> (lista uczestników czatu)
        \end{itemize}
    \end{tabitemize}
}

\cardendpoint
{EP39}
{/chats/\{chatId\}/send-files}
{Wyślij jeden lub wiele plików w ramach czatu}
{%
    \begin{tabitemize}
        \item \textbf{chatId}: Long (identyfikator czatu w ścieżce)
        \item \textbf{media}: List<MultipartFile> (lista załączanych plików do wysłania w wiadomości)
    \end{tabitemize}
}
{Brak}
{201 Created, 401 Unauthorized, 404 Not Found, 413 Payload Too Large, 415 Unsupported Media Type, 500 Internal Server Error}
{%
    \begin{tabitemize}
        \item Brak (pusta odpowiedź; wiadomości z plikami pojawią się w historii czatu)
    \end{tabitemize}
}

\cardendpoint
{EP40}
{/chats/create/group}
{Utwórz nowy czat grupowy}
{%
    \begin{tabitemize}
        \item \textbf{body}: CreateGroupChatDto (dane nowego czatu grupowego), zawiera:
        \begin{itemize}
            \item \textbf{usermes}: List<String> (lista nazw użytkowników, którzy mają zostać uczestnikami czatu)
            \item \textbf{ownerUsername}: String (nazwa właściciela / twórcy czatu)
        \end{itemize}
    \end{tabitemize}
}
{Brak}
{201 Created, 400 Bad Request, 401 Unauthorized}
{%
    \begin{tabitemize}
        \item \textbf{body}: ChatDto (utworzony czat grupowy), zawiera:
        \begin{itemize}
            \item \textbf{id}: Long (identyfikator czatu)
            \item \textbf{name}: String (nazwa czatu – nazwa grupy)
            \item \textbf{lastMessage}: ChatMessageDto (ostatnia wiadomość w czacie, jeśli istnieje)
            \item \textbf{imgUrl}: String (URL avatara czatu)
            \item \textbf{messages}: List<ChatMessageDto> (lista wiadomości zwróconych razem z czatem)
            \item \textbf{chatType}: ChatType (typ czatu, \texttt{PRIVATE} lub \texttt{GROUP})
            \item \textbf{participants}: List<ChatParticipantDto> (lista uczestników czatu)
        \end{itemize}
    \end{tabitemize}
}

\cardendpoint
{EP41}
{/chats/\{chatId\}}
{Zaktualizuj dane czatu grupowego (nazwa, zdjęcie)}
{%
    \begin{tabitemize}
        \item \textbf{chatId}: Long (identyfikator czatu grupowego)
        \item \textbf{updateGroupChatDto}: UpdateGroupChatDto (wysyłany jako \texttt{multipart/form-data}, zawiera dane do zmiany, nowa nazwa, nowe zdjęcie)
    \end{tabitemize}
}
{Brak}
{200 OK, 401 Unauthorized, 404 Not Found, 413 Payload Too Large, 415 Unsupported Media Type, 500 Internal Server Error}
{%
    \begin{tabitemize}
        \item \textbf{body}: UpdatedGroupChatDto (zaktualizowane dane czatu grupowego), zawiera:
        \begin{itemize}
            \item \textbf{newName}: String (aktualna nazwa czatu po zmianie)
            \item \textbf{newImgUrl}: String (aktualny URL obrazka grupy po zmianie)
        \end{itemize}
    \end{tabitemize}
}

\cardendpoint
{EP42}
{/chats/group-chat/add/search/\{chatId\}}
{Wyszukaj potencjalnych użytkowników do dodania do czatu grupowego}
{%
    \begin{tabitemize}
        \item \textbf{chatId}: Long (identyfikator czatu grupowego, do którego chcemy dodać użytkowników)
    \end{tabitemize}
}
{%
    \begin{tabitemize}
        \item \textbf{query}: String (fraza wyszukiwania po nazwie użytkownika)
        \item \textbf{page}: Integer (numer strony wyników, domyślnie 0)
        \item \textbf{size}: Integer (liczba wyników na stronę, domyślnie 20)
    \end{tabitemize}
}
{200 OK, 401 Unauthorized, 404 Not Found}
{%
    \begin{tabitemize}
        \item \textbf{body}: SimpleSliceDto<PotentialChatMemberDto> (stronicowana lista potencjalnych uczestników czatu), zawiera:
        \begin{itemize}
            \item \textbf{hasNext}: boolean (czy istnieje kolejna „strona” wyników)
            \item \textbf{collection}: Collection<PotentialChatMemberDto> (kolekcja potencjalnych użytkowników), każdy element:
            \begin{itemize}
                \item \textbf{username}: String (nazwa użytkownika)
                \item \textbf{profileImg}: String (URL zdjęcia profilowego użytkownika)
            \end{itemize}
        \end{itemize}
    \end{tabitemize}
}
