%! Author = Mateusz
%! Date = 30/11/2025

\subsubsection{Czat}
\label{subsubsec:chat}

\cardendpoint
{EP08}
{/chats/\{chatId\}/messages}
{Pobierz stronicowane wiadomości dla wybranego czatu}
{%
    \begin{tabitemize}
        \item \textbf{chatId}: Long (identyfikator czatu)
    \end{tabitemize}
}
{%
    \begin{tabitemize}
        \item \textbf{pageParam}: Integer (numer strony wiadomości, domyślnie 1 – pierwsza strona po wstępnym pobraniu)
        \item \textbf{numberOfMessagesPerPage}: Integer (liczba wiadomości na stronę, domyślnie 20)
    \end{tabitemize}
}
{200 OK, 401 Unauthorized, 404 Not Found}
{%
    \begin{tabitemize}
        \item \textbf{body}: ChatMessageDtoSlice, zawiera:
        \begin{itemize}
            \item \textbf{messages}: List<ChatMessageDto>, każdy element:
            \begin{itemize}
                \item \textbf{id}: Long (identyfikator wiadomości)
                \item \textbf{sender}: ChatMessageSenderDto (dane nadawcy wiadomości)
                \item \textbf{sentAt}: LocalDateTime (data i godzina wysłania wiadomości)
                \item \textbf{content}: String (treść wiadomości; dla wiadomości plikowych może być pusty)
                \item \textbf{chatId}: Long (identyfikator czatu, do którego należy wiadomość)
                \item \textbf{attachedFiles}: \newline List<ChatMessageAttachedFileDto> (lista załączonych plików)
            \end{itemize}
            \item \textbf{hasNextSlice}: Boolean (czy istnieje kolejna „strona” / porcja wiadomości)
            \item \textbf{numberOfMessages}: Integer (liczba wiadomości zwróconych w tej odpowiedzi)
            \item \textbf{sliceNumber}: Integer (numer bieżącej porcji wiadomości)
        \end{itemize}
    \end{tabitemize}
}
