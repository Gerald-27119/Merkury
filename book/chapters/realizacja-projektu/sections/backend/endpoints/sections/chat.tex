%! Author = Mateusz
%! Date = 30/11/2025

\noindent\textbf{GIF-y (Tenor) – integracja czatu}

\cardendpoint
{EP81}
{/gifs/trending}
{Pobierz listę trendujących kategorii GIF-ów z Tenor}
{Brak}
{Brak}
{200 OK, 500 Internal Server Error}
{%
    \begin{tabitemize}
        \item \textbf{items}: List<TenorGifCategoryDto>, każdy element zawiera:
        \begin{itemize}
            \item \textbf{searchTerm}: String (fraza wyszukiwania powiązana z kategorią)
            \item \textbf{path}: String (ścieżka/slug kategorii w Tenor)
            \item \textbf{gifUrl}: String (URL przykładowego GIF-a z kategorii)
        \end{itemize}
    \end{tabitemize}
}

\cardendpoint
{EP82}
{/gifs/search}
{Wyszukaj GIF-y po frazie tekstowej}
{Brak}
{%
    \begin{tabitemize}
        \item \textbf{searchPhrase}: String (fraza wyszukiwania, np. \texttt{surf}, \texttt{windy})
        \item \textbf{next}: String (token paginacji zwrócony z poprzedniego wywołania; dla pierwszego zapytania może być pusty)
    \end{tabitemize}
}
{200 OK, 500 Internal Server Error}
{%
    \begin{tabitemize}
        \item \textbf{body}: TenorGifSearchWrapperDto, zawiera:
        \begin{itemize}
            \item \textbf{gifs}: List<TenorGifSearchDto>, każdy element:
            \begin{itemize}
                \item \textbf{url}: String (URL GIF-a)
            \end{itemize}
            \item \textbf{next}: String (token do pobrania kolejnej strony wyników)
        \end{itemize}
    \end{tabitemize}
}

\noindent\textbf{Czat – komunikacja w czasie rzeczywistym (STOMP)}

\cardendpoint
{EP83}
{/app/send/\{chatId\}/message}
{Wyślij wiadomość czatu (WebSocket/STOMP)}
{%
    \begin{tabitemize}
        \item \textbf{chatId}: String (identyfikator czatu w ścieżce STOMP)
        \item \textbf{body}: IncomingChatMessageDto (JSON z danymi wiadomości, m.in. treść, nadawca, UUID optymistycznej wiadomości)
    \end{tabitemize}
}
{Brak}
{Brak klasycznego kodu HTTP (komunikacja STOMP)}
{%
    \begin{tabitemize}
        \item Brak bezpośredniej odpowiedzi – wiadomość jest:
        \begin{itemize}
            \item zapisywana przez \textbf{ChatService}
            \item rozsyłana do wszystkich uczestników czatu
            \item do nadawcy wysyłana jest wersja ACK z potwierdzeniem (z wykorzystaniem \textbf{optimisticMessageUUID})
        \end{itemize}
    \end{tabitemize}
}

\noindent\textbf{Czat – REST API}

\cardendpoint
{EP84}
{/chats/\{chatId\}/messages}
{Pobierz stronicowane wiadomości dla wybranego czatu}
{%
    \begin{tabitemize}
        \item \textbf{chatId}: Long (identyfikator czatu)
    \end{tabitemize}
}
{%
    \begin{tabitemize}
        \item \textbf{pageParam}: Integer (numer strony wiadomości, domyślnie 1 – pierwsza strona po wstępnym pobraniu)
        \item \textbf{numberOfMessagesPerPage}: Integer (liczba wiadomości na stronę, domyślnie 20)
    \end{tabitemize}
}
{200 OK, 401 Unauthorized, 404 Not Found}
{%
    \begin{tabitemize}
        \item \textbf{body}: ChatMessageDtoSlice, zawiera:
        \begin{itemize}
            \item \textbf{messages}: List<ChatMessageDto>, każdy element m.in.:
            \begin{itemize}
                \item \textbf{id}: Long
                \item \textbf{sender}: ChatMessageSenderDto
                \item \textbf{sentAt}: LocalDateTime
                \item \textbf{content}: String (treść wiadomości; dla wiadomości plikowych może być pusty)
                \item \textbf{chatId}: Long
                \item \textbf{attachedFiles}: List<ChatMessageAttachedFileDto>
            \end{itemize}
            \item \textbf{hasNextSlice}: Boolean (czy istnieje kolejna „strona” wiadomości)
            \item \textbf{numberOfMessages}: Integer (liczba wiadomości w tej odpowiedzi)
            \item \textbf{sliceNumber}: Integer (numer bieżącej „porcji”)
        \end{itemize}
    \end{tabitemize}
}

\cardendpoint
{EP85}
{/chats/user-chats}
{Pobierz listę czatów użytkownika z ostatnimi wiadomościami}
{Brak}
{%
    \begin{tabitemize}
        \item \textbf{pageNumber}: Integer (numer strony listy czatów, domyślnie 0)
        \item \textbf{numberOfChatsPerPage}: Integer (liczba czatów na stronę, domyślnie 10)
    \end{tabitemize}
}
{200 OK, 401 Unauthorized}
{%
    \begin{tabitemize}
        \item \textbf{items}: List<ChatDto>, każdy element:
        \begin{itemize}
            \item \textbf{id}: Long
            \item \textbf{name}: String (nazwa czatu; dla prywatnych może być nazwą drugiego użytkownika)
            \item \textbf{lastMessage}: ChatMessageDto (ostatnia wiadomość na czacie)
            \item \textbf{imgUrl}: String (URL avatara czatu / rozmówcy)
            \item \textbf{messages}: List<ChatMessageDto> (ostatnie wiadomości, do wyświetlenia szczegółów)
            \item \textbf{chatType}: ChatType (np. PRIVATE, GROUP)
            \item \textbf{participants}: List<ChatParticipantDto> (uczestnicy czatu)
        \end{itemize}
    \end{tabitemize}
}

\cardendpoint
{EP86}
{/chats/get-or-create-private-chat}
{Pobierz istniejący lub utwórz nowy prywatny czat z użytkownikiem}
{Brak}
{%
    \begin{tabitemize}
        \item \textbf{receiverUsername}: String (nazwa użytkownika, z którym chcemy rozmawiać)
        \item \textbf{chatId}: Long (opcjonalnie, ID istniejącego czatu – jeśli znany)
    \end{tabitemize}
}
{200 OK, 401 Unauthorized, 404 Not Found}
{%
    \begin{tabitemize}
        \item \textbf{body}: ChatDto (szczegóły czatu prywatnego – patrz EP85)
    \end{tabitemize}
}

\cardendpoint
{EP87}
{/chats/\{chatId\}/send-files}
{Wyślij jeden lub wiele plików w ramach czatu}
{%
    \begin{tabitemize}
        \item \textbf{chatId}: Long (identyfikator czatu w ścieżce)
        \item \textbf{media}: List<MultipartFile> (pole formularza \texttt{media} – lista załączanych plików)
    \end{tabitemize}
}
{Brak}
{201 Created, 401 Unauthorized, 404 Not Found, 413 Payload Too Large, 415 Unsupported Media Type, 500 Internal Server Error}
{%
    \begin{tabitemize}
        \item Brak (pusta odpowiedź; wiadomości z plikami pojawią się w historii czatu)
    \end{tabitemize}
}

\cardendpoint
{EP88}
{/chats/create/group}
{Utwórz nowy czat grupowy}
{%
    \begin{tabitemize}
        \item \textbf{body}: CreateGroupChatDto (JSON z danymi nowej grupy, np. nazwa, uczestnicy, opcjonalny avatar)
    \end{tabitemize}
}
{Brak}
{201 Created, 400 Bad Request, 401 Unauthorized}
{%
    \begin{tabitemize}
        \item \textbf{body}: ChatDto (nowo utworzony czat grupowy – patrz EP85)
    \end{tabitemize}
}

\cardendpoint
{EP89}
{/chats/\{chatId\}}
{Zaktualizuj dane czatu grupowego (np. nazwa, zdjęcie)}
{%
    \begin{tabitemize}
        \item \textbf{chatId}: Long (identyfikator czatu grupowego)
        \item \textbf{updateGroupChatDto}: UpdateGroupChatDto (wysyłany jako \texttt{multipart/form-data}, zawiera dane do zmiany, np. nowa nazwa, nowe zdjęcie)
    \end{tabitemize}
}
{Brak}
{200 OK, 401 Unauthorized, 404 Not Found, 413 Payload Too Large, 415 Unsupported Media Type, 500 Internal Server Error}
{%
    \begin{tabitemize}
        \item \textbf{body}: UpdatedGroupChatDto, zawiera:
        \begin{itemize}
            \item \textbf{newName}: String (aktualna nazwa czatu)
            \item \textbf{newImgUrl}: String (aktualny URL obrazka grupy)
        \end{itemize}
    \end{tabitemize}
}

\cardendpoint
{EP90}
{/chats/group-chat/add/search/\{chatId\}}
{Wyszukaj potencjalnych użytkowników do dodania do czatu grupowego}
{%
    \begin{tabitemize}
        \item \textbf{chatId}: Long (identyfikator czatu grupowego)
    \end{tabitemize}
}
{%
    \begin{tabitemize}
        \item \textbf{query}: String (fraza wyszukiwania po nazwie użytkownika)
        \item \textbf{page}: Integer (numer strony, domyślnie 0)
        \item \textbf{size}: Integer (liczba wyników na stronę, domyślnie 20)
    \end{tabitemize}
}
{200 OK, 401 Unauthorized, 404 Not Found}
{%
    \begin{tabitemize}
        \item \textbf{body}: SimpleSliceDto<PotentialChatMemberDto>, zawiera:
        \begin{itemize}
            \item \textbf{hasNext}: boolean (czy istnieje kolejna „strona” wyników)
            \item \textbf{collection}: Collection<PotentialChatMemberDto>, każdy element:
            \begin{itemize}
                \item \textbf{username}: String (nazwa użytkownika)
                \item \textbf{profileImg}: String (URL zdjęcia profilowego)
            \end{itemize}
        \end{itemize}
    \end{tabitemize}
}

\cardendpoint
{EP91}
{/chats/add/users/\{chatId\}}
{Dodaj użytkowników do istniejącego czatu grupowego}
{%
    \begin{tabitemize}
        \item \textbf{chatId}: Long (identyfikator czatu grupowego)
        \item \textbf{body}: AddUsersToExistingGroupChatDto (JSON, zawiera listę nazw użytkowników do dodania)
    \end{tabitemize}
}
{Brak}
{200 OK, 401 Unauthorized, 404 Not Found, 409 Conflict}
{%
    \begin{tabitemize}
        \item \textbf{body}: ChatDto (zaktualizowany czat grupowy z nową listą uczestników)
    \end{tabitemize}
}
