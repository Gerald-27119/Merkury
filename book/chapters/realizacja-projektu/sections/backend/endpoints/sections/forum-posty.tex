%! Author = Mateusz
%! Date = 30/11/2025

\subsubsection{Forum – posty}
\label{subsubsec:forum-posty}

\cardendpoint
{EP05}
{/public/post/\{postId\}}
{Pobierz szczegółowe informacje o poście}
{%
    \begin{tabitemize}
        \item \textbf{postId}: Long (identyfikator posta w ścieżce URL)
    \end{tabitemize}
}
{Brak}
{200 OK, 404 Not Found}
{%
    \begin{tabitemize}
        \item \textbf{PostDetailsDto}, zawiera:
        \begin{itemize}
            \item \textbf{id}: Long (identyfikator posta)
            \item \textbf{title}: String (tytuł posta)
            \item \textbf{content}: String (pełna treść posta)
            \item \textbf{category}: ForumCategoryDto (kategoria forum, do której należy post)
            \item \textbf{tags}: List<ForumTagDto> (lista tagów przypisanych do posta)
            \item \textbf{author}: AuthorDto (dane autora posta)
            \item \textbf{isAuthor}: Boolean (czy bieżący użytkownik jest autorem posta)
            \item \textbf{isFollowed}: Boolean (czy bieżący użytkownik obserwuje ten post)
            \item \textbf{publishDate}: LocalDateTime (data i godzina publikacji posta)
            \item \textbf{views}: Integer (liczba wyświetleń posta)
            \item \textbf{upVotes}: Integer (liczba głosów w górę na post)
            \item \textbf{downVotes}: Integer (liczba głosów w dół na post)
            \item \textbf{isUpVoted}: Boolean (czy bieżący użytkownik oddał głos w górę na post)
            \item \textbf{isDownVoted}: Boolean (czy bieżący użytkownik oddał głos w dół na post)
            \item \textbf{commentsCount}: Integer (łączna liczba komentarzy pod postem)
        \end{itemize}
    \end{tabitemize}
}
