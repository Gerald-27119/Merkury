%! Author = Mateusz
%! Date = 30/11/2025

\subsubsection{Spoty i pogoda}
\label{subsubsec:spoty-i-pogoda}

\cardendpoint
{EP02}
{/public/spot/current-view}
{Pobierz listę spotów w aktualnym widoku mapy}
{Brak}
{%
    \begin{tabitemize}
        \item \textbf{swLng}: double (długość geograficzna lewego dolnego rogu)
        \item \textbf{swLat}: double (szerokość geograficzna lewego dolnego rogu)
        \item \textbf{neLng}: double (długość geograficzna prawego górnego rogu)
        \item \textbf{neLat}: double (szerokość geograficzna prawego górnego rogu)
        \item \textbf{name}: String (fragment nazwy spotu, domyślnie pusty)
        \item \textbf{sorting}: String (tryb sortowania, domyślnie \texttt{none})
        \item \textbf{ratingFrom}: double (minimalna ocena, domyślnie 0.0)
        \item \textbf{page}: Integer (numer strony, domyślnie 0)
    \end{tabitemize}
}
{200 OK}
{%
    \begin{tabitemize}
        \item \textbf{Page<SearchSpotDto>}: stronicowana lista spotów, gdzie każdy element zawiera:
        \begin{itemize}
            \item \textbf{id}: Long (identyfikator spota)
            \item \textbf{name}: String (nazwa spota)
            \item \textbf{rating}: Double (0--5)
            \item \textbf{ratingCount}: Integer (liczba ocen)
            \item \textbf{firstPhoto}: String (URL pierwszego zdjęcia)
            \item \textbf{tags}: Set<SpotTagDto> (tagi spota)
            \item \textbf{centerPoint}: BorderPoint (środek obszaru spota)
        \end{itemize}
    \end{tabitemize}
}
