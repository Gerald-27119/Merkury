%! Author = Mateusz
%! Date = 30/11/2025

\noindent\textbf{Spoty, wyszukiwarka i pogoda}

\cardendpoint
{EP28}
{/public/spot/gallery}
{Pobierz stronę galerii mediów dla spotu}
{Brak}
{%
    \begin{tabitemize}
        \item \textbf{spotId}: Long (identyfikator spotu)
        \item \textbf{mediaType}: String (typ mediów, wartość enum \texttt{GenericMediaType}, np. \texttt{PHOTO}, \texttt{VIDEO})
        \item \textbf{sorting}: String (kryterium sortowania, np. po dacie / popularności)
        \item \textbf{page}: Integer (numer strony, domyślnie 0)
        \item \textbf{size}: Integer (rozmiar strony, domyślnie 6)
    \end{tabitemize}
}
{200 OK, 404 Not Found}
{%
    \begin{tabitemize}
        \item \textbf{Page<SpotSidebarMediaGalleryDto>}: stronicowana lista elementów galerii, gdzie każdy element zawiera:
        \begin{itemize}
            \item \textbf{id}: Long (ID media)
            \item \textbf{url}: String (URL pliku)
            \item \textbf{mediaType}: GenericMediaType (typ media)
        \end{itemize}
    \end{tabitemize}
}

\cardendpoint
{EP29}
{/public/spot/gallery-media-position}
{Pobierz numer strony galerii, na której znajduje się dane medium}
{Brak}
{%
    \begin{tabitemize}
        \item \textbf{spotId}: Long (ID spotu)
        \item \textbf{mediaId}: Long (ID media)
        \item \textbf{mediaType}: String (typ media, \texttt{GenericMediaType})
        \item \textbf{sorting}: String (kryterium sortowania używane w galerii)
        \item \textbf{pageSize}: Integer (rozmiar strony, domyślnie 6)
    \end{tabitemize}
}
{200 OK, 404 Not Found}
{%
    \begin{tabitemize}
        \item \textbf{mediaPagePosition}: Integer (numer strony, na której znajduje się medium)
    \end{tabitemize}
}

\cardendpoint
{EP30}
{/public/spot/gallery-fullscreen-media}
{Pobierz dane media do wyświetlenia w trybie pełnoekranowym}
{Brak}
{%
    \begin{tabitemize}
        \item \textbf{spotId}: Long (ID spotu)
        \item \textbf{mediaId}: Long (ID media)
        \item \textbf{mediaType}: String (typ media, \texttt{GenericMediaType})
    \end{tabitemize}
}
{200 OK, 404 Not Found}
{%
    \begin{tabitemize}
        \item \textbf{id}: Long (ID media)
        \item \textbf{url}: String (URL pliku)
        \item \textbf{mediaType}: GenericMediaType
        \item \textbf{likesNumber}: Integer (liczba polubień)
        \item \textbf{publishDate}: LocalDate (data publikacji)
        \item \textbf{authorName}: String (nazwa autora)
        \item \textbf{authorProfilePhotoUrl}: String (URL zdjęcia profilowego autora)
    \end{tabitemize}
}

\cardendpoint
{EP31}
{/public/spot/current-view}
{Pobierz listę spotów w aktualnym widoku mapy}
{Brak}
{%
    \begin{tabitemize}
        \item \textbf{swLng}: double (długość geograficzna lewego dolnego rogu)
        \item \textbf{swLat}: double (szerokość geograficzna lewego dolnego rogu)
        \item \textbf{neLng}: double (długość geograficzna prawego górnego rogu)
        \item \textbf{neLat}: double (szerokość geograficzna prawego górnego rogu)
        \item \textbf{name}: String (fragment nazwy spotu, domyślnie pusty)
        \item \textbf{sorting}: String (tryb sortowania, domyślnie \texttt{none})
        \item \textbf{ratingFrom}: double (minimalna ocena, domyślnie 0.0)
        \item \textbf{page}: Integer (numer strony, domyślnie 0)
    \end{tabitemize}
}
{200 OK}
{%
    \begin{tabitemize}
        \item \textbf{Page<SearchSpotDto>}: stronicowana lista spotów, gdzie każdy element zawiera:
        \begin{itemize}
            \item \textbf{id}: Long
            \item \textbf{name}: String
            \item \textbf{rating}: Double (0--5)
            \item \textbf{ratingCount}: Integer (liczba ocen)
            \item \textbf{firstPhoto}: String (URL pierwszego zdjęcia)
            \item \textbf{tags}: Set<SpotTagDto> (tagi spotu)
            \item \textbf{centerPoint}: BorderPoint (środek obszaru spotu)
        \end{itemize}
    \end{tabitemize}
}

\cardendpoint
{EP32}
{/public/spot/current-view/spot-names}
{Pobierz nazwy spotów w aktualnym widoku mapy}
{Brak}
{%
    \begin{tabitemize}
        \item \textbf{swLng}: double
        \item \textbf{swLat}: double
        \item \textbf{neLng}: double
        \item \textbf{neLat}: double
        \item \textbf{name}: String (filtr po nazwie, domyślnie pusty)
    \end{tabitemize}
}
{200 OK}
{%
    \begin{tabitemize}
        \item \textbf{List<String>}: lista nazw spotów znajdujących się w widoku mapy
    \end{tabitemize}
}

\cardendpoint
{EP33}
{/public/spot/\{spotId\}}
{Pobierz szczegóły spotu na podstawie ID}
{%
    \begin{tabitemize}
        \item \textbf{spotId}: Long (ID spotu w ścieżce URL)
    \end{tabitemize}
}
{Brak}
{200 OK, 404 Not Found}
{%
    \begin{tabitemize}
        \item \textbf{id}: Long
        \item \textbf{name}: String
        \item \textbf{country}: String
        \item \textbf{city}: String
        \item \textbf{street}: String
        \item \textbf{description}: String
        \item \textbf{rating}: Double (0--5)
        \item \textbf{ratingCount}: Integer
        \item \textbf{media}: List<SpotMediaDto> (lista mediów powiązanych ze spotem)
        \item \textbf{centerPoint}: SpotCoordinatesDto (koordynaty środka)
        \item \textbf{tags}: Set<SpotTagDto> (tagi spotu)
    \end{tabitemize}
}

\cardendpoint
{EP34}
{/public/spot/increase-view-count}
{Zwiększ licznik wyświetleń spotu}
{Brak}
{%
    \begin{tabitemize}
        \item \textbf{spotId}: Long (ID spotu)
    \end{tabitemize}
}
{200 OK, 404 Not Found}
{Brak (pusta odpowiedź)}

\cardendpoint
{EP35}
{/public/spot/search/map}
{Wyszukaj spoty na mapie (prosta wyszukiwarka)}
{Brak}
{%
    \begin{tabitemize}
        \item \textbf{name}: String (fragment nazwy, domyślnie pusty)
    \end{tabitemize}
}
{200 OK, 404 Not Found}
{%
    \begin{tabitemize}
        \item \textbf{items}: List<GeneralSpotDto>, każdy element zawiera:
        \begin{itemize}
            \item \textbf{id}: Long
            \item \textbf{areaColor}: String (kolor obszaru na mapie)
            \item \textbf{name}: String
            \item \textbf{region}: String
            \item \textbf{city}: String
            \item \textbf{rating}: Double (0--5)
            \item \textbf{contourCoordinates}: List<Double[]> (lista punktów konturu)
            \item \textbf{centerPoint}: BorderPoint
            \item \textbf{area}: Double (powierzchnia obszaru)
        \end{itemize}
    \end{tabitemize}
}

\cardendpoint
{EP36}
{/public/spot/search/list}
{Wyszukaj spoty i zwróć stronę wyników w formie listy}
{Brak}
{%
    \begin{tabitemize}
        \item \textbf{name}: String (fragment nazwy, domyślnie pusty)
        \item \textbf{sorting}: String (tryb sortowania, domyślnie \texttt{none})
        \item \textbf{page}: Integer (numer strony, domyślnie 0)
    \end{tabitemize}
}
{200 OK}
{%
    \begin{tabitemize}
        \item \textbf{Page<SearchSpotDto>}: stronicowana lista spotów (jak w EP31)
    \end{tabitemize}
}

\cardendpoint
{EP37}
{/public/spot/names}
{Pobierz listę nazw spotów pasujących do podanego tekstu}
{Brak}
{%
    \begin{tabitemize}
        \item \textbf{text}: String (fragment nazwy, domyślnie pusty)
    \end{tabitemize}
}
{200 OK, 404 Not Found}
{%
    \begin{tabitemize}
        \item \textbf{List<String>}: lista pasujących nazw spotów
    \end{tabitemize}
}

\cardendpoint
{EP38}
{/public/spot/most-popular}
{Pobierz 18 najpopularniejszych spotów}
{Brak}
{Brak}
{200 OK}
{%
    \begin{tabitemize}
        \item \textbf{items}: List<TopRatedSpotDto>, każdy element zawiera:
        \begin{itemize}
            \item \textbf{id}: Long
            \item \textbf{name}: String
            \item \textbf{city}: String
            \item \textbf{imageUrl}: String (URL głównego zdjęcia)
        \end{itemize}
    \end{tabitemize}
}

\cardendpoint
{EP39}
{/public/spot/search/home-page}
{Wyszukaj spoty na stronie głównej na podstawie lokalizacji}
{Brak}
{%
    \begin{tabitemize}
        \item \textbf{country}: String (opcjonalnie, kraj)
        \item \textbf{region}: String (opcjonalnie, region)
        \item \textbf{city}: String (opcjonalnie, miasto)
        \item \textbf{userLongitude}: Double (opcjonalnie, długość geograficzna użytkownika)
        \item \textbf{userLatitude}: Double (opcjonalnie, szerokość geograficzna użytkownika)
        \item \textbf{page}: Integer (numer strony, domyślnie 0)
        \item \textbf{size}: Integer (rozmiar strony, domyślnie 20)
    \end{tabitemize}
}
{200 OK}
{%
    \begin{tabitemize}
        \item \textbf{items}: List<HomePageSpotDto>, każdy element:
        \begin{itemize}
            \item \textbf{id}: Long
            \item \textbf{name}: String
            \item \textbf{rating}: Double
            \item \textbf{ratingCount}: Integer
            \item \textbf{firstPhoto}: String (URL zdjęcia)
            \item \textbf{tags}: Set<SpotTagDto>
            \item \textbf{centerPoint}: BorderPoint
            \item \textbf{city}: String
            \item \textbf{distanceToUser}: Double (odległość od użytkownika, jeśli dostępna)
        \end{itemize}
        \item \textbf{hasNext}: boolean (czy istnieje kolejna strona wyników)
    \end{tabitemize}
}

\cardendpoint
{EP40}
{/public/spot/search/home-page/locations}
{Pobierz listę podpowiedzi lokalizacji dla wyszukiwarki na stronie głównej}
{Brak}
{%
    \begin{tabitemize}
        \item \textbf{query}: String (fraza wyszukiwania)
        \item \textbf{type}: String (typ lokalizacji, np. kraj/region/miasto)
    \end{tabitemize}
}
{200 OK}
{%
    \begin{tabitemize}
        \item \textbf{List<String>}: lista podpowiedzi (nazwy lokalizacji)
    \end{tabitemize}
}

\cardendpoint
{EP41}
{/public/spot/search/home-page/advance}
{Wyszukaj spoty na stronie głównej (zaawansowane filtrowanie)}
{Brak}
{%
    \begin{tabitemize}
        \item \textbf{city}: String (opcjonalnie, miasto)
        \item \textbf{tags}: List<String> (opcjonalnie, lista tagów)
        \item \textbf{userLongitude}: Double (opcjonalnie, długość geograficzna użytkownika)
        \item \textbf{userLatitude}: Double (opcjonalnie, szerokość geograficzna użytkownika)
        \item \textbf{sort}: SpotSortType (opcjonalnie, typ sortowania)
        \item \textbf{filter}: SpotRatingFilterType (opcjonalnie, filtr po ocenie)
        \item \textbf{page}: Integer (numer strony, domyślnie 0)
        \item \textbf{size}: Integer (rozmiar strony, domyślnie 20)
    \end{tabitemize}
}
{200 OK}
{%
    \begin{tabitemize}
        \item \textbf{HomePageSpotPageDto}: tak jak w EP39 (items + hasNext)
    \end{tabitemize}
}

\cardendpoint
{EP42}
{/public/spot/get-spot-basic-weather}
{Pobierz podstawowe informacje pogodowe dla wskazanej lokalizacji}
{Brak}
{%
    \begin{tabitemize}
        \item \textbf{latitude}: double (szerokość geograficzna)
        \item \textbf{longitude}: double (długość geograficzna)
    \end{tabitemize}
}
{200 OK}
{%
    \begin{tabitemize}
        \item \textbf{temperature}: Double (temperatura)
        \item \textbf{weatherCode}: int (kod warunków pogodowych)
        \item \textbf{windSpeed}: Double (prędkość wiatru)
        \item \textbf{isDay}: boolean (czy jest dzień)
    \end{tabitemize}
}

\cardendpoint
{EP43}
{/public/spot/get-spot-detailed-weather}
{Pobierz szczegółowe informacje pogodowe dla wskazanej lokalizacji}
{Brak}
{%
    \begin{tabitemize}
        \item \textbf{latitude}: double
        \item \textbf{longitude}: double
    \end{tabitemize}
}
{200 OK}
{%
    \begin{tabitemize}
        \item \textbf{temperature}: Double
        \item \textbf{weatherCode}: int
        \item \textbf{precipitationProbability}: Double (prawdopodobieństwo opadów)
        \item \textbf{dewPoint}: Double (punkt rosy)
        \item \textbf{relativeHumidity}: Double (wilgotność względna)
        \item \textbf{isDay}: boolean
        \item \textbf{uvIndexMax}: Double (maksymalny indeks UV)
    \end{tabitemize}
}

\cardendpoint
{EP44}
{/public/spot/get-spot-wind-speeds}
{Pobierz prędkości wiatru dla spotu na różnych wysokościach}
{Brak}
{%
    \begin{tabitemize}
        \item \textbf{latitude}: double
        \item \textbf{longitude}: double
        \item \textbf{spotId}: long (ID spotu)
    \end{tabitemize}
}
{200 OK}
{%
    \begin{tabitemize}
        \item \textbf{windSpeeds100m}: Double
        \item \textbf{windSpeeds200m}: Double
        \item \textbf{windSpeeds300m}: Double
        \item \textbf{windSpeeds500m}: Double
        \item \textbf{windSpeeds750m}: Double
        \item \textbf{windSpeeds1000m}: Double
    \end{tabitemize}
}

\cardendpoint
{EP45}
{/public/spot/get-spot-weather-timeline-plot-data}
{Pobierz dane pogodowe w czasie do wykresu dla spotu}
{Brak}
{%
    \begin{tabitemize}
        \item \textbf{latitude}: double
        \item \textbf{longitude}: double
        \item \textbf{spotId}: long
    \end{tabitemize}
}
{200 OK, 404 Not Found}
{%
    \begin{tabitemize}
        \item \textbf{List<SpotWeatherTimelinePlotDataDto>}: lista punktów na osi czasu, każdy element:
        \begin{itemize}
            \item \textbf{time}: String (znacznik czasu)
            \item \textbf{temperature}: Double
            \item \textbf{weatherCode}: Integer
            \item \textbf{precipitationProbability}: Double
            \item \textbf{isDay}: Boolean
            \item \textbf{timeZone}: String (strefa czasowa)
        \end{itemize}
    \end{tabitemize}
}

\cardendpoint
{EP46}
{/public/spot/increase-spot-media-views-count}
{Zwiększ licznik wyświetleń konkretnego media spotu}
{Brak}
{%
    \begin{tabitemize}
        \item \textbf{spotMediaId}: long (ID media)
    \end{tabitemize}
}
{200 OK, 404 Not Found}
{Brak (pusta odpowiedź)}

\cardendpoint
{EP47}
{/public/spot/edit-spot-media-likes}
{Przełącz polubienie media spotu przez aktualnego użytkownika}
{Brak}
{%
    \begin{tabitemize}
        \item \textbf{spotMediaId}: long (ID media)
    \end{tabitemize}
}
{200 OK, 404 Not Found}
{Brak (pusta odpowiedź)}

\cardendpoint
{EP48}
{/spot/check-is-spot-media-liked}
{Sprawdź, czy media spotu jest polubione przez zalogowanego użytkownika}
{Brak}
{%
    \begin{tabitemize}
        \item \textbf{spotMediaId}: long (ID media)
    \end{tabitemize}
}
{200 OK, 404 Not Found, 401 Unauthorized}
{%
    \begin{tabitemize}
        \item \textbf{isSpotMediaLiked}: boolean (czy aktualny użytkownik polubił to media)
    \end{tabitemize}
}

\cardendpoint
{EP49}
{/public/spot/get-spot-time-zone}
{Pobierz strefę czasową spotu}
{Brak}
{%
    \begin{tabitemize}
        \item \textbf{spotId}: long (ID spotu)
    \end{tabitemize}
}
{200 OK, 404 Not Found}
{%
    \begin{tabitemize}
        \item \textbf{timeZone}: String (identyfikator strefy czasowej, np. \texttt{Europe/Warsaw})
    \end{tabitemize}
}
