%! Author = Adam
%! Date = 23/12/2025

\subsection{Bezpieczeństwo aplikacji}
\label{subsec:bezpieczenstwo-aplikacji}

%1. Po co uzylismy spring security
%2. czym jest spring security
%https://docs.spring.io/spring-security/reference/index.html
%
%diagramy filtry: https://docs.spring.io/spring-security/reference/servlet/architecture.html
%
%JwtProperties
%UrlsProperties
%JwtConfig
%SecuritCOnfig
%
%folder security
%
%diagram ze springa jak dziala security? odwolania do chain of responsibility
%diagram jak sie ansze te wpisuaj w to

Niniejszy rozdział prezentuje implementację zabezpieczeń aplikacji na backendzie.
Rozdział przygotowano w oparciu o podane źródła:
\begin{itemize}
    \item https://docs.spring.io/spring-security/reference/index.html.
    \item https://docs.spring.io/spring-security/reference/servlet/architecture.html.
\end{itemize}


Podczas tworzenia aplikacji webowej opartej o serwer zapewniający \gls{api}
kwestie bezpieczeństwa są jednym z kluczowych aspektów projektowych.
Wystawienie usług do sieci oznacza, że system jest stale narażony na próby
nieautoryzowanego dostępu, nadużycia mechanizmów logowania, a także ataki
prowadzące do ujawnienia lub modyfikacji danych.
Brak odpowiednich zabezpieczeń może skutkować przejęciem kont użytkowników,
wyciekiem informacji lub wykorzystaniem zasobów systemu
w sposób niezgodny z przeznaczeniem.


Na bezpieczeństwo aplikacji składają się przede wszystkim następujące obszary:

\begin{itemize}
    \item \textbf{Uwierzytelnianie} -- proces potwierdzania tożsamości klienta
    komunikującego się z systemem. Oznacza to zapewnienie, że żądanie
    pochodzi od konkretnego, znanego użytkownika.

    \item \textbf{Autoryzacja} -- określenie i przestrzeganie polityki dostępu oraz operacji na zasobach,
    do jakich dostęp ma uwierzytelniony użytkownik.
    Nawet poprawnie uwierzytelniony klient
    nie powinien mieć możliwości wykonania akcji, do których nie posiada uprawnień.

    \item \textbf{Ochrona przed typowymi atakami} -- zestaw mechanizmów, które mają
    na celu ograniczenie ryzyka podatności i nadużyć, takich jak ataki typu brute-force,
    nieprawidłowa konfiguracja \gls{cors}, czy wykorzystanie luk
    w warstwie aplikacyjnej. Istotne są również bezpieczne nagłówki HTTP,
    poprawna obsługa błędów oraz kontrola sposobu przetwarzania danych wejściowych.

    \item \textbf{Integralność i poufność komunikacji} -- zapewnienie, że dane przesyłane
    pomiędzy klientem a serwerem nie mogą zostać łatwo podejrzane lub zmodyfikowane
    po drodze. W kontekście aplikacji internetowych sprowadza się to głównie do
    stosowania szyfrowania transportowego (HTTPS/TLS) oraz konsekwentnego egzekwowania
    polityk bezpieczeństwa w warstwie serwera.
\end{itemize}

W projekcie wykorzystano ekosystem Spring Boot, który zapewnia gotową integrację
mechanizmów bezpieczeństwa w ramach Spring Security.

\subsubsection{Spring Security}
\label{subsubsec:spring-security}

Spring Security jest frameworkiem dostarczanym w ramach ekosystemu Spring'a,
który można dołączyć do projektu jako zależność.
Jego zadaniem jest zapewnienie kompletnego zestawu mechanizmów związanych z bezpieczeństwem
warstwy serwerowej: od uwierzytelniania i autoryzacji, po ochronę przed najczęściej
spotykanymi zagrożeniami w aplikacjach webowych.

Istotną cechą Spring Security jest oparcie działania o łańcuch filtrów,
przez który przechodzi każde żądanie HTTP trafiające do aplikacji.
Twórcy frameworka skorzystali w tym celu ze wzorca [odniesienie do opisu CoR]. %TODO: uzupelnic potem
Dzięki temu możliwe jest przechwycenie żądania na wczesnym etapie,
zbudowanie kontekstu bezpieczeństwa oraz podjęcie decyzji,
czy żądanie może zostać przekazane dalej.
Oznacza to, że kontrola dostępu jest realizowana konsekwentnie i centralnie,
bez konieczności ręcznego powielania sprawdzeń w wielu miejscach kodu.

\subsubsection{Konfiguracja Spring Security}
\label{subsubsec:konfiguracja-spring-security}

diagram 1
daigram 2



