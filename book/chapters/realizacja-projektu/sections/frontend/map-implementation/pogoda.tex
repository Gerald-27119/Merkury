%! Author = Stanisław Oziemczuk
%! Date = 22.12.2025

\subsubsection{Panel z informacjami pogodowymi}
\label{subsubsec:panel-z-informacjami-pogodowymi}

Panel ten jest odpowiedzialny za wyświetlanie informacji pogodowych wybranego \glslink{spot}{spota}.
Składa się z dwóch głównych części:
\begin{itemize}
    \item panelu z ogólnymi danymi takimi jak ikona stanu pogody, temperatura, prędkość wiatru
    \item zakładki zawierającej szczegółowe informacje: aktualny czas, ikonę stanu pogody, prawdopodobieństwo opadów,
    wskaźnik UV, punkt rosy, wilgotność, prędkość wiatru na różnych wysokościach, wykres z prognozowaną pogodą na trzy dni
\end{itemize}

\subsubsubsection{Panel z danymi ogólnymi}
\label{subsubsec:panel-z-danymi-ogolnymi}

Składa się z dwóch postawowych \glslink{react-component}{komponentów}:
\begin{itemize}
    \item \texttt{BasicSpotWeather}
    \item \texttt{WeatherIcon}
\end{itemize}

\textbf{\texttt{BasicSpotWeather}} (rys. \ref{fig:basic-spot-weather-1} i \ref{fig:basic-spot-weather-2})
\glslink{react-component}{Komponent} jest odpowiedzialny za wyświetlanie następujących informacji pogodowych: temperatura, prędkość wiatru,
także ikony obrazującej obecny stan warunków atmosferycznych (komponent \texttt{WeatherIcon (por. sekcja~\ref{subsubsec:komponent-wspolny})}).
Dane pobierane są z odpowiedniego \glslink{endpoint}{endpointu} \glslink{backend}{backendu}.
W czasie ich ładowania wyświetlany jest \glslink{react-component}{komponent} \texttt{LoadingSpinner}.
Jeśli operacja zakończy się niepowodzeniem, użytkownikowi wyświetlany jest komunikat \emph{Failed to load weather for this spot!}.
Przycisk \emph{Show more} powoduje zamknięcie obecnego panelu i otworzenie zakładki ze szczegółowymi informacjami
(komponent \texttt{DetailedSpotWeather})

\begin{figure}[H]
    \centering
    \includegraphics[width=1\textwidth]{attachments/implementacja-frontendu/mapa/pogoda/basic_spot_weather1}
    \caption{Implementacja komponentu BasicSpotWeather (1/2)}
    \label{fig:basic-spot-weather-1}
\end{figure}
\noindent

\begin{figure}[H]
    \centering
    \includegraphics[width=1\textwidth]{attachments/implementacja-frontendu/mapa/pogoda/basic_spot_weather2}
    \caption{Implementacja komponentu BasicSpotWeather (2/2)}
    \label{fig:basic-spot-weather-2}
\end{figure}
\noindent

\subsubsubsection{Panel z danymi szczegółowymi}
\label{subsubsec:panel-z-danymi-szczegolowymi}

Został zbudowany z x \glslink{react-component}{komponentów}:
\begin{itemize}
    \item \texttt{DetailedSpotWeather}
    \item \texttt{WeatherOverview}
    \item \texttt{WeatherDetails}
    \item \texttt{WeatherTile}
    \item \texttt{WindSpeeds}
    \item \texttt{WindSpeedDisplay}
    \item \texttt{SelectHeightButton}
    \item \texttt{WeatherTimelinePlot}
    \item \texttt{VictoryChart}
    \item \texttt{VictoryContainer}
    \item \texttt{VictoryLine}
    \item \texttt{VictoryScatter}
    \item \texttt{CustomTickLabel}
    \item \texttt{VictoryLabel}
    \item \texttt{WeatherIcon}
\end{itemize}



\textbf{\texttt{DetailedSpotWeather}} (rys. \ref{fig:detailed-spot-weather-1} i \ref{fig:detailed-spot-weather-2})
Odpowiedzialny jest za układ wszystkich części panelu.
Zawiera w sobie \glslink{react-component}{komponenty} \texttt{WeatherOverview}, \texttt{WeatherDetails}, \texttt{WindSpeeds}
oraz \texttt{WeatherTimelinePlot}, które wyświetlają poszczególne częsci informacji pogodowych.
W celu zapewnienia płynnych przejść, renderowanie i zamykanie panelu animowane jest za pomocą \glslink{biblioteka}{biblioteki} \emph{motion}.
Dane przekazywane \glslink{react-component}{komponentom} \texttt{WeatherOverview} i \texttt{WeatherDetails} pobierane są z
odpowiedniego \glslink{endpoint}{endpointu} \glslink{backend}{backendu} przy użyciu \glslink{hook}{hook'a} \emph{useQuery}
z \glslink{tanstack-query}{tanstack query}.
Podczas ich pobierania wyświetlany jest \texttt{LoadingSpinner}, a jeśli operacja zakończy się błędem, pokazywany jest
komunikat \emph{Failed to load weather.}.
Zamknięcie panelu spowoduje wyświetlenie \glslink{react-component}{komponentu} \texttt{BasicSpotWeather}.

\begin{figure}[H]
    \centering
    \includegraphics[width=1\textwidth]{attachments/implementacja-frontendu/mapa/pogoda/detailed_spot_weather1}
    \caption{Implementacja komponentu DetailedSpotWeather (1/2)}
    \label{fig:detailed-spot-weather-1}
\end{figure}
\noindent

\begin{figure}[H]
    \centering
    \includegraphics[width=1\textwidth]{attachments/implementacja-frontendu/mapa/pogoda/detailed_spot_weather2}
    \caption{Implementacja komponentu DetailedSpotWeather (2/2)}
    \label{fig:detailed-spot-weather-2}
\end{figure}
\noindent


\textbf{\texttt{WeatherOverview}} (rys. \ref{fig:weather-overview-1} i \ref{fig:weather-overview-2})
Wyświetla informacje ogólne informacje o \glslink{spot}{spocie} i jego pogodzie: temperaturę, aktualny czas,
ikonę wraz ze słownym opisem, nazwę, lokalizację.
Do poprawnego wyświetlania czasu, z \glslink{backend}{backendu} pobierana jest strefa czasowa \glslink{spot}{spota}.
Gdy operacja zakończy się sukcesem w \glslink{hook}{hook'u} \emph{useEffect} ustawiany jest interwał czasowy, aktualizujący
się co minutę.
Po usunięcia \glslink{react-component}{komponentu} z drzewa renderowania, funkcja czyszcząca usunie interwał.
Obrazowanie stanu pogody realizowane jest za pomocą \texttt{WeatherIcon} (por. sekcja~\ref{subsubsec:komponent-wspolny})
oraz funkcji \texttt{getWeatherAdjective}, która na podstawie kodu pogody oraz parametru \emph{isDay} zwraca słowny
opis warunków atmosferycznych.
Nazwa oraz lokalizacja \glslink{spot}{spota} pobierane są ze stanu \glslink{redux}{Redux'a} za pomoca
\glslink{hook}{hook'a} \texttt{useSelectorTyped}.

\begin{figure}[H]
    \centering
    \includegraphics[width=1\textwidth]{attachments/implementacja-frontendu/mapa/pogoda/weather_overwiev1}
    \caption{Implementacja komponentu WeatherOverview (1/2)}
    \label{fig:weather-overview-1}
\end{figure}
\noindent

\begin{figure}[H]
    \centering
    \includegraphics[width=1\textwidth]{attachments/implementacja-frontendu/mapa/pogoda/weather_overwiev2}
    \caption{Implementacja komponentu WeatherOverview (2/2)}
    \label{fig:weather-overview-2}
\end{figure}
\noindent

\textbf{\texttt{WeatherDetails}}
\glslink{react-component}{Komponent} odpowiedzialny za wyświetlanie następujących danych pogodowych:
prawdopodobieństwo opadów, punkt rosy, wskaźnik indeksu UV, poziom wilgotności.
Indeks UV przedstawiony jest słownie poprzez funkcję \texttt{getUvIndexTextLevel}, która określa go na podstawie
wartości liczbowej przekazywanej w jej argumencie.
Poszczególne dane prezentowane są w formie kafelek przez \glslink{react-component}{komponenty} \texttt{WeatherTile},
razem tworzących siatkę 2x2.

\textbf{\texttt{WindSpeeds}} (rys. \ref{fig:wind-speeds-1} i \ref{fig:wind-speeds-2})
W tym \glslink{react-component}{komponencie} wyświetlane są prędkości wiatru na różnych wysokościach w \texttt{km\\h} lub \texttt{m\\s}.
Dane pobierane są z odpowiedniego \glslink{endpoint}{endpointu} \glslink{backend}{backendu} przy użyciu
\glslink{hook}{hook'a} \emph{useQuery}.
W trakcie ich ładowania wyświetlany jest \texttt{LoadingSpinner}, a w razie błędu komunikat \emph{Failed to load wind speeds data}.
W lokalnym stanie (\glslink{hook}{hook} \texttt{useState}) przechowywana jest informacja, która wysokość została wybrana.
Natomiast za jego zmianę odpowiedzialny jest \glslink{react-component}{komponent} \texttt{SelectHeightButton}.
Oprócz tego, przycisk ten informuje o wybranej wysokości poprzez warunkową zmianę wyglądu.
Za obsługę zmiany jednostek prędkości wiatru oraz wyświetlanie jej wartości odpowiada \texttt{WindSpeedDisplay}.
\glslink{react-component}{Komponent} w stanie lokalnym przechowuje informacje o wybranej jednostce, a po każdej
jej zmianie przelicza nową wartość wyświetlaną z dokładnością do części dziesiętnej.

\begin{figure}[H]
    \centering
    \includegraphics[width=1\textwidth]{attachments/implementacja-frontendu/mapa/pogoda/wind_speeds1}
    \caption{Implementacja komponentu WindSpeeds (1/2)}
    \label{fig:wind-speeds-1}
\end{figure}
\noindent

\begin{figure}[H]
    \centering
    \includegraphics[width=1\textwidth]{attachments/implementacja-frontendu/mapa/pogoda/wind_speeds2}
    \caption{Implementacja komponentu WindSpeeds (2/2)}
    \label{fig:wind-speeds-2}
\end{figure}
\noindent

\textbf{\texttt{WeatherTimelinePlot}} (rys. \ref{fig:weather-timeline-plot-1} - \ref{fig:weather-timeline-plot-4})
Ten \glslink{react-component}{komponent} przedstawia wykres prezentujący prognozę pogody na trzy dni.
Zawiera informacje o temperaturze, prawdopodobieństwie opadów, godzinie oraz ikonę symbolizującą warunki atmosferyczne.
Dane pobierane są z \glslink{endpoint}{endpointu} \glslink{backend}{backendu} poprzez \glslink{hook}{hook}
\texttt{useQuery}.
Pobieranie danych oznaczone jest przez \texttt{LoadingSpinner}, a błąd komunikatem \emph{Failed to load data for timeline plot.}.
Wykres został stworzony przy użyciu elementów \glslink{biblioteka}{biblioteki} \texttt{victory}.
Głównym \glslink{react-component}{komponentem} jest \texttt{VictoryChart}, który przyjmuje dane na osie X i Y, a także
pozostałe elementy wykresu.
Linia obrazująca wartości temperatury zbudowana jest przy użyciu \texttt{VictoryLine} oraz \texttt{VictoryScatter}.
Natomiast sposób wyświetlenia osi poziomej określany jest przez \texttt{VictoryAxis}, przyjmujący \texttt{CustomTickLabel},
w którym dokładnie poszczególne elementy opisu.
W celu zachowania stałej wysokości obszaru wyświetlanych danych niezależnie od ich rozpiętości,
wartości temperatur są znormalizowane do wirtualnego zakresu od 0 do 100.

\begin{figure}[H]
    \centering
    \includegraphics[width=1\textwidth]{attachments/implementacja-frontendu/mapa/pogoda/weather_timeline_plot1}
    \caption{Implementacja komponentu WeatherTimelinePlot (1/4)}
    \label{fig:weather-timeline-plot-1}
\end{figure}
\noindent

\begin{figure}[H]
    \centering
    \includegraphics[width=1\textwidth]{attachments/implementacja-frontendu/mapa/pogoda/weather_timeline_plot2}
    \caption{Implementacja komponentu WeatherTimelinePlot (2/4)}
    \label{fig:weather-timeline-plot-2}
\end{figure}
\noindent

\begin{figure}[H]
    \centering
    \includegraphics[width=1\textwidth]{attachments/implementacja-frontendu/mapa/pogoda/weather_timeline_plot3}
    \caption{Implementacja komponentu WeatherTimelinePlot (3/4)}
    \label{fig:weather-timeline-plot-3}
\end{figure}
\noindent

\begin{figure}[H]
    \centering
    \includegraphics[width=1\textwidth]{attachments/implementacja-frontendu/mapa/pogoda/weather_timeline_plot4}
    \caption{Implementacja komponentu WeatherTimelinePlot (4/4)}
    \label{fig:weather-timeline-plot-4}
\end{figure}
\noindent


\subsubsubsection{Komponent wspólny}
\label{subsubsec:komponent-wspolny}

W tym rozdziale przedstawiono komponent, z którego korzystają obydwie części modułu pogody.


\textbf{\texttt{WeatherIcon}}
To \glslink{react-component}{komponent}, którego zadaniem jest na podstawie kodu pogody (\emph{code}) i informacji czy
jest dzień (\emph{isDay}) wybrać odpowiednią ikonę symbolizującą stan pogody.
W przypadku braku znalezienia odpowiedniego elementu zwracana jest ikona znaku zapytania.
Komponent opcjonalnie przyjmuje parametr określający rozmiar ikony (\emph{textSize}), będący klasą \glslink{tailwind-css}{Tailwind}.
Jeśli nie zostanie podany, ustawiona zostanie domyślna opcja \emph{text-3xl}.
Poniżej przedstawiono implementację tego komponentu.

\begin{figure}[H]
    \centering
    \includegraphics[width=1\textwidth]{attachments/implementacja-frontendu/mapa/pogoda/weather_icon1}
    \caption{Implementacja komponentu WeatherIcon (1/2)}
    \label{fig:weather-icon-1}
\end{figure}
\noindent

\begin{figure}[H]
    \centering
    \includegraphics[width=1\textwidth]{attachments/implementacja-frontendu/mapa/pogoda/weather_icon2}
    \caption{Implementacja komponentu WeatherIcon (2/2)}
    \label{fig:weather-icon-2}
\end{figure}
\noindent
