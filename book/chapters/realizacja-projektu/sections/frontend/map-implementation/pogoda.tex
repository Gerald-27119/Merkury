%! Author = Stanisław Oziemczuk
%! Date = 22.12.2025

\subsubsection{Panel z informacjami pogodowymi}
\label{subsubsec:panel-z-informacjami-pogodowymi}

Panel ten jest odpowiedzialny za wyświetlanie informacji pogodowych wybranego \glslink{spot}{spota}.
Składa się z dwóch głównych części:
\begin{itemize}
    \item panelu z ogólnymi danymi takimi jak ikona stanu pogody, temperatura, prędkość wiatru
    \item zakładki zawierającej szczegółowe informacje: aktualny czas, ikonę stanu pogody, prawdopodobieństwo opadów,
    wskaźnik UV, punkt rosy, wilgotność, prędkość wiatru na różnych wysokościach, wykres z prognozowaną pogodą na trzy dni
\end{itemize}

\subsubsubsection{Panel z danymi ogólnymi}
\label{subsubsec:panel-z-danymi-ogolnymi}

Składa się z dwóch postawowych komponentów:
\begin{itemize}
    \item \texttt{BasicSpotWeather}
    \item \texttt{WeatherIcon}
\end{itemize}

\textbf{\texttt{BasicSpotWeather}} (rys. \ref{fig:basic-spot-weather-1} i \ref{fig:basic-spot-weather-2})
Komponent jest odpowiedzialny za wyświetlanie następujących informacji pogodowych: temperatura, prędkość wiatru,
także ikony obrazującej obecny stan warunków atmosferycznych (komponent \texttt{WeatherIcon (por. sekcja~\ref{subsubsec:komponenty-wspolne})}).
Dane pobierane są z odpowiedniego \glslink{endpoint}{endpointu} \glslink{backend}{backendu}.
W czasie ich ładowania wyświetlany jest komponent \texttt{LoadingSpinner}.
Jeśli operacja zakończy się niepowodzeniem, użytkownikowi wyświetlany jest komunikat \emph{Failed to load weather for this spot!}.
Przycisk \emph{Show more} powoduje zamknięcie obecnego panelu i otworzenie zakładki ze szczegółowymi informacjami
(komponent \texttt{DetailedSpotWeather})

\begin{figure}[H]
    \centering
    \includegraphics[width=1\textwidth]{attachments/implementacja-frontendu/mapa/pogoda/basic_spot_weather1}
    \caption{Implementacja komponentu BasicSpotWeather (1/2)}
    \label{fig:basic-spot-weather-1}
\end{figure}
\noindent

\begin{figure}[H]
    \centering
    \includegraphics[width=1\textwidth]{attachments/implementacja-frontendu/mapa/pogoda/basic_spot_weather2}
    \caption{Implementacja komponentu BasicSpotWeather (2/2)}
    \label{fig:basic-spot-weather-2}
\end{figure}
\noindent

\subsubsubsection{Komponenty wspólne}
\label{subsubsec:komponenty-wspolne}
