%! Author = Stanisław Oziemczuk
%! Date = 22.12.2025

\subsubsection{Duża galeria zdjęć i filmów}
\label{subsubsec:duza-galeria-zdjec-i-filmow}

Wyświetlane są w niej wszystkie zdjęcia i filmy dodane do \glslink{spot}{spota}, zarówno poprzez komentarze jak i
dedykowany formularz.

Duża galeria została zbudowana z dwunastu \glslink{react-component}{komponentów}:
\begin{itemize}
    \item \texttt{ExpandedSpotMediaGallery}
    \item \texttt{ExpandedGallerySidebar}
    \item \texttt{SortingAndFilterPanel}
    \item \texttt{OptionButton}
    \item \texttt{ExpandedMediaGalleryActions}
    \item \texttt{ExpandedMediaDisplay}
    \item \texttt{MediaInfoDisplay}
    \item \texttt{MediaDisplayFilterPanel}
    \item \texttt{ExpandedGalleryPhoto}
    \item \texttt{ExpandedGalleryVideo}
    \item \texttt{ExpandedGalleryPanel}
    \item \texttt{FullscreenMediaModal}
\end{itemize}

\textbf{\texttt{ExpandedSpotMediaGallery}}
\glslink{react-component}{Komponent}, który jest kontenerem ustalającym położenie pozostałych elementów.
Jego otworzenie oraz zamknięcie jest animowane przy użyciu \glslink{biblioteka}{biblioteki} \texttt{motion}.

\textbf{\texttt{ExpandedGallerySidebar}} (rys. \ref{fig:expanded-media-gallery-sidebar-1} - \ref{fig:expanded-media-gallery-sidebar-13})
Wyświetla listę zdjęć lub filmów przewijalną przy użyciu \texttt{\glslink{infinite-scroll}{infinite scroll}}.
Dane pobierane są z \glslink{paginacja}{paginacją} z \glslink{backend}{backendu} przy pomocy \glslink{hook}{hook'a}
\texttt{useInfiniteQuery}.
Mechanizm ten został zrealizowany za pomocą \\ \texttt{IntersectionObserver}, który nasłuchuje na element powiązany z
referencją \newline \texttt{containerRef}.
Zbliżenie się do \texttt{loadNextPageRef} lub \texttt{loadPreviousPageRef} określa odpowiedni kierunek pobierania danych
(kolejne lub poprzednie strony), który jest zapisywany w \texttt{fetchDirection}.
W czasie ładowania stron wyświetlany jest \newline \texttt{LoadingSpinner}.
W \glslink{hook}{hook'u} \texttt{useEffect} po zmianie danych na podstawie zapisanej wartości, dane dodawane
są na początek lub koniec listy przechowywanej w stanie \glslink{redux}{Redux}.
Pierwszy element z listy jest wyświetlany w \glslink{react-component}{komponencie} \texttt{ExpandedMediaDisplay}.
Po zmianie sortowania lub typu medii (zdjęcia lub filmy), lista jest resetowana, a element ustalany na nowo.
Kliknięcie w dowolnego media spowoduje wyświetlenie go w \texttt{ExpandedMediaDisplay}.
Lista może być zwijana lub rozwijana przy użyciu odpowiedniego przycisku, a przejścia są animowane za pomocą
\glslink{biblioteka}{biblioteki} \texttt{motion}.

\begin{figure}[H]
    \centering
    \includegraphics[width=1\textwidth]{attachments/implementacja-frontendu/mapa/duza-galeria/expanded-gallery-sidebar-1}
    \caption{Implementacja komponentu ExpandedGallerySidebar (1/13)}
    \label{fig:expanded-media-gallery-sidebar-1}
\end{figure}
\noindent

\begin{figure}[H]
    \centering
    \includegraphics[width=1\textwidth]{attachments/implementacja-frontendu/mapa/duza-galeria/expanded-gallery-sidebar-2}
    \caption{Implementacja komponentu ExpandedGallerySidebar (2/13)}
    \label{fig:expanded-media-gallery-sidebar-2}
\end{figure}
\noindent

\begin{figure}[H]
    \centering
    \includegraphics[width=1\textwidth]{attachments/implementacja-frontendu/mapa/duza-galeria/expanded-gallery-sidebar-3}
    \caption{Implementacja komponentu ExpandedGallerySidebar (3/13)}
    \label{fig:expanded-media-gallery-sidebar-3}
\end{figure}
\noindent

\begin{figure}[H]
    \centering
    \includegraphics[width=1\textwidth]{attachments/implementacja-frontendu/mapa/duza-galeria/expanded-gallery-sidebar-4}
    \caption{Implementacja komponentu ExpandedGallerySidebar (4/13)}
    \label{fig:expanded-media-gallery-sidebar-4}
\end{figure}
\noindent

\begin{figure}[H]
    \centering
    \includegraphics[width=1\textwidth]{attachments/implementacja-frontendu/mapa/duza-galeria/expanded-gallery-sidebar-5}
    \caption{Implementacja komponentu ExpandedGallerySidebar (5/13)}
    \label{fig:expanded-media-gallery-sidebar-5}
\end{figure}
\noindent

\begin{figure}[H]
    \centering
    \includegraphics[width=1\textwidth]{attachments/implementacja-frontendu/mapa/duza-galeria/expanded-gallery-sidebar-6}
    \caption{Implementacja komponentu ExpandedGallerySidebar (6/13)}
    \label{fig:expanded-media-gallery-sidebar-6}
\end{figure}
\noindent

\begin{figure}[H]
    \centering
    \includegraphics[width=1\textwidth]{attachments/implementacja-frontendu/mapa/duza-galeria/expanded-gallery-sidebar-7}
    \caption{Implementacja komponentu ExpandedGallerySidebar (7/13)}
    \label{fig:expanded-media-gallery-sidebar-7}
\end{figure}
\noindent

\begin{figure}[H]
    \centering
    \includegraphics[width=1\textwidth]{attachments/implementacja-frontendu/mapa/duza-galeria/expanded-gallery-sidebar-8}
    \caption{Implementacja komponentu ExpandedGallerySidebar (8/13)}
    \label{fig:expanded-media-gallery-sidebar-8}
\end{figure}
\noindent

\begin{figure}[H]
    \centering
    \includegraphics[width=1\textwidth]{attachments/implementacja-frontendu/mapa/duza-galeria/expanded-gallery-sidebar-9}
    \caption{Implementacja komponentu ExpandedGallerySidebar (9/13)}
    \label{fig:expanded-media-gallery-sidebar-9}
\end{figure}
\noindent

\begin{figure}[H]
    \centering
    \includegraphics[width=1\textwidth]{attachments/implementacja-frontendu/mapa/duza-galeria/expanded-gallery-sidebar-10}
    \caption{Implementacja komponentu ExpandedGallerySidebar (10/13)}
    \label{fig:expanded-media-gallery-sidebar-10}
\end{figure}
\noindent

\begin{figure}[H]
    \centering
    \includegraphics[width=1\textwidth]{attachments/implementacja-frontendu/mapa/duza-galeria/expanded-gallery-sidebar-11}
    \caption{Implementacja komponentu ExpandedGallerySidebar (11/13)}
    \label{fig:expanded-media-gallery-sidebar-11}
\end{figure}
\noindent

\begin{figure}[H]
    \centering
    \includegraphics[width=1\textwidth]{attachments/implementacja-frontendu/mapa/duza-galeria/expanded-gallery-sidebar-12}
    \caption{Implementacja komponentu ExpandedGallerySidebar (12/13)}
    \label{fig:expanded-media-gallery-sidebar-12}
\end{figure}
\noindent

\begin{figure}[H]
    \centering
    \includegraphics[width=1\textwidth]{attachments/implementacja-frontendu/mapa/duza-galeria/expanded-gallery-sidebar-13}
    \caption{Implementacja komponentu ExpandedGallerySidebar (13/13)}
    \label{fig:expanded-media-gallery-sidebar-13}
\end{figure}
\noindent

\textbf{\texttt{SortingAndFilterPanel}}
\glslink{react-component}{Komponent} ten odpowiedzialny jest za obsługę ustawiania filtrów i sortowania.
Dostępne opcje wyświetlane są w \texttt{OptionButton}, a aktywne oznaczane poprzez wyróżniający się wygląd.
Obecnie wybrane wartości pobierane są ze stanu \glslink{redux}{Redux'a}, a nowe w nim ustawiane.
\glslink{react-component}{Komponent} umożliwia jednoczesne wybranie jednego filtru i sortowania.
Typ początkowy filtru określany jest na podstawie klikniętego media, natomiast sortowanie ustawiane jest
jako od najnowszych.


\textbf{\texttt{ExpandedMediaDisplay}}
Zbudowany jest z \glslink{react-component}{komponentów} \texttt{MediaInfoDisplay}, \\ \texttt{MediaDisplayFilterPanel} oraz \texttt{ExpandedGalleryPanel}.
Pobiera wcześniej wybrane media z \glslink{backend}{backendu}, a następnie w zależności od typu
wyświetla je w \newline \texttt{ExpandedGalleryPhoto} (zdjęcie) lub \\ \texttt{ExpandedGalleryVideo} (film).
Obsługa video odbywa się za pomocą elementów \glslink{biblioteka}{bilbioteki} \texttt{react-player}.
W trakcie pobierania danych wyświetlany jest \texttt{LoadingSpinner}, a błędzie informuje komunikat
\emph{Failed to fetch media.}.

\textbf{\texttt{MediaInfoDisplay}}
\glslink{react-component}{Komponent} odpowiedzialny za wyświetlanie informacji o wybranym zdjęciu.
Zawiera dane o autorze: nazwę użytkownika i jego zdjęcie profilowe oraz datę publikacji media.

\textbf{\texttt{MediaDisplayFilterPanel}}
Służy do przełączania typu miediów (zdjęcia lub filmy), a wybrana opcja zapisywana jest
w stanie \glslink{redux}{Redux'a}.

\textbf{\texttt{ExpandedGalleryPanel}}
\glslink{react-component}{Komponent} obsługujący akcje, które użytkownik może wykonać w stosunku do wyświetlanego
media.
Polubienie elementu realizowane jest poprzez funkcję \texttt{handleClickEditSpotMediaLikes}.
Jeźeli użytkownik nie jest zalogowany, w \texttt{Notification} wyświetlany jest odpowiedni komunikat.
W przeciwnym wypadku w zależności od tego czy polubienie jest dodawane, czy usuwane, wykonywana jest
optymistyczna zmiana \glslink{ui}{UI} (por. sekcja~\ref{subsubsec:Optimistic UI}).
Wysłanie odpowiedniego żądania odbywa się za pomocą mutacji z \glslink{biblioteka}{biblioteki}
\texttt{Tanstack Query}.
W przypadku błędu w \glslink{react-component}{komponencie} \newline \texttt{Notification} wyświetlany jest komunikat
\emph{Failed to edit spot media likes.}.
W celu utrzymania aktualności danych, powodzenie operacji spowoduje unieważnienie wszystkich zapytań (queries)
powiązanych z wybranym media, co wymusi ponowne pobranie danych.
Funkcja \texttt{handleSetMediaFullscreen} obsługuje powiększenie media na cały ekran poprzez
otworzenie \glslink{modal}{modalu} \texttt{FullscreenMediaModal}.
Kliknięcie przycisku z ikoną \texttt{BiDownload} powoduje przekierowanie do wybranego zasobu, który
jest pobierany przez przeglądarkę.
Montowanie \glslink{react-component}{komponentu} wykonuje kod w \texttt{useEffect}, wysyłący
przy użyciu \texttt{mutateAsync} z \glslink{hook}{hook'a} \texttt{useMutation} asynchroniczne
żądanie zwiększające liczbę wyświetleń media.

\textbf{\texttt{ExpandedMediaGalleryActions}}
Element realizujący akcje udostępnienia media oraz zamknięcia dużej galerii zdjęć i filmów.
Udostępnienie obsługuje funkcja \newline \texttt{handleClickShare}, która przy użyciu
\texttt{navigator.clipboard.writeText} kopiuje link do wybranego
elementu do pomięci podręcznej systemu użytkownika.
Komunikaty o stanie zakończenia operacji wyświetlane są w \glslink{react-component}{komponencie}
\texttt{Notification}.
