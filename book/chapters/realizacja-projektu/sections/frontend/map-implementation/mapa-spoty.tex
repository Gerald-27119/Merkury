%! Author = Stanisław Oziemczuk
%! Date = 22.12.2025

\subsubsection{Mapa z zaznaczonymi spotami}
\label{subsubsec:mapa-z-zaznaczonymi-spotami}

Mapa zbudowana jest z następujących \glslink{react-component}{komponentów}:
\begin{itemize}
    \item \texttt{MapPage}
    \item \texttt{Spots}
    \item \texttt{ZoomControlPanel}
    \item \texttt{ZoomControlButton}
    \item \texttt{UserLocationPanel}
    \item \texttt{SearchCurrentViewButton}
    \item \texttt{CurrentViewSpotsList}
    \item \texttt{CurrentViewSpotsFormsContainer}
    \item \texttt{CurrentViewSpotsNameSearchBar}
    \item \texttt{SpotsSortingForm}
    \item \texttt{RatingFromForm}
    \item \texttt{SpotsNameSearchBar}
    \item \texttt{SearchedSpotsList}
    \item \texttt{ListedSpotInfo}
    \item \texttt{SpotTag}
\end{itemize}

\textbf{\texttt{MapPage}} (rys. \ref{fig:map-page-1} - \ref{fig:map-page-5})
To główny \glslink{react-component}{komponent} prezentujący mapę.
Zawiera w sobie \texttt{Map} z \glslink{biblioteka}{biblioteki} \texttt{react-maplibre}, w którym ustawia się domyślną
lokalizację, na którą ustawiony jest obraz mapy oraz poziom przybliżenia.
Przyjmuje on także inne \glslink{react-component}{komponenty}, które mają być w nim wyświetlone.
Ustalenie, które z nich mają być widoczne realizowane jest poprzez odczytanie wartości ze stanu \glslink{redux}{Redux'a}.
Mapa ładowana jest z linków ustawionych w pliku \newline \texttt{map\_style.json}, w nim również opisany jest jej wygląd.
\glslink{hook}{Hook} \texttt{useEffect} uruchamia się tylko przy pierwszej inicjalizacji \glslink{react-component}{komponentu}
i sprawdza czy w \glslink{url}{url'u} umieszczone są parametry informujące, że link pochodzi z udostępnienia
\glslink{spot}{spota}.
Jeśli tak jest, odczytywane są odpowiednie dane, obraz mapy przekierowany jest na wskazaną lokalizację oraz
otwierane są panel ze szczegółami \glslink{spot}{spota} (por. sekcja~\ref{subsubsec:panel-ze-szczegolami-spota})
i jego pogodą (por. sekcja~\ref{subsubsec:panel-z-informacjami-pogodowymi}).

\begin{figure}[H]
    \centering
    \includegraphics[width=1\textwidth]{attachments/implementacja-frontendu/mapa/mapa-spoty/map-page-1}
    \caption{Implementacja komponentu MapPage (1/5)}
    \label{fig:map-page-1}
\end{figure}
\noindent

\begin{figure}[H]
    \centering
    \includegraphics[width=1\textwidth]{attachments/implementacja-frontendu/mapa/mapa-spoty/map-page-2}
    \caption{Implementacja komponentu MapPage (2/5)}
    \label{fig:map-page-2}
\end{figure}
\noindent

\begin{figure}[H]
    \centering
    \includegraphics[width=1\textwidth]{attachments/implementacja-frontendu/mapa/mapa-spoty/map-page-3}
    \caption{Implementacja komponentu MapPage (3/5)}
    \label{fig:map-page-3}
\end{figure}
\noindent

\begin{figure}[H]
    \centering
    \includegraphics[width=1\textwidth]{attachments/implementacja-frontendu/mapa/mapa-spoty/map-page-4}
    \caption{Implementacja komponentu MapPage (4/5)}
    \label{fig:map-page-4}
\end{figure}
\noindent

\begin{figure}[H]
    \centering
    \includegraphics[width=1\textwidth]{attachments/implementacja-frontendu/mapa/mapa-spoty/map-page-5}
    \caption{Implementacja komponentu MapPage (5/5)}
    \label{fig:map-page-5}
\end{figure}
\noindent

\textbf{\texttt{Spots}} (rys. \ref{fig:spots-1} - \ref{fig:spots-5})
\glslink{react-component}{Komponent} ten odpowiedzialny jest za nałożenie \glslink{spot}{spotów} na mapę.
Z \glslink{backend}{backendu} za pomocą \glslink{hook}{hook'a} \texttt{useQuery} pobierana jest ich lista, a następnie
zgodnie z wartością zwróconą przez funkcję \texttt{shouldRenderMarker},
przyjmującej pole powierzchni elementu oraz poziom przybliżenia mapy, \glslink{spot}{spot} wyświetlany jest w formie
punktu (\texttt{Marker}) lub wielokąta (\texttt{Source} oraz \texttt{Layer}).
Wymienione \glslink{react-component}{komponenty} pochodzą z \glslink{biblioteka}{biblioteki} \texttt{react-maplibre}.
Zawiera ona również \glslink{hook}{hook} \texttt{useMap}, który zwraca obecną instancję mapy.
Pozwala to dodać do wyświetlanych elementów event-listener'ów obsługujących najechanie na dany element oraz funkcji
wywoływanej po jego kliknięciu.
Jest to funkcja \texttt{handleSpotClick}.
Powoduje ustawienie w stanie \glslink{redux}{Redux'a} danych \glslink{spot}{spota} oraz otworzenie
panelu z jego szczegółami (por. sekcja~\ref{subsubsec:panel-ze-szczegolami-spota})
i pogodą (por. sekcja~\ref{subsubsec:panel-z-informacjami-pogodowymi}).
Za pomocą \glslink{mutacja}{mutacji} zostaje również wysłane żądanie zwiększające liczbę wyświetleń elementu.
Informacja o błędzie podczas pobierana danych wyświetlana jest w \glslink{react-component}{komponencie}
\texttt{Notification}.

\begin{figure}[H]
    \centering
    \includegraphics[width=1\textwidth]{attachments/implementacja-frontendu/mapa/mapa-spoty/spots1}
    \caption{Implementacja komponentu Spots (1/5)}
    \label{fig:spots-1}
\end{figure}
\noindent

\begin{figure}[H]
    \centering
    \includegraphics[width=1\textwidth]{attachments/implementacja-frontendu/mapa/mapa-spoty/spots2}
    \caption{Implementacja komponentu Spots (2/5)}
    \label{fig:spots-2}
\end{figure}
\noindent

\begin{figure}[H]
    \centering
    \includegraphics[width=1\textwidth]{attachments/implementacja-frontendu/mapa/mapa-spoty/spots3}
    \caption{Implementacja komponentu Spots (3/5)}
    \label{fig:spots-3}
\end{figure}
\noindent

\begin{figure}[H]
    \centering
    \includegraphics[width=1\textwidth]{attachments/implementacja-frontendu/mapa/mapa-spoty/spots4}
    \caption{Implementacja komponentu Spots (4/5)}
    \label{fig:spots-4}
\end{figure}
\noindent

\begin{figure}[H]
    \centering
    \includegraphics[width=1\textwidth]{attachments/implementacja-frontendu/mapa/mapa-spoty/spots5}
    \caption{Implementacja komponentu Spots (5/5)}
    \label{fig:spots-5}
\end{figure}
\noindent

\textbf{\texttt{ZoomControlPanel}}
\glslink{react-component}{Komponent} obsługujący ustawianie przybliżenia przybliżenia mapy.
Zawiera dwa przyciski wyświetlane przez \texttt{ZoomControlButton}, przyjmujące funkcje, które na instacji mapy
pobieranej z \glslink{hook}{hook'a} \texttt{useMap} (\glslink{biblioteka}{biblioteka} \texttt{react-maplibre}),
wywołują na niej odpowiednio wbudowane operacje \texttt{zoomIn()} lub \texttt{zoomOut()}.

\textbf{\texttt{UserLocationPanel}}
Odpowiada za wyświetlenie na mapie aktualnej lokalizacji użytkownika.
Po kliknięciu przycisku za pomocą \\ \texttt{navigator.geolocation.getCurrentPosition} pobiera jego aktualną pozycję, a
następnie zapisuje ją w lokalnym stanie \glslink{react-component}{komponentu} (\texttt{useState}).
Na ich podstawie na mapie wyświetlany jest \texttt{Marker} z \glslink{biblioteka}{biblioteki} \texttt{react-maplibre}.
Pochodzący z niej \glslink{hook}{hook} \texttt{useMap} zwraca aktualną instancję mapy, na której wywoływana jest
funkcja \texttt{flyTo} przenosząca płynnie obraz na wskazane koordynaty.
Informacje o błędach podczas operacji wyświetlane są w \texttt{Notification}.

\textbf{\texttt{SearchCurrentViewButton}} (rys. \ref{fig:search-current-view-btn-1} i rys. \ref{fig:search-current-view-btn-2})
\glslink{react-component}{Komponent} będący przyciskiem, po kliknięciu którego pokazywana jest lista \glslink{spot}{spotów}
znajdujących się w widocznym obszarze mapy.
Naciśnięcie wywołuje funkcję \texttt{handleClickSearchCurrentView}, która na obecnej instacji mapy zwróconej
przez \glslink{hook}{hook} \texttt{useMap} (\glslink{biblioteka}{biblioteka} \texttt{react-maplibre}) za pomocą
wbudowanej funkcji \texttt{getBounds} pobiera dwa punkty wyznaczające granicę szerokości i długości geograficznej
obecnie wyświetlanego fragmentu mapy.
Następnie na \glslink{backend}{backend} wysyłane jest odpowiednie zapytanie i otwierany jest panel z listą
znalezionych elementów.

\begin{figure}[H]
    \centering
    \includegraphics[width=1\textwidth]{attachments/implementacja-frontendu/mapa/mapa-spoty/search-current-view-btn-1}
    \caption{Implementacja komponentu SearchCurrentViewButton (1/2)}
    \label{fig:search-current-view-btn-1}
\end{figure}
\noindent

\begin{figure}[H]
    \centering
    \includegraphics[width=1\textwidth]{attachments/implementacja-frontendu/mapa/mapa-spoty/search-current-view-btn-2}
    \caption{Implementacja komponentu SearchCurrentViewButton (2/2)}
    \label{fig:search-current-view-btn-2}
\end{figure}
\noindent

\textbf{\texttt{CurrentViewSpotsList}}
Wyświetla listę \glslink{spot}{spotów} znajdujących się w widocznym obszarze mapy.
Poszczególne elementy wyświetlane są przez \texttt{ListedSpotInfo} (por. sekcja~\ref{subsubsec:komponenty-wspolne-map-spots}).
Lista przewijana jest przy użyciu mechanizmu \glslink{infinite-scroll}{infinite scroll}, zrealizowanego przez
\\ \texttt{IntersectionObserver} nasłuchującego na element powiązany referencją oraz \glslink{hook}{hook'a}
\texttt{useInfiniteQuery} z \glslink{biblioteka}{biblioteki} \texttt{Tanstack Query}.
Podczas ładowania danych wyświetlany jest \texttt{LoadingSpinner}, a o wystąpieniu błędu informuje komunikat
\emph{Failed to load spots data.}.
W \glslink{react-component}{komponencie} \\ \texttt{CurrentViewSpotsFormsContainer} umieszczone są formularze
do filtrowania i sortowania listy.
Filtrowanie \glslink{spot}{spotów} po nazwie obsługuje \texttt{CurrentViewSpotsNameSearchBar}.
W trakcie wpisywania nazwy wyświetlana jest lista z podpowiedziami ułatwiająca uzupełnienie pola.
Po kliknięciu przycisku do szukania lub wyborze podpowiedzi wyświetlane są \glslink{spot}{spoty}, których nazwa
zawiera w sobie wpisaną frazę.
Filtrowanie po ocenie „od”, tzn. ustawienia minimalnej wartości tejże, umożliwia \glslink{react-component}{komponent}
\texttt{RatingFromForm}.
Za pomocą \texttt{Rate} z \glslink{biblioteka}{biblioteki} \texttt{antd} wybierana jest wartość oceny, a nstępnie
lista wyników jest aktualizowana.
Jeżeli żaden \glslink{spot}{spot} nie spełnia warunków ustawionych przez opisane filtry, wyświetlany jest komunikat
\emph{No spots match criteria!}.
Sortowanie listy realizowane jest przez \glslink{react-component}{komponent} \texttt{SpotsSortingForm}
(por. sekcja~\ref{subsubsec:komponenty-wspolne-map-spots}).
Otworzenie i zamknięcie panelu animowane jest przy użyciu elementów \glslink{biblioteka}{biblioteki} \texttt{motion}.

\textbf{\texttt{SpotsNameSearchBar}}
\glslink{react-component}{Komponent} umożliwiający wyszukiwanie \glslink{spot}{spotów} po nazwie.
Podczas wpisywania frazy z \glslink{backend}{backendu} za pomocą \glslink{hook}{hook'a} \texttt{useQuery}
(\glslink{biblioteka}{biblioteka} \texttt{Tanstack Query}) pobierane są nazwy wszystkich \glslink{spot}{spotów}
zawierających podany tekst i wyświetlane jako lista z podpowiedziami.
Nazwy pobierane są przy wykorzystaniu \glslink{debounce}{debounce'owania} zrealizowanego za pomocą \glslink{hook}{hook'a} \texttt{useDebounce},
co pozwala ograniczyć liczbę wysyłanych zapytań.
Po kliknięciu przycisku wyszukiwania lub wybraniu podpowiedzi wyświetlany jest panel z listą wyników (\texttt{SearchedSpotsList}).
Kliknięcie ikony \texttt{IoClose} powoduje usunięcie filtra.

\textbf{\texttt{SearchedSpotsList}}
Wyświetla wynik wyszukiwania \glslink{spot}{spotów} po nazwie.
Lista jest przewijalna z wykorzystaniem mechanizmu \glslink{infinite-scroll}{infinite scroll} zrealizowanego
za pomocą \\ \texttt{IntersectionObserver}, \glslink{hook}{hook'a} \texttt{useInfiniteQuery} (\glslink{biblioteka}{biblioteka}
\texttt{Tanstack Query}) oraz referencji.
Podczas pobierania danych prezentowany jest \texttt{LoadingSpinner}, a w wypadku błędu komunikat \emph{Failed to load searched spots data.}.
Jeżeli lista wyników jest pusta, pojawia się informacja \emph{No spots match criteria!}.
Elementy można sortować poprzez \glslink{react-component}{komponent} \texttt{SpotsSortingForm}
(por. sekcja~\ref{subsubsec:komponenty-wspolne-map-spots}).
Otworzenie i zamknięcie panelu animowane jest za pomocą elementów \glslink{biblioteka}{biblioteki} \texttt{motion}.

\subsubsubsection{Komponenty wspólne}
\label{subsubsec:komponenty-wspolne-map-spots}

W niniejszym rozdziale opisano \glslink{react-component}{komponenty}, z których korzystają inne \glslink{react-component}{komponenty}
zawarte w tej części.

\textbf{\texttt{ListedSpotInfo}}
Wyświetla pojedycznego \glslink{spot}{spota} będącego wynikiem wyszukiwania po nazwie lub w widocznym obszarze mapy.
Zawiera jego pierwsze zdjęcie, nazwę, ocenę w gwiazdkach, liczbę ocen oraz tagi.
Ocena zrealizowana jest za pomocą \texttt{Rate} z \glslink{biblioteka}{biblioteki} \texttt{antd}, a
tagi prezentowane są przez listę \texttt{SpotTag}.
Po kliknięciu elementu na obecnej instancji mapy zwracanej przez \glslink{hook}{hook} \texttt{useMap} (\texttt{react-maplibre})
wywoływana jest funkcja \texttt{flyTo}, przenosząca w płynnym przejściu obraz mapy na lokalizację \glslink{spot}{spota}.

\textbf{\texttt{SpotsSortingForm}}
\glslink{react-component}{Komponent} realizujący ustawianie sortowania listy \glslink{spot}{spotów}.
Do wyboru dostępne są opcje: Default, Rating ascending, Rating descending, Rating count ascending oraz Rating count descending,
zapisane jako tablica wartości w stałej \emph{options}.
Otworzenie i zamknięcie listy z tymi elementami animowane jest przy użyciu \glslink{react-component}{komponentu}
\texttt{AnimatePresence} z \glslink{biblioteka}{biblioteki} \texttt{framer-motion}.
W \glslink{props}{propsach} przyjmowane są funkcje obsługujące zmianę wyboru opcji oraz czyszczenia poprzednich wyników.
Przekazywana jest również tablica określająca klucz zapytań wymagających unieważnienia po zmianie sortowania, a także
opcja sortowania wybrana w danej liście.
W celu rozróżnienia gdzie używany jest \glslink{react-component}{komponent} przez \glslink{props}{propa} \emph{variant}
określany jest jego typ (\texttt{SEARCH} lub \texttt{CURRENT\_VIEW}).
Wybrana opcja przechowywana jest w lokalnym stanie (\texttt{useState}), \glslink{hook}{hook} \texttt{useEffect}
nasłuchuje na jej zmianę jeśli jest różna od opcji wybranej wcześniej (\glslink{props}{prop} \emph{sorting}), wykonuje
przekazaną funkcję do obsługi zmiany wyboru sortowania oraz funkcję czyszczącą.
Unieważanianie są również zapytania zawierające podany klucz.
\glslink{hook}{Hook} \texttt{useMemo} zwraca stałą będącą „stabilnym kluczem” \textendash \space po każdym odświeżeniu
\glslink{react-component}{komponentu} nadrzędnego tablica określająca klucz z zapytaniami wymagającymi unieważnienia jest
przekazywana ponownie.
\texttt{useMemo} porównuje jej wartości z poprzednią i jeśli są takie same, nie zwraca nowej wartości, która jest w
tablicy dependecji \texttt{useEffect} wspomnianego wcześniej.
Pozwala to na optymalizację poprzez uniknięcie niepotrzebnych uruchomień tego \glslink{hook}{hook'a}.
