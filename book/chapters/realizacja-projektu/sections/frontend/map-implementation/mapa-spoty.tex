%! Author = Stanisław Oziemczuk
%! Date = 22.12.2025

\subsubsection{Mapa z zaznaczonymi spotami}
\label{subsubsec:mapa-z-zaznaczonymi-spotami}

Mapa zbudowana jest z następujących \glslink{react-component}{komponentów}:
\begin{itemize}
    \item \texttt{MapPage}
    \item \texttt{Spots}
    \item \texttt{ZoomControlPanel}
    \item \texttt{ZoomControlButton}
    \item \texttt{UserLocationPanel}
    \item \texttt{SearchCurrentViewButton}
    \item \texttt{CurrentViewSpotsList}
    \item \texttt{CurrentViewSpotsFormsContainer}
    \item \texttt{SpotsSortingForm}
    \item \texttt{RatingFromForm}
    \item \texttt{SpotsNameSearchBar}
    \item \texttt{SearchedSpotsList}
    \item \texttt{ListedSpotInfo}
    \item \texttt{SpotTag}
\end{itemize}

\textbf{\texttt{MapPage}} (rys. \ref{fig:map-page-1} - \ref{fig:map-page-3})
To główny \glslink{react-component}{komponent} prezentujący mapę.
Zawiera w sobie \texttt{Map} z \glslink{biblioteka}{biblioteki} \texttt{react-maplibre}, w którym ustawia się domyślną
lokalizację, na którą ustawiony jest obraz mapy oraz poziom przybliżenia.
Przyjmuje on także inne \glslink{react-component}{komponenty}, które mają być w nim wyświetlone.
Ustalenie, które z nich mają być widoczne realizowane jest poprzez odczytanie wartości ze stanu \glslink{redux}{Redux'a}.
Mapa ładowana jest z linków ustawionych w pliku \texttt{map\_style.json}, w nim również opisany jest jej wygląd.
\glslink{hook}{Hook} \texttt{useEffect} uruchamia się tylko przy pierwszej inicjalizacji \glslink{react-component}{komponentu}
i sprawdza czy w \glslink{url}{url'u} umieszczone są parametry informujące, że link pochodzi z udostępnienia
\glslink{spot}{spota}.
Jeśli tak jest, odczytywane są odpowiednie dane, obraz mapy przekierowany jest na wskazaną lokalizację oraz
otwierane są panel ze szczegółami \glslink{spot}{spota} (por. sekcja~\ref{subsubsec:panel-ze-szczegolami-spota})
i jego pogodą (por. sekcja~\ref{subsubsec:panel-z-informacjami-pogodowymi}).

\begin{figure}[H]
    \centering
    \includegraphics[width=1\textwidth]{attachments/implementacja-frontendu/mapa/mapa-spoty/map_page1}
    \caption{Implementacja komponentu MapPage (1/3)}
    \label{fig:map-page-1}
\end{figure}
\noindent

\begin{figure}[H]
    \centering
    \includegraphics[width=1\textwidth]{attachments/implementacja-frontendu/mapa/mapa-spoty/map_page2}
    \caption{Implementacja komponentu MapPage (2/3)}
    \label{fig:map-page-2}
\end{figure}
\noindent

\begin{figure}[H]
    \centering
    \includegraphics[width=1\textwidth]{attachments/implementacja-frontendu/mapa/mapa-spoty/map_page3}
    \caption{Implementacja komponentu MapPage (3/3)}
    \label{fig:map-page-3}
\end{figure}
\noindent

\textbf{\texttt{Spots}} (rys. \ref{fig:spots-1} - \ref{fig:spots-4})
\glslink{react-component}{Komponent} ten odpowiedzialny jest za nałożenie \glslink{spot}{spotów} na mapę.
Z \glslink{backend}{backendu} za pomocą \glslink{hook}{hook'a} \texttt{useQuery} pobierana jest ich lista, a następnie
zgodnie z wartością zwróconą przez funkcję \texttt{shouldRenderMarker},
przyjmującej pole powierzchni elementu oraz poziom przybliżenia mapy, \glslink{spot}{spot} wyświetlany jest w formie
punktu (\texttt{Marker}) lub wielokąta (\texttt{Source} oraz \texttt{Layer}).
Wymienione \glslink{react-component}{komponenty} pochodzą z \glslink{biblioteka}{biblioteki} \texttt{react-maplibre}.
Zawiera ona również \glslink{hook}{hook} \texttt{useMap}, który zwraca obecną instancję mapy.
Pozwala to dodać do wyświetlanych elementów event-listener'ów obsługujących najechanie na dany element oraz funkcji
wywoływanej po jego kliknięciu.
Jest to funkcja \texttt{handleSpotClick}.
Powoduje ustawienie w stanie \glslink{redux}{Redux'a} danych \glslink{spot}{spota} oraz otworzenie
panelu z jego szczegółami (por. sekcja~\ref{subsubsec:panel-ze-szczegolami-spota})
i pogodą (por. sekcja~\ref{subsubsec:panel-z-informacjami-pogodowymi}).
Za pomocą \glslink{mutacja}{mutacji} zostaje również wysłane żądanie zwiększające liczbę wyświetleń elementu.
Informacja o błędzie podczas pobierana danych wyświetlana jest w \glslink{react-component}{komponencie}
\texttt{Notification}.

\begin{figure}[H]
    \centering
    \includegraphics[width=1\textwidth]{attachments/implementacja-frontendu/mapa/mapa-spoty/spots1}
    \caption{Implementacja komponentu Spots (1/4)}
    \label{fig:spots-1}
\end{figure}
\noindent

\begin{figure}[H]
    \centering
    \includegraphics[width=1\textwidth]{attachments/implementacja-frontendu/mapa/mapa-spoty/spots2}
    \caption{Implementacja komponentu Spots (2/4)}
    \label{fig:spots-2}
\end{figure}
\noindent

\begin{figure}[H]
    \centering
    \includegraphics[width=1\textwidth]{attachments/implementacja-frontendu/mapa/mapa-spoty/spots3}
    \caption{Implementacja komponentu Spots (3/4)}
    \label{fig:spots-3}
\end{figure}
\noindent

\begin{figure}[H]
    \centering
    \includegraphics[width=1\textwidth]{attachments/implementacja-frontendu/mapa/mapa-spoty/spots4}
    \caption{Implementacja komponentu Spots (4/4)}
    \label{fig:spots-4}
\end{figure}
\noindent

\textbf{\texttt{ZoomControlPanel}}
\glslink{react-component}{Komponent} obsługujący ustawianie przybliżenia przybliżenia mapy.
Zawiera dwa przyciski wyświetlane przez \texttt{ZoomControlButton}, przyjmujące funkcje, które na instacji mapy
pobieranej z \glslink{hook}{hook'a} \texttt{useMap} (\glslink{biblioteka}{biblioteka} \texttt{react-maplibre}),
wywołują na niej odpowiednio wbudowane operacje \texttt{zoomIn()} lub \texttt{zoomOut()}.

\textbf{\texttt{UserLocationPanel}}
Odpowiada za wyświetlenie na mapie aktualnej lokalizacji użytkownika.
Po kliknięciu przycisku za pomocą \\ \texttt{navigator.geolocation.getCurrentPosition} pobiera jego aktualną pozycję, a
następnie zapisuje ją w lokalnym stanie \glslink{react-component}{komponentu} (\texttt{useState}).
Na ich podstawie na mapie wyświetlany jest \texttt{Marker} z \glslink{biblioteka}{biblioteki} \texttt{react-maplibre}.
Pochodzący z niej \glslink{hook}{hook} \texttt{useMap} zwraca aktualną instancję mapy, na której wywoływana jest
funkcja \texttt{flyTo} przenosząca płynnie obraz na wskazane koordynaty.
Informacje o błędach podczas operacji wyświetlane są w \texttt{Notification}.

\textbf{\texttt{SearchCurrentViewButton}}

\textbf{\texttt{CurrentViewSpotsList}}

\textbf{\texttt{RatingFromForm}}

\textbf{\texttt{SpotsNameSearchBar}}

\textbf{\texttt{SearchedSpotsList}}

\subsubsubsection{Komponenty wspólne}
\label{subsubsec:komponenty-wspolne-map-spots}

W niniejszym rozdziale opisano \glslink{react-component}{komponenty}, z których korzystają inne \glslink{react-component}{komponenty}
zawarte w tej części.

\textbf{\texttt{ListedSpotInfo}}
Wyświetla pojedycznego \glslink{spot}{spota} będącego wynikiem wyszukiwania po nazwie lub w widocznym obszarze mapy.
Zawiera jego pierwsze zdjęcie, nazwę, ocenę w gwiazdkach, liczbę ocen oraz tagi.
Ocena zrealizowana jest za pomocą \texttt{Rate} z \glslink{biblioteka}{biblioteki} \texttt{antd}, a
tagi prezentowane są przez listę \texttt{SpotTag}.
Po kliknięciu elementu na obecnej instancji mapy zwracanej przez \glslink{hook}{hook} \texttt{useMap} (\texttt{react-maplibre})
wywoływana jest funkcja \texttt{flyTo}, przenosząca w płynnym przejściu obraz mapy na lokalizację \glslink{spot}{spota}.

\textbf{\texttt{SpotsSortingForm}}
\glslink{react-component}{Komponent} realizujący ustawianie sortowania listy \glslink{spot}{spotów}.
Do wyboru dostępne są opcje: Default, Rating ascending, Rating descending, Rating count ascending oraz Rating count descending,
zapisane jako tablica wartości w stałej \emph{options}.
Otworzenie i zamknięcie listy z tymi elementami animowane jest przy użyciu \glslink{react-component}{komponentu}
\texttt{AnimatePresence} z \glslink{biblioteka}{biblioteki} \texttt{framer-motion}.
W \glslink{props}{propsach} przyjmowane są funkcje obsługujące zmianę wyboru opcji oraz czyszczenia poprzednich wyników.
Przekazywana jest również tablica określająca klucz zapytań wymagających unieważnienia po zmianie sortowania, a także
opcja sortowania wybrana w danej liście.
W celu rozróżnienia gdzie używany jest \glslink{react-component}{komponent} przez \glslink{props}{propa} \emph{variant}
określany jest jego typ (\texttt{SEARCH} lub \texttt{CURRENT\_VIEW}).
Wybrana opcja przechowywana jest w lokalnym stanie (\texttt{useState}), \glslink{hook}{hook} \texttt{useEffect}
nasłuchuje na jej zmianę jeśli jest różna od opcji wybranej wcześniej (\glslink{props}{prop} \emph{sorting}), wykonuje
przekazaną funkcję do obsługi zmiany wyboru sortowania oraz funkcję czyszczącą.
Unieważanianie są również zapytania zawierające podany klucz.
\glslink{hook}{Hook} \texttt{useMemo} zwraca stałą będącą „stabilnym kluczem” \textendash \space po każdym odświeżeniu
\glslink{react-component}{komponentu} nadrzędnego tablica określająca klucz z zapytaniami wymagającymi unieważnienia jest
przekazywana ponownie.
\texttt{useMemo} porównuje jej wartości z poprzednią i jeśli są takie same, nie zwraca nowej wartości, która jest w
tablicy dependecji \texttt{useEffect} wspomnianego wcześniej.
Pozwala to na optymalizację poprzez uniknięcie niepotrzebnych uruchomień tego \glslink{hook}{hook'a}.
