%! Author = Stanisław Oziemczuk
%! Date = 22.12.2025

\subsubsection{Panel ze szczegółami spota}
\label{subsubsec:panel-ze-szczegolami-spota}

Panel ze szczegółami \glslink{spot}{spota} zbudowany jest z czterech głównych części:
\begin{itemize}
    \item informacji o \glslink{spot}{spocie}
    \item galerii zdjęć i filmów
    \item przycisków akcji
    \item komentarzy \glslink{spot}{spota}
\end{itemize}

\textbf{\texttt{SpotDetails}} (rys. \ref{fig:spot-details-1} - \ref{fig:spot-details-3})
To główny \glslink{react-component}{komponent}, który ustawia ułożenie wyżej wymienionych elementów.
Odpowiedzialny jest za pobranie danych o wybranym \glslink{spot}{spoice}.
W trakcie ich ładowania wyświetlany jest \texttt{LoadingSpinner}, a informacja o błędzie w
\texttt{Notification}.
Przejścia otwarcia oraz zamknięcia panelu animowane są za pomocą \glslink{biblioteka}{biblioteki} \texttt{motion}.

\begin{figure}[H]
    \centering
    \includegraphics[width=1\textwidth]{attachments/implementacja-frontendu/mapa/szczegoly-spota/spot_details1}
    \caption{Implementacja komponentu SpotDetails (1/3)}
    \label{fig:spot-details-1}
\end{figure}
\noindent

\begin{figure}[H]
    \centering
    \includegraphics[width=1\textwidth]{attachments/implementacja-frontendu/mapa/szczegoly-spota/spot_details2}
    \caption{Implementacja komponentu SpotDetails (2/3)}
    \label{fig:spot-details-2}
\end{figure}
\noindent

\begin{figure}[H]
    \centering
    \includegraphics[width=1\textwidth]{attachments/implementacja-frontendu/mapa/szczegoly-spota/spot_details3}
    \caption{Implementacja komponentu SpotDetails (3/3)}
    \label{fig:spot-details-3}
\end{figure}
\noindent