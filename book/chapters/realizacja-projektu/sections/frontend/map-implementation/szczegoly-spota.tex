%! Author = Stanisław Oziemczuk
%! Date = 22.12.2025

\subsubsection{Panel ze szczegółami spota}
\label{subsubsec:panel-ze-szczegolami-spota}

Panel ze szczegółami \glslink{spot}{spota} zbudowany jest z czterech głównych części:
\begin{itemize}
    \item informacji o \glslink{spot}{spocie}
    \item galerii zdjęć i filmów
    \item przycisków akcji
    \item komentarzy \glslink{spot}{spota}
\end{itemize}

\textbf{\texttt{SpotDetails}} (rys. \ref{fig:spot-details-1} - \ref{fig:spot-details-3})
To główny \glslink{react-component}{komponent}, który ustawia ułożenie wyżej wymienionych elementów.
Odpowiedzialny jest za pobranie danych o wybranym \glslink{spot}{spoice}.
W trakcie ich ładowania wyświetlany jest \texttt{LoadingSpinner}, a informacja o błędzie w
\texttt{Notification}.
Przejścia otwarcia oraz zamknięcia panelu animowane są za pomocą \glslink{biblioteka}{biblioteki} \texttt{motion}.

\begin{figure}[H]
    \centering
    \includegraphics[width=1\textwidth]{attachments/implementacja-frontendu/mapa/szczegoly-spota/spot_details1}
    \caption{Implementacja komponentu SpotDetails (1/3)}
    \label{fig:spot-details-1}
\end{figure}
\noindent

\begin{figure}[H]
    \centering
    \includegraphics[width=1\textwidth]{attachments/implementacja-frontendu/mapa/szczegoly-spota/spot_details2}
    \caption{Implementacja komponentu SpotDetails (2/3)}
    \label{fig:spot-details-2}
\end{figure}
\noindent

\begin{figure}[H]
    \centering
    \includegraphics[width=1\textwidth]{attachments/implementacja-frontendu/mapa/szczegoly-spota/spot_details3}
    \caption{Implementacja komponentu SpotDetails (3/3)}
    \label{fig:spot-details-3}
\end{figure}
\noindent

\subsubsubsection{Informacje o spocie}
\label{subsubsec:informacje-o-spocie}

Sekcja ta składa się z trzech \glslink{react-component}{komponentów}:
\begin{itemize}
    \item \texttt{SpotAddressInfo}
    \item \texttt{SpotGeneralInfo}
    \item \texttt{SpotTag}
\end{itemize}

\textbf{\texttt{SpotAddressInfo}}
Wyświetla informacje o lokalizacji \glslink{spot}{spota}: państwo, miasto oraz nazwę ulicy.


\textbf{\texttt{SpotGeneralInfo}} (rys. \ref{fig:spot-general-info-1} i \ref{fig:spot-general-info-2})
\glslink{react-component}{Komponent} odpowiedzialny za wyświetlanie danych takich jak: nazwa \glslink{spot}{spota},
opis, ocena, liczba ocen, lista tagów, które przyjmuje w \glslink{props}{propsach}.
Do prezentacji oceny w formie gwiazdek użyto \texttt{Rate} z \glslink{biblioteka}{biblioteki} \texttt{antd},
a jego konfiguracja ustawiana jest w \texttt{ConfigProvider}.
\glslink{responsywnosc}{Responywność} tego \glslink{react-component}{komponentu} realizowana jest poprzez
ustawianie wielkości gwiazdek.
Własny \glslink{hook}{hook} \texttt{useScreenSize} podaje aktualne rozmiary ekranu, a
\texttt{useEffect} reaguje na zmiany i za pomocą funkcji \texttt{calculateRateStarSize} ustawia
odpowiednią wartość przechowywaną w lokalnym stanie \glslink{react-component}{komponentu}.
Tagi wyświetlane są przez listę \texttt{SpotTag}.

\begin{figure}[H]
    \centering
    \includegraphics[width=1\textwidth]{attachments/implementacja-frontendu/mapa/szczegoly-spota/spot_general_info1}
    \caption{Implementacja komponentu SpotGeneralInfo (1/2)}
    \label{fig:spot-general-info-1}
\end{figure}
\noindent

\begin{figure}[H]
    \centering
    \includegraphics[width=1\textwidth]{attachments/implementacja-frontendu/mapa/szczegoly-spota/spot_general_info2}
    \caption{Implementacja komponentu SpotGeneralInfo (2/2)}
    \label{fig:spot-general-info-2}
\end{figure}
\noindent

\subsubsubsection{Galeria zdjęć i filmów}
\label{subsubsec:galeria-zdjec-i-filmow}

Zbudowana jest z następujących \glslink{react-component}{komponentów}:
\begin{itemize}
    \item \texttt{SpotDetailsGallery}
    \item \texttt{Photo}
    \item \texttt{Video}
\end{itemize}

\textbf{\texttt{SpotDetailsGallery}} (rys. \ref{fig:spot-details-gallery-1} i \ref{fig:spot-details-gallery-2})
\glslink{react-component}{Komponent} odpowiedzialny za wyświetlanie galerii zdjęć i filmów \glslink{spot}{spota}.
Galeria w formie karuzeli została zrealizowana poprzez \texttt{Carousel}, który jest elementem
\glslink{biblioteka}{biblioteki} \texttt{antd}.
Poprzez \glslink{react-component}{komponent} konfiguracyjny (\texttt{ConfigProvider}) ustawiane jest położenie
strzałek przełączających media.
Za wyświetlanie odpowiednio zdjęć i filmów odpowiadają \texttt{Photo} i \texttt{Video}.
W lokalnym stanie (\texttt{useState}) ustawiany jest obecny indeks elementu, poprzez porównanie go z
indekem właściwym elementu ustalana jest jedna ze składowych określających czy video powinno być odtworzone.
Pozwala to na automatyczne zatrzymanie filmu po zmianie media.

\begin{figure}[H]
    \centering
    \includegraphics[width=1\textwidth]{attachments/implementacja-frontendu/mapa/szczegoly-spota/spot_details_gallery1}
    \caption{Implementacja komponentu SpotDetailsGallery (1/2)}
    \label{fig:spot-details-gallery-1}
\end{figure}
\noindent

\begin{figure}[H]
    \centering
    \includegraphics[width=1\textwidth]{attachments/implementacja-frontendu/mapa/szczegoly-spota/spot_details_gallery2}
    \caption{Implementacja komponentu SpotDetailsGallery (2/2)}
    \label{fig:spot-details-gallery-2}
\end{figure}
\noindent

\textbf{\texttt{Photo}}
Odpowiada za wyświetlenie zdjęcia oraz obsługę jego kliknięcia.
Funkcja \texttt{handleClickPhoto} poprzez \glslink{mutacja}{mutację} wysyła na \glslink{backend}{backend}
żądanie ustalające pozycję zdjęcia w liście wszystkich zdjęć \glslink{spot}{spota} z uwzględnieniem sortowania.
\glslink{hook}{Hook} \texttt{useEffect} reaguje na zmianę danych i ustawia w stanie \glslink{redux}{Redux'a}
id zdjęcia, jego pozycję w całej liście, typ media na \emph{PHOTO} oraz otwiera \glslink{modal}{modal}
\\ \texttt{ExpandedSpotMediaGallery} (por. sekcja~\ref{subsubsec:duza-galeria-zdjec-i-filmow}).

\textbf{\texttt{Video}}
Obsługuje wyświetlanie oraz odtwarzanie filmów.
Operacja zrealizowana jest za pomocą \glslink{react-component}{komponentu} \texttt{ReactPlayer} z
\glslink{biblioteka}{biblioteki} \texttt{react-player}.
Panel sterujący filmem (czas trwania wraz z możliwością przesuwania, powiększenie, ustawienie głośności,
przyciski do odtwarzania i puzowania) zbudowany został z elementów \glslink{biblioteka}{biblioteki}
\texttt{media-chrome}.

\subsubsubsection{Przyciski akcji}
\label{subsubsec:przyciski-akcji}

Sekcja składa się z dwóch komponentów:
\begin{itemize}
    \item \texttt{SpotActionButtonsContainer}
    \item \texttt{SpotActionButton}
\end{itemize}

\subsubsubsection{Komentarze spota}
\label{subsubsec:komentarze-spota}
