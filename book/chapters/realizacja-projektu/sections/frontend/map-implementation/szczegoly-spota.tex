%! Author = Stanisław Oziemczuk
%! Date = 22.12.2025

\subsubsection{Panel ze szczegółami spota}
\label{subsubsec:panel-ze-szczegolami-spota}

Panel ze szczegółami \glslink{spot}{spota} zbudowany jest z czterech głównych części:
\begin{itemize}
    \item informacji o \glslink{spot}{spocie}
    \item galerii zdjęć i filmów
    \item przycisków akcji
    \item komentarzy \glslink{spot}{spota}
\end{itemize}

\textbf{\texttt{SpotDetails}} (rys. \ref{fig:spot-details-1} - \ref{fig:spot-details-3})
To główny \glslink{react-component}{komponent}, który ustawia ułożenie wyżej wymienionych elementów.
Odpowiedzialny jest za pobranie danych o wybranym \glslink{spot}{spocie}.
W trakcie ich ładowania wyświetlany jest \texttt{LoadingSpinner}, a informacja o błędzie w
\texttt{Notification}.
Przejścia otwarcia oraz zamknięcia panelu animowane są za pomocą \glslink{biblioteka}{biblioteki} \texttt{motion}.

\begin{figure}[H]
    \centering
    \includegraphics[width=1\textwidth]{attachments/implementacja-frontendu/mapa/szczegoly-spota/spot_details1}
    \caption{Implementacja komponentu SpotDetails (1/3)}
    \label{fig:spot-details-1}
\end{figure}
\noindent

\begin{figure}[H]
    \centering
    \includegraphics[width=1\textwidth]{attachments/implementacja-frontendu/mapa/szczegoly-spota/spot_details2}
    \caption{Implementacja komponentu SpotDetails (2/3)}
    \label{fig:spot-details-2}
\end{figure}
\noindent

\begin{figure}[H]
    \centering
    \includegraphics[width=1\textwidth]{attachments/implementacja-frontendu/mapa/szczegoly-spota/spot_details3}
    \caption{Implementacja komponentu SpotDetails (3/3)}
    \label{fig:spot-details-3}
\end{figure}
\noindent

\subsubsubsection{Informacje o spocie}
\label{subsubsec:informacje-o-spocie}

Sekcja ta składa się z trzech \glslink{react-component}{komponentów}:
\begin{itemize}
    \item \texttt{SpotAddressInfo}
    \item \texttt{SpotGeneralInfo}
    \item \texttt{SpotTag}
\end{itemize}

\textbf{\texttt{SpotAddressInfo}}
Wyświetla informacje o lokalizacji \glslink{spot}{spota}: państwo, miasto oraz nazwę ulicy.


\textbf{\texttt{SpotGeneralInfo}} (rys. \ref{fig:spot-general-info-1} i \ref{fig:spot-general-info-2})
\glslink{react-component}{Komponent} odpowiedzialny za wyświetlanie danych takich jak: nazwa \glslink{spot}{spota},
opis, ocena, liczba ocen, lista tagów, które przyjmuje w \glslink{props}{propsach}.
Do prezentacji oceny w formie gwiazdek użyto \texttt{Rate} z \glslink{biblioteka}{biblioteki} \texttt{antd},
a jego konfiguracja ustawiana jest w \texttt{ConfigProvider}.
\glslink{responsywnosc}{Responywność} tego \glslink{react-component}{komponentu} realizowana jest poprzez
ustawianie wielkości gwiazdek.
Własny \glslink{hook}{hook} \texttt{useScreenSize} podaje aktualne rozmiary ekranu, a
\texttt{useEffect} reaguje na zmiany i za pomocą funkcji \texttt{calculateRateStarSize} ustawia
odpowiednią wartość przechowywaną w lokalnym stanie \glslink{react-component}{komponentu}.
Tagi wyświetlane są przez listę \texttt{SpotTag}.

\begin{figure}[H]
    \centering
    \includegraphics[width=1\textwidth]{attachments/implementacja-frontendu/mapa/szczegoly-spota/spot_general_info1}
    \caption{Implementacja komponentu SpotGeneralInfo (1/2)}
    \label{fig:spot-general-info-1}
\end{figure}
\noindent

\begin{figure}[H]
    \centering
    \includegraphics[width=1\textwidth]{attachments/implementacja-frontendu/mapa/szczegoly-spota/spot_general_info2}
    \caption{Implementacja komponentu SpotGeneralInfo (2/2)}
    \label{fig:spot-general-info-2}
\end{figure}
\noindent

\subsubsubsection{Galeria zdjęć i filmów}
\label{subsubsec:galeria-zdjec-i-filmow}

Zbudowana jest z następujących \glslink{react-component}{komponentów}:
\begin{itemize}
    \item \texttt{SpotDetailsGallery}
    \item \texttt{Photo}
    \item \texttt{Video}
\end{itemize}

\textbf{\texttt{SpotDetailsGallery}} (rys. \ref{fig:spot-details-gallery-1} i \ref{fig:spot-details-gallery-2})
\glslink{react-component}{Komponent} odpowiedzialny za wyświetlanie galerii zdjęć i filmów \glslink{spot}{spota}.
Galeria w formie karuzeli została zrealizowana poprzez \texttt{Carousel}, który jest elementem
\glslink{biblioteka}{biblioteki} \texttt{antd}.
Poprzez \glslink{react-component}{komponent} konfiguracyjny \\ (\texttt{ConfigProvider}) ustawiane jest położenie
strzałek przełączających media.
Za wyświetlanie odpowiednio zdjęć i filmów odpowiadają \texttt{Photo} i \texttt{Video}.
W lokalnym stanie (\texttt{useState}) ustawiany jest obecny indeks elementu, poprzez porównanie go z
indekem właściwym elementu ustalana jest jedna ze składowych określających czy video powinno być odtworzone.
Pozwala to na automatyczne zatrzymanie filmu po zmianie media.

\begin{figure}[H]
    \centering
    \includegraphics[width=1\textwidth]{attachments/implementacja-frontendu/mapa/szczegoly-spota/spot_details_gallery1}
    \caption{Implementacja komponentu SpotDetailsGallery (1/2)}
    \label{fig:spot-details-gallery-1}
\end{figure}
\noindent

\begin{figure}[H]
    \centering
    \includegraphics[width=1\textwidth]{attachments/implementacja-frontendu/mapa/szczegoly-spota/spot_details_gallery2}
    \caption{Implementacja komponentu SpotDetailsGallery (2/2)}
    \label{fig:spot-details-gallery-2}
\end{figure}
\noindent

\textbf{\texttt{Photo}}
Odpowiada za wyświetlenie zdjęcia oraz obsługę jego kliknięcia.
Funkcja \\ \texttt{handleClickPhoto} poprzez \glslink{mutacja}{mutację} wysyła na \glslink{backend}{backend}
żądanie ustalające pozycję zdjęcia w liście wszystkich zdjęć \glslink{spot}{spota} z uwzględnieniem sortowania.
\glslink{hook}{Hook} \texttt{useEffect} reaguje na zmianę danych i ustawia w stanie \glslink{redux}{Redux'a}
id zdjęcia, jego pozycję w całej liście, typ media na \emph{PHOTO} oraz otwiera \glslink{modal}{modal}
\\ \texttt{ExpandedSpotMediaGallery} (por. sekcja~\ref{subsubsec:duza-galeria-zdjec-i-filmow}).

\textbf{\texttt{Video}}
Obsługuje wyświetlanie oraz odtwarzanie filmów.
Operacja zrealizowana jest za pomocą \glslink{react-component}{komponentu} \texttt{ReactPlayer} z
\glslink{biblioteka}{biblioteki} \texttt{react-player}.
Panel sterujący filmem (czas trwania wraz z możliwością przesuwania, powiększenie, ustawienie głośności,
przyciski do odtwarzania i puzowania) zbudowany został z elementów \glslink{biblioteka}{biblioteki}
\texttt{media-chrome}.

\subsubsubsection{Przyciski akcji}
\label{subsubsec:przyciski-akcji}

Sekcja składa się z trzech \glslink{react-component}{komponentów}:
\begin{itemize}
    \item \texttt{SpotActionButtonsContainer}
    \item \texttt{SpotActionButton}
    \item \texttt{SpotAddMediaModal}
\end{itemize}

\textbf{\texttt{SpotActionButtonsContainer}}
Zawiera w sobie logikę obsługującą akcje możliwe do wykonania na danym \glslink{spot}{spocie}.
Przyciski wyświetlane są poprzez \texttt{SpotActionButton}, którym poszczególne funkcje przekazywane są
przez \glslink{props}{propsy}.
Nawigacja do wybranego \glslink{spot}{spota} zrealizowana jest poprzez pobranie lokalizacji użytkownika
\\ (\texttt{navigator.geolocation.getCurrentPosition}), a następnie przeniesienie na \\ \texttt{google maps}
z wyznaczoną trasą.
Funkcja \texttt{clickShareSpotHandler} obsługuje udostępnienie \glslink{spot}{spota} przez
skopiowanie odpowiedniego linka do pamięci podręcznej systemu korzystając z \texttt{navigator.clipboard}.
Dane zapisane w \glslink{query-params}{query params} pozwalają po wklejeniu linka na otworzenie
panelu ze szczegółami \glslink{spot}{spota} oraz przybliżenie jego lokalizacji na mapie.
Przycisk z ikoną \texttt{MdOutlineAddPhotoAlternate} sprawdza czy użytkownik jest zalogowany i w
\glslink{react-component}{komponencie} \texttt{Notification} wyświetla odpowiedni komunikat.
Gdy ten warunek jest spełniony, otwierany jest \glslink{modal}{modal} \texttt{SpotAddMediaModal}, zwierający
formularz do dodawania media do \glslink{spot}{spota}.
Poprzez \glslink{hook}{hook} \texttt{useQuery} wysyłane jest zapytanie sprawdzające czy wybrany \glslink{spot}{spot}
znajduje się w liscię polubionych użytkownika.
Od wyniku operacji uzależniony jest wygląd przycisku oraz typ wysyłanego żądania po jego kliknięciu
(wysyłanego \glslink{mutacja}{mutacją}).
Wynik operacji dodania lub usunięcia \glslink{spot}{spota} z listy ulubionych pokazywany jest w
\glslink{react-component}{komponencie} \texttt{Notification.}
Zapytanie dotyczące listy ulubionych \glslink{spot}{spotów} użytownika jest unieważniane w celu
wymuszenia ponownego pobrania danych i ich aktualizacji.

\textbf{\texttt{SpotAddMediaModal}}
\glslink{react-component}{Komponent} zawierający formularz do dodawania mediów do \glslink{spot}{spota}.
Pozwala na wybór plików (zdjęć oraz filmów), które wraz z podglądem wyświetlane są w formie listy.
Kliknięcie elementu na liście powoduje jego usunięcie.
Dane przysłane są \glslink{mutacja}{mutacją} na \glslink{backend}{backend}, a zapytanie dotyczące
szczegółów \glslink{spot}{spota} są unieważnianie, dzięki czemu dodane media będą znajdywać się w
galerii zdjęć i filmów.
Komunikat o wyniku operacji wyświetlany jest w \texttt{Notification}.
Otworzenie oraz zamknięcie elementu animowane jest \glslink{biblioteka}{biblioteką} \texttt{motion}.

\subsubsubsection{Komentarze spota}
\label{subsubsec:komentarze-spota}

Część panelu szczegółów \glslink{spot}{spota}, w której znajdują się zarówno komentarze, jak i \glslink{modal}{modal}
do ich dodawania.
Zbudowana jest z następujących \glslink{react-component}{komponentów}:
\begin{itemize}
    \item \texttt{SpotCommentsList}
    \item \texttt{SpotCommentHeader}
    \item \texttt{AddSpotCommentModal}
    \item \texttt{SpotComment}
    \item \texttt{SpotCommentAuthor}
    \item \texttt{SpotCommentMediaGallery}
    \item \texttt{SpotCommentVotesPanel}
    \item \texttt{SpotCommentVoteDisplay}
\end{itemize}

\textbf{\texttt{SpotCommentsList}}
\glslink{react-component}{Komponent} wyświetlający listę komentarzy przewijalnej przy użyciu
\glslink{infinite-scroll}{infinite scroll}.
Mechanizm ten jest realizowany za pomocą \texttt{IntersectionObserver}, który nasłuchuje na element powiązany
referencją i wywołuje funkcję \texttt{fetchNextPage} z \glslink{hook}{hook'a} \texttt{useInfiniteQuery}
(\glslink{biblioteka}{biblioteka} \texttt{tanstack query}).
Na początku pobierana jest tylko pierwsza strona komentarzy oraz wyświetlany jest przycik \texttt{Show More}.
Po jego kliknięciu jest ukrywany i wysyłane jest zapytanie o drugą stronę.
Pobieranie kolejnych odbywa się w sposób opisany powyżej.
W trakcie ładowania wyświetlany jest \texttt{LoadingSpinner}, a w razie błędu komunikat
\emph{Failed to load comments.}.

\textbf{\texttt{SpotCommentHeader}}
Zawiera w sobie nagłówek „Comments” oraz przcisk otwierający \glslink{modal}{modal} będący formularzem
do dodania nowego komentarza (\texttt{AddSpotCommentModal}).
Po jego kliknięciu jeżeli użytkownik nie jest zalogowany, w \glslink{react-component}{komponencie}
\texttt{Notification} wyswietlany jest komunikat z odpowiednią informacją.

\textbf{\texttt{AddSpotCommentModal}} (rys. \ref{fig:add-spot-comment-1} - \ref{fig:add-spot-comment-6})
\glslink{modal}{Modal} będący formularzem do dodania nowego komentarza.
Zawiera w sobie pola do określenia oceny w gwiazdkach, dodania tekstu oraz zdjęć i filmów.
Ustawienie oceny \glslink{spot}{spota} jest polem obowiązkowym z domyślną wartością 0.
Zrealizowane jest przy użyciu \glslink{react-component}{komponentu} \texttt{Rate} skonfigurowanego
przez \texttt{ConfigProvider}.
Oba elementy pochodzą z \glslink{biblioteka}{biblioteki} \texttt{antd}.
Innym obowiązkowym polem jest pole tekstowe, w którym użytkownik umieszcza swoją opinię.
Wartości wpisane przechodzą proces walidacji przez \texttt{addSpotCommentSchema} i funkcje
\glslink{biblioteka}{biblioteki} \texttt{zod}.
Informacje o błędach wyświetlane są w \texttt{Notification}.
Dodawanie zdjęć oraz filmów jest opcjonalne i zaimplementowane zostało za pomocą \glslink{react-component}{komponentu}
\texttt{UploadButton}, który otwiera eksplorator plików i umożliwia wybranie odpowiednich plików.
Wskazane elementy wyświetlane są w formie listy, a ich kliknięcie usuwa je z niej.
Możliwe jest dodanie maksymalnie dwudziestu elementów, o czym użytkownik jest informowany odpowiednim
komunikatem.
Wartości wpisane w poszczególne pola formularza przechowywane są w lokalnym stanie
\glslink{react-component}{komponentu}.
Wysłanie danych do \glslink{backend}{backendu} odbywa się poprzez \glslink{mutacja}{mutację}, a w oczekiwaniu na
odpowiedź wyświetlany jest \texttt{LoadingSpinner}.
Pomyślne zakończenie operacji skutkuje unieważnieniem zapytań powiązanych z danym \glslink{spot}{spotem}
oraz zamknięciem \glslink{modal}{modalu}.
Kliknięcie przycisku „Cancel” powoduje wywołanie funkcji \texttt{handleCancelAddSpotComment}, która
zamyka formularz, wpisane dane nie zostają zapisane.

\begin{figure}[H]
    \centering
    \includegraphics[width=1\textwidth]{attachments/implementacja-frontendu/mapa/szczegoly-spota/add-spot-comment-modal-1}
    \caption{Implementacja komponentu AddSpotCommentModal (1/6)}
    \label{fig:add-spot-comment-1}
\end{figure}
\noindent

\begin{figure}[H]
    \centering
    \includegraphics[width=1\textwidth]{attachments/implementacja-frontendu/mapa/szczegoly-spota/add-spot-comment-modal-2}
    \caption{Implementacja komponentu AddSpotCommentModal (2/6)}
    \label{fig:add-spot-comment-2}
\end{figure}
\noindent

\begin{figure}[H]
    \centering
    \includegraphics[width=1\textwidth]{attachments/implementacja-frontendu/mapa/szczegoly-spota/add-spot-comment-modal-3}
    \caption{Implementacja komponentu AddSpotCommentModal (3/6)}
    \label{fig:add-spot-comment-3}
\end{figure}
\noindent

\begin{figure}[H]
    \centering
    \includegraphics[width=1\textwidth]{attachments/implementacja-frontendu/mapa/szczegoly-spota/add-spot-comment-modal-4}
    \caption{Implementacja komponentu AddSpotCommentModal (4/6)}
    \label{fig:add-spot-comment-4}
\end{figure}
\noindent

\begin{figure}[H]
    \centering
    \includegraphics[width=1\textwidth]{attachments/implementacja-frontendu/mapa/szczegoly-spota/add-spot-comment-modal-5}
    \caption{Implementacja komponentu AddSpotCommentModal (5/6)}
    \label{fig:add-spot-comment-5}
\end{figure}
\noindent

\begin{figure}[H]
    \centering
    \includegraphics[width=1\textwidth]{attachments/implementacja-frontendu/mapa/szczegoly-spota/add-spot-comment-modal-6}
    \caption{Implementacja komponentu AddSpotCommentModal (6/6)}
    \label{fig:add-spot-comment-6}
\end{figure}
\noindent

\textbf{\texttt{SpotComment}}
\glslink{react-component}{Komponent} wyświetlający pojedynczy komentarz.
Zbudowany jest z wyspecjalizowanych \glslink{react-component}{komponentów} odpowiedzialnych za prezentację jego poszczególnych fragmentów.
\texttt{SpotCommentAuthor} zawiera informacje o autorze komentarza: nazwę oraz zdjęcie profilowe,
kliknięcie go powoduje przeniesienie na stronę profilu danego użytkownika.
Ocena komentarza zrealizowana jest poprzez \texttt{Rate} z \glslink{biblioteka}{biblioteki} \texttt{antd}.
Pozostałe dwa \glslink{react-component}{komponenty}, \texttt{SpotCommentMediaGallery}
oraz \newline \texttt{SpotCommentVotesPanel} zostały opisane poniżej.

\textbf{\texttt{SpotCommentMediaGallery}}
Wyświetla zdjęcia i filmy dodane do komentarza.
Pobieranie danych zrealizowane jest za pomocą \glslink{hook}{hook'a} \texttt{useQuery} z \glslink{biblioteka}{biblioteki}
\texttt{tanstack query}.
W czasie ich ładowania prezentowany jest \glslink{react-component}{komponent} \texttt{LoadingSpinner}, a o
niepowodzeniu operacji informuje komunikat \emph{Failed to download rest of the photos!}.
Pierwsze trzy elementy są widoczne od razu, jeżeli ich ogólna liczba jest większa
(informacja o tym jest przesyłana z \glslink{backend}{backendu}), to na trzeci z nich nakładany jest częściowo
przezroczysty przycisk z napisem „See more”.
Po jego kliknięciu funkcja \texttt{handleShowMoreMedia} powoduje wysłanie zapytania i wyświetlenie wszyskich multimediów.
Wybór elementu z listy powoduje wysłanie żądanie za pomocą \glslink{mutacja}{mutacji} ustalającej jego położenie
pośród wszyskich multimediów \glslink{spot}{spota}.
Na wynik operacji nasłuchuje \glslink{hook}{hook} \texttt{useEffect}, który po otrzymaniu danych w stanie
\glslink{redux}{Redux} ustawia \texttt{ID} elementu, jego pozycję w całej liście oraz typ (zdjęcie lub film).
Następnie otwiera dużą galerię multimediów \glslink{spot}{spota} (por. sekcja~\ref{subsubsec:duza-galeria-zdjec-i-filmow}).

\textbf{\texttt{SpotCommentVotesPanel}} (rys. \ref{fig:spot-comment-vote-panel-0} - \ref{fig:spot-comment-vote-panel-3})
\glslink{react-component}{Komponent} odpowiedzialny za wyświetlanie polubień i niepolubień komentarza oraz obsługę operacji
oddania głosu przez użytkownika.
Za pomocą \glslink{hook}{hook'a} \texttt{useQuery} (\glslink{biblioteka}{biblioteka} \texttt{tanstack query}) pobierana
jest informacja czy i jaki głos oddał użytkownik.
Na jej podstawie ustawiany jest wygląd odpowiedniej ikony.
Kliknięcie przycisku powoduje następujące operacje:
\begin{itemize}
    \item jeśli użytkownik polubił komentarz, ponowne kliknięcie przycisku usuwa je; analogicznie jest z niepolubieniami
    \item jeśli użytkownik klika przycisk przeciwny do wcześniejszego, poprzednia operacja jest cofana na rzecz wykonania nowej
    \item jeśli użytkownik nie oddał wcześniej głosu na komentarz, wykonywana jest wybrana akcja
\end{itemize}
Zagłosowanie wymaga zalogowania, o czym informuje komunikat w \glslink{react-component}{komponencie} \newline \texttt{Notification}.
W powyższych operacjach zastosowano optymistyczne aktualizowane \glslink{ui}{UI} (por. sekcja~\ref{subsubsec:Optimistic UI}) poprzez
przechowywanie liczby głosów w stanie lokalnym (\glslink{hook}{hook} \texttt{useState}) i zwiększanie lub pomniejszanie jej w
zależności od wykonanej akcji.
Żądanie na \glslink{backend}{backend} wysyłane jest za pomocą \glslink{mutacja}{mutacji}.
Po pozytywnej odpowiedzi unieważniane są wszystkie zapytania powiązane z danym \glslink{spot}{spotem} i komentarzem, co
gwarantuje pobranie aktualnych danych.
Natomiast jeśli wystąpi błąd, informacja o tym wyświetlana jest w \texttt{Notification}.
\glslink{react-component}{Komponent} \texttt{SpotCommentVoteDisplay} wyświetla liczbę oddanych głosów wraz z odpowiednią
ikoną.

\begin{figure}[H]
    \centering
    \includegraphics[width=1\textwidth]{attachments/implementacja-frontendu/mapa/szczegoly-spota/spot_comment_vote_panel0}
    \caption{Implementacja komponentu SpotCommentVotesPanel (1/4)}
    \label{fig:spot-comment-vote-panel-0}
\end{figure}
\noindent

\begin{figure}[H]
    \centering
    \includegraphics[width=1\textwidth]{attachments/implementacja-frontendu/mapa/szczegoly-spota/spot_comment_vote_panel1}
    \caption{Implementacja komponentu SpotCommentVotesPanel (2/4)}
    \label{fig:spot-comment-vote-panel-1}
\end{figure}
\noindent

\begin{figure}[H]
    \centering
    \includegraphics[width=1\textwidth]{attachments/implementacja-frontendu/mapa/szczegoly-spota/spot_comment_vote_panel2}
    \caption{Implementacja komponentu SpotCommentVotesPanel (3/4)}
    \label{fig:spot-comment-vote-panel-2}
\end{figure}
\noindent

\begin{figure}[H]
    \centering
    \includegraphics[width=1\textwidth]{attachments/implementacja-frontendu/mapa/szczegoly-spota/spot_comment_vote_panel3}
    \caption{Implementacja komponentu SpotCommentVotesPanel (4/4)}
    \label{fig:spot-comment-vote-panel-3}
\end{figure}
\noindent
