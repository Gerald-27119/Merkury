%! Author = Stanisław Oziemczuk
%! Date = 22.12.2025

\subsubsection{Panel ze szczegółami spota}
\label{subsubsec:panel-ze-szczegolami-spota}

Panel ze szczegółami \glslink{spot}{spota} zbudowany jest z czterech głównych części:
\begin{itemize}
    \item informacji o \glslink{spot}{spocie}
    \item galerii zdjęć i filmów
    \item przycisków akcji
    \item komentarzy \glslink{spot}{spota}
\end{itemize}

\textbf{\texttt{SpotDetails}} (rys. \ref{fig:spot-details-1} - \ref{fig:spot-details-3})
To główny \glslink{react-component}{komponent}, który ustawia ułożenie wyżej wymienionych elementów.
Odpowiedzialny jest za pobranie danych o wybranym \glslink{spot}{spoice}.
W trakcie ich ładowania wyświetlany jest \texttt{LoadingSpinner}, a informacja o błędzie w
\texttt{Notification}.
Przejścia otwarcia oraz zamknięcia panelu animowane są za pomocą \glslink{biblioteka}{biblioteki} \texttt{motion}.

\begin{figure}[H]
    \centering
    \includegraphics[width=1\textwidth]{attachments/implementacja-frontendu/mapa/szczegoly-spota/spot_details1}
    \caption{Implementacja komponentu SpotDetails (1/3)}
    \label{fig:spot-details-1}
\end{figure}
\noindent

\begin{figure}[H]
    \centering
    \includegraphics[width=1\textwidth]{attachments/implementacja-frontendu/mapa/szczegoly-spota/spot_details2}
    \caption{Implementacja komponentu SpotDetails (2/3)}
    \label{fig:spot-details-2}
\end{figure}
\noindent

\begin{figure}[H]
    \centering
    \includegraphics[width=1\textwidth]{attachments/implementacja-frontendu/mapa/szczegoly-spota/spot_details3}
    \caption{Implementacja komponentu SpotDetails (3/3)}
    \label{fig:spot-details-3}
\end{figure}
\noindent

\subsubsubsection{Informacje o spocie}
\label{subsubsec:informacje-o-spocie}

Sekcja ta składa się z trzech \glslink{react-component}{komponentów}:
\begin{itemize}
    \item \texttt{SpotAddressInfo}
    \item \texttt{SpotGeneralInfo}
    \item \texttt{SpotTag}
\end{itemize}

\textbf{\texttt{SpotAddressInfo}}
Wyświetla informacje o lokalizacji \glslink{spot}{spota}: państwo, miasto oraz nazwę ulicy.


\textbf{\texttt{SpotGeneralInfo}} (rys. \ref{fig:spot-general-info-1} i \ref{fig:spot-general-info-2})
\glslink{react-component}{Komponent} odpowiedzialny za wyświetlanie danych takich jak: nazwa \glslink{spot}{spota},
opis, ocena, liczba ocen, lista tagów, które przyjmuje w \glslink{props}{propsach}.
Do prezentacji oceny w formie gwiazdek użyto \texttt{Rate} z \glslink{biblioteka}{biblioteki} \texttt{antd},
a jego konfiguracja ustawiana jest w \texttt{ConfigProvider}.
\glslink{responsywnosc}{Responywność} tego \glslink{react-component}{komponentu} realizowana jest poprzez
ustawianie wielkości gwiazdek.
Własny \glslink{hook}{hook} \texttt{useScreenSize} podaje aktualne rozmiary ekranu, a
\texttt{useEffect} reaguje na zmiany i za pomocą funkcji \texttt{calculateRateStarSize} ustawia
odpowiednią wartość przechowywaną w lokalnym stanie \glslink{react-component}{komponentu}.
Tagi wyświetlane są przez listę \texttt{SpotTag}.

\begin{figure}[H]
    \centering
    \includegraphics[width=1\textwidth]{attachments/implementacja-frontendu/mapa/szczegoly-spota/spot_general_info1}
    \caption{Implementacja komponentu SpotGeneralInfo (1/2)}
    \label{fig:spot-general-info-1}
\end{figure}
\noindent

\begin{figure}[H]
    \centering
    \includegraphics[width=1\textwidth]{attachments/implementacja-frontendu/mapa/szczegoly-spota/spot_general_info2}
    \caption{Implementacja komponentu SpotGeneralInfo (2/2)}
    \label{fig:spot-general-info-2}
\end{figure}
\noindent

\subsubsubsection{Galeria zdjęć i filmów}
\label{subsubsec:galeria-zdjec-i-filmow}

Zbudowana jest z następujących \glslink{react-component}{komponentów}:
\begin{itemize}
    \item \texttt{SpotDetailsGallery}
    \item \texttt{Photo}
    \item \texttt{Video}
\end{itemize}

\textbf{\texttt{SpotDetailsGallery}} (rys. \ref{fig:spot-details-gallery-1} i \ref{fig:spot-details-gallery-2})
\glslink{react-component}{Komponent} odpowiedzialny za wyświetlanie galerii zdjęć i filmów \glslink{spot}{spota}.
Galeria w formie karuzeli została zrealizowana poprzez \texttt{Carousel}, który jest elementem
\glslink{biblioteka}{biblioteki} \texttt{antd}.
Poprzez \glslink{react-component}{komponent} konfiguracyjny (\texttt{ConfigProvider}) ustawiane jest położenie
strzałek przełączających media.
Za wyświetlanie odpowiednio zdjęć i filmów odpowiadają \texttt{Photo} i \texttt{Video}.
W lokalnym stanie (\texttt{useState}) ustawiany jest obecny indeks elementu, poprzez porównanie go z
indekem właściwym elementu ustalana jest jedna ze składowych określających czy video powinno być odtworzone.
Pozwala to na automatyczne zatrzymanie filmu po zmianie media.

\begin{figure}[H]
    \centering
    \includegraphics[width=1\textwidth]{attachments/implementacja-frontendu/mapa/szczegoly-spota/spot_details_gallery1}
    \caption{Implementacja komponentu SpotDetailsGallery (1/2)}
    \label{fig:spot-details-gallery-1}
\end{figure}
\noindent

\begin{figure}[H]
    \centering
    \includegraphics[width=1\textwidth]{attachments/implementacja-frontendu/mapa/szczegoly-spota/spot_details_gallery2}
    \caption{Implementacja komponentu SpotDetailsGallery (2/2)}
    \label{fig:spot-details-gallery-2}
\end{figure}
\noindent

\textbf{\texttt{Photo}}
Odpowiada za wyświetlenie zdjęcia oraz obsługę jego kliknięcia.
Funkcja \texttt{handleClickPhoto} poprzez \glslink{mutacja}{mutację} wysyła na \glslink{backend}{backend}
żądanie ustalające pozycję zdjęcia w liście wszystkich zdjęć \glslink{spot}{spota} z uwzględnieniem sortowania.
\glslink{hook}{Hook} \texttt{useEffect} reaguje na zmianę danych i ustawia w stanie \glslink{redux}{Redux'a}
id zdjęcia, jego pozycję w całej liście, typ media na \emph{PHOTO} oraz otwiera \glslink{modal}{modal}
\\ \texttt{ExpandedSpotMediaGallery} (por. sekcja~\ref{subsubsec:duza-galeria-zdjec-i-filmow}).

\textbf{\texttt{Video}}
Obsługuje wyświetlanie oraz odtwarzanie filmów.
Operacja zrealizowana jest za pomocą \glslink{react-component}{komponentu} \texttt{ReactPlayer} z
\glslink{biblioteka}{biblioteki} \texttt{react-player}.
Panel sterujący filmem (czas trwania wraz z możliwością przesuwania, powiększenie, ustawienie głośności,
przyciski do odtwarzania i puzowania) zbudowany został z elementów \glslink{biblioteka}{biblioteki}
\texttt{media-chrome}.

\subsubsubsection{Przyciski akcji}
\label{subsubsec:przyciski-akcji}

Sekcja składa się z trzech \glslink{react-component}{komponentów}:
\begin{itemize}
    \item \texttt{SpotActionButtonsContainer}
    \item \texttt{SpotActionButton}
    \item \texttt{SpotAddMediaModal}
\end{itemize}

\textbf{\texttt{SpotActionButtonsContainer}}
Zawiera w sobie logikę obsługującą akcje możliwe do wykonania na danym \glslink{spot}{spocie}.
Przyciski wyświetlane są poprzez \texttt{SpotActionButton}, którym poszczególne funkcje przekazywane są
przez \glslink{props}{propsy}.
Nawigacja do wybranego \glslink{spot}{spota} zrealizowana jest poprzez pobranie lokalizacji użytkownika
\\ (\texttt{navigator.geolocation.getCurrentPosition}), a następnie przeniesienie na \texttt{google maps}
z wyznaczoną trasą.
Funkcja \texttt{clickShareSpotHandler} obsługuje udostępnienie \glslink{spot}{spota} przez
skopiowanie odpowiedniego linka do pamięci podręcznej systemu korzystając z \texttt{navigator.clipboard}.
Dane zapisane w \glslink{query-params}{query params} pozwalają po wklejeniu linka na otworzenie
panelu ze szczegółami \glslink{spot}{spota} oraz przybliżenie jego lokalizacji na mapie.
Przycisk z ikoną \texttt{MdOutlineAddPhotoAlternate} sprawdza czy użytkownik jest zalogowany i w
\glslink{react-component}{komponencie} \texttt{Notification} wyświetla odpowiedni komunikat.
Gdy ten warunek jest spełniony, otwierany jest \glslink{modal}{modal} \texttt{SpotAddMediaModal}, zwierający
formularz do dodawania media do \glslink{spot}{spota}.
Poprzez \glslink{hook}{hook} \texttt{useQuery} wysyłane jest zapytanie sprawdzające czy wybrany \glslink{spot}{spot}
znajduje się w liscię polubionych użytkownika.
Od wyniku operacji uzależniony jest wygląd przycisku oraz typ wysyłanego żądania po jego kliknięciu
(wysyłanego \glslink{mutacja}{mutacją}).
Wynik operacji dodania lub usunięcia \glslink{spot}{spota} z listy ulubionych pokazywany jest w
\glslink{react-component}{komponencie} \texttt{Notification.}
Zapytanie dotyczące listy ulubionych \glslink{spot}{spotów} użytownika jest unieważniane w celu
wymuszenia ponownego pobrania danych i ich aktualizacji.

\textbf{\texttt{SpotAddMediaModal}}
\glslink{react-component}{Komponent} zawierający formularz do dodawania mediów do \glslink{spot}{spota}.
Pozwala na wybór plików (zdjęć oraz filmów), które wraz z podlądem wyświetlane są w formie listy.
Kliknięcie elementu na liście powoduje jego usunięcie.
Dane przysłane są \glslink{mutacja}{mutacją} na \glslink{backend}{backend}, a zapytanie dotyczące
szczegółów \glslink{spot}{spota} są unieważnianie, dzięki czemu dodane media będą znajdywać się w
galerii zdjęć i filmów.
Komunikat o wyniku operacji wyświetlany jest w \texttt{Notification}.
Otworzenie oraz zamknięcie elementu animowane jest \glslink{biblioteka}{biblioteką} \texttt{motion}.

\subsubsubsection{Komentarze spota}
\label{subsubsec:komentarze-spota}

Część panelu szczegółów \glslink{spot}{spota}, w której znajdują się zarówno komentarze, jak i \glslink{modal}{modal}
do ich dodawania.
Zbudowana jest z następujących \glslink{react-component}{komponentów}:
\begin{itemize}
    \item \texttt{SpotCommentsList}
    \item \texttt{SpotCommentHeader}
    \item \texttt{AddSpotCommentModal}
    \item \texttt{SpotComment}
    \item \texttt{SpotCommentAuthor}
    \item \texttt{SpotCommentMediaGallery}
    \item \texttt{SpotCommentVotesPanel}
    \item \texttt{SpotCommentVoteDisplay}
\end{itemize}

\textbf{\texttt{SpotCommentsList}}
\glslink{react-component}{Komponent} wyświetlający listę komentarzy przewijalnej przy użyciu
\glslink{infinite-scroll}{infinite scroll}.
Mechanizm ten jest realizowany za pomocą \texttt{IntersectionObserver}, który nasłuchuje na element powiązany
referencją i wywołuje funkcję \texttt{fetchNextPage} z \glslink{hook}{hook'a} \texttt{useInfiniteQuery}
(\glslink{biblioteka}{biblioteka} \texttt{tanstack query}).
Na początku pobierana jest tylko pierwsza strona komentarzy oraz wyświetlany jest przycik \texttt{Show More}.
Po jego kliknięciu jest ukrywany i wysyłane jest zapytanie o drugą stronę.
Pobieranie kolejnych odbywa się w sposób opisany powyżej.
W trakcie ładowania wyświetlany jest \texttt{LoadingSpinner}, a w razie błędu komunikat
\emph{Failed to load comments.}.

\textbf{\texttt{SpotCommentHeader}}
Zawiera w sobie nagłówek „Comments” oraz przcisk otwierający \glslink{modal}{modal} będący formularzem
do dodania nowego komentarza (\texttt{AddSpotCommentModal}).
Po jego kliknięciu jeżeli użytkownik nie jest zalogowany, w \glslink{react-component}{komponencie}
\texttt{Notification} wyswietlany jest komunikat z odpowiednią informacją.

\textbf{\texttt{AddSpotCommentModal}} (rys. \ref{fig:add-spot-comment-1} - \ref{fig:add-spot-comment-4})
\glslink{modal}{Modal} będący formularzem do dodania nowego komentarza.
Zawiera w sobie pola do określenia oceny w gwiazdkach, dodania tekstu oraz zdjęć i filmów.
Ustawienie oceny \glslink{spot}{spota} jest polem obowiązkowym z domyślną wartością 0.
Zrealizowane jest przy użyciu \glslink{react-component}{komponentu} \texttt{Rate} skonfigurowanego
przez \texttt{ConfigProvider}.
Oba elementy pochodzą z \glslink{biblioteka}{biblioteki} \texttt{antd}.
Innym obowiązkowym polem jest pole tekstowe, w którym użytkownik umieszcza swoją opinię.
Wartości wpisane przechodzą proces walidacji przez \texttt{addSpotCommentSchema} i funkcje
\glslink{biblioteka}{biblioteki} \texttt{zod}.
Informacje o błędach wyświetlane są w \texttt{Notification}.
Dodawanie zdjęć oraz filmów jest opcjonalne i zaimplementowane zostało za pomocą \glslink{react-component}{komponentu}
\texttt{UploadButton}, który otwiera eksplorator plików i umożliwia wybranie odpowiednich plików.
Wskazane elementy wyświetlane są w formie listy, a ich kliknięcie usuwa je z niej.
Możliwe jest dodanie maksymalnie dwudziestu elementów, o czym użytkownik jest informowany odpowiednim
komunikatem.
Wartości wpisane w poszczególne pola formularza przechowywane są w lokalnym stanie
\glslink{react-component}{komponentu}.
Wysłanie danych do \glslink{backend}{backendu} odbywa się poprzez \glslink{mutacja}{mutację}, a w oczekiwaniu na
odpowiedź wyświetlany jest \texttt{LoadingSpinner}.
Pomyślne zakończenie operacji skutkuje unieważnieniem zapytań powiązanych z danym \glslink{spot}{spotem}
oraz zamknięciem \glslink{modal}{modalu}.
Kliknięcie przycisku „Cancel” powoduje wywołanie funkcji \texttt{handleCancelAddSpotComment}, która
zamyka formularz, wpisane dane nie zostają zapisane.

\begin{figure}[H]
    \centering
    \includegraphics[width=1\textwidth]{attachments/implementacja-frontendu/mapa/szczegoly-spota/add_spot_comment1}
    \caption{Implementacja komponentu AddSpotCommentModal (1/4)}
    \label{fig:add-spot-comment-1}
\end{figure}
\noindent

\begin{figure}[H]
    \centering
    \includegraphics[width=1\textwidth]{attachments/implementacja-frontendu/mapa/szczegoly-spota/add_spot_comment2}
    \caption{Implementacja komponentu AddSpotCommentModal (2/4)}
    \label{fig:add-spot-comment-2}
\end{figure}
\noindent

\begin{figure}[H]
    \centering
    \includegraphics[width=1\textwidth]{attachments/implementacja-frontendu/mapa/szczegoly-spota/add_spot_comment3}
    \caption{Implementacja komponentu AddSpotCommentModal (3/4)}
    \label{fig:add-spot-comment-3}
\end{figure}
\noindent

\begin{figure}[H]
    \centering
    \includegraphics[width=1\textwidth]{attachments/implementacja-frontendu/mapa/szczegoly-spota/add_spot_comment4}
    \caption{Implementacja komponentu AddSpotCommentModal (4/4)}
    \label{fig:add-spot-comment-4}
\end{figure}
\noindent

\textbf{\texttt{SpotComment}}
\glslink{react-component}{Komponent} wyświetlający pojedynczy komentarz.
Zbudowany jest z wyspecjalizowanych komponentów odpowiedzialnych za prezentację jego poszczególnych fragmentów.
\texttt{SpotCommentAuthor} zawiera informacje o autorze komentarza: nazwę oraz zdjęcie profilowe,
kliknięcie go powoduje przeniesienie na stronę profilu danego użytkownika.
Ocena komentarza zrealizowana jest poprzez \texttt{Rate} z biblioteki \texttt{antd}.
Pozostałe dwa \glslink{react-component}{komponenty}, \texttt{SpotCommentMediaGallery}
oraz \texttt{SpotCommentVotesPanel} zostały opisane poniżej.
