%! Author = Mateusz
%! Date = 02/11/2025

\subsection{Integracja i komunikacja z backendem}
\label{subsec:integracja-i-komunikacja-z-backendem}

Niniejszy podrozdział opisuje sposób integracji aplikacji \glslink{frontend}{frontendowej} z \glslink{backend}{backendem} oraz
mechanizmy odpowiedzialne za bezpieczną i efektywną komunikację z serwerem. \newline

Jest to kluczowy element aplikacji, ponieważ wymaga bezpiecznego przesyłania danych użytkownika.
W celu uproszczenia komunikacji z serwerem skorzystaliśmy z
biblioteki \texttt{axios}~\cite{axios} oraz \glslink{biblioteka}{Biblioteki} \texttt{TanStack Query}~\cite{tanstack-query}.
We wszystkich ścieżkach, które wymagają aby użytkownik był zalogowany, do zapytania dołączany jest token \gls{jwt}.
Token jest przekazywany w ciasteczku dzięki ustawieniu parametru \texttt{withCredentials} na wartość \texttt{true}.
Przykładem pliku odpowiedzialnego za taką komunikację jest \texttt{account.js}
(rys. \ref{img:account-axios1} i \ref{img:account-axios2}), który obsługuje operacje związane z logowaniem,
rejestracją, zmianą hasła oraz wylogowaniem.

\begin{figure}[H]
    \centering
    \includegraphics[width=1\textwidth]{attachments/implementacja-frontendu/account-axios1}
    \caption{Implementacja modułu \texttt{account} (1)}
    \label{img:account-axios1}
\end{figure}

\begin{figure}[H]
    \centering
    \includegraphics[width=1\textwidth]{attachments/implementacja-frontendu/account-axios2}
    \caption{Implementacja modułu \texttt{account} (2)}
    \label{img:account-axios2}
\end{figure}

Funkcje odpowiedzialne za komunikację z backendem zostały umieszczone w katalogu \texttt{/http}.
Dzięki temu są one scentralizowane i mogą być w prosty sposób wykorzystywane w różnych częściach aplikacji.
Zastosowaliśmy TanStack Query, ponieważ znacząco ogranicza on powtarzalny kod oraz upraszcza obsługę błędów i
stanów zapytania (takich jak ładowanie danych, błąd, sukces). \glslink{biblioteka}{biblioteka} udostępnia m.in.
wartość \texttt{isLoading},
dzięki czemu komponent może łatwo wyświetlić ekran ładowania bez ręcznego zarządzania własnym stanem.
Dodatkowo \glslink{hook}{hook} \texttt{useQuery} z tej \glslink{biblioteka}{biblioteki} umożliwia
automatyczne pobieranie danych po wejściu na daną podstronę.
Oznacza to, że komponent deklaruje jedynie „jakie dane są mu potrzebne”, a TanStack Query zajmuje się ich pobraniem,
cache’owaniem oraz odświeżaniem.
Do operacji, które wymagają wywołania akcji po stronie użytkownika (np. wysłania formularza logowania),
wykorzystujemy \glslink{hook}{hook} \texttt{useMutation} z TanStack Query.
Przykład użycia tego rozwiązania w procesie logowania został przedstawiony na rys. \ref{img:login-tanstack}.

\begin{figure}[H]
    \centering
    \includegraphics[width=1\textwidth]{attachments/implementacja-frontendu/login-tanstack}
    \caption{Wykorzystanie TanStack Query przy logowaniu użytkownika}
    \label{img:login-tanstack}
\end{figure}


