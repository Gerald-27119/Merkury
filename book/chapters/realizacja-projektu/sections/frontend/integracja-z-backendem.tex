%! Author = Mateusz
%! Date = 02/11/2025

\subsection{Integracja i komunikacja z backendem}
\label{subsec:integracja-i-komunikacja-z-backendem}

W niniejszym podrozdziale opisano sposób integracji aplikacji \glslink{frontend}{frontendowej} z \glslink{backend}{backendem} oraz mechanizmy odpowiedzialne za bezpieczną i efektywną komunikację z serwerem.\newline

Jest to kluczowy element aplikacji, ponieważ obejmuje przesyłanie oraz przetwarzanie danych użytkownika.
W celu uproszczenia komunikacji z serwerem zdecydowano się na wykorzystanie \glslink{biblioteka}{biblioteki} Axios~\cite{axios} oraz TanStack Query~\cite{tanstack-query},
które zapewniają spójny sposób definiowania zapytań, obsługi błędów oraz zarządzania stanem asynchronicznym po stronie klienta.

Zadania asynchroniczne związane z komunikacją z \glslink{api}{API} realizowano z użyciem \texttt{async}/\texttt{await} zgodnie z opisem w \cite[rozdz.~13.3]{flanagan-js-definitive-guide-pl}.

W przypadku ścieżek wymagających uwierzytelnienia do zapytania dołączany jest \glslink{token}{token} \glslink{jwt}{JWT}.
Token przekazywany jest w ciasteczku \glslink{http-only-cookie}{HttpOnly}, a jego wysyłanie realizowane jest automatycznie przez przeglądarkę dzięki ustawieniu parametru
\texttt{withCredentials} na wartość \texttt{true}. W razie braku tokena lub jego nieważności dostęp do danych nie jest przyznawany.
Dodatkowo, po odświeżeniu strony lub wejściu na widok w sytuacji, gdy ciasteczko utraciło ważność, sesja jest uznawana za nieaktywną,
a użytkownik zostaje automatycznie wylogowany. Podczas wylogowania inicjowanego przez użytkownika ciasteczko jest usuwane,
co skutkuje utratą uprawnień do zasobów chronionych.

Przykładem pliku odpowiedzialnego za taką komunikację jest \texttt{account.js}
(rys. \ref{img:account-axios1} i \ref{img:account-axios2}), który obsługuje operacje związane z logowaniem
rejestracją, zmianą hasła oraz wylogowaniem.

\begin{figure}[H]
    \centering
    \includegraphics[width=1\textwidth]{attachments/implementacja-frontendu/account-axios1}
    \caption{Implementacja modułu \texttt{account} (1/2)}
    \label{img:account-axios1}
\end{figure}

\begin{figure}[H]
    \centering
    \includegraphics[width=1\textwidth]{attachments/implementacja-frontendu/account-axios2}
    \caption{Implementacja modułu \texttt{account} (2/2)}
    \label{img:account-axios2}
\end{figure}

Funkcje odpowiedzialne za komunikację z \glslink{backend}{backendem} umieszczono w katalogu \texttt{/http}.
Dzięki temu są one scentralizowane i mogą być w prosty sposób wykorzystywane w różnych częściach aplikacji.
Zastosowanie TanStack Query umożliwiło znaczące ograniczenie powtarzalnego kodu oraz uprościło obsługę błędów i
stanów zapytania (takich jak ładowanie danych, błąd czy sukces).
\glslink{biblioteka}{Biblioteka} udostępnia m.in. wartość \texttt{isLoading}, dzięki czemu komponent może łatwo wyświetlić ekran ładowania
bez konieczności ręcznego zarządzania własnym stanem.
Dodatkowo \glslink{hook}{hook} \texttt{useQuery} pozwala na automatyczne pobieranie danych po wejściu na daną podstronę.
Komponent deklaruje jedynie, jakie dane są mu potrzebne, a TanStack Query realizuje ich pobranie, cache’owanie oraz odświeżanie.
Do operacji wymagających wywołania akcji po stronie użytkownika (np. wysłania formularza logowania)
wykorzystywany jest \glslink{hook}{hook} \texttt{useMutation} z TanStack Query.
Przykład użycia tego rozwiązania w procesie logowania przedstawiono na rys. \ref{img:login-tanstack}.

\begin{figure}[H]
    \centering
    \includegraphics[width=1\textwidth]{attachments/implementacja-frontendu/login-tanstack}
    \caption{Wykorzystanie TanStack Query przy logowaniu użytkownika}
    \label{img:login-tanstack}
\end{figure}


