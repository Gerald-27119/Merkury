%! Author = Mateusz
%! Date = 02/11/2025

\subsection{Style}
\label{subsec:style}

Do stylowania interfejsu wykorzystaliśmy \gls{framework} Tailwind CSS~\cite{tailwind}.
Dzięki gotowym klasom udostępnianym przez Tailwind mogliśmy definiować wygląd
elementów bezpośrednio w kodzie komponentu, bez konieczności przechodzenia do osobnych plików ze stylami.
Ułatwia to zarówno tworzenie widoków, jak i późniejsze modyfikacje — w przypadku zmiany stylu
dokładnie wiadomo, gdzie należy jej dokonać.
Korzystanie ze zdefiniowanych klas pozwoliło nam również zachować spójność wizualną w całej aplikacji.
W pliku \texttt{index.css} zdefiniowaliśmy zmienne
kolorystyczne (rys. \ref{img:css-kolory1} i \ref{img:css-kolory2}).
Dzięki temu zmiana motywu kolorystycznego w przyszłości sprowadza się do edycji wartości w jednym miejscu.

\begin{figure}[H]
    \centering
    \includegraphics[width=1\textwidth]{attachments/css-kolory1}
    \caption{Implementacja zmiennych kolorystycznych (1)}
    \label{img:css-kolory1}
\end{figure}

\begin{figure}[H]
    \centering
    \includegraphics[width=1\textwidth]{attachments/css-kolory2}
    \caption{Implementacja zmiennych kolorystycznych (2)}
    \label{img:css-kolory2}
\end{figure}

W niektórych miejscach konieczne było zapisanie stylów w czystym \gls{css}, ponieważ część
użytych \glslink{biblioteka}{Bibliotek} tego wymagała. W innych przypadkach wystarczyło skorzystać z klas
zdefiniowanych w \texttt{index.css} oraz klas Tailwinda.
Cała aplikacja jest \glslink{responsywnosc}{Responsywna}. Tailwind udostępnia predefiniowane prefiksy
\glslink{responsywnosc}{Responsywne} (np. \texttt{md:}, \texttt{lg:}) (rys. \ref{img:tailwind-responsywnosc}),
stworzyliśmy również własny (\texttt{3xl:}) na ekrany o rozdzielczości 2560px
które pozwalają przypisywać style zależnie od
szerokości ekranu bez pisania własnych reguł @media.
Dzięki temu implementacja widoków mobilnych i desktopowych była znacząco szybsza.

\begin{figure}[H]
    \centering
    \includegraphics[width=1\textwidth]{attachments/tailwind-responsywnosc}
    \caption{Przykładowe użycie klas Tailwind (w tym prefiksów responsywności)}
    \label{img:tailwind-responsywnosc}
\end{figure}

Tailwind został też wykorzystany do obsługi trybu jasnego i ciemnego.
Wystarczy dodać klasę z prefiksem \texttt{dark:} (np. \texttt{dark:bg-black}), aby zmienić kolorystykę
elementu, gdy aplikacja jest w trybie ciemnym (rys. \ref{img:tailwind-motyw}).

\begin{figure}[H]
    \centering
    \includegraphics[width=1\textwidth]{attachments/tailwind-motyw}
    \caption{Przykładowe użycie klas Tailwind (w tym wariantu \texttt{dark:})}
    \label{img:tailwind-motyw}
\end{figure}

Aby uzyskać płynniejsze i przyjemniejsze animacje, wykorzystaliśmy \glslink{biblioteka}{Bibliotekę} Motion~\cite{motion}.
Pozwala ona w prosty sposób tworzyć animacje elementów interfejsu, bez potrzeby ręcznego pisania
złożonych reguł \gls{css}. W naszej aplikacji użyliśmy jej m.in. w polach formularza logowania i rejestracji
(rys. \ref{img:motion}).
Na początku etykieta pola (np. „username”) jest wyświetlana wewnątrz pola tekstowego,
natomiast po kliknięciu w pole jest płynnie przesuwana nad to pole, co poprawia czytelność i ergonomię formularza.

\begin{figure}[H]
    \centering
    \includegraphics[width=1\textwidth]{attachments/motion}
    \caption{Implementacja animacji z wykorzystaniem Motion}
    \label{img:motion}
\end{figure}
