%! Author = Mateusz
%! Date = 19/11/2025
%!TEX root = ../../sprz.tex

\subsection{Panel użytkownika}
\label{subsec:panel-uzytkownika-frontend}

W niniejszym rozdziale przedstawiono implementację panelu użytkownika na frontendzie.\newline

Kolejnym z głównych modółów aplikacji jest panel użytkownika.
Został on podzielony na 8 zakładek:

\begin{itemize}
    \item Profile - profil użytkownika wraz z możliwością zmiany zdjęcia profilowego,
    \item Spots - lista z spotami dodanymi do list (np. ulubionych, do ponownego odwiedzenia),
    \item Photos - lista zdjęć które użytkownik doddał do spotów, komentarzy pod spotami oraz na forum,
    \item Movies - lista filmów które użytkownik doddał do spotów, komentarzy pod spotami oraz na forum,
    \item Social - listy z znajomymi, obserwowanymi i obserwującymi,
    \item Add spot - lista ze spotami które dodał użytkownik oraz formularz do dodania nowego spota,
    \item Comments - lista komentarzy które użytkownik dodał pod spotami,
    \item Settings - ustawienia użytkownika (zmiana: nazwy użytkownika, e-mail, hasła),
\end{itemize}

\subsubsection{Profile}
Profil użytkownika został zbudowany z pięciu komponentów:
- UserOwnProfile
- ProfileForViewer
- Profile
- ProfileStats
- MostPopularImage

\texttt{UserOwnProfile} - Jest to komponent używany do wyświetlenia własnego profilu uzytkownika.
JEst on bardzo prosto zbudowany korzysta tylko z komponentu Profile (numer) który jest uniwersalny
dla obu widoków.
Dodatkowo znajduje się tam pasek ładowania informujący o postępie ładowania.
Całe dane są pobierane za pomocą hooka useQuery z bibiloteki tanstack query,
korzysta on z funkcji "getUserOwnProfile" która nie przyjmuje rzadnych parametrów.

\texttt{ProfileForViewer} - Ten komponent jest używany do wyśwetlenia profilu innego użytkownika.
Poza tym co zawiera komponet USerOwnProfile ten zawera dodatkowo komponenty:
-Button - który jest przekazywany do Profil za pomocą propu children,
-Modal.
Znajdują się tam dwa przyciski które w zależności od tego czy użytkownik jest zalogowany
i czy obserwuje bądź ma go w znajomych wyśwetlają stosowne informacje na przyciskach.
Komponent modal wyśwetla się jeżeli użytkownik chce usunąć znajomego lub przestać obserwować,
wyśwetla się wtedy komunikat czy napewno chce to zrobić.
Jeżeli profil o danej nazwie użytkownika nie istnieje wyśwetla się komunikat "No profile data available".
Do pobrania informacji o profilu tutaj również wykorzystano hook useQuery, nazwa uśytkownika jest pobierana z
ścieżki URL za pomocą hooka useParams.
Jeżeli pobrane dane zawieraja informację że jest to profil własny użytkownika to za
pomocą hooka useNavigate w useEffect zostaje on przeniesiony na adres "/account/profile"
Natomiast do edycji danych:
- edycji statusu znajomego - funkcja "changeUserFriendsStatus"
- edycji stnu obserwacji - funckja "editUserFollowed"
- edycji stanu znajomego - funkcja "editUserFriends"
wykorzystano hook useMutation.
Przy błędzie wyświetlana jest stosowna informacja w zalezności czego ten błąd dotyczy i za pomocą
dispatcha z reduxa wyświetlany w komponencie Notification.

\texttt{Profile} - jest to główny komponent używany w tej sekcji panelu użytkownika
ponieważ jest używany w obu wariantach.
Jest tutaj używany hook useMutation do zmiany zdjęcia profilowego ale
tylko i wyłacznie jeżeli jest to profil własny użytkownika, wtedy po najechaniu
na zjęcie profilowe pojawi się komunikat "Change profile photo." a po kliknięciu w
ten komunikat wyświetli się okno do wyboru nowego zdjęcia profilowego.
Komponent ten korzysta również z uniwersalnego komponentu AccountWrapper który jest używany w każdej
sekcji panelu uzytkownika dzięki czemu widok jest powtarzalny bez powtarzania tych samych komponentów.
Następnie wyśwetlają się cztery komponenty ProfileStat które wyśwetlają informacje o liczbie
danych rzeczy w panelu użytkownika.
Na samym dole wyśwetla się komunikat "most popular photos" a pod nim
albo komunikat "This user hasn't added any photos." jezeli jest to konto innego użytkownika albo
"You haven't added any photos" jeżeli jest to konto własne.
Jeżeli użytkownik posiada liste zdjęć to za pomocą komponentu MostPopularImage wyśweitlą się cztery
najpopularniajesze zdjęcia użytkownika.

\texttt{ProfileStat} - komponent ten służy do wyświetlania informacji o liczbie
obserwowanych, obserwujacych, znajomych oraz djeć danego użytkownika.
przyjmuje on 3 propy:
- value - liczbę
- label - nazwę kategorii
- onClick - funkcję która uruchomi się po kliknieciu w ten element

\texttt{MostPopularImage} - komponent służy do wyświetlenia zdjęć wraz z jego statystykami.
Przyjmuje on tylko jednego propa: image - zjęcie z listy czterach najpopularniajeszych zdjęć.
Wyświetla on zdjęcie a na nim za pomocą klasy relative wyświetla liczbę polubień oraz
wyświetleń zdjęcia wraz z odpowiednimi ikonkami.

\subsubsection{Spots}
KOmponent Spots składa się z 3 głównych komponentów:
- FavoriteSpots
- FavoriteSpotTile
- FavoriteSpotTags

\texttt{FavoriteSpots} - Jest to główny komponent tej sekcji który składa wszystko w jedno.
Na samej górze znajduje się lista "menuTypes" która służy do wyśweitlenia przycisków nawigacyjnych
do przełaczania różnego typu list;
Wybarany aktualnie typ listy jest przechowywany w tym tomponencie za pomocą hooka useState.
Do pobrania danych z backendu użyto hooka useInfinityQuery z biblioteki Tanstack Query,
co pozowliło w łatwy sposób zaimplementować infinity scrolla.
TAk jak w Profilu cały ten komponent jest opakowany w komponent AccountWrapper,
ale tutaj jest jeszcze dodatakowo AccountTitle który wyświetla nagłówek danej podstorny.
Nastęnie wyśwetlane są przyciski służące do zmiany typu listy za pomocą komponentu Button.
Przy pierwszym ładowaniu strony oraz przy ładowaniu kolejnych stron danych pokazuje się Loading spinner.
Jeżeli użytkownik nie posiada żadnych dodanych spotów poajwia się komunikat:
"You don't have any spots in your list." w przeciwnym razie zostaną wyświetlnone
kafelki spotów za pomocą komponentu FavoriteSpotTile.

\texttt{FavoriteSpotTile} - Przyjmuje on dwa parametry:
-spot
- selectedType
Do usnięcia danego spota z którejkolwiek listy wykorzystano hook useMutatnion
przyjmuje ona funkcję "editFavoriteSpotList" która przyjmuje następujące parametry:
- type - typ listy
- spotId - identyfikator spota
- operationType - typ operacji (dodanie lub usunięcie).
Wyśweitlają się na nim następujące informacje:
- zdjęcie
- liczba wyświetleń
- lokaalizacja
- śrenia ocen
- nazwa spota
- tagi spota (numer komponentu)
- przycisk do zobaczenia spota na mapie
- przycisk do usunięcia spota z listy
Wyśwetla się tu również okno modalne które służy do potwierdznei usunięcia spota z danej listy,
informacja czy modal jest otwarty przechowuje się w stanie hooka useBoolean.

\texttt{FavoriteSpotTags} - przyjmuje on tylko listę tagów które, wyśwetla tylko ich nazwę w liście

\subsubsection{Photos}
Zdjęcia składają się z trzech głównych komponentów:
- Photos
- Media
- Photo

\texttt{Photos} - Bardzo prosty komponent który wyświetla komponent Media,
dane są pobierane za pomocą useInfiniteQuery dzięki czemu korzysta z infinite scrolla.
W useEffect za pomocą dispatcha z biblioteki redux ustawiany jest komunikat o błędzie,
jeżeli takowy się pokaże.

\texttt{Photo} - przyjmuje on tylko jeden parametr: photo czyli zdjęcie z listy.
Wyśwetla on zdjęcie z listy a na zdjęciu za pomocą klas relative i absolute
ukazuje pasek z informacją o liczbie polubień oraz liczbie wyświetleń wraz
z odpowiednimi ikonami.

\texttt{Media} - Jest to uniwersalny komponent dla zdjęć oraz filmów użytkownika.
Przyjmuje on osiem parametrów:
- variant - używany do komponentu AccountWrapper
- searchDate - obiekt który przetrzymuje dane do filtrowania po dacie
- onSortChange - funkcja używana do sortowania
- onDateChange - funckja używana do ustawienia filtrowania po dacie
- isLoading - informuje czy dane sę pobierane
- mediaList - typ który określa czy są to zdjęcia czy filmy
- loadMoreRef
- isFetchingNextPage - informuje czy pobiera kolejną stronę z danych użytkownika
Tak jak w pozostałych sekcjach korzysta on z AccountWrapper do opakowania całego widoku w identyczny wygkląd.
Korzysta również z komponentu AccountTitle wyświetla on albo "Photos" albo "Movies"
w zależności od typu wariantu.
NAstępnie wyświetla komponent SortAndDateFilters które wyświetlają panel sortowania
oraz kormularze do wyboru daty od oraz do służące do filtrowania.
Wtedy jest wyśweitlana jest lista filmów lub zdjęć które są zgrupowne po
dacie Za pomocą komponentu DateBadge który przyjmuje datę i ją wyświetla.
Wtedy w liście w zależności od tego czy variat to zdjęcia czy filmy wyświetla odwpowieni
komponent (Photos lub Movies).
Jeżeli użytkownik nie posiada żadnych mediów wyświetla się komunikat:
"You haven't added any (photos lub movies)".
Jak dane się ładują albo ładuje się kolejna strona danych pokazuje sie komponent LoadingSpinner.

\subsubsection{Movies}
Filmy składają się z trzech głównych komponentów:
- Movies
- Media
- Movie

\texttt{Movies} - Komponent który wyświetla komponent Media wygląda niemal identycznie jak
komponent photos różni się tylko przyjmowaną metodą,
dane są pobierane za pomocą useInfiniteQuery dzięki czemu korzysta z infinite scrolla.
W useEffect za pomocą dispatcha z biblioteki redux ustawiany jest komunikat o błędzie,
jeżeli takowy się pokaże.

\texttt{Movie} - przyjmuje on tylko jeden parametr: movie czyli zdjęcie z listy.
Wyśwetla on film wraz z możliwością odtworzenia takowego
z listy a na zdjęciu za pomocą klas relative i absolute ukazuje pasek z informacją
o liczbie polubień oraz liczbie wyświetleń wraz z odpowiednimi ikonami.
Do odtwarzania filmów wykorzystano bibliotkę "react-player".

\subsubsection{Social}


\subsubsection{Add spot}
Sekcja z dodanymi spotami składa się z siedmiu głównych komponentów:
- AddedSpot
- UploadButton
- SpotMap
- PolygonDrawer
- AddSpotModal
- AddSpotInput
- AddSpotTile

\texttt{AddedSpot} - Jest to główny komponent tej sekcji służy on do wyświetlenia listy spotów
oraz otworzenia okna modlanego do dodania nowego spota, stan do tego jest zarządzany przez hooka useBoolean.
Na początku pobierane dane są za pomocą useInfinityQuery dzięki czemu jest tu zaimplementowany mechanizm
infinity scroll.
Tak jak w pozostałych sekcjach użyto tutaj komponentów AccountWrapper oraz AccountTitle.
Dodano tutaj też przycisk do otwarcia okna modalnego do dodania nowego spota, okno to
otworzy się tylko na ekranach których szerokość jest większa od 900px.
Jeżeli użytkownik dodał jakiekolwiek spoty wyśweitli się lista komponentów
AddedSpotTile lecz jeżeli nie dodał żadnego wyśweitli się następujący komunikat  "You haven't added any spots yet."
Przy pierwszym ładowaniu strony oraz przy scrolowaniu w dół i ładowaniu kolejych
elementów listy pokaże się LoadingSpinner.

\texttt{UploadButton} - przyjmuje on tylko jeden parametr, onFileSelect jest to funkcja
która służy do ustawienia listy zdjęć oraz filmów które użytkownik dodaje do spota.
W komponencie trzymany jest stan z dodanymi mediami.
Użytkownik może też tu usunąć zdjęcie lub film które dodał.
Wyśweilane są tu przycisk do dodania multimediów oraz ich lista jako miniaturowe
zdjęcia oraz filmy po najechaniu i naciśnięciu na taką miniaturkę zostaje ona usunięta.

\texttt{SpotMap} - Jest to komponent który przymuje dwie współżędne szerokość i wysokość geograficzną
położenia spota.
Wyświetla ona lokalizację spota na małej mapie za pomocą pineski.
Do wyświetlenia mapy wykorzystano bibliotekę maplibregl

\texttt{PolygonDrawer} - Jest to mapa służąca do zaznaczenia konturów dodawanego spota.
Przyjmuje ona trzy parametry:
- onPolygonComplete - funkcja służąca do potwierdznia konturów
- initialPosition - pozycja na mapie
- borderPointsError - komunikaty błędów
Ustawione koordynaty spota są trzymane w useState w tym komponencie.
Przy zmianie danych w parametrze initialPosition użytkownik zostaje automatycznie przeniesiony w nowe miejsce.
Po kliknięciu w dane miejsce na mapie jego współżedne sa oddawane do listy w useState,
po dodaniu trzeciego pokazuje się również pole danego spota.
Głównym komponentem jest mapa z maplibregl która za pomocą inych komponentów
wyświetla odpowiednie pole.
Poniżej znajdują się dwa przycsiki jeden do cofnięcia dodania ostaniego punktu,
drugi do zatwierdznie punktów.

\texttt{AddSpotModal} - Komponent ten służy do dodawania nowego spota przez użytkownika.
Przyjmuje on dwa parametry:
- onClose - funkcja do zamykania modala
- isOpen - informacja czy okno modalne jest otwarte
Zawiera on stany do błędów walidacji oraz dane dodawanego spota.
Po wpisaniu wszystkich danych adresowych mapka zostanie automatycznie przeniesiona
na daną lokalizację.
Do dodania nowego spota wykorzystno hooka useMutation który obsługuje również komunikaty o
pozytywnym zapisie oraz o błędach.
Do płynnego pojawienia się okna modalnego wykorzystano komponent AnimateProsence z biblioteki motion
wraz z motion.div z ustawionymi odpowiednimi wartościami.
Komponent ten wyświetla nagłówek który mówi gdzie użytkownik się znajduje oraz w prawym górnum roku przycisk
do zamknięcia.
Jest podzielony rownież na dwie części po lewej wyświetlają się formularze
do podania podstawowych danych o spocie oraz adresie (komponent AddSpotInput),
oraz komponent do dadania multimediów.
Po prawej natomiast znajduje się mapa (komponent PolygonDrawer).
Na samym dole są dwa przyciski (komponenty Button) jeden do zamknięcia okna modalnego
drugi do dodania spota.
Do wyświetlenia modala użyto hooka createPortal dzięki czemu można w łatwy i szybki
sposób określić gdzie w drzewie DOM ma się on znaleść.

\texttt{AddSpotInput} - Jest to komponent do wyświetlenia i obsługi listy pól formularza
posaiada stan błędów do walidacji w hooku useState.
Przyjmuje on trzy parametry
- config - lista pól do wyświetlenia
- spot - dane do wyświetlenia w polach
- onChange - funkcja do obsługi pól
Wyświetla on listę komponentów FormInput jest to ogólny komponent do pola formularzu
używany w wielu miejscach aplikacji.
Do walidacji wykorzystano tu bibliotekę zod a schemat walidacji znajduje się w plik spotSchema.

\texttt{AddedSpotTile} - Jest to kafelek do wyświetlnia daych o dodanym spocie w komponencie AddedSpot,
przyjmuje on tylko parametr spot.
Wyśwwietla on:
- zdjęcie spota,
- małą mapke (komponent SpotMap) z informacją o jego lokalizacji,
- nazwę spota,
- opis spota,
- adres ( kraj, miast i ulicę)


\subsubsection{Comments}
Komentarze składają się z dwóch głównych komponentów:
- Comments
- CommentsList

\texttt{Comments} - wyglądem i budową przypomina bardzo sekcję filmów i zdjęć;
Tylko tutaj do sortowania i filrtowania stworzono customowy hook useDateSortFilter.
Do pobierania danych użyto hooka useIfinityQuery dzięki czemu skorzystano z infinity scrolla.
W hooku useEffect z pomocą dispatch ustawiono komunikat o błędzie który jest wyświetlany w komponencie notification.
Tak jak w pozostałych sekcjach tuttaj również skorzystano z AccountWrapper oraz AccountTitle,
dodatkowo jak w filmach i zdjęciach użyto komponentu SortAndDateFilters.
Następenie wyśwetla się lista komentarzy jeżeli użytkownik posiada jakieś,
została ona pogrupowana po dacie oraz spocie i obie te inforamcje wyśweitlajane są za pomocą
komponentu DateBadge data jest przesyłana za pomocą parametru date, a komunikat do jakiego
spota został dodany ten komentarz za pomocą parametru children.
Same komentarze są wyświetlane za pomocą komponentu CommentsList.
Jeżeli uzytkownik nie posiada żadnych dodanych komentarzy wyświetli się komunikat
"You haven't added any comments.".
Przy pierwszym ładowaniu stron oraz przy scrolowaniu w dół pokazuje się Loading spinner.

\texttt{CommentsList} - przyjmuje on tylko jeden parametr czyli listę komentarzy które ma wyświetlić.
Każdy komentarz posiada godzinę dodania oraz jego treść.

\subsubsection{Settings}

Do ustawień stworzono dwa główne komponenty:
- Settings
- DisableInput

\texttt{Settings} - Jest to główny komponent do ustawień zarządza on którą informację
użytkownik planuje zmienić oraz wyświetleniem odpowiednich pól formularzy.
Typ danych które uzytkonik planuje edytować jest przetrzymuwany w useState.
Na początku za pomocą hooka useQuery pobierane są dane użytkownik i wyśweitlane w
komponentach DisableInput.
Po wyborze danych które użytkownik chce zmienić po prawej stronie ukazuje się sekcja z
z odpowiednimi polami.
Po wpisaniu zmienionych danych i kliknięciu przycisku dane są wysyłane za pomocą
hooka useMutation jeżeli pojawi się błądą zostaje on wyśwetlony w komponencie Notification,
jeżeli wszystko zakończy się sukcesem ukazany sotanie komunikat o pozytywnej zmianie.
Ten komponent również korzysta z AccountWrapper oraz AccountTitle,
Jeżeli użytkownik założył konto za pomocą formularza rejestracji może zmienić dane
i wyświetlą się podane wyżej pola formularzy.
Lecz jeżeli użytkownik zalogował się za pomocą Google lub Github pojawi się komunikat
"Your account was created via (Google lub Github)" w zależniści czym zalogował się,
oraz "Your email address is (...) and cannot be changed.".
Dla zmieny nazwy użytkownika oraz e-maila pokaże się tylko jedno pole,
natomiast jeżeli użytkownik planuje zmienić hasło pokażą się trzy pola
do starego hasła, nowego hasła i powtórzenia nowego hasła.
Wszytkie pola są odpowiednio walidowane za pomocą biblioteki zod jego schamat
znajdujje się w pliku validationSchema.

\texttt{DisableInput} - przyjmuje on pięć parametrów:
- label - nazwę pola
- value - jego wartość
- type - password lub text
- id - identyfikator pola
- onEdit - funkcję do ustawienia co użytkownik chce edytować
Komponent ten wyświetla tylko informacje o nazwie oraz pole z wypisanymi wartościami
dla hasła wyświetlane są gwiazdki, oraz przycisk edit który słóży do wybrania co
użytkownik planuje edytować.