%! Author = Mateusz
%! Date = 19/11/2025

\subsection{Panel użytkownika}
\label{subsec:panel-uzytkownika-frontend}

W niniejszym rozdziale przedstawiono implementację panelu użytkownika na frontendzie.\newline

Kolejnym z głównych modółów aplikacji jest panel użytkownika.
Został on podzielony na 8 zakładek:

\begin{itemize}
    \item Profile - profil użytkownika wraz z możliwością zmiany zdjęcia profilowego,
    \item Spots - lista z spotami dodanymi do list (np. ulubionych, do ponownego odwiedzenia),
    \item Photos - lista zdjęć które użytkownik doddał do spotów, komentarzy pod spotami oraz na forum,
    \item Movies - lista filmów które użytkownik doddał do spotów, komentarzy pod spotami oraz na forum,
    \item Social - listy z znajomymi, obserwowanymi i obserwującymi,
    \item Add spot - lista ze spotami które dodał użytkownik oraz formularz do dodania nowego spota,
    \item Comments - lista komentarzy które użytkownik dodał pod spotami,
    \item Settings - ustawienia użytkownika (zmiana: nazwy użytkownika, e-mail, hasła),
\end{itemize}

\subsubsection{Profile}
Profil użytkownika został zbudowany z pięciu komponentów:
- UserOwnProfile
- ProfileForViewer
- Profile
- ProfileStats
- MostPopularImage

\texttt{UserOwnProfile} - Jest to komponent używany do wyświetlenia własnego profilu uzytkownika.
JEst on bardzo prosto zbudowany korzysta tylko z komponentu Profile (numer) który jest uniwersalny
dla obu widoków.
Dodatkowo znajduje się tam pasek ładowania informujący o postępie ładowania.
Całe dane są pobierane za pomocą hooka useQuery z bibiloteki tanstack query,
korzysta on z funkcji "getUserOwnProfile" która nie przyjmuje rzadnych parametrów.

\texttt{ProfileForViewer} - Ten komponent jest używany do wyśwetlenia profilu innego użytkownika.
Poza tym co zawiera komponet USerOwnProfile ten zawera dodatkowo komponenty:
-Button - który jest przekazywany do Profil za pomocą propu children,
-Modal.
Znajdują się tam dwa przyciski które w zależności od tego czy użytkownik jest zalogowany
i czy obserwuje bądź ma go w znajomych wyśwetlają stosowne informacje na przyciskach.
Komponent modal wyśwetla się jeżeli użytkownik chce usunąć znajomego lub przestać obserwować,
wyśwetla się wtedy komunikat czy napewno chce to zrobić.
Jeżeli profil o danej nazwie użytkownika nie istnieje wyśwetla się komunikat "No profile data available".
Do pobrania informacji o profilu tutaj również wykorzystano hook useQuery, nazwa uśytkownika jest pobierana z
ścieżki URL za pomocą hooka useParams.
Jeżeli pobrane dane zawieraja informację że jest to profil własny użytkownika to za
pomocą hooka useNavigate w useEffect zostaje on przeniesiony na adres "/account/profile"
Natomiast do edycji danych:
- edycji statusu znajomego - funkcja "changeUserFriendsStatus"
- edycji stnu obserwacji - funckja "editUserFollowed"
- edycji stanu znajomego - funkcja "editUserFriends"
wykorzystano hook useMutation.
Przy błędzie wyświetlana jest stosowna informacja w zalezności czego ten błąd dotyczy i za pomocą
dispatcha z reduxa wyświetlany w komponencie Notification.

\texttt{Profile} - jest to główny komponent używany w tej sekcji panelu użytkownika
ponieważ jest używany w obu wariantach.
Jest tutaj używany hook useMutation do zmiany zdjęcia profilowego ale
tylko i wyłacznie jeżeli jest to profil własny użytkownika, wtedy po najechaniu
na zjęcie profilowe pojawi się komunikat "Change profile photo." a po kliknięciu w
ten komunikat wyświetli się okno do wyboru nowego zdjęcia profilowego.
Komponent ten korzysta również z uniwersalnego komponentu AccountWrapper który jest używany w każdej
sekcji panelu uzytkownika dzięki czemu widok jest powtarzalny bez powtarzania tych samych komponentów.
Następnie wyśwetlają się cztery komponenty ProfileStat które wyśwetlają informacje o liczbie
danych rzeczy w panelu użytkownika.
Na samym dole wyśwetla się komunikat "most popular photos" a pod nim
albo komunikat "This user hasn't added any photos." jezeli jest to konto innego użytkownika albo
"You haven't added any photos" jeżeli jest to konto własne.
Jeżeli użytkownik posiada liste zdjęć to za pomocą komponentu MostPopularImage wyśweitlą się cztery
najpopularniajesze zdjęcia użytkownika.

\texttt{ProfileStat} - komponent ten służy do wyświetlania informacji o liczbie
obserwowanych, obserwujacych, znajomych oraz djeć danego użytkownika.
przyjmuje on 3 propy:
- value - liczbę
- label - nazwę kategorii
- onClick - funkcję która uruchomi się po kliknieciu w ten element

\texttt{MostPopularImage} - komponent służy do wyświetlenia zdjęć wraz z jego statystykami.
Przyjmuje on tylko jednego propa: image - zjęcie z listy czterach najpopularniajeszych zdjęć.
Wyświetla on zdjęcie a na nim za pomocą klasy relative wyświetla liczbę polubień oraz
wyświetleń zdjęcia wraz z odpowiednimi ikonkami.

\subsubsection{Spots}
Listy z zapisanymi spotami użytkownika.
Wydzielono 4 takie listy:
\begin{itemize}
    \item Favorites - lista z ulubionymi,
    \item Plan to visit - lista ze spotami planowanymi do odwiedzenia,
    \item Visited liked it - lista z spotami które użytkownik odwiedził i mu się spodobały,
    \item Visited didn't like it - lista z spotami które użytkownik odwiedził i mu się nie spodobały.
\end{itemize}



\subsubsection{Photos}
\subsubsection{Movies}
\subsubsection{Social}
\subsubsection{Add spot}
\subsubsection{Comments}
\subsubsection{Settings}