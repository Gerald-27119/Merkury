%! Author = Mateusz
%! Date = 19/11/2025

\subsection{Panel użytkownika}
\label{subsec:panel-uzytkownika-frontend}

W niniejszym rozdziale przedstawiono implementację panelu użytkownika po stronie \glslink{frontend}{frontendu}.

Jednym z głównych modułów aplikacji jest panel użytkownika.
Został on podzielony na osiem zakładek:

\begin{itemize}
    \item \textbf{Profile} -- profil użytkownika wraz z możliwością zmiany zdjęcia profilowego;
    \item \textbf{Spots} -- listy \glslink{spot}{spotów} dodanych do różnych kategorii (ulubione, planowane do odwiedzenia,
    odwiedzone i ocenione pozytywnie, odwiedzone i ocenione negatywnie);
    \item \textbf{Photos} -- lista zdjęć dodanych przez użytkownika do \glslink{spot}{spotów}, komentarzy pod \glslink{spot}{spotami} oraz na forum;
    \item \textbf{Movies} -- lista filmów dodanych przez użytkownika do \glslink{spot}{spotów}, komentarzy pod \glslink{spot}{spotami} oraz na forum;
    \item \textbf{Social} -- listy znajomych, obserwowanych i obserwujących;
    \item \textbf{Add spot} -- lista \glslink{spot}{spotów} dodanych przez użytkownika oraz formularz dodawania nowego \glslink{spot}{spota};
    \item \textbf{Comments} -- lista komentarzy dodanych przez użytkownika pod \glslink{spot}{spotami};
    \item \textbf{Settings} -- ustawienia konta użytkownika (zmiana nazwy użytkownika, adresu e-mail oraz hasła).
\end{itemize}

Wszystkie zakładki korzystają z uniwersalnego komponentu \texttt{AccountWrapper}, który zapewnia spójny układ i styl
widoków (por. sekcja~\ref{subsubsec:common-components}).

%-----------------------------------
\subsubsection{Profile}
\label{subsubsec:profile}

Zakładka \texttt{Profile} odpowiada za prezentację profilu użytkownika.
Została zbudowana z pięciu podstawowych komponentów:

\begin{itemize}
    \item \texttt{UserOwnProfile},
    \item \texttt{ProfileForViewer},
    \item \texttt{Profile},
    \item \texttt{ProfileStat},
    \item \texttt{MostPopularImage}.
\end{itemize}

\textbf{\texttt{UserOwnProfile}}
Ten komponent służy do wyświetlania profilu zalogowanego użytkownika.
Dane profilu są pobierane za pomocą \glslink{hook}{hooka} \texttt{useQuery} z \glslink{biblioteka}{biblioteki}
\glslink{tanstack-query}{TanStack Query}, który odwołuje się do odpowiedniego \glslink{endpoint}{endpointu} \glslink{backend}{backendu}.
Na czas pobierania danych wyświetlany jest komponent \texttt{LoadingSpinner}, a po poprawnym załadowaniu
profilu renderowany jest komponent \texttt{Profile} z przekazanym obiektem \texttt{userData}.

\textbf{\texttt{ProfileForViewer}}
Komponent wyświetla profil innego użytkownika.
Nazwa użytkownika pobierana jest ze ścieżki \gls{url} za pomocą \glslink{hook}{hooka} \texttt{useParams}, a dane profilu -- za
pomocą \texttt{useQuery}, które wysyła zapytanie do \glslink{backend}{backendu} z przekazaną nazwą.

Oprócz wspólnej części logicznej z widokiem własnego profilu wykorzystywane są:
\begin{itemize}
    \item przyciski \texttt{Button} przekazywane do \texttt{Profile} jako \texttt{children}, obsługujące
    akcje \texttt{follow} oraz zarządzanie znajomymi;
    \item komponent \glslink{modal}{\texttt{Modal}} do potwierdzania usunięcia znajomego lub zaprzestania obserwowania.
\end{itemize}

Zachowanie przycisków zależy od tego, czy użytkownik jest zalogowany, czy obserwuje dany profil oraz czy osoba
ta znajduje się na liście znajomych.
W przypadku błędu autoryzacji wyświetlany jest komunikat zachęcający do zalogowania, a inne błędy są mapowane
na komunikaty w komponencie \texttt{Notification}.

Relacje społecznościowe modyfikowane są przy użyciu kilku operacji po stronie \glslink{api}{API}, odpowiadających za dodawanie
lub usuwanie znajomych, zarządzanie listą obserwowanych oraz akceptowanie lub odrzucanie zaproszeń.
Po powodzeniu odświeżane są dane profilu użytkownika.
Jeżeli z \glslink{backend}{backendu} zwrócona zostanie informacja, że jest to profil zalogowanego użytkownika,
w \glslink{hook}{hooku} \texttt{useEffect} następuje przekierowanie na adres \texttt{/account/profile}.
W sytuacji, gdy profil o podanej nazwie użytkownika nie istnieje, renderowany jest komunikat:
\texttt{"No profile data available"}.

\textbf{\texttt{Profile}}
Jest to główny komponent odpowiedzialny za prezentację profilu i wykorzystywany zarówno w widoku własnym, jak i
w widoku innego użytkownika.
Cały widok opakowany jest w \texttt{AccountWrapper} z wariantem \texttt{PROFILE} (por. sekcja~\ref{subsubsec:common-components}).

Po lewej stronie wyświetlane jest zdjęcie profilowe.
Dla własnego profilu wykorzystano \glslink{hook}{hook} \texttt{useMutation}, który umożliwia zmianę zdjęcia.
Po najechaniu na nie kursorem pojawia się półprzezroczyste tło z ikoną edycji i napisem
\texttt{"Change profile photo."}, a kliknięcie uruchamia wybór nowego pliku.
Po udanej aktualizacji zdjęcia odświeżane są dane profilu oraz wyświetlany jest komunikat sukcesu.
W przypadku błędu prezentowany jest odpowiedni komunikat informujący o niepowodzeniu aktualizacji.

Po prawej stronie wyświetlana jest nazwa użytkownika oraz cztery kafelki \newline \texttt{ProfileStat},
przedstawiające liczbę znajomych, obserwowanych, obserwujących oraz zdjęć.
Kliknięcie w wybraną statystykę powoduje przejście do odpowiedniej sekcji społecznościowej (por. sekcja~\ref{subsubsec:social});
nawigacja realizowana jest przez \texttt{useNavigate} oraz akcję \texttt{socialAction.setType}.

Pod główną częścią profilu wyświetlany jest nagłówek \texttt{"most popular photos"}.
Jeżeli użytkownik posiada popularne zdjęcia, wyświetlane są maksymalnie cztery miniatury za pomocą komponentu \texttt{MostPopularImage}.
W przeciwnym razie prezentowany jest komunikat:
\texttt{"This user hasn't added any photos."} (dla profilu innego użytkownika) lub
\texttt{"You haven't added any photos."} (dla własnego profilu).

\textbf{\texttt{ProfileStat}}
Komponent prezentuje pojedynczą statystykę profilu (liczbę znajomych, obserwujących, obserwowanych oraz zdjęć) w formie kafelka,
który pełni rolę przycisku nawigacyjnego do odpowiedniej listy społecznościowej (por. sekcja~\ref{subsubsec:social}).

\textbf{\texttt{MostPopularImage}}
Komponent odpowiada za wyświetlenie pojedynczego zdjęcia wraz z liczbą polubień i wyświetleń.
Informacje umieszczone są na półprzezroczystym pasku w dolnej części kafelka.

%-----------------------------------
\subsubsection{Spots}
\label{subsubsec:spots}

Zakładka \texttt{Spots} odpowiada za prezentację list \glslink{spot}{spotów} przypisanych do różnych kategorii
(ulubione, planowane do odwiedzenia, odwiedzone itp.).
Składa się z trzech głównych komponentów:

\begin{itemize}
    \item \texttt{FavoriteSpots},
    \item \texttt{FavoriteSpotTile},
    \item \texttt{FavoriteSpotTags}.
\end{itemize}

\textbf{\texttt{FavoriteSpots}}
Jest to główny komponent zakładki.\newline
Wykorzystuje tablicę \texttt{menuTypes}, która definiuje dostępne typy list
(wszystkie, ulubione, planowane do odwiedzenia, odwiedzone i ocenione pozytywnie, odwiedzone i ocenione negatywnie),
a aktualnie wybrany typ przechowywany jest w stanie za pomocą \glslink{hook}{hooka} \texttt{useState}.

Dane \glslink{spot}{spotów} są pobierane za pomocą \glslink{hook}{hooka} \texttt{useInfiniteQuery}, który wysyła zapytania do
\glslink{backend}{backendu} z uwzględnieniem wybranego typu listy.
Mechanizm \glslink{infinite-scroll}{\textit{infinite scrolla}} realizowany jest przy użyciu \texttt{IntersectionObserver},
który nasłuchuje na element powiązany z referencją.
Widok opakowany jest w \texttt{AccountWrapper} z wariantem \texttt{FAVORITE\_SPOTS} oraz nagłówek
\texttt{AccountTitle} z tekstem \texttt{"spots lists"}.
Pod nagłówkiem renderowane są przyciski (komponent \texttt{Button}) do zmiany typu listy.

Podczas ładowania danych wyświetlany jest \newline \texttt{LoadingSpinner}.
Jeżeli użytkownik nie posiada żadnych \glslink{spot}{spotów} w wybranej liście, prezentowany jest komunikat:
\texttt{"You don't have any spots in your list."},
w przeciwnym razie renderowana jest lista kafelków \texttt{FavoriteSpotTile}.

\textbf{\texttt{FavoriteSpotTile}}
Komponent reprezentuje pojedynczy kafelek \glslink{spot}{spota}.
Wyświetlane są: zdjęcie, liczba wyświetleń, lokalizacja, średnia ocena
(komponent \texttt{Rate}), nazwa, tagi (komponent \texttt{FavoriteSpotTags}), krótki opis oraz przyciski
otwarcia \glslink{spot}{spota} na mapie i usunięcia z listy.

Do modyfikacji listy wykorzystywana jest \glslink{mutacja}{mutacja} obsługująca usuwanie spotów z danej kategorii.
Po powodzeniu odświeżane są dane listy, tak aby kafelek został usunięty z widoku.
Przed usunięciem wyświetlany jest modal potwierdzenia (komponent \glslink{modal}{\texttt{Modal}}), którego stan widoczności
kontrolowany jest za pomocą \glslink{hook}{hooka} \texttt{useBoolean}.

\textbf{\texttt{FavoriteSpotTags}}
Komponent prezentuje listę tagów \glslink{spot}{spota} jako niewielkie etykiety z nazwą tagu.

%-----------------------------------
\subsubsection{Photos}
\label{subsubsec:photos}

Zakładka \texttt{Photos} odpowiada za prezentację zdjęć dodanych przez użytkownika.
Korzysta z następujących głównych komponentów:

\begin{itemize}
    \item \texttt{Photos},
    \item \texttt{Media},
    \item \texttt{Photo}.
\end{itemize}

\textbf{\texttt{Photos}}
Komponent pełni rolę warstwy konfigurującej widok zdjęć.
Do zarządzania sortowaniem i filtrowaniem po dacie wykorzystywany jest \glslink{hook}{customowy hook} \texttt{useDateSortFilter},
który zwraca typ sortowania oraz zakres dat.
Dane są pobierane za pomocą \texttt{useInfiniteQuery}, które wysyła zapytania do \glslink{backend}{backendu} z uwzględnieniem aktualnych filtrów.
Mechanizm \glslink{infinite-scroll}{\textit{infinite scroll}} zrealizowano przy użyciu \texttt{IntersectionObserver}.

W przypadku błędu podczas pobierania zdjęć w \glslink{hook}{hooku} \texttt{useEffect} ustawiany jest komunikat
błędu w komponencie \texttt{Notification}.
Na końcu komponent przekazuje dane do \texttt{Media} z wariantem \texttt{AccountWrapperType.PHOTOS}.

\textbf{\texttt{Photo}}
Komponent reprezentuje pojedyncze zdjęcie użytkownika.
Zdjęcie prezentowane jest jako kwadratowy kafelek z cieniem, a w dolnej części na półprzezroczystym pasku wyświetlana
jest liczba polubień i wyświetleń wraz z odpowiednimi ikonami.

\textbf{\texttt{Media}}
Jest to uniwersalny komponent używany zarówno w zakładce \texttt{Photos}, jak i \texttt{Movies} (por. sekcja~\ref{subsubsec:movies}).
Widok jest opakowany w \texttt{AccountWrapper}, a nagłówek sekcji wyświetlany przez \texttt{AccountTitle} z tekstem zależnym od wariantu.

Poniżej nagłówka znajduje się panel \texttt{SortAndDateFilters} (por. sekcja~\ref{subsubsec:common-components})
umożliwiający zmianę sortowania oraz zakresu dat.
Lista mediów grupowana jest po dacie; dla każdej grupy wyświetlany jest \texttt{DateBadge} z datą, a poniżej siatka miniatur.
W zależności od wariantu wykorzystywany jest komponent \texttt{Photo} lub \texttt{Movie}.

Jeżeli użytkownik nie posiada żadnych mediów, wyświetlany jest komunikat:
\texttt{"You haven't added any photos."} lub
\texttt{"You haven't added any movies."}.
Podczas ładowania danych początkowych oraz kolejnych stron prezentowany jest \newline \texttt{LoadingSpinner}.

%-----------------------------------
\subsubsection{Movies}
\label{subsubsec:movies}

Zakładka \texttt{Movies} odpowiada za prezentację filmów dodanych przez użytkownika.
Pod względem struktury jest bardzo podobna do zakładki \texttt{Photos} (por. sekcja~\ref{subsubsec:photos}) i
korzysta z tych samych komponentów ogólnych:

\begin{itemize}
    \item \texttt{Movies},
    \item \texttt{Media},
    \item \texttt{Movie}.
\end{itemize}

\textbf{\texttt{Movies}}
Komponent konfiguruje widok filmów i przekazuje dane do \texttt{Media} z wariantem \texttt{AccountWrapperType.MOVIES}.
Ponownie wykorzystywany jest \glslink{hook}{hook} \texttt{useDateSortFilter}, a dane pobierane są za pomocą \texttt{useInfiniteQuery}
z uwzględnieniem wybranego sortowania i zakresu dat.
W przypadku błędu ustawiany jest komunikat w komponencie \texttt{Notification}.

\textbf{\texttt{Movie}}
Komponent reprezentuje pojedynczy film.
Do odtwarzania wykorzystano bibliotekę \texttt{react-player} oraz zestaw komponentów kontrolnych z
\glslink{biblioteka}{biblioteki} \texttt{media-chrome}.
\glslink{stan}{Stan} odtwarzania i ewentualnego błędu zarządzany jest za pomocą customowego \glslink{hook}{hooka} \texttt{useBoolean}.
Po zakończeniu odtwarzania film jest zatrzymywany, a w razie problemu z odczytem wyświetlany jest komunikat:
\texttt{"Failed to load the video. Please try again later."}.
Na filmie wyświetlany jest także pasek z liczbą polubień i wyświetleń.

%-----------------------------------
\subsubsection{Social}
\label{subsubsec:social}

Zakładka \texttt{Social} odpowiada za obsługę funkcji społecznościowych: znajomych, obserwowanych, obserwujących
oraz (w widoku innego użytkownika) zdjęć tego użytkownika.
Składa się z następujących głównych komponentów:

\begin{itemize}
    \item \texttt{UserOwnSocial},
    \item \texttt{SocialForViewer},
    \item \texttt{Social},
    \item \texttt{SocialCardList},
    \item \texttt{SocialCard},
    \item \texttt{SocialButton},
    \item \texttt{SearchFriendsList},
    \item \texttt{PhotosList},
    \item \texttt{FriendInvitesList}.
\end{itemize}

\textbf{\texttt{UserOwnSocial}}
Komponent odpowiada za wyświetlanie list społecznościowych zalogowanego użytkownika.
\newline Aktualnie wybrany typ listy (\texttt{FRIENDS}, \texttt{FOLLOWED}, \texttt{FOLLOWERS}) jest pobierany ze store
\glslink{redux}{Reduxa} za pomocą \texttt{useSelectorTyped}.
W celu uniknięcia powielania logiki zapytań użyto mapowania \texttt{queryConfigMap}, które łączy typ listy z odpowiednią
funkcją pobierającą.

Dane pobierane są za pomocą \texttt{useInfiniteQuery}, a kolejne strony ładowane przy wykorzystaniu \texttt{IntersectionObserver}.
Po zebraniu wszystkich stron dane są spłaszczane i przekazywane do uniwersalnego komponentu \texttt{Social}
z parametrem \newline \texttt{isSocialForViewer = false}.

\textbf{\texttt{SocialForViewer}}
Komponent realizuje analogiczną funkcjonalność, ale dla profilu innego użytkownika.
Nazwa użytkownika pobierana jest z parametrów ścieżki (\texttt{useParams}), a dane list społecznościowych oraz zdjęć
tego użytkownika -- z odpowiednich \glslink{endpoint}{endpointów} powiązanych w rozszerzonym \texttt{queryConfigMap}.

W zależności od typu listy pobierane są dane użytkowników (\texttt{SocialDto}) lub zdjęć (\texttt{DatedMediaGroup}).
Po spłaszczeniu danych całość przekazywana jest do komponentu \texttt{Social} z parametrem \newline \texttt{isSocialForViewer = true}.

\textbf{\texttt{Social}}
Jest to główny, uniwersalny komponent wykorzystywany zarówno dla widoku własnego, jak i cudzego profilu.
Widok opakowany jest w \newline \texttt{AccountWrapper} z wariantem \texttt{SOCIAL}, a nagłówek sekcji wyświetlany
przez \newline \texttt{AccountTitle} z tekstem \texttt{"social list"}.

W przypadku widoku własnego i wybranego typu \texttt{FRIENDS} po prawej stronie nagłówka pojawiają się dwa przyciski:
\begin{itemize}
    \item \texttt{"See friend invites"} -- otwiera modal z listą zaproszeń \newline (\texttt{FriendInvitesList});
    \item \texttt{"Add new friend"} -- otwiera modal z wyszukiwarką znajomych \newline (\texttt{SearchFriendsList}).
\end{itemize}

Poniżej wyświetlane jest menu z przyciskami \texttt{SocialButton}, które pozwalają przełączać się pomiędzy
listą znajomych, obserwowanych, obserwujących oraz (w widoku innego użytkownika) listą zdjęć.
Zmiana typu listy zapisywana jest w \glslink{store}{store} \glslink{redux}{Reduxa} za pomocą akcji \texttt{socialAction.setType}.

Dalsza część widoku zależy od typu listy:
\begin{itemize}
    \item dla \texttt{PHOTOS} używany jest komponent \texttt{PhotosList};
    \item dla pozostałych typów -- komponent \texttt{SocialCardList}.
\end{itemize}
Na dole listy znajduje się element powiązany z \texttt{loadMoreRef}, obsługujący \glslink{infinite-scroll}{\textit{infinite scroll}}, oraz
\texttt{LoadingSpinner} wyświetlany podczas pobierania kolejnych stron.

\textbf{\texttt{SocialCardList}}
Komponent wyświetla listę kafelków \texttt{SocialCard} oraz prezentuje odpowiednie komunikaty w przypadku pustych list.
Treść komunikatu zależy od typu listy oraz tego, czy jest to profil własny, czy innego użytkownika.
Dla trybu wyszukiwania znajomych (\texttt{isSearchFriend}) wyświetlany jest komunikat:
\texttt{"We can't find a user with this username."}.
W przypadku typu \texttt{POTENTIAL\_GROUP\_CHAT\_MEMBER} stosowany jest komunikat:
\texttt{"There are no other people in the world besides you."}.

\textbf{\texttt{SocialCard}}
Komponent reprezentuje pojedynczego użytkownika na listach społecznościowych.
Wyświetlane są: zdjęcie profilowe, nazwa użytkownika oraz zestaw przycisków akcji które służą do przejścia do profilu,
otwarcia prywatnego czatu oraz dodania lub usunięcia z listy znajomych bądź obserwowanych.

Relacje znajomości i obserwowania modyfikowane są za pomocą \glslink{mutacja}{mutacji} po stronie \glslink{api}{API},
odpowiedzialnych za zarządzanie listą znajomych oraz obserwowanych.
Po ich powodzeniu odświeżane są odpowiednie listy.
Otwarcie lub utworzenie prywatnego czatu realizowane jest przez osobną \glslink{mutacja}{mutację}, która zwraca obiekt czatu; następnie
jest on zapisywany w store \glslink{redux}{Reduxa}, ustawiany jako aktualnie wybrany, a aplikacja przełącza widok na \texttt{/chat}.

Przed usunięciem znajomego lub zakończeniem obserwowania wyświetlany jest modal potwierdzenia, którego stan zarządzany
jest przez \glslink{hook}{hook} \texttt{useBoolean}.
Treść komunikatu ma postać: \newline
\texttt{"Are you sure you want to remove \{username\} as a friend?"} lub \newline
\texttt{"Are you sure you want to remove \{username\} as a follower?"}.

\textbf{\texttt{SocialButton}}
Jest to uniwersalny przycisk wykorzystywany jako element nawigacji w \texttt{Social} oraz jako przycisk akcji w \texttt{SocialCard}.
Umożliwia wyróżnienie przycisku aktywnego oraz dostosowanie szerokości do treści.

\textbf{\texttt{SearchFriendsList}}
Komponent odpowiada za wyszukiwanie potencjalnych znajomych.
Wykorzystywany jest w modalu otwieranym z poziomu sekcji \texttt{Social}.
Zapytanie tekstowe jest \glslink{debounce}{debouncowane} za pomocą \glslink{hook}{hooka} \texttt{useDebounce}, a dane
pobierane są z użyciem \texttt{useInfiniteQuery}, które wysyła zapytania wyszukujące użytkowników po nazwie.
Interfejs zawiera przycisk zamknięcia, nagłówek, pole wyszukiwania oraz listę wyników w postaci \texttt{SocialCardList}.

\textbf{\texttt{PhotosList}}
Komponent wyświetla zdjęcia innego użytkownika, gdy z widoku profilu następuje przejście do listy jego zdjęć
(por. sekcja~\ref{subsubsec:profile}).
Zdjęcia grupowane są po dacie; dla każdej daty wyświetlany jest \texttt{DateBadge}, a poniżej siatka zdjęć z
wykorzystaniem komponentu \texttt{Photo}.
W przypadku braku zdjęć wyświetlany jest komunikat:
\texttt{"This user hasn't added any photos."}.

\textbf{\texttt{FriendInvitesList}}
Komponent służy do wyświetlania listy zaproszeń do znajomych otrzymanych przez użytkownika. \newline
Dane pobierane są za pomocą \texttt{useInfiniteQuery}, a kolejne strony ładowane są przy użyciu \texttt{IntersectionObserver}.
Zmiana statusu zaproszenia (zaakceptowanie lub odrzucenie) realizowana jest przez \glslink{mutacja}{mutację},
która wysyła odpowiednie żądanie do \glslink{backend}{backendu}.
Interfejs zawiera przycisk zamknięcia, nagłówek, listę zaproszeń oraz przyciski akceptacji i odrzucenia dla każdego elementu.
Kliknięcie w awatar użytkownika powoduje przejście do jego profilu.
Jeżeli brak zaproszeń, wyświetlany jest komunikat:
\texttt{"You don't have any invites yet"}.

%-----------------------------------
\subsubsection{Add spot}
\label{subsubsec:add-spot}

Zakładka \texttt{Add spot} służy do zarządzania \glslink{spot}{spotami} dodanymi przez użytkownika oraz do
dodawania nowych \glslink{spot}{spotów}.
Składa się z następujących komponentów:

\begin{itemize}
    \item \texttt{AddedSpot},
    \item \texttt{UploadButton},
    \item \texttt{SpotMap},
    \item \texttt{PolygonDrawer},
    \item \texttt{AddSpotModal},
    \item \texttt{AddSpotInput},
    \item \texttt{AddedSpotTile}.
\end{itemize}

Dodatkowo wykorzystywany jest schemat walidacji \texttt{spotSchema} utworzony z użyciem \glslink{biblioteka}{biblioteki} \texttt{zod}.

\textbf{\texttt{AddedSpot}}
Jest to główny komponent zakładki.
Odpowiada za wyświetlenie listy spotów dodanych przez użytkownika oraz za otwieranie \glslink{modal}{okna modalnego}
\texttt{AddSpotModal} służącego do dodania nowego \glslink{spot}{spota}.
Dane pobierane są za pomocą \texttt{useInfiniteQuery}, które pobiera kolejne strony wyników z \glslink{backend}{backendu},
a mechanizm \glslink{infinite-scroll}{\textit{infinite scroll}} zaimplementowano przy użyciu \texttt{IntersectionObserver}.

\glslink{stan}{Stan} widoczności modala zarządzany jest przez \glslink{hook}{hook} \texttt{useBoolean}.
Dodatkowo sprawdzana jest szerokość okna przeglądarki; formularz dodawania \glslink{spot}{spota} dostępny jest tylko
na ekranach o szerokości co najmniej 900~px.
W przypadku niespełnienia tego warunku wyświetlany jest komunikat informacyjny w komponencie \texttt{Notification}.

Widok opakowany jest w \texttt{AccountWrapper} z wariantem \texttt{ADD\_SPOT} oraz nagłówek \texttt{AccountTitle}.
Jeżeli użytkownik nie dodał żadnych \glslink{spot}{spotów}, wyświetlany jest komunikat:
\texttt{"You haven't added any spots yet."},
w przeciwnym razie renderowana jest lista kafelków \texttt{AddedSpotTile}.
Podczas ładowania danych prezentowany jest \texttt{LoadingSpinner}.

\textbf{\texttt{UploadButton}}
Komponent służy do wyboru i podglądu multimediów (zdjęć i filmów) dodawanych do \glslink{spot}{spota}.
W lokalnym stanie utrzymywana jest lista wybranych plików wraz z tymczasowymi adresami tworzonymi za pomocą
\newline \texttt{URL.createObjectURL}.
Użytkownik może usuwać pojedyncze elementy listy, a przy usuwaniu zwalniane są zasoby (\texttt{URL.revokeObjectURL}).
Dozwolone są wybrane typy plików graficznych oraz wideo.

\textbf{\texttt{SpotMap}}
Komponent wyświetla miniaturową mapę z zaznaczoną lokalizacją spota.
Wykorzystywana jest \glslink{biblioteka}{biblioteka} \texttt{maplibregl} oraz przygotowany styl mapy \newline \texttt{map\_style.json}.
Na mapie umieszczany jest marker w pozycji \glslink{spot}{spota}.

\textbf{\texttt{PolygonDrawer}}
Komponent umożliwia interaktywne narysowanie konturu \glslink{spot}{spota} na mapie.
Zbudowany jest w oparciu o \texttt{@vis.gl/react-maplibre}.
Kliknięcia w mapę dodają kolejne wierzchołki, a po dodaniu co najmniej trzech punktów prezentowany jest zamknięty wielokąt.
Użytkownik może cofnąć ostatni punkt lub zatwierdzić wielokąt, przekazując współrzędne do komponentu nadrzędnego.
W razie błędu walidacji konturu wyświetlany jest komunikat pod mapą.

\textbf{\texttt{AddSpotModal}}
Komponent odpowiada za dodawanie nowego \glslink{spot}{spota}.
W stanie przechowywany jest obiekt z danymi \glslink{spot}{spota}, komunikaty błędów walidacyjnych oraz aktualna pozycja mapy.
Adres budowany jest z pól formularza (ulica, miasto, region, kraj) i \glslink{debounce}{debouncowany} za pomocą
\glslink{hook}{hooka} \texttt{useDebounce}.
Na podstawie pełnego adresu \glslink{hook}{hook} \texttt{useQuery} pobiera współrzędne geograficzne, a mapa przesuwana
jest do odpowiedniej lokalizacji.

Walidacja danych formularza realizowana jest przy użyciu \texttt{spotSchema}.
W przypadku niepowodzenia walidacji wypełniane są pola błędów oraz wyświetlany jest komunikat:
\texttt{"Please fill in all fields correctly"}.
Po poprawnej walidacji wywoływana jest \glslink{mutacja}{mutacja} odpowiedzialna za dodanie \glslink{spot}{spota} po stronie \glslink{backend}{backendu}.
Po jej powodzeniu \glslink{modal}{modal} jest zamykany, formularz resetowany, dane listy odświeżane, a użytkownik otrzymuje
komunikat o poprawnym dodaniu \glslink{spot}{spota}.

Wyświetlenie \glslink{modal}{modala} zrealizowano za pomocą \texttt{createPortal}, co umożliwia umieszczenie go w dedykowanym elemencie
\gls{dom} (\texttt{<div id="modal">}).
Za animacje otwierania i zamykania odpowiadają komponenty \texttt{AnimatePresence} oraz
\texttt{motion.div} z \glslink{biblioteka}{biblioteki} motion.

\textbf{\texttt{AddSpotInput}}
Komponent odpowiada za wyświetlanie i obsługę listy pól formularza.
Korzysta z lokalnego \glslink{stan}{stanu} błędów oraz schematu \texttt{spotSchema} (za pomocą \texttt{pick}) do walidacji pojedynczych pól.
Do prezentacji pól wykorzystuje uniwersalny komponent \texttt{FormInput}.

\textbf{\texttt{AddedSpotTile}}
Komponent wyświetla pojedynczy kafelek z informacjami o dodanym \glslink{spot}{spocie}.
Prezentowane są: zdjęcie spota, mini mapa (\texttt{SpotMap}) z lokalizacją, nazwa, opis oraz adres (kraj, miasto, ulica).

%-----------------------------------
\subsubsection{Comments}
\label{subsubsec:comments}

Zakładka \texttt{Comments} wyświetla wszystkie komentarze dodane przez użytkownika pod \glslink{spot}{spotami}.
Składa się z dwóch głównych komponentów:

\begin{itemize}
    \item \texttt{Comments},
    \item \texttt{CommentsList}.
\end{itemize}

\textbf{\texttt{Comments}}
Komponent korzysta z \glslink{hook}{hooka} \texttt{useDateSortFilter}, analogicznie jak zakładki \texttt{Photos} i \texttt{Movies}.
Dane pobierane są za pomocą \texttt{useInfiniteQuery}, które pobiera kolejne strony komentarzy z \glslink{backend}{backendu},
a \glslink{infinite-scroll}{\textit{infinite scroll}} realizowany jest przez \texttt{IntersectionObserver}.

W przypadku błędu podczas pobierania komentarzy w \glslink{hook}{hooku} \texttt{useEffect} ustawiany jest komunikat błędu,
wyświetlany przez \texttt{Notification}.
Widok opakowany jest w \texttt{AccountWrapper} z wariantem \texttt{COMMENTS} i nagłówek \texttt{AccountTitle}.
Poniżej znajduje się panel filtrów \texttt{SortAndDateFilters}, a następnie lista komentarzy
pogrupowanych po dacie i nazwie \glslink{spot}{spota}.

Dla każdej grupy wyświetlany jest \texttt{DateBadge} z datą oraz informacją o \glslink{spot}{spocie},
a poniżej lista komentarzy renderowana przez \texttt{CommentsList}.
Jeżeli użytkownik nie dodał żadnych komentarzy, wyświetlany jest komunikat:
\texttt{"You haven't added any comments."}.
Podczas ładowania danych oraz kolejnych stron prezentowany jest \newline \texttt{LoadingSpinner}.

\textbf{\texttt{CommentsList}}
Komponent otrzymuje listę komentarzy i wyświetla je w postaci kafelków zawierających godzinę dodania oraz treść komentarza.

%-----------------------------------
\subsubsection{Settings}
\label{subsubsec:settings}

Zakładka \texttt{Settings} umożliwia użytkownikowi edycję podstawowych danych konta.
Została zrealizowana za pomocą dwóch głównych komponentów:

\begin{itemize}
    \item \texttt{Settings},
    \item \texttt{DisableInput}.
\end{itemize}

\textbf{\texttt{Settings}}
Komponent zarządza procesem zmiany danych użytkownika oraz wyświetlaniem odpowiednich formularzy.
Typ danych podlegających edycji (\texttt{USERNAME}, \texttt{EMAIL}, \texttt{PASSWORD}) przechowywany jest w stanie (\texttt{editType}).

Na początku, za pomocą \glslink{hook}{hooka} \texttt{useQuery}, pobierane są dane użytkownika (nazwa, e-mail, dostawca logowania).
Dopóki dane są ładowane, wyświetlany jest \newline \texttt{LoadingSpinner}.
Podstawowe informacje prezentowane są przez komponenty \newline \texttt{DisableInput}, które wyświetlają aktualne wartości
oraz przycisk \texttt{"Edit"}.

Po wyborze rodzaju edycji po prawej stronie pojawia się formularz dopasowany do aktualnego \texttt{editType}.
Walidacja realizowana jest przy użyciu schematów \texttt{baseSchemas} zdefiniowanych w \glslink{biblioteka}{bibliotece}
\texttt{zod}, a obsługę formularza zapewnia \texttt{react-hook-form} w połączeniu z \texttt{zodResolver}.
Dane wysyłane są do \glslink{backend}{backendu} za pomocą \glslink{mutacja}{mutacji}, która aktualizuje
odpowiednie ustawienia użytkownika.
W przypadku błędu prezentowany jest komunikat z odpowiedzi \glslink{backend}{backendu}; po sukcesie wyświetlany
jest komunikat o poprawnej zmianie ustawień oraz odświeżane są dane użytkownika.

Dla kont utworzonych przez formularz rejestracji (\texttt{Provider.NONE}) dostępna jest edycja nazwy użytkownika,
adresu e-mail oraz hasła.
Jeżeli konto zostało utworzone za pomocą zewnętrznego dostawcy (Google, GitHub), wyświetlany jest komunikat:
\texttt{"Your account was created via (Google/Github)"} oraz \newline
\texttt{"Your email address is (...) and cannot be changed."},
a formularze edycyjne są wówczas ukryte.

\textbf{\texttt{DisableInput}}
Komponent prezentuje pojedyncze pole z aktualną wartością danej informacji (nazwy użytkownika, e-mail lub hasła) w trybie
tylko do odczytu oraz przyciskiem \texttt{"Edit"}.
Dla pola hasła wyświetlane są jedynie gwiazdki.
Logika wyboru rodzaju edycji pozostaje w komponencie nadrzędnym.

%-----------------------------------
\subsubsection{Komponenty wspólne}
\label{subsubsec:common-components}

Panel użytkownika wykorzystuje także zestaw komponentów uniwersalnych, z których część była już wspomniana w
poprzednich sekcjach.
Dla przejrzystości poniżej zebrano ich krótkie opisy.

\textbf{\texttt{AccountWrapper}}
Komponent odpowiada za wspólny układ i podstawowe stylowanie poszczególnych zakładek panelu.
Przyjmuje wariant widoku i na jego podstawie stosuje odpowiednie klasy \glslink{tailwind-css}{Tailwind}, dzięki czemu
zachowana jest spójność wizualna przy jednoczesnej możliwości dostosowania przestrzeni danej sekcji.

\textbf{\texttt{AccountTitle}}
Komponent wyświetla nagłówek danej sekcji jako element \texttt{h1}.
Dodatkowo ustawia atrybut \texttt{data-testid}, co ułatwia testowanie widoków.

\textbf{\texttt{SortDropdown}}
Komponent udostępnia możliwość wyboru typu sortowania po dacie (malejąco/rosnąco).
\glslink{stan}{Stan} rozwinięcia listy zarządzany jest przez \glslink{hook}{hook} \texttt{useBoolean},
a animacje zrealizowano za pomocą \glslink{biblioteka}{biblioteki} \texttt{framer-motion}.
Komponent używany jest pośrednio poprzez \texttt{SortAndDateFilters}.

\textbf{\texttt{DateChooser}}
Jest to wrapper wokół komponentu \texttt{DatePicker} z \glslink{biblioteka}{biblioteki} Ant Design.
Służy do wyboru daty w polach \texttt{"From"} oraz \texttt{"To"} w panelu filtrowania.

\textbf{\texttt{SortAndDateFilters}}
Komponent łączy \texttt{SortDropdown} oraz dwa \newline \texttt{DateChooser}, tworząc panel sterujący sortowaniem i
filtrowaniem po dacie.
Wykorzystywany jest w zakładkach \texttt{Photos}, \texttt{Movies} oraz \texttt{Comments}.

\textbf{\texttt{DateBadge}}
Komponent służy do czytelnej prezentacji daty (formatowanej zgodnie z lokalizacją \texttt{"pl-PL"}) oraz
opcjonalnej dodatkowej informacji przekazywanej jako \texttt{children}.
Wykorzystywany jest do grupowania zdjęć, filmów oraz komentarzy po dacie.

Komponent \textbf{\texttt{LoadingSpinner}} jest wspólnym komponentem informującym o
trwającym ładowaniu danych i pojawia się
w większości widoków w \glslink{stan}{stanach} \texttt{isLoading} oraz \texttt{isFetchingNextPage}.

\textbf{\texttt{Notification}} (przywoływany pośrednio poprzez akcje \glslink{redux}{Reduxa} \newline
\texttt{notificationAction.addSuccess}, \texttt{addError} oraz \texttt{addInfo}) odpowiada za prezentację komunikatów
informacyjnych, błędów oraz potwierdzeń operacji.
