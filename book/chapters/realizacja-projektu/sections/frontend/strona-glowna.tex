%! Author = Mateusz
%! Date = 13/11/2025

\subsection{Strona główna}
\label{subsec:strona-glowna-frontend}

Jednym z głównych modółów aplikacji jest strona główna.
Pełni ona rolę głównej wyszukiwarki spotów, dzięki której użytkownik może w
łatwy sposób znaleść interesujące go lokacje.
Posiadan ona dwa tryby prosty i zaawansowany, dzięki przyciskowi na samej górze strnony można się łatwo
przełączyć między tymi widokami.
Na prostym znajduje się karuzela z 12 najpopularniejszymi spotami w całej aplikacji, użytkownik może tutaj
wyszukać spoty po lokalizacji ( kraj, region, miasto).
Na zaawansowanym widoku jest wyszukiwarka która filtruje po mieście, tagach oraz ocenie dodatkowo
sortuje po popularności i ocenach.

Strona główna została zbudowana z dwóch głównych komponentów HomePaga oraz AdvanceHomePage.
W skład prostej wersji wchodzą następujące komponenty:
\begin{itemize}
  \item Switch - służy do przełączania widoku między trybem podstawowym a zaawansowanym
  \item SearchBar - wyszukiwarka spotów
  \item Carousel - wyświetla najpopularnejsze spoty
  \item SearchSpotList - wyświetla wyszukane spoty
\end{itemize}

W skład zaawansowanej wersji wchodzą następujące komponenty:
\begin{itemize}
    \item Switch - służy do przełączania widoku między trybem podstawowym a zaawansowanym
    \item AdvanceSearchBar - wyszukiwarka spotów
    \item SearchSpotList - wyświetla wyszukane spoty
\end{itemize}

Komponent Switch zawiera w sobie dwa NavLink z biblioteki React Router dzięki temu można przełącznyć widok bez niepotrzebnych
odświeżeń strony.

W koponencie SearchBar po wpisaniu conajmniej 2 znaków pojawi się lista z podpowiedziami
do kraju, regionu oraz miasta w zależności od tego które aktualnie uzupełniamy.
Po pokazaniu się tej listy można wybrać interesujące nas miejsce dzięki czemu
wiemy w jakich lokalizacjach znajdują się spoty.

SearchSpotList zawiera listę komponentów SpotTile, LoadingSpiner oraz komunikat który
wyświetli się jeżeli nie zostanie wyświetlony żaden spot.

Spot tile zawiera informacje takie jak:
\begin{itemize}
    \item Zdjęcie spota
    \item Miasto w którym  się znajduje
    \item Nazwę
    \item Oceny i ich liczbę
    \item Tagi
    \item Podstawowe informacje pogodowe (temperatura i typ)
    \item Dwa przyciski, jeden do przejścia do szczegółów a drugi z informacją jak daleko znajduje
    się dany spot, po kliknięciu pokazuje go na mapie.
\end{itemize}

Komponent AdvanceSearchBar wygląda bardzo podobnie tylko z lokalizacji można podać tylko miasto,
dodatko jest mozżliwość dodania tagów z listy.
Jest też filtrowanie po ocenie oraz sortowanie po ocenie i popularności (komponenty Dropdown).

%\begin{figure}[H]
%    \centering
%    \includegraphics[width=1\textwidth]{attachments/implementacja-strona-glowna1}
%    \caption{Implementacja strony głównej}
%    \label{img:implementacja-strona-glowna1}
%\end{figure}