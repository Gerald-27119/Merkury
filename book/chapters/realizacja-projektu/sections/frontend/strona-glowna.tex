\subsection{Strona główna}
\label{subsec:strona-glowna-frontend}

Jednym z głównych modułów aplikacji jest strona główna.
Pełni ona rolę centralnej wyszukiwarki \gls{spot}ów, dzięki której użytkownik może w prosty sposób znaleźć interesujące go lokalizacje.
Strona działa w dwóch trybach: prostym oraz zaawansowanym (rys. \ref{img:implementacja-prosta-strona-glowna} i \ref{img:implementacja-zaawansowana-strona-glowna}).

\begin{figure}[H]
    \centering
    \includegraphics[width=1\textwidth]{attachments/implementacja-prosta-strona-glowna}
    \caption{Implementacja prostej wersji strony głównej}
    \label{img:implementacja-prosta-strona-glowna}
\end{figure}

\begin{figure}[H]
    \centering
    \includegraphics[width=1\textwidth]{attachments/implementacja-zaawansowana-strona-glowna}
    \caption{Implementacja zaawansowanej wersji strony głównej}
    \label{img:implementacja-zaawansowana-strona-glowna}
\end{figure}

Przełączanie pomiędzy tymi widokami odbywa się za pomocą przycisku umieszczonego w górnej części strony (rys. \ref{img:implementacja-switch}).

\begin{figure}[H]
    \centering
    \includegraphics[width=1\textwidth]{attachments/implementacja-switch}
    \caption{Implementacja komponentu do przełączania trybów}
    \label{img:implementacja-switch}
\end{figure}

W trybie prostym prezentowana jest karuzela (rys. \ref{img:implementacja-karuzela-spotow}) z dwunastoma najpopularniejszymi spotami w całej aplikacji.
Użytkownik może w tym miejscu wyszukiwać \gls{spot}y po lokalizacji (kraj, region, miasto).

\begin{figure}[H]
    \centering
    \includegraphics[width=1\textwidth]{attachments/implementacja-karuzela-spotow}
    \caption{Implementacja karuzeli z najpopularniejszymi spotami}
    \label{img:implementacja-karuzela-spotow}
\end{figure}

Widok zaawansowany udostępnia rozszerzoną wyszukiwarkę, która umożliwia filtrowanie wyników po mieście,
tagach oraz ocenie, a także ich sortowanie według popularności i średniej oceny (rys. \ref{img:implementacja-zaawansowana-strona-glowna}).

Strona główna została zbudowana z dwóch głównych komponentów: \texttt{HomePage} oraz \texttt{AdvanceHomePage}.
W skład prostej wersji wchodzą następujące komponenty:
\begin{itemize}
    \item \texttt{Switch} -- służy do przełączania widoku między trybem podstawowym a zaawansowanym,
    \item \texttt{SearchBar} -- podstawowa wyszukiwarka \gls{spot}ów,
    \item \texttt{Carousel} -- wyświetla najpopularniejsze \gls{spot}y,
    \item \texttt{SearchSpotList} -- wyświetla wyszukane \gls{spot}y.
\end{itemize}

W skład zaawansowanej wersji wchodzą następujące komponenty:
\begin{itemize}
    \item \texttt{Switch} -- służy do przełączania widoku między trybem podstawowym a zaawansowanym,
    \item \texttt{AdvanceSearchBar} -- zaawansowana wyszukiwarka \gls{spot}ów,
    \item \texttt{SearchSpotList} -- wyświetla wyszukane \gls{spot}y.
\end{itemize}

Komponent \texttt{Switch} (rys. \ref{img:implementacja-switch}) zawiera dwa elementy \texttt{NavLink} z \glslink{biblioteka}{Biblioteki} React Router,
co pozwala na przełączanie widoków bez konieczności przeładowywania całej strony.

W komponencie \texttt{SearchBar} (rys. \ref{img:implementacja-prosta-wyszukiwarka}) po wpisaniu co najmniej dwóch znaków wyświetlana jest lista podpowiedzi
dla kraju, regionu oraz miasta, w zależności od aktualnie uzupełnianego pola.
Po pojawieniu się listy użytkownik może wybrać interesującą go lokalizację,
co ułatwia określenie, w jakich miejscach znajdują się dostępne \gls{spot}y.

\begin{figure}[H]
    \centering
    \includegraphics[width=1\textwidth]{attachments/implementacja-prosta-wyszukiwarka}
    \caption{Implementacja prostej wyszukiwarki}
    \label{img:implementacja-prosta-wyszukiwarka}
\end{figure}

Komponent \texttt{SearchSpotList} (rys. \ref{img:implementacja-lista-spotow-strona-glowna}) odpowiada za prezentację wyników wyszukiwania.
Został w nim zaimplementowany mechanizm przewijania nieskończonego (\glslink{infinity-scroll}{infinite scroll}),
który automatycznie pobiera kolejne strony wyników w momencie, gdy użytkownik zbliża się do końca listy.
Wykorzystuje on listę komponentów \texttt{SpotTile}, a także komponent \texttt{LoadingSpinner}
oraz komunikat informujący o braku wyników, jeżeli nie zostanie odnaleziony żaden \gls{spot}.

\begin{figure}[H]
    \centering
    \includegraphics[width=1\textwidth]{attachments/implementacja-lista-spotow-strona-glowna}
    \caption{Implementacja listy do wyświetlania spotów}
    \label{img:implementacja-lista-spotow-strona-glowna}
\end{figure}

Komponent \texttt{SpotTile} zawiera następujące informacje:
\begin{itemize}
    \item zdjęcie \gls{spot}a,
    \item miasto, w którym się znajduje,
    \item nazwę \gls{spot}a,
    \item ocenę oraz liczbę ocen,
    \item tagi,
    \item podstawowe informacje pogodowe (temperatura i typ pogody),
    \item dwa przyciski: jeden prowadzący do widoku szczegółów \gls{spot}a oraz drugi informujący, jak daleko znajduje się dany \gls{spot}; po kliknięciu przycisku lokalizacja \gls{spot}a jest prezentowana na mapie.
\end{itemize}

Komponent \texttt{AdvanceSearchBar} jest zbliżony wyglądem i strukturą do wersji podstawowej,
jednak w polu lokalizacji można podać wyłącznie miasto.
Dodatkowo dostępna jest możliwość dodawania tagów z przygotowanej listy.
Wyszukiwarka umożliwia także filtrowanie po ocenie oraz sortowanie wyników według oceny i popularności
z wykorzystaniem komponentów typu \texttt{Dropdown}.

Oba widoki (\texttt{HomePage} i \texttt{AdvanceHomePage}) współdzielą część komponentów,
między innymi \texttt{Switch} oraz \texttt{SearchSpotList}.
Dzięki temu kod odpowiedzialny za wyświetlanie listy wyników jest zdefiniowany w jednym miejscu,
a zmiany w sposobie prezentacji spotów wymagają modyfikacji tylko w komponentach współdzielonych.
