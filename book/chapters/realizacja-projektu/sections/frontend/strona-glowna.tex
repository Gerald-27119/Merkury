%! Author = Mateusz
%! Date = 13/11/2025

\subsection{Strona główna}
\label{subsec:strona-glowna-frontend}

Jednym z głównych modułów aplikacji jest strona główna.
Pełni ona rolę centralnej wyszukiwarki spotów, dzięki której użytkownik może w prosty i intuicyjny sposób znaleźć interesujące go lokalizacje.
Strona działa w dwóch trybach: prostym oraz zaawansowanym.
Przełączanie pomiędzy tymi widokami odbywa się za pomocą przycisku umieszczonego w górnej części strony.

W trybie prostym prezentowana jest karuzela z dwunastoma najpopularniejszymi spotami w całej aplikacji.
Użytkownik może w tym miejscu wyszukiwać spoty po lokalizacji (kraj, region, miasto).
Widok zaawansowany udostępnia rozszerzoną wyszukiwarkę, która umożliwia filtrowanie wyników po mieście,
tagach oraz ocenie, a także ich sortowanie według popularności i średniej oceny.

Strona główna została zbudowana z dwóch głównych komponentów: \texttt{HomePage} oraz \texttt{AdvanceHomePage}.
W skład prostej wersji wchodzą następujące komponenty:
\begin{itemize}
    \item \texttt{Switch} -- służy do przełączania widoku między trybem podstawowym a zaawansowanym,
    \item \texttt{SearchBar} -- podstawowa wyszukiwarka spotów,
    \item \texttt{Carousel} -- wyświetla najpopularniejsze spoty,
    \item \texttt{SearchSpotList} -- wyświetla wyszukane spoty.
\end{itemize}

W skład zaawansowanej wersji wchodzą następujące komponenty:
\begin{itemize}
    \item \texttt{Switch} -- służy do przełączania widoku między trybem podstawowym a zaawansowanym,
    \item \texttt{AdvanceSearchBar} -- zaawansowana wyszukiwarka spotów,
    \item \texttt{SearchSpotList} -- wyświetla wyszukane spoty.
\end{itemize}

Komponent \texttt{Switch} zawiera dwa elementy \texttt{NavLink} z biblioteki React Router,
co pozwala na przełączanie widoków bez konieczności przeładowywania całej strony.

W komponencie \texttt{SearchBar} po wpisaniu co najmniej dwóch znaków wyświetlana jest lista podpowiedzi
dla kraju, regionu oraz miasta, w zależności od aktualnie uzupełnianego pola.
Po pojawieniu się listy użytkownik może wybrać interesującą go lokalizację,
co ułatwia określenie, w jakich miejscach znajdują się dostępne spoty.

Komponent \texttt{SearchSpotList} odpowiada za prezentację wyników wyszukiwania.
W zależności od stanu zapytania do API wyświetlana jest lista komponentów \texttt{SpotTile}, komponent \texttt{LoadingSpinner}
lub komunikat informujący o braku wyników.

Komponent \texttt{SpotTile} zawiera następujące informacje:
\begin{itemize}
    \item zdjęcie spota,
    \item miasto, w którym się znajduje,
    \item nazwę spota,
    \item ocenę oraz liczbę ocen,
    \item tagi,
    \item podstawowe informacje pogodowe (temperatura i typ pogody),
    \item dwa przyciski: jeden prowadzący do widoku szczegółów spota oraz drugi informujący, jak daleko znajduje się dany spot; po kliknięciu przycisku lokalizacja spota jest prezentowana na mapie.
\end{itemize}

Komponent \texttt{AdvanceSearchBar} jest zbliżony wyglądem do wersji podstawowej,
jednak w polu lokalizacji można podać wyłącznie miasto.
Dodatkowo dostępna jest możliwość dodawania tagów z przygotowanej listy.
Użytkownik może również filtrować wyniki po ocenie oraz sortować je według oceny i
popularności z wykorzystaniem komponentów typu \texttt{Dropdown}.

% Implementację widoku strony głównej przedstawiono na rysunku~\ref{img:implementacja-strona-glowna1}.
%
%\begin{figure}[H]
%    \centering
%    \includegraphics[width=1\textwidth]{attachments/implementacja-strona-glowna1}
%    \caption{Implementacja strony głównej}
%    \label{img:implementacja-strona-glowna1}
%\end{figure}
