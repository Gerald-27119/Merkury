%! Author = mateusz
%! Date = 20/10/2025

\subsection{Struktura aplikacji}
\label{subsec:struktura-aplikacji}

Architektura aplikacji frontendowej została
zaprojektowana w strukturze folder-by-type, polegający na podziale kodu względem
typu (komponenty, strony, modele, itd.).
Każdy plik jest w katalogu odpowiadającemu
jego zastosowaniu.



Głównym elementem aplikacji jest mechanizm
routingu z biblioteki React Router. Definiuje on ścieżki do każdej
funkcjonalności aplikacji. Dzięki temu aplikacja posiada płynną nawigację
między różnymi widokami bez konieczności przeładowywania strony.

%\putimage{Implementacja routera 1}{attachments/router1}{img:router1}{0.2\textwidth}
%\putimage{Implementacja routera 2}{attachments/router2}{img:router2}{0.2\textwidth}

\begin{figure}[H]
    \centering
    \includegraphics[width=0.7\textwidth]{attachments/router1}
    \caption{Implementacja routera 1}
    \label{img:router1}
\end{figure}

\begin{figure}[H]
    \centering
    \includegraphics[width=0.7\textwidth]{attachments/router2}
    \caption{Implementacja routera 2}
    \label{img:router2}
\end{figure}

Został zastosowany również wzorzec Protected
Route, służy on do zabezpieczenia niektórych tras przed dostępem użytkowników
niezalogowanych. W pliku router.tsx w głównym katalogu projektu w metodzie
createBrowserRouter widocznej na rysunku(numerRysunku) została ona opakowana w
komponent ProtectedRoute który pełni rolę bramki. Przykładem takiej ścieżki
może być trasa /account/profile która prowadzi do profilu zalogowanego
użytkownika. Jeżeli na podaną ścieżkę wejdzie użytkownik niezalogowany zostanie
on automatycznie przeniesiony na stronę główną.

\putimage{Implementacja komponentu zabezpieczjącego}{attachments/protected-route}{img:protected-route}{1\textwidth}