%! Author = mateusz
%! Date = 20/10/2025

\subsection{Struktura aplikacji}
\label{subsec:struktura-aplikacji}

Architektura aplikacji frontendowej została zaprojektowana w strukturze
\gls{folder-by-type}, która polega na podziale kodu według typu zasobu
(komponenty, strony, modele itd.). Każdy plik znajduje się w katalogu
odpowiadającym jego przeznaczeniu, co widać na rysunkach
\ref{img:struktura-frontend1} oraz \ref{img:struktura-frontend2}.

\begin{figure}[H]
    \centering
    \includegraphics[width=0.7\textwidth]{attachments/struktura-frontend1}
    \caption{Struktura katalogów (1)}
    \label{img:struktura-frontend1}
\end{figure}

\begin{figure}[H]
    \centering
    \includegraphics[width=0.7\textwidth]{attachments/struktura-frontend2}
    \caption{Struktura katalogów (2)}
    \label{img:struktura-frontend2}
\end{figure}

Głównym elementem aplikacji jest mechanizm routingu oparty na bibliotece
React Router. Definiuje on ścieżki do poszczególnych funkcjonalności
aplikacji. Dzięki temu możliwa jest płynna nawigacja między różnymi
widokami bez konieczności przeładowywania strony.

\begin{figure}[H]
    \centering
    \includegraphics[width=0.7\textwidth]{attachments/router1}
    \caption{Implementacja routera (1)}
    \label{img:router1}
\end{figure}

\begin{figure}[H]
    \centering
    \includegraphics[width=0.7\textwidth]{attachments/router2}
    \caption{Implementacja routera (2)}
    \label{img:router2}
\end{figure}

W projekcie zastosowano również wzorzec \gls{protected-route}, który służy
do zabezpieczania wybranych tras przed dostępem użytkowników
niezalogowanych. W pliku \texttt{router.tsx}, znajdującym się w głównym
katalogu projektu, w konfiguracji przekazywanej do funkcji
\texttt{createBrowserRouter} (rysunki \ref{img:router1} oraz
\ref{img:router2}), wybrane ścieżki zostały opakowane w komponent
\texttt{ProtectedRoute}. Komponent ten pełni rolę bramki
(rysunek~\ref{img:protected-route}).

Przykładem takiej chronionej ścieżki jest trasa \texttt{/chat},
prowadząca do modułu czatu dostępnego wyłącznie dla zalogowanych
użytkowników. Jeśli niezalogowany użytkownik spróbuje uzyskać dostęp do
tej ścieżki, zostanie automatycznie przekierowany na stronę główną.

\begin{figure}[H]
    \centering
    \includegraphics[width=1\textwidth]{attachments/protected-route}
    \caption{Implementacja komponentu bramki (\texttt{ProtectedRoute})}
    \label{img:protected-route}
\end{figure}
