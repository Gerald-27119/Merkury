%! Author = mateusz
%! Date = 20/10/2025

\subsection{Struktura aplikacji}
\label{subsec:struktura-aplikacji}

Architektura aplikacji frontendowej została
zaprojektowana w strukturze \gls{folder-by-type}, polegający na podziale kodu względem
typu (komponenty, strony, modele, itd.).
Każdy plik jest w katalogu odpowiadającemu jego zastosowaniu co widać na rysunkach
\ref{img:struktura-frontend1} i \ref{img:struktura-frontend2}.

\begin{figure}[H]
    \centering
    \includegraphics[width=0.7\textwidth]{attachments/struktura-frontend1}
    \caption{Struktura katalogów 1}
    \label{img:struktura-frontend1}
\end{figure}

\begin{figure}[H]
    \centering
    \includegraphics[width=0.7\textwidth]{attachments/struktura-frontend2}
    \caption{Struktura katalogów 2}
    \label{img:struktura-frontend2}
\end{figure}

Głównym elementem aplikacji jest mechanizm
routingu z \gls{bibliote}ki React Router. Definiuje on ścieżki do każdej
funkcjonalności aplikacji. Dzięki temu aplikacja posiada płynną nawigację
między różnymi widokami bez konieczności przeładowywania strony.

\begin{figure}[H]
    \centering
    \includegraphics[width=0.7\textwidth]{attachments/router1}
    \caption{Implementacja routera 1}
    \label{img:router1}
\end{figure}

\begin{figure}[H]
    \centering
    \includegraphics[width=0.7\textwidth]{attachments/router2}
    \caption{Implementacja routera 2}
    \label{img:router2}
\end{figure}

Został zastosowany również \gls{wzorzec} \gls{protected-route}, służy on do zabezpieczenia
niektórych tras przed dostępem użytkowników
niezalogowanych. W pliku router.tsx w głównym katalogu projektu w metodzie
createBrowserRouter widocznej na rysunkach \ref{img:router1} i \ref{img:router2} została ona opakowana w
komponent ProtectedRoute który pełni rolę bramki \ref{img:protected-route}. Przykładem takiej ścieżki
może być trasa /chat która prowadzi do chatu dla zalogowanych użytkowników.
Jeżeli na podaną ścieżkę wejdzie użytkownik niezalogowany zostanie
on automatycznie przeniesiony na stronę główną.

\begin{figure}[H]
    \centering
    \includegraphics[width=1\textwidth]{attachments/protected-route}
    \caption{Implementacja komponentu bramki}
    \label{img:protected-route}
\end{figure}