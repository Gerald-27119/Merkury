%! Author = kacper
%! Date = 08/01/2026

\subsection{Forum}
\label{subsec:forum-frontend}

W niniejszym rozdziale przedstawiono implementację forum po stronie \glslink{frontend}{frontendu}.
Opisane zostały najważniejsze \glslink{react-component}{komponenty} dla tego modułu projektu.

%! Author = kacper
%! Date = 08.01.2026

\subsubsection{Układ strony}
\label{subsubsec:uklad-strony}

\texttt{ForumLayout} jest \glslink{react-component}{komponentem} nadrzędnym wszystkich podstron forum (rys. \ref{img:forum-layout-1}~\ref{img:forum-layout-2}).
Zapewnia on spójną strukturę strony składającej się z trzech głównych sekcji: lewego panelu bocznego, centralnej części z główną treścią oraz prawego panelu bocznego.
Komponent ten stanowi wspólny układ dla wszystkich tras forum i jest używany w \glslink{routing}{routingu} aplikacji.

\texttt{ForumLayout} obejmuje:
\begin{itemize}
    \item lewy panel boczny, zawierający przycisk dodawania nowego posta oraz panel kategorii i tagów;
    \item część centralną, w której wyświetlana jest właściwa zawartość podstron forum (lista postów, widok posta, wyniki wyszukiwania itp.);
    \item prawy panel boczny, zawierający pasek wyszukiwania oraz panel popularnych postów z ostatniego miesiąca.
\end{itemize}

Komponent jest wykorzystywany w \glslink{routing}{routingu} aplikacji poprzez zagnieżdżenie tras forum, co umożliwia dostęp do paneli bocznych oraz modali na wszystkich podstronach forum.

Dane kategorii, tagów oraz popularnych postów pobierane są przy użyciu \glslink{hook}{hooka} \texttt{useQuery} z \glslink{tanstack-query}{biblioteki TanStack Query}, który komunikuje się z odpowiednimi endpointami \glslink{backend}{backendu}.
Stany ładowania oraz błędów przekazywane są do komponentów potomnych.

Przycisk dodawania posta (\texttt{AddPostButton}) sprawdza stan zalogowania użytkownika.
W przypadku braku autoryzacji wyświetlany jest komunikat informacyjny zachęcający do zalogowania się.
Obsługa modali tworzenia posta oraz zgłaszania treści realizowana jest globalnie za pomocą stanu Redux, dzięki czemu formularze te są dostępne niezależnie od aktualnie wyświetlanej podstrony forum.

\texttt{ForumLayout} zawiera również dwa modale: \texttt{ForumAddPostModal} oraz \texttt{ForumReportModal}, które są dostępne na wszystkich podstronach forum.
Służą one do obsługi formularzy tworzenia i edycji postów oraz zgłaszania postów i komentarzy.

\begin{figure}[H]
    \centering
    \includegraphics[width=1\textwidth]{attachments/implementacja-frontendu/forum/forum_layout_1}
    \caption{Komponent ForumLayout (1)}
    \label{img:forum-layout-1}
\end{figure}

\begin{figure}[H]
    \centering
    \includegraphics[width=1\textwidth]{attachments/implementacja-frontendu/forum/forum_layout_2}
    \caption{Komponent ForumLayout (2)}
    \label{img:forum-layout-2}
\end{figure}
%! Author = kacper
%! Date = 08.01.2026

\subsubsection{Lewy panel boczny}
\label{subsubsec:lewy-panel-boczny}

Komponent \texttt{ForumAddPostModal} jest globalnym modalem tworzonym w ramach \texttt{ForumLayout} i odpowiada za dodawanie oraz edycję postów na forum.
Jest dostępny na wszystkich podstronach forum i otwierany programowo (np. z poziomu przycisku „Add post” lub opcji edycji posta).

Modal działa w dwóch trybach:
\begin{itemize}
    \item \textbf{create} – tworzenie nowego posta,
    \item \textbf{edit} – edycja istniejącego posta, z wypełnionymi danymi początkowymi.
\end{itemize}

\texttt{ForumAddPostModal}:
\begin{itemize}
    \item wykorzystuje React Portal (\texttt{createPortal}) do umieszczania modala poza głównym drzewem \glslink{react-component}{komponentów},
    \item wyświetla półprzezroczyste tło (overlay), które zamyka modal po kliknięciu,
    \item pokazuje komponent \texttt{PostForm}, odpowiedzialny za formularz posta.
\end{itemize}

Komponent \texttt{PostForm} zawiera formularz dodawania i edycji posta.

Formularz został zaimplementowany z użyciem:
\begin{itemize}
    \item \texttt{React Hook Form} – do zarządzania stanem formularza,
    \item \texttt{Zod} – do walidacji danych wejściowych (schema \texttt{PostFormSchema}).
\end{itemize}

Obsługiwane pola:
\begin{itemize}
    \item tytuł posta,
    \item kategoria (pojedynczy wybór),
    \item tagi (wielokrotny wybór),
    \item treść posta (\texttt{richTextEditor}).
\end{itemize}

W trybie edycji formularz jest inicjalizowany danymi istniejącego posta.

Pole \texttt{content} wykorzystuje kontrolowany edytor tekstu (\texttt{ControlledEditor}) oparty o \glslink{tiptap}{bibliotekę Tiptap}, działający w wariancie \glslink{modal}{modalnym}.

Formularz zawiera również pola wyboru kategorii i tagów zaimplementowane przy użyciu komponentu \texttt{ControlledSelect}, który integruje React Hook Form z \glslink{react-select}{biblioteką react-select}, zapewniając obsługę wyszukiwania oraz wyboru pojedynczego i wielokrotnego.

Komponenty z prefiksem \texttt{Controlled} wykorzystują mechanizm \texttt{Controller} z biblioteki React Hook Form, co umożliwia integrację niestandardowych komponentów wejściowych (np. z zewnętrznej \glslink{biblioteka}{biblioteki}) z systemem walidacji i zarządzania stanem formularza, mimo że nie udostępniają one standardowych właściwości pól formularza.

Obrazy wstawiane w edytorze są tymczasowo przechowywane w treści jako dane Base64.
Przed wysłaniem formularza treść HTML jest analizowana, a osadzone obrazy są konwertowane do plików, asynchronicznie przesyłane do zewnętrznego storage (\glslink{azure-blob-storage}{Azure}), a następnie ich lokalne referencje są zastępowane docelowymi \glslink{url}{adresami URL}.

W przypadku błędu podczas przesyłania obrazów formularz wyświetla odpowiedni komunikat walidacyjny i blokuje zapis posta.


Komponent \texttt{ControlledEditor} pełni rolę warstwy integracyjnej pomiędzy React Hook Form a \glslink{rich-text-editor}{edytorem rich text} opartym o \glslink{tiptap}{bibliotekę Tiptap}.
Został zaimplementowany jako komponent kontrolowany przy użyciu \texttt{Controller}, co umożliwia poprawne przekazywanie wartości, obsługę zdarzeń \texttt{onChange} / \texttt{onBlur} oraz integrację z systemem walidacji formularza.

Edytor obsługuje różne warianty wizualne (\texttt{default}, \texttt{modal}), które wpływają na jego wysokość, układ oraz stylowanie, co pozwala na jego wielokrotne użycie w różnych kontekstach interfejsu.

Sam komponent \texttt{Tiptap} został skonfigurowany od podstaw z wykorzystaniem zestawu rozszerzeń, m.in.:
\begin{itemize}
    \item \texttt{StarterKit} – zapewniający podstawowe elementy edycji (akapity, listy, formatowanie tekstu),
    \item \texttt{Placeholder} – obsługa tekstu zastępczego dla pustej treści,
    \item \texttt{TextAlign} – wyrównanie tekstu,
    \item \texttt{Image} – obsługa obrazów (z włączoną obsługą danych Base64),
    \item \texttt{FileHandler} – umożliwiający wklejanie i przeciąganie plików graficznych bezpośrednio do edytora.
\end{itemize}

Dodawanie obrazów realizowane jest poprzez przeciąganie plików, wklejanie ze schowka lub dedykowany przycisk w pasku narzędzi.
Na etapie edycji obrazy są osadzane w treści jako dane Base64, a typ pliku jest walidowany przed dodaniem do dokumentu.
Informacje o błędach lub poprawnym dodaniu pliku są przekazywane do formularza poprzez callbacki \texttt{onFileError} oraz \texttt{onFileSuccess}, co pozwala na wyświetlanie komunikatów walidacyjnych na poziomie pola formularza.

Edytor posiada własny pasek narzędzi (\texttt{MenuBar}) zsynchronizowany ze stanem edytora, oferujący m.in.:
\begin{itemize}
    \item formatowanie tekstu (\texttt{bold}, \texttt{italic}, \texttt{underline}),
    \item listy uporządkowane i nieuporządkowane,
    \item nagłówki,
    \item wyrównanie tekstu,
    \item linki,
    \item wstawianie obrazów.
\end{itemize}

Całość stanowi w pełni kontrolowany, rozszerzalny edytor treści, spójnie zintegrowany z logiką formularza i dalszym procesem przetwarzania treści posta.


\texttt{PostCategoriesTagsPanel} jest komponentem lewego panelu bocznego forum, odpowiedzialnym za prezentację listy kategorii oraz tagów postów.
Komponent obsługuje stany ładowania i błędów danych pobieranych z \glslink{backend}{backendu}.
W trakcie pobierania wyświetlany jest komponent \texttt{SkeletonPostCategoryTag}, natomiast w przypadku błędu pokazywany jest komponent \texttt{Error}.

Po poprawnym załadowaniu danych panel wyświetla dwie sekcje: listę kategorii oraz listę tagów.
Każda sekcja prezentuje skróconą listę elementów, wyświetlanych odpowiednio za pomocą komponentów \texttt{PostCategory} oraz \texttt{PostTag}.

Na końcu obu list dostępny jest przycisk \texttt{ExpansionButton}, który umożliwia przejście do dedykowanych podstron zawierających pełne, alfabetyczne listy kategorii lub tagów.
Nawigacja realizowana jest z wykorzystaniem \glslink{hook}{hooka} \texttt{useNavigate}.
W przypadku braku danych wyświetlany jest komunikat informujący o braku dostępnych elementów.
%! Author = kacper
%! Date = 08.01.2026

\subsubsection{Prawy panel boczny}
\label{subsubsec:prawy-panel-boczny}

\glslink{react-component}{Komponent} \texttt{ForumSearchBar} (rys. \ref{img:searchbar-1} - \ref{img:searchbar-4}) odpowiada za zaawansowane wyszukiwanie postów na forum i jest wykorzystywany w \texttt{ForumLayout}.
Logika wyszukiwania oparta jest o lokalny \glslink{stan}{stan} \glslink{react-component}{komponentu}, którego strukturę definiuje interfejs \texttt{SearchState}.
Interfejs ten opisuje komplet aktywnych filtrów wyszukiwania, takich jak:
\begin{itemize}
    \item fraza wyszukiwania
    \item nazwa autora
    \item kategoria
    \item tagi
    \item zakres dat publikacji (od–do)
\end{itemize}

Dla wyboru kategorii i tagów wykorzystywany jest \glslink{react-component}{komponent} \texttt{SelectWithSearch}, oparty o \glslink{react-select}{bibliotekę react-select}, umożliwiający wyszukiwanie oraz wybór pojedynczy i wielokrotny.

Pola odpowiedzialne za filtrowanie po dacie („From” / „To”) zostały zaimplementowane przy użyciu \glslink{react-component}{komponent} \texttt{DatePicker} z \glslink{biblioteka}{biblioteki} Ant Design (\texttt{antd}), z wykorzystaniem \texttt{dayjs} do obsługi i formatowania dat.
\glslink{react-component}{Komponent} dat dynamicznie dostosowuje wygląd do aktualnego trybu jasnego i ciemnego aplikacji.

Po zatwierdzeniu formularza dane wyszukiwania są mapowane na parametry \glslink{url}{URL} (\texttt{URLSearchParams}), a użytkownik jest przekierowywany do dedykowanego widoku wyników (\texttt{/forum/search}).
Dzięki temu wyszukiwanie jest w pełni odtwarzalne na podstawie adresu \glslink{url}{URL}.

Dodatkowo pola tekstowe wykorzystują \texttt{HintedSearchField}, który umożliwia podpowiedzi wyszukiwania w trakcie wpisywania.

\begin{figure}[H]
    \centering
    \includegraphics[width=1\textwidth]{attachments/implementacja-frontendu/forum/searchbar_1}
    \caption{Komponent ForumSearchBar (1/4)}
    \label{img:searchbar-1}
\end{figure}

\begin{figure}[H]
    \centering
    \includegraphics[width=1\textwidth]{attachments/implementacja-frontendu/forum/searchbar_2}
    \caption{Komponent ForumSearchBar (2/4)}
    \label{img:searchbar-2}
\end{figure}

\begin{figure}[H]
    \centering
    \includegraphics[width=1\textwidth]{attachments/implementacja-frontendu/forum/searchbar_3}
    \caption{Komponent ForumSearchBar (3/4)}
    \label{img:searchbar-3}
\end{figure}

\begin{figure}[H]
    \centering
    \includegraphics[width=1\textwidth]{attachments/implementacja-frontendu/forum/searchbar_4}
    \caption{Komponent ForumSearchBar (4/4)}
    \label{img:searchbar-4}
\end{figure}

\glslink{react-component}{Komponent} \texttt{ForumSearch} odpowiada za prezentację listy postów spełniających kryteria wyszukiwania przekazane w parametrach \glslink{url}{URL}.

Na podstawie \texttt{useSearchParams} tworzony jest obiekt \texttt{PostSearchRequestDto}, który zawiera wszystkie aktywne filtry (fraza, kategoria, tagi, daty, autor).
Dane te są następnie wykorzystywane w zapytaniu do \glslink{api}{API} po stronie \glslink{backend}{backendu}.

Pobieranie wyników wyszukiwania realizowane jest przy użyciu \glslink{hook}{hooka} \texttt{useInfiniteQuery}, co umożliwia stronicowanie danych oraz dynamiczne ładowanie kolejnych stron.
Kolejne wyniki są pobierane automatycznie z wykorzystaniem mechanizmu \glslink{infinite-scroll}{nieskończonego przewijania}.

Widok obsługuje również sortowanie wyników, a wybrana opcja sortowania jest częścią klucza zapytania, co powoduje ponowne pobranie danych przy jego zmianie.

Wyniki wyszukiwania są wyświetlane przy użyciu \glslink{react-component}{komponentu} \texttt{ForumPostsPage}, który odpowiada za:
\begin{itemize}
    \item wyświetlanie listy postów
    \item obsługę sortowania
    \item prezentację informacji o liczbie znalezionych wyników
    \item komunikaty o braku dalszych rezultatów
\end{itemize}

Po wykonaniu wyszukiwania \glslink{react-component}{komponent} \texttt{SearchResults} (rys. \ref{img:search-results}) prezentuje liczbę znalezionych postów oraz czytelne podsumowanie zastosowanych filtrów.
\glslink{react-component}{Komponent} ten znajduje się w \texttt{ForumPostList} i wyświetlany jest warunkowo, tylko gdy zapytanie wyszukiwania zawiera jakiekolwiek kryteria.

\begin{figure}[H]
    \centering
    \includegraphics[width=1\textwidth]{attachments/implementacja-frontendu/forum/search_result}
    \caption{Komponent SearchResults}
    \label{img:search-results}
\end{figure}

W przypadku błędów zapytania użytkownik otrzymuje stosowną notyfikację, natomiast podczas ładowania wyświetlane są \glslink{react-component}{komponenty} typu skeleton, zapewniające spójne \glslink{ux}{UX}.

\texttt{TrendingPostPanel} wyświetla listę popularnych postów za pomocą \texttt{TrendingPostList}.
Każdy element w liście używa \glslink{react-component}{komponentu} \texttt{TrendingPost}.
%! Author = Mateusz
%! Date = 13/11/2025

\subsection{Strona główna}
\label{subsec:strona-glowna-frontend}

Jednym z głównych modółów aplikacji jest strona główna.
Pełni ona rolę głównej wyszukiwarki spotów, dzięki której użytkownik może w
łatwy sposób znaleść interesujące go lokacje.
Posiadan ona dwa tryby prosty i zaawansowany, dzięki przyciskowi na samej górze strnony można się łatwo
przełączyć między tymi widokami.
Na prostym znajduje się karuzela z 12 najpopularniejszymi spotami w całej aplikacji, użytkownik może tutaj
wyszukać spoty po lokalizacji ( kraj, region, miasto).
Na zaawansowanym widoku jest wyszukiwarka która filtruje po mieście, tagach oraz ocenie dodatkowo
sortuje po popularności i ocenach.

Strona główna została zbudowana z dwóch głównych komponentów HomePaga oraz AdvanceHomePage.
W skład prostej wersji wchodzą następujące komponenty:
\begin{itemize}
  \item Switch - służy do przełączania widoku między trybem podstawowym a zaawansowanym
  \item SearchBar - wyszukiwarka spotów
  \item Carousel - wyświetla najpopularnejsze spoty
  \item SearchSpotList - wyświetla wyszukane spoty
\end{itemize}

W skład zaawansowanej wersji wchodzą następujące komponenty:
\begin{itemize}
    \item Switch - służy do przełączania widoku między trybem podstawowym a zaawansowanym
    \item AdvanceSearchBar - wyszukiwarka spotów
    \item SearchSpotList - wyświetla wyszukane spoty
\end{itemize}

Komponent Switch zawiera w sobie dwa NavLink z biblioteki React Router dzięki temu można przełącznyć widok bez niepotrzebnych
odświeżeń strony.

W koponencie SearchBar po wpisaniu conajmniej 2 znaków pojawi się lista z podpowiedziami
do kraju, regionu oraz miasta w zależności od tego które aktualnie uzupełniamy.
Po pokazaniu się tej listy można wybrać interesujące nas miejsce dzięki czemu
wiemy w jakich lokalizacjach znajdują się spoty.

SearchSpotList zawiera listę komponentów SpotTile, LoadingSpiner oraz komunikat który
wyświetli się jeżeli nie zostanie wyświetlony żaden spot.

Spot tile zawiera informacje takie jak:
\begin{itemize}
    \item Zdjęcie spota
    \item Miasto w którym  się znajduje
    \item Nazwę
    \item Oceny i ich liczbę
    \item Tagi
    \item Podstawowe informacje pogodowe (temperatura i typ)
    \item Dwa przyciski, jeden do przejścia do szczegółów a drugi z informacją jak daleko znajduje
    się dany spot, po kliknięciu pokazuje go na mapie.
\end{itemize}

Komponent AdvanceSearchBar wygląda bardzo podobnie tylko z lokalizacji można podać tylko miasto,
dodatko jest mozżliwość dodania tagów z listy.
Jest też filtrowanie po ocenie oraz sortowanie po ocenie i popularności (komponenty Dropdown).

%\begin{figure}[H]
%    \centering
%    \includegraphics[width=1\textwidth]{attachments/implementacja-strona-glowna1}
%    \caption{Implementacja strony głównej}
%    \label{img:implementacja-strona-glowna1}
%\end{figure}
%! Author = kacper
%! Date = 08.01.2026

\subsubsection{Szczegóły posta}
\label{subsubsec:szczegoly-posta}

\glslink{react-component}{Komponent} \texttt{ForumThread} odpowiada za wyświetlanie pełnej treści pojedynczego posta wraz z komentarzami.
Otrzymuje jego identyfikator z parametrów \glslink{url}{URL} i wykorzystuje \glslink{hook}{hooka} \texttt{useQuery} do pobrania szczegółów (\texttt{DetailedPost}) oraz \texttt{useInfiniteQuery} do stronicowanego pobierania komentarzy.

\texttt{ForumThread} składa się z następujących \glslink{react-component}{komponentów}:
\begin{itemize}
    \item \texttt{ReturnButton} – umożliwia powrót do poprzedniej strony
    \item \texttt{FollowPostButton} – pozwala obserwować post, dostępny tylko dla użytkowników zalogowanych lub niebędących autorem
    \item \texttt{DetailedPost} – prezentuje pełną treść posta, kategorię, tagi, liczbę wyświetleń, komentarzy oraz przycisk dodawania komentarza
    \item \texttt{PostCommentForm} – formularz dodawania komentarza, otwierany po kliknięciu w przycisk w \texttt{DetailedPost}.
    Widoczność kontrolowana jest lokalnym \glslink{stan}{stanem} \glslink{react-component}{komponentu}
    \item \texttt{PostCommentList} – lista komentarzy do posta, obsługująca sortowanie i paginację
\end{itemize}

\texttt{ForumThread} obsługuje następujące akcje użytkownika:
\begin{itemize}
    \item dodawanie komentarzy (\texttt{PostCommentForm})
    \item obserwowanie posta (\texttt{FollowPostButton})
    \item edycję i usuwanie postów oraz zgłaszanie treści poprzez komponenty \texttt{DetailedPost} i współdzielone menu kontekstowe
\end{itemize}

Akcje te realizowane są przy użyciu \glslink{mutacja}{mutacji} \glslink{api}{API} z \glslink{tanstack-query}{biblioteki TanStack Query}, z automatycznym odświeżaniem danych po wykonaniu akcji.
W przypadku braku autoryzacji użytkownika wyświetlane są odpowiednie komunikaty zachęcające do zalogowania się.

\glslink{react-component}{Komponent} \texttt{DetailedPost} odpowiada za prezentację pełnej treści pojedynczego posta w widoku wątku forum.
Otrzymuje obiekt \texttt{PostDetails} oraz stany ładowania i błędu, a także funkcje do obsługi akcji użytkownika (dodawanie komentarza, follow, edycja, usuwanie, zgłaszanie, głosowanie, udostępnianie).

\texttt{DetailedPost} składa się z następujących \glslink{react-component}{komponentów}:
\begin{itemize}
    \item \texttt{ForumContentHeader} – nagłówek posta z informacjami o autorze i dacie publikacji.
    Kliknięcie w autora przenosi do jego profilu
    \item \texttt{PostMetaData} – wyświetla kategorię i tagi posta
    \item \texttt{DetailedPostContent} – prezentuje tytuł i treść posta
    Treść (\texttt{content}) jest wyświetlana jako HTML, co pozwala na zachowanie formatowania oraz wstawianie elementów multimedialnych
    \item \texttt{DetailedPostActions} – panel akcji użytkownika, zawiera wszystkie interakcje związane z postem (głosowanie, obserwowanie, edycja, usuwanie, raportowanie, udostępnianie, dodawanie komentarza)
\end{itemize}

\texttt{DetailedPost} obsługuje akcje przy użyciu \glslink{mutacja}{mutacji} \glslink{api}{API} z \glslink{tanstack-query}{biblioteki TanStack Query}, z automatycznym odświeżaniem odpowiednich zapytań.
Dostępność poszczególnych akcji zależy od stanu zalogowania użytkownika oraz jego uprawnień.
W przypadku braku autoryzacji wyświetlane są komunikaty informujące o konieczności logowania.

\glslink{react-component}{Komponent} \texttt{DetailedPostActions} wyświetla panel akcji użytkownika dla posta w widoku szczegółowym.
Zawiera:
\begin{itemize}
    \item Głosy (\texttt{upvote}/\texttt{downvote}) – za pomocą \glslink{react-component}{komponentu} \texttt{ActionIconWithCount}, pokazującego liczbę głosów i aktywność użytkownika
    \item Komentarze – licznik komentarzy z ikoną
    \item Udostępnianie – przycisk kopiujący adres \glslink{url}{URL} posta do schowka
    \item Menu kontekstowe (\texttt{ForumPostMenu}) – pozwala autorowi edytować lub usuwać post, a innym użytkownikom obserwować (\texttt{follow}) lub zgłaszać (\texttt{report})
    \item Przycisk dodania komentarza (\texttt{AddCommentButton}) – otwiera formularz dodawania komentarza w \glslink{react-component}{komponencie} \texttt{PostCommentForm}
\end{itemize}

Menu kontekstowe (\texttt{ForumContentMenu}) obsługuje dynamicznie dostępne akcje w zależności od tego, czy użytkownik jest autorem posta oraz czy obserwuje dany post.
Widoczność menu i jego zamykanie po kliknięciu poza obszarem realizowane są przy użyciu \glslink{hook}{hooka} \texttt{useClickOutside}.

Wszystkie akcje użytkownika, takie jak głosowanie, follow, report, edycja czy usuwanie, realizowane są za pomocą \glslink{mutacja}{mutacji} \glslink{api}{API} z \glslink{tanstack-query}{biblioteki TanStack Query}, z automatycznym odświeżaniem odpowiednich zapytań i wyświetlaniem komunikatów o sukcesie lub braku autoryzacji.

\glslink{react-component}{Komponent} \texttt{PostCommentList} odpowiada za wyświetlanie listy komentarzy przypisanych do posta lub do innego komentarza (w przypadku odpowiedzi).
Obsługuje zarówno komentarze główne, jak i zagnieżdżone odpowiedzi.

Funkcjonalności \glslink{react-component}{komponentu} obejmują:
\begin{itemize}
    \item wyświetlanie listy komentarzy przy użyciu \glslink{react-component}{komponentu} \texttt{PostComment}
    \item obsługę stanów ładowania i błędu
    \item sortowanie komentarzy (tylko dla komentarzy głównych) przy użyciu \newline \texttt{ForumSortDropdown}
    \item animacje przejść pomiędzy \glslink{stan}{stanami} (ładowanie / lista / brak danych) z wykorzystaniem \glslink{biblioteka}{biblioteki} \texttt{Framer Motion}
\end{itemize}

W zależności od wartości flagi \texttt{areReplies}, \glslink{react-component}{komponent}:
\begin{itemize}
    \item dla komentarzy głównych – umożliwia sortowanie
    \item dla odpowiedzi – prezentuje listę bez sortowania i w uproszczonym układzie wizualnym
\end{itemize}

\glslink{react-component}{Komponent} \texttt{PostComment} odpowiada za prezentację pojedynczego komentarza oraz obsługę wszystkich interakcji użytkownika z nim związanych.

Składa się z następujących elementów:
\begin{itemize}
    \item \texttt{ForumContentHeader} – nagłówek z informacjami o autorze i dacie publikacji; kliknięcie przenosi do profilu autora
    \item \texttt{PostCommentContent} – treść komentarza
    \item \texttt{PostCommentActions} – panel akcji (głosowanie, edycja, usuwanie, odpowiedź, zgłoszenie)
\end{itemize}

\glslink{react-component}{Komponent} obsługuje również:
\begin{itemize}
    \item edycję komentarza
    \item dodawanie odpowiedzi
    \item usuwanie
    \item głosowanie (\texttt{upvote}/\texttt{downvote})
    \item zgłaszanie komentarza (\texttt{report})
\end{itemize}

Wszystkie operacje modyfikujące dane realizowane są przy użyciu \glslink{mutacja}{mutacji} \glslink{api}{API} z \glslink{tanstack-query}{biblioteki TanStack Query}, z odpowiednim odświeżaniem zapytań dla komentarzy głównych lub odpowiedzi.

Komentarze w systemie forum obsługują zagnieżdżone odpowiedzi, co zostało zaimplementowane w formie \glslink{rekurencja}{rekurencyjnego} użycia \glslink{react-component}{komponentów}:
\begin{itemize}
    \item \texttt{PostComment} może zawierać \texttt{PostCommentList} z odpowiedziami
    \item \texttt{PostCommentList} wyświetla kolejne komponenty \texttt{PostComment}
    \item każdy komentarz może posiadać własną listę odpowiedzi, które są ładowane i wyświetlane niezależnie
\end{itemize}

Odpowiedzi do komentarza:
\begin{itemize}
    \item są pobierane stronicowo (\texttt{infinite query})
    \item ładowane dopiero po rozwinięciu listy odpowiedzi
    \item mogą być dalej rozwijane, zachowując tę samą strukturę komponentów
\end{itemize}



\subsubsection{Mechanizm nieskończonego przewijania}

W widokach forum prezentujących listy danych ładowane stronicowo (lista postów, wyniki wyszukiwania oraz komentarze) zastosowano mechanizm nieskończonego przewijania (\glslink{infinite-scroll}{infinite scroll}).
We wszystkich tych przypadkach wykorzystano \glslink{intersection-observer}{intersection-observer}, który obserwuje element referencyjny umieszczony na końcu listy.
Po jego pojawieniu się w obszarze widoku wyzwalane jest pobranie kolejnej strony danych, o ile dostępne są dalsze wyniki i nie trwa aktualnie inne zapytanie.

\subsubsection{Komponenty Wspólne}

Forum wykorzystuje kilka \glslink{react-component}{komponentów} uniwersalnych, o których była już mowa we wcześniejszych podrozdziałach.
Dla przejrzystości poniżej zebrano ich krótkie opisy:

\paragraph{ForumLayout}
\glslink{react-component}{Komponent} odpowiadający za wspólny układ wszystkich podstron forum.
Zapewnia trójkolumnową strukturę widoku, obejmującą lewy i prawy panel boczny oraz centralną część z główną treścią.
Jest wykorzystywany w \glslink{routing}{routingu} aplikacji jako komponent nadrzędny dla tras forum, co umożliwia dostęp do paneli bocznych oraz globalnych \glslink{modal}{modali} na każdej podstronie.

\paragraph{ForumSortDropdown}
\glslink{react-component}{Komponent} pozwalający wybrać sposób sortowania listy postów lub komentarzy (rys. \ref{img:forum-sort-dropdown}).
Obsługuje zmianę kryteriów sortowania, takich jak data publikacji czy liczba interakcji i jest wykorzystywany w widokach listy postów, wyników wyszukiwania oraz komentarzy.

\begin{figure}[H]
    \centering
    \includegraphics[width=1\textwidth]{attachments/implementacja-frontendu/forum/forum_sort_dropdown}
    \caption{Komponent ForumSortDropdown}
    \label{img:forum-sort-dropdown}
\end{figure}

\paragraph{ControlledEditor}
\glslink{react-component}{Komponent} stanowiący warstwę integracyjną pomiędzy React Hook Form a \glslink{rich-text-editor}{edytorem rich text} opartym o \glslink{tiptap}{bibliotekę Tiptap} (rys. \ref{img:controlled-editor}).
Umożliwia kontrolowane zarządzanie wartością pola treści, obsługę walidacji oraz przekazywanie zdarzeń edytora do formularza.
Jest wykorzystywany w formularzach tworzenia i edycji postów oraz komentarzy.

\begin{figure}[H]
    \centering
    \includegraphics[width=1\textwidth]{attachments/implementacja-frontendu/forum/controlled_editor}
    \caption{Komponent ControlledEditor}
    \label{img:controlled-editor}
\end{figure}

\paragraph{SelectWithSearch}
\glslink{react-component}{Komponent} oparty o \glslink{react-select}{bibliotekę react-select}, umożliwiający wybór pojedynczych lub wielu wartości z listy z obsługą wyszukiwania (rys. \ref{img:select-with-search-1}, \ref{img:select-with-search-2}).
Wykorzystywany jest m.in. w formularzach tworzenia i edycji postów oraz w panelu wyszukiwania forum do wyboru kategorii i tagów.

\begin{figure}[H]
    \centering
    \includegraphics[width=1\textwidth]{attachments/implementacja-frontendu/forum/select_with_search_1}
    \caption{Komponent SelectWithSearch (1/2)}
    \label{img:select-with-search-1}
\end{figure}

\begin{figure}[H]
    \centering
    \includegraphics[width=1\textwidth]{attachments/implementacja-frontendu/forum/select_with_search_2}
    \caption{Komponent SelectWithSearch (2/2)}
    \label{img:select-with-search-2}
\end{figure}

\paragraph{ForumContentMenu}
Współdzielony \glslink{react-component}{komponent} menu kontekstowego, wykorzystywany w postach oraz komentarzach (rys. \ref{img:forum-content-menu-1}, \ref{img:forum-content-menu-2}).
Dynamicznie dostosowuje dostępne akcje w zależności od typu treści, stanu zalogowania użytkownika oraz jego uprawnień (np. autor, obserwator).
Obsługuje operacje takie jak edycja, usuwanie, obserwowanie oraz zgłaszanie treści.

\begin{figure}[H]
    \centering
    \includegraphics[width=1\textwidth]{attachments/implementacja-frontendu/forum/forum_content_menu_1}
    \caption{Komponent ForumContentMenu (1/2)}
    \label{img:forum-content-menu-1}
\end{figure}

\begin{figure}[H]
    \centering
    \includegraphics[width=1\textwidth]{attachments/implementacja-frontendu/forum/forum_content_menu_2}
    \caption{Komponent ForumContentMenu (2/2)}
    \label{img:forum-content-menu-2}
\end{figure}

\paragraph{Notification}
\glslink{react-component}{Komponent} odpowiedzialny za wyświetlanie komunikatów systemowych, takich jak informacje o powodzeniu lub błędzie wykonywanych operacji.
Wykorzystywany jest globalnie w aplikacji, w szczególności w odpowiedzi na \glslink{mutacja}{mutacje} \glslink{api}{API}.

