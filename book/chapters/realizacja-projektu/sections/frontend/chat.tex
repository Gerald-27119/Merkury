%! Author = Adam
%! Date = 30/12/2025

\subsection{Czat}
\label{subsec:chat-frontend}

\newcommand{\chatimplfig}[1]{./attachments/implementacja-frontendu/czat/#1}

W niniejszym rozdziale przedstawiono implementację modułu czatu po stronie \glslink{frontend}{frontendu}. \newline
Czat stanowi jeden z kluczowych elementów aplikacji, zapewniając komunikację w wariancie rozmów prywatnych
oraz grupowych.

Na główne \glslink{react-component}{komponenty} modułu składają się:
\begin{itemize}
    \item \texttt{ChatsPage} -- główny kontener układu,
    \item \texttt{ChatList} oraz \texttt{ListedChat} -- \glslink{react-component}{komponenty} odpowiedzialne za prezentację i obsługę listy rozmów,
    \item \texttt{ChatTopBar}, \texttt{ChatMessagingWindow}, \texttt{ChatBottomBar} -- elementy składające się na okno konwersacji,
    \item \texttt{EmojiGifWindowWrapper} wraz z \texttt{EmojiWindow} i \texttt{GifWindow} -- panel wyboru \glslink{emoji}{emoji}/\glslink{gif}{GIF-ów},
    \item \texttt{GroupChatParticipantsSideBar} -- panel listy uczestników rozmowy grupowej.
\end{itemize}

\subsubsection{ChatsPage -- główny kontener modułu}

\begin{itemize}
    \item \glslink{react-component}{Komponent} \texttt{ChatsPage} jest nadrzędnym kontenerem widoku czatu.
    \item Odpowiada za rozłożenie interfejsu na trzy sekcje:
    \begin{enumerate}
        \item lewy panel z listą rozmów,
        \item centralne okno konwersacji,
        \item opcjonalny panel boczny widoczny wyłącznie dla czatów grupowych.
    \end{enumerate}
\end{itemize}

\glslink{stan}{Stan} wyboru rozmowy oraz widoczność panelu bocznego pobierane są ze \glslink{stan}{stanu} globalnego \glslink{redux}{Redux}.

\subsubsection{Lista czatów}

\paragraph{Komponent ChatList}

\glslink{react-component}{Komponent} \texttt{ChatList} odpowiada za pobieranie oraz prezentację listy rozmów użytkownika.
Dane są ładowane z \glslink{backend}{backendu} przy użyciu \glslink{hook}{hook'a}
\texttt{useInfiniteQuery} z \glslink{tanstack-query}{TanStack Query}.
Zaimplementowano mechanizm \glslink{infinite-scroll}{przewijania nieskończonego},
który doładowuje kolejne strony wyników, gdy \glslink{sentinel}{element-strażnik} pojawi się na ekranie użytkownika.
W tym celu użyto mechanizmu \glslink{intersection-observer}{Intersection Observer}.

Po każdorazowym pobraniu strony dane są mapowane do lokalnego formatu i zapisywane w \glslink{redux}{Redux},
dzięki czemu \glslink{stan}{stan} listy rozmów jest współdzielony z innymi komponentami modułu.
Dodatkowo przy pierwszym renderowaniu, jeżeli użytkownik nie ma wybranego czatu,
ustawiana jest domyślna rozmowa.

W trakcie ładowania wyświetlane są elementy typu \glslink{skeleton-loader}{skeleton loader},
natomiast w przypadku doładowywania kolejnych stron prezentowany jest wskaźnik ładowania.

\paragraph{Komponent ListedChat}

\glslink{react-component}{Komponent} \texttt{ListedChat} reprezentuje pojedynczy element listy rozmów.
Wyświetla podstawowe informacje o czacie: nazwę, awatar, treść ostatniej wiadomości oraz czas jej wysłania.
\glslink{react-component}{Komponent} wizualnie rozróżnia stan:
\begin{itemize}
    \item aktualnie wybranego czatu,
    \item czatu posiadającego nowe, nieodczytane wiadomości,
    \item czatu nieaktywnego.
\end{itemize}

Kliknięcie w element listy aktualizuje identyfikator wybranego czatu w \glslink{redux}{Redux} i
jednocześnie czyści znacznik nowych wiadomości dla danej rozmowy.

\subsubsection{Okno konwersacji}

\paragraph{Komponent ChatContent}

\glslink{react-component}{Komponent} \texttt{ChatContent} pełni rolę kontenera na \glslink{react-component}{komponenty} okna konwersacji.
Jego zadaniem jest pobranie z \glslink{redux}{Redux} identyfikatora aktualnie wybranego czatu, a następnie odczytanie
pełnych danych rozmowy na podstawie selektora \newline \texttt{selectChatById}.

\texttt{ChatContent} składa okno rozmowy z trzech części:
paska nagłówka (\texttt{ChatTopBar}), listy wiadomości (\texttt{ChatMessagingWindow}) oraz paska tworzenia wiadomości
\newline (\texttt{ChatBottomBar}).

\paragraph{Komponent ChatTopBar}

\glslink{react-component}{Komponent} \texttt{ChatTopBar} pełni rolę nagłówka konwersacji.
W zależności od typu rozmowy (prywatna/grupowa) udostępnia różne akcje:
dla czatu prywatnego umożliwia przejście do profilu rozmówcy, natomiast w czacie grupowym
udostępnia przełącznik panelu bocznego oraz operacje zarządzania rozmową.
Część funkcjonalności realizowana jest w \glslink{modal}{oknach modalnych}
(tworzenie nowego czatu grupowego, edycja czatu lub dodawanie uczestników).

\paragraph{Komponent ChatMessagingWindow}

\glslink{react-component}{Komponent} \texttt{ChatMessagingWindow} odpowiada za wyświetlanie wiadomości
w obrębie wybranej konwersacji. Dane pobierane z \glslink{backend}{backendu} przy użyciu
\texttt{useInfiniteQuery}.
Wiadomości są renderowane w układzie „od dołu” (ostatnia wiadomość na końcu),
a doładowywanie starszych fragmentów odbywa się po przewinięciu w górę.

Dodatkowo komponent porządkuje wiadomości w czytelny sposób:
\begin{itemize}
    \item wstawia separatory dat przy zmianie dnia,
    \item grupuje wiadomości tego samego autora wysłane w krótkim odstępie czasu.
\end{itemize}

Okno konwersacji współpracuje również z aktualizacją danych w czasie rzeczywistym.
Po odebraniu nowej wiadomości jest ona natychmiast wyświetlana.

\paragraph{Komponent ChatBottomBar}

\glslink{react-component}{Komponent} \texttt{ChatBottomBar} realizuje wysyłanie wiadomości tekstowych,
załączników oraz interakcję z oknami wyboru \glslink{emoji}{emoji} i \glslink{gif}{GIF-ów}.
Wysyłanie wiadomości w czasie rzeczywistym odbywa się z użyciem \glslink{websocket}{WebSocket}
oraz protokołu \glslink{stomp}{STOMP}.
Obsługa załączników obejmuje wybór wielu plików oraz ich podgląd (jeżeli są zdjęciem).

\subsubsection{Okna emoji i GIF}

\paragraph{Komponent EmojiGifWindowWrapper}

\glslink{react-component}{Komponent} \texttt{EmojiGifWindowWrapper} stanowi warstwę pośrednią,
która w zależności od aktualnego trybu wyświetla jedno z dwóch okien:
\texttt{EmojiWindow} lub \texttt{GifWindow}.

\paragraph{Komponent EmojiWindow}

\glslink{react-component}{Komponent} \texttt{EmojiWindow} integruje zewnętrzny selektor \glslink{emoji}{emoji}
i umożliwia wyszukiwanie oraz wybór emotikonów.
Po kliknięciu \glslink{emoji}{emoji}, \glslink{react-component}{komponent} dopisuje go do aktualnie budowanej treści wiadomości.

\paragraph{Komponent GifWindow}

\glslink{react-component}{Komponent} \texttt{GifWindow} umożliwia wyszukiwanie oraz wysyłanie \glslink{gif}{GIF-ów}
z wykorzystaniem integracji z dostawcą \glslink{tenor}{Tenor} po stronie \glslink{backend}{backendu}.
Widok udostępnia dwa scenariusze:
\begin{itemize}
    \item prezentację popularnych kategorii,
    \item wyszukiwanie \glslink{gif}{GIF-ów} po frazie.
\end{itemize}

Wyszukiwanie realizowane jest przy użyciu \texttt{useInfiniteQuery}.
W przypadku kliknięcia w wybrany \glslink{gif}{GIF} jego adres \glslink{url}{URL} jest przesyłany jako treść wiadomości
przez \glslink{websocket}{WebSocket}/\glslink{stomp}{STOMP}, a okno selektora zostaje zamknięte.

\subsubsection{Lista uczestników czatu grupowego}

\paragraph{Komponent GroupChatParticipantsSideBar}

\glslink{react-component}{Komponent} \texttt{GroupChatParticipantsSideBar} jest panelem bocznym,
który pojawia się wyłącznie dla rozmów grupowych.
Panel prezentuje listę uczestników czatu wraz z awatarem i nazwą użytkownika,
a kliknięcie w wybraną osobę przekierowuje do jej profilu.
