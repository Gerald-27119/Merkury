%! Author = Adam
%! Date = 30/12/2025

\subsection{Czat}
\label{subsec:chat-frontend}

\newcommand{\chatimplfig}[1]{./attachments/implementacja-frontendu/czat/#1}

W niniejszym rozdziale przedstawiono implementację modułu czatu po stronie \glslink{frontend}{frontendu}. \newline
Czat stanowi jeden z kluczowych elementów aplikacji, zapewniając komunikację w wariancie rozmów prywatnych
oraz grupowych.

Na główne \glslink{react-component}{komponenty} modułu składają się:
\begin{itemize}
    \item \texttt{ChatsPage} -- główny kontener układu,
    \item \texttt{ChatList} oraz \texttt{ListedChat} -- \glslink{react-component}{komponenty} odpowiedzialne za prezentację i obsługę listy rozmów,
    \item \texttt{ChatTopBar}, \texttt{ChatMessagingWindow}, \texttt{ChatBottomBar} -- elementy składające się na okno konwersacji,
    \item \texttt{EmojiGifWindowWrapper} wraz z \texttt{EmojiWindow} i \texttt{GifWindow} -- panel wyboru \glslink{emoji}{emoji}/\glslink{gif}{GIF-ów},
    \item \texttt{GroupChatParticipantsSideBar} -- panel listy uczestników rozmowy grupowej.
\end{itemize}

\subsubsection{ChatsPage -- główny kontener modułu}

\begin{figure}[H]
    \centering
    \includegraphics[width=1\textwidth]{\chatimplfig{czat_page}}
    \caption{Implementacja komponentu \texttt{ChatsPage}}
    \label{fig:chat:chats-page-1}
\end{figure}

\glslink{react-component}{Komponent} \texttt{ChatsPage} (rys.~\ref{fig:chat:chats-page-1}) jest nadrzędnym kontenerem widoku czatu.
Odpowiada za rozłożenie interfejsu na trzy sekcje:
(1) lewy panel z listą rozmów, (2) centralne okno konwersacji oraz (3) opcjonalny panel boczny
widoczny wyłącznie dla czatów grupowych.
\glslink{stan}{Stan} wyboru rozmowy oraz widoczność panelu bocznego pobierane są ze \glslink{stan}{stanu} globalnego \glslink{redux}{Redux}.

\subsubsection{Lista czatów}

\paragraph{Komponent ChatList}

\begin{figure}[H]
    \centering
    \includegraphics[width=1\textwidth]{\chatimplfig{czat_list_1}}
    \caption{Implementacja komponentu \texttt{ChatList} (1/4)}
    \label{fig:chat:chat-list-1}
\end{figure}

\begin{figure}[H]
    \centering
    \includegraphics[width=1\textwidth]{\chatimplfig{czat_list_2}}
    \caption{Implementacja komponentu \texttt{ChatList} (2/4)}
    \label{fig:chat:chat-list-2}
\end{figure}

\begin{figure}[H]
    \centering
    \includegraphics[width=1\textwidth]{\chatimplfig{czat_list_3}}
    \caption{Implementacja komponentu \texttt{ChatList} (3/4)}
    \label{fig:chat:chat-list-3}
\end{figure}

\begin{figure}[H]
    \centering
    \includegraphics[width=1\textwidth]{\chatimplfig{czat_list_4}}
    \caption{Implementacja komponentu \texttt{ChatList} (4/4)}
    \label{fig:chat:chat-list-4}
\end{figure}

\glslink{react-component}{Komponent} \texttt{ChatList} (rys.~\ref{fig:chat:chat-list-1}--\ref{fig:chat:chat-list-4}) odpowiada za pobieranie oraz prezentację listy rozmów użytkownika.
Dane są ładowane z \glslink{backend}{backendu} przy użyciu \glslink{hook}{hook'a}
\texttt{useInfiniteQuery} z \glslink{tanstack-query}{TanStack Query}.
Zaimplementowano mechanizm \glslink{infinite-scroll}{przewijania nieskończonego},
który doładowuje kolejne strony wyników, gdy \glslink{sentinel}{element-strażnik} pojawi się na ekranie użytkownika.
W tym celu użyto mechanizmu \glslink{intersection-observer}{Intersection Observer}.

Po każdorazowym pobraniu strony dane są mapowane do lokalnego formatu i zapisywane w \glslink{redux}{Redux},
dzięki czemu \glslink{stan}{stan} listy rozmów jest współdzielony z innymi komponentami modułu.
Dodatkowo przy pierwszym renderowaniu, jeżeli użytkownik nie ma wybranego czatu,
ustawiana jest domyślna rozmowa.

W trakcie ładowania wyświetlane są elementy typu \glslink{skeleton-loader}{skeleton loader},
natomiast w przypadku doładowywania kolejnych stron prezentowany jest wskaźnik ładowania.

\paragraph{Komponent ListedChat}

\begin{figure}[H]
    \centering
    \includegraphics[width=1\textwidth]{\chatimplfig{listed_czat_1}}
    \caption{Implementacja komponentu \texttt{ListedChat} (1/2)}
    \label{fig:chat:listed-chat-1}
\end{figure}

\begin{figure}[H]
    \centering
    \includegraphics[width=1\textwidth]{\chatimplfig{listed_czat_2}}
    \caption{Implementacja komponentu \texttt{ListedChat} (2/2)}
    \label{fig:chat:listed-chat-2}
\end{figure}

\glslink{react-component}{Komponent} \texttt{ListedChat} (rys.~\ref{fig:chat:listed-chat-1}--\ref{fig:chat:listed-chat-2}) reprezentuje pojedynczy element listy rozmów.
Wyświetla podstawowe informacje o czacie: nazwę, awatar, treść ostatniej wiadomości oraz czas jej wysłania.
\glslink{react-component}{Komponent} wizualnie rozróżnia stan:
\begin{itemize}
    \item aktualnie wybranego czatu,
    \item czatu posiadającego nowe, nieodczytane wiadomości,
    \item czatu nieaktywnego.
\end{itemize}

Kliknięcie w element listy aktualizuje identyfikator wybranego czatu w \glslink{redux}{Redux} i
jednocześnie czyści znacznik nowych wiadomości dla danej rozmowy.

\subsubsection{Okno konwersacji}

\paragraph{Komponent ChatContent}

\begin{figure}[H]
    \centering
    \includegraphics[width=1\textwidth]{\chatimplfig{czat_content}}
    \caption{Implementacja komponentu \texttt{ChatContent}}
    \label{fig:chat:chat-content-1}
\end{figure}

\glslink{react-component}{Komponent} \texttt{ChatContent} (rys.~\ref{fig:chat:chat-content-1}) pełni rolę kontenera na \glslink{react-component}{komponenty} okna konwersacji.
Jego zadaniem jest pobranie z \glslink{redux}{Redux} identyfikatora aktualnie wybranego czatu, a następnie odczytanie
pełnych danych rozmowy na podstawie selektora \newline \texttt{selectChatById}.

\texttt{ChatContent} składa okno rozmowy z trzech części:
paska nagłówka (\texttt{ChatTopBar}), listy wiadomości (\texttt{ChatMessagingWindow}) oraz paska tworzenia wiadomości
\newline (\texttt{ChatBottomBar}).

\paragraph{Komponent ChatTopBar}

\begin{figure}[H]
    \centering
    \includegraphics[width=1\textwidth]{\chatimplfig{czat_top_bar_1}}
    \caption{Implementacja komponentu \texttt{ChatTopBar} (1/6)}
    \label{fig:chat:top-bar-1}
\end{figure}

\begin{figure}[H]
    \centering
    \includegraphics[width=1\textwidth]{\chatimplfig{czat_top_bar_2}}
    \caption{Implementacja komponentu \texttt{ChatTopBar} (2/6)}
    \label{fig:chat:top-bar-2}
\end{figure}

\begin{figure}[H]
    \centering
    \includegraphics[width=1\textwidth]{\chatimplfig{czat_top_bar_3}}
    \caption{Implementacja komponentu \texttt{ChatTopBar} (3/6)}
    \label{fig:chat:top-bar-3}
\end{figure}

\begin{figure}[H]
    \centering
    \includegraphics[width=1\textwidth]{\chatimplfig{czat_top_bar_4}}
    \caption{Implementacja komponentu \texttt{ChatTopBar} (4/6)}
    \label{fig:chat:top-bar-4}
\end{figure}

\begin{figure}[H]
    \centering
    \includegraphics[width=1\textwidth]{\chatimplfig{czat_top_bar_5}}
    \caption{Implementacja komponentu \texttt{ChatTopBar} (5/6)}
    \label{fig:chat:top-bar-5}
\end{figure}

\begin{figure}[H]
    \centering
    \includegraphics[width=1\textwidth]{\chatimplfig{czat_top_bar_6}}
    \caption{Implementacja komponentu \texttt{ChatTopBar} (6/6)}
    \label{fig:chat:top-bar-6}
\end{figure}

\glslink{react-component}{Komponent} \texttt{ChatTopBar} (rys.~\ref{fig:chat:top-bar-1}--\ref{fig:chat:top-bar-6}) pełni rolę nagłówka konwersacji.
W zależności od typu rozmowy (prywatna/grupowa) udostępnia różne akcje:
dla czatu prywatnego umożliwia przejście do profilu rozmówcy, natomiast w czacie grupowym
udostępnia przełącznik panelu bocznego oraz operacje zarządzania rozmową.
Część funkcjonalności realizowana jest w \glslink{modal}{oknach modalnych}
(tworzenie nowego czatu grupowego, edycja czatu lub dodawanie uczestników).

\paragraph{Komponent ChatMessagingWindow}

\begin{figure}[H]
    \centering
    \includegraphics[width=1\textwidth]{\chatimplfig{czat_msg_w_1}}
    \caption{Implementacja komponentu \texttt{ChatMessagingWindow} (1/6)}
    \label{fig:chat:messaging-window-1}
\end{figure}

\begin{figure}[H]
    \centering
    \includegraphics[width=1\textwidth]{\chatimplfig{czat_msg_w_2}}
    \caption{Implementacja komponentu \texttt{ChatMessagingWindow} (2/6)}
    \label{fig:chat:messaging-window-2}
\end{figure}

\begin{figure}[H]
    \centering
    \includegraphics[width=1\textwidth]{\chatimplfig{czat_msg_w_3}}
    \caption{Implementacja komponentu \texttt{ChatMessagingWindow} (3/6)}
    \label{fig:chat:messaging-window-3}
\end{figure}

\begin{figure}[H]
    \centering
    \includegraphics[width=1\textwidth]{\chatimplfig{czat_msg_w_4}}
    \caption{Implementacja komponentu \texttt{ChatMessagingWindow} (4/6)}
    \label{fig:chat:messaging-window-4}
\end{figure}

\begin{figure}[H]
    \centering
    \includegraphics[width=1\textwidth]{\chatimplfig{czat_msg_w_5}}
    \caption{Implementacja komponentu \texttt{ChatMessagingWindow} (5/6)}
    \label{fig:chat:messaging-window-5}
\end{figure}

\begin{figure}[H]
    \centering
    \includegraphics[width=1\textwidth]{\chatimplfig{czat_msg_w_6}}
    \caption{Implementacja komponentu \texttt{ChatMessagingWindow} (6/6)}
    \label{fig:chat:messaging-window-6}
\end{figure}

\glslink{react-component}{Komponent} \texttt{ChatMessagingWindow} (rys.~\ref{fig:chat:messaging-window-1}--\ref{fig:chat:messaging-window-6}) odpowiada za wyświetlanie wiadomości
w obrębie wybranej konwersacji. Dane pobierane z \glslink{backend}{backendu} przy użyciu
\texttt{useInfiniteQuery}.
Wiadomości są renderowane w układzie „od dołu” (ostatnia wiadomość na końcu),
a doładowywanie starszych fragmentów odbywa się po przewinięciu w górę.

Dodatkowo komponent porządkuje wiadomości w czytelny sposób:
\begin{itemize}
    \item wstawia separatory dat przy zmianie dnia,
    \item grupuje wiadomości tego samego autora wysłane w krótkim odstępie czasu.
\end{itemize}

Okno konwersacji współpracuje również z aktualizacją danych w czasie rzeczywistym.
Po odebraniu nowej wiadomości jest ona natychmiast wyświetlana.

\paragraph{Komponent ChatBottomBar}

\begin{figure}[H]
    \centering
    \includegraphics[width=1\textwidth]{\chatimplfig{czat_bottom_bar_1}}
    \caption{Implementacja komponentu \texttt{ChatBottomBar} (1/8)}
    \label{fig:chat:bottom-bar-1}
\end{figure}

\begin{figure}[H]
    \centering
    \includegraphics[width=1\textwidth]{\chatimplfig{czat_bottom_bar_2}}
    \caption{Implementacja komponentu \texttt{ChatBottomBar} (2/8)}
    \label{fig:chat:bottom-bar-2}
\end{figure}

\begin{figure}[H]
    \centering
    \includegraphics[width=1\textwidth]{\chatimplfig{czat_bottom_bar_3}}
    \caption{Implementacja komponentu \texttt{ChatBottomBar} (3/8)}
    \label{fig:chat:bottom-bar-3}
\end{figure}

\begin{figure}[H]
    \centering
    \includegraphics[width=1\textwidth]{\chatimplfig{czat_bottom_bar_4}}
    \caption{Implementacja komponentu \texttt{ChatBottomBar} (4/8)}
    \label{fig:chat:bottom-bar-4}
\end{figure}

\begin{figure}[H]
    \centering
    \includegraphics[width=1\textwidth]{\chatimplfig{czat_bottom_bar_5}}
    \caption{Implementacja komponentu \texttt{ChatBottomBar} (5/8)}
    \label{fig:chat:bottom-bar-5}
\end{figure}

\begin{figure}[H]
    \centering
    \includegraphics[width=1\textwidth]{\chatimplfig{czat_bottom_bar_6}}
    \caption{Implementacja komponentu \texttt{ChatBottomBar} (6/8)}
    \label{fig:chat:bottom-bar-6}
\end{figure}

\begin{figure}[H]
    \centering
    \includegraphics[width=1\textwidth]{\chatimplfig{czat_bottom_bar_7}}
    \caption{Implementacja komponentu \texttt{ChatBottomBar} (7/8)}
    \label{fig:chat:bottom-bar-7}
\end{figure}

\begin{figure}[H]
    \centering
    \includegraphics[width=1\textwidth]{\chatimplfig{czat_bottom_bar_8}}
    \caption{Implementacja komponentu \texttt{ChatBottomBar} (8/8)}
    \label{fig:chat:bottom-bar-8}
\end{figure}

\glslink{react-component}{Komponent} \texttt{ChatBottomBar} (rys.~\ref{fig:chat:bottom-bar-1}--\ref{fig:chat:bottom-bar-8}) realizuje wysyłanie wiadomości tekstowych,
załączników oraz interakcję z oknami wyboru \glslink{emoji}{emoji} i \glslink{gif}{GIF-ów}.
Wysyłanie wiadomości w czasie rzeczywistym odbywa się z użyciem \glslink{websocket}{WebSocket}
oraz protokołu \glslink{stomp}{STOMP}.
Obsługa załączników obejmuje wybór wielu plików oraz ich podgląd (jeżeli są zdjęciem).

\subsubsection{Okna emoji i GIF}

\paragraph{Komponent EmojiGifWindowWrapper}

\begin{figure}[H]
    \centering
    \includegraphics[width=1\textwidth]{\chatimplfig{egwrapper}}
    \caption{Implementacja komponentu \texttt{EmojiGifWindowWrapper}}
    \label{fig:chat:emoji-gif-wrapper-1}
\end{figure}

\glslink{react-component}{Komponent} \texttt{EmojiGifWindowWrapper} (rys.~\ref{fig:chat:emoji-gif-wrapper-1}) stanowi warstwę pośrednią,
która w zależności od aktualnego trybu wyświetla jedno z dwóch okien:
\texttt{EmojiWindow} lub \texttt{GifWindow}.

\paragraph{Komponent EmojiWindow}

\begin{figure}[H]
    \centering
    \includegraphics[width=1\textwidth]{\chatimplfig{ew}}
    \caption{Implementacja komponentu \texttt{EmojiWindow}}
    \label{fig:chat:emoji-window-1}
\end{figure}

\glslink{react-component}{Komponent} \texttt{EmojiWindow} (rys.~\ref{fig:chat:emoji-window-1}) integruje zewnętrzny selektor \glslink{emoji}{emoji}
i umożliwia wyszukiwanie oraz wybór emotikonów.
Po kliknięciu \glslink{emoji}{emoji}, \glslink{react-component}{komponent} dopisuje go do aktualnie budowanej treści wiadomości.

\paragraph{Komponent GifWindow}

\begin{figure}[H]
    \centering
    \includegraphics[width=1\textwidth]{\chatimplfig{gw_1}}
    \caption{Implementacja komponentu \texttt{GifWindow} (1/5)}
    \label{fig:chat:gif-window-1}
\end{figure}

\begin{figure}[H]
    \centering
    \includegraphics[width=1\textwidth]{\chatimplfig{gw_2}}
    \caption{Implementacja komponentu \texttt{GifWindow} (2/5)}
    \label{fig:chat:gif-window-2}
\end{figure}

\begin{figure}[H]
    \centering
    \includegraphics[width=1\textwidth]{\chatimplfig{gw_3}}
    \caption{Implementacja komponentu \texttt{GifWindow} (3/5)}
    \label{fig:chat:gif-window-3}
\end{figure}

\begin{figure}[H]
    \centering
    \includegraphics[width=1\textwidth]{\chatimplfig{gw_4}}
    \caption{Implementacja komponentu \texttt{GifWindow} (4/5)}
    \label{fig:chat:gif-window-4}
\end{figure}

\begin{figure}[H]
    \centering
    \includegraphics[width=1\textwidth]{\chatimplfig{gw_5}}
    \caption{Implementacja komponentu \texttt{GifWindow} (5/5)}
    \label{fig:chat:gif-window-5}
\end{figure}

\glslink{react-component}{Komponent} \texttt{GifWindow} (rys.~\ref{fig:chat:gif-window-1}--\ref{fig:chat:gif-window-5}) umożliwia wyszukiwanie oraz wysyłanie \glslink{gif}{GIF-ów}
z wykorzystaniem integracji z dostawcą \glslink{tenor}{Tenor} po stronie \glslink{backend}{backendu}.
Widok udostępnia dwa scenariusze:
\begin{itemize}
    \item prezentację popularnych kategorii,
    \item wyszukiwanie \glslink{gif}{GIF-ów} po frazie.
\end{itemize}

Wyszukiwanie realizowane jest przy użyciu \texttt{useInfiniteQuery}.
W przypadku kliknięcia w wybrany \glslink{gif}{GIF} jego adres \glslink{url}{URL} jest przesyłany jako treść wiadomości
przez \glslink{websocket}{WebSocket}/\glslink{stomp}{STOMP}, a okno selektora zostaje zamknięte.

\subsubsection{Lista uczestników czatu grupowego}

\paragraph{Komponent GroupChatParticipantsSideBar}

\begin{figure}[H]
    \centering
    \includegraphics[width=1\textwidth]{\chatimplfig{gcpsb}}
    \caption{Implementacja komponentu \texttt{GroupChatParticipantsSideBar}}
    \label{fig:chat:group-participants-sidebar-1}
\end{figure}

\glslink{react-component}{Komponent} \texttt{GroupChatParticipantsSideBar} (rys.~\ref{fig:chat:group-participants-sidebar-1}) jest panelem bocznym,
który pojawia się wyłącznie dla rozmów grupowych.
Panel prezentuje listę uczestników czatu wraz z awatarem i nazwą użytkownika,
a kliknięcie w wybraną osobę przekierowuje do jej profilu.
