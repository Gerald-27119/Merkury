%! Author = Mateusz
%! Date = 19/12/2025

\subsection{Sidebar}
\label{subsec:sidebar-frontend}

\glslink{sidebar}{Sidebar} stanowi główny komponent nawigacyjny aplikacji.
Implementacja obejmuje:
\begin{itemize}
    \item zestaw komponentów interfejsu odpowiedzialnych za renderowanie pozycji, podmenu, tooltipów oraz obsługę
    interakcji użytkownika,
    \item pliki pomocnicze służące do budowania listy linków i rozpoznawania typów elementów nawigacji
    (\texttt{functions}, \texttt{sidebarLinks}),
    \item definicje typów i interfejsów opisujących strukturę pozycji paska bocznego (\texttt{link}),
    \item moduł \texttt{Redux} przechowujący stan rozwinięcia paska (\texttt{sidebar.ts}).
\end{itemize}

\subsubsection{Komponenty}

\textbf{\texttt{Sidebar}} \\
Komponent odpowiada za złożenie całej nawigacji oraz sterowanie jej zachowaniem (rys. \ref{img:sidebar}).
Wykorzystywane są \glslink{hook}{hooki} \texttt{useDarkMode} (obsługa motywu), \newline \texttt{useSelectorTyped}
(odczyt \texttt{isLogged} oraz \texttt{isSidebarOpen}) oraz \texttt{useLocation} (sprawdzenie aktualnej ścieżki).
Na podstawie lokalizacji ustalany jest tryb pozycjonowania: dla stron o układzie „sticky”
(panel użytkownika, czat, strona główna) stosowana jest klasa \texttt{xl:sticky}, w pozostałych przypadkach \texttt{absolute}.
Zestawy linków nawigacyjnych budowane są w pliku \texttt{sidebarLinks} (opis w podrozdz. \ref{subsubsec:sidebar-links-files}).

W strukturze komponentu renderowane są:
\begin{itemize}
    \item \texttt{SidebarToggleButton},
    \item sekcja nawigacyjna: \texttt{SidebarSection} $\rightarrow$ \texttt{SidebarList} (linki główne),
    \item sekcja opcji: \texttt{SidebarSection} $\rightarrow$ \texttt{SidebarList} (akcje i opcje).
\end{itemize}

\begin{figure}[H]
    \centering
    \includegraphics[width=1\textwidth]{attachments/implementacja-frontendu/sidebar/sidebar}
    \caption{Fragment kodu komponentu \texttt{Sidebar}.}
    \label{img:sidebar}
\end{figure}

\textbf{\texttt{Tooltip}} \\
Komponent służy do prezentacji nazwy pozycji oraz (w przypadku submenu) listy pozycji podrzędnych, gdy pasek
boczny pozostaje zwinięty.
Po kliknięciu w element następuje przejście do przypisanej podstrony (rys. \ref{img:sidebar-tooltip}).

\begin{figure}[H]
    \centering
    \includegraphics[width=1\textwidth]{attachments/implementacja-frontendu/sidebar/tooltip}
    \caption{Fragment kodu komponentu \texttt{Tooltip}.}
    \label{img:sidebar-tooltip}
\end{figure}

\textbf{\texttt{SidebarToggleButton}} \\
Komponent wyświetla przycisk zwijania/rozwijania paska bocznego (rys. \ref{img:sidebar-toggle-button}).
Po kliknięciu wywoływana jest akcja \texttt{toggleSidebar} z \glslink{redux}{\texttt{Redux}}.
Na mniejszych ekranach (poniżej \texttt{xl}) prezentowany jest również napis „Merkury”.

\begin{figure}[H]
    \centering
    \includegraphics[width=1\textwidth]{attachments/implementacja-frontendu/sidebar/sidebar-toggle-button}
    \caption{Fragment kodu komponentu \texttt{SidebarToggleButton}.}
    \label{img:sidebar-toggle-button}
\end{figure}

\textbf{\texttt{SidebarSection}} \\
Komponent grupuje listy nawigacyjne przekazane przez \texttt{children} oraz opcjonalnie dodaje separator u
góry lub na dole (rys. \ref{img:sidebar-section}).

\begin{figure}[H]
    \centering
    \includegraphics[width=1\textwidth]{attachments/implementacja-frontendu/sidebar/sidebar-section}
    \caption{Fragment kodu komponentu \texttt{SidebarSection}.}
    \label{img:sidebar-section}
\end{figure}

\textbf{\texttt{SidebarList}} \\
Komponent renderuje listę elementów nawigacji poprzez mapowanie na \texttt{SidebarItem} (rys. \ref{img:sidebar-list}).

\begin{figure}[H]
    \centering
    \includegraphics[width=1\textwidth]{attachments/implementacja-frontendu/sidebar/sidebar-list}
    \caption{Fragment kodu komponentu \texttt{SidebarList}.}
    \label{img:sidebar-list}
\end{figure}

\textbf{\texttt{SidebarLabel}} \\
Komponent odpowiada za wyświetlenie etykiety pozycji, gdy pasek jest rozwinięty.
Dla płynnej animacji zastosowano \texttt{AnimatePresence} oraz \texttt{motion.p}.
W przypadku \texttt{submenu} z niepustą listą dzieci prezentowana jest dodatkowo ikona
strzałki (rys. \ref{img:sidebar-label}).

\begin{figure}[H]
    \centering
    \includegraphics[width=1\textwidth]{attachments/implementacja-frontendu/sidebar/sidebar-label}
    \caption{Fragment kodu komponentu \texttt{SidebarLabel}.}
    \label{img:sidebar-label}
\end{figure}

\textbf{\texttt{SidebarIcon}} \\
Komponent odpowiada za prezentację ikony danej pozycji.
W przypadku zwiniętego paska i pozycji typu \texttt{submenu} ikona stanowi \texttt{NavLink}
(umożliwia przejście do strony nadrzędnej), natomiast dla \texttt{link} i \texttt{action}
zwracana jest sama ikona (rys. \ref{img:sidebar-icon}).

\begin{figure}[H]
    \centering
    \includegraphics[width=1\textwidth]{attachments/implementacja-frontendu/sidebar/sidebar-icon}
    \caption{Fragment kodu komponentu \texttt{SidebarIcon}.}
    \label{img:sidebar-icon}
\end{figure}

\textbf{\texttt{SidebarItemSubmenuLink}} \\
Komponent renderuje elementy podrzędne submenu jako \texttt{NavLink}.
Przy każdym kliknięciu wywoływana jest akcja \texttt{closeSidebar}, która zwija pasek wyłącznie na
ekranach o szerokości mniejszej niż \texttt{1280px} (rys. \ref{img:sidebar-item-submenu-link}).

\begin{figure}[H]
    \centering
    \includegraphics[width=1\textwidth]{attachments/implementacja-frontendu/sidebar/sidebar-item-submenu-link}
    \caption{Fragment kodu komponentu \texttt{SidebarItemSubmenuLink}.}
    \label{img:sidebar-item-submenu-link}
\end{figure}

\textbf{\texttt{SidebarItemSubmenu}} \\
Komponent odpowiada za obsługę pozycji typu \texttt{submenu} (rys. \ref{img:sidebar-item-submenu-1}--\ref{img:sidebar-item-submenu-3}).
Zastosowano:
\begin{itemize}
    \item \texttt{useToggleState} do przełączania stanu rozwinięcia,
    \item \texttt{useState} do przechowywania nazwy aktualnie otwartego submenu (w celu utrzymania zasady „jedno submenu otwarte naraz”),
    \item \texttt{useLocation} oraz \texttt{useEffect} do automatycznego rozwinięcia submenu, gdy aktywna jest jedna z podstron potomnych.
\end{itemize}
Pozycja submenu renderowana jest jako przycisk zawierający \texttt{SidebarItemContent}, a lista elementów potomnych
pojawia się animacyjnie (przez \texttt{AnimatePresence} i \texttt{motion.div}) tylko wtedy, gdy pasek jest rozwinięty.

\begin{figure}[H]
    \centering
    \includegraphics[width=1\textwidth]{attachments/implementacja-frontendu/sidebar/sidebar-item-submenu1}
    \caption{Fragment kodu komponentu \texttt{SidebarItemSubmenu} (1).}
    \label{img:sidebar-item-submenu-1}
\end{figure}
\begin{figure}[H]
    \centering
    \includegraphics[width=1\textwidth]{attachments/implementacja-frontendu/sidebar/sidebar-item-submenu2}
    \caption{Fragment kodu komponentu \texttt{SidebarItemSubmenu} (2).}
    \label{img:sidebar-item-submenu-2}
\end{figure}
\begin{figure}[H]
    \centering
    \includegraphics[width=1\textwidth]{attachments/implementacja-frontendu/sidebar/sidebar-item-submenu3}
    \caption{Fragment kodu komponentu \texttt{SidebarItemSubmenu} (3).}
    \label{img:sidebar-item-submenu-3}
\end{figure}

\textbf{\texttt{SidebarItemLink}} \\
Komponent renderuje pojedynczą pozycję typu \texttt{link} jako \texttt{NavLink} zawierający \texttt{SidebarItemContent}.
Dodatkowo, po kliknięciu wywoływana jest akcja \texttt{closeSidebar} (działająca tylko dla szerokości poniżej
\texttt{1280px}) (rys. \ref{img:sidebar-item-link-1}--\ref{img:sidebar-item-link-2}).

\begin{figure}[H]
    \centering
    \includegraphics[width=1\textwidth]{attachments/implementacja-frontendu/sidebar/sidebar-item-link1}
    \caption{Fragment kodu komponentu \texttt{SidebarItemLink} (1).}
    \label{img:sidebar-item-link-1}
\end{figure}
\begin{figure}[H]
    \centering
    \includegraphics[width=1\textwidth]{attachments/implementacja-frontendu/sidebar/sidebar-item-link2}
    \caption{Fragment kodu komponentu \texttt{SidebarItemLink} (2).}
    \label{img:sidebar-item-link-2}
\end{figure}

\textbf{\texttt{SidebarItemContent}} \\
Komponent stanowi wspólne „wnętrze” elementów \texttt{SidebarItem}.
Zawiera \texttt{SidebarIcon}, warunkowo \texttt{Tooltip} (gdy pasek jest zwinięty i aktywny jest stan tooltipa)
oraz \texttt{SidebarLabel} (rys. \ref{img:sidebar-item-content}).

\begin{figure}[H]
    \centering
    \includegraphics[width=1\textwidth]{attachments/implementacja-frontendu/sidebar/sidebar-item-content}
    \caption{Fragment kodu komponentu \texttt{SidebarItemContent}.}
    \label{img:sidebar-item-content}
\end{figure}

\textbf{\texttt{SidebarItemAction}} \\
Komponent obsługuje pozycje typu \texttt{action}, które nie prowadzą do \glslink{routing}{routingu}, lecz uruchamiają funkcje.
Zaimplementowano akcję wylogowania (wywołanie \texttt{logout} oraz aktualizacja stanu konta)
oraz przełączanie motywu poprzez \texttt{onChangeTheme}.
Renderowany jest przycisk zawierający \texttt{SidebarItemContent} (rys. \ref{img:sidebar-item-action-1}--\ref{img:sidebar-item-action-2}).

\begin{figure}[H]
    \centering
    \includegraphics[width=1\textwidth]{attachments/implementacja-frontendu/sidebar/sidebar-item-action1}
    \caption{Fragment kodu komponentu \texttt{SidebarItemAction} (1).}
    \label{img:sidebar-item-action-1}
\end{figure}
\begin{figure}[H]
    \centering
    \includegraphics[width=1\textwidth]{attachments/implementacja-frontendu/sidebar/sidebar-item-action2}
    \caption{Fragment kodu komponentu \texttt{SidebarItemAction} (2).}
    \label{img:sidebar-item-action-2}
\end{figure}

\textbf{\texttt{SidebarItem}} \\
Komponent wybiera właściwy podtyp elementu na podstawie pola \texttt{type} (rys. \ref{img:sidebar-item-1}--\ref{img:sidebar-item-2}).
Wykorzystywany jest \glslink{hook}{hook} \texttt{useBoolean} do sterowania widocznością tooltipa oraz \texttt{useLocation} i
\texttt{useEffect} do ukrycia tooltipa po zmianie ścieżki.
Dobór implementacji realizowany jest przez instrukcję \texttt{switch}:
\texttt{submenu} $\rightarrow$ \texttt{SidebarItemSubmenu}, \texttt{action} $\rightarrow$ \texttt{SidebarItemAction},
\texttt{link} $\rightarrow$ \texttt{SidebarItemLink}.

\begin{figure}[H]
    \centering
    \includegraphics[width=1\textwidth]{attachments/implementacja-frontendu/sidebar/sidebar-item1}
    \caption{Fragment kodu komponentu \texttt{SidebarItem} (1).}
    \label{img:sidebar-item-1}
\end{figure}
\begin{figure}[H]
    \centering
    \includegraphics[width=0.7\textwidth]{attachments/implementacja-frontendu/sidebar/sidebar-item2}
    \caption{Fragment kodu komponentu \texttt{SidebarItem} (2).}
    \label{img:sidebar-item-2}
\end{figure}

\subsubsection{Pliki pomocnicze}
\label{subsubsec:sidebar-links-files}

\textbf{\texttt{functions}} \\
Plik zawiera funkcje typu \emph{type guard} umożliwiające rozróżnienie wariantów linków:
\texttt{isSidebarSubmenu}, \texttt{isSidebarAction} oraz \texttt{isSidebarLink} (rys. \ref{img:sidebar-functions}).

\begin{figure}[H]
    \centering
    \includegraphics[width=1\textwidth]{attachments/implementacja-frontendu/sidebar/sidebar-functions}
    \caption{Fragment kodu pliku \texttt{functions} (funkcje typu \emph{guard}).}
    \label{img:sidebar-functions}
\end{figure}

\textbf{\texttt{sidebarLinks}} \\
Plik definiuje strukturę nawigacji w postaci obiektów:
\texttt{staticLinks}, \texttt{userLoggedLinks} oraz \texttt{getOptionsLinks} (rys. \ref{img:sidebar-links-1}--\ref{img:sidebar-links-3}).
Zależnie od stanu zalogowania (\texttt{isLogged}) generowane są różne zestawy pozycji:
\begin{itemize}
    \item dla użytkownika niezalogowanego: strona główna, mapa oraz forum (strona główna forum, regulamin),
    \item dla użytkownika zalogowanego: dodatkowo lista obserwowanych postów na forum, czat oraz panel użytkownika
    \item (submenu „account” z podstronami profilu, ustawień itd.).
\end{itemize}
Sekcja opcji (\texttt{getOptionsLinks}) zawiera:
\begin{itemize}
    \item pozycję logowania lub wylogowania (zależnie od \texttt{isLogged}),
    \item przełączenie motywu jasny/ciemny (zależnie od \texttt{isDark}).
\end{itemize}

\begin{figure}[H]
    \centering
    \includegraphics[width=1\textwidth]{attachments/implementacja-frontendu/sidebar/sidebar-links1}
    \caption{Fragment kodu pliku \texttt{sidebarLinks} (linki statyczne).}
    \label{img:sidebar-links-1}
\end{figure}
\begin{figure}[H]
    \centering
    \includegraphics[width=1\textwidth]{attachments/implementacja-frontendu/sidebar/sidebar-links2}
    \caption{Fragment kodu pliku \texttt{sidebarLinks} (linki dostępne dla użytkownika zalogowanego).}
    \label{img:sidebar-links-2}
\end{figure}
\begin{figure}[H]
    \centering
    \includegraphics[width=0.98\textwidth]{attachments/implementacja-frontendu/sidebar/sidebar-links3}
    \caption{Fragment kodu pliku \texttt{sidebarLinks} (linki opcji: logowanie/wylogowanie oraz zmiana motywu).}
    \label{img:sidebar-links-3}
\end{figure}

\textbf{\texttt{link}} \\
Plik z typami definiuje \texttt{LinkType} oraz interfejsy: \texttt{BaseLink}, \texttt{SidebarLink},
\texttt{SidebarSubmenuLink}, \texttt{SidebarSubmenu}, \texttt{SidebarAction} (rys. \ref{img:sidebar-types}).

\begin{figure}[H]
    \centering
    \includegraphics[width=0.9\textwidth]{attachments/implementacja-frontendu/sidebar/sidebar-types}
    \caption{Fragment kodu pliku \texttt{link}.}
    \label{img:sidebar-types}
\end{figure}

\subsubsection{Redux}

\glslink{stan}{Stan} paska bocznego utrzymywany jest w module \texttt{Redux} w pliku \texttt{sidebar.ts}.
Przechowywana jest wyłącznie flaga \texttt{isOpen} informująca o stanie rozwinięcia.
Zdefiniowano:
\begin{itemize}
    \item reducer \texttt{setIsSidebarOpen} (jawne ustawienie wartości),
    \item reducer \texttt{toggleSidebar} (przełączenie stanu),
    \item thunk \texttt{closeSidebar}, który zwija pasek wyłącznie dla szerokości mniejszej niż \texttt{1280px}.
\end{itemize}
Rozwiązanie to umożliwia automatyczne zwijanie paska po kliknięciu w link na urządzeniach mobilnych, bez wpływu na
zachowanie na dużych ekranach (rys. \ref{img:sidebar-redux}).

\begin{figure}[H]
    \centering
    \includegraphics[width=1\textwidth]{attachments/implementacja-frontendu/sidebar/sidebar-redux}
    \caption{Slice \texttt{Redux} odpowiadający za stan paska bocznego.}
    \label{img:sidebar-redux}
\end{figure}

\subsubsection{Użycie w układzie aplikacji}

\textbf{\texttt{Layout}} \\
Stanowi szkielet widoków aplikacji: renderuje \texttt{Sidebar} oraz część
główną (\texttt{main}) zawierającą \texttt{MobileBar}, listę powiadomień oraz \texttt{Outlet}
(miejsce wstrzyknięcia aktualnej podstrony przez router).
Dla strony mapy ustawiany jest układ \texttt{relative}, a dla pozostałych widoków układ elastyczny (\texttt{flex}).
Dodatkowo, w \texttt{useEffect} wykrywana jest nawigacja do ścieżek panelu konta; na ekranach o szerokości
co najmniej \texttt{1280px} pasek boczny jest wówczas automatycznie rozwijany poprzez
\texttt{dispatch(sidebarAction.setIsSidebarOpen(true))} (rys. \ref{img:layout}).

\begin{figure}[H]
    \centering
    \includegraphics[width=1\textwidth]{attachments/implementacja-frontendu/sidebar/layout}
    \caption{Fragment kodu komponentu \texttt{Layout}.}
    \label{img:layout}
\end{figure}

\textbf{\texttt{MobileBar}} \\
Wyświetlany jest wyłącznie poniżej progu \texttt{xl} (klasa \texttt{xl:hidden}).
Stanowi górny pasek nawigacyjny na urządzeniach mobilnych: zawiera przycisk z ikoną menu, który przełącza
\glslink{stan}{stan} paska bocznego (\texttt{toggleSidebar}), oraz tytuł aplikacji „Merkury”.
Dzięki umieszczeniu w części głównej (\texttt{main}) komponent pozostaje widoczny nad zawartością strony,
jednocześnie nie wpływając na układ desktopowy (rys. \ref{img:mobile-bar}).

\begin{figure}[H]
    \centering
    \includegraphics[width=1\textwidth]{attachments/implementacja-frontendu/sidebar/mobile-bar}
    \caption{Fragment kodu komponentu \texttt{MobileBar}.}
    \label{img:mobile-bar}
\end{figure}
