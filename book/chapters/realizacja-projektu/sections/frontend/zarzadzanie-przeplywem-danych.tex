%! Author = mateusz
%! Date = 20/10/2025

\subsection{Zarządzanie stanem i przepływ danych}
\label{subsec:zarzadzanie-stanem-i-przeplyw-danych}

W projekcie postawiliśmy na zrównoważone podejście do zarządzania
\gls{stan}em. Korzystamy zarówno z lokalnego \gls{stan}u komponentów
(za pomocą hooka \texttt{useState})~\cite{react-use-state}, jak i ze \gls{stan}u globalnego,
utrzymywanego przez \gls{bibliote}kę React~Redux~\cite{redux}. 
Globalny \gls{stan}
został wprowadzony po to, aby możliwie najbardziej ograniczyć
przekazywanie propsów w głąb drzewa komponentów oraz uniknąć
niepotrzebnych ponownych renderów.

Do przechowywania \gls{stan}u lokalnego, ograniczonego tylko do danego
komponentu (lub jego najbliższych elementów podrzędnych), wykorzystujemy
hooki \texttt{useState} oraz \texttt{useEffect}. Natomiast w przypadku
bardziej złożonej logiki lub potrzeby ponownego wykorzystania kodu
powstały hooki niestandardowe, takie jak \texttt{useScreenSize},
\texttt{useDarkMode} czy \texttt{useClickOutside}. Dzięki temu większość
logiki biznesowej została wydzielona z warstwy \gls{ui}, co poprawia
czytelność i ułatwia utrzymanie kodu.

Z racji tego, że korzystamy z Reacta w połączeniu z TypeScriptem,
przygotowaliśmy również własne hooki wspomagające typowanie, takie jak
\texttt{useDispatchTyped} oraz \texttt{useSelectorTyped}. Pozwalają one
na bezpieczne typowanie akcji oraz selektorów Reduxa bez konieczności
powtarzania adnotacji typów w każdym komponencie.
Fragmenty tej implementacji przedstawiono na rysunkach
\ref{img:redux-store} oraz \ref{img:redux-account}.

\begin{figure}[H]
    \centering
    \includegraphics[width=1\textwidth]{attachments/redux-store}
    \caption{Konfiguracja sklepu (\texttt{Redux store})}
    \label{img:redux-store}
\end{figure}

\begin{figure}[H]
    \centering
    \includegraphics[width=1\textwidth]{attachments/redux-account}
    \caption{Przykładowy slice odpowiedzialny za sprawdzenie czy użytkownik jest zalogowany}
    \label{img:redux-account}
\end{figure}
