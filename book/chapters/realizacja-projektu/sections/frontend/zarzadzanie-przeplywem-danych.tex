%! Author = Mateusz
%! Date = 20/10/2025

\subsection{Zarządzanie stanem i przepływ danych}
\label{subsec:zarzadzanie-stanem-i-przeplyw-danych}

W niniejszym podrozdziale opisano zastosowane w projekcie podejście do zarządzania \glslink{stan}{stanem}
oraz organizację przepływu danych w aplikacji frontendowej. \newline

W projekcie postawiono na zrównoważone podejście do zarządzania \glslink{stan}{stanem}.
Korzysta się zarówno z lokalnego \glslink{stan}{stanu} komponentów
(za pomocą \glslink{hook}{hooka} \texttt{useState})~\cite{react-use-state}, jak i ze \glslink{stan}{stanu} globalnego,
utrzymywanego przez \glslink{biblioteka}{bibliotekę} \gls{react}~\gls{redux}~\cite{redux}.
Globalny \glslink{stan}{stan} wprowadzono w celu możliwie jak największego ograniczenia
przekazywanie \glslink{props}{propsów} w głąb drzewa komponentów oraz uniknąć
niepotrzebnych ponownych renderów.

Do przechowywania \glslink{stan}{stanu} lokalnego, ograniczonego tylko do danego komponentu
(lub jego najbliższych elementów podrzędnych), wykorzystuje się \glslink{hook}{hook} \texttt{useState}.
Natomiast efekty uboczne i synchronizację realizuje się za pomocą \texttt{useEffect}.
W przypadku bardziej złożonej logiki lub potrzeby ponownego wykorzystania kodu
powstały \glslink{hook}{hooki} niestandardowe, takie jak \texttt{useScreenSize},
\texttt{useDarkMode} czy \newline \texttt{useClickOutside}.
Dzięki temu większość logiki prezentacji wydzielono
z warstwy \gls{ui}, co poprawia czytelność i ułatwia utrzymanie kodu.

Z racji tego, że korzystamy z \glslink{react}{reacta} w połączeniu z \glslink{type-script}{TypeScriptem},
przygotowano również własne \glslink{hook}{hooki} wspomagające typowanie, takie jak
\texttt{useDispatchTyped} oraz \texttt{useSelectorTyped}.
Pozwalają one na bezpieczne typowanie akcji oraz selektorów \glslink{redux}{reduxa}
bez konieczności powtarzania adnotacji typów w każdym komponencie.
Fragmenty tej implementacji przedstawiono na rysunkach
\ref{img:redux-store} oraz \ref{img:redux-account}.

\begin{figure}[H]
    \centering
    \includegraphics[width=1\textwidth]{attachments/implementacja-frontendu/redux-store}
    \caption{Konfiguracja sklepu (\texttt{Redux store})}
    \label{img:redux-store}
\end{figure}

\begin{figure}[H]
    \centering
    \includegraphics[width=1\textwidth]{attachments/implementacja-frontendu/redux-account}
    \caption{Przykładowy slice odpowiedzialny za sprawdzenie czy użytkownik jest zalogowany}
    \label{img:redux-account}
\end{figure}
