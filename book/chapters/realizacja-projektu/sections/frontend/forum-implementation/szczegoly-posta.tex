%! Author = kacper
%! Date = 08.01.2026

\subsubsection{Szczegóły posta}
\label{subsubsec:szczegoly-posta}

\glslink{react-component}{Komponent} \texttt{ForumThread} odpowiada za wyświetlanie pełnej treści pojedynczego posta wraz z komentarzami.
Otrzymuje jego identyfikator z parametrów \glslink{url}{URL} i wykorzystuje \glslink{hook}{hooka} \texttt{useQuery} do pobrania szczegółów (\texttt{DetailedPost}) oraz \texttt{useInfiniteQuery} do stronicowanego pobierania komentarzy.

\texttt{ForumThread} składa się z następujących \glslink{react-component}{komponentów}:
\begin{itemize}
    \item \texttt{ReturnButton} – umożliwia powrót do poprzedniej strony
    \item \texttt{FollowPostButton} – pozwala obserwować post, dostępny tylko dla użytkowników zalogowanych lub niebędących autorem
    \item \texttt{DetailedPost} – prezentuje pełną treść posta, kategorię, tagi, liczbę wyświetleń, komentarzy oraz przycisk dodawania komentarza
    \item \texttt{PostCommentForm} – formularz dodawania komentarza, otwierany po kliknięciu w przycisk w \texttt{DetailedPost}.
    Widoczność kontrolowana jest lokalnym \glslink{stan}{stanem} \glslink{react-component}{komponentu}
    \item \texttt{PostCommentList} – lista komentarzy do posta, obsługująca sortowanie i paginację
\end{itemize}

\texttt{ForumThread} obsługuje następujące akcje użytkownika:
\begin{itemize}
    \item dodawanie komentarzy (\texttt{PostCommentForm})
    \item obserwowanie posta (\texttt{FollowPostButton})
    \item edycję i usuwanie postów oraz zgłaszanie treści poprzez komponenty \texttt{DetailedPost} i współdzielone menu kontekstowe
\end{itemize}

Akcje te realizowane są przy użyciu \glslink{mutacja}{mutacji} \glslink{api}{API} z \glslink{tanstack-query}{biblioteki TanStack Query}, z automatycznym odświeżaniem danych po wykonaniu akcji.
W przypadku braku autoryzacji użytkownika wyświetlane są odpowiednie komunikaty zachęcające do zalogowania się.

\glslink{react-component}{Komponent} \texttt{DetailedPost} odpowiada za prezentację pełnej treści pojedynczego posta w widoku wątku forum.
Otrzymuje obiekt \texttt{PostDetails} oraz stany ładowania i błędu, a także funkcje do obsługi akcji użytkownika (dodawanie komentarza, follow, edycja, usuwanie, zgłaszanie, głosowanie, udostępnianie).

\texttt{DetailedPost} składa się z następujących \glslink{react-component}{komponentów}:
\begin{itemize}
    \item \texttt{ForumContentHeader} – nagłówek posta z informacjami o autorze i dacie publikacji.
    Kliknięcie w autora przenosi do jego profilu
    \item \texttt{PostMetaData} – wyświetla kategorię i tagi posta
    \item \texttt{DetailedPostContent} – prezentuje tytuł i treść posta
    Treść (\texttt{content}) jest wyświetlana jako HTML, co pozwala na zachowanie formatowania oraz wstawianie elementów multimedialnych
    \item \texttt{DetailedPostActions} – panel akcji użytkownika, zawiera wszystkie interakcje związane z postem (głosowanie, obserwowanie, edycja, usuwanie, raportowanie, udostępnianie, dodawanie komentarza)
\end{itemize}

\texttt{DetailedPost} obsługuje akcje przy użyciu \glslink{mutacja}{mutacji} \glslink{api}{API} z \glslink{tanstack-query}{biblioteki TanStack Query}, z automatycznym odświeżaniem odpowiednich zapytań.
Dostępność poszczególnych akcji zależy od stanu zalogowania użytkownika oraz jego uprawnień.
W przypadku braku autoryzacji wyświetlane są komunikaty informujące o konieczności logowania.

\glslink{react-component}{Komponent} \texttt{DetailedPostActions} wyświetla panel akcji użytkownika dla posta w widoku szczegółowym.
Zawiera:
\begin{itemize}
    \item Głosy (\texttt{upvote}/\texttt{downvote}) – za pomocą \glslink{react-component}{komponentu} \texttt{ActionIconWithCount}, pokazującego liczbę głosów i aktywność użytkownika
    \item Komentarze – licznik komentarzy z ikoną
    \item Udostępnianie – przycisk kopiujący adres \glslink{url}{URL} posta do schowka
    \item Menu kontekstowe (\texttt{ForumPostMenu}) – pozwala autorowi edytować lub usuwać post, a innym użytkownikom obserwować (\texttt{follow}) lub zgłaszać (\texttt{report})
    \item Przycisk dodania komentarza (\texttt{AddCommentButton}) – otwiera formularz dodawania komentarza w \glslink{react-component}{komponencie} \texttt{PostCommentForm}
\end{itemize}

Menu kontekstowe (\texttt{ForumContentMenu}) obsługuje dynamicznie dostępne akcje w zależności od tego, czy użytkownik jest autorem posta oraz czy obserwuje dany post.
Widoczność menu i jego zamykanie po kliknięciu poza obszarem realizowane są przy użyciu \glslink{hook}{hooka} \texttt{useClickOutside}.

Wszystkie akcje użytkownika, takie jak głosowanie, follow, report, edycja czy usuwanie, realizowane są za pomocą \glslink{mutacja}{mutacji} \glslink{api}{API} z \glslink{tanstack-query}{biblioteki TanStack Query}, z automatycznym odświeżaniem odpowiednich zapytań i wyświetlaniem komunikatów o sukcesie lub braku autoryzacji.

\glslink{react-component}{Komponent} \texttt{PostCommentList} odpowiada za wyświetlanie listy komentarzy przypisanych do posta lub do innego komentarza (w przypadku odpowiedzi).
Obsługuje zarówno komentarze główne, jak i zagnieżdżone odpowiedzi.

Funkcjonalności \glslink{react-component}{komponentu} obejmują:
\begin{itemize}
    \item wyświetlanie listy komentarzy przy użyciu \glslink{react-component}{komponentu} \texttt{PostComment}
    \item obsługę stanów ładowania i błędu
    \item sortowanie komentarzy (tylko dla komentarzy głównych) przy użyciu \newline \texttt{ForumSortDropdown}
    \item animacje przejść pomiędzy \glslink{stan}{stanami} (ładowanie / lista / brak danych) z wykorzystaniem \glslink{biblioteka}{biblioteki} \texttt{Framer Motion}
\end{itemize}

W zależności od wartości flagi \texttt{areReplies}, \glslink{react-component}{komponent}:
\begin{itemize}
    \item dla komentarzy głównych – umożliwia sortowanie
    \item dla odpowiedzi – prezentuje listę bez sortowania i w uproszczonym układzie wizualnym
\end{itemize}

\glslink{react-component}{Komponent} \texttt{PostComment} odpowiada za prezentację pojedynczego komentarza oraz obsługę wszystkich interakcji użytkownika z nim związanych.

Składa się z następujących elementów:
\begin{itemize}
    \item \texttt{ForumContentHeader} – nagłówek z informacjami o autorze i dacie publikacji; kliknięcie przenosi do profilu autora
    \item \texttt{PostCommentContent} – treść komentarza
    \item \texttt{PostCommentActions} – panel akcji (głosowanie, edycja, usuwanie, odpowiedź, zgłoszenie)
\end{itemize}

\glslink{react-component}{Komponent} obsługuje również:
\begin{itemize}
    \item edycję komentarza
    \item dodawanie odpowiedzi
    \item usuwanie
    \item głosowanie (\texttt{upvote}/\texttt{downvote})
    \item zgłaszanie komentarza (\texttt{report})
\end{itemize}

Wszystkie operacje modyfikujące dane realizowane są przy użyciu \glslink{mutacja}{mutacji} \glslink{api}{API} z \glslink{tanstack-query}{biblioteki TanStack Query}, z odpowiednim odświeżaniem zapytań dla komentarzy głównych lub odpowiedzi.

Komentarze w systemie forum obsługują zagnieżdżone odpowiedzi, co zostało zaimplementowane w formie \glslink{rekurencja}{rekurencyjnego} użycia \glslink{react-component}{komponentów}:
\begin{itemize}
    \item \texttt{PostComment} może zawierać \texttt{PostCommentList} z odpowiedziami
    \item \texttt{PostCommentList} wyświetla kolejne komponenty \texttt{PostComment}
    \item każdy komentarz może posiadać własną listę odpowiedzi, które są ładowane i wyświetlane niezależnie
\end{itemize}

Odpowiedzi do komentarza:
\begin{itemize}
    \item są pobierane stronicowo (\texttt{infinite query})
    \item ładowane dopiero po rozwinięciu listy odpowiedzi
    \item mogą być dalej rozwijane, zachowując tę samą strukturę komponentów
\end{itemize}
