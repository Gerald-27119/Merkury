%! Author = kacper
%! Date = 08.01.2026

\subsubsection{Prawy panel boczny}
\label{subsubsec:prawy-panel-boczny}

Komponent \texttt{ForumSearchBar} odpowiada za zaawansowane wyszukiwanie postów na forum i jest wykorzystywany w \texttt{ForumLayout}.
Logika wyszukiwania oparta jest o lokalny stan komponentu, którego strukturę definiuje interfejs \texttt{SearchState}.
Interfejs ten opisuje komplet aktywnych filtrów wyszukiwania, takich jak:
\begin{itemize}
    \item fraza wyszukiwania
    \item nazwa autora
    \item kategoria
    \item tagi
    \item zakres dat publikacji (od–do)
\end{itemize}

Dla wyboru kategorii i tagów wykorzystywany jest komponent \texttt{SelectWithSearch}, oparty o \glslink{react-select}{bibliotekę react-select}, umożliwiający wyszukiwanie oraz wybór pojedynczy i wielokrotny.

Pola odpowiedzialne za filtrowanie po dacie („From” / „To”) zostały zaimplementowane przy użyciu komponentu \texttt{DatePicker} z \glslink{biblioteka}{biblioteki} Ant Design (\texttt{antd}), z wykorzystaniem \texttt{dayjs} do obsługi i formatowania dat.
Komponent dat dynamicznie dostosowuje wygląd do aktualnego trybu jasnego i ciemnego aplikacji.

Po zatwierdzeniu formularza dane wyszukiwania są mapowane na parametry \glslink{url}{URL} (\texttt{URLSearchParams}), a użytkownik jest przekierowywany do dedykowanego widoku wyników (\texttt{/forum/search}).
Dzięki temu wyszukiwanie jest w pełni odtwarzalne na podstawie adresu \glslink{url}{URL}.

Dodatkowo pola tekstowe wykorzystują \texttt{HintedSearchField}, który umożliwia podpowiedzi wyszukiwania w trakcie wpisywania.

Komponent \texttt{ForumSearch} odpowiada za prezentację listy postów spełniających kryteria wyszukiwania przekazane w parametrach \glslink{url}{URL}.

Na podstawie \texttt{useSearchParams} tworzony jest obiekt \texttt{PostSearchRequestDto}, który zawiera wszystkie aktywne filtry (fraza, kategoria, tagi, daty, autor).
Dane te są następnie wykorzystywane w zapytaniu do API po stronie \glslink{backend}{backendu}.

Pobieranie wyników wyszukiwania realizowane jest przy użyciu hooka \texttt{useInfiniteQuery}, co umożliwia stronicowanie danych oraz dynamiczne ładowanie kolejnych stron.
Kolejne wyniki są pobierane automatycznie z wykorzystaniem mechanizmu nieskończonego przewijania.

Widok obsługuje również sortowanie wyników, a wybrana opcja sortowania jest częścią klucza zapytania, co powoduje ponowne pobranie danych przy jego zmianie.

Wyniki wyszukiwania są wyświetlane przy użyciu komponentu \texttt{ForumPostsPage}, który odpowiada za:
\begin{itemize}
    \item wyświetlanie listy postów
    \item obsługę sortowania
    \item prezentację informacji o liczbie znalezionych wyników
    \item komunikaty o braku dalszych rezultatów
\end{itemize}

Po wykonaniu wyszukiwania komponent \texttt{SearchResults} prezentuje liczbę znalezionych postów oraz czytelne podsumowanie zastosowanych filtrów.
Komponent ten znajduje się w \texttt{ForumPostList} i wyświetlany jest warunkowo, tylko gdy zapytanie wyszukiwania zawiera jakiekolwiek kryteria.

W przypadku błędów zapytania użytkownik otrzymuje stosowną notyfikację, natomiast podczas ładowania wyświetlane są komponenty typu skeleton, zapewniające spójne UX\@.

\texttt{TrendingPostPanel} wyświetla listę popularnych postów za pomocą \texttt{TrendingPostList}.
Każdy element w liście używa komponentu \texttt{TrendingPost}.