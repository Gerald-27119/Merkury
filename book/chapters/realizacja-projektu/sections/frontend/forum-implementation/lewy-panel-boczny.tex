%! Author = kacper
%! Date = 08.01.2026

\subsubsection{Lewy panel boczny}
\label{subsubsec:lewy-panel-boczny}

Komponent \texttt{ForumAddPostModal} jest globalnym modalem tworzonym w ramach \texttt{ForumLayout} i odpowiada za dodawanie oraz edycję postów na forum.
Jest dostępny na wszystkich podstronach forum i otwierany programowo (np. z poziomu przycisku „Add post” lub opcji edycji posta).

Modal działa w dwóch trybach:
\begin{itemize}
    \item \textbf{create} – tworzenie nowego posta,
    \item \textbf{edit} – edycja istniejącego posta, z wypełnionymi danymi początkowymi.
\end{itemize}

\texttt{ForumAddPostModal}:
\begin{itemize}
    \item wykorzystuje React Portal (\texttt{createPortal}) do umieszczania modala poza głównym drzewem \glslink{react-component}{komponentów},
    \item wyświetla półprzezroczyste tło (overlay), które zamyka modal po kliknięciu,
    \item pokazuje komponent \texttt{PostForm}, odpowiedzialny za formularz posta.
\end{itemize}

Komponent \texttt{PostForm} zawiera formularz dodawania i edycji posta.

Formularz został zaimplementowany z użyciem:
\begin{itemize}
    \item \texttt{React Hook Form} – do zarządzania stanem formularza,
    \item \texttt{Zod} – do walidacji danych wejściowych (schema \texttt{PostFormSchema}).
\end{itemize}

Obsługiwane pola:
\begin{itemize}
    \item tytuł posta,
    \item kategoria (pojedynczy wybór),
    \item tagi (wielokrotny wybór),
    \item treść posta (\texttt{richTextEditor}).
\end{itemize}

W trybie edycji formularz jest inicjalizowany danymi istniejącego posta.

Pole \texttt{content} wykorzystuje kontrolowany edytor tekstu (\texttt{ControlledEditor}) oparty o \glslink{tiptap}{bibliotekę Tiptap}, działający w wariancie \glslink{modal}{modalnym}.

Formularz zawiera również pola wyboru kategorii i tagów zaimplementowane przy użyciu komponentu \texttt{ControlledSelect}, który integruje React Hook Form z \glslink{react-select}{biblioteką react-select}, zapewniając obsługę wyszukiwania oraz wyboru pojedynczego i wielokrotnego.

Komponenty z prefiksem \texttt{Controlled} wykorzystują mechanizm \texttt{Controller} z biblioteki React Hook Form, co umożliwia integrację niestandardowych komponentów wejściowych (np. z zewnętrznej \glslink{biblioteka}{biblioteki}) z systemem walidacji i zarządzania stanem formularza, mimo że nie udostępniają one standardowych właściwości pól formularza.

Obrazy wstawiane w edytorze są tymczasowo przechowywane w treści jako dane Base64.
Przed wysłaniem formularza treść HTML jest analizowana, a osadzone obrazy są konwertowane do plików, asynchronicznie przesyłane do zewnętrznego storage (\glslink{azure-blob-storage}{Azure}), a następnie ich lokalne referencje są zastępowane docelowymi \glslink{url}{adresami URL}.

W przypadku błędu podczas przesyłania obrazów formularz wyświetla odpowiedni komunikat walidacyjny i blokuje zapis posta.


Komponent \texttt{ControlledEditor} pełni rolę warstwy integracyjnej pomiędzy React Hook Form a \glslink{rich-text-editor}{edytorem rich text} opartym o \glslink{tiptap}{bibliotekę Tiptap}.
Został zaimplementowany jako komponent kontrolowany przy użyciu \texttt{Controller}, co umożliwia poprawne przekazywanie wartości, obsługę zdarzeń \texttt{onChange} / \texttt{onBlur} oraz integrację z systemem walidacji formularza.

Edytor obsługuje różne warianty wizualne (\texttt{default}, \texttt{modal}), które wpływają na jego wysokość, układ oraz stylowanie, co pozwala na jego wielokrotne użycie w różnych kontekstach interfejsu.

Sam komponent \texttt{Tiptap} został skonfigurowany od podstaw z wykorzystaniem zestawu rozszerzeń, m.in.:
\begin{itemize}
    \item \texttt{StarterKit} – zapewniający podstawowe elementy edycji (akapity, listy, formatowanie tekstu),
    \item \texttt{Placeholder} – obsługa tekstu zastępczego dla pustej treści,
    \item \texttt{TextAlign} – wyrównanie tekstu,
    \item \texttt{Image} – obsługa obrazów (z włączoną obsługą danych Base64),
    \item \texttt{FileHandler} – umożliwiający wklejanie i przeciąganie plików graficznych bezpośrednio do edytora.
\end{itemize}

Dodawanie obrazów realizowane jest poprzez przeciąganie plików, wklejanie ze schowka lub dedykowany przycisk w pasku narzędzi.
Na etapie edycji obrazy są osadzane w treści jako dane Base64, a typ pliku jest walidowany przed dodaniem do dokumentu.
Informacje o błędach lub poprawnym dodaniu pliku są przekazywane do formularza poprzez callbacki \texttt{onFileError} oraz \texttt{onFileSuccess}, co pozwala na wyświetlanie komunikatów walidacyjnych na poziomie pola formularza.

Edytor posiada własny pasek narzędzi (\texttt{MenuBar}) zsynchronizowany ze stanem edytora, oferujący m.in.:
\begin{itemize}
    \item formatowanie tekstu (\texttt{bold}, \texttt{italic}, \texttt{underline}),
    \item listy uporządkowane i nieuporządkowane,
    \item nagłówki,
    \item wyrównanie tekstu,
    \item linki,
    \item wstawianie obrazów.
\end{itemize}

Całość stanowi w pełni kontrolowany, rozszerzalny edytor treści, spójnie zintegrowany z logiką formularza i dalszym procesem przetwarzania treści posta.


\texttt{PostCategoriesTagsPanel} jest komponentem lewego panelu bocznego forum, odpowiedzialnym za prezentację listy kategorii oraz tagów postów.
Komponent obsługuje stany ładowania i błędów danych pobieranych z \glslink{backend}{backendu}.
W trakcie pobierania wyświetlany jest komponent \texttt{SkeletonPostCategoryTag}, natomiast w przypadku błędu pokazywany jest komponent \texttt{Error}.

Po poprawnym załadowaniu danych panel wyświetla dwie sekcje: listę kategorii oraz listę tagów.
Każda sekcja prezentuje skróconą listę elementów, wyświetlanych odpowiednio za pomocą komponentów \texttt{PostCategory} oraz \texttt{PostTag}.

Na końcu obu list dostępny jest przycisk \texttt{ExpansionButton}, który umożliwia przejście do dedykowanych podstron zawierających pełne, alfabetyczne listy kategorii lub tagów.
Nawigacja realizowana jest z wykorzystaniem \glslink{hook}{hooka} \texttt{useNavigate}.
W przypadku braku danych wyświetlany jest komunikat informujący o braku dostępnych elementów.