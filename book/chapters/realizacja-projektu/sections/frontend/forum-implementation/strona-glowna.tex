%! Author = kacper
%! Date = 08.01.2026

\subsubsection{Strona główna}
\label{subsubsec:strona-glowna}

\glslink{react-component}{Komponent} \texttt{Forum} odpowiada za wyświetlanie strony głównej forum, zawierającej listę postów.
Dane pobierane są z \glslink{backend}{backendu} przy użyciu \glslink{hook}{hooka} \newline \texttt{useInfiniteQuery} z \glslink{tanstack-query}{biblioteki TanStack Query}, co umożliwia stronicowanie wyników oraz dynamiczne ładowanie kolejnych stron.

Lista postów może być sortowana według wybranych kryteriów (data publikacji, liczba wyświetleń, liczba komentarzy).
Zmiana opcji sortowania powoduje ponowne wykonanie zapytania z odpowiednimi parametrami.
Za sortowanie odpowiada \glslink{react-component}{komponent} \texttt{ForumSortDropdown}.

W trakcie początkowego ładowania danych wyświetlane są \glslink{react-component}{komponenty} szkieletowe (\texttt{SkeletonListedForumPost}), natomiast w przypadku błędu wyświetlany jest \glslink{react-component}{komponent} \texttt{Error}.

Logika została wydzielona do \glslink{react-component}{komponentu} \texttt{InfiniteScroll}, który odpowiada za:
\begin{itemize}
    \item wyświetlanie wskaźnika ładowania kolejnych danych
    \item prezentację komunikatu końcowego po dotarciu do ostatniej strony wyników
\end{itemize}

\glslink{react-component}{Komponent} \texttt{ForumFollowed} w analogiczny sposób wyświetla posty obserwowane dla zalogowanego użytkownika.

\glslink{react-component}{Komponent} \texttt{Post} odpowiada za prezentację pojedynczego posta na ich liście.
Otrzymuje on jego dane w postaci obiektu \texttt{PostGeneral} i wyświetla jego podstawowe informacje, takie jak tytuł, kategoria, tagi, skrócona treść, liczba wyświetleń oraz komentarzy.

\glslink{react-component}{Komponent} wykorzystuje podział na mniejsze elementy:
\begin{itemize}
    \item \texttt{PostHeader} – nagłówek posta zawierający tytuł oraz menu kontekstowe
    \item \texttt{PostMetaData} – prezentacja kategorii oraz tagów
    \item \texttt{PostContent} – skrócona treść posta wraz z informacjami statystycznymi (liczba wyświetleń i komentarzy)
\end{itemize}

Kliknięcie w tytuł posta powoduje przejście do widoku szczegółowego wątku forum, realizowane za pomocą \glslink{routing}{routingu} aplikacji.

Post umożliwia wykonywanie podstawowych operacji na poście, takich jak:
\begin{itemize}
    \item usuwanie
    \item edycja
    \item obserwowanie (follow)
    \item zgłaszanie treści
\end{itemize}

Operacje te realizowane są przy użyciu \glslink{mutacja}{mutacji} \glslink{api}{API} zaimplementowanych z wykorzystaniem \glslink{tanstack-query}{biblioteki TanStack Query}, co umożliwia automatyczne odświeżanie danych po wykonaniu akcji.
W przypadku braku autoryzacji użytkownika wyświetlany jest komunikat informacyjny zachęcający do zalogowania się.

\glslink{react-component}{Komponent} \texttt{PostHeader} wyświetla tytuł posta oraz menu kontekstowe dostępne po kliknięciu ikony.
Logika menu została wydzielona do \glslink{react-component}{komponentu} \newline \texttt{ForumContentMenu}, który jest wykorzystywany również w widokach szczegółowych postów oraz komentarzy.

\texttt{ForumContentMenu} dynamicznie dostosowuje dostępne akcje w zależności od:
\begin{itemize}
    \item tego, czy zalogowany użytkownik jest autorem treści
    \item stanu obserwowania posta
    \item rodzaju wyświetlanej treści
\end{itemize}

Menu umożliwia wykonywanie operacji takich jak obserwowanie posta, zgłaszanie treści, edycja oraz usuwanie.
Widoczność menu kontrolowana jest lokalnym stanem komponentu oraz mechanizmem zamykania po kliknięciu poza jego obszarem, zrealizowanym przy użyciu dedykowanego \glslink{hook}{hooka} \texttt{useClickOutside}.
