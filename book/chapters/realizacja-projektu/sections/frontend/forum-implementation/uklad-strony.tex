%! Author = kacper
%! Date = 08.01.2026

\subsubsection{Układ strony}
\label{subsubsec:uklad-strony}

\texttt{ForumLayout} jest \glslink{react-component}{komponentem} nadrzędnym wszystkich podstron forum (rys. \ref{img:forum-layout-1}~\ref{img:forum-layout-2}).
Zapewnia on spójną strukturę strony składającej się z trzech głównych sekcji: lewego panelu bocznego, centralnej części z główną treścią oraz prawego panelu bocznego.
Komponent ten stanowi wspólny układ dla wszystkich tras forum i jest używany w \glslink{routing}{routingu} aplikacji.

\texttt{ForumLayout} obejmuje:
\begin{itemize}
    \item lewy panel boczny, zawierający przycisk dodawania nowego posta oraz panel kategorii i tagów;
    \item część centralną, w której wyświetlana jest właściwa zawartość podstron forum (lista postów, widok posta, wyniki wyszukiwania itp.);
    \item prawy panel boczny, zawierający pasek wyszukiwania oraz panel popularnych postów z ostatniego miesiąca.
\end{itemize}

Komponent jest wykorzystywany w \glslink{routing}{routingu} aplikacji poprzez zagnieżdżenie tras forum, co umożliwia dostęp do paneli bocznych oraz modali na wszystkich podstronach forum.

Dane kategorii, tagów oraz popularnych postów pobierane są przy użyciu \glslink{hook}{hooka} \texttt{useQuery} z \glslink{tanstack-query}{biblioteki TanStack Query}, który komunikuje się z odpowiednimi endpointami \glslink{backend}{backendu}.
Stany ładowania oraz błędów przekazywane są do komponentów potomnych.

Przycisk dodawania posta (\texttt{AddPostButton}) sprawdza stan zalogowania użytkownika.
W przypadku braku autoryzacji wyświetlany jest komunikat informacyjny zachęcający do zalogowania się.
Obsługa modali tworzenia posta oraz zgłaszania treści realizowana jest globalnie za pomocą stanu Redux, dzięki czemu formularze te są dostępne niezależnie od aktualnie wyświetlanej podstrony forum.

\texttt{ForumLayout} zawiera również dwa modale: \texttt{ForumAddPostModal} oraz \texttt{ForumReportModal}, które są dostępne na wszystkich podstronach forum.
Służą one do obsługi formularzy tworzenia i edycji postów oraz zgłaszania postów i komentarzy.

\begin{figure}[H]
    \centering
    \includegraphics[width=1\textwidth]{attachments/implementacja-frontendu/forum/forum_layout_1}
    \caption{Komponent ForumLayout (1)}
    \label{img:forum-layout-1}
\end{figure}

\begin{figure}[H]
    \centering
    \includegraphics[width=1\textwidth]{attachments/implementacja-frontendu/forum/forum_layout_2}
    \caption{Komponent ForumLayout (2)}
    \label{img:forum-layout-2}
\end{figure}