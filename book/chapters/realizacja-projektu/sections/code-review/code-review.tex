%! Author = Stanisław Oziemczuk
%! Date = 31/12/2025


\section{Rieview kodu}
\label{sec:review-kodu}

W projekcie do każdego zadania była tworzona dedykowana gałąź (ang. \emph{branch}), która po jego wykonaniu była
\glslink{merge}{mergowana} do gałęzi głównej.
Aby proces ten był możliwy, tworzone były \glslink{pull-request}{pull requesty}, czyli mechanizm pozwalający innym członkom
zespołu przeczytanie kodu oraz zgłaszanie ewentualnych uwag lub pytań.
Poniżej przedstawiono przykład komentarza do kodu (rys. \ref{fig:code-review-comment}):

\begin{figure}[H]
    \centering
    \includegraphics[width=0.78\textwidth]{attachments/realizacja-projektu/review-kodu/review_comment}
    \caption{Przykład komentarza w review kodu}
    \label{fig:code-review-comment}
\end{figure}

Jeżeli autor komentarza uznał, że poruszona kwestia została poprawnie rozwiązana, zamykał go.
\\
\\
Uwagi mógł też zgłaszać \texttt{Copilot} (por. sekcja~\ref{subsec:narzedzia}), będący modelem sztucznej inteligencji dostarczanym
przez \glslink{github}{GitHub'a}.
Zespół projektowy miał do niego dostęp ze względu na posiadanie studenckiej subskrypcji konta, która umożliwia korzystanie z
usług premium w celach edukacyjnych.
Aby \texttt{Copilot} przeprowadził \glslink{review-kodu}{review kodu}, podczas zgłaszania \glslink{pull-request}{pull request'a}
należy zaznaczyć odpowiednią opcję.
Czas oczekiwania na przeprowadzenie recenzji wynosił od kilku do kilkunastu minut i był zależny od ilości przesłanego kodu.
Uwagi zgłaszane przez \texttt{Copilot'a} były rozważane i zamykane przez osobę odpowiedzialną za dane zadanie.
Poniżej przedstawiono przykładowy komentarz do kodu napisany przez ten model sztucznej inteligencji (rys. \ref{fig:code-review-copilot-comment}):

\begin{figure}[H]
    \centering
    \includegraphics[width=0.78\textwidth]{attachments/realizacja-projektu/review-kodu/review_copilot_comment}
    \caption{Przykład komentarza Copilot'a w review kodu}
    \label{fig:code-review-copilot-comment}
\end{figure}
