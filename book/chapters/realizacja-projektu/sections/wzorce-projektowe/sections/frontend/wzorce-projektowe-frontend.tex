%! Author = Stanisław Oziemczuk
%! Date = 21.12.2025

\subsection{Frontend}
\label{subsec:Frontend}

\begin{itemize}
    \item \textbf{Hooks Pattern} \textemdash \space \glslink{wzorzec}{wzorzec} projektowy, który umożliwia korzystanie z
    mechanizmów \glslink{react}{Reacta} takich jak \glslink{stan}{stan} czy cykl życia \glslink{react-component}{komponentu} w
    \glslink{react-component}{komponentach} funkcyjnych.
    Przed wprowadzeniem \glslink{hook}{hook'ów} do zarządzania tymi właściwościami trzeba było tworzyć \glslink{react-component}{komponenty} klasowe,
    co powodowało większe skomplikowanie kodu.
    Hooks Pattern pozwala również na tworzenie własnych \glslink{hook}{hook'ów} zawierających logikę używaną w wielu \glslink{react-component}{komponentach}.

    W projekcie wykorzystano ten \glslink{wzorzec}{wzorzec} do zarządzania stanem oraz efektami ubocznymi w \glslink{react-component}{komponentach}
    za pomocą wbudowanych \glslink{hook}{hook'ów} \emph{useState} i \emph{useEffect}.
    \begin{figure}[H]
        \centering
        \includegraphics[width=1\textwidth]{attachments/wzorce-projektowe/useState_usage}
        \caption{Przykładowe użycie hook'a useState}
        \label{fig:hooks-useState-usage}
    \end{figure}
    \noindent
    \begin{figure}[H]
        \centering
        \includegraphics[width=1\textwidth]{attachments/wzorce-projektowe/useEffect_usage}
        \caption{Przykładowe użycie hook'a useEffect}
        \label{fig:hooks-useEffect-usage}
    \end{figure}
    \noindent
    Zaimplementowano też własne \glslink{hook}{hooki}.
    Jednym z nich jest \emph{useBoolean} obługujący logikę do zarządzania zmiennymi przyjmujących wartości
    \textit{true} lub \textit{false}.
    Udostępniane są funckje do ustawienia wartości, a także do zmiany jej na przeciwną do poprzedniej.
    Korzystanie z tego \glslink{hook}{hook'a} pozwoliło uniknąć powtarzalności kodu oraz poprawiło jego czytelność.
    Poniżej przedstawiono implementację oraz przykładowe użycie \glslink{hook}{hooka} \emph{useBoolean}.
    \begin{figure}[H]
        \centering
        \includegraphics[width=1\textwidth]{attachments/wzorce-projektowe/useBoolean_implementation}
        \caption{Implementacja hook'a useBoolean}
        \label{fig:hooks-implementation}
    \end{figure}
    \noindent
    \begin{figure}[H]
        \centering
        \includegraphics[width=1\textwidth]{attachments/wzorce-projektowe/useBoolean_usage}
        \caption{Przykładowe użycie hook'a useBoolean}
        \label{fig:hooks-useBoolean-usage}
    \end{figure}
    \noindent
    \item \textbf{Optimistic UI} \textemdash \space \glslink{wzorzec}{wzorzec}, w którym \glslink{ui}{UI} jest aktualizowane
    natychmiast po tym gdy użytkownik wykona czynność, przy założeniu że ta akcja po przetworzniu przez \glslink{backend}{backend}
    zostanie zakończona sukcesem.
    Jeżeli operacja nie powiedzie się, użytkownik zostanie o tym poinformowany, a stan \glslink{ui}{UI} wróci do poprzedniego.
    Dzięki zastosowaniu tego \glslink{wzorzec}{wzorca} zmniejszone jest odczucie opóźnień w działaniu aplikacji oraz
    poprawa w płynnym jej użytkowaniu.
    W projekcie Optimistic \glslink{ui}{UI} zostało wykorzystane w module Chatu.
    Gdy użytkownik wyśle nową wiadomość, natychmiast jest ona wyświetlana w oknie konwersacji.
    Poniżej przedstawiono implementację tego \glslink{wzorzec}{wzorca} w \glslink{react-component}{komponentach} \textit{ChatBottomBar} oraz \textit{ChatMessagingWindow}.
    \begin{figure}[H]
        \centering
        \includegraphics[width=1\textwidth]{attachments/wzorce-projektowe/chatBottomBar_implementation1}
        \caption{Implementacja komponentu ChatBottomBar (1)}
        \label{fig:optimistic-ui-ChatBottomBar1}
    \end{figure}
    \noindent
    \begin{figure}[H]
        \centering
        \includegraphics[width=1\textwidth]{attachments/wzorce-projektowe/chatBottomBar_implementation2}
        \caption{Implementacja komponentu ChatBottomBar (2)}
        \label{fig:optimistic-ui-ChatBottomBar2}
    \end{figure}
    \noindent
    \begin{figure}[H]
        \centering
        \includegraphics[width=1\textwidth]{attachments/wzorce-projektowe/chatBottomBar_implementation3}
        \caption{Implementacja komponentu ChatBottomBar (3)}
        \label{fig:optimistic-ui-ChatBottomBar3}
    \end{figure}
    \noindent

    \glslink{react-component}{Komponent} ten odpowiedzialny jest za przygotowanie wiadomości, przesłanie jej do wyświetlenia \glslink{react-component}{komponentowi}
    \textit{ChatMessegingWindow} oraz równoczesne wysłanie danych na odpowiedni \glslink{endpoint}{endpoint} do \glslink{backend}{backendu}.

    \begin{figure}[H]
        \centering
        \includegraphics[width=1\textwidth]{attachments/wzorce-projektowe/chatMessagingWindow_implementation1}
        \caption{Implementacja komponentu ChatMessagingWindow (1)}
        \label{fig:optimistic-ui-ChatMessagingWindow1}
    \end{figure}
    \noindent
    \begin{figure}[H]
        \centering
        \includegraphics[width=1\textwidth]{attachments/wzorce-projektowe/chatMessagingWindow_implementation2}
        \caption{Implementacja komponentu ChatMessagingWindow (2)}
        \label{fig:optimistic-ui-ChatMessagingWindow2}
    \end{figure}
    \noindent
    \begin{figure}[H]
        \centering
        \includegraphics[width=1\textwidth]{attachments/wzorce-projektowe/chatMessagingWindow_implementation3}
        \caption{Implementacja komponentu ChatMessagingWindow (3)}
        \label{fig:optimistic-ui-ChatMessagingWindow3}
    \end{figure}
    \noindent
    \begin{figure}[H]
        \centering
        \includegraphics[width=1\textwidth]{attachments/wzorce-projektowe/chatMessagingWindow_implementation4}
        \caption{Implementacja komponentu ChatMessagingWindow (4)}
        \label{fig:optimistic-ui-ChatMessagingWindow4}
    \end{figure}
    \noindent

    W tym \glslink{react-component}{komponencie} sprawdzane jest czy wiadomość pochodzi z optymistycznego przebiegu realizacji oraz jest wyświetlana
    w oknie konwersacji.
    \item \textbf{Portal} \textemdash \space to renderowanie \glslink{react-component}{komponentów} w innym miejscu drzewa \glslink{dom}{DOM} niż wynika to
    z ich hierarchii ułożenia.
    Mimo takiego ustawienia, propagacja zdarzeń przebiega w sposób niezmieniony, tzn. jest obsługiwana przez \glslink{react-component}{komponent} nadrzędny.
    Stosowane gdy \glslink{react-component}{komponent} musi być wyświetlony nad innymi elementami \glslink{ui}{UI}, aby uniknąć ograniczeń stylów rodzica.
    W projekcie ten \glslink{wzorzec}{wzorzec} zastosowano poprzez \glslink{react-component}{komponent} \textit{Modal}, którego implementację oraz przykładowe
    użycie przedstawiono poniżej.
    \begin{figure}[H]
        \centering
        \includegraphics[width=1\textwidth]{attachments/wzorce-projektowe/modal_implementation1}
        \caption{Implementacja komponentu Modal (1)}
        \label{fig:portal-modal-implementation1}
    \end{figure}
    \noindent
    \begin{figure}[H]
        \centering
        \includegraphics[width=1\textwidth]{attachments/wzorce-projektowe/modal_implementation2}
        \caption{Implementacja komponentu Modal (2)}
        \label{fig:portal-modal-implementation2}
    \end{figure}
    \noindent
    \begin{figure}[H]
        \centering
        \includegraphics[width=1\textwidth]{attachments/wzorce-projektowe/modal_usage}
        \caption{Przykładowe użycie komponentu Modal}
        \label{fig:portal-modal-usage}
    \end{figure}
    \noindent
    \item \textbf{Protected route} \textemdash \space \glslink{wzorzec}{wzorzec} polegający na uniemożliweniu dostępu
    do stron lub podstron aplikacji nieuwierzytelnionym użytkownikom.
    W projekcie do realizacji tego wzorca wykorzystano \glslink{react-component}{komponent} \textit{ProtectedRoute} oraz bibliotekę \textit{react-router-dom}.
    Gdy użytkownik będzie próbował przejść do sekcji wymagającej wcześniejszego zalogowania bez uczynienia tego,
    zostanie automatycznie przekierowany na stronę główną aplikacji.
    \begin{figure}[H]
        \centering
        \includegraphics[width=1\textwidth]{attachments/wzorce-projektowe/protectedRoute_implementation}
        \caption{Implementacja komponentu ProtectedRoute}
        \label{fig:protected-route-implementation}
    \end{figure}
    \noindent
    \begin{figure}[H]
        \centering
        \includegraphics[width=1\textwidth]{attachments/wzorce-projektowe/protectedRoute_usage}
        \caption{Przykładowe użycie komponentu ProtectedRoute}
        \label{fig:protected-route-usage}
    \end{figure}
    \noindent
\end{itemize}
