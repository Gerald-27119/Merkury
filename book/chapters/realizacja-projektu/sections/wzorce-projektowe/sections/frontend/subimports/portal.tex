%! Author = Stanisław Oziemczuk
%! Date = 21.12.2025

\subsubsection{Portal}
\label{subsubsec:Portal}

\textbf{Portal} \textemdash \space to renderowanie \glslink{react-component}{komponentów} w innym miejscu drzewa \glslink{dom}{DOM} niż wynika to
z ich hierarchii ułożenia.
Mimo takiego ustawienia, propagacja zdarzeń przebiega w sposób niezmieniony, tzn. jest obsługiwana przez \glslink{react-component}{komponent} nadrzędny.
Stosowane gdy \glslink{react-component}{komponent} musi być wyświetlony nad innymi elementami \glslink{ui}{UI}, aby uniknąć ograniczeń stylów rodzica.
W projekcie ten \glslink{wzorzec}{wzorzec} zastosowano poprzez \glslink{react-component}{komponent} \textit{Modal}, którego implementację oraz przykładowe
użycie przedstawiono poniżej.
\begin{figure}[H]
    \centering
    \includegraphics[width=1\textwidth]{attachments/wzorce-projektowe/modal_implementation1}
    \caption{Implementacja komponentu Modal (1)}
    \label{fig:portal-modal-implementation1}
\end{figure}
\noindent
\begin{figure}[H]
    \centering
    \includegraphics[width=1\textwidth]{attachments/wzorce-projektowe/modal_implementation2}
    \caption{Implementacja komponentu Modal (2)}
    \label{fig:portal-modal-implementation2}
\end{figure}
\noindent
\begin{figure}[H]
    \centering
    \includegraphics[width=1\textwidth]{attachments/wzorce-projektowe/modal_usage}
    \caption{Przykładowe użycie komponentu Modal}
    \label{fig:portal-modal-usage}
\end{figure}
\noindent
