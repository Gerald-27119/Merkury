%! Author = Stanisław Oziemczuk
%! Date = 21.12.2025

\subsubsection{Optimistic UI}
\label{subsubsec:Optimistic UI}

\textbf{Optimistic UI} \textemdash \space \glslink{wzorzec}{wzorzec}, w którym \glslink{ui}{UI} jest aktualizowane
natychmiast po tym gdy użytkownik wykona czynność, przy założeniu że ta akcja po przetworzniu przez \glslink{backend}{backend}
zostanie zakończona sukcesem.
Jeżeli operacja nie powiedzie się, użytkownik zostanie o tym poinformowany, a stan \glslink{ui}{UI} wróci do poprzedniego.
Dzięki zastosowaniu tego \glslink{wzorzec}{wzorca} zmniejszone jest odczucie opóźnień w działaniu aplikacji oraz
poprawa w płynnym jej użytkowaniu.
W projekcie Optimistic \glslink{ui}{UI} zostało wykorzystane w module Chatu.
Gdy użytkownik wyśle nową wiadomość, natychmiast jest ona wyświetlana w oknie konwersacji.
Poniżej przedstawiono implementację tego \glslink{wzorzec}{wzorca} w \glslink{react-component}{komponentach} \textit{ChatBottomBar} oraz \textit{ChatMessagingWindow}.
\begin{figure}[H]
    \centering
    \includegraphics[width=1\textwidth]{attachments/wzorce-projektowe/chatBottomBar_implementation1}
    \caption{Implementacja komponentu ChatBottomBar (1)}
    \label{fig:optimistic-ui-ChatBottomBar1}
\end{figure}
\noindent
\begin{figure}[H]
    \centering
    \includegraphics[width=1\textwidth]{attachments/wzorce-projektowe/chatBottomBar_implementation2}
    \caption{Implementacja komponentu ChatBottomBar (2)}
    \label{fig:optimistic-ui-ChatBottomBar2}
\end{figure}
\noindent
\begin{figure}[H]
    \centering
    \includegraphics[width=1\textwidth]{attachments/wzorce-projektowe/chatBottomBar_implementation3}
    \caption{Implementacja komponentu ChatBottomBar (3)}
    \label{fig:optimistic-ui-ChatBottomBar3}
\end{figure}
\noindent

\glslink{react-component}{Komponent} ten odpowiedzialny jest za przygotowanie wiadomości, przesłanie jej do wyświetlenia \glslink{react-component}{komponentowi}
\textit{ChatMessegingWindow} oraz równoczesne wysłanie danych na odpowiedni \glslink{endpoint}{endpoint} do \glslink{backend}{backendu}.

\begin{figure}[H]
    \centering
    \includegraphics[width=1\textwidth]{attachments/wzorce-projektowe/chatMessagingWindow_implementation1}
    \caption{Implementacja komponentu ChatMessagingWindow (1)}
    \label{fig:optimistic-ui-ChatMessagingWindow1}
\end{figure}
\noindent
\begin{figure}[H]
    \centering
    \includegraphics[width=1\textwidth]{attachments/wzorce-projektowe/chatMessagingWindow_implementation2}
    \caption{Implementacja komponentu ChatMessagingWindow (2)}
    \label{fig:optimistic-ui-ChatMessagingWindow2}
\end{figure}
\noindent
\begin{figure}[H]
    \centering
    \includegraphics[width=1\textwidth]{attachments/wzorce-projektowe/chatMessagingWindow_implementation3}
    \caption{Implementacja komponentu ChatMessagingWindow (3)}
    \label{fig:optimistic-ui-ChatMessagingWindow3}
\end{figure}
\noindent
\begin{figure}[H]
    \centering
    \includegraphics[width=1\textwidth]{attachments/wzorce-projektowe/chatMessagingWindow_implementation4}
    \caption{Implementacja komponentu ChatMessagingWindow (4)}
    \label{fig:optimistic-ui-ChatMessagingWindow4}
\end{figure}
\noindent

W tym \glslink{react-component}{komponencie} sprawdzane jest czy wiadomość pochodzi z optymistycznego przebiegu realizacji oraz jest wyświetlana
w oknie konwersacji.
