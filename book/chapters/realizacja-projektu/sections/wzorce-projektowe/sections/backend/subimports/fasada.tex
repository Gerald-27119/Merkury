%! Author = Stanisław Oziemczuk
%! Date = 21.12.2025

\subsubsection{Fasada}
\label{subsubsec:Fasada}

Jest to strukturalny \glslink{wzorzec}{wzorzec} projektowy, który nakłada na
\glslink{biblioteka}{bibliotekę} lub zestaw klas interfejs ułatwiający korzystanie z zawartych w nich operacji.
\begin{figure}[H]
    \centering
    \includegraphics[width=1\textwidth]{attachments/wzorce-projektowe/facade}
    \caption{Diagram klas wzorca projektowego Fasada}
    \label{fig:diagram-facade}
\end{figure}
\noindent

W projekcie Fasada została zaimplementowana jako \emph{PolygonAreaCalculator}.
To klasa odpowiedzialna za obliczanie pola powierzchni \glslink{spot}{spota} na podstawie ograniczających go punktów.
Do wykonywania konkretnych obliczeń wykorzystano \glslink{biblioteka}{bibliotekę} \emph{geographiclib}, której komponenty
są używane podczas wywołania metody \emph{calculateArea}.
Dzięki zastosowaniu Fasady, gdy zajdzie potrzeba zmiany \glslink{biblioteka}{biblioteki}, ponownej implementacji będzie wymagać tylko
metoda \emph{calculateArea} \textendash \space sposób jej wywoływania pozostanie bez zmian.

Poniżej przedstawiono implentację klasy \emph{PolygonAreaCalculator} oraz jej przykładowe użycie podczas operacji dodawania nowego \glslink{spot}{spota}.
\begin{figure}[H]
    \centering
    \includegraphics[width=1\textwidth]{attachments/wzorce-projektowe/facade_implementation}
    \caption{Implementacja wzorca Fasada}
    \label{fig:facade-implementation}
\end{figure}
\noindent
\begin{figure}[H]
    \centering
    \includegraphics[width=1\textwidth]{attachments/wzorce-projektowe/facade_usage}
    \caption{Przykładowe użycie klasy PolygonAreaCalculator}
    \label{fig:facade-usage}
\end{figure}
\noindent
