%! Author = Stanisław Oziemczuk
%! Date = 21.12.2025

\subsubsection{Builder}
\label{subsubsec:Builder}

Kreacyjny \glslink{wzorzec}{wzorzec} projektowy, który ułatwia tworzenie
skomplikowanych obiektów poprzez rozbicie tego procesu na mniejsze, konfigurowalne etapy.
Eliminuje potrzebę korzystania z konstruktorów zawierających wiele parametrów.
Stworzony zostaje obiekt budujący (Budowniczy), który implementuje poszczególne kroki konstrukcji obiektu, a na końcu
wywoływana jest metoda inicjalizująca go.
Nie jest wymagane wywołanie wszystkich kroków, ponadto można stworzyć wielu Budowniczych, kreujących różne warianty obiektu.
\begin{figure}[H]
    \centering
    \includegraphics[width=1\textwidth]{attachments/wzorce-projektowe/builder}
    \caption{Diagram klas wzorca projektowego Builder}
    \label{fig:diagram-builder}
\end{figure}
\noindent
Do zaimplementowania tego \glslink{wzorzec}{wzorca} projektowego wykrzystano \glslink{annotation}{adnotację} \emph{@Builder} z biblioteki \emph{Lombok},
która powoduje utworzenie Budowniczego dla danej klasy.
Builder został zastosowany między innymi w klasach reprezentujących encje, co poprawiło czytelność kodu przy ich tworzeniu.

Poniżej przedstawiono zastosowanie \glslink{annotation}{adnotacji} \emph{@Builder} oraz przykładowe użycie tego wzorca podczas konstruowania encji.
\begin{figure}[H]
    \centering
    \includegraphics[width=1\textwidth]{attachments/wzorce-projektowe/builder_implementation}
    \caption{Implementacja wzorca projektowego Builder}
    \label{fig:builder-implementation}
\end{figure}
\noindent
\begin{figure}[H]
    \centering
    \includegraphics[width=1\textwidth]{attachments/wzorce-projektowe/builder_usage}
    \caption{Przykładowe użycie wzorca projektowego Builder}
    \label{fig:builder-usage}
\end{figure}
\noindent
