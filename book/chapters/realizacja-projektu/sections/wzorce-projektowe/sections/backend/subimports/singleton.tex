%! Author = Stanisław Oziemczuk
%! Date = 21.12.2025

\subsubsection{Singleton}
\label{subsubsec:Singleton}

\textbf{Singleton} \textemdash \space to kreacyjny \glslink{wzorzec}{wzorzec} projektowy zapewniający
istnienie dokładnie jednej instancji danego obiektu, która jest dostępna globalnie.
Kontruktor takiego obiektu jest prywatny, a dostęp do niego odbywa się poprzez statyczną metodę zwracającą
istniejącą instancję lub jeśli jej nie ma, tworzącą nową.
Używany jest między innymi do zarządzania konfiguracjami czy połączeniami do bazy danych.
\glslink{wzorzec}{Wzorzec} Singleton łamie zasadę \glslink{srp}{Single Responsibilty}, ponieważ taki obiekt oprócz wykonywania swojej logiki,
dba o swoją unikatowość.
\begin{figure}[H]
    \centering
    \includegraphics[width=1\textwidth]{attachments/wzorce-projektowe/singleton}
    \caption{Diagram klas wzorca projektowego Singleton}
    \label{fig:diagram-singleton}
\end{figure}
\noindent
W projekcie skorzystano z \glslink{framework}{frameworka} Spring Boot, w którym wszystkie klasy oznaczone jako \glslink{bean}{Beany}
są Singletonami.
Przykładem jest nadanie klasie \glslink{annotation}{adnotacji} \emph{@Service}.
Raz stworzony obiekt jest wykorzystywany przez wszystkie inne potrzebujące go obiekty.

Poniżej przedstawiono wykorzystanie \glslink{annotation}{adnotacji} \emph{@Service} oraz użycie tej klasy jako zależność w innej klasie.
\begin{figure}[H]
    \centering
    \includegraphics[width=1\textwidth]{attachments/wzorce-projektowe/singleton_annotation}
    \caption{Adnotacja frameworka Spring Boot tworząca Singleton}
    \label{fig:singleton-annotation}
\end{figure}
\noindent
\begin{figure}[H]
    \centering
    \includegraphics[width=1\textwidth]{attachments/wzorce-projektowe/singleton_dependency}
    \caption{Użycie Singletona jako zależności}
    \label{fig:singleton-dependency}
\end{figure}
\noindent
