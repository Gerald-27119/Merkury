%! Author = Stanisław Oziemczuk
%! Date = 08/12/2025


\section{Wzorce projektowe}
\label{sec:wzorce-projektowe}

Podczas prac deweloperskich nad projektem skorzystano z różnych \glslink{wzorzec}{wzorców projektowych}.
Zarówno na \glslink{frontend}{frontendzie}, jak i \glslink{backend}{backendzie} istnieje wiele propozycji, z których członkowie zespołu starali się korzystać
w celu poszerzenia wiedzy, umiejętności oraz otrzymania wysokiej jakości pisanego kodu.
Poniżej przedstawiono wybrane rozwiązania.

\begin{itemize}
    \item \textbf{\glslink{backend}{Backend}}
    \begin{itemize}
        \item \textbf{Chain of Responsibility}
        \item \textbf{Fasada} \textemdash \space jest strukturalnym \glslink{wzorzec}{wzorcem} projektowym, który nakłada na
        bibliotekę lub zestaw klas interfejs ułatwiający korzystanie z zawartych w nich operacji.
        \begin{figure}[H]
            \centering
            \includegraphics[width=1\textwidth]{attachments/wzorce-projektowe/facade}
            \caption{Diagram klas wzorca projektowego Fasada}
            \label{fig:diagram-facade}
        \end{figure}
        \noindent

        W projekcie Fasada została zastosowana zaimplementowana jako \emph{PolygonAreaCalculator}.
        To klasa odpowiedzialna za obliczanie pola powierzchni \glslink{spot}{spota} na podstawie ograniczających go punktów.
        Do wykonywania konkretnych obliczeń wykorzystano \glslink{biblioteka}{bibliotekę} \emph{geographiclib}, której komponenty
        są używane podczas wywołana metody \emph{calculateArea}.
        Dzięki zastosowaniu Fasady, gdy zajdzie potrzeba zmiany biblioteki, ponownej implementacji będzie wymagać tylko
        metoda \emph{calculateArea} \textendash \space sposób jej wywoływania pozostanie bez zmian.

        Poniżej przedstawiono implentację klasy \emph{PolygonAreaCalculator} oraz jej przykładowe użycie podczas operacji dodawania nowego \glslink{spot}{spota}.
        \begin{figure}[H]
            \centering
            \includegraphics[width=1\textwidth]{attachments/wzorce-projektowe/facade_implementation}
            \caption{Implementacja wzorca Fasada}
            \label{fig:facade-implementation}
        \end{figure}
        \noindent
        \begin{figure}[H]
            \centering
            \includegraphics[width=1\textwidth]{attachments/wzorce-projektowe/facade_usage}
            \caption{Przykładowe użycie klasy PolygonAreaCalculator}
            \label{fig:facade-usage}
        \end{figure}
        \noindent
        \item \textbf{Singleton} \textemdash \space to kreacyjny \glslink{wzorzec}{wzorzec} projektowy zapewniający
        istnienie dokładnie jednej instancji danego obiektu, która jest dostępna globalnie.
        Kontruktor takiego obiektu jest prywatny, a dostęp do niego odbywa się poprzez statyczną metodę zwracającą
        istniejącą instancję lub jeśli jej nie ma, tworzącą nową.
        Używany jest między innymi do zarządzania konfiguracjami czy połączeniami do bazy danych.
        \glslink{wzorzec}{Wzorzec} Singleton łamie zasadę \glslink{srp}{Single Responsibilty}, ponieważ taki obiekt oprócz wykonywania swojej logiki,
        dba o swoją unikatowość.
        \begin{figure}[H]
            \centering
            \includegraphics[width=1\textwidth]{attachments/wzorce-projektowe/singleton}
            \caption{Diagram klas wzorca projektowego Singleton}
            \label{fig:diagram-singleton}
        \end{figure}
        \noindent
        \item \textbf{Builder} \textemdash \space kreacyjny \glslink{wzorzec}{wzorzec} projektowy, który ułatwia tworzenie
        skomplikowanych obiektów poprzez rozbicie tego procesu na mniejsze, konfigurowalne etapy.
        Eliminuje potrzebę korzystania z kontruktorów zawierjących wiele parametrów.
        Stworzony zostaje obiekt budujący (Budowniczy), który implemnetuje poszczególne kroki kontrukcji obiektu, a na końcu
        wywoływana jest metoda inicjalizująca go.
        Nie jest wymagane wywołanie wszystkich kroków, ponadto można stworzyć wielu Budowniczych, kreujących różne warianty obiektu.
        \begin{figure}[H]
            \centering
            \includegraphics[width=1\textwidth]{attachments/wzorce-projektowe/builder}
            \caption{Diagram klas wzorca projektowego Builder}
            \label{fig:diagram-builder}
        \end{figure}
        \noindent
    \end{itemize}
    \item \textbf{\glslink{frontend}{Frontend}}
    \begin{itemize}
        \item \textbf{Hooks Pattern}
        \item \textbf{Error Boundary}
        \item \textbf{Portal}
        \item \textbf{Protected route}
    \end{itemize}
\end{itemize}

Opisy \glslink{wzorzec}{wzorców} projektowych użytych na \glslink{backend}{backendzie} (oprócz MVC) zostały
wykonane na podstawie treści zawartych w książce \cite{wzorce-projektowe}.