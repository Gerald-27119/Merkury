%! Author = Stanisław Oziemczuk
%! Date = 08/12/2025


\section{Wzorce projektowe}
\label{sec:wzorce-projektowe}

Podczas prac deweloperskich nad projektem skorzystano z różnych \glslink{wzorzec}{wzorców projektowych}.
Zarówno na \glslink{frontend}{frontendzie}, jak i \glslink{backend}{backendzie} istnieje wiele propozycji, z których członkowie zespołu starali się korzystać
w celu poszerzenia wiedzy, umiejętności oraz otrzymania wysokiej jakości pisanego kodu.
Poniżej przedstawiono wybrane rozwiązania.

\begin{itemize}
    \item \textbf{\glslink{backend}{Backend}}
    \begin{itemize}
        \item \textbf{Chain of Responsibility}
        \item \textbf{Fasada} \textemdash \space jest strukturalnym \glslink{wzorzec}{wzorcem} projektowym, który nakłada na
        bibliotekę lub zestaw klas interfejs ułatwiający korzystanie z zawartych w nich operacji.
        \begin{figure}[H]
            \centering
            \includegraphics[width=1\textwidth]{attachments/wzorce-projektowe/facade}
            \caption{Diagram klas wzorca projektowego Fasada}
            \label{fig:diagram-facade}
        \end{figure}
        \noindent

        W projekcie Fasada została zastosowana zaimplementowana jako \emph{PolygonAreaCalculator}.
        To klasa odpowiedzialna za obliczanie pola powierzchni \glslink{spot}{spota} na podstawie ograniczających go punktów.
        Do wykonywania konkretnych obliczeń wykorzystano \glslink{biblioteka}{bibliotekę} \emph{geographiclib}, której komponenty
        są używane podczas wywołana metody \emph{calculateArea}.
        Dzięki zastosowaniu Fasady, gdy zajdzie potrzeba zmiany biblioteki, ponownej implementacji będzie wymagać tylko
        metoda \emph{calculateArea} \textendash \space sposób jej wywoływania pozostanie bez zmian.

        Poniżej przedstawiono implentację klasy \emph{PolygonAreaCalculator} oraz jej przykładowe użycie podczas operacji dodawania nowego \glslink{spot}{spota}.
        \begin{figure}[H]
            \centering
            \includegraphics[width=1\textwidth]{attachments/wzorce-projektowe/facade_implementation}
            \caption{Implementacja wzorca Fasada}
            \label{fig:facade-implementation}
        \end{figure}
        \noindent
        \begin{figure}[H]
            \centering
            \includegraphics[width=1\textwidth]{attachments/wzorce-projektowe/facade_usage}
            \caption{Przykładowe użycie klasy PolygonAreaCalculator}
            \label{fig:facade-usage}
        \end{figure}
        \noindent
        \item \textbf{Singleton} \textemdash \space to kreacyjny \glslink{wzorzec}{wzorzec} projektowy zapewniający
        istnienie dokładnie jednej instancji danego obiektu, która jest dostępna globalnie.
        Kontruktor takiego obiektu jest prywatny, a dostęp do niego odbywa się poprzez statyczną metodę zwracającą
        istniejącą instancję lub jeśli jej nie ma, tworzącą nową.
        Używany jest między innymi do zarządzania konfiguracjami czy połączeniami do bazy danych.
        \glslink{wzorzec}{Wzorzec} Singleton łamie zasadę \glslink{srp}{Single Responsibilty}, ponieważ taki obiekt oprócz wykonywania swojej logiki,
        dba o swoją unikatowość.
        \begin{figure}[H]
            \centering
            \includegraphics[width=1\textwidth]{attachments/wzorce-projektowe/singleton}
            \caption{Diagram klas wzorca projektowego Singleton}
            \label{fig:diagram-singleton}
        \end{figure}
        \noindent
        W projekcie skorzystano z \glslink{framework}{frameworka} Spring Boot, w którym wszystkie klasy oznaczone jako \glslink{bean}{Beany}
        są Singletonami.
        Przykładem jest nadanie klasie \glslink{annotaion}{adnotacji} \emph{@Service}.
        Raz stworzony obiekt jest wykorzystywany przez wszystkie inne potrzebujące go obiekty.

        Poniżej przedstawiono wykorzystanie \glslink{annotaion}{adnotacji} \emph{@Service} oraz użycie tej klasy jako zależność w innej klasie.
        \begin{figure}[H]
            \centering
            \includegraphics[width=1\textwidth]{attachments/wzorce-projektowe/singleton_annotation}
            \caption{Adnotacja frameworka Spring Boot tworząca Singleton}
            \label{fig:singleton-annotation}
        \end{figure}
        \noindent
        \begin{figure}[H]
            \centering
            \includegraphics[width=1\textwidth]{attachments/wzorce-projektowe/singleton_dependency}
            \caption{Użycie Singletona jako zależności}
            \label{fig:singleton-dependency}
        \end{figure}
        \noindent
        \item \textbf{Builder} \textemdash \space kreacyjny \glslink{wzorzec}{wzorzec} projektowy, który ułatwia tworzenie
        skomplikowanych obiektów poprzez rozbicie tego procesu na mniejsze, konfigurowalne etapy.
        Eliminuje potrzebę korzystania z kontruktorów zawierjących wiele parametrów.
        Stworzony zostaje obiekt budujący (Budowniczy), który implemnetuje poszczególne kroki kontrukcji obiektu, a na końcu
        wywoływana jest metoda inicjalizująca go.
        Nie jest wymagane wywołanie wszystkich kroków, ponadto można stworzyć wielu Budowniczych, kreujących różne warianty obiektu.
        \begin{figure}[H]
            \centering
            \includegraphics[width=1\textwidth]{attachments/wzorce-projektowe/builder}
            \caption{Diagram klas wzorca projektowego Builder}
            \label{fig:diagram-builder}
        \end{figure}
        \noindent
        Do zaimplementowania tego \glslink{wzorzec}{wzorca} projektowego wykrzystano \glslink{annotaion}{adnotację} \emph{@Builder} z biblioteki \emph{Lombok},
        która powoduje utworzenie Budowniczego dla danej klasy.
        Builder został zastosowany do wszystkich klas reprezentujących encje, co poprawiło czytelność kodu przy ich tworzeniu.

        Poniżej przedstawiono zastosowanie \glslink{annotaion}{adnotacji} \emph{@Builder} oraz przykładowe użycie tego wzorca podczas konstruowania encji.
        \begin{figure}[H]
            \centering
            \includegraphics[width=1\textwidth]{attachments/wzorce-projektowe/builder_implementation}
            \caption{Implementacja wzorca projektowego Builder}
            \label{fig:builder-implementation}
        \end{figure}
        \noindent
        \begin{figure}[H]
            \centering
            \includegraphics[width=1\textwidth]{attachments/wzorce-projektowe/builder_usage}
            \caption{Przykładowe użycie wzorca projektowego Builder}
            \label{fig:builder-usage}
        \end{figure}
        \noindent
    \end{itemize}
    \item \textbf{\glslink{frontend}{Frontend}}
    \begin{itemize}
        \item \textbf{Hooks Pattern} \textemdash \space \glslink{wzorzec}{wzorzec} projektowy, który umożliwia korzystanie z
        mechanizmów \glslink{react}{Reacta} takich jak \glslink{stan}{stan} czy cykl życia \glslink{react-component}{komponentu} w
        \glslink{react-component}{komponentach} funkcyjnych.
        Przed wprowadzeniem \glslink{hook}{hook'ów} do zarządzania tymi właściwościami trzeba było tworzyć \glslink{react-component}{komponenty} klasowe,
        co powodowało większe skomplikowanie kodu.
        Hooks Pattern pozwala również na tworzenie własnych \glslink{hook}{hook'ów} zawierających logikę używaną w wielu \glslink{react-component}{komponentach}.

        W projekcie wykorzystano ten \glslink{wzorzec}{wzorzec} do zarządzania stanem oraz efektami ubocznymi w \glslink{react-component}{komponentach}
        za pomocą wbudowanych \glslink{hook}{hook'ów} \emph{useState} i \emph{useEffect}.
        \begin{figure}[H]
            \centering
            \includegraphics[width=1\textwidth]{attachments/wzorce-projektowe/useState_usage}
            \caption{Przykładowe użycie hook'a useState}
            \label{fig:hooks-useState-usage}
        \end{figure}
        \noindent
        \begin{figure}[H]
            \centering
            \includegraphics[width=1\textwidth]{attachments/wzorce-projektowe/useEffect_usage}
            \caption{Przykładowe użycie hook'a useEffect}
            \label{fig:hooks-useEffect-usage}
        \end{figure}
        \noindent
        Zaimplementowano też własne \glslink{hook}{hooki}.
        Jednym z nich jest \emph{useBoolean} obługujący logikę do zarządzania zmiennymi przyjmujących wartości
        \textit{true} lub \textit{false}.
        Udostępniane są funckje do ustawienia wartości, a także do zmiany jej na przeciwną do poprzedniej.
        Korzystanie z tego \glslink{hook}{hook'a} pozwoliło uniknąć powtarzalności kodu oraz poprawiło jego czytelność.
        Poniżej przedstawiono implementację oraz przykładowe użycie \glslink{hook}{hooka} \emph{useBoolean}.
        \begin{figure}[H]
            \centering
            \includegraphics[width=1\textwidth]{attachments/wzorce-projektowe/useBoolean_implementation}
            \caption{Implementacja hook'a useBoolean}
            \label{fig:hooks-implementation}
        \end{figure}
        \noindent
        \begin{figure}[H]
            \centering
            \includegraphics[width=1\textwidth]{attachments/wzorce-projektowe/useBoolean_usage}
            \caption{Przykładowe użycie hook'a useBoolean}
            \label{fig:hooks-useBoolean-usage}
        \end{figure}
        \noindent
        \item \textbf{Optimistic UI} \textemdash \space \glslink{wzorzec}{wzorzec}, w którym \glslink{ui}{UI} jest aktualizowane
        natychmiast po tym gdy użytkownik wykona czynność, przy założeniu że ta akcja po przetworzniu przez \glslink{backend}{backend}
        zostanie zakończona sukcesem.
        Jeżeli operacja nie powiedzie się, użytkownik zostanie o tym poinformowany, a stan \glslink{ui}{UI} wróci do poprzedniego.
        Dzięki zastosowaniu tego \glslink{wzorzec}{wzorca} zmniejszone jest odczucie opóźnień w działaniu aplikacji oraz
        poprawa w płynnym jej użytkowaniu.
        W projekcie Optimistic \glslink{ui}{UI} zostało wykorzystane w module Chatu.
        Gdy użytkownik wyśle nową wiadomość, natychmiast jest ona wyświetlana w oknie konwersacji.
        Poniżej przedstawiono implementację tego \glslink{wzorzec}{wzorca} w \glslink{react-component}{komponentach} \textit{ChatBottomBar} oraz \textit{ChatMessagingWindow}.
        \begin{figure}[H]
            \centering
            \includegraphics[width=1\textwidth]{attachments/wzorce-projektowe/chatBottomBar_implementation1}
            \caption{Implementacja komponentu ChatBottomBar (1)}
            \label{fig:optimistic-ui-ChatBottomBar1}
        \end{figure}
        \noindent
        \begin{figure}[H]
            \centering
            \includegraphics[width=1\textwidth]{attachments/wzorce-projektowe/chatBottomBar_implementation2}
            \caption{Implementacja komponentu ChatBottomBar (2)}
            \label{fig:optimistic-ui-ChatBottomBar2}
        \end{figure}
        \noindent
        \begin{figure}[H]
            \centering
            \includegraphics[width=1\textwidth]{attachments/wzorce-projektowe/chatBottomBar_implementation3}
            \caption{Implementacja komponentu ChatBottomBar (3)}
            \label{fig:optimistic-ui-ChatBottomBar3}
        \end{figure}
        \noindent

        \glslink{react-component}{Komponent} ten odpowiedzialny jest za przygotowanie wiadomości, przesłanie jej do wyświetlenia \glslink{react-component}{komponentowi}
        \textit{ChatMessegingWindow} oraz równoczesne wysłanie danych na odpowiedni endpoint do \glslink{backend}{backendu}.

        \begin{figure}[H]
            \centering
            \includegraphics[width=1\textwidth]{attachments/wzorce-projektowe/chatMessagingWindow_implementation1}
            \caption{Implementacja komponentu ChatMessagingWindow (1)}
            \label{fig:optimistic-ui-ChatMessagingWindow1}
        \end{figure}
        \noindent
        \begin{figure}[H]
            \centering
            \includegraphics[width=1\textwidth]{attachments/wzorce-projektowe/chatMessagingWindow_implementation2}
            \caption{Implementacja komponentu ChatMessagingWindow (2)}
            \label{fig:optimistic-ui-ChatMessagingWindow2}
        \end{figure}
        \noindent
        \begin{figure}[H]
            \centering
            \includegraphics[width=1\textwidth]{attachments/wzorce-projektowe/chatMessagingWindow_implementation3}
            \caption{Implementacja komponentu ChatMessagingWindow (3)}
            \label{fig:optimistic-ui-ChatMessagingWindow3}
        \end{figure}
        \noindent
        \begin{figure}[H]
            \centering
            \includegraphics[width=1\textwidth]{attachments/wzorce-projektowe/chatMessagingWindow_implementation4}
            \caption{Implementacja komponentu ChatMessagingWindow (4)}
            \label{fig:optimistic-ui-ChatMessagingWindow4}
        \end{figure}
        \noindent

        W tym \glslink{react-component}{komponencie} sprawdzane jest czy wiadomość pochodzi z optymistycznego przebiegu realizacji oraz jest wyświetlana
        w oknie konwersacji.
        \item \textbf{Portal} \textemdash \space to renderowanie \glslink{react-component}{komponentów} w innym miejscu drzewa \glslink{dom}{DOM} niż wynika to
        z ich hierarchii ułożenia.
        Mimo takiego ustawienia, propagacja zdarzeń przebiega w sposób niezmieniony, tzn. jest obsługiwana przez \glslink{react-component}{komponent} nadrzędny.
        Stosowane gdy \glslink{react-component}{komponent} musi być wyświetlony nad innymi elementami \glslink{ui}{UI}, aby uniknąć ograniczeń stylów rodzica.
        W projekcie ten \glslink{wzorzec}{wzorzec} zastosowano poprzez \glslink{react-component}{komponent} \textit{Modal}, którego implementację oraz przykładowe
        użycie przedstawiono poniżej.
        \begin{figure}[H]
            \centering
            \includegraphics[width=1\textwidth]{attachments/wzorce-projektowe/modal_implementation1}
            \caption{Implementacja komponentu Modal (1)}
            \label{fig:portal-modal-implementation1}
        \end{figure}
        \noindent
        \begin{figure}[H]
            \centering
            \includegraphics[width=1\textwidth]{attachments/wzorce-projektowe/modal_implementation2}
            \caption{Implementacja komponentu Modal (2)}
            \label{fig:portal-modal-implementation2}
        \end{figure}
        \noindent
        \begin{figure}[H]
            \centering
            \includegraphics[width=1\textwidth]{attachments/wzorce-projektowe/modal_usage}
            \caption{Przykładowe użycie komponentu Modal}
            \label{fig:portal-modal-usage}
        \end{figure}
        \noindent
        \item \textbf{Protected route} \textemdash \space \glslink{wzorzec}{wzorzec} polegający na uniemożliweniu dostępu
        do stron lub podstron aplikacji nieuwierzytelnionym użytkownikom.
        W projekcie do realizacji tego wzorca wykorzystano \glslink{react-component}{komponent} \textit{ProtectedRoute} oraz bibliotekę \textit{react-router-dom}.
        Gdy użytkownik będzie próbował przejść do sekcji wymagającej wcześniejszego zalogowania bez uczynienia tego,
        zostanie automatycznie przekierowany na stronę główną aplikacji.
        \begin{figure}[H]
            \centering
            \includegraphics[width=1\textwidth]{attachments/wzorce-projektowe/protectedRoute_implementation}
            \caption{Implementacja komponentu ProtectedRoute}
            \label{fig:protected-route-implementation}
        \end{figure}
        \noindent
        \begin{figure}[H]
            \centering
            \includegraphics[width=1\textwidth]{attachments/wzorce-projektowe/protectedRoute_usage}
            \caption{Przykładowe użycie komponentu ProtectedRoute}
            \label{fig:protected-route-usage}
        \end{figure}
        \noindent
    \end{itemize}
\end{itemize}

Opisy \glslink{wzorzec}{wzorców} projektowych użytych na \glslink{backend}{backendzie} zostały
wykonane na podstawie treści zawartych w książce \cite{wzorce-projektowe}.
