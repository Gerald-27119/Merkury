%! Author = Stanisław Oziemczuk
%! Date = 30/12/2025

\section{Implementacja OAuth}
\label{sec:implementacja-oauth}

\glslink{oauth}{OAuth} w wersji 2.0 to \glslink{framework}{framework} umożliwiający autoryzację w aplikacji poprzez serwisy
stron trzecich.
Często stosowany jest w aplikacjach internetowych dając użytkownikom opcję zalogowania lub założenia konta za pomocą
istniejących już kont Google, Facebook, \glslink{github}{GitHub} itd..
\\
Poniżej przedstawiono diagram opisujący przepływ \glslink{oauth}{OAuth'a} 2.0.
\begin{figure}[H]
    \centering
    \includegraphics[width=1\textwidth]{attachments/implementacja-oauth/oauth_diagram_rfc}
    \caption{Diagram przedstawiający przepły OAuth 2.0; zacytowany z \cite{oauth-docs-rfc}}
    \label{fig:oauth-flow-diagram}
\end{figure}

\begin{itemize}
    \item \texttt{Client} \textendash \space to aplikacja prosząca o dostęp do zasobu, np. adresu email, nazwy użytkownika
    oraz określająca rodzaj dostępu (odczyt, edycja)
    \item \texttt{Resource Owner} \textendash \space właściciel zasobu, do krórego dostęp prosi aplikacja; w tym wypadku
    jest to użytkownik systemu
    \item \texttt{Authorization Server} \textendash \space serwer nadający \texttt{Client'owi} uprawnienia do danego zasobu
    \item \texttt{Resource Server} \textendash \space serwer, na którym przechowywany jest zasób; na podstawie uprawnień
    \texttt{Client'a} przekazuje mu go lub odmawia dostępu
\end{itemize}
