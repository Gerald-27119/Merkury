%! Author = Stanisław Oziemczuk
%! Date = 30/12/2025


\section{Implementacja OAuth}
\label{sec:implementacja-oauth}

\glslink{oauth}{OAuth} w wersji 2.0 to \glslink{framework}{framework} umożliwiający autoryzację w systemach poprzez serwisy
stron trzecich.
Często stosowany jest w aplikacjach internetowych dając użytkownikom opcję zalogowania lub założenia konta za pomocą
istniejących już kont Google, Facebook, \glslink{github}{GitHub} itd..
\\
Poniżej przedstawiono diagram opisujący przepływ \glslink{oauth}{OAuth'a} 2.0.
\begin{figure}[H]
    \centering
    \includegraphics[width=1\textwidth]{attachments/implementacja-oauth/oauth_diagram_rfc}
    \caption{Diagram przedstawiający przepły OAuth 2.0; zacytowany z \cite{oauth-docs-rfc}}
    \label{fig:oauth-flow-diagram}
\end{figure}

\begin{itemize}
    \item \texttt{Client} \textendash \space to aplikacja prosząca o dostęp do zasobu, np. adresu email, nazwy użytkownika
    oraz określająca rodzaj dostępu (odczyt, edycja)
    \item \texttt{Resource Owner} \textendash \space właściciel zasobu, do krórego dostęp prosi aplikacja; w tym wypadku
    jest to użytkownik systemu
    \item \texttt{Authorization Server} \textendash \space serwer nadający \texttt{Client'owi} uprawnienia do danego zasobu
    \item \texttt{Resource Server} \textendash \space serwer, na którym przechowywany jest zasób; na podstawie uprawnień
    \texttt{Client'a} przekazuje mu go lub odmawia dostępu
\end{itemize}

Opis przepływu:
\begin{enumerate}[label=\Alph*.]
    \item \texttt{Client} prosi \texttt{Resource Owner'a} o dostęp do zasobu określając zakres oraz typy operacji, które będzie
    na nim wykonywać.
    Często zapytanie zamiast trafiać bezpośrednio do \texttt{Resource Owner'a} obsługiwane jest przez serwer autentykacji, czyli
    miejsce w którym użytkownik musi się zalogować do systemu.
    \item \texttt{Client'owi} zwracane jest potwierdzenie przyznania dostępu, na przykład w formie tokenu.
    \item \texttt{Client} przesyła potwierdzenie dostępu do \texttt{Authorization Server'a} uwierzytelniając się i prosząc
    o token dostępu do zasobu z \texttt{Resource Server'a}.
    \item \texttt{Authorization Server} sprawdza przesłane dane i jeśli proces walidacji zostanie zakończony pomyślnie,
    przekazuje token dostępu do zasobu.
    \item \texttt{Client} wysyła zapytanie z uzyskanym wcześniej tokenem do \texttt{Resource Server'a} prosząc o przesłanie
    zasobu.
    \item \texttt{Resource Server} przeprowadza walidację tokena i jeśli zakończy się ona powodzeniem, przesyła zasób
    \texttt{Client'owi}.
\end{enumerate}

W projekcie do implementacji \glslink{oauth}{OAutha 2.0} wykorzystano \glslink{biblioteka}{bilioteki} \texttt{Spring Boot}:
\texttt{spring-boot-starter-oauth2-resource-server} oraz \\ \texttt{spring-boot-starter-oauth2-client}.
Zdecydowano się na umożliwienie użytkownikom logowania i rejestracji do aplikacji poprzez konta Google i \glslink{github}{GitHub},
dalej nazywanych \emph{Provider'ami}.
Aby móc korzystać z usługi autoryzacji, u każdego z \texttt{Provider'ów} należało zarejestrować aplikację poprzez
utworzenie konta deweloperskiego i podanie linków, na które użytkonicy po pomyślnym procesie będą przekierowywani.
Od każdego z nich otrzymano parę: \texttt{CLIENT\_ID}, \texttt{CLIENT\_SECRET}, które pozwalają na uwierzytelnienie aplikacji.
Dane te nie powinny być dostępne dla osób trzecich, dlatego ze względów bezpieczeństwa umieszczono jest w zmiennych
środowikowych systemu.
Wymienione wyżej \glslink{biblioteka}{biblioteki} wymagają skonfigurowania dla każdego z \texttt{Provider'ów} zakresu dostępu do zasobu, linków do
przekierowań, autoryzacji oraz wartości danych uwierzytelniających (rys. \ref{fig:oauth-config}).

\begin{figure}[H]
    \centering
    \includegraphics[width=1\textwidth]{attachments/implementacja-oauth/oauth_konfiguracja}
    \caption{Konfiguracja OAuth w Spring Boot}
    \label{fig:oauth-config}
\end{figure}

W formularzu rejestracji i logowania znajdują się przyciski umożliwiające skorzystanie z kont w serwisach Google lub \glslink{github}{GitHub}.
Kliknięcie ich powoduje przekierowanie na \glslink{endpoint}{endpoint} wygenerowany przez użyte \glslink{biblioteka}{biblioteki},
następnie użytkownik przeniesiony jest na stronę wybranego \texttt{Provider'a}.
Po pomyślnej autoryzacji następuje przekierowanie na \glslink{endpoint}{endpoint} \glslink{backend}{backendu} aplikacji
\texttt{/account/login-success}, w którym z serwisu \texttt{accountService} wywoływana jest
metoda \texttt{handleOAuth2User} (rys. \ref{fig:oauth-service-impl}).

\begin{figure}[H]
    \centering
    \includegraphics[width=1\textwidth]{attachments/implementacja-oauth/oauth_service_impl}
    \caption{Implementacja metody handleOAuth2User}
    \label{fig:oauth-service-impl}
\end{figure}

Z tokenu przesłanego przez \texttt{Provider'a} ustalany jest jego typ (Google lub \glslink{github}{GitHub}) oraz
obiekt zakodowanego użytkownika.
Następnie odczytywane są nazwa i adres email użytkownika, w przypadku niepowodzenia rzucany jest odpowiedni wyjątek.
Po sprawdzeniu czy w bazie danych użytkownik istnieje, wywoływane są odpowiednio metody do logowania lub
rejestracji.
Pomyślny przebieg operacji skutkuje przekierowaniem na adres części \glslink{frontend}{frontendowej} aplikacji.

