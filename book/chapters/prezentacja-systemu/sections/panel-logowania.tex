%! Author = Mateusz
%! Date = 01/01/2026

\section{Panel uwierzytelnienia}
\label{sec:panel-uwierzytelnienia}

Panel uwierzytelniania użytkownika obejmuje trzy widoki: logowanie, rejestrację oraz
resetowanie hasła.
Każdy z formularzy został przygotowany w spójnej stylistyce oraz posiada wariant w motywie ciemnym.
W formularzach zastosowano etykiety pływające -- po aktywacji pola (kliknięciu lub rozpoczęciu wpisywania)
etykieta przenosi się nad pole, co poprawia czytelność podczas uzupełniania danych (rys. \ref{fig:auth-login-focus}).

\subsection{Logowanie}
\label{ssec:logowanie}

Widok logowania przedstawiono na rys. \ref{fig:auth-login}.
Formularz składa się z pól \textit{Username} oraz \textit{Password}, przycisku \textit{Sign In} oraz
odnośnika \textit{Forgot Password?}, który przenosi do modułu resetowania hasła
(opisano w podrozdz. \ref{ssec:forgot-password}). Dodatkowo udostępniono logowanie z użyciem zewnętrznych
dostawców (Google oraz \gls{github}), co umożliwia uwierzytelnienie bez podawania danych konta lokalnego.

Zachowanie etykiet pól po aktywacji pokazano na rys. \ref{fig:auth-login-focus}.
W przypadku próby wysłania formularza bez wymaganych danych wyświetlane są komunikaty walidacyjne bezpośrednio
pod odpowiednimi polami (rys. \ref{fig:auth-login-errors}).
Przygotowano również wersję w motywie ciemnym (rys. \ref{fig:auth-login-dark}).

\begin{figure}[H]
    \centering
    \includegraphics[width=1\textwidth]{attachments/prezentacja-systemu/panel-logowania/login}
    \caption{Panel logowania -- widok podstawowy.}
    \label{fig:auth-login}
\end{figure}

\begin{figure}[H]
    \centering
    \includegraphics[width=1\textwidth]{attachments/prezentacja-systemu/panel-logowania/login-hover}
    \caption{Panel logowania -- etykiety pływające po aktywacji pól formularza.}
    \label{fig:auth-login-focus}
\end{figure}

\begin{figure}[H]
    \centering
    \includegraphics[width=1\textwidth]{attachments/prezentacja-systemu/panel-logowania/login-errors}
    \caption{Panel logowania -- komunikaty walidacyjne przy braku wymaganych danych.}
    \label{fig:auth-login-errors}
\end{figure}

\begin{figure}[H]
    \centering
    \includegraphics[width=1\textwidth]{attachments/prezentacja-systemu/panel-logowania/login-dark}
    \caption{Panel logowania -- motyw ciemny.}
    \label{fig:auth-login-dark}
\end{figure}

\subsection{Rejestracja}
\label{ssec:rejestracja}

Widok rejestracji przedstawiono na rys. \ref{fig:auth-register}.
Formularz zawiera pola: \textit{Username}, \textit{E-Mail}, \textit{Password} oraz \textit{Confirm Password}.
Proces rejestracji uruchamiany jest przyciskiem \textit{Sign Up}.
Analogicznie do logowania, dostępna jest rejestracja z użyciem zewnętrznych dostawców (Google oraz \gls{github}),
co umożliwia utworzenie konta bez ręcznego uzupełniania wszystkich pól.

W przypadku podania niekompletnych danych (pustych pól) wyświetlane są komunikaty walidacyjne w
bezpośrednim sąsiedztwie pól, których dotyczy błąd (rys. \ref{fig:auth-register-errors}).
Opracowano również wariant w motywie ciemnym (rys. \ref{fig:auth-register-dark}).

\begin{figure}[H]
    \centering
    \includegraphics[width=1\textwidth]{attachments/prezentacja-systemu/panel-logowania/register}
    \caption{Panel rejestracji -- widok podstawowy.}
    \label{fig:auth-register}
\end{figure}

\begin{figure}[H]
    \centering
    \includegraphics[width=1\textwidth]{attachments/prezentacja-systemu/panel-logowania/register-errors}
    \caption{Panel rejestracji -- komunikaty walidacyjne przy braku wymaganych danych.}
    \label{fig:auth-register-errors}
\end{figure}

\begin{figure}[H]
    \centering
    \includegraphics[width=1\textwidth]{attachments/prezentacja-systemu/panel-logowania/register-dark}
    \caption{Panel rejestracji -- motyw ciemny.}
    \label{fig:auth-register-dark}
\end{figure}

\subsection{Resetowanie hasła}
\label{ssec:forgot-password}

Widok resetowania hasła pokazano na rys. \ref{fig:auth-forgot}.
Formularz składa się z pola \textit{Email} oraz przycisku \textit{Remind me}.
Wprowadzenie adresu e-mail jest wymagane, a w przypadku próby wysłania pustego formularza
wyświetlany jest komunikat walidacyjny (rys. \ref{fig:auth-forgot-errors}).
Dostępny jest również wariant w motywie ciemnym (rys. \ref{fig:auth-forgot-dark}).

\begin{figure}[H]
    \centering
    \includegraphics[width=1\textwidth]{attachments/prezentacja-systemu/panel-logowania/forgot-password}
    \caption{Panel resetowania hasła -- widok podstawowy.}
    \label{fig:auth-forgot}
\end{figure}

\begin{figure}[H]
    \centering
    \includegraphics[width=1\textwidth]{attachments/prezentacja-systemu/panel-logowania/forgot-password-error}
    \caption{Panel resetowania hasła -- komunikat walidacyjny przy braku adresu e-mail.}
    \label{fig:auth-forgot-errors}
\end{figure}

\begin{figure}[H]
    \centering
    \includegraphics[width=1\textwidth]{attachments/prezentacja-systemu/panel-logowania/forgot-password-dark}
    \caption{Panel resetowania hasła -- motyw ciemny.}
    \label{fig:auth-forgot-dark}
\end{figure}
