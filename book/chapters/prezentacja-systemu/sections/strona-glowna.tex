%! Author = Mateusz
%! Date = 31/12/2025

\section{Wyszukiwarka spotów}
\label{sec:wyszukiwarka-spotow}

W systemie udostępniono wyszukiwarkę \glslink{spot}{spotów} działającą w dwóch trybach: prostym oraz zaawansowanym.
Przełączanie trybu realizowane jest za pomocą przycisków znajdujących się w górnej części widoku
(rys.~\ref{fig:search-simple-empty} oraz rys.~\ref{fig:search-advanced-empty}).
W obu trybach wyniki prezentowane są w postaci kart \glslink{spot}{spotów} zawierających nazwę, lokalizację,
ocenę, tagi oraz przyciski akcji (\textit{Details} oraz \textit{See on map}).

\subsection{Tryb prosty}
\label{subsec:wyszukiwarka-prosty}

W trybie prostym główny nacisk położono na szybkie wyszukiwanie.
W górnej części ekranu znajduje się przycisk przejścia do trybu zaawansowanego, a poniżej umieszczono
pola wyszukiwania lokalizacji wraz z mechanizmem podpowiedzi (autouzupełniania).
W sytuacji, gdy nie wprowadzono żadnych parametrów, wyświetlany jest komunikat zachęcający do rozpoczęcia
wyszukiwania (rys.~\ref{fig:search-simple-empty}).

\begin{figure}[H]
    \centering
    \includegraphics[width=1\textwidth]{attachments/prezentacja-systemu/strona-glowna/main-page-empty}
    \caption{Tryb prosty wyszukiwarki spotów — stan początkowy bez wyników.}
    \label{fig:search-simple-empty}
\end{figure}

Po wprowadzeniu danych i wykonaniu wyszukiwania prezentowana jest lista wyników w formie kart
\glslink{spot}{spotów} (rys.~\ref{fig:search-simple-list}). Każda karta zawiera podstawowe informacje o obiekcie,
w tym ocenę oraz zestaw tagów, co ułatwia szybkie porównanie wyników.

\begin{figure}[H]
    \centering
    \includegraphics[width=1\textwidth]{attachments/prezentacja-systemu/strona-glowna/main-page-list}
    \caption{Tryb prosty wyszukiwarki — lista znalezionych spotów w postaci kart.}
    \label{fig:search-simple-list}
\end{figure}

Podczas wpisywania nazwy miasta system wyświetla podpowiedzi, co pozwala ograniczyć liczbę błędów oraz
przyspiesza wybór poprawnej lokalizacji (rys.~\ref{fig:search-simple-suggestions}).

\begin{figure}[H]
    \centering
    \includegraphics[width=1\textwidth]{attachments/prezentacja-systemu/strona-glowna/main-page-tooltip}
    \caption{Tryb prosty — podpowiedzi (autouzupełnianie) w polu wyszukiwania lokalizacji.}
    \label{fig:search-simple-suggestions}
\end{figure}

%Wyszukiwarka w trybie prostym została przygotowana również w wersji ciemnej (rys.~\ref{fig:search-simple-dark}).

%\begin{figure}[H]
%    \centering
%    \includegraphics[width=1\textwidth]{attachments/prezentacja-systemu/strona-glowna/main-page-dark}
%    \caption{Tryb prosty wyszukiwarki w motywie ciemnym.}
%    \label{fig:search-simple-dark}
%\end{figure}

\subsection{Tryb zaawansowany}
\label{subsec:wyszukiwarka-zaawansowany}

Tryb zaawansowany umożliwia precyzyjniejsze wyszukiwanie poprzez zastosowanie dodatkowych filtrów oraz sortowania.
W górnej części widoku dostępny jest przycisk przejścia do trybu prostego, a poniżej umieszczono rozbudowany
panel wyszukiwania z polami lokalizacji, tagów, sortowaniem oraz filtrowaniem po ocenie.

W przypadku braku wprowadzonych parametrów prezentowany jest stan początkowy analogiczny do trybu prostego
(rys.~\ref{fig:search-advanced-empty}).

\begin{figure}[H]
    \centering
    \includegraphics[width=1\textwidth]{attachments/prezentacja-systemu/strona-glowna/advance-page-empty}
    \caption{Tryb zaawansowany wyszukiwarki — stan początkowy bez wyników.}
    \label{fig:search-advanced-empty}
\end{figure}

Po wykonaniu wyszukiwania wyniki wyświetlane są w postaci listy kart \glslink{spot}{spotów} (rys.~\ref{fig:search-advanced-list}).
W porównaniu do trybu prostego użytkownik otrzymuje możliwość zawężenia wyników bezpośrednio w panelu filtrów.

\begin{figure}[H]
    \centering
    \includegraphics[width=1\textwidth]{attachments/prezentacja-systemu/strona-glowna/advance-page-list}
    \caption{Tryb zaawansowany — lista spotów po wyszukaniu.}
    \label{fig:search-advanced-list}
\end{figure}

Dostępny jest filtr oceny w formie listy progów, co pozwala ograniczyć wyniki do \glslink{spot}{spotów} spełniających minimalną
ocenę (rys.~\ref{fig:search-advanced-rating-filter}).

\begin{figure}[H]
    \centering
    \includegraphics[width=1\textwidth]{attachments/prezentacja-systemu/strona-glowna/advance-page-filter}
    \caption{Tryb zaawansowany — rozwinięty filtr minimalnej oceny.}
    \label{fig:search-advanced-rating-filter}
\end{figure}

W trybie zaawansowanym można również zmienić sposób sortowania wyników (po popularności lub po ocenie)
(rys.~\ref{fig:search-advanced-sort}).

\begin{figure}[H]
    \centering
    \includegraphics[width=1\textwidth]{attachments/prezentacja-systemu/strona-glowna/advance-page-sort}
    \caption{Tryb zaawansowany — wybór sposobu sortowania wyników.}
    \label{fig:search-advanced-sort}
\end{figure}

Po zastosowaniu filtrów lista wyników jest odświeżana automatycznie, a prezentowane \glslink{spot}{spoty} odpowiadają aktualnym
kryteriom (rys.~\ref{fig:search-advanced-filtered}).

\begin{figure}[H]
    \centering
    \includegraphics[width=1\textwidth]{attachments/prezentacja-systemu/strona-glowna/advance-page-filtered}
    \caption{Tryb zaawansowany — wyniki po zastosowaniu filtrów.}
    \label{fig:search-advanced-filtered}
\end{figure}

%Analogicznie jak w pozostałych widokach aplikacji, przygotowano wersję w motywie ciemnym (rys.~\ref{fig:search-advanced-dark}).

%\begin{figure}[H]
%    \centering
%    \includegraphics[width=1\textwidth]{attachments/prezentacja-systemu/strona-glowna/advance-page-dark}
%    \caption{Tryb zaawansowany wyszukiwarki w motywie ciemnym.}
%    \label{fig:search-advanced-dark}
%\end{figure}
