%! Author = Stanisław Oziemczuk
%! Date = 02/01/2026


\section{Strona mapy}
\label{sec:strona-mapy}

Moduł mapy umożliwia przeglądanie dostępnych \glslink{spot}{spotów}.
Został podzielony na kilka głównych części:
\begin{itemize}
    \item mapę ze \glslink{spot}{spotami}
    \item panel szczegółów \glslink{spot}{spota}
    \item panel z danymi pogodowymi
    \item galerię multimediów
\end{itemize}

\subsection{Mapa ze spotami}
\label{subsec:mapa-ze-spotami}

W tej części modułu użytkownik może przeglądać dostępne \glslink{spot}{spoty} na mapie wyświetlane w formie wielokątów (rys. \ref{fig:mapa-spoty-wielokaty}).
Po odpowiednim oddaleniu widoku, w zależności od pola powierzchni obszaru, zostanie ono zaprezentowane jako marker rys. \ref{fig:mapa-spoty-markery}.

\begin{figure}[H]
    \centering
    \includegraphics[width=1\textwidth]{attachments/prezentacja-systemu/mapa/mapa_spoty_wielokaty}
    \caption{Spoty na mapie prezentowane w formie wielokątów.}
    \label{fig:mapa-spoty-wielokaty}
\end{figure}

\begin{figure}[H]
    \centering
    \includegraphics[width=1\textwidth]{attachments/prezentacja-systemu/mapa/mapa_spoty_markery}
    \caption{Spoty na mapie prezentowane w formie markerów.}
    \label{fig:mapa-spoty-markery}
\end{figure}

Zaimplementowany został przycisk do sprawdzania aktualnej pozycji użytkownika.
Po jego kliknięciu na mapie wyświetlane jest koło oznaczające jego pozycję rys. \ref{fig:mapa-lokalizacja-uzytkownika}, a widok mapy zostaje
płynnie na nią przybliżony.

\begin{figure}[H]
    \centering
    \includegraphics[width=1\textwidth]{attachments/prezentacja-systemu/mapa/mapa_spoty_markery}
    \caption{Aktualna lokalizacja użytkownika zaznaczona na mapie.}
    \label{fig:mapa-lokalizacja-uzytkownika}
\end{figure}
