%! Author = kacper
%! Date = 04/01/2026

\section{Strona forum}
\label{sec:strona-forum}

Forum jest dostępne od momentu wejścia na stronę również dla użytkowników niezalogowanych, jednak wyłącznie do przeglądania treści.
Wszelkie próby interakcji z nimi wymagają poprawnego zalogowania.
Forum składa się z sześciu podstron:
\begin{itemize}
    \item Home Page
    \item Post Details
    \item All Categories
    \item All Tags
    \item Followed Posts
    \item Guidelines
\end{itemize}

Interfejs forum został zaprojektowany w wersji jasnej oraz ciemnej, przy czym większość zdjęć w tym rozdziale zaprezentowano w trybie jasnym.

\subsection{Home Page}
\label{subsec:home-page}

Strona główna forum (rys. \ref{img:home-page}) składa się z trzech podstawowych obszarów funkcjonalnych:
\begin{itemize}
    \item centralnej listy postów
    \item lewego panelu bocznego
    \item prawego panelu bocznego
\end{itemize}


\subsubsection{Lista postów}

Centralna część strony zawiera listę postów posortowanych według wybranej opcji.
Nad listą umieszczony został przycisk umożliwiający zmianę sposobu sortowania.
Dostępne opcje to: \textit{Newest, Oldest, Most Viewed, Least Viewed, Most Commented, Least Commented}.

Każdy post prezentowany jest w formie kafelka zawierającego:
\begin{itemize}
    \item tytuł posta
    \item kategorię
    \item tagi
    \item skróconą treść posta (bez obrazów)
    \item liczbę komentarzy
    \item liczbę wyświetleń
    \item menu kontekstowe
\end{itemize}

Menu kontekstowe ukryte jest pod ikoną trzech poziomych kropek i umożliwia zgłoszenie posta (rys. \ref{img:report-form}) lub dodanie go do obserwowanych (rys. \ref{img:menu-not-author}).
Próba wykonania tych operacji przez użytkownika niezalogowanego skutkuje wyświetleniem odpowiedniego komunikatu.
W przypadku zalogowanego autora posta dostępne są dodatkowe opcje edycji oraz usunięcia treści (rys. \ref{img:menu-author}).
Kliknięcie kafelka powoduje przejście do widoku szczegółów posta.

\begin{figure}[H]
    \centering
    \includegraphics[width=1\textwidth]{attachments/prezentacja-systemu/forum/home_page}
    \caption{Strona główna w trybie jasnym}
    \label{img:home-page}
\end{figure}

%\begin{figure}[H]
%    \centering
%    \includegraphics[width=1\textwidth]{attachments/prezentacja-systemu/forum/home_page_dark}
%    \caption{Strona główna w trybie ciemnym}
%    \label{img:home-page-dark}
%\end{figure}

\begin{figure}[H]
    \centering
    \includegraphics[width=0.5\textwidth]{attachments/prezentacja-systemu/forum/menu_kontekstowe_not_author}
    \caption{Menu kontekstowe posta dla użytkownika niebędącego autorem posta}
    \label{img:menu-not-author}
\end{figure}

\begin{figure}[H]
    \centering
    \includegraphics[width=0.5\textwidth]{attachments/prezentacja-systemu/forum/menu_kontekstowe_author}
    \caption{Menu kontekstowe posta dla autora posta}
    \label{img:menu-author}
\end{figure}

\begin{figure}[H]
    \centering
    \includegraphics[width=1\textwidth]{attachments/prezentacja-systemu/forum/report_form}
    \caption{Formularz zgłoszenia posta lub komentarza}
    \label{img:report-form}
\end{figure}

\subsubsection{Lewy panel boczny}

Lewy panel boczny (rys. \ref{img:left-panel}) zawiera przycisk umożliwiający otwarcie formularza tworzenia nowego posta w formie \glslink{modal}{modalu}.
Poniżej wyświetlana jest lista dostępnych kategorii oraz tagów.
Przyciski \textit{All Categories} oraz \textit{All Tags} przekierowują do podstron prezentujących alfabetyczne listy tych elementów.

\begin{figure}[H]
    \centering
    \includegraphics[width=0.5\textwidth]{attachments/prezentacja-systemu/forum/left_panel}
    \caption{Lewy panel boczny forum}
    \label{img:left-panel}
\end{figure}

Formularz tworzenia posta (rys. \ref{img:create-post}) składa się z następujących pól:
\begin{itemize}
    \item \textbf{title} -- tytuł posta
    \item \textbf{category} -- kategoria posta
    \item \textbf{tags} -- lista tagów
    \item \textbf{content} -- treść posta tworzona za pomocą edytora \glslink{rich-text-editor}{typu rich text}
\end{itemize}

Na dole formularza znajdują się dwa przyciski:
\begin{itemize}
    \item \textbf{create} -- zapisanie posta
    \item \textbf{cancel} -- anulowanie operacji i zamknięcie formularza
\end{itemize}

Otwarcie formularza przez użytkownika niezalogowanego powoduje wyświetlenie odpowiedniego komunikatu informacyjnego (rys. \ref{img:create-post-warning}).
W formularzu zastosowano walidację danych.
W przypadku braku wymaganych danych pojawia się informacja pod polem nie spełniającym wymagań (rys. \ref{img:create-post-validation}).

\begin{figure}[H]
    \centering
    \includegraphics[width=1\textwidth]{attachments/prezentacja-systemu/forum/create_post_form}
    \caption{Formularz tworzenia nowego posta}
    \label{img:create-post}
\end{figure}

\begin{figure}[H]
    \centering
    \includegraphics[width=1\textwidth]{attachments/prezentacja-systemu/forum/post_validation}
    \caption{Walidacja formularza tworzenia posta}
    \label{img:create-post-validation}
\end{figure}

\begin{figure}[H]
    \centering
    \includegraphics[width=1\textwidth]{attachments/prezentacja-systemu/forum/create_post_warning}
    \caption{Komunikat informujący o konieczności zalogowania w celu utworzenia posta}
    \label{img:create-post-warning}
\end{figure}

\subsubsection{Prawy panel boczny}

Prawy panel boczny (rys. \ref{img:right-panel}) składa się z dwóch elementów.

\paragraph{SearchBar}

Komponent \textit{SearchBar} (rys. \ref{img:search-bar}) umożliwia wyszukiwanie postów na forum.
Po jego rozwinięciu dostępne są dodatkowe filtry, w tym:
\begin{itemize}
    \item kategoria
    \item tagi
    \item zakres dat (from, to)
    \item autor
\end{itemize}

W prawym dolnym rogu rozwiniętego paska wyszukiwania znajdują się dwa przyciski:
\begin{itemize}
    \item \textbf{clear} -- wyczyszczenie ustawionych filtrów,
    \item \textbf{search} -- wykonanie wyszukiwania.
\end{itemize}

Po zakończeniu wyszukiwania strona wyświetla listę wyników wraz z informacją o liczbie znalezionych postów oraz zastosowanych filtrach (rys. \ref{img:search-example}).

\begin{figure}[H]
    \centering
    \includegraphics[width=0.5\textwidth]{attachments/prezentacja-systemu/forum/search_bar}
    \caption{Panel wyszukiwania postów}
    \label{img:search-bar}
\end{figure}

\begin{figure}[H]
    \centering
    \includegraphics[width=1\textwidth]{attachments/prezentacja-systemu/forum/search_example}
    \caption{Przykładowy widok wyników wyszukiwania postów}
    \label{img:search-example}
\end{figure}

\paragraph{Top Posts}

Sekcja \textit{Top Posts} prezentuje listę trzech najpopularniejszych postów z okresu ostatniego miesiąca.
Każdy element zawiera nazwę użytkownika wraz ze zdjęciem profilowym autora oraz datę publikacji.
Kliknięcie tytułu przenosi do widoku szczegółów posta, natomiast kliknięcie nazwy autora prowadzi do jego profilu.

\begin{figure}[H]
    \centering
    \includegraphics[width=0.5\textwidth]{attachments/prezentacja-systemu/forum/right_panel}
    \caption{Prawy panel boczny forum}
    \label{img:right-panel}
\end{figure}

Lewy oraz prawy panel boczny są widoczne na wszystkich podstronach forum, co zapewnia spójną i wygodną nawigację.

\subsection{Post Details}
\label{subsec:post-details}

Widok szczegółów posta prezentuje jego pełną treść wraz z osadzonymi obrazami (rys. \ref{img:detailed-post}).
Nad kafelkiem posta umieszczone są dwa przyciski:
\begin{itemize}
    \item po lewej stronie -- przycisk nawigacyjny umożliwiający powrót do strony głównej forum
    \item po prawej stronie -- przycisk w postaci ikony dzwonka służący do dodania posta do obserwowanych
\end{itemize}

Górna część kafelka zawiera zdjęcie profilowe autora oraz jego nazwę użytkownika, a po przeciwnej stronie datę utworzenia posta.
Poniżej wyświetlane są kategoria i tagi, następnie tytuł oraz pełna treść wpisu.

W dolnej części znajdują się przyciski interakcji, obejmujące:
\begin{itemize}
    \item przycisk \textit{upvote} z liczbą polubień
    \item przycisk \textit{downvote} z liczbą negatywnych ocen
    \item liczbę komentarzy
    \item przycisk \textit{share} umożliwiający skopiowanie adresu \glslink{url}{URL} posta
    \item menu kontekstowe służące do zarządzania postem, dodania go do obserwowanych lub zgłoszenia
    \item przycisk dodania nowego komentarza
\end{itemize}

Przyciski \textit{upvote} oraz \textit{downvote} zmieniają kolor w przypadku oddania głosu przez zalogowanego użytkownika (rys. \ref{img:upvoted-post}).
Naciśnięcie przycisku \textit{Add comment} spowoduje otwarcie formularza dodania komentarza (rys. \ref{img:add-comment}), natomiast w przypadku użytkownika niezalogowanego wyświetlany jest odpowiedni komunikat informujący o konieczności zalogowania.
W formularzu zastosowano walidację danych.
W przypadku braku wymaganych danych pojawia się informacja pod polem nie spełniającym wymagań (rys. \ref{img:comment-validation}).

\begin{figure}[H]
    \centering
    \includegraphics[width=1\textwidth]{attachments/prezentacja-systemu/forum/detailed_post}
    \caption{Widok szczegółów posta}
    \label{img:detailed-post}
\end{figure}

\begin{figure}[H]
    \centering
    \includegraphics[width=1\textwidth]{attachments/prezentacja-systemu/forum/upvoted_post}
    \caption{Polubiony post}
    \label{img:upvoted-post}
\end{figure}

\begin{figure}[H]
    \centering
    \includegraphics[width=1\textwidth]{attachments/prezentacja-systemu/forum/add_comment}
    \caption{Formularz dodawania komentarza}
    \label{img:add-comment}
\end{figure}

\begin{figure}[H]
    \centering
    \includegraphics[width=1\textwidth]{attachments/prezentacja-systemu/forum/comment_validation}
    \caption{Walidacja formularza dodawania komentarza}
    \label{img:comment-validation}
\end{figure}

Pod kafelkiem posta wyświetlana jest lista komentarzy (rys. \ref{img:comments-replies-closed}).
Elementy mają formę wizualnie zbliżoną do posta, z wyłączeniem funkcji udostępniania.
Każdy z nich zawiera przycisk \textit{reply} umożliwiający dodanie odpowiedzi oraz menu kontekstowe pozwalające autorowi na edycję lub usunięcie treści.
Usunięty komentarz pozostaje widoczny w formie komunikatu „Comment was deleted by the user.”, a wszelka interakcja z nim jest zablokowana (rys. \ref{img:deleted-comment}).
W przypadku elementów posiadających odpowiedzi wyświetlana jest ikona strzałki wraz z liczbą odpowiedzi, umożliwiająca rozwinięcie ich listy (rys. \ref{img:comments-replies-open}).

Próba dodania komentarza lub odpowiedzi przez użytkownika niezalogowanego skutkuje wyświetleniem odpowiedniego komunikatu.

\begin{figure}[H]
    \centering
    \includegraphics[width=1\textwidth]{attachments/prezentacja-systemu/forum/comments_replies_closed}
    \caption{Widok listy komentarzy pod postem}
    \label{img:comments-replies-closed}
\end{figure}

\begin{figure}[H]
    \centering
    \includegraphics[width=1\textwidth]{attachments/prezentacja-systemu/forum/comments_replies_open}
    \caption{Widok listy komentarzy z rozwiniętymi odpowiedziami}
    \label{img:comments-replies-open}
\end{figure}

\begin{figure}[H]
    \centering
    \includegraphics[width=1\textwidth]{attachments/prezentacja-systemu/forum/deleted_comment}
    \caption{Widok usuniętego komentarza}
    \label{img:deleted-comment}
\end{figure}

\subsection{All Categories}

Podstrona \textit{All Categories} (rys. \ref{img:categories}) prezentuje alfabetyczną listę wszystkich kategorii postów.
Wybranie konkretnej kategorii powoduje przejście do widoku wyników wyszukiwania w obrębie strony głównej forum, zawierającego posty przypisane do danej kategorii.

\begin{figure}[H]
    \centering
    \includegraphics[width=1\textwidth]{attachments/prezentacja-systemu/forum/all_categories}
    \caption{Widok listy kategorii}
    \label{img:categories}
\end{figure}

\subsection{All Tags}

Podstrona \textit{All Tags} (rys. \ref{img:tags}) wyświetla alfabetyczną listę wszystkich tagów.
Kliknięcie wybranego tagu przekierowuje do widoku wyników wyszukiwania na stronie głównej forum, zawierającego posty oznaczone tym tagiem.

\begin{figure}[H]
    \centering
    \includegraphics[width=1\textwidth]{attachments/prezentacja-systemu/forum/all_tags}
    \caption{Widok listy tagów}
    \label{img:tags}
\end{figure}

\subsection{Followed Posts}

Podstrona \textit{Followed Posts} (rys. \ref{img:followed-posts}) dostępna jest wyłącznie dla zalogowanych użytkowników.
Zawiera listę obserwowanych postów z możliwością sortowania analogiczną do strony głównej forum.
Dostęp do podstrony realizowany jest poprzez wybór odpowiedniej sekcji w \glslink{sidebar}{sidebarze} po poprawnym zalogowaniu.

\begin{figure}[H]
    \centering
    \includegraphics[width=1\textwidth]{attachments/prezentacja-systemu/forum/followed_posts}
    \caption{Lista obserwowanych postów}
    \label{img:followed-posts}
\end{figure}

\subsection{Guidelines}

Podstrona \textit{Guidelines} (rys. \ref{img:guidelines}) zawiera regulamin korzystania z forum i jest stale dostępna dla wszystkich użytkowników.
Przejście do niej odbywa się poprzez wybór odpowiedniej sekcji w \glslink{sidebar}{sidebarze}.

\begin{figure}[H]
    \centering
    \includegraphics[width=1\textwidth]{attachments/prezentacja-systemu/forum/guidelines}
    \caption{Regulamin korzystania z forum}
    \label{img:guidelines}
\end{figure}