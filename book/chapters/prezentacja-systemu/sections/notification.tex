%! Author = Mateusz
%! Date = 01/01/2026

\section{Powiadomienia (Notification)}
\label{sec:notification}

W aplikacji zastosowano komponent powiadomień (\textit{Notification}), którego zadaniem
jest prezentowanie krótkich komunikatów systemowych.
Element ten informuje o błędach walidacji i błędach serwera, poprawnym wykonaniu
akcji (zalogowaniu) oraz o konieczności zalogowania się w celu uzyskania dostępu do
wybranych funkcji.

Komponent występuje w trzech wariantach:
\begin{itemize}
    \item \textbf{error} -- komunikat o błędzie,
    \item \textbf{success} -- komunikat o pomyślnym wykonaniu operacji,
    \item \textbf{info} -- komunikat informacyjny (o wymaganym zalogowaniu).
\end{itemize}

Przykładowe warianty powiadomień przedstawiono na rys. \ref{fig:notif-error}--\ref{fig:notif-info}.

\begin{figure}[H]
    \centering
    \includegraphics[width=1\textwidth]{attachments/prezentacja-systemu/nottification/error}
    \caption{Powiadomienie w trybie \textit{error} -- informacja o błędzie.}
    \label{fig:notif-error}
\end{figure}

\begin{figure}[H]
    \centering
    \includegraphics[width=1\textwidth]{attachments/prezentacja-systemu/nottification/success}
    \caption{Powiadomienie w trybie \textit{success} -- potwierdzenie poprawnego wykonania operacji.}
    \label{fig:notif-success}
\end{figure}

\begin{figure}[H]
    \centering
    \includegraphics[width=1\textwidth]{attachments/prezentacja-systemu/nottification/info}
    \caption{Powiadomienie w trybie \textit{info} -- komunikat informacyjny.}
    \label{fig:notif-info}
\end{figure}
