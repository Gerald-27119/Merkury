%! Author = Mateusz
%! Date = 01/01/2026

\section{Sidebar}
\label{sec:sidebar}

\glslink{sidebar}{Sidebar} jest głównym elementem nawigacyjnym aplikacji.
Umożliwia przechodzenie pomiędzy kluczowymi modułami,
a także szybkie wykonanie akcji takich jak logowanie/wylogowanie oraz zmiana motywu.

Dostępne są dwa warianty widoku: zwinięty (ikony) oraz rozwinięty (ikony wraz z etykietami).
W wariancie zwiniętym wyświetlane są jedynie ikony sekcji (rys. \ref{fig:sidebar-collapsed}).
Po najechaniu kursorem na element w tym trybie prezentowana jest podpowiedź (tooltip), co ułatwia identyfikację akcji
(rys. \ref{fig:sidebar-collapsed-hover}).

\begin{figure}[H]
    \centering
    \includegraphics[height=0.9\textheight]{attachments/prezentacja-systemu/sidebar/collapsed}
    \caption{Sidebar w trybie zwiniętym (użytkownik niezalogowany).}
    \label{fig:sidebar-collapsed}
\end{figure}

\begin{figure}[H]
    \centering
    \includegraphics[height=0.9\textheight]{attachments/prezentacja-systemu/sidebar/collapsed-hover}
    \caption{Podpowiedź (tooltip) w trybie zwiniętym.}
    \label{fig:sidebar-collapsed-hover}
\end{figure}

Po zalogowaniu w sidebarze pojawiają się dodatkowe pozycje nawigacyjne związane z funkcjami dostępnymi wyłącznie
dla zalogowanego użytkownika (chat, account oraz zapisane posty na forum), a akcja logowania zastępowana jest wylogowaniem
(rys. \ref{fig:sidebar-collapsed-logged-in}).

\begin{figure}[H]
    \centering
    \includegraphics[height=0.9\textheight]{attachments/prezentacja-systemu/sidebar/collapsed-logged-in}
    \caption{Sidebar w trybie zwiniętym po zalogowaniu.}
    \label{fig:sidebar-collapsed-logged-in}
\end{figure}

W trybie rozwiniętym oprócz ikon wyświetlane są także nazwy pozycji menu (rys. \ref{fig:sidebar-extended}),
a aktualnie wybrana podstrona jest wyróżniona.
Wybrane sekcje da się rozwinąć, aby pokazać podpozycje, takie jak forum (rys. \ref{fig:sidebar-extended-forum-open})
oraz account po zalogowaniu (rys. \ref{fig:sidebar-extended-logged-in-open}).

\begin{figure}[H]
    \centering
    \includegraphics[height=0.9\textheight]{attachments/prezentacja-systemu/sidebar/extended}
    \caption{Sidebar w trybie rozwiniętym (użytkownik niezalogowany).}
    \label{fig:sidebar-extended}
\end{figure}

\begin{figure}[H]
    \centering
    \includegraphics[height=0.9\textheight]{attachments/prezentacja-systemu/sidebar/extended-forum-open}
    \caption{Rozwinięta sekcja \textit{Forum} w sidebarze.}
    \label{fig:sidebar-extended-forum-open}
\end{figure}

Po zalogowaniu w trybie rozwiniętym widoczne są dodatkowe sekcje (np. \textit{Chat} oraz \textit{Account})
oraz akcja \textit{Sign Out} (rys. \ref{fig:sidebar-extended-logged-in}).
Dodatkowo możliwe jest rozwinięcie sekcji forum i konta w celu przejścia do szczegółowych podstron użytkownika
(rys. \ref{fig:sidebar-extended-logged-in-open}).

\begin{figure}[H]
    \centering
    \includegraphics[height=0.9\textheight]{attachments/prezentacja-systemu/sidebar/extended-logged-in}
    \caption{Sidebar w trybie rozwiniętym po zalogowaniu.}
    \label{fig:sidebar-extended-logged-in}
\end{figure}

\begin{figure}[H]
    \centering
    \includegraphics[height=0.9\textheight]{attachments/prezentacja-systemu/sidebar/extended-logged-in-open}
    \caption{Rozwinięte sekcje \textit{Forum} oraz \textit{Account} po zalogowaniu.}
    \label{fig:sidebar-extended-logged-in-open}
\end{figure}