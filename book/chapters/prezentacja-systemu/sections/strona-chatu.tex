%! Author = mateusz
%! Date = 20/10/2025

\section{Strona chatu}
\label{sec:strona-chatu}

%TODO: Jak będziecie opisywać wygląd aplikacji to usuńcie na frontendzie notification z sidebara

\subsection{Widok bazowy modułu czatu}
\label{subsec:widok-bazowy-czatu}

EKRAN 1

Rysunek~X przedstawia główny widok modułu czatu. Interfejs zaprojektowano
z wyraźnym podziałem na dwie strefy: listę rozmów oraz okno aktywnej konwersacji.

Bezpośrednio obok paska nawigacyjnego umieszczono panel prezentujący listę
dostępnych konwersacji. Każdy element listy zawiera miniaturę (awatar), nazwę rozmowy,
fragment ostatniej wiadomości oraz jej datę wysłania.
Aktualnie wybrana rozmowa jest wyróżniona wizualnie.

Centralną część ekranu zajmuje okno aktywnej rozmowy. W górnym pasku widoczna jest nazwa
konwersacji. W przypadku czatu prywatnego nazwę konwersacji stanowi nazwa użytkownika
z którym obecnie prowadzona jest rozmowa. W przypadku czatu grupowego nazwę czatu stanowi
lista nazw użytkownika członków czatu. W przypadku czatu grupowego nazwa czatu jest edytowalna. Tak samo awatar czatu.
W przypadku rozmowy prywatnej nazwa czatu po jej nacisnieciu przenosi uzytkownika do panelu uzytkownika,
do modulu social do konta osoby z ktora jest ta konwersacja.
oraz zestaw akcji kontekstowych związanych z rozmową. W przypadku czatu prywatnego wyswietla sie przycisk
umozwloiajacy stwroenie czatu grupowego z aktualnym uczestnikiem czatu prwyatnego.
Poniżej znajduje się przewijalna historia wiadomości ułożona
chronologicznie. Każda wiadomość prezentuje awatar nadawcy, nazwę użytkownika, znacznik czasu
oraz treść.

W dolnej części widoku umieszczono pasek wprowadzania wiadomości z polem tekstowym
oraz przyciskami akcji (dodawanie załącznika do wiadomości, wysylanie gifów oraz emoji) i przyciskiem wysyłania.
Taki układ wspiera typowy scenariusz użycia: wybór rozmowy z listy po lewej stronie,
przegląd historii w części centralnej oraz tworzenie i wysyłanie kolejnych wiadomości.

EKRAN 2

Na ekranie pozakane jest okno edycji czatu grupowego. WYswietla sie ono po nacisnieciu na kafelek
z awatarem i nazwa czatu grupwoego. Jak widac na ekranie mozna zmienic wawatar i anzw czatu.

