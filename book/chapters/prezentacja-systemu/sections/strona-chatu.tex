%! Author = Adam
%! Date = 01/01/2025

\section{Strona czatu}
\label{sec:strona-chatu}

\subsection{Widok modułu czatu (rozmowa prywatna)}
\label{subsec:chat-screen-1-private}

\begin{figure}[H]
    \centering
    \includegraphics[width=\textwidth]{./attachments/prezentacja-systemu/czat/prywatny_1}
    \caption{Widok modułu czatu z otwartą rozmową prywatną.}
    \label{fig:chat:screen-1-main-view}
\end{figure}

Rysunek~\ref{fig:chat:screen-1-main-view} przedstawia główny widok modułu czatu w scenariuszu rozmowy prywatnej.
Interfejs został podzielony na dwa obszary: listę dostępnych konwersacji po lewej stronie oraz okno aktywnej rozmowy
w części centralnej.

Panel listy rozmów zawiera nagłówek \textit{Chats} oraz zestaw kafelków reprezentujących konwersacje.
Każdy kafelek prezentuje awatar rozmowy, jej nazwę, podgląd ostatniej wiadomości
oraz jej znacznik czasu. Aktualnie otwarta konwersacja jest wyróżniona wizualnie poprzez podświetlenie tła.

W górnej części obszaru rozmowy znajduje się pasek nagłówka z awatarem oraz nazwą rozmówcy.
Kliknięcie w kafelek z nazwą użytkownika i jego awatarem przenosi do profilu tej osoby.
W prawym górnym rogu widoczna jest ikona służąca do inicjowania utworzenia czatu grupowego.

Poniżej nagłówka prezentowana jest historia wiadomości w układzie chronologicznym. Każda wiadomość zawiera awatar nadawcy,
nazwę użytkownika, znacznik czasu oraz treść wiadomości. Dodatkowo, wiadomości są grupowane według daty.

W dolnej części widoku umieszczono pasek tworzenia wiadomości z polem tekstowym oraz zestawem przycisków akcji.
Po lewej stronie dostępny jest przycisk dodawania załączników do wiadomości, natomiast po prawej stronie
znajdują się skróty do wstawianie  GIFów oraz emoji. Na prawo od nich jest przycisk wysyłania wiadomości.
W celu wysłania wiadomości oprócz przycisku można użyć klawisza Enter.

\subsection{Wysyłanie GIF-ów}
\label{subsec:panel-wyszukiwania-i-wysyania-gif-ow}}

\begin{figure}[H]
    \centering
    \includegraphics[width=\textwidth]{./attachments/prezentacja-systemu/czat/gif/1}
    \caption{Panel wyboru GIF-ów -- widok początkowy po otwarciu.}
    \label{fig:chat:gif-panel-1}
\end{figure}

Rysunek~\ref{fig:chat:gif-panel-1} przedstawia panel wyboru GIF-ów, uruchamiany z poziomu paska tworzenia wiadomości.
Panel wyświetlany jest jako nakładka w obrębie widoku konwersacji.
W górnej części panelu dostępne są zakładki \textit{Gif} oraz \textit{Emoji}, umożliwiające przełączenie trybu wyboru treści.
Poniżej umieszczono pole wyszukiwania, niżej widać predefiniowane kategorie.

\begin{figure}[H]
    \centering
    \includegraphics[width=\textwidth]{./attachments/prezentacja-systemu/czat/gif/2}
    \caption{Panel wyboru GIF-ów -- wyszukiwanie po kategorii \textit{omg}.}
    \label{fig:chat:gif-panel-2}
\end{figure}

Na rysunku~\ref{fig:chat:gif-panel-2} pokazano działanie mechanizmu wyszukiwania po kategorii. Po wyborze kategorii
lista wyników jest zawężana do pasujących elementów.

\begin{figure}[H]
    \centering
    \includegraphics[width=\textwidth]{./attachments/prezentacja-systemu/czat/gif/3}
    \caption{Panel wyboru GIF-ów -- wyszukiwanie po frazie \textit{thank you}.}
    \label{fig:chat:gif-panel-3}
\end{figure}

Rysunek~\ref{fig:chat:gif-panel-3} przedstawia analogiczny scenariusz dla ręcznie wprowadzonej frazy.
Użytkownik może wielokrotnie modyfikować frazę wyszukiwania, przeglądać zwrócone propozycje oraz wybrać konkretną miniaturę.
Wybranie GIF-a powoduje natychmiastowe jego wysłanie.

\subsection{Wysyłanie emoji}
\label{subsec:chat-emoji}

\begin{figure}[H]
\centering
\includegraphics[width=\textwidth]{./attachments/prezentacja-systemu/czat/emoji/1}
\caption{Panel wyboru emoji oraz przykład wysłania wiadomości z emoji.}
\label{fig:chat:emoji-panel}
\end{figure}

Rysunek~\ref{fig:chat:emoji-panel} przedstawia mechanizm wyboru emoji dostępny z poziomu paska tworzenia wiadomości.
Po użyciu skrótu \textit{Emoji} w dolnej części widoku konwersacji otwierany jest panel w formie nakładki,
wyświetlany w obrębie okna rozmowy (bez opuszczania aktualnego czatu).

W górnej części panelu znajdują się zakładki \textit{Gif} oraz \textit{Emoji}, umożliwiające przełączenie trybu wyboru treści.
Poniżej umieszczono pole wyszukiwania, które pozwala filtrować dostępne emoji po wpisanej frazie (na rysunku pokazano wyszukiwanie dla hasła \textit{cat}).
Wyniki prezentowane są w sekcji \textit{Search results} w postaci listy ikon, które użytkownik może wybierać pojedynczo.

Wybranie emoji powoduje wstawienie go do pola edycji wiadomości w dolnym pasku, dzięki czemu użytkownik może skomponować treść
złożoną wyłącznie z emotikon lub użyć ich jako uzupełnienia tekstu. Po wysłaniu wiadomości (przyciskiem wysyłania lub klawiszem Enter)
treść pojawia się w historii rozmowy, zgodnie z układem chronologicznym.


\subsection{Dodawanie załączników do wiadomości}
\label{subsec:chat-attachments}

\begin{figure}[H]
\centering
\includegraphics[width=\textwidth]{./attachments/prezentacja-systemu/czat/att/1}
\caption{Systemowe okno wyboru pliku uruchamiane z poziomu czatu.}
\label{fig:chat:attachments-1}
\end{figure}

Rysunek~\ref{fig:chat:attachments-1} przedstawia pierwszy etap dodawania załącznika.
Po użyciu przycisku dodawania załączników w dolnym pasku tworzenia wiadomości aplikacja
otwiera systemowe okno wyboru pliku. Dzięki wykorzystaniu natywnego selektora użytkownik
może wskazać zasób z lokalnego dysku.

\begin{figure}[H]
\centering
\includegraphics[width=\textwidth]{./attachments/prezentacja-systemu/czat/att/2}
\caption{Załącznik dodany do wiadomości -- podgląd w polu edycji przed wysłaniem.}
\label{fig:chat:attachments-2}
\end{figure}

Na rysunku~\ref{fig:chat:attachments-2} pokazano widok czatu po wskazaniu pliku.
Wybrany załącznik pojawia się nad polem wprowadzania treści w formie kafelka zawierającego
ikonę typu pliku oraz jego miniaturę (o ile jest dostępna). Taki podgląd umożliwia szybkie
zweryfikowanie, jaki plik zostanie wysłany, bez opuszczania bieżącej konwersacji.
Użytkownik może w tym samym miejscu dopisać treść wiadomości, a następnie wysłać ją
standardowo (przyciskiem wysyłania lub klawiszem Enter).

\begin{figure}[H]
\centering
\includegraphics[width=\textwidth]{./attachments/prezentacja-systemu/czat/att/3}
\caption{Wiadomość z wysłanym załącznikiem w historii konwersacji.}
\label{fig:chat:attachments-3}
\end{figure}

Rysunek~\ref{fig:chat:attachments-3} przedstawia sposób prezentacji załącznika po wysłaniu wiadomości.
W historii rozmowy wyświetlana jest karta pliku zawierająca jego nazwę oraz rozmiar, co ułatwia
identyfikację przesłanego zasobu. Po prawej stronie karty znajduje się ikona pobierania,
umożliwiająca zapisanie pliku na urządzeniu odbiorcy. Dodatkowo (jeżeli jest to możliwe dla danego formatu)
poniżej karty prezentowana jest miniatura podglądu, dzięki czemu użytkownik może rozpoznać zawartość
bez konieczności pobierania pliku.

\begin{figure}[H]
\centering
\includegraphics[width=\textwidth]{./attachments/prezentacja-systemu/czat/att/4}
\caption{Interakcja z załącznikiem -- podgląd miniatury oraz możliwość pobrania pliku.}
\label{fig:chat:attachments-4}
\end{figure}

Na rysunku~\ref{fig:chat:attachments-4} pokazano interakcję z wiadomością zawierającą załącznik.
Załącznik pozostaje dostępny w obrębie konwersacji, a użytkownik może skorzystać z miniatury podglądu
lub bezpośrednio użyć przycisku pobierania na karcie pliku. Takie rozwiązanie wspiera typowy scenariusz
wymiany materiałów w rozmowie (np. dokumentów i zrzutów ekranu), zapewniając szybki dostęp do plików
również po upływie czasu od ich wysłania.

\subsection{Tworzenie czatu grupowego z poziomu rozmowy prywatnej}
\label{subsec:chat-group-create-from-private}

\begin{figure}[H]
\centering
\includegraphics[width=\textwidth]{./attachments/prezentacja-systemu/czat/cg/1}
\caption{Okno wyboru uczestników podczas tworzenia czatu grupowego z rozmowy prywatnej.}
\label{fig:chat:group-create-modal}
\end{figure}

Rysunek~\ref{fig:chat:group-create-modal} przedstawia okno dialogowe służące do utworzenia czatu grupowego bezpośrednio
z poziomu rozmowy prywatnej. Funkcja uruchamiana jest poprzez przycisk w prawym górnym rogu paska nagłówka czatu
(przy aktywnej konwersacji prywatnej). Po jego użyciu interfejs wyświetla nakładkę na bieżący widok rozmowy,
a w tle pozostaje przyciemniony ekran czatu.

W górnej części okna dostępne jest pole wyszukiwania, umożliwiające filtrowanie listy znajomych po nazwie.
Wybrani użytkownicy prezentowani są w postaci etykiet (tzw. \textit{chipów}) umieszczonych pod polem wyszukiwania,
z możliwością usunięcia danej osoby z listy. Istotnym elementem jest domyślne dodanie aktualnego rozmówcy
z czatu prywatnego do nowo tworzonej rozmowy grupowej --- dzięki temu użytkownik nie musi ponownie wskazywać osoby,
z którą już prowadzi konwersację. W interfejsie widoczna jest także informacja o limicie liczby uczestników
(np. maksymalnie 6 osób), co zapobiega tworzeniu nadmiernie rozbudowanych grup.

Lista wyników wyszukiwania prezentowana jest w formie kafelków zawierających awatar oraz nazwę użytkownika.
Dodanie osoby odbywa się poprzez użycie przycisku z ikoną \textit{+}, natomiast dla już wybranych uczestników
przycisk zmienia stan na nieaktywny (np. \textit{Added}), co jednoznacznie sygnalizuje przynależność do tworzonej grupy.
Po skompletowaniu składu użytkownik zatwierdza operację przyciskiem \textit{Create Chat}.

\begin{figure}[H]
\centering
\includegraphics[width=\textwidth]{./attachments/prezentacja-systemu/czat/cg/2}
\caption{Nowo utworzony czat grupowy po zatwierdzeniu listy uczestników.}
\label{fig:chat:group-created-view}
\end{figure}

Rysunek~\ref{fig:chat:group-created-view} pokazuje widok aplikacji po utworzeniu czatu grupowego.
Nowa konwersacja pojawia się na liście rozmów po lewej stronie, a użytkownik zostaje automatycznie przeniesiony
do okna aktywnego czatu grupowego. W nagłówku rozmowy prezentowana jest informacja o uczestnikach (np. w postaci listy nazw),
co pozwala szybko rozpoznać, z kim prowadzona jest konwersacja.

W początkowym stanie historia wiadomości jest pusta (widoczna jest informacja o początku rozmowy), a użytkownik może od razu
rozpocząć wymianę wiadomości, korzystając z tego samego paska tworzenia wiadomości co w czacie prywatnym.
Dzięki temu mechanizm tworzenia grupy stanowi naturalne rozszerzenie rozmowy 1--1 o dodatkowych uczestników,
bez konieczności opuszczania modułu czatu.

\subsection{Edycja czatu grupowego}
\label{subsec:chat-group-edit}

\begin{figure}[H]
\centering
\includegraphics[width=\textwidth]{./attachments/prezentacja-systemu/czat/edit/1}
\caption{Widok czatu grupowego z przyciskiem uruchamiającym edycję.}
\label{fig:chat:group-edit-1}
\end{figure}

Rysunek~\ref{fig:chat:group-edit-1} przedstawia czat grupowy w widoku głównym, z widocznym paskiem nagłówka konwersacji.
W nagłówku prezentowana jest nazwa czatu (lub lista uczestników w przypadku braku własnej nazwy), a po prawej stronie
znajduje się przycisk z ikoną ołówka, służący do przejścia do edycji ustawień rozmowy. Użycie tej akcji nie wymaga
opuszczania aktualnej konwersacji --- w kolejnym kroku wyświetlane jest okno dialogowe w formie nakładki.

\begin{figure}[H]
\centering
\includegraphics[width=\textwidth]{./attachments/prezentacja-systemu/czat/edit/2}
\caption{Okno edycji czatu grupowego: zmiana nazwy oraz awatara.}
\label{fig:chat:group-edit-2}
\end{figure}

Na rysunku~\ref{fig:chat:group-edit-2} pokazano okno \textit{Edit Chat}, które umożliwia zmianę podstawowych danych czatu grupowego.
W górnej części znajduje się podgląd aktualnego awatara, a kliknięcie obszaru \textit{Change chat image} pozwala wskazać nową grafikę
(standardowo poprzez systemowy selektor plików). Poniżej umieszczono pole tekstowe przeznaczone do edycji nazwy czatu.
Użytkownik może zatwierdzić zmiany przyciskiem \textit{Save} lub przerwać operację przyciskiem \textit{Cancel} (bądź ikoną zamknięcia),
co pozwala szybko wycofać się bez modyfikowania istniejących ustawień.

\begin{figure}[H]
\centering
\includegraphics[width=\textwidth]{./attachments/prezentacja-systemu/czat/edit/3}
\caption{Widok czatu grupowego po zapisaniu zmian nazwy oraz awatara.}
\label{fig:chat:group-edit-3}
\end{figure}

Rysunek~\ref{fig:chat:group-edit-3} przedstawia efekt zapisania zmian.
Zaktualizowana nazwa oraz nowy awatar są prezentowane w interfejsie czatu, a także na liście konwersacji po lewej stronie,
co ułatwia identyfikację rozmowy wśród pozostałych czatów. Dzięki temu edycja stanowi prosty mechanizm personalizacji
czatu grupowego i porządkowania listy rozmów w przypadku większej liczby konwersacji.
