%! Author = Adam
%! Date = 01/01/2025

\section{Strona chatu}
\label{sec:strona-chatu}

%TODO: Jak będziecie opisywać wygląd aplikacji to usuńcie na frontendzie notification z sidebara

\subsection{Widok bazowy modułu czatu}
\label{subsec:widok-bazowy-czatu}

\paragraph{Ekran 1. Widok główny rozmów}
Rysunek~X przedstawia główny widok modułu czatu. Interfejs zaprojektowano
z wyraźnym podziałem na dwie strefy: panel listy rozmów po lewej stronie oraz
obszar aktywnej konwersacji w części centralnej.

Bezpośrednio na prawo od paska nawigacyjnego znajduje się panel prezentujący listę
dostępnych konwersacji. Każdy element listy zawiera awatar, nazwę rozmowy,
fragment ostatniej wiadomości oraz informację o czasie wysłania.
Aktualnie wybrana rozmowa jest wyróżniona wizualnie, co ułatwia orientację w module
i jednoznacznie wskazuje konwersację, której treść jest prezentowana w części centralnej.

Centralną część ekranu zajmuje okno aktywnej rozmowy. W górnym pasku widoczna jest nazwa
konwersacji oraz zestaw akcji kontekstowych związanych z bieżącym czatem.
W przypadku rozmowy prywatnej nazwę konwersacji stanowi nazwa użytkownika,
z którym prowadzona jest wymiana wiadomości. Kliknięcie nazwy przenosi użytkownika
do profilu tej osoby w module społecznościowym.
W przypadku czatu grupowego nazwa rozmowy oraz jej awatar mogą zostać zmodyfikowane,
a wejście w tryb edycji jest dostępne z poziomu nagłówka konwersacji.
Dodatkowo, dla czatu prywatnego udostępniono akcję umożliwiającą utworzenie czatu grupowego
z udziałem aktualnego rozmówcy.

Poniżej nagłówka znajduje się przewijalna historia wiadomości ułożona chronologicznie.
Każda wiadomość prezentuje awatar nadawcy, nazwę użytkownika, znacznik czasu oraz treść,
co pozwala jednoznacznie identyfikować autora wypowiedzi i kontekst czasowy rozmowy.

W dolnej części widoku umieszczono pasek wprowadzania wiadomości, składający się z pola tekstowego
oraz przycisków akcji (dodawanie załączników, wysyłanie GIF-ów oraz emoji) i przycisku wysyłania.
Taki układ wspiera typowy scenariusz użycia modułu: wybór rozmowy z listy po lewej stronie,
przegląd historii w części centralnej oraz tworzenie i wysyłanie kolejnych wiadomości.

\paragraph{Ekran 2. Edycja czatu grupowego}
Rysunek~Y przedstawia okno edycji czatu grupowego, wyświetlane po kliknięciu w obszar nagłówka
konwersacji (kafelek z awatarem i nazwą czatu). W ramach tego widoku użytkownik może zmienić
awatar oraz nazwę czatu grupowego. Rozwiązanie to umożliwia dostosowanie identyfikacji rozmowy
do potrzeb jej uczestników, bez konieczności opuszczania modułu czatu.

\paragraph{Ekran 3. Dodawanie załączników do wiadomości}
Rysunek~Z przedstawia proces dołączania plików do wiadomości. Po użyciu przycisku
dodawania załączników w pasku wprowadzania treści wyświetlane jest systemowe okno
wyboru pliku, pozwalające wskazać zasób z lokalnego dysku. Po zatwierdzeniu wyboru
plik zostaje dołączony do aktualnie tworzonej wiadomości i może zostać wysłany do pozostałych
uczestników rozmowy jako załącznik.

\paragraph{Ekran 4. Panel wyszukiwania i wysyłania GIF-ów}
Rysunek~W prezentuje panel wyboru GIF-ów dostępny z poziomu paska edycji wiadomości.
Panel otwierany jest w obrębie widoku konwersacji i nie wymaga opuszczania czatu.
Użytkownik może przeszukiwać zasoby poprzez pole wyszukiwania, a wyniki prezentowane są
w formie siatki miniatur. Wybranie elementu powoduje dodanie GIF-a do wiadomości i umożliwia
jego wysłanie w ramach rozmowy.

\paragraph{Ekran 5. Panel wyboru emoji}
Rysunek~V przedstawia panel wyboru emoji, uruchamiany analogicznie jak panel GIF-ów.
Emoji wybierane są z poziomu \textit{picker-a}, który udostępnia kategorie predefiniowane
(np.\ według typu emotikon) oraz wyszukiwanie tekstowe po frazie. Zaznaczenie emoji wstawia je
do pola tekstowego wiadomości, dzięki czemu może ono zostać wykorzystane samodzielnie lub jako
uzupełnienie treści.

\paragraph{Ekran 6. Tworzenie czatu grupowego z rozmowy prywatnej}
Rysunek~U przedstawia okno dialogowe tworzenia czatu grupowego uruchamiane z poziomu rozmowy prywatnej
(przycisk akcji w nagłówku konwersacji). W oknie dostępne jest pole wyszukiwania użytkowników oraz
lista wybranych osób prezentowana w formie etykiet. Domyślnie uwzględniony jest aktualny rozmówca
oraz użytkownik tworzący czat; możliwe jest wskazanie dodatkowych uczestników (do zdefiniowanego limitu).
Jeżeli nie zostaną dodane kolejne osoby, tworzony jest dwuosobowy czat grupowy, który zachowuje cechy rozmowy
grupowej (m.in.\ możliwość późniejszej rozbudowy składu oraz edycji danych czatu).

\paragraph{Ekran 7. Widok czatu grupowego}
Rysunek~T przedstawia widok aktywnej konwersacji grupowej po jej utworzeniu. Interfejs zachowuje układ znany
z rozmów prywatnych (lista czatów po lewej stronie oraz historia wiadomości w części centralnej), natomiast w nagłówku
prezentowana jest nazwa czatu grupowego (lub skrócona lista uczestników, zależnie od konfiguracji).
W prawym górnym rogu dostępne są dwa przyciski: pierwszy otwiera listę uczestników czatu, natomiast drugi
umożliwia dodawanie kolejnych osób do grupy.

\paragraph{Ekran 8. Lista uczestników czatu grupowego}
Rysunek~S przedstawia panel boczny z listą uczestników, wyświetlany po kliknięciu przycisku podglądu członków grupy.
Panel pojawia się po prawej stronie widoku konwersacji i zawiera nagłówek z liczbą uczestników oraz listę profili
(wraz z awatarami i nazwami użytkowników). Rozwiązanie to umożliwia szybkie zweryfikowanie składu czatu bez przerywania
ciągłości rozmowy.

\paragraph{Ekran 9. Dodawanie nowych uczestników do czatu grupowego}
Rysunek~R przedstawia okno dialogowe dodawania użytkowników do istniejącej konwersacji grupowej.
W oknie dostępne jest pole wyszukiwania oraz mechanizm wyboru użytkowników wraz z licznikiem określającym
maksymalną liczbę osób możliwych do dodania w ramach jednej operacji. Po zatwierdzeniu akcją kontekstową
nowi członkowie zostają dopisani do czatu i są widoczni na liście uczestników.
