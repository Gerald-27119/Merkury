%! Author = Adam
%! Date = 01/01/2025

\section{Strona czatu}
\label{sec:strona-chatu}

\subsection{Widok modułu czatu (rozmowa prywatna)}
\label{subsec:chat-screen-1-private}

\begin{figure}[H]
    \centering
    \includegraphics[width=\textwidth]{./attachments/prezentacja-systemu/czat/prywatny_1}
    \caption{Widok modułu czatu z otwartą rozmową prywatną.}
    \label{fig:chat:screen-1-main-view}
\end{figure}

Rysunek~\ref{fig:chat:screen-1-main-view} przedstawia główny widok modułu czatu z otwartą rozmową prywatną.
\glslink{ui}{Interfejs} został podzielony na dwa obszary: listę dostępnych konwersacji po lewej stronie oraz okno aktywnej rozmowy
w części centralnej.

Panel listy rozmów zawiera nagłówek \textit{Chats} oraz listę kafelków reprezentujących konwersacje.
Każdy kafelek prezentuje awatar rozmowy, jej nazwę, podgląd ostatniej wiadomości
oraz jej znacznik czasu. Aktualnie otwarta konwersacja jest wyróżniona wizualnie poprzez podświetlenie tła.

W górnej części obszaru rozmowy znajduje się pasek nagłówka z awatarem oraz nazwą rozmówcy.
Kliknięcie w kafelek z nazwą użytkownika i jego awatarem przenosi do profilu tej osoby.
W prawym górnym rogu widoczna jest ikona służąca do inicjowania utworzenia czatu grupowego.

Poniżej nagłówka prezentowana jest historia wiadomości w układzie chronologicznym. Każda wiadomość zawiera awatar nadawcy,
nazwę użytkownika, znacznik czasu oraz treść wiadomości. Dodatkowo, wiadomości są grupowane według daty.

W dolnej części widoku umieszczono pasek tworzenia wiadomości z polem tekstowym oraz zestawem przycisków akcji.
Po lewej stronie dostępny jest przycisk dodawania załączników do wiadomości, natomiast po prawej stronie
znajdują się skróty do otwarcia paneli \glslink{gif}{GIF-ów} lub \glslink{emoji}{emoji}. Na prawo od nich jest przycisk wysyłania wiadomości.
W celu wysłania wiadomości, zamiast przycisku można użyć klawisza Enter.

\subsection{Wysyłanie GIF-ów}
\label{subsec:panel-wyszukiwania-i-wysyania-gif-ow}

\begin{figure}[H]
    \centering
    \includegraphics[width=\textwidth]{./attachments/prezentacja-systemu/czat/gif/1}
    \caption{Panel wyboru GIF-ów -- widok początkowy po otwarciu.}
    \label{fig:chat:gif-panel-1}
\end{figure}

Rysunek~\ref{fig:chat:gif-panel-1} przedstawia panel wyboru \glslink{gif}{GIF-ów}, uruchamiany z poziomu paska tworzenia wiadomości.
Panel wyświetlany jest jako nakładka w obrębie widoku konwersacji.
W górnej części panelu dostępne są zakładki \textit{Gif} oraz \textit{Emoji}, umożliwiające przełączenie trybu wyboru treści.
Poniżej umieszczono pole wyszukiwania, niżej widać predefiniowane kategorie.

\begin{figure}[H]
    \centering
    \includegraphics[width=\textwidth]{./attachments/prezentacja-systemu/czat/gif/2}
    \caption{Panel wyboru GIF-ów -- wyszukiwanie po kategorii \textit{omg}.}
    \label{fig:chat:gif-panel-2}
\end{figure}

Na rysunku~\ref{fig:chat:gif-panel-2} pokazano działanie mechanizmu wyszukiwania po kategorii. Po wyborze kategorii
lista wyników jest zawężana do pasujących elementów.

\begin{figure}[H]
    \centering
    \includegraphics[width=\textwidth]{./attachments/prezentacja-systemu/czat/gif/3}
    \caption{Panel wyboru GIF-ów -- wyszukiwanie po frazie \textit{thank you}.}
    \label{fig:chat:gif-panel-3}
\end{figure}

Rysunek~\ref{fig:chat:gif-panel-3} przedstawia analogiczny scenariusz dla ręcznie wprowadzonej frazy.
Użytkownik może wielokrotnie modyfikować frazę wyszukiwania, przeglądać zwrócone propozycje oraz wybrać konkretną miniaturę.
Wybranie \glslink{gif}{GIF-a} powoduje natychmiastowe jego wysłanie.

\begin{figure}[H]
    \centering
    \includegraphics[width=\textwidth]{./attachments/prezentacja-systemu/czat/gif/4}
    \caption{Wysłanie GIF-a.}
    \label{fig:chat:gif-wyslanie}
\end{figure}

Rysunek~\ref{fig:chat:gif-wyslanie} przedstawia efekt wysłania \glslink{gif}{GIF-a}.

\subsection{Wysyłanie emoji}
\label{subsec:chat-emoji}

\begin{figure}[H]
\centering
\includegraphics[width=\textwidth]{./attachments/prezentacja-systemu/czat/emoji/1}
\caption{Panel wyboru emoji oraz przykład wysłania wiadomości z emoji.}
\label{fig:chat:emoji-panel}
\end{figure}

Rysunek~\ref{fig:chat:emoji-panel} przedstawia mechanizm wyboru \glslink{emoji}{emoji} dostępny z poziomu paska tworzenia wiadomości.
Po naciśnięciu ikony reprezentującej \glslink{emoji}{emoji} otwierany jest panel w formie nakładki, analogiczny do panelu wyboru \glslink{gif}{GIF-ów}.

W górnej części panelu znajdują się zakładki \textit{Gif} oraz \textit{Emoji}, umożliwiające przełączenie trybu wyboru treści.
Poniżej umieszczono pole wyszukiwania, które pozwala filtrować dostępne \glslink{emoji}{emoji} po wpisanej frazie
(na rysunku pokazano wyszukiwanie dla hasła \textit{cat}).
Wyniki prezentowane są w sekcji \textit{Search results} w postaci listy ikon, które użytkownik może wybierać pojedynczo.

Wybranie \glslink{emoji}{emoji} powoduje wstawienie go do pola edycji wiadomości w dolnym pasku, dzięki czemu użytkownik może skomponować treść
złożoną wyłącznie z emotikon lub użyć ich jako uzupełnienia tekstu. Po wysłaniu wiadomości
treść pojawia się w historii rozmowy.

\subsection{Dodawanie załączników do wiadomości}
\label{subsec:chat-attachments}

\begin{figure}[H]
\centering
\includegraphics[width=\textwidth]{./attachments/prezentacja-systemu/czat/att/1}
\caption{Systemowe okno wyboru pliku uruchamiane z poziomu czatu.}
\label{fig:chat:attachments-1}
\end{figure}

Rysunek~\ref{fig:chat:attachments-1} przedstawia pierwszy etap dodawania załącznika.
Po użyciu przycisku dodawania załączników aplikacja
otwiera systemowe okno wyboru pliku. Dzięki wykorzystaniu natywnego selektora użytkownik
może wskazać zasób z lokalnego dysku. Aplikacja umożliwia wybór wielu formatów m.in. PDF, JPG, PNG, DOCX.

\begin{figure}[H]
\centering
\includegraphics[width=\textwidth]{./attachments/prezentacja-systemu/czat/att/2}
\caption{Załącznik dodany do wiadomości -- podgląd w polu edycji przed wysłaniem.}
\label{fig:chat:attachments-2}
\end{figure}

Na rysunku~\ref{fig:chat:attachments-2} pokazano widok czatu po wskazaniu pliku.
Wybrany załącznik pojawia się nad polem wprowadzania treści w formie kafelka zawierającego
ikonę typu pliku oraz jego miniaturę (o ile jest dostępna). Taki podgląd umożliwia szybkie
zweryfikowanie, jaki plik zostanie wysłany.
Użytkownik może w tym samym miejscu dopisać treść wiadomości, a następnie ją wysłać.

\begin{figure}[H]
\centering
\includegraphics[width=\textwidth]{./attachments/prezentacja-systemu/czat/att/3}
\caption{Wiadomość z wysłanym załącznikiem w historii konwersacji.}
\label{fig:chat:attachments-3}
\end{figure}

Rysunek~\ref{fig:chat:attachments-3} przedstawia sposób prezentacji załącznika po wysłaniu wiadomości.
W historii rozmowy wyświetlana jest karta pliku zawierająca jego nazwę oraz rozmiar, co ułatwia
identyfikację przesłanego zasobu. Po prawej stronie karty znajduje się ikona pobierania,
umożliwiająca zapisanie pliku na urządzeniu odbiorcy.
Z kolei, jeżeli wysłanym załącznikiem jest plik multimedialny, prezentowana jest miniatura jego podglądu.

\begin{figure}[H]
\centering
\includegraphics[width=\textwidth]{./attachments/prezentacja-systemu/czat/att/4}
\caption{Możliwość pobrania pliku.}
\label{fig:chat:attachments-4}
\end{figure}

Na rysunku~\ref{fig:chat:attachments-4} pokazano interakcję z wiadomością zawierającą załącznik.
Użytkownik może go pobrać na swoje urządzenie po naciśnięciu na odpowiednią ikonę.

\subsection{Tworzenie czatu grupowego z poziomu rozmowy prywatnej}
\label{subsec:chat-group-create-from-private}

\begin{figure}[H]
\centering
\includegraphics[width=\textwidth]{./attachments/prezentacja-systemu/czat/cg/1}
\caption{Okno wyboru uczestników podczas tworzenia czatu grupowego z rozmowy prywatnej.}
\label{fig:chat:group-create-modal}
\end{figure}

Rysunek~\ref{fig:chat:group-create-modal} przedstawia okno dialogowe służące do utworzenia czatu grupowego bezpośrednio
z poziomu rozmowy prywatnej. Funkcja uruchamiana jest poprzez przycisk w prawym górnym rogu paska nagłówka czatu
(przy aktywnej konwersacji prywatnej). Po jego użyciu \glslink{ui}{interfejs} wyświetla \glslink{modal}{okno modalne},
a w tle pozostaje przyciemniony ekran czatu.

W górnej części okna dostępne jest pole wyszukiwania, umożliwiające filtrowanie listy znajomych po nazwie.
Wybrani użytkownicy prezentowani są w postaci kafelków z nazwą użytkownika, umieszczonych pod polem wyszukiwania,
z możliwością usunięcia danej osoby z listy. Istotnym elementem jest domyślne dodanie aktualnego rozmówcy
z czatu prywatnego do nowo tworzonej rozmowy grupowej --- dzięki temu użytkownik nie musi ponownie wskazywać osoby,
z którą już prowadzi konwersację. W \glslink{ui}{interfejsie} widoczna jest także informacja o limicie liczby uczestników.

Lista wyników wyszukiwania prezentowana jest w formie kafelków zawierających awatar oraz nazwę użytkownika.
Dodanie osoby odbywa się poprzez użycie przycisku z ikoną \textit{+}, natomiast dla już wybranych uczestników
przycisk zmienia stan na nieaktywny, co sygnalizuje przynależność do tworzonej grupy.
Po skompletowaniu składu użytkownik zatwierdza operację przyciskiem \textit{Create Chat}.

\begin{figure}[H]
\centering
\includegraphics[width=\textwidth]{./attachments/prezentacja-systemu/czat/cg/2}
\caption{Nowo utworzony czat grupowy.}
\label{fig:chat:group-created-view}
\end{figure}

Rysunek~\ref{fig:chat:group-created-view} pokazuje widok aplikacji po utworzeniu czatu grupowego.
Nowa konwersacja pojawia się na liście rozmów po lewej stronie, a użytkownik zostaje automatycznie przeniesiony
do okna aktywnego czatu grupowego.

W początkowym stanie historia wiadomości jest pusta (widoczna jest informacja o początku rozmowy), a użytkownik może od razu
rozpocząć wymianę wiadomości, korzystając z tego samego paska tworzenia wiadomości co w czacie prywatnym.

\subsection{Edycja czatu grupowego}
\label{subsec:chat-group-edit}

\begin{figure}[H]
\centering
\includegraphics[width=\textwidth]{./attachments/prezentacja-systemu/czat/eg/1}
\caption{Widok czatu grupowego z przyciskiem uruchamiającym edycję.}
\label{fig:chat:group-edit-1}
\end{figure}

Rysunek~\ref{fig:chat:group-edit-1} przedstawia czat grupowy, z podświetlonym po najechaniu kursorem kafelkiem z nazwą czatu oraz awatarem.
Naciśnięcie kafelka powoduje otwarcie \glslink{modal}{okna modalnego} pozwalającego edytować czat grupowy.

\begin{figure}[H]
\centering
\includegraphics[width=\textwidth]{./attachments/prezentacja-systemu/czat/eg/2}
\caption{Okno edycji czatu grupowego: zmiana nazwy oraz awatara.}
\label{fig:chat:group-edit-2}
\end{figure}

Na rysunku~\ref{fig:chat:group-edit-2} pokazano okno \textit{Edit Chat}, które umożliwia zmianę podstawowych danych czatu grupowego.
W górnej części znajduje się podgląd aktualnego awatara, a kliknięcie obszaru \textit{Change chat image} pozwala wskazać nową grafikę
poprzez systemowy selektor plików. Poniżej umieszczono pole tekstowe przeznaczone do edycji nazwy czatu.
Użytkownik może zatwierdzić zmiany przyciskiem \textit{Save} lub przerwać operację przyciskiem \textit{Cancel} (bądź ikoną zamknięcia).

\begin{figure}[H]
\centering
\includegraphics[width=\textwidth]{./attachments/prezentacja-systemu/czat/eg/3}
\caption{Widok czatu grupowego po zapisaniu zmian nazwy oraz awatara.}
\label{fig:chat:group-edit-3}
\end{figure}

Rysunek~\ref{fig:chat:group-edit-3} przedstawia efekt zapisania zmian.

\subsection{Lista uczestników czatu grupowego}
\label{subsec:chat-group-members}

\begin{figure}[H]
\centering
\includegraphics[width=\textwidth]{./attachments/prezentacja-systemu/czat/g_czlonkowie}
\caption{Panel listy uczestników czatu grupowego oraz możliwość przejścia do profilu.}
\label{fig:chat:group-members}
\end{figure}

Rysunek~\ref{fig:chat:group-members} przedstawia panel uczestników dostępny w widoku czatu grupowego, po naciśnięciu ikony reprezentującej grupę użytkowników.
Panel jest wyświetlany po prawej stronie ekranu i prezentuje zestawienie członków rozmowy wraz z ich
awatarami oraz nazwami użytkowników. W nagłówku umieszczono informację o liczbie uczestników ( \textit{Chat Participants -- 3}).

Każdy element listy ma formę kafelka, który pełni jednocześnie funkcję nawigacyjną.
Kliknięcie w wybranego członka przenosi użytkownika do profilu tej osoby.

\subsection{Dodawanie uczestników do istniejącego czatu grupowego}
\label{subsec:chat-group-add-existing-members}

\begin{figure}[H]
\centering
\includegraphics[width=\textwidth]{./attachments/prezentacja-systemu/czat/dg/1}
\caption{Okno dodawania nowych uczestników do istniejącego czatu grupowego.}
\label{fig:chat:group-add-members-modal}
\end{figure}

Rysunek~\ref{fig:chat:group-add-members-modal} przedstawia mechanizm rozszerzania składu czatu grupowego
o kolejne osoby bez konieczności tworzenia nowej rozmowy. Funkcja uruchamiana jest z poziomu aktywnego czatu,
z górnej części widoku (ikona dodawania użytkownika widoczna obok skrótu do panelu uczestników).
Po użyciu tej akcji wyświetlane jest \glslink{modal}{okno modalne}, a widok czatu w tle zostaje przyciemniony.

W górnej części okna znajduje się pole wyszukiwania umożliwiające filtrowanie po nazwie użytkownika
oraz informacja o limicie liczby osób możliwych do dodania (prezentowana wraz z licznikiem wybranych osób).
Wybrane osoby są pokazywane pod polem wyszukiwania w formie etykiet z możliwością usunięcia
pojedynczego wyboru (ikona \textit{x}).

Poniżej prezentowana jest lista kandydatów w formie kafelków z awatarem i nazwą użytkownika.
Dodanie osoby odbywa się przyciskiem \textit{Add}, natomiast dla osób już wybranych przycisk zmienia stan na
\textit{Added} i staje się nieaktywny.
Zatwierdzenie operacji następuje poprzez przycisk \textit{Add users}, co skutkuje dopisaniem wskazanych osób do rozmowy.

\begin{figure}[H]
\centering
\includegraphics[width=\textwidth]{./attachments/prezentacja-systemu/czat/dg/2}
\caption{Widok czatu grupowego po dodaniu nowych uczestników i aktualizacji listy członków.}
\label{fig:chat:group-add-members-updated}
\end{figure}

Na rysunku~\ref{fig:chat:group-add-members-updated} pokazano efekt dodania użytkowników.
Po zamknięciu okna dialogowego panel uczestników po prawej stronie zostaje zaktualizowany ---
widoczna jest nowa liczba członków (\textit{Chat Participants}) oraz dopisane osoby na liście.
Dzięki temu użytkownik może w prosty sposób rozbudowywać istniejącą rozmowę grupową,
zachowując ciągłość historii wiadomości w ramach jednego czatu.
