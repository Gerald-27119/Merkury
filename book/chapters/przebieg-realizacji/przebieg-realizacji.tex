 %! Author = Adam
%! Date = 10/01/2025

% Informacja, kto jaki rozdział książki napisał ma być w rozdziale o pracy indywidualnej,
% tutaj nie. Komentarz informacyjny.

\chapter{Przebieg realizacji projektu}
\label{ch:przebieg-realizacji-projektu}

W niniejszym rozdziale przedstawiono przebieg realizacji projektu
w ujęciu funkcjonalnym, z podziałem na główne moduły aplikacji.
Opis koncentruje się na tym, co zostało zaimplementowane
w ramach kolejnych obszarów systemu.
Taki sposób ujęcia odpowiada charakterowi prac prowadzonych
zgodnie z metodyką \gls{DAD_LLC}, gdzie rozwój funkcjonalności
odbywał się równolegle, a zakres modułów był stopniowo doprecyzowywany.

Rozwój był wspierany przez stałe \glslink{review-kodu}{review kodu},
automatyzację testów oraz rozwój infrastruktury deweloperskiej.


\section{Elementy niezależne od modułu}
\label{sec:niezalezne}

W całym okresie realizacji rozwijano także elementy wspólne.

\begin{itemize}
    \item \textbf{Stos technologiczny i struktura projektu} --
    doprecyzowanie architektury rozwiązania oraz organizacji repozytorium
    dla \glslink{backend}{backendu} i \glslink{frontend}{frontendu}.

    \item \textbf{Podstawy warstwy klienckiej} --
    stworzenie fundamentów aplikacji \glslink{frontend}{frontendowej}
    (\glslink{routing}{routingu}, konfiguracja stylów, struktury projektu),
    a także dodanie do projektu potrzebnych \glslink{biblioteka}{bibliotek} m.in.: \glslink{tanstack-query}{TanStack Query}, \glslink{redux}{Redux} oraz \glslink{tailwind-css}{Tailwind CSS}.

    \item \textbf{Uwierzytelnianie i bezpieczeństwo} --
    implementacja logowania i rejestracji, doprecyzowanie ról i uprawnień
    w \glslink{spring-security}{Spring Security}, a także logowanie za pomocą \glslink{oauth}{OAuth} poprzez Google oraz \glslink{github}{GitHuba}.

    \item \textbf{Odzyskiwanie dostępu do konta} --
    implementacja resetowania hasła.

    \item \textbf{Komunikacja e-mail} --
    integracja z zewnętrzną usługą wysyłania wiadomości mailowych.

    \item \textbf{Środowisko deweloperskie} --
    konteneryzacja serwisów, na które składa się aplikacja,
    z wykorzystaniem \glslink{docker}{Dockera}.
    Przygotowanie skryptów ułatwiających uruchamianie środowiska deweloperskiego.

    \item \textbf{Jakość i automatyzacja} --
    rozwój testów (\glslink{integration-tests}{integracyjnych}, \glslink{unit-tests}{jednostkowych} i \glslink{e2e-tests}{E2E}),
    uruchamianie testów w potokach \glslink{cicd}{CI/CD},
    ujednolicenie formatowania za pomocą \glslink{prettier}{Prettier'a}
    i kontrola jakości kodu w repozytorium za pomocą \glslink{eslint}{ESLint}.

    \item \textbf{Wydajność} --
    wprowadzenie mechanizmu optymalizacji w postaci \glslink{cache}{cache'a}.

    \item \textbf{Dane deweloperskie} --
    przygotowanie danych testowych wykorzystywanych do demonstracji
    i weryfikacji działania produktu.

    \item \textbf{Komunikaty systemowe} --
    implementacja mechanizmu prezentacji informacji o błędach i sukcesach.

    \item \textbf{Dokumentacja} --
    opracowanie dokumentacji projektu w \glslink{latex}{LaTeX'ie}.
\end{itemize}

Przed rozpoczęciem implementacji modułów przygotowano projekt \glslink{ui}{interfejsu użytkownika (UI)},
obejmujący kluczowe widoki.
Stanowił on punkt odniesienia dla dalszych prac i ułatwił iteracyjne dopasowywanie
funkcjonalności do sposobu użycia systemu.

\setcounter{secnumdepth}{4}
\setcounter{tocdepth}{4}


\section{Moduły aplikacji}
\label{sec:moduly-aplikacji}

\subsection{Rozwój funkcjonalności w modułach}
\label{subsec:rozwoj-modulow}

Poniżej przedstawiono rozwój funkcjonalności w kluczowych modułach aplikacji.
Każdy moduł rozwijano iteracyjnie, stopniowo rozszerzając jego zakres.

\subsubsection{Mapa}
\label{subsubsec:modul-mapa}

\begin{itemize}
    \item \textbf{Prototyp mapy} --
    przygotowanie wstępnej wersji mapy (proof-of-concept)
    w celu weryfikacji możliwości interaktywnej prezentacji danych.

    \item \textbf{Migracja do rozwiązania docelowego} --
    przejście na \glslink{react-maplibre}{React-MapLibre} oraz ponowna implementacja podstaw
    wyświetlania \glslink{spot}{spotów}.

    \item \textbf{Model \glslink{spot}{spota}} --
    rozszerzanie modelu o dane lokalizacyjne oraz ujednolicenie sposobu wyznaczania punktu
    reprezentującego \glslink{spot}{spota}.

    \item \textbf{Widok szczegółów \glslink{spot}{spota}} --
    rozwój prezentacji informacji o \glslink{spot}{spocie}.

    \item \textbf{Komentarze i oceny} --
    dodanie możliwości dodawania komentarzy przez użytkownika.

    \item \textbf{Media dla \glslink{spot}{spotów}} --
    ujednolicenie obsługi zdjęć i filmów oraz implementacja prezentacji
    multimediów w widoku \glslink{spot}{spota}.

    \item \textbf{Galeria multimediów} --
    wprowadzenie rozszerzonej galerii multimediów dla danego \glslink{spot}{spota}.

    \item \textbf{Ulubione \glslink{spot}{spoty}} --
    możliwość dodawania \glslink{spot}{spotów} do ulubionych oraz integracja
    tej funkcji z panelem użytkownika.

    \item \textbf{Responsywność} --
    dostosowanie modułu mapy i widoku szczegółów \glslink{spot}{spotów}
    do różnych rozmiarów ekranu.

    \item \textbf{Podstawowa pogoda} --
    pobieranie danych pogodowych dla wybranego \glslink{spot}{spota}.

    \item \textbf{Rozszerzona pogoda} --
    wprowadzenie pogody szczegółowej.

    \item \textbf{Wyszukiwanie w obszarze mapy} --
    dodanie możliwości wyszukiwania \glslink{spot}{spotów}
    w aktualnie widocznym obszarze mapy.
\end{itemize}

\subsubsection{Czat}
\label{subsubsec:modul-czat}

\begin{itemize}
    \item \textbf{Podstawy czatu} --
    implementacja listy czatów wraz z wyświetlaniem aktualnie otwartego.

    \item \textbf{Komunikacja w czasie rzeczywistym} --
    wdrożenie \glslink{websocket}{WebSocket} do pobierania i wysyłania wiadomości.

    \item \textbf{Usprawnienia UX} --
    grupowanie wiadomości po dacie, wielowierszowe pole tekstowe, dodanie mechanizmu nieskończonego przewijania wiadomości w danym czacie.

    \item \textbf{Emoji i GIF} --
    dodanie obsługi \glslink{emoji}{emoji} oraz wysyłania animowanych obrazów \glslink{gif}{GIF}.

    \item \textbf{Wydajność i niezawodność} --
    optymistyczne wysyłanie wiadomości oraz implementacja stronicowania dla
    pobierania starszych wiadomości.

    \item \textbf{Rozmowy prywatne} --
    możliwość rozpoczęcia i kontynuowania rozmów bezpośrednio
    z list relacji społecznościowych w panelu użytkownika.

    \item \textbf{Czaty grupowe} --
    tworzenie czatów grupowych, edycja nazwy/obrazu
    oraz dodawanie kolejnych uczestników.

    \item \textbf{Obsługa załączników} --
    wysyłanie i wyświetlanie plików oraz obrazów.
\end{itemize}

\subsubsection{Forum}
\label{subsubsec:modul-forum}

\begin{itemize}
    \item \textbf{Podstawowa struktura forum} --
    obsługa postów, kategorii, tagów oraz \glslink{paginacja}{paginacji}.

    \item \textbf{Edytor treści} --
    integracja \glslink{rich-text-editor}{edytora rich-text} na potrzeby tworzenia postów.

    \item \textbf{Media w postach} --
    integracja z \glslink{azure-blob-storage}{Azure Blob Storage} do przesyłania
    i przechowywania mediów.

    \item \textbf{Przeglądanie i sortowanie} --
    dodanie opcji sortowań listy postów i mechanizmów sortowania.

    \item \textbf{Interakcje społecznościowe} --
    komentarze do postów (dodawanie/edycja/usuwanie/głosowanie),
    możliwość zgłaszania treści oraz obserwowanie postów.
\end{itemize}

\subsubsection{Wyszukiwarka spotów}
\label{subsubsec:modul-wyszukiwarka}

\begin{itemize}

    \item \textbf{Wyszukiwanie spotów} --
    wdrożenie wyszukiwania po nazwie z panelem wyników.

    \item \textbf{Wyszukiwanie po lokalizacji} --
    dodanie wyszukiwania po lokalizacji wraz z listą wyników
    i dystansem od użytkownika.

    \item \textbf{Najpopularniejsze spoty} --
    dodanie karuzeli z najpopularniejszymi \glslink{spot}{spotami}.

    \item \textbf{Funkcje zaawansowane} --
    \glslink{paginacja}{Paginacja} wyników, sortowanie (po popularność i ocenie)
    oraz filtrowanie po minimalnej ocenie.

\end{itemize}

\subsubsection{Panel użytkownika}
\label{subsubsec:modul-panel}

\begin{itemize}
    \item \textbf{Profil użytkownika} --
    wdrożenie profilu oraz rozdzielenie widoku własnego profilu
    i profilu innego użytkownika.

    \item \textbf{Relacje społecznościowe} --
    obsługa znajomych, obserwowanych i obserwujących,
    wraz z nawigacją oraz dedykowanymi widokami list.

    \item \textbf{Zaproszenia do znajomych} --
    statusy relacji (wysłane/otrzymane/zakończone),
    widok listy zaproszeń oraz akcje akceptowania i odrzucania.

    \item \textbf{Wyszukiwanie użytkowników} --
    funkcja dodawania znajomych z paginacją i wyszukiwaniem
    po nazwie użytkownika.

    \item \textbf{Aktywność} --
    sekcje zdjęć i filmów, widok dodanych komentarzy do spota.

    \item \textbf{Ustawienia konta} --
    edycja danych konta (nazwa użytkownika, e-mail, hasło)
    oraz obsługa zmiany zdjęcia profilowego.

    \item \textbf{Moje spoty i ulubione} --
    lista dodanych \glslink{spot}{spotów} oraz zarządzanie ulubionymi
    \glslink{spot}{spotami} z integracją z mapą.

\end{itemize}

\setcounter{secnumdepth}{3}
\setcounter{tocdepth}{3}


\section{Podsumowanie etapu finalizacji}
\label{sec:podsumowanie-finalizacja}

Końcowy okres realizacji koncentrował się na domykaniu funkcjonalności
oraz uzupełnieniu dokumentacji technicznej i tekstowej.
Przyjętą datą zakończenia prac projektowych jest
\textbf{10 stycznia 2026 roku}.
