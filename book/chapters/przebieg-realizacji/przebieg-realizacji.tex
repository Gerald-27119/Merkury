%! Author = Adam
%! Date = 10/01/2025


\chapter{Przebieg realizacji projektu}
\label{ch:przebieg-realizacji-projektu}

W niniejszym rozdziale przedstawiono przebieg realizacji
projektu w kolejnych Etapach zdefiniowanych w harmonogramie.
Sposób realizacji projektu odzwierciedla model pracy zespołu zgodny z metodyką
\gls{DAD_LLC}, w której prace deweloperskie, planowanie
i doprecyzowywanie wymagań przebiegają iteracyjnie i równolegle.

Warto podkreślić, że przez cały czas trwania projektu członkowie zespołu
poświęcali znaczącą część czasu na wzajemne przeglądy kodu
(\gls{review-kodu}) oraz ciągły feedback dotyczący implementacji
poszczególnych funkcjonalności.
Taki sposób pracy pozwolił na szybkie wychwytywanie błędów,
ujednolicenie stylu implementacji oraz bieżące korygowanie
wymagań.

Dla przejrzystości opisu, w ramach każdego miesiąca przedstawiono
zadania w podziale na poszczególnych członków zespołu.

\newenvironment{zadaniaosoby}[1]{%
    \par\medskip
    \noindent\textbf{#1}\par
    \vspace{-0.4\baselineskip}
    \begin{itemize}
}{%
    \end{itemize}
}


\section{Faza przedprojektowa (lipiec–wrzesień 2024)}
\label{sec:faza-przedprojektowa}

Faza przedprojektowa, wyróżniona w harmonogramie jako okres od lipca do
września 2024 roku, poprzedzała formalne zatwierdzenie tematu pracy
oraz opracowanie szczegółowego harmonogramu.

Na początku fazy przedprojektowej doprecyzowano koncepcję systemu
i wybrano główny stos technologiczny, co umożliwiło szybkie przejście do
bardziej zaawansowanych działań opisanych w kolejnych podsekcjach.
Niezależnie od ostatecznego sformułowania tematu zakładano stworzenie
aplikacji webowej wymagającej kont użytkowników, co wynikało wprost
ze specyfiki specjalizacji „Aplikacje Internetowe”, na której studiowali
wszyscy członkowie zespołu.

\subsection*{Lipiec 2024}

\begin{zadaniaosoby}{Cały zespół}
    \item Określenie struktury plików na \gls{backend}zie
    \item Wstępny wybór stosu technologicznego:
    \gls{backend} w \gls{spring-boot}, \gls{frontend} w \gls{react} z wykorzystaniem
    \gls{type-script} oraz system kontroli wersji oparty na \gls{github},
    a także ogólne założenia dotyczące dalszego rozwoju architektury.
\end{zadaniaosoby}

\begin{zadaniaosoby}{Adam Langmesser}
    \item Przygotowanie lokalnego środowiska deweloperskiego dla \glslink{backend}{backendu}
    i \gls{frontend}u (konfiguracja \gls{ide}, testowe projekty, podstawowe
    ustawienia narzędzi budujących), co ułatwiło płynne przejście
    do właściwych prac w kolejnych miesiącach.
\end{zadaniaosoby}

\subsection*{Sierpień 2024}

\begin{zadaniaosoby}{Cały zespół}
    \item Utworzenie projektu w narzędziu \gls{jira}, zdefiniowanie
    typów zgłoszeń (epik, zadanie, podzadanie) oraz przygotowanie
    podstawowego workflow z wykorzystaniem \glslink{tablica_kanban}{tablicy kanban}.
\end{zadaniaosoby}

\begin{zadaniaosoby}{Mateusz Redosz}
    \item Konfiguracja biblioteki \gls{tailwind-css} w części
    frontendowej.
    \item Konfiguracja narzędzia \gls{prettier} do automatycznego
    formatowania kodu \glslink{frontend}{frontendu}.
    \item Implementacja rejestracji użytkownika zarówno na \glslink{backend}{backendzie} jak i \glslink{frontend}{frotendzie}
    \item Dodanie i konfiguracja biblioteki \gls{tanstack-query}.
    \item Stworzenie wstępnej struktury \glslink{routing}{routingu} na \glslink{frontend}{frontendzie}.
\end{zadaniaosoby}

\begin{zadaniaosoby}{Adam Langmesser}
    \item Rozpoczęcie prac nad modułem logowania i rejestracji
    użytkownika po stronie backendu; logika ta była następnie
    kilkukrotnie rozwijana i refaktoryzowana w kolejnych etapach
    projektu.
    \item Implementacja obsługi przypadku, w którym podczas rejestracji
    wybrana nazwa użytkownika jest już zajęta (walidacja unikalności
    loginu oraz odpowiednie komunikaty błędów).
\end{zadaniaosoby}

\begin{zadaniaosoby}{Kacper Badek}
    \item Stworzenie formularza logowania użytkownika po stronie
    frontendu oraz jego integracja z przygotowanym backendowym
    API logowania.
\end{zadaniaosoby}

\begin{zadaniaosoby}{Stanisław Oziemczuk}
    \item Integracja systemu z zewnętrzną usługą wysyłania wiadomości
    e-mail oraz implementacja wysyłania wiadomości powitalnej do
    nowo zarejestrowanego użytkownika.
    \item Dodanie pliku konfiguracyjnego do formatowania kodu w Intellij IDEA.
\end{zadaniaosoby}

\subsection*{Wrzesień 2024}

\begin{zadaniaosoby}{Adam Langmesser}
    \item Dodanie cyklicznego testu sprawdzającego, czy backend
    uruchamiany w środowisku \texttt{docker-compose} działa poprawnie
    i odpowiada na podstawowe żądania, co ułatwiło wczesne wykrywanie
    problemów konfiguracyjnych.
\end{zadaniaosoby}

\begin{zadaniaosoby}{Mateusz Redosz}
    \item Dodanie przycisków na frontendzie do rejestracji kontem Google lub Github
\end{zadaniaosoby}

\begin{zadaniaosoby}{Kacper Badek}
    \item Implementacja funkcjonalności resetowania hasła użytkownika,
    obejmującej generowanie i weryfikację tokenów resetu oraz
    formularz zmiany hasła.
\end{zadaniaosoby}


\section{Etap 1 (październik 2024 – styczeń 2025)}
\label{sec:etap1}

Etap 1 był przede wszystkim poświęcony dopracowaniu wymagań wstępnych,
stabilizacji modułu uwierzytelniania oraz pierwszym eksperymentom
z mapą spotów.
W tym okresie zespół łączył prace deweloperskie z działaniami
analitycznymi i planistycznymi (opracowanie harmonogramu, założeń
oraz wymagań w ramach przedmiotów projektowych na uczelni).

\subsection*{Październik 2024}

\begin{zadaniaosoby}{Cały zespół}
    \item Formalny wybór tematu projektu inżynierskiego.
\end{zadaniaosoby}

\begin{zadaniaosoby}{Mateusz Redosz}
    \item Poprawa konfiguracji \gls{cors} na backendzie, tak aby aplikacja
    frontendowa mogła komunikować się z serwerem w sposób bezpieczny
    i zgodny z przeglądarkowymi ograniczeniami.
    \item Zmiana sposobu przechowywania tokena \gls{jwt} – umieszczenie go
    w ciasteczku \gls{http-only-cookie}, co poprawiło bezpieczeństwo
    aplikacji.
\end{zadaniaosoby}

\begin{zadaniaosoby}{Stanisław Oziemczuk}
    \item Implementacja logowania \gls{oauth} z wykorzystaniem Google
    i GitHub po stronie backendu oraz integracja z frontem.
\end{zadaniaosoby}

\begin{zadaniaosoby}{Kacper Badek}
    \item Uporządkowanie pliku \texttt{.gitignore}
    oraz struktury repozytorium.
\end{zadaniaosoby}

\subsection*{Listopad 2024}

\begin{zadaniaosoby}{Adam Langmesser}
    \item Implementacja demonstracyjnej mapy z wykorzystaniem
    biblioteki \gls{leaflet} – prototyp miał na celu pokazanie zespołowi
    możliwości interaktywnej mapy.
    Ze względu na ograniczone możliwości customizacji wyglądu
    biblioteki zdecydowano się później na zmianę dostawcy
    kafelków mapowych na usługę
    \item Poprawki konfiguracji narzędzia \gls{eslint}
    po stronie frontendu.
\end{zadaniaosoby}

\begin{zadaniaosoby}{Mateusz Redosz}
    \item Dalsze poprawki obsługi \gls{jwt} na backendzie oraz logowania
    błędów związanych z procesem logowania i rejestracji.
\end{zadaniaosoby}

\begin{zadaniaosoby}{Stanisław Oziemczuk}
    \item Poprawki działania logowania użytkownika po stronie backendu
    i frontendu, obejmujące obsługę błędów oraz komunikaty
    dla użytkownika.
\end{zadaniaosoby}

\begin{zadaniaosoby}{Kacper Badek}
    \item Implementacja logiki cyklicznego usuwania przeterminowanych
    tokenów resetu hasła.
\end{zadaniaosoby}

\begin{zadaniaosoby}{Cały zespół}
    \item Opracowanie harmonogramu projektu w formie opisowej
    oraz w postaci wykresu Gantta.
    \item Przygotowanie wstępnych założeń i wymagań
    w ramach przedmiotu \gls{pro}.
\end{zadaniaosoby}

\subsection*{Grudzień 2024}

\begin{zadaniaosoby}{Adam Langmesser}
    \item Dalsze poprawki konfiguracji \gls{spring-security} na backendzie,
    w tym doprecyzowanie ról i uprawnień.
    \item Rozszerzenie encji użytkownika o dane deweloperskie
    oraz przygotowanie inicjalnych danych w bazie
    (np. konta testowe).
    \item Implementacja testów automatycznych związanych
    z bezpieczeństwem (scenariusze logowania i rejestracji
    użytkownika po stronie backendu).
\end{zadaniaosoby}

\begin{zadaniaosoby}{Mateusz Redosz}
    \item Dodanie biblioteki \gls{redux} do frontendu i wstępna
    konfiguracja store.
    \item Implementacja automatycznego wylogowywania użytkownika
    po wygaśnięciu tokena \gls{jwt}.
    \item Stworzenie komponentu odpowiedzialnego za prezentację błędów
    systemowych użytkownikowi (globalny mechanizm powiadomień).
\end{zadaniaosoby}

\begin{zadaniaosoby}{Stanisław Oziemczuk}
    \item Stworzenie komponentu frontendu odpowiedzialnego
    za wyświetlanie szczegółów pojedynczego spota.
\end{zadaniaosoby}

\begin{zadaniaosoby}{Kacper Badek}
    \item Poprawa logowania błędów związanych z resetowaniem hasła
    użytkownika po stronie backendu.
    \item Dostosowanie wyglądu i treści wiadomości e-mail wysyłanych
    przez system (np. w procesie resetu hasła).
\end{zadaniaosoby}

\subsection*{Styczeń 2025}

\begin{zadaniaosoby}{Adam Langmesser}
    \item Poprawa testów logowania i rejestracji na backendzie
    oraz ujednolicenie asercji.
\end{zadaniaosoby}

\begin{zadaniaosoby}{Mateusz Redosz}
    \item Rozszerzenie stanu \gls{redux} o informację o tym,
    czy użytkownik jest aktualnie zalogowany.
    \item Implementacja \glslink{e2e-tests}{testów E2E} dla procesów logowania
    i rejestracji użytkownika.
    \item Implementacja \glslink{integraion-tests}{testów integracyjnych} i \glslink{unit-tests}{testów jednostkowych} dla \glslink{sidebar}{sidebara}.
    \item Dodanie uruchamiania testów frontendu do \gls{cicd}.
    \item Poprawa sposobu wyświetlania szczegółów spota na froncie.
\end{zadaniaosoby}

\begin{zadaniaosoby}{Stanisław Oziemczuk}
    \item Poprawki logowania i rejestracji z wykorzystaniem \gls{oauth}
    (Google/GitHub) – dopracowanie scenariuszy brzegowych.
    \item Implementacja logiki filtrowania spotów na mapie
    po różnych kryteriach (np. nazwa) po stronie backendu.
    \item Dalsze dopracowanie logiki filtrowania spotów po nazwie
    po stronie backendu.
    \item Implementacja logiki zmiany danych konta użytkownika.
\end{zadaniaosoby}

\begin{zadaniaosoby}{Kacper Badek}
    \item Dodanie danych deweloperskich dla mapy
    (przykładowe spoty, dane do prezentacji i testów).
    \item Implementacja logiki dodawania spota do ulubionych
    po stronie backendu.
\end{zadaniaosoby}


\section{Etap 2 (luty 2025 – wrzesień 2025)}
\label{sec:etap2}

Etap 2 obejmował zasadniczą część prac deweloperskich nad aplikacją.
W tym okresie rozwijano kolejne moduły (mapa, forum, czat,
panel użytkownika), równolegle doprecyzowując dokumentację oraz
weryfikując i aktualizując wymagania na podstawie bieżących
doświadczeń zespołu.

Równolegle, w ramach przedmiotu \gls{prz1} opracowano projekt interfejsu
użytkownika, konsultowany z prowadzącym zajęcia, mgr. inż. Adamem
Urbanowiczem, który na bieżąco przekazywał zespołowi uwagi i
rekomendacje dotyczące ergonomii oraz spójności interfejsu
z założeniami funkcjonalnymi.

\subsection*{Luty 2025}

\begin{zadaniaosoby}{Mateusz Redosz}
    \item Poprawa wyglądu strony logowania, w tym dopracowanie trybu jasnego tejże strony.
    \item Implementacja automatycznego wylogowania użytkownika.
    \item Implementacja obsługi błędów \gls{jakarta-validation}.
    \item Refactor wyglądu komponentu pogodowego na \glslink{frontend}{frontendzie}.
    \item Poprawa działa \gls{cicd}.
    \item Wypracowanie propozycji palety kolorystycznej interfejsu użytkownika.
\end{zadaniaosoby}

\begin{zadaniaosoby}{Stanisław Oziemczuk}
    \item Implementacja logiki dodawania spota do ulubionych
    po stronie frontendu oraz integracja z backendem.
    \item Implementacja integracji z zewnętrznym \gls{api} pogodowym
    służącym do pobierania podstawowych danych pogodowych
    dla danego spota.
    \item Drobna refaktoryzacja kodu na frontendzie.
    \item Obsługa błędów przy wysyłaniu maili.
    \item Zmiana dostawy usługi do wysyłania maili.
    \item Optymalizacja zapytań do bazy danych.
    \item Testy Integracyjne logiki logowania i rejestracji OAuth2 na backendzie oraz filtrowania spotów.
    \item Poprawa plików Dockerfile oraz docker-compose.
\end{zadaniaosoby}

\begin{zadaniaosoby}{Adam Langmesser}
    \item Dalsze poprawki konfiguracji \gls{spring-security} na backendzie,
    obejmujące doprecyzowanie reguł autoryzacji i konfiguracji filtrów.
    \item Poprawa konfiguracji pliku \texttt{.gitignore} w repozytorium,
    tak aby obejmował również nowe katalogi i pliki generowane przez
    narzędzia deweloperskie.
\end{zadaniaosoby}

\subsection*{Marzec 2025}

\begin{zadaniaosoby}{Cały Zespół}
    \item Wypracowanie wstępnej wersji wyglądu interfejsu użytkownika w ramach \gls{prz1}.
\end{zadaniaosoby}

\begin{zadaniaosoby}{Kacper Badek}
    \item Refaktoryzacja szablonów wiadomości e-mail związanych
    z rejestracją i resetowaniem hasła – zastąpienie układu
    opartego na \textit{flexboxie} layoutem tabelarycznym
    z \textit{inline-stylami}, ujednolicenie struktury HTML,
    metadanych oraz stylu logo, co poprawiło kompatybilność
    z różnymi klientami pocztowymi.
    \item Rozbudowa systemu komentarzy do spotów: dodanie możliwości
    edycji, usuwania i głosowania na komentarze, wydzielenie
    dedykowanego kontrolera po stronie backendu oraz lepsza integracja
    z widokiem szczegółów spota (paginacja, powiadomienia,
    prezentacja ocen).
\end{zadaniaosoby}

\begin{zadaniaosoby}{Stanisław Oziemczuk}
    \item Przygotowanie skryptu uruchamiającego wszystkie wymagane
    kontenery \gls{docker} (bazy danych i usługi pomocnicze) na potrzeby
    środowiska deweloperskiego, co uprościło proces uruchamiania
    projektu na nowych stanowiskach.
    \item Przygotowanie instrukcji uruchamiania projektu w README.
\end{zadaniaosoby}

\begin{zadaniaosoby}{Mateusz Redosz}
    \item Migracja wszystkich adresów obrazów (galerie zdjęć spotów oraz logotypy
    w szablonach e-mail) z usługi Google Drive do \gls{cdn} ucarecdn.com
\end{zadaniaosoby}

\subsection*{Kwiecień 2025}

\begin{zadaniaosoby}{Cały Zespół}
    \item Wypracowanie finalnej wersji wyglądu interfejsu użytkownika w ramach \gls{prz1}.
\end{zadaniaosoby}

\begin{zadaniaosoby}{Adam Langmesser}
    \item Konfiguracja projektu dokumentacji w \LaTeX{}: dodanie pliku
    \texttt{.gitignore} z typowymi plikami pomocniczymi \LaTeX{}a,
    przygotowanie \texttt{Makefile} automatyzującego proces budowania,
    utworzenie pliku bibliografii oraz głównego dokumentu
    \texttt{engineer.tex} z przykładową strukturą rozdziałów
    i klasą \texttt{sprz.cls}.
    \item Implementacja pełnej funkcjonalności czatu w warstwie
    backendowej i frontendowej (podstawowe API, integracja z \gls{redux},
    widoki ekranów czatu), a także dostosowanie layoutu i paska bocznego
    do nowej sekcji.
\end{zadaniaosoby}

\begin{zadaniaosoby}{Mateusz Redosz}
    \item Integracja \glslink{frontend}{frontendu} z \gls{type-script}.
    \item Integracja \gls{tailwind-css} z projektem \glslink{frontend}{frontendu} oraz dodanie
    predefiniowanych kolorów zgodnych z nową szatą kolorystyczną interfejsu użytkownika
    \item Poprawa i rozbudowa \glslink{sidebar}{sidebara}.
    \item Implementacja strony profilu użytkownika.
\end{zadaniaosoby}

\subsection*{Maj 2025}

\begin{zadaniaosoby}{Stanisław Oziemczuk}
    \item Migracja mapy z biblioteki \gls{leaflet} na \gls{maplibre},
    oraz implementacja od nowa podstaw wyświetlania spotów i interakcji użytkownika z mapą.
    \item Zmiana wyglądu znacznika lokalizacji użytkownika na mapie.
    \item Migracja funkcji API związanych ze spotami do \gls{type-script},
    rozszerzenie modelu szczegółów spota o dodatkowe informacje
    lokalizacyjne i statystyczne (np. kraj, miasto, tagi, liczniki ocen)
    oraz refaktoryzacja logiki interakcji z markerami MapLibre
    (wygodniejsze otwieranie modali ze szczegółami, dopracowany wygląd
    markerów oraz layout sekcji ulubionych spotów).
\end{zadaniaosoby}

\begin{zadaniaosoby}{Mateusz Redosz}
    \item Reorganizacja panelu bocznego (\textit{Sidebar}) do zestawu
    modularnych komponentów z dynamicznie generowanymi linkami,
    dodanie ogólnych hooków do zarządzania stanem (w tym trybem
    ciemnym) oraz wprowadzenie zależności do biblioteki \texttt{motion}
    na potrzeby animacji.
    \item Dodanie sekcji \textit{„Znajomi”} w panelu konta, obejmującej
    obsługę znajomych, obserwowanych i obserwujących, wprowadzenie
    nowych modeli i endpointów do zarządzania relacjami oraz dodanie
    testów jednostkowych dla komponentu \textit{FriendCard}.
    \item Refaktoryzacja endpointów użytkownika tak, aby login był
    pobierany z kontekstu uwierzytelnienia zamiast z parametru
    ścieżki, dostosowanie do tego wywołań na froncie oraz przeniesienie
    slice'a konta na \gls{type-script} z uproszczonym stanem i
    uporządkowanymi importami.
    \item poprawa wyglądu formularzy logowania i rejestracji
    \item Poprawa zarządzania stanem sidebara.
\end{zadaniaosoby}

\begin{zadaniaosoby}{Adam Langmesser}
    \item Wprowadzenie usprawnień w potokach \gls{cicd} backendu,
    obejmujących optymalizację czasu budowania oraz doprecyzowanie
    kroków uruchamiania testów, co zwiększyło niezawodność procesu
    wdrażania.
\end{zadaniaosoby}

\begin{zadaniaosoby}{Kacper Badek}
    \item Dodanie pełnoprawnego forum po stronie frontendu, obejmującego
    obsługę postów, kategorii, tagów i paginacji, oraz integracja
    backendu z usługą \gls{azure-blob-storage} do uploadu i
    przechowywania mediów, wraz z dostosowaniem layoutu, routingu
    i podstawowych styli.
\end{zadaniaosoby}

\subsection*{Czerwiec 2025}

W czerwcu 2025 roku rozwijano przede wszystkim panel konta użytkownika,
w tym ulubione spoty, oraz refaktoryzowano sekcję komentarzy do spotów.

\begin{zadaniaosoby}{Mateusz Redosz}
    \item Przebudowa profilu użytkownika poprzez rozdzielenie widoków
    własnego profilu i profili innych użytkowników (nowe endpointy
    \textit{public/private profile}, rozszerzony model danych), dodanie
    przycisków akcji (follow/unfollow, zaproszenie do znajomych),
    usprawnienie nawigacji z kart znajomych i sidebaru oraz aktualizacja
    routingu i testów do nowego podziału.
    \item Dodanie sekcji \textit{„Ulubione spoty”} w panelu konta
    (lista, filtrowanie, usuwanie oraz podgląd na mapie), wprowadzenie
    współdzielonych komponentów i modeli dla tej funkcji oraz
    uporządkowanie przestarzałego kodu i modeli współrzędnych.
    \item Przebudowa funkcjonalności społecznościowej: rozdzielenie
    widoków sekcji \textit{social} dla właściciela profilu i osoby
    oglądającej, aktualizacja modelu danych i routingu oraz dodanie
    nawigacji z liczników profilu do odpowiednich list (friends,
    followers, followed), wraz z wprowadzeniem osobnego slice'a Redux
    do zarządzania aktywnym typem listy social.
\end{zadaniaosoby}

\begin{zadaniaosoby}{Stanisław Oziemczuk}
    \item Refaktoryzacja komentarzy do spotów – zastąpienie ogólnego
    modelu szczegółowym, spot-specyficznym typowaniem i nowymi
    komponentami UI, uporządkowanie warstwy API oraz poprawa wyglądu
    i układu sekcji komentarzy.
\end{zadaniaosoby}

\subsection*{Lipiec 2025}

W lipcu 2025 roku intensywnie rozwijano część społecznościową,
panel konta oraz czat, a także usprawniano obsługę mapy
i mediów w systemie.

\begin{zadaniaosoby}{Mateusz Redosz}
    \item Dodanie nowej sekcji \textit{„Zdjęcia”} w panelu użytkownika,
    z możliwością sortowania i filtrowania po dacie, wyodrębnienie
    wspólnego wrappera layoutu dla podstron konta, aktualizacja
    routingu i styli oraz dopisanie testów dla wyboru daty.
    \item Dodanie sekcji komentarzy w panelu konta (z grupowaniem
    po dacie i spocie oraz filtrowaniem/sortowaniem po dacie),
    co ułatwiło użytkownikom przeglądanie własnej aktywności.
    \item Dodanie strony ustawień konta umożliwiającej edycję nazwy
    użytkownika, adresu e-mail i hasła (z walidacją w oparciu
    o \texttt{zod} i \texttt{react-hook-form}), aktualizacja routingu
    i modeli (nowe enumy/interfejsy, wariant przycisku), usunięcie
    starych endpointów edycji danych z \textit{AccountController}
    oraz dodanie zależności potrzebnych do obsługi formularzy.
    \item Ujednolicenie obsługi zdjęć i filmów do wspólnego modelu
    \textit{media}, dodanie nowej sekcji \textit{„Filmy”}
    w panelu użytkownika (routing, endpointy, komponenty UI) oraz
    refaktoryzacja istniejących widoków i DTO tak, aby korzystały
    z nowych, współdzielonych struktur.
\end{zadaniaosoby}

\begin{zadaniaosoby}{Adam Langmesser}
    \item Rozbudowa funkcjonalności czatu – dodanie szczegółowego
    pobierania danych czatu i wysyłania wiadomości po \gls{websocket},
    odświeżenie interfejsu (spinner, skeletony list, okno rozmowy
    z separatorami dat i pustym stanem) oraz wprowadzenie nowych
    zależności i zmian w stylach poprawiających UX.
    \item Uporządkowanie formatowania kodu z wykorzystaniem
    narzędzia \gls{prettier}, dodanie skryptu \texttt{format:check}
    oraz sprawdzania formatowania w potokach \gls{cicd}.
    \item Przeprowadzenie dużej refaktoryzacji czatu: ujednolicenie
    modelu danych do jednego \textit{ChatDto}, uproszczenie powiązanego
    API i struktury store'a Redux, dostosowanie komponentów czatu
    do nowego modelu oraz poprawa wyglądu (tło, paski przewijania),
    połączona z czyszczeniem i formatowaniem kodu.
    \item Poprawa danych deweloperskich w bazie (m.in. przykładowych
    spotów i użytkowników), tak aby lepiej odzwierciedlały realistyczne
    scenariusze działania systemu.
    \item Usprawnienie mechanizmu nieskończonego przewijania i nazewnictwa
    listy czatów oraz redukcja drobnego długu technicznego
    (lepsze wartości domyślne, uproszczone typy).
    \item Stworzenie ogólnej logiki i komponentów pomocniczych
    umożliwiających wszystkim członkom zespołu wygodne korzystanie
    z \gls{websocket}ów na froncie (uogólnione hooki i funkcje
    narzędziowe).
    \item Poprawa UX czatu poprzez dodanie grupowania wiadomości
    w czasie z podpowiedzią pełnej daty, wielowierszowego pola tekstowego
    z obsługą Enter/Shift+Enter, uproszczonych klas ikon oraz
    dodatkowych poprawek layoutu i efektów najechania.
    \item Wprowadzenie drobnych poprawek w potokach \gls{cicd}
    (konfiguracja kroków, stabilność zadań).
\end{zadaniaosoby}

\begin{zadaniaosoby}{Stanisław Oziemczuk}
    \item Przebudowa wyszukiwania spotów na mapie – zastąpienie
    dotychczasowych filtrów nowym paskiem wyszukiwania po nazwie
    z bocznym panelem wyników i paginacją oraz dopracowanie interfejsu
    (gwiazdki ocen, warstwy sidebarów) i logiki cache'owania zapytań.
    \item Ujednolicenie obsługi mediów dla spotów i komentarzy
    (wspólny model dla zdjęć i wideo), aktualizacja powiązanych
    struktur danych i galerii w UI oraz dodanie zależności i styli
    potrzebnych do odtwarzania oraz wygodnego wyświetlania multimediów.
    \item Refaktoryzacja struktury plików na backendzie związanych
    z modułem mapy, co uprościło nawigację po kodzie i ułatwiło
    dalszy rozwój funkcjonalności.
\end{zadaniaosoby}

\begin{zadaniaosoby}{Kacper Badek}
    \item Zapoznanie sie z dokumentacją tinyMce richTextEditor.
\end{zadaniaosoby}

\subsection*{Sierpień 2025}

W sierpniu 2025 roku kontynuowano rozwój czatu, wyszukiwania
oraz panelu użytkownika, a także rozbudowano stronę główną
i usprawniono forum.

\begin{zadaniaosoby}{Adam Langmesser}
    \item Dodanie do czatu możliwości wyszukiwania i wysyłania animowanych
    obrazów (GIF) oraz wstawiania emoji z dedykowanego okna przy polu
    wpisywania wiadomości, co znacząco poprawiło komfort rozmowy.
    \item Usprawnienie czatu poprzez wprowadzenie optymistycznego
    wysyłania wiadomości z natychmiastową aktualizacją okna rozmowy
    i listy czatów, paginację oraz wygodniejsze wybieranie konwersacji
    (z obsługą nieprzeczytanych), dodanie nowych endpointów
    do stronicowanego pobierania wiadomości, wzmocnienie typowania
    i refaktoryzacja komponentów oraz przygotowanie zależności
    pod wirtualizowane listy i \gls{infinite-scroll}.
\end{zadaniaosoby}

\begin{zadaniaosoby}{Stanisław Oziemczuk}
    \item Uporządkowanie tematu współrzędnych spotów: dodanie
    jednoznacznego „punktu środka” spota i konsekwentne wykorzystywanie
    go w backendzie i frontendzie, co poprawiło dokładność
    pozycjonowania markerów na mapie.
\end{zadaniaosoby}

\begin{zadaniaosoby}{Mateusz Redosz}
    \item Rozbudowa strony głównej: dodanie karuzeli z najpopularniejszymi
    spotami, wyszukiwarki po lokalizacji (kraj/region/miasto)
    z listą wyników i dystansem od użytkownika, wprowadzenie
    odpowiednich endpointów i modeli danych po stronie API oraz
    poprawa zachowania i wyglądu sidebaru, co ułatwiło odkrywanie
    nowych miejsc do latania dronem.
    \item Dodanie zaawansowanego wyszukiwania spotów na stronie głównej
    (osobny widok z filtrami po mieście i tagach oraz przełącznik
    między prostym a zaawansowanym trybem), dostosowanie modeli
    i endpointów API (m.in. obsługa sugestii tagów i pól opcjonalnych)
    oraz dopracowanie interfejsu listy wyników, w tym komunikatu
    o braku dopasowanych spotów.
    \item Wprowadzenie paginacji do endpointów panelu użytkownika
    i przebudowa powiązanych komponentów frontendu na nieskończone
    przewijanie oparte na \gls{tanstack-query} i \gls{intersection-observer},
    dzięki czemu listy (znajomi, obserwujący, ulubione spoty, media,
    komentarze) ładują się porcjami z poprawioną obsługą stanów
    ładowania i pustych wyników.
    \item Dodanie sekcji \textit{„Zdjęcia”} w panelu społecznościowym
    użytkownika (ze zdjęciami pogrupowanymi po dacie i stronicowaniem),
    uogólnienie i uproszczenie logiki nieskończonego przewijania
    dla różnych zakładek social (friends/followers/followed/photos)
    oraz poprawa drobnych błędów w backendzie i kontraktach komponentów.
    \item Dodanie do panelu użytkownika funkcji dodawania własnych
    spotów (z adresem, multimediami i wielokątem na mapie) wraz
    z nowymi endpointami i modelami danych oraz widokiem listy
    dodanych spotów z formularzem w modalu i nieskończonym przewijaniem.
\end{zadaniaosoby}

\begin{zadaniaosoby}{Kacper Badek}
    \item Przygotowanie formularza dodawania postów z wykorzystaniem edytora TinyMCE.
    \item Skonfigurowanie biblioteki jsoup na backendzie.
\end{zadaniaosoby}

\subsection*{Wrzesień 2025}

\begin{zadaniaosoby}{Adam Langmesser}
    \item Wprowadzenie możliwości rozpoczynania lub kontynuowania
    prywatnych rozmów czatowych bezpośrednio z list znajomych
    i obserwujących poprzez dodanie przycisku
    \textit{„Wiadomość”} na kartach społecznościowych oraz integrację
    z logiką wyszukiwania i tworzenia czatów prywatnych.
    \item Dodanie nowoczesnej obsługi emoji w komunikatorze – integracja
    komponentu wyboru emoji z polem wpisywania wiadomości oraz
    uporządkowanie wyglądu okna wyboru animowanych obrazów (GIF),
    co poprawiło ergonomię korzystania z czatu.
\end{zadaniaosoby}

\begin{zadaniaosoby}{Mateusz Redosz}
    \item Refaktoryzacja systemu powiadomień na froncie, umożliwiająca
    jednoczesne wyświetlanie wielu komunikatów oraz zwiększająca
    modularność i możliwość ponownego wykorzystania komponentów
    powiadomień.
    \item Wprowadzenie stronicowania (paginacji) dla endpointów
    wyszukiwania spotów na backendzie oraz dostosowanie komponentów
    stron głównych i list wyników wyszukiwania do pracy
    w trybie nieskończonego przewijania (\gls{infinite-scroll})
    z mechanizmem \textit{„load more”}, z wykorzystaniem tzw.
    obserwatora przecięcia (\gls{intersection-observer}) oraz
    przekazywaniem referencji i stanów ładowania.
    \item Dodanie walidacji formularza dodawania spota
    w oparciu o schemat walidacyjny, w tym sprawdzanie kompletności
    danych, wymaganego zestawu multimediów oraz punktów poligonu,
    a także wyświetlanie komunikatów błędów bezpośrednio pod sekcjami
    mediów i obszaru na mapie.
    \item Rozszerzenie komponentu przesyłania multimediów o podgląd
    wybranych obrazów i materiałów wideo oraz poprawę zarządzania
    cyklem życia adresów URL podglądu; jednocześnie podniesiono
    priorytet wyświetlania listy powiadomień (warstwa \gls{z-index}),
    aby były lepiej widoczne.
    \item Wprowadzenie pełnej obsługi zmiany zdjęcia profilowego
    użytkownika, obejmującej wysyłanie plików do backendu, usuwanie
    poprzednich zdjęć z magazynu obiektowego oraz aktualizację adresu
    zdjęcia w profilu użytkownika.
    \item Refaktoryzacja struktury projektu \LaTeX{} (\gls{latex}):
    zastąpienie dotychczasowej treści demonstracyjnej rzeczywistymi
    rozdziałami opisującymi projekt \textit{„spoty-na-drony.pl”},
    aktualizacja metadanych (tytuł, autorzy, promotor, cele projektu)
    oraz dodanie osobnych plików rozdziałów dla głównych części pracy
    (wstęp, opis problemu, kontekst projektu, analiza wymagań,
    decyzje projektowe, projekt, planowanie, implementacja, testowanie,
    prezentacja systemu, podsumowanie) i włączenie ich do pliku głównego.
    \item Rozszerzenie zaawansowanych możliwości wyszukiwania spotów
    o sortowanie wyników według popularności lub oceny oraz filtrowanie
    po minimalnej ocenie, wraz z odpowiednimi listami rozwijanymi
    w interfejsie użytkownika i obsługą nowych parametrów
    po stronie backendu.
\end{zadaniaosoby}

\begin{zadaniaosoby}{Stanisław Oziemczuk}
    \item Implementacja kompletnej funkcjonalności pogody dla spotów:
    dodanie podstawowego i szczegółowego modalu pogodowego
    (\gls{modal}) oraz zestawu komponentów interfejsu prezentujących
    m.in. temperaturę, prędkość wiatru, opady oraz dodatkowe
    parametry meteorologiczne.
    \item Przebudowa obsługi pogody tak, aby zamiast bezpośrednich
    wywołań publicznego API wykorzystywany był backend jako warstwa
    pośrednia, z wprowadzeniem dedykowanych struktur \gls{dto} oraz
    dostosowaniem komponentów prezentujących dane pogodowe
    do nowego modelu danych i wymagań responsywności.
    \item Implementacja wyświetlania zdjęcia profilowego użytkownika w komentarzach spota.
\end{zadaniaosoby}

\begin{zadaniaosoby}{Kacper Badek}
    \item Zamiana edytora TinyMCE na Tiptap.
    \item Implementacja przeglądania postów na forum oraz ich sortowania.
    \item wprowadzenie czytelniejszych adresów URL opartych na slugach,
    poprawa walidacji dodawania nowego posta.
    \item Poprawa ergonomii poruszania się po forum.
\end{zadaniaosoby}

\section{Etap 3 (październik 2025 – styczeń 2026)}
\label{sec:etap3}

Etap 3 obejmował finalizację prac nad systemem,
dopracowanie dokumentacji technicznej i tekstowej pracy inżynierskiej
oraz przygotowanie projektu do oddania.
W tym czasie większość nowych funkcjonalności była już zaimplementowana,
a nacisk położono na stabilizację, testy oraz spójny opis
w dokumentacji.

W ramach przedmiotu \gls{psem}, prowadzonego przez dr. inż. Marka
Bednarczyka, postępy w przygotowywaniu dokumentacji były na bieżąco
konsultowane, a na podstawie uzyskiwanej informacji zwrotnej wprowadzano
kolejne poprawki i uzupełnienia.
Analogiczny tryb pracy przyjęto w ramach przedmiotu \gls{prz2}, którego
opiekunem był promotor pracy, mgr. inż. Adam Urbanowicz.

\subsection*{Październik 2025}

\begin{zadaniaosoby}{Adam Langmesser}
    \item Wprowadzenie możliwości tworzenia czatów grupowych,
    w tym obsługi wyboru uczestników, komunikacji z backendem
    oraz integracji z istniejącą listą czatów.
    \item Dodanie funkcjonalności wysyłania i wyświetlania załączników
    w wiadomościach czatu (pliki oraz obrazy), wraz z logiką wyboru,
    podglądu i wysyłania samych plików bez treści tekstowej.
    \item Rozszerzenie istniejącej funkcjonalności czatów grupowych
    o możliwość edycji ich parametrów (zmiana nazwy oraz obrazu czatu)
    po stronie backendu i frontendu.
    \item Dodanie możliwości dołączania nowych użytkowników
    do istniejących czatów grupowych, wraz z odpowiednimi
    endpointami HTTP i modyfikacją interfejsu użytkownika.
\end{zadaniaosoby}

\begin{zadaniaosoby}{Mateusz Redosz}
    \item Rozbudowa systemu statusów znajomych w części społecznościowej
    aplikacji, w tym rozróżnienie zaproszeń wysłanych, otrzymanych
    oraz relacji zakończonych, a także dostosowanie interfejsu
    do prezentacji odpowiednich komunikatów i akcji.
    \item Wprowadzenie zaawansowanego zarządzania zaproszeniami
    do znajomych: dodanie modalnego widoku listy zaproszeń, obsługi
    akceptowania i odrzucania oraz integracji z dedykowanymi
    endpointami backendowymi.
    \item Dodanie funkcji \textit{„Dodaj znajomego”} w sekcji
    społecznościowej, obejmującej wyszukiwarkę użytkowników
    (paginacja, wyszukiwanie po nazwie użytkownika) oraz spójny
    wygląd modalnego okna wyszukiwania.
    \item Rozbudowa komponentu przycisku przesyłania plików
    o możliwość podglądu wielu plików, nadawania im unikalnych
    identyfikatorów oraz usuwania pojedynczych plików przed wysłaniem.
    \item Przygotowanie struktury rozdziałów w \glslink{latex}{latexie}.
    \item Opracowanie rozdziału ~\ref{subsec:mateusz-redosz} \textit{\nameref{subsec:mateusz-redosz}}.
\end{zadaniaosoby}

\begin{zadaniaosoby}{Stanisław Oziemczuk}
    \item Wprowadzenie rozszerzonej galerii multimediów dla spotów,
    obejmującej obsługę paginacji, podglądu w trybie pełnoekranowym
    oraz dodatkowych akcji (np. udostępnianie odnośnika do zasobu).
    \item Dostosowanie modułu mapy oraz widoku szczegółów spotów
    pod kątem responsywności i skalowania na dużych ekranach,
    w tym korekta wysokości komponentów, układu galerii multimediów
    oraz przewijania list komentarzy.
\end{zadaniaosoby}

\begin{zadaniaosoby}{Kacper Badek}
    \item Dodanie obsługi komentarzy do postów (dodawanie, edycję,
    usuwanie oraz głosowanie).
    \item Zastąpienie klasycznych wskaźników ładowania
    loaderami typu \gls{skeleton-loader} dla listy postów oraz paneli
    kategorii i tagów.
    \item Poprawa działania stanu formularza dodawania postów.
    \item Dodanie możliwości zgłaszania postów i komentarzy.
    \item Implementacja możliwości obserwowania postów.
\end{zadaniaosoby}

\subsection*{Listopad 2025}

%TODO: do uzupełnienia
TODO: do uzupełnienia

\subsection*{Grudzień 2025}

%TODO: do uzupełnienia
TODO: do uzupełnienia

\subsection*{Styczeń 2026}

%TODO: do uzupełnienia
TODO: do uzupełnienia

Na zakończenie prac projektowych przyjęto datę
\textbf{10 stycznia 2026 roku}.
