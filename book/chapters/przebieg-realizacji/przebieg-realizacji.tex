%! Author = Adam
%! Date = 10/01/2025

\chapter{Przebieg realizacji projektu}
\label{ch:przebieg-realizacji-projektu}

W niniejszym rozdziale przedstawiono rzeczywisty przebieg realizacji
projektu w kolejnych fazach czasowych.
Opis odzwierciedla sposób pracy zespołu zgodny z metodyką
\gls{DAD_LLC}, w której prace deweloperskie, planowanie
i doprecyzowywanie wymagań przebiegają iteracyjnie i równolegle.
TODO: odwołanie do rozdziału opisującego metodykę pracy
oraz do odpowiednich pozycji w bibliografii.

Warto podkreślić, że przez cały czas trwania projektu członkowie zespołu
poświęcali znaczącą część czasu na wzajemne przeglądy kodu
(\gls{review-kodu}) oraz ciągły feedback dotyczący implementacji
poszczególnych funkcjonalności (zarówno w obrębie backendu,
jak i frontendu).
Taki sposób pracy pozwolił na szybkie wychwytywanie błędów,
ujednolicenie stylu implementacji oraz bieżące korygowanie
wymagań funkcjonalnych.
TODO: odwołanie do rozdziału dotyczącego testów i jakości
oprogramowania.

Dla przejrzystości opisu, w ramach każdego miesiąca przedstawiono
zadania w podziale na poszczególnych członków zespołu.
Najpierw podawane jest krótkie podsumowanie prac w danym miesiącu,
a następnie – w postaci list punktowanych – wyszczególniono działania
zrealizowane przez konkretne osoby.

\newenvironment{zadaniaosoby}[1]{%
    \par\medskip
    \noindent\textbf{#1}\par
    \vspace{-0.4\baselineskip}
    \begin{itemize}
}{%
    \end{itemize}
}

\section{Faza przedprojektowa (czerwiec–wrzesień 2024)}
\label{sec:faza-przedprojektowa}

Faza przedprojektowa obejmowała okres od czerwca do września 2024 roku
i poprzedzała formalne zatwierdzenie tematu pracy oraz opracowanie
harmonogramu przedstawionego w rozdziale planistycznym.
TODO: odwołanie do rozdziału z harmonogramem projektu.

Zespół rozpoczął prace deweloperskie już w czerwcu 2024 roku,
co było możliwe dzięki temu, że niezależnie od ostatecznego tematu
zakładano stworzenie aplikacji internetowej wymagającej kont
użytkowników.
Wynikało to bezpośrednio ze specyfiki specjalizacji
„Aplikacje Internetowe”, na której studiują wszyscy członkowie zespołu.
W efekcie część prac technicznych została wykonana jeszcze przed
formalnym wyborem tematu projektu.
TODO: ewentualne odwołanie do opisu kontekstu studiów
w rozdziale wstępnym.

\subsection*{Czerwiec 2024}

W czerwcu 2024 roku wykonano następujące działania przygotowawcze:

\begin{zadaniaosoby}{Adam Langmesser}
    \item Utworzenie repozytorium w serwisie \texttt{GitHub}
    i skonfigurowanie podstawowych reguł pracy z repozytorium.
    \item Przygotowanie pierwszego potoku \gls{cicd} dla backendu –
    automatyczne budowanie projektu i uruchamianie testów jednostkowych
    na serwerze ciągłej integracji.
\end{zadaniaosoby}

\begin{zadaniaosoby}{Kacper Badek}
    \item Konfiguracja narzędzia \texttt{Jira}:
    zapoznanie się z typowym podziałem na epiki, taski i podtaski,
    zdefiniowanie statusów przepływu pracy oraz utworzenie
    \gls{tablica_kanban}.
\end{zadaniaosoby}

\begin{zadaniaosoby}{Cały zespół}
    \item Przygotowanie szkieletu projektu backendu w technologii
    Spring Boot – utworzono podstawowe pakiety, konfigurację aplikacji
    oraz minimalną strukturę modułów.
\end{zadaniaosoby}

\subsection*{Lipiec 2024}

W lipcu 2024 roku skupiono się na warstwie frontendowej
i pierwszych funkcjonalnościach związanych z kontami użytkowników:

\begin{zadaniaosoby}{Adam Langmesser}
    \item Rozpoczęcie prac nad potokiem \gls{cicd} dla frontendu,
    którego konfigurację finalnie ukończono we wrześniu 2024 roku.
    \item Implementacja podstawowej logiki logowania
    i rejestracji użytkownika zarówno po stronie backendu
    (endpointy \gls{rest_api}), jak i frontendu
    (formularze oraz obsługa żądań).
\end{zadaniaosoby}

\begin{zadaniaosoby}{Cały zespół}
    \item Przygotowanie szkieletu aplikacji frontendowej
    (\gls{react} + \gls{type-script}) wraz z podstawową strukturą
    komponentów i konfiguracją narzędzi budujących.
\end{zadaniaosoby}

\subsection*{Sierpień 2024}

W sierpniu 2024 roku zespół skupił się na zagadnieniach bezpieczeństwa
oraz na dopracowaniu pierwszych ekranów aplikacji:

\begin{zadaniaosoby}{Adam Langmesser}
    \item Dodanie Spring Security i implementacja logiki
    uwierzytelniania oraz autoryzacji użytkownika po stronie backendu.
\end{zadaniaosoby}

\begin{zadaniaosoby}{Mateusz Redosz}
    \item Konfiguracja biblioteki Tailwind \gls{css} na froncie,
    umożliwiająca spójne i responsywne stylowanie komponentów.
    \item Konfiguracja narzędzia \texttt{Prettier}
    do automatycznego formatowania kodu frontendu.
    \item Stworzenie formularza rejestracji użytkownika
    wraz z podstawową walidacją po stronie frontendu.
    \item Dodanie i konfiguracja biblioteki TanStack Query
    do obsługi komunikacji z backendem i cachowania danych.
\end{zadaniaosoby}

\begin{zadaniaosoby}{Stanisław Oziemczuk}
    \item Implementacja logiki resetowania hasła
    (proces „zapomniałem hasła”) po stronie frontendu.
\end{zadaniaosoby}

\begin{zadaniaosoby}{Kacper Badek}
    \item Implementacja strony logowania użytkownika
    po stronie frontendu.
\end{zadaniaosoby}

\subsection*{Wrzesień 2024}

We wrześniu 2024 roku kontynuowano prace nad bezpieczeństwem
i przygotowaniem środowiska uruchomieniowego:

\begin{zadaniaosoby}{Adam Langmesser}
    \item Zastąpienie bazy danych działającej w pamięci (in-memory)
    instancją uruchamianą w kontenerze \texttt{Docker},
    co urealniło środowisko deweloperskie i testowe.
    \item Dostosowanie potoku \gls{cicd} backendu tak,
    aby uwzględniał uruchamianie bazy danych w kontenerze
    podczas wykonywania testów.
    \item Dokończenie konfiguracji potoku \gls{cicd} dla frontendu.
\end{zadaniaosoby}

\begin{zadaniaosoby}{Kacper Badek}
    \item Implementacja logiki resetowania hasła po stronie backendu –
    generowanie tokenów, ich weryfikacja oraz integracja
    z istniejącym procesem resetowania na froncie.
\end{zadaniaosoby}

\begin{zadaniaosoby}{Zespół frontendowy}
    \item Implementacja przycisków na stronie logowania,
    umożliwiających logowanie i rejestrację z wykorzystaniem
    \gls{oauth} (GitHub i Google) po stronie frontendu.
    TODO: doprecyzować osobę odpowiedzialną za implementację.
\end{zadaniaosoby}

\section{Etap 1 (październik 2024 – styczeń 2025)}
\label{sec:etap1}

Etap 1 był przede wszystkim poświęcony dopracowaniu wymagań wstępnych,
stabilizacji modułu uwierzytelniania oraz pierwszym eksperymentom
z mapą spotów.
W tym okresie zespół łączył prace deweloperskie z działaniami
analitycznymi i planistycznymi (opracowanie harmonogramu, założeń
oraz wymagań w ramach przedmiotów projektowych na uczelni).

\subsection*{Październik 2024}

\begin{zadaniaosoby}{Mateusz Redosz}
    \item Poprawa konfiguracji CORS na backendzie, tak aby aplikacja
    frontendowa mogła komunikować się z serwerem w sposób bezpieczny
    i zgodny z przeglądarkowymi ograniczeniami.
    \item Zmiana sposobu przechowywania tokena \gls{jwt} – umieszczenie go
    w ciasteczku \texttt{HttpOnly}, co poprawiło bezpieczeństwo
    aplikacji.
\end{zadaniaosoby}

\begin{zadaniaosoby}{Stanisław Oziemczuk}
    \item Implementacja logowania \gls{oauth} z wykorzystaniem Google
    i GitHub po stronie backendu oraz integracja z frontem.
\end{zadaniaosoby}

\begin{zadaniaosoby}{Kacper Badek}
    \item Uporządkowanie pliku \texttt{.gitignore}
    oraz struktury repozytorium.
\end{zadaniaosoby}

\begin{zadaniaosoby}{Cały zespół}
    \item Formalny wybór tematu projektu inżynierskiego.
\end{zadaniaosoby}

\subsection*{Listopad 2024}

\begin{zadaniaosoby}{Adam Langmesser}
    \item Implementacja demonstracyjnej mapy z wykorzystaniem
    biblioteki Leaflet – prototyp miał na celu pokazanie zespołowi
    możliwości interaktywnej mapy.
    Ze względu na ograniczone możliwości customizacji wyglądu
    biblioteki zdecydowano się później na zmianę dostawcy
    kafelków mapowych na usługę
    TODO: uzupełnić nazwę docelowej usługi mapowej.
    \item Poprawki konfiguracji narzędzia \texttt{ESLint}
    po stronie frontendu.
\end{zadaniaosoby}

\begin{zadaniaosoby}{Mateusz Redosz}
    \item Dalsze poprawki obsługi JWT na backendzie oraz logowania
    błędów związanych z procesem logowania i rejestracji.
\end{zadaniaosoby}

\begin{zadaniaosoby}{Stanisław Oziemczuk}
    \item Poprawki działania logowania użytkownika po stronie backendu
    i frontendu, obejmujące obsługę błędów oraz komunikaty
    dla użytkownika.
\end{zadaniaosoby}

\begin{zadaniaosoby}{Kacper Badek}
    \item Implementacja logiki cyklicznego usuwania przeterminowanych
    tokenów resetu hasła.
\end{zadaniaosoby}

\begin{zadaniaosoby}{Cały zespół}
    \item Opracowanie harmonogramu projektu w formie opisowej
    oraz w postaci wykresu Gantta.
    \item Przygotowanie wstępnych założeń i wymagań
    w ramach przedmiotu PRO.
\end{zadaniaosoby}

TODO: odwołanie do rozdziału z harmonogramem projektu.\\
TODO: odwołanie do rozdziału z analizą wymagań.

\subsection*{Grudzień 2024}

\begin{zadaniaosoby}{Adam Langmesser}
    \item Dalsze poprawki konfiguracji Spring Security na backendzie,
    w tym doprecyzowanie ról i uprawnień.
    \item Rozszerzenie encji użytkownika o dane deweloperskie
    oraz przygotowanie inicjalnych danych w bazie
    (np. konta testowe).
    \item Implementacja testów automatycznych związanych
    z bezpieczeństwem (scenariusze logowania i rejestracji
    użytkownika po stronie backendu).
\end{zadaniaosoby}

\begin{zadaniaosoby}{Mateusz Redosz}
    \item Dodanie biblioteki \gls{redux} do frontendu i wstępna
    konfiguracja store.
    \item Implementacja automatycznego wylogowywania użytkownika
    po wygaśnięciu tokena JWT.
    \item Stworzenie komponentu odpowiedzialnego za prezentację błędów
    systemowych użytkownikowi (globalny mechanizm powiadomień).
\end{zadaniaosoby}

\begin{zadaniaosoby}{Stanisław Oziemczuk}
    \item Stworzenie komponentu frontendu odpowiedzialnego
    za wyświetlanie szczegółów pojedynczego spota.
\end{zadaniaosoby}

\begin{zadaniaosoby}{Kacper Badek}
    \item Poprawa logowania błędów związanych z resetowaniem hasła
    użytkownika po stronie backendu.
    \item Dostosowanie wyglądu i treści wiadomości e-mail wysyłanych
    przez system (np. w procesie resetu hasła).
\end{zadaniaosoby}

\subsection*{Styczeń 2025}

\begin{zadaniaosoby}{Adam Langmesser}
    \item Poprawa testów logowania i rejestracji na backendzie
    oraz ujednolicenie asercji.
\end{zadaniaosoby}

\begin{zadaniaosoby}{Mateusz Redosz}
    \item Rozszerzenie stanu \gls{redux} o informację o tym,
    czy użytkownik jest aktualnie zalogowany.
    \item Implementacja testów E2E dla procesów logowania
    i rejestracji użytkownika.
    \item Dodanie uruchamiania testów frontendu do potoku \gls{cicd}.
    \item Poprawa sposobu wyświetlania szczegółów spota na froncie.
\end{zadaniaosoby}

\begin{zadaniaosoby}{Stanisław Oziemczuk}
    \item Poprawki logowania i rejestracji z wykorzystaniem \gls{oauth}
    (Google/GitHub) – dopracowanie scenariuszy brzegowych.
    \item Implementacja logiki filtrowania spotów na mapie
    po różnych kryteriach (np. nazwa) po stronie backendu.
    \item Dalsze dopracowanie logiki filtrowania spotów po nazwie
    po stronie backendu.
\end{zadaniaosoby}

\begin{zadaniaosoby}{Kacper Badek}
    \item Dodanie danych deweloperskich dla mapy
    (przykładowe spoty, dane do prezentacji i testów).
    \item Implementacja logiki dodawania spota do ulubionych
    po stronie backendu.
\end{zadaniaosoby}

\section{Etap 2 (luty 2025 – wrzesień 2025)}
\label{sec:etap2}

Etap 2 obejmował zasadniczą część prac deweloperskich nad aplikacją.
W tym okresie rozwijano kolejne moduły (mapa, forum, czat,
panel użytkownika), równolegle doprecyzowując dokumentację oraz
weryfikując i aktualizując wymagania na podstawie bieżących
doświadczeń zespołu.
TODO: odwołanie do rozdziału z analizą wymagań
oraz do rozdziału opisującego projekt architektury.

Równolegle, w ramach przedmiotu PRZ~1 opracowano projekt interfejsu
użytkownika, konsultowany z prowadzącym zajęcia, mgr inż. Adamem
Urbanowiczem, który na bieżąco przekazywał zespołowi uwagi i
rekomendacje dotyczące ergonomii oraz spójności interfejsu
z założeniami funkcjonalnymi.

\subsection*{Luty 2025}

\begin{zadaniaosoby}{Mateusz Redosz}
    \item Poprawa wyglądu strony logowania, w tym dopracowanie stylistyki
    komponentów oraz zachowania w trybie ciemnym i jasnym.
    \item Poprawa logiki wylogowywania użytkownika na froncie
    (m.in. czyszczenie stanu \gls{redux}, przekierowania).
    \item Poprawa logiki dodawania komentarzy do spotów na mapie
    po stronie frontendu.
\end{zadaniaosoby}

\begin{zadaniaosoby}{Stanisław Oziemczuk}
    \item Implementacja logiki dodawania spota do ulubionych
    po stronie frontendu oraz integracja z backendem.
    \item Implementacja integracji z zewnętrznym API pogodowym
    służącym do pobierania podstawowych danych pogodowych
    dla danego spota.
\end{zadaniaosoby}

TODO: uzupełnić nazwę i dostawcę API pogodowego
oraz odwołać się do kart usług zewnętrznych i bibliografii.

\subsection*{Marzec 2025}

TODO: do uzupełnienia – opis prac wykonanych w marcu 2025 roku
(np. rozwój forum, pierwsza wersja czatu, dalsze prace nad mapą).

\subsection*{Kwiecień 2025}

TODO: do uzupełnienia – opis prac wykonanych w kwietniu 2025 roku.

\subsection*{Maj 2025}

TODO: do uzupełnienia – opis prac wykonanych w maju 2025 roku.

\subsection*{Czerwiec 2025}

TODO: do uzupełnienia – opis prac wykonanych w czerwcu 2025 roku.

\subsection*{Lipiec 2025}

TODO: do uzupełnienia – opis prac wykonanych w lipcu 2025 roku.

\subsection*{Sierpień 2025}

TODO: do uzupełnienia – opis prac wykonanych w sierpniu 2025 roku.

\subsection*{Wrzesień 2025}

We wrześniu 2025 roku skoncentrowano się na dalszym rozwijaniu
funkcjonalności wyszukiwania spotów, modułu pogody, systemu
powiadomień, części społecznościowej oraz czatu, a także na
uporządkowaniu struktury dokumentacji pracy inżynierskiej.

\begin{zadaniaosoby}{Adam Langmesser}
    \item Wprowadzenie możliwości rozpoczynania lub kontynuowania
    prywatnych rozmów czatowych bezpośrednio z list znajomych
    i obserwujących poprzez dodanie przycisku
    \textit{„Wiadomość”} na kartach społecznościowych oraz integrację
    z logiką wyszukiwania i tworzenia czatów prywatnych.
    \item Dodanie nowoczesnej obsługi emoji w komunikatorze – integracja
    komponentu wyboru emoji z polem wpisywania wiadomości oraz
    uporządkowanie wyglądu okna wyboru animowanych obrazów (GIF),
    co poprawiło ergonomię korzystania z czatu.
\end{zadaniaosoby}

\begin{zadaniaosoby}{Mateusz Redosz}
    \item Refaktoryzacja systemu powiadomień na froncie, umożliwiająca
    jednoczesne wyświetlanie wielu komunikatów oraz zwiększająca
    modularność i możliwość ponownego wykorzystania komponentów
    powiadomień.
    \item Wprowadzenie stronicowania (paginacji) dla endpointów
    wyszukiwania spotów na backendzie oraz dostosowanie komponentów
    stron głównych i list wyników wyszukiwania do pracy
    w trybie nieskończonego przewijania (\gls{infinite-scroll})
    z mechanizmem \textit{„load more”}, z wykorzystaniem obserwatora
    przecięcia oraz przekazywaniem referencji i stanów ładowania.
    \item Dodanie walidacji formularza dodawania spota
    w oparciu o schemat walidacyjny, w tym sprawdzanie kompletności
    danych, wymaganego zestawu multimediów oraz punktów poligonu,
    a także wyświetlanie komunikatów błędów bezpośrednio pod sekcjami
    mediów i obszaru na mapie.
    \item Rozszerzenie komponentu przesyłania multimediów o podgląd
    wybranych obrazów i materiałów wideo oraz poprawę zarządzania
    cyklem życia adresów URL podglądu; jednocześnie podniesiono
    priorytet wyświetlania listy powiadomień (warstwa \textit{z-index}),
    aby były lepiej widoczne.
    \item Wprowadzenie pełnej obsługi zmiany zdjęcia profilowego
    użytkownika, obejmującej wysyłanie plików do backendu, usuwanie
    poprzednich zdjęć z magazynu obiektowego oraz aktualizację adresu
    zdjęcia w profilu użytkownika.
    \item Refaktoryzacja struktury projektu \LaTeX: zastąpienie
    dotychczasowej treści demonstracyjnej rzeczywistymi rozdziałami
    opisującymi projekt \textit{„spoty-na-drony.pl”}, aktualizacja
    metadanych (tytuł, autorzy, promotor, cele projektu) oraz dodanie
    osobnych plików rozdziałów dla głównych części pracy
    (wstęp, opis problemu, kontekst projektu, analiza wymagań,
    decyzje projektowe, projekt, planowanie, implementacja, testowanie,
    prezentacja systemu, podsumowanie) i włączenie ich do pliku głównego.
    \item Rozszerzenie zaawansowanych możliwości wyszukiwania spotów
    o sortowanie wyników według popularności lub oceny oraz filtrowanie
    po minimalnej ocenie, wraz z odpowiednimi listami rozwijanymi
    w interfejsie użytkownika i obsługą nowych parametrów
    po stronie backendu.
\end{zadaniaosoby}

\begin{zadaniaosoby}{Stanisław Oziemczuk}
    \item Implementacja kompletnej funkcjonalności pogody dla spotów:
    dodanie podstawowego i szczegółowego modalu pogodowego oraz
    zestawu komponentów interfejsu prezentujących m.in. temperaturę,
    prędkość wiatru, opady oraz dodatkowe parametry meteorologiczne.
    \item Przebudowa obsługi pogody tak, aby zamiast bezpośrednich
    wywołań publicznego API wykorzystywany był backend jako warstwa
    pośrednia, z wprowadzeniem dedykowanych struktur DTO oraz
    dostosowaniem komponentów prezentujących dane pogodowe
    do nowego modelu danych i wymagań responsywności.
\end{zadaniaosoby}

\section{Etap 3 (październik 2025 – styczeń 2026)}
\label{sec:etap3}

Etap 3 obejmował finalizację prac nad systemem,
dopracowanie dokumentacji technicznej i tekstowej pracy inżynierskiej
oraz przygotowanie projektu do oddania.
W tym czasie większość nowych funkcjonalności była już zaimplementowana,
a nacisk położono na stabilizację, testy oraz spójny opis
w dokumentacji.

W ramach przedmiotu PSEM, prowadzonego przez dr. inż. Marka
Bednarczyka, postępy w przygotowywaniu dokumentacji były na bieżąco
konsultowane, a na podstawie uzyskiwanej informacji zwrotnej wprowadzano
kolejne poprawki i uzupełnienia.
Analogiczny tryb pracy przyjęto w ramach przedmiotu PRZ~2, którego
opiekunem był promotor pracy, mgr inż. Adam Urbanowicz.

\subsection*{Październik 2025}

\begin{zadaniaosoby}{Adam Langmesser}
    \item Wprowadzenie możliwości tworzenia czatów grupowych,
    w tym obsługi wyboru uczestników, komunikacji z backendem
    oraz integracji z istniejącą listą czatów.
    \item Dodanie funkcjonalności wysyłania i wyświetlania załączników
    w wiadomościach czatu (pliki oraz obrazy), wraz z logiką wyboru,
    podglądu i wysyłania samych plików bez treści tekstowej.
    \item Rozszerzenie istniejącej funkcjonalności czatów grupowych
    o możliwość edycji ich parametrów (zmiana nazwy oraz obrazu czatu)
    po stronie backendu i frontendu.
    \item Dodanie możliwości dołączania nowych użytkowników
    do istniejących czatów grupowych, wraz z odpowiednimi
    endpointami HTTP i modyfikacją interfejsu użytkownika.
\end{zadaniaosoby}

\begin{zadaniaosoby}{Mateusz Redosz}
    \item Rozbudowa systemu statusów znajomych w części społecznościowej
    aplikacji, w tym rozróżnienie zaproszeń wysłanych, otrzymanych
    oraz relacji zakończonych, a także dostosowanie interfejsu
    do prezentacji odpowiednich komunikatów i akcji.
    \item Wprowadzenie zaawansowanego zarządzania zaproszeniami
    do znajomych: dodanie modalnego widoku listy zaproszeń, obsługi
    akceptowania i odrzucania oraz integracji z dedykowanymi
    endpointami backendowymi.
    \item Dodanie funkcji \textit{„Dodaj znajomego”} w sekcji
    społecznościowej, obejmującej wyszukiwarkę użytkowników
    (paginacja, wyszukiwanie po nazwie użytkownika) oraz spójny
    wygląd modalnego okna wyszukiwania.
    \item Rozbudowa komponentu przycisku przesyłania plików
    o możliwość podglądu wielu plików, nadawania im unikalnych
    identyfikatorów oraz usuwania pojedynczych plików przed wysłaniem.
    \item Przebudowa struktury rozdziału \textit{Implementacja}:
    wydzielenie osobnych sekcji dla backendu, frontendu i CI/CD,
    uzupełnienie dokumentacji backendu o listę endpointów z przykładami
    odpowiedzi oraz przygotowanie miejsc na opis implementacji
    pozostałych części.
    \item Przygotowanie struktury rozdziału \textit{Analiza wymagań},
    w tym włączenie plików dotyczących przypadków użycia, wymagań
    funkcjonalnych i pozafunkcjonalnych oraz wymagań dotyczących
    środowiska docelowego.
    \item Przygotowanie struktury rozdziału  \textit{Nakład pracy} wraz
    z podrozdziałami opisującymi indywidualny wkład każdego członka
    zespołu oraz włączenie go do głównej struktury dokumentu.
    \item Przebudowa i uporządkowanie struktury dokumentacji
    (m.in. „Ogólny nakład pracy”, „Indywidualne nakłady pracy”
    oraz „Aspekty społeczne i biznesowe”) poprzez wydzielenie ich
    do osobnych plików, usunięcie przestarzałych fragmentów
    oraz ujednolicenie poziomów nagłówków i oznaczeń sekcji.
\end{zadaniaosoby}

\begin{zadaniaosoby}{Stanisław Oziemczuk}
    \item Wprowadzenie rozszerzonej galerii multimediów dla spotów,
    obejmującej obsługę paginacji, podglądu w trybie pełnoekranowym
    oraz dodatkowych akcji (np. udostępnianie odnośnika do zasobu).
    \item Dostosowanie modułu mapy oraz widoku szczegółów spotów
    pod kątem responsywności i skalowania na dużych ekranach,
    w tym korekty wysokości komponentów, układu galerii multimediów
    oraz przewijania list komentarzy.
\end{zadaniaosoby}

\begin{zadaniaosoby}{Kacper Badek}
    \item Przeprowadzenie dużej refaktoryzacji funkcjonalności forum,
    obejmującej obsługę komentarzy do postów (dodawanie, edycję,
    usuwanie oraz głosowanie) po stronie backendu i frontendu.
    \item Przebudowa stron forum w kierunku architektury
    z nieskończonym przewijaniem (\gls{infinite-scroll}),
    z wykorzystaniem mechanizmów stronicowania i sortowania postów
    po stronie klienta i serwera.
    \item Zastąpienie klasycznych wskaźników ładowania (spinnerów)
    loaderami typu „skeleton” dla listy postów oraz paneli
    kategorii i tagów, co poprawiło odbiór interfejsu w trakcie
    ładowania danych.
\end{zadaniaosoby}

\subsection*{Listopad 2025}

\begin{zadaniaosoby}{Adam Langmesser}
    \item Opracowanie rozdziału \textit{Analiza wymagań} – przygotowanie
    listy aktorów systemu, diagramu przypadków użycia oraz scenariuszy
    przypadków użycia.
\end{zadaniaosoby}

TODO: odwołanie do odpowiedniego rozdziału i podrozdziałów
poświęconych analizie wymagań.\\
TODO: dokończenie

\subsection*{Grudzień 2025}

\begin{zadaniaosoby}{Adam Langmesser}
    \item Opracowanie podrozdziału poświęconego wymaganiom
    dla modułu czatu, w tym wymagań funkcjonalnych
    i pozafunkcjonalnych.
    \item Dopracowanie kart usług zewnętrznych (m.in. usług mapowych,
    pogodowych, poczty e-mail) wykorzystywanych przez system.
\end{zadaniaosoby}

TODO: odwołanie do numeru podrozdziału z wymaganiami dla czatu.\\
TODO: odwołanie do sekcji z kartami usług zewnętrznych.\\
TODO: dokończenie

\subsection*{Styczeń 2026}

TODO: do uzupełnienia

Na zakończenie prac projektowych przyjęto datę
\textbf{10 stycznia 2026 roku}.
