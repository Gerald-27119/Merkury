%! Author = Adam
%! Date = 10/01/2025

% Informacja, kto jaki rozdział książki napisał ma być w rozdziale o pracy indywidualnej, tutaj nie. Komentarz informacyjny.
\chapter{Przebieg realizacji projektu}
\label{ch:przebieg-realizacji-projektu}

W niniejszym rozdziale przedstawiono przebieg realizacji projektu w ujęciu funkcjonalnym,
z podziałem na główne moduły aplikacji. Opis koncentruje się na tym, jakie
elementy systemu powstawały i jak ewoluowały w czasie. Taki sposób prezentacji odzwierciedla
iteracyjny charakter realizacji zgodny z metodyką \gls{DAD_LLC}.

Prace nad modułami były prowadzone równolegle, a rozwój funkcjonalności był wspierany
przez stałe \glslink{review-kodu}{review kodu}, automatyzację testów oraz rozwój infrastruktury.

\section{Elementy przekrojowe (niezależne od modułu)}
\label{sec:przekrojowe}

W całym okresie realizacji rozwijano również elementy wspólne dla całego systemu, które
stanowiły fundament dla kolejnych modułów:

\begin{itemize}
    \item \textbf{Stos technologiczny i struktura projektu} -- doprecyzowanie architektury rozwiązania oraz organizacji repozytorium dla \gls{backend} i \gls{frontend}.
    \item \textbf{Podstawy warstwy klienckiej} -- budowa fundamentów aplikacji \glslink{frontend}{frontendowej} (routing, konfiguracja stylów, porządkowanie struktury projektu), a także wprowadzenie bibliotek wspierających rozwój (np. \gls{tanstack-query}, \gls{redux}).
    \item \textbf{Uwierzytelnianie i bezpieczeństwo} -- rozwój logowania i rejestracji, obsługa \gls{jwt} (w tym przechowywanie w \glslink{http-only-cookie}{HttpOnly}), konfiguracja \gls{cors}, doprecyzowanie ról i uprawnień w \gls{spring-security}, a także logowanie \gls{oauth} (Google/\gls{github}).
    \item \textbf{Odzyskiwanie dostępu do konta} -- wdrożenie procesu resetowania hasła (generowanie i weryfikacja tokenów resetu, formularz zmiany hasła) oraz mechanizmy utrzymaniowe, np. cykliczne usuwanie przeterminowanych tokenów.
    \item \textbf{Komunikacja e-mail} -- integracja z zewnętrzną usługą wysyłania wiadomości (np. e-mail powitalny i wiadomości resetu hasła), dopracowanie szablonów HTML oraz obsługa błędów po stronie serwera; w trakcie projektu dostosowywano również sposób dostawy tej usługi.
    \item \textbf{Środowisko uruchomieniowe} -- konteneryzacja (\gls{docker-compose}), dopracowanie plików \gls{dockerfile} oraz skryptów ułatwiających uruchomienie środowiska deweloperskiego.
    \item \textbf{Jakość i automatyzacja} -- rozwój testów (integracyjnych, jednostkowych i E2E tam, gdzie zasadne), uruchamianie testów w potokach \gls{cicd}, ujednolicenie formatowania (m.in. \gls{prettier}) i kontrola jakości w repozytorium (np. \gls{eslint}).
    \item \textbf{Wydajność i obserwowalność} -- dopracowanie logowania błędów oraz wprowadzenie mechanizmów optymalizacji (m.in. cache w postaci \gls{redis}).
    \item \textbf{Dane deweloperskie} -- przygotowanie danych testowych wykorzystywanych do demonstracji i weryfikacji działania modułów (np. przykładowi użytkownicy i \glslink{spot}{spoty}).
    \item \textbf{Dokumentacja} -- uruchomienie projektu dokumentacji w \gls{latex} (bibliografia, struktura rozdziałów) oraz iteracyjne uzupełnianie treści równolegle do rozwoju systemu.
\end{itemize}

Przed rozpoczęciem implementacji modułów przygotowano projekt interfejsu użytkownika,
obejmujący kluczowe widoki oraz założenia nawigacji i ergonomii. Stanowił on punkt odniesienia
dla dalszych prac i ułatwił iteracyjne dopasowywanie funkcjonalności do sposobu użycia systemu.

\setcounter{secnumdepth}{4}
\setcounter{tocdepth}{4}

\section{Moduły aplikacji}
\label{sec:moduly-aplikacji}

\subsection{Rozwój funkcjonalności w modułach}
\label{subsec:rozwoj-modulow}

Poniżej przedstawiono rozwój funkcjonalności w kluczowych modułach aplikacji.
Każdy moduł rozwijano iteracyjnie, stopniowo rozszerzając jego zakres oraz dopracowując
aspekty ergonomii i spójności z pozostałymi elementami systemu.

\subsubsection{Mapa}
\label{subsubsec:modul-mapa}

Moduł mapy stanowił jeden z kluczowych elementów aplikacji. Jego rozwój przebiegał od
prototypu demonstracyjnego do docelowego rozwiązania obejmującego interakcje użytkownika,
widok szczegółów oraz integrację z pozostałymi funkcjami.

\begin{itemize}
    \item \textbf{Prototyp mapy} -- przygotowanie wstępnej wersji mapy (proof-of-concept) w celu weryfikacji możliwości interaktywnej prezentacji danych.
    \item \textbf{Migracja i stabilizacja rozwiązania docelowego} -- przejście na \gls{react-maplibre} oraz ponowna implementacja podstaw wyświetlania \glslink{spot}{spotów} i interakcji użytkownika.
    \item \textbf{Ergonomia mapy} -- dopracowanie interakcji z mapą (m.in. obsługa znaczników i zachowanie mapy w typowych scenariuszach użytkownika) oraz aktualizacja sposobu prezentacji lokalizacji użytkownika.
    \item \textbf{Model \glslink{spot}{spota}} -- rozszerzanie modelu o dane lokalizacyjne i statystyczne (np. kraj/miasto, tagi, liczniki ocen) oraz ujednolicenie sposobu wyznaczania punktu reprezentującego \glslink{spot}{spota}.
    \item \textbf{Widok szczegółów \glslink{spot}{spota}} -- rozwój prezentacji informacji o \glslink{spot}{spocie} oraz dopracowanie ergonomii interfejsu.
    \item \textbf{Komentarze i oceny} -- rozwój mechanizmów komentarzy (w tym refaktoryzacja modelu komentarzy), dopracowanie prezentacji ocen oraz rozszerzenia UI (np. prezentacja zdjęcia profilowego autora komentarza).
    \item \textbf{Media dla \glslink{spot}{spotów}} -- ujednolicenie obsługi zdjęć i filmów oraz implementacja prezentacji multimediów w widoku \glslink{spot}{spota}.
    \item \textbf{Galeria multimediów} -- wprowadzenie rozszerzonej galerii (paginacja, tryb pełnoekranowy, dodatkowe akcje, np. udostępnianie odnośnika).
    \item \textbf{Ulubione \glslink{spot}{spoty}} -- możliwość dodawania \glslink{spot}{spotów} do ulubionych oraz integracja tej funkcji z panelem użytkownika i mapą.
    \item \textbf{Responsywność} -- dostosowanie modułu mapy i widoku szczegółów \glslink{spot}{spotów} do różnych rozmiarów ekranu, w tym skalowanie na dużych ekranach.
    \item \textbf{Porządkowanie modułu} -- refaktoryzacja struktury plików \glslink{backend}{backendu} związanych z modułem mapy w miarę rozrastania się funkcji.
\end{itemize}

\paragraph{Pogoda dla \glslink{spot}{spotów}}
\label{par:pogoda-spoty}

Funkcjonalność pogody rozwijano jako element wspierający korzystanie z mapy i widoku \glslink{spot}{spotów}.

\begin{itemize}
    \item \textbf{Integracja podstawowa} -- pobieranie danych pogodowych dla wybranego \glslink{spot}{spota} oraz przygotowanie panelu pogodowego w interfejsie.
    \item \textbf{Rozszerzenie funkcjonalności} -- wprowadzenie pogody podstawowej i szczegółowej oraz dopracowanie sposobu prezentacji danych (np. w formie \gls{modal}u z zestawem parametrów meteorologicznych).
    \item \textbf{Warstwa pośrednia w \gls{backend}zie} -- przebudowa integracji tak, aby dane pogodowe były pobierane przez \glslink{backend}{backend} (zamiast bezpośrednich wywołań z \glslink{frontend}{frontendu}).
    \item \textbf{Dopasowanie UI} -- poprawa responsywności i ergonomii panelu pogodowego.
\end{itemize}

\subsubsection{Czat}
\label{subsubsec:modul-czat}

Czat był rozwijany iteracyjnie: od podstaw komunikacji do rozbudowanych scenariuszy
(rozmowy prywatne i grupowe, multimedia, usprawnienia wydajności i ergonomii).

\begin{itemize}
    \item \textbf{Podstawy komunikacji} -- implementacja fundamentów czatu w \glslink{backend}{backendzie} i \glslink{frontend}{frontendzie} oraz przygotowanie UI.
    \item \textbf{Komunikacja w czasie rzeczywistym} -- wdrożenie \gls{websocket} do pobierania i wysyłania wiadomości; przygotowanie wspólnych komponentów/ułatwień po stronie frontendu dla obsługi websocketów.
    \item \textbf{Usprawnienia UX} -- grupowanie wiadomości po dacie, wielowierszowe pole tekstowe, usprawnienia listy czatów oraz mechanizmu nieskończonego przewijania.
    \item \textbf{Emoji i GIF} -- dodanie obsługi emoji oraz wysyłania animowanych obrazów (\glslink{gif}{GIF}).
    \item \textbf{Wydajność i niezawodność} -- optymistyczne wysyłanie wiadomości oraz endpointy do stronicowanego pobierania starszych wiadomości.
    \item \textbf{Rozmowy prywatne} -- możliwość rozpoczęcia i kontynuowania rozmów bezpośrednio z list relacji społecznościowych.
    \item \textbf{Czaty grupowe} -- tworzenie czatów grupowych, edycja nazwy/obrazu oraz dołączanie kolejnych uczestników.
    \item \textbf{Załączniki} -- wysyłanie i wyświetlanie plików oraz obrazów, wraz z obsługą wyboru, podglądu i wysyłania samych plików bez treści tekstowej.
\end{itemize}

\subsubsection{Forum}
\label{subsubsec:modul-forum}

Forum rozwijano jako moduł publikacji treści i dyskusji, uzupełniający funkcjonalności mapy i społeczności.

\begin{itemize}
    \item \textbf{Podstawowa struktura forum} -- obsługa postów, kategorii, tagów oraz paginacji.
    \item \textbf{Edytor treści} -- integracja edytora rich-text na potrzeby tworzenia postów; w toku rozwoju modułu dopracowano i zmieniano zastosowane rozwiązanie edytora (np. \gls{tinymce} $\rightarrow$ \gls{tiptap}).
    \item \textbf{Sanityzacja / przetwarzanie treści} -- wprowadzenie przetwarzania HTML po stronie \glslink{backend}{backendu} (np. z użyciem \gls{jsoup}) w celu kontroli i porządkowania treści publikowanych przez użytkowników.
    \item \textbf{Media w postach} -- integracja z \gls{azure-blob-storage} do przesyłania i przechowywania mediów.
    \item \textbf{Przeglądanie i sortowanie} -- rozwój listy postów i mechanizmów sortowania.
    \item \textbf{Czytelne adresy i walidacja} -- wprowadzenie czytelniejszych adresów \gls{url} opartych na \glslink{slug}{slugach} oraz dopracowanie walidacji i ergonomii formularza dodawania postów.
    \item \textbf{Interakcje społecznościowe} -- komentarze do postów (dodawanie/edycja/usuwanie/głosowanie), możliwość zgłaszania treści oraz obserwowanie postów.
    \item \textbf{Usprawnienia UX} -- uspójnienie zachowania formularzy, a także poprawa sposobu komunikowania stanu ładowania (np. \gls{skeleton-loader} w listach i panelach).
\end{itemize}

\subsubsection{Wyszukiwarka \glslink{spot}{spotów}}
\label{subsubsec:modul-wyszukiwarka}

Wyszukiwanie \glslink{spot}{spotów} rozwijano stopniowo, przechodząc od prostych filtrów do
zaawansowanych funkcji wspierających odkrywanie miejsc przez użytkowników.

\begin{itemize}
    \item \textbf{Filtrowanie podstawowe} -- pierwsze kryteria wyszukiwania, w tym filtrowanie po nazwie.
    \item \textbf{Nowy pasek wyszukiwania} -- wdrożenie wyszukiwania po nazwie z panelem wyników oraz dopracowanie interfejsu (m.in. prezentacja ocen).
    \item \textbf{Wyszukiwanie w obszarze mapy} -- dodanie możliwości wyszukiwania \glslink{spot}{spotów} w aktualnie widocznym obszarze mapy.
    \item \textbf{Wyszukiwanie po lokalizacji} -- dodanie wyszukiwania po lokalizacji wraz z listą wyników i dystansem od użytkownika.
    \item \textbf{Odkrywanie treści} -- dodanie elementów wspierających znajdowanie interesujących miejsc, np. karuzeli z najpopularniejszymi \glslink{spot}{spotami} na stronie głównej.
    \item \textbf{Funkcje zaawansowane} -- paginacja wyników, sortowanie (np. popularność/ocena) oraz filtrowanie po minimalnej ocenie.
    \item \textbf{Optymalizacja} -- dopracowanie mechanizmów cache'owania zapytań oraz uspójnienie sposobu pobierania danych.
\end{itemize}

\subsubsection{Panel użytkownika}
\label{subsubsec:modul-panel}

Moduł panelu użytkownika rozwijano w kierunku kompletnego zestawu widoków i funkcji
związanych z zarządzaniem kontem oraz aktywnością użytkownika w systemie.

\begin{itemize}
    \item \textbf{Profil użytkownika} -- wdrożenie profilu oraz rozdzielenie widoku własnego profilu i profilu innego użytkownika.
    \item \textbf{Relacje społecznościowe} -- obsługa znajomych, obserwowanych i obserwujących, wraz z nawigacją oraz dedykowanymi widokami list.
    \item \textbf{Zaproszenia do znajomych} -- statusy relacji (wysłane/otrzymane/zakończone), widok listy zaproszeń oraz akcje akceptowania i odrzucania.
    \item \textbf{Wyszukiwanie użytkowników} -- funkcja dodawania znajomych z paginacją i wyszukiwaniem po nazwie użytkownika (np. w formie okna modalnego).
    \item \textbf{Aktywność i zasoby} -- sekcje zdjęć i filmów, ujednolicenie modelu \textit{media}, widoki komentarzy oraz listy z filtrowaniem i sortowaniem (np. po dacie).
    \item \textbf{Paginacja i nieskończone przewijanie} -- przebudowa wybranych widoków panelu i części społecznościowej na mechanizmy stronicowania lub nieskończonego przewijania.
    \item \textbf{Ustawienia konta} -- edycja danych konta (np. nazwa użytkownika, e-mail, hasło) oraz obsługa zmiany zdjęcia profilowego.
    \item \textbf{Moje \glslink{spot}{spoty} i ulubione} -- lista dodanych \glslink{spot}{spotów} oraz zarządzanie ulubionymi \glslink{spot}{spotami} z integracją z mapą.
    \item \textbf{Komunikaty systemowe} -- rozwój mechanizmu prezentacji informacji o błędach i sukcesach w interfejsie (w tym możliwość wyświetlania wielu komunikatów), wykorzystywanego w panelu i innych częściach systemu.
\end{itemize}

\setcounter{secnumdepth}{3}
\setcounter{tocdepth}{3}

\section{Podsumowanie etapu finalizacji}
\label{sec:podsumowanie-finalizacja}

Końcowy okres realizacji koncentrował się na domykaniu funkcji, stabilizacji działania systemu,
poprawie ergonomii interfejsu oraz uzupełnieniu dokumentacji technicznej i tekstowej.
Przyjętą datą zakończenia prac projektowych był \textbf{10 stycznia 2026 roku}.
