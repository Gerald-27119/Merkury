%! Author = Adam
%! Date = 18/02/2025

\chapter{Przebieg realizacji projektu}
\label{ch:przebieg-realizacji-projektu}


Rozdział prezentuje chronologiczny przebieg realizacji projektu, opisując najważniejsze etapy prac oraz decyzje projektowe.


\section{Przebieg realizacji projektu}
\label{sec:przebieg-realizacji-projektu}

\noindent
\textbf{Lipiec -- wrzesień 2024.}
W tym okresie zebrano zespół projektowy oraz przygotowano środowisko deweloperskie
(repozytorium kodu, konfiguracja narzędzi, podstawowa infrastruktura).
Równolegle rozpoczęto prace nad modułem logowania i rejestracji użytkownika,
który miał stać się fundamentem dalszego rozwoju aplikacji.

\medskip

\noindent
\textbf{Październik 2024 -- styczeń 2025.}
Z początkiem roku akademickiego wybrano temat projektu inżynierskiego
oraz w ramach przedmiotu PRO opracowano wstępne wymagania systemu.
Przeprowadzono analizę grupy docelowej i doprecyzowano główne scenariusze użycia.
Od tego momentu nastąpiło właściwe rozpoczęcie prac deweloperskich --
implementowano kolejne moduły backendu i frontend,
opierając się na ustalonych wymaganiach.

\medskip

\noindent
\textbf{Luty 2025.}
Kontynuowano prace deweloperskie.
Rozbudowywano istniejące funkcjonalności
oraz stabilizowano fundamenty architektury aplikacji.

\medskip

\noindent
\textbf{Marzec -- czerwiec 2025.}
W tym okresie przeprowadzono analizę aspektów społecznych i biznesowych produktu
oraz doprecyzowano model wartości dla użytkownika końcowego.
Opracowano szczegółowy projekt interfejsu użytkownika,
uwzględniając zebrane wymagania i wyniki analiz.
Projekt UI był kilkukrotnie iterowany na podstawie wewnętrznych przeglądów
oraz testów użyteczności w zespole.

\medskip

\noindent
\textbf{Lipiec -- wrzesień 2025.}
Prowadzono intensywne prace deweloperskie,
skupiając się na implementacji zaprojektowanych wcześniej ekranów
i przepływów w interfejsie użytkownika
oraz na integracji z usługami zewnętrznymi.
W tym czasie zespół zidentyfikował część funkcji
o zbyt dużej złożoności jak na dostępny czas,
dlatego ograniczono zakres MVP na rzecz dopracowania kluczowych modułów.

\medskip

\noindent
\textbf{Październik -- listopad 2025.}
Rozpoczęto prace nad częścią tekstową pracy inżynierskiej.
Zebrano i uporządkowano materiały powstałe na wcześniejszych semestrach
(diagramy, dokumenty analityczne i projektowe),
a następnie włączono je do spójnej struktury pracy.
Do końca listopada sfinalizowano główne prace deweloperskie --
od tego momentu wprowadzano jedynie poprawki błędów
oraz drobne usprawnienia.

\medskip

\noindent
\textbf{Grudzień 2025 -- luty 2026.}
Zakończenie prac deweloperskich zostało zwieńczone prezentacją aplikacji promotorowi.
Przygotowano i uruchomiono testy automatyczne dla najważniejszych funkcjonalności.
W kodzie wprowadzano już tylko niewielkie poprawki,
a główny nacisk przeniesiono na ukończenie części tekstowej pracy.
Prace nad projektem zakończono 20 lutego 2026 roku.

\section{Podsumowanie}
\label{sec:podsumowanie}

Zespół rozpoczął prace deweloperskie w lipcu 2024 roku,
jeszcze przed formalnym wyborem tematu projektu.
Było to możliwe dzięki temu,
że niezależnie od ostatecznego tematu,
zakładano stworzenie aplikacji internetowej wymagającej kont użytkowników,
co wynikało ze specjalizacji ,,Aplikacje Internetowe'',
na której studiują wszyscy członkowie zespołu.

Wraz z nadejściem października 2024 roku wybrano temat projektu,
a w ramach przedmiotu PRO opracowano wstępne wymagania systemu.
Od tego momentu prace deweloperskie nad produktem trwały nieprzerwanie
aż do początku grudnia 2025 roku.
Część programistyczna projektu zajęła łącznie około 18 miesięcy.
Równolegle z implementacją tworzono dokumentację systemu.

W trakcie rozwoju systemu zespół na bieżąco weryfikował
i aktualizował wymagania stawiane aplikacji.
Wielokrotnie zmieniano i ulepszano wygląd interfejsu użytkownika,
uwzględniając pojawiające się wnioski z analiz i testów.
Wraz ze zbliżaniem się terminu zakończenia projektu
urealniono zakres produktu oraz skorygowano priorytety --
część mniej istotnych funkcjonalności przesunięto poza zakres MVP
na rzecz dopracowania kluczowych modułów.
\newline
\newline
\textbf{Prace nad projektem zakończono 20 lutego 2026 roku.}
