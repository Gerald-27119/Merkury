%! Author = Adam
%! Date = 10/01/2025

\chapter{Przebieg realizacji projektu}
\label{ch:przebieg-realizacji-projektu}

W niniejszym rozdziale przedstawiono rzeczywisty przebieg realizacji projektu
w kolejnych fazach czasowych.
Opis odzwierciedla sposób pracy zespołu zgodny z metodyką
Disciplined Agile Delivery (DAD) w wariancie Lean Life Cycle%
\footnote{\textit{TODO: odwołanie do rozdziału opisującego metodykę pracy
oraz do pozycji w bibliografii.}},
w której prace deweloperskie, planowanie i doprecyzowywanie wymagań
przebiegają iteracyjnie i równolegle.

Warto podkreślić, że przez cały czas trwania projektu członkowie zespołu
poświęcali znaczącą część czasu na wzajemne przeglądy kodu oraz
ciągły feedback dotyczący implementacji poszczególnych funkcjonalności
(zarówno w obrębie backendu, jak i frontendu).
Taki sposób pracy pozwolił na szybkie wychwytywanie błędów,
ujednolicenie stylu implementacji oraz bieżące korygowanie
wymagań funkcjonalnych%
\footnote{\textit{TODO: odwołanie do rozdziału dotyczącego testów
i jakości oprogramowania.}}.

\section{Faza przedprojektowa (czerwiec–wrzesień 2024)}
\label{sec:faza-przedprojektowa}

Faza przedprojektowa obejmowała okres od czerwca do września 2024 roku
i poprzedzała formalne zatwierdzenie tematu pracy oraz opracowanie harmonogramu
przedstawionego w rozdziale planistycznym%
\footnote{\textit{TODO: odwołanie do rozdziału z harmonogramem projektu.}}.

Zespół rozpoczął prace deweloperskie już w czerwcu 2024 roku,
co było możliwe dzięki temu, że niezależnie od ostatecznego tematu
zakładano stworzenie aplikacji internetowej wymagającej kont użytkowników.
Wynikało to bezpośrednio ze specyfiki specjalizacji
„Aplikacje Internetowe”, na której studiują wszyscy członkowie zespołu%
\footnote{\textit{TODO: ewentualne odwołanie do opisu kontekstu studiów
w rozdziale wstępnym.}}.
W efekcie część prac technicznych została wykonana jeszcze przed
formalnym wyborem tematu projektu.

\subsection{Czerwiec 2024}

W czerwcu 2024 roku wykonano następujące działania przygotowawcze:

\begin{itemize}
    \item Przygotowanie szkieletu projektu backendu w technologii Spring Boot –
    utworzono podstawowe pakiety, konfigurację aplikacji oraz minimalną
    strukturę modułów (\textbf{cały zespół}).
    \item Konfiguracja narzędzia \texttt{Jira}:
    zespół zapoznał się z typowym podziałem na epiki, taski i podtaski,
    zdefiniował statusy przepływu pracy oraz utworzył tablicę
    w stylu Kanban (\textbf{Kacper Badek}).
    \item Utworzenie repozytorium w serwisie \texttt{GitHub}
    i skonfigurowanie podstawowych reguł pracy z repozytorium
    (\textbf{Adam Langmesser}).
    \item Przygotowanie pierwszego potoku CI/CD dla backendu –
    automatyczne budowanie projektu i uruchamianie testów jednostkowych
    na serwerze ciągłej integracji (\textbf{Adam Langmesser}).
\end{itemize}

\subsection{Lipiec 2024}

W lipcu 2024 roku skupiono się na warstwie frontendowej
i pierwszych funkcjonalnościach związanych z kontami użytkowników:

\begin{itemize}
    \item Przygotowanie szkieletu aplikacji frontendowej
    (React + TypeScript) wraz z podstawową strukturą komponentów
    i konfiguracją narzędzi budujących (\textbf{cały zespół}).
    \item Rozpoczęcie prac nad potokiem CI/CD dla frontendu,
    którego konfigurację finalnie ukończono we wrześniu 2024 roku
    (\textbf{Adam Langmesser}).
    \item Implementacja podstawowej logiki logowania
    i rejestracji użytkownika zarówno po stronie backendu
    (endpointy REST/GraphQL), jak i frontendu
    (formularze oraz obsługa żądań)
    (\textbf{Adam Langmesser}).
\end{itemize}

\subsection{Sierpień 2024}

W sierpniu 2024 roku zespół skupił się na zagadnieniach bezpieczeństwa
oraz na dopracowaniu pierwszych ekranów aplikacji:

\begin{itemize}
    \item Dodanie Spring Security i implementacja logiki
    uwierzytelniania oraz autoryzacji użytkownika po stronie backendu
    (\textbf{Adam Langmesser}).
    \item Konfiguracja biblioteki Tailwind CSS na froncie,
    umożliwiająca spójne i responsywne stylowanie komponentów
    (\textbf{Mateusz Redosz}).
    \item Konfiguracja narzędzia \texttt{Prettier}
    do automatycznego formatowania kodu frontendu
    (\textbf{Mateusz Redosz}).
    \item Stworzenie formularza rejestracji użytkownika
    wraz z podstawową walidacją po stronie frontendu
    (\textbf{Mateusz Redosz}).
    \item Dodanie i konfiguracja biblioteki TanStack Query
    do obsługi komunikacji z backendem i cachowania danych
    (\textbf{Mateusz Redosz}).
    \item Implementacja strony logowania użytkownika
    po stronie frontendu (\textbf{Kacper Badek}).
    \item Implementacja logiki resetowania hasła
    (proces „zapomniałem hasła”) po stronie frontendu
    (\textbf{Stanisław Oziemczuk}).
\end{itemize}

\subsection{Wrzesień 2024}

We wrześniu 2024 roku kontynuowano prace nad bezpieczeństwem
i przygotowaniem środowiska uruchomieniowego:

\begin{itemize}
    \item Implementacja przycisków na stronie logowania
    umożliwiających logowanie i rejestrację z wykorzystaniem OAuth
    (GitHub i Google) po stronie frontendu.
    \item Zastąpienie bazy danych działającej w pamięci (in-memory)
    instancją uruchamianą w kontenerze \texttt{Docker},
    co urealniło środowisko deweloperskie i testowe
    (\textbf{Adam Langmesser}).
    \item Dostosowanie potoku CI/CD backendu tak,
    aby uwzględniał uruchamianie bazy danych w kontenerze
    podczas wykonywania testów (\textbf{Adam Langmesser}).
    \item Implementacja logiki resetowania hasła po stronie backendu –
    generowanie tokenów, ich weryfikacja oraz integracja
    z istniejącym procesem resetowania na froncie
    (\textbf{Kacper Badek}).
    \item Dokończenie konfiguracji potoku CI/CD dla frontendu
    (\textbf{Adam Langmesser}).
\end{itemize}

\section{Etap 1 (październik 2024 – styczeń 2025)}
\label{sec:etap1}

Etap 1 był przede wszystkim poświęcony dopracowaniu wymagań wstępnych,
stabilizacji modułu uwierzytelniania oraz pierwszym eksperymentom
z mapą spotów.
W tym okresie zespół łączył prace deweloperskie z działaniami
analitycznymi i planistycznymi (opracowanie harmonogramu, założeń
oraz wymagań w ramach przedmiotów projektowych na uczelni).

\subsection{Październik 2024}

\begin{itemize}
    \item Poprawa konfiguracji CORS na backendzie, tak aby aplikacja frontendowa
    mogła komunikować się z serwerem w sposób bezpieczny
    i zgodny z przeglądarkowymi ograniczeniami
    (\textbf{Mateusz Redosz}).
    \item Zmiana sposobu przechowywania tokena JWT – umieszczenie go
    w ciasteczku \texttt{HttpOnly}, co poprawiło bezpieczeństwo aplikacji
    (\textbf{Mateusz Redosz}).
    \item Uporządkowanie pliku \texttt{.gitignore}
    oraz struktury repozytorium (\textbf{Kacper Badek}).
    \item Implementacja logowania OAuth z wykorzystaniem Google i GitHub
    po stronie backendu oraz integracja z frontem
    (\textbf{Stanisław Oziemczuk}).
    \item Formalny wybór tematu projektu inżynierskiego przez zespół.
\end{itemize}

\subsection{Listopad 2024}

\begin{itemize}
    \item Opracowanie harmonogramu projektu w formie opisowej oraz
    w postaci wykresu Gantta (\textbf{cały zespół})%
    \footnote{\textit{TODO: odwołanie do rozdziału z harmonogramem projektu.}}.
    \item Przygotowanie wstępnych założeń i wymagań w ramach przedmiotu PRO
    (\textbf{cały zespół})%
    \footnote{\textit{TODO: odwołanie do rozdziału z analizą wymagań.}}.
    \item Implementacja demonstracyjnej mapy z wykorzystaniem biblioteki Leaflet –
    prototyp miał na celu pokazanie zespołowi możliwości interaktywnej mapy
    (\textbf{Adam Langmesser}).
    Ze względu na ograniczone możliwości customizacji wyglądu biblioteki
    zdecydowano się później na zmianę dostawcy kafelków mapowych na usługę
    \textit{TODO: uzupełnić nazwę docelowej usługi mapowej}.
    \item Poprawki konfiguracji narzędzia \texttt{ESLint}
    po stronie frontendu (\textbf{Adam Langmesser}).
    \item Implementacja logiki cyklicznego usuwania przeterminowanych
    tokenów resetu hasła (\textbf{Kacper Badek}).
    \item Poprawki działania logowania użytkownika po stronie backendu
    i frontendu, obejmujące obsługę błędów oraz komunikaty dla użytkownika
    (\textbf{Stanisław Oziemczuk}).
    \item Dalsze poprawki obsługi JWT na backendzie oraz logowania błędów
    związanych z procesem logowania i rejestracji
    (\textbf{Mateusz Redosz}).
\end{itemize}

\subsection{Grudzień 2024}

\begin{itemize}
    \item Dalsze poprawki konfiguracji Spring Security na backendzie,
    w tym doprecyzowanie ról i uprawnień (\textbf{Adam Langmesser}).
    \item Poprawa logowania błędów związanych z resetowaniem hasła
    użytkownika po stronie backendu (\textbf{Kacper Badek}).
    \item Rozszerzenie encji użytkownika o dane deweloperskie
    oraz przygotowanie inicjalnych danych w bazie
    (np. konta testowe) (\textbf{Adam Langmesser}).
    \item Dodanie biblioteki Redux do frontendu i wstępna konfiguracja store
    (\textbf{Mateusz Redosz}).
    \item Implementacja automatycznego wylogowywania użytkownika
    po wygaśnięciu tokena JWT (\textbf{Mateusz Redosz}).
    \item Stworzenie komponentu frontendu odpowiedzialnego za wyświetlanie
    szczegółów pojedynczego spota (\textbf{Stanisław Oziemczuk}).
    \item Stworzenie komponentu odpowiedzialnego za prezentację błędów
    systemowych użytkownikowi (globalny mechanizm powiadomień)
    (\textbf{Mateusz Redosz}).
    \item Dostosowanie wyglądu i treści wiadomości e-mail wysyłanych
    przez system (np. w procesie resetu hasła)
    (\textbf{Kacper Badek}).
    \item Implementacja testów automatycznych związanych z bezpieczeństwem
    (scenariusze logowania i rejestracji użytkownika po stronie backendu)
    (\textbf{Adam Langmesser}).
\end{itemize}

\subsection{Styczeń 2025}

\begin{itemize}
    \item Dodanie danych deweloperskich dla mapy
    (przykładowe spoty, dane do prezentacji i testów)
    (\textbf{Kacper Badek}).
    \item Rozszerzenie stanu Redux o informację o tym,
    czy użytkownik jest aktualnie zalogowany
    (\textbf{Mateusz Redosz}).
    \item Implementacja testów E2E dla procesów logowania
    i rejestracji użytkownika (\textbf{Mateusz Redosz}).
    \item Poprawki logowania i rejestracji z wykorzystaniem OAuth
    (Google/GitHub) – dopracowanie scenariuszy brzegowych
    (\textbf{Stanisław Oziemczuk}).
    \item Dodanie uruchamiania testów frontendu do potoku CI/CD
    (\textbf{Mateusz Redosz}).
    \item Poprawa testów logowania i rejestracji na backendzie
    oraz ujednolicenie asercji (\textbf{Adam Langmesser}).
    \item Implementacja logiki filtrowania spotów na mapie
    po różnych kryteriach (np. nazwa) po stronie backendu
    (\textbf{Stanisław Oziemczuk}).
    \item Poprawa sposobu wyświetlania szczegółów spota na froncie
    (\textbf{Mateusz Redosz}).
    \item Dalsze dopracowanie logiki filtrowania spotów po nazwie
    po stronie backendu (\textbf{Stanisław Oziemczuk}).
    \item Implementacja logiki dodawania spota do ulubionych
    po stronie backendu (\textbf{Kacper Badek}).
\end{itemize}

\section{Etap 2 (luty 2025 – wrzesień 2025)}
\label{sec:etap2}

Etap 2 obejmował zasadniczą część prac deweloperskich nad aplikacją.
W tym okresie rozwijano kolejne moduły (mapa, forum, czat, panel użytkownika),
równolegle doprecyzowując dokumentację oraz weryfikując i aktualizując
wymagania na podstawie bieżących doświadczeń zespołu%
\footnote{\textit{TODO: odwołanie do rozdziału z analizą wymagań
i do rozdziału opisującego projekt architektury.}}.

\subsection{Luty 2025}

\begin{itemize}
    \item Poprawa wyglądu strony logowania, w tym dopracowanie stylistyki
    komponentów oraz zachowania w trybie ciemnym i jasnym
    (\textbf{Mateusz Redosz}).
    \item Poprawa logiki wylogowywania użytkownika na froncie
    (m.in. czyszczenie stanu Redux, przekierowania)
    (\textbf{Mateusz Redosz}).
    \item Poprawa logiki dodawania komentarzy do spotów na mapie
    po stronie frontendu (\textbf{Mateusz Redosz}).
    \item Implementacja logiki dodawania spota do ulubionych
    po stronie frontendu oraz integracja z backendem
    (\textbf{Stanisław Oziemczuk}).
    \item Implementacja integracji z zewnętrznym API pogodowym
    służącym do pobierania podstawowych danych pogodowych
    dla danego spota (\textbf{Stanisław Oziemczuk}).
    \textit{TODO: uzupełnić nazwę i dostawcę API oraz odwołać się
    do kart usług zewnętrznych i bibliografii.}
\end{itemize}

\subsection{Marzec 2025}

\textit{TODO: do uzupełnienia – opis prac wykonanych w marcu 2025 roku
    (np. rozwój forum, pierwsza wersja czatu, dalsze prace nad mapą).}

\subsection{Kwiecień 2025}

\textit{TODO: do uzupełnienia – opis prac wykonanych w kwietniu 2025 roku.}

\subsection{Maj 2025}

\textit{TODO: do uzupełnienia – opis prac wykonanych w maju 2025 roku.}

\subsection{Czerwiec 2025}

\textit{TODO: do uzupełnienia – opis prac wykonanych w czerwcu 2025 roku.}

\subsection{Lipiec 2025}

\textit{TODO: do uzupełnienia – opis prac wykonanych w lipcu 2025 roku.}

\subsection{Sierpień 2025}

\textit{TODO: do uzupełnienia – opis prac wykonanych w sierpniu 2025 roku.}

\subsection{Wrzesień 2025}

\textit{TODO: do uzupełnienia – opis prac wykonanych we wrześniu 2025 roku
oraz podsumowanie Etapu 2.}

\section{Etap 3 (październik 2025 – styczeń 2026)}
\label{sec:etap3}

Etap 3 obejmował finalizację prac nad systemem,
dopracowanie dokumentacji technicznej i tekstowej pracy inżynierskiej
oraz przygotowanie projektu do oddania.
W tym czasie większość nowych funkcjonalności była już zaimplementowana,
a nacisk położono na stabilizację, testy oraz spójny opis w dokumentacji.

\subsection{Październik 2025}

\textit{TODO: do uzupełnienia – opis prac wykonanych w październiku 2025 roku
    (np. domykanie brakujących funkcjonalności, porządki w kodzie,
    stabilizacja środowiska).}

\subsection{Listopad 2025}

\begin{itemize}
    \item Opracowanie rozdziału \textit{Analiza wymagań} –
    przygotowanie listy aktorów systemu, diagramu przypadków użycia
    oraz scenariuszy przypadków użycia
    (\textbf{Adam Langmesser}).
    \textit{TODO: odwołanie do odpowiedniego rozdziału i podrozdziałów.}
\end{itemize}

\subsection{Grudzień 2025}

\begin{itemize}
    \item Opracowanie podrozdziału poświęconego wymaganiom
    dla modułu czatu, w tym wymagań funkcjonalnych
    i pozafunkcjonalnych (\textbf{Adam Langmesser}).
    \textit{TODO: odwołanie do numeru podrozdziału z wymaganiami dla czatu.}
    \item Dopracowanie kart usług zewnętrznych
    (m.in. usług mapowych, pogodowych, poczty e-mail)
    wykorzystywanych przez system (\textbf{Adam Langmesser}).
    \textit{TODO: odwołanie do sekcji z kartami usług zewnętrznych.}
\end{itemize}

\subsection{Styczeń 2026}

\textit{TODO: do uzupełnienia – szczegółowy opis ostatnich prac wykonanych
w styczniu 2026 roku (np. końcowe testy, poprawki kosmetyczne,
    formatowanie pracy).}

Na zakończenie prac projektowych przyjęto datę \textbf{10 stycznia 2026 roku}.
