%! Author = Adam
%! Date = 10/01/2025


\chapter{Przebieg realizacji projektu}
\label{ch:przebieg-realizacji-projektu}

W niniejszym rozdziale przedstawiono przebieg realizacji
projektu w kolejnych Etapach zdefiniowanych w harmonogramie.
Sposób realizacji projektu odzwierciedla model pracy zespołu zgodny z metodyką
\gls{DAD_LLC}, w której prace deweloperskie, planowanie
i doprecyzowywanie wymagań przebiegają iteracyjnie i równolegle.

Warto podkreślić, że przez cały czas trwania projektu członkowie zespołu
poświęcali znaczącą część czasu na wzajemne przeglądy kodu
(\gls{review-kodu}) oraz ciągły feedback dotyczący implementacji
poszczególnych funkcjonalności.
Taki sposób pracy pozwolił na szybkie wychwytywanie błędów,
ujednolicenie stylu implementacji oraz bieżące korygowanie
wymagań.

Dla przejrzystości opisu, w ramach każdego miesiąca przedstawiono
zadania w podziale na poszczególnych członków zespołu.

\newenvironment{zadaniaosoby}[1]{%
    \par\medskip
    \noindent\textbf{#1}\par
    \vspace{-0.4\baselineskip}
    \begin{itemize}
}{%
    \end{itemize}
}


\section{Faza przedprojektowa (lipiec–wrzesień 2024)}
\label{sec:faza-przedprojektowa}

Faza przedprojektowa, wyróżniona w harmonogramie jako okres od lipca do
września 2024 roku, poprzedzała formalne zatwierdzenie tematu pracy
oraz opracowanie szczegółowego harmonogramu.

Na początku fazy przedprojektowej doprecyzowano koncepcję systemu
i wybrano główny stos technologiczny, co umożliwiło szybkie przejście do
bardziej zaawansowanych działań opisanych w kolejnych podsekcjach.
Niezależnie od ostatecznego sformułowania tematu zakładano stworzenie
aplikacji webowej wymagającej kont użytkowników, co wynikało wprost
ze specyfiki specjalizacji „Aplikacje Internetowe”, na której studiowali
wszyscy członkowie zespołu.

\subsection*{Lipiec 2024}

\begin{zadaniaosoby}{Cały zespół}
    \item Określenie struktury plików na \glslink{backend}{backendzie}.
    \item Wstępny wybór stosu technologicznego:
    \gls{backend} w \gls{spring-boot}, \gls{frontend} w \gls{react} z wykorzystaniem
    \gls{type-script} oraz system kontroli wersji oparty na \gls{github},
    a także ogólne założenia dotyczące dalszego rozwoju architektury.
\end{zadaniaosoby}

\begin{zadaniaosoby}{Adam Langmesser}
    \item Przygotowanie lokalnego środowiska deweloperskiego dla \glslink{backend}{backendu}
    i \glslink{frontend}{frontendu} (konfiguracja \gls{ide}, testowe projekty, podstawowe
    ustawienia narzędzi budujących), co ułatwiło płynne przejście
    do właściwych prac w kolejnych miesiącach.
\end{zadaniaosoby}

\subsection*{Sierpień 2024}

\begin{zadaniaosoby}{Cały zespół}
    \item Utworzenie projektu w narzędziu \gls{jira}, zdefiniowanie
    typów zgłoszeń (epik, zadanie, podzadanie) oraz przygotowanie
    podstawowego workflow z wykorzystaniem \glslink{tablica_kanban}{tablicy kanban}.
\end{zadaniaosoby}

\begin{zadaniaosoby}{Mateusz Redosz}
    \item Konfiguracja biblioteki \gls{tailwind-css} w części
    \glslink{frontend}{frontendowej}.
    \item Konfiguracja narzędzia \gls{prettier} do automatycznego
    formatowania kodu \glslink{frontend}{frontendu}.
    \item Implementacja rejestracji użytkownika zarówno na \glslink{backend}{backendzie} jak i \glslink{frontend}{frontendzie}.
    \item Dodanie i konfiguracja biblioteki \gls{tanstack-query}.
    \item Stworzenie wstępnej struktury \glslink{routing}{routingu} na \glslink{frontend}{frontendzie}.
\end{zadaniaosoby}

\begin{zadaniaosoby}{Adam Langmesser}
    \item Implementacja podstawowej strony powitalnej na \glslink{frontend}{frontendzie}.
    \item Implementacja podstaw logiki logowania i rejestracji
    użytkownika po stronie \glslink{backend}{backendu}.
    \item Dodanie obsługi przypadku, w którym podczas rejestracji
    wybrana nazwa użytkownika jest już zajęta.
\end{zadaniaosoby}

\begin{zadaniaosoby}{Kacper Badek}
    \item Stworzenie formularza logowania użytkownika po stronie
    \glslink{frontend}{frontendu} oraz jego integracja z przygotowanym
    \glslink{backend}{backendowym} \gls{api} logowania.
\end{zadaniaosoby}

\begin{zadaniaosoby}{Stanisław Oziemczuk}
    \item Integracja systemu z zewnętrzną usługą wysyłania wiadomości
    e-mail oraz implementacja wysyłania wiadomości powitalnej do
    nowo zarejestrowanego użytkownika.
    \item Dodanie pliku konfiguracyjnego do formatowania kodu w \gls{intellij-idea}.
\end{zadaniaosoby}

\subsection*{Wrzesień 2024}

\begin{zadaniaosoby}{Adam Langmesser}
    \item Dodanie \gls{docker-compose} dla \glslink{backend}{backendu} i bazy danych.
    \item Dodanie cyklicznego testu sprawdzającego, czy \glslink{backend}{backend}
    uruchamiany w środowisku \gls{docker-compose} działa poprawnie
    i odpowiada na podstawowe żądania.
\end{zadaniaosoby}

\begin{zadaniaosoby}{Mateusz Redosz}
    \item Dodanie przycisków na \glslink{frontend}{frontendzie} do rejestracji kontem Google lub \gls{github}.
    \item Poprawa logiki obsługi \gls{jwt}.
\end{zadaniaosoby}

\begin{zadaniaosoby}{Kacper Badek}
    \item Implementacja funkcjonalności resetowania hasła użytkownika,
    obejmującej generowanie i weryfikację tokenów resetu oraz
    formularz zmiany hasła.
\end{zadaniaosoby}


\section{Etap 1 (październik 2024 – styczeń 2025)}
\label{sec:etap1}

Etap 1 był przede wszystkim poświęcony dopracowaniu wymagań wstępnych,
stabilizacji modułu uwierzytelniania oraz pierwszym eksperymentom
z mapą \glslink{spot}{spotów}.
W tym okresie zespół łączył prace deweloperskie z działaniami
analitycznymi i planistycznymi (opracowanie harmonogramu, założeń
oraz wymagań w ramach przedmiotów projektowych na uczelni).

\subsection*{Październik 2024}

\begin{zadaniaosoby}{Cały zespół}
    \item Formalny wybór tematu projektu inżynierskiego.
\end{zadaniaosoby}

\begin{zadaniaosoby}{Adam Langmesser}
    \item Zaprezentowanie zespołowi przykładowych \glslink{integration-tests}{testów integracyjnych}
    oraz \glslink{e2e-tests}{testów E2E} na \glslink{backend}{backendzie}.
\end{zadaniaosoby}

\begin{zadaniaosoby}{Mateusz Redosz}
    \item Poprawa konfiguracji \gls{cors} na \glslink{backend}{backendzie}, tak aby aplikacja
    \glslink{frontend}{frontendowa} mogła komunikować się z serwerem w sposób bezpieczny
    i zgodny z przeglądarkowymi ograniczeniami.
    \item Zmiana sposobu przechowywania tokena \gls{jwt} – umieszczenie go
    w \glslink{http-only-cookie}{ciasteczku HttpOnly}, co poprawiło bezpieczeństwo
    aplikacji.
\end{zadaniaosoby}

\begin{zadaniaosoby}{Stanisław Oziemczuk}
    \item Implementacja logowania \gls{oauth} z wykorzystaniem Google
    i \gls{github} po stronie \glslink{backend}{backendu} oraz integracja z \glslink{frontend}{frontendem}.
    \item Napisanie testów na \glslink{backend}{backendzie} do \gls{oauth}.
\end{zadaniaosoby}

\begin{zadaniaosoby}{Kacper Badek}
    \item Uporządkowanie pliku \texttt{.gitignore}
    oraz struktury repozytorium.
\end{zadaniaosoby}

\subsection*{Listopad 2024}

\begin{zadaniaosoby}{Adam Langmesser}
    \item Implementacja demonstracyjnej mapy z wykorzystaniem
    biblioteki \gls{leaflet} – prototyp miał na celu pokazanie zespołowi
    możliwości interaktywnej mapy.
    \item Poprawki konfiguracji narzędzia \gls{eslint}
    po stronie \glslink{frontend}{frontendu}.
\end{zadaniaosoby}

\begin{zadaniaosoby}{Mateusz Redosz}
    \item Dalsze poprawki obsługi \gls{jwt} na \glslink{backend}{backendzie} oraz dodanie logowania
    błędów związanych z procesem logowania i rejestracji.
    \item Implementacja przykładowych \glslink{e2e-tests}{testów E2E} na \glslink{frontend}{frontendzie}.
\end{zadaniaosoby}

\begin{zadaniaosoby}{Stanisław Oziemczuk}
    \item Poprawa logiki logowania użytkownika po stronie \glslink{backend}{backendu}
    i \glslink{frontend}{frontendu}.
\end{zadaniaosoby}

\begin{zadaniaosoby}{Kacper Badek}
    \item Implementacja logiki cyklicznego usuwania przeterminowanych
    tokenów resetu hasła.
\end{zadaniaosoby}

\begin{zadaniaosoby}{Cały zespół}
    \item Opracowanie harmonogramu projektu w formie opisowej
    oraz w postaci \glslink{gantt-chart}{diagramu Gantta}.
    \item Przygotowanie wstępnych założeń i wymagań
    w ramach przedmiotu \gls{pro}.
\end{zadaniaosoby}

\subsection*{Grudzień 2024}

\begin{zadaniaosoby}{Adam Langmesser}
    \item Dalsze poprawki konfiguracji \gls{spring-security} na \glslink{backend}{backendzie},
    w tym doprecyzowanie ról i uprawnień.
    \item Dodanie nowych użytkowników do bazy danych, w celach deweloperskich.
    \item Implementacja testów automatycznych związanych
    z bezpieczeństwem (scenariusze logowania i rejestracji
    użytkownika po stronie \glslink{backend}{backendu}).
\end{zadaniaosoby}

\begin{zadaniaosoby}{Mateusz Redosz}
    \item Dodanie biblioteki \gls{redux} do \glslink{frontend}{frontendu} i wstępna
    konfiguracja store.
    \item Implementacja automatycznego wylogowywania użytkownika
    po wygaśnięciu tokena \gls{jwt}.
    \item Stworzenie komponentu odpowiedzialnego za prezentację błędów
    systemowych użytkownikowi (globalny mechanizm powiadomień).
\end{zadaniaosoby}

\begin{zadaniaosoby}{Stanisław Oziemczuk}
    \item Stworzenie komponentu \glslink{frontend}{frontendu} odpowiedzialnego
    za wyświetlanie szczegółów pojedynczego \glslink{spot}{spota}.
\end{zadaniaosoby}

\begin{zadaniaosoby}{Kacper Badek}
    \item Poprawa logowania błędów związanych z resetowaniem hasła
    użytkownika po stronie \glslink{backend}{backendu}.
    \item Dostosowanie wyglądu i treści wiadomości e-mail wysyłanych
    przez system (np. w procesie resetu hasła).
\end{zadaniaosoby}

\subsection*{Styczeń 2025}

\begin{zadaniaosoby}{Adam Langmesser}
    \item Poprawa testów logowania i rejestracji na \glslink{backend}{backendzie}
    oraz ujednolicenie asercji.
    \item Rozwiązanie problemu z relacjami między obiektami w testach integracyjnych.
\end{zadaniaosoby}

\begin{zadaniaosoby}{Mateusz Redosz}
    \item Rozszerzenie stanu \gls{redux} o informację o tym,
    czy użytkownik jest aktualnie zalogowany.
    \item Implementacja testów dla procesów logowania
    i rejestracji użytkownika.
    \item Implementacja \glslink{integration-tests}{testów integracyjnych} i \glslink{unit-tests}{testów jednostkowych} dla \glslink{sidebar}{sidebara}.
    \item Dodanie uruchamiania testów \glslink{frontend}{frontendu} do \gls{cicd}.
    \item Poprawa sposobu wyświetlania szczegółów \glslink{spot}{spota}.
\end{zadaniaosoby}

\begin{zadaniaosoby}{Stanisław Oziemczuk}
    \item Poprawki logowania i rejestracji z wykorzystaniem \gls{oauth}
    (Google/\gls{github}) – dopracowanie scenariuszy brzegowych.
    \item Implementacja logiki filtrowania \glslink{spot}{spotów} na mapie
    po różnych kryteriach (np. nazwa) po stronie \glslink{backend}{backendu}.
    \item Dalsze dopracowanie logiki filtrowania \glslink{spot}{spotów} po nazwie
    po stronie \glslink{backend}{backendu}.
    \item Implementacja logiki zmiany danych konta użytkownika zarówno na \glslink{backend}{backendzie}
    jak i na \glslink{frontend}{frontendzie}.
\end{zadaniaosoby}

\begin{zadaniaosoby}{Kacper Badek}
    \item Dodanie danych deweloperskich dla mapy
    (przykładowe \glslink{spot}{spoty}, dane do prezentacji i testów).
    \item Implementacja logiki dodawania \glslink{spot}{spota} do ulubionych
    po stronie \glslink{backend}{backendu}.
\end{zadaniaosoby}


\section{Etap 2 (luty 2025 – wrzesień 2025)}
\label{sec:etap2}

Etap 2 obejmował zasadniczą część prac deweloperskich nad aplikacją.
W tym okresie rozwijano kolejne moduły (mapa, forum, czat,
panel użytkownika), równolegle doprecyzowując dokumentację oraz
weryfikując i aktualizując wymagania na podstawie bieżących
doświadczeń zespołu.

Równolegle, w ramach przedmiotu \gls{prz1} opracowano projekt interfejsu
użytkownika, konsultowany z prowadzącym zajęcia, mgr. inż. Adamem
Urbanowiczem, który na bieżąco przekazywał zespołowi uwagi i
rekomendacje dotyczące ergonomii oraz spójności interfejsu
z założeniami funkcjonalnymi.

\subsection*{Luty 2025}

\begin{zadaniaosoby}{Mateusz Redosz}
    \item Poprawa wyglądu strony logowania, w tym dopracowanie trybu jasnego tejże strony.
    \item Implementacja automatycznego wylogowania użytkownika.
    \item Implementacja obsługi błędów \gls{jakarta-validation}.
    \item Refactor wyglądu komponentu pogodowego na \glslink{frontend}{frontendzie}.
    \item Poprawa działania \gls{cicd}.
    \item Wypracowanie propozycji palety kolorystycznej interfejsu użytkownika.
    \item Implementacja integracji z zewnętrznym \gls{api} pogodowym
    służącym do pobierania podstawowych danych pogodowych
    dla danego \glslink{spot}{spota}.
\end{zadaniaosoby}

\begin{zadaniaosoby}{Stanisław Oziemczuk}
    \item Implementacja logiki dodawania \glslink{spot}{spota} do ulubionych
    po stronie \glslink{frontend}{frontendu} oraz integracja z \glslink{backend}{backendem}.
    \item Refaktoryzacja kodu na \glslink{frontend}{frontendzie}.
    \item Obsługa błędów przy wysyłaniu maili.
    \item Zmiana dostawy usługi do wysyłania maili.
    \item Optymalizacja zapytań do bazy danych.
    \item \glslink{integration-tests}{Testy integracyjne} logiki logowania i rejestracji \gls{oauth}
    na \glslink{backend}{backendzie} oraz filtrowania \glslink{spot}{spotów}.
    \item Poprawa plików \gls{dockerfile} oraz \gls{docker-compose}.
    \item Przygotowanie skryptu uruchamiającego wszystkie wymagane
    kontenery \gls{docker} (bazy danych i usługi pomocnicze) na potrzeby
    środowiska deweloperskiego, co uprościło proces uruchamiania
    projektu na nowych stanowiskach.
\end{zadaniaosoby}

\begin{zadaniaosoby}{Adam Langmesser}
    \item Dalsze poprawki konfiguracji \gls{spring-security} na \glslink{backend}{backendzie},
    obejmujące doprecyzowanie reguł autoryzacji i konfiguracji filtrów.
    \item Poprawa konfiguracji pliku \texttt{.gitignore} w repozytorium,
    tak aby obejmował również nowe katalogi i pliki generowane przez
    narzędzia deweloperskie.
    \item Poprawa działania \gls{cicd} dla \glslink{backend}{backendu}.
    \item Poprawa struktury plików na \glslink{backend}{backendzie}.
    \item Dodanie cachowania na \glslink{backend}{backendzie} w postaci \gls{redis}.
    \item Poprawa konfiguracji poziomów logowania o błędach na \glslink{backend}{backendzie}.
\end{zadaniaosoby}

\subsection*{Marzec 2025}

\begin{zadaniaosoby}{Cały Zespół}
    \item Wypracowanie wstępnej wersji wyglądu interfejsu użytkownika w ramach \gls{prz1}.
\end{zadaniaosoby}

\begin{zadaniaosoby}{Kacper Badek}
    \item Refaktoryzacja szablonów wiadomości e-mail związanych
    z rejestracją i resetowaniem hasła – ujednolicenie struktury HTML,
    metadanych oraz stylu logo aplikacji.
    \item Dodanie możliwości
    edycji, usuwania i głosowania na komentarze \glslink{spot}{spota} oraz lepsza integracja
    z widokiem szczegółów \glslink{spot}{spota} (\gls{paginacja}, powiadomienia,
    prezentacja ocen).
\end{zadaniaosoby}

\begin{zadaniaosoby}{Stanisław Oziemczuk}

    \item Przygotowanie instrukcji uruchamiania projektu w README.
\end{zadaniaosoby}

\begin{zadaniaosoby}{Mateusz Redosz}
    \item Migracja wszystkich adresów obrazów (galerie zdjęć \glslink{spot}{spotów} oraz logotypy
    w szablonach e-mail) z usługi Google Drive do \gls{cdn} ucarecdn.com.
\end{zadaniaosoby}

\subsection*{Kwiecień 2025}

\begin{zadaniaosoby}{Cały Zespół}
    \item Wypracowanie finalnej wersji wyglądu interfejsu użytkownika w ramach \gls{prz1}.
\end{zadaniaosoby}

\begin{zadaniaosoby}{Adam Langmesser}
    \item Konfiguracja projektu dokumentacji w \gls{latex},
    utworzenie pliku bibliografii oraz głównego dokumentu z przykładową strukturą rozdziałów.
    \item Implementacja podstaw funkcjonalności czatu w warstwie
    \glslink{backend}{backendowej} i \glslink{frontend}{frontendowej}, a także dostosowanie layoutu i paska bocznego
    do nowej sekcji.
\end{zadaniaosoby}

\begin{zadaniaosoby}{Mateusz Redosz}
    \item Integracja \glslink{frontend}{frontendu} z \gls{type-script}.
    \item Integracja \gls{tailwind-css} z projektem \glslink{frontend}{frontendu} oraz dodanie
    predefiniowanych kolorów zgodnych z nową szatą kolorystyczną interfejsu użytkownika.
    \item Poprawa i rozbudowa \glslink{sidebar}{sidebara}.
    \item Implementacja strony profilu użytkownika.
\end{zadaniaosoby}

\subsection*{Maj 2025}

\begin{zadaniaosoby}{Stanisław Oziemczuk}
    \item Migracja mapy z biblioteki \gls{leaflet} na \gls{react-maplibre},
    oraz implementacja od nowa podstaw wyświetlania \glslink{spot}{spotów} i interakcji użytkownika z mapą.
    \item Zmiana wyglądu znacznika lokalizacji użytkownika na mapie.
    \item Migracja funkcji \gls{api} związanych ze \glslink{spot}{spotami} do \gls{type-script},
    rozszerzenie modelu szczegółów \glslink{spot}{spota} o dodatkowe informacje
    lokalizacyjne i statystyczne (np. kraj, miasto, tagi, liczniki ocen)
    oraz refaktoryzacja logiki interakcji z markerami \gls{react-maplibre}.
\end{zadaniaosoby}

\begin{zadaniaosoby}{Mateusz Redosz}
    \item Refactor panelu bocznego,
    dodanie ogólnych hooków do zarządzania stanem (w tym trybem
    ciemnym) oraz wprowadzenie zależności do biblioteki \texttt{motion}
    na potrzeby animacji.
    \item Dodanie sekcji \textit{„Znajomi”} w panelu konta, obejmującej
    obsługę znajomych, obserwowanych i obserwujących, wprowadzenie
    nowych modeli i \glslink{endpoint}{endpointów} do zarządzania relacjami oraz dodanie
    testów jednostkowych.
    \item Refaktoryzacja \glslink{endpoint}{endpointów} użytkownika tak, aby login był
    pobierany z kontekstu uwierzytelnienia zamiast z parametru
    ścieżki, dostosowanie do tego wywołań na froncie oraz przeniesienie
    \gls{redux-slice} odpowiedzialnego za konto użytkownika na \gls{type-script}
    z uproszczonym stanem i uporządkowanymi importami.
    \item Poprawa wyglądu formularzy logowania i rejestracji.
    \item Poprawa zarządzania stanem \glslink{sidebar}{sidebara}.
\end{zadaniaosoby}

\begin{zadaniaosoby}{Adam Langmesser}
    \item Wprowadzenie usprawnień w potokach \gls{cicd} \glslink{backend}{backendu},
    obejmujących optymalizację czasu budowania oraz doprecyzowanie
    kroków uruchamiania testów, co zwiększyło niezawodność procesu
    wdrażania.
\end{zadaniaosoby}

\begin{zadaniaosoby}{Kacper Badek}
    \item Dodanie pełnoprawnego forum po stronie \glslink{frontend}{frontendu}, obejmującego
    obsługę postów, kategorii, tagów i \glslink{paginacja}{paginacji}, oraz integracja
    \glslink{backend}{backendu} z usługą \gls{azure-blob-storage} do uploadu i
    przechowywania mediów.
\end{zadaniaosoby}

\subsection*{Czerwiec 2025}

\begin{zadaniaosoby}{Mateusz Redosz}
    \item Przebudowa profilu użytkownika poprzez rozdzielenie widoków
    własnego profilu i profili innych użytkowników, dodanie
    przycisków akcji (follow/unfollow, zaproszenie do znajomych),
    usprawnienie nawigacji z kart znajomych i \glslink{sidebar}{sidebara} oraz aktualizacja testów.
    \item Dodanie sekcji \textit{„Ulubione \glslink{spot}{spoty}”} w panelu konta
    (lista, filtrowanie, usuwanie oraz podgląd na mapie), wprowadzenie
    współdzielonych komponentów i modeli dla tej funkcji oraz
    uporządkowanie przestarzałego kodu i modeli współrzędnych.
    \item Przebudowa funkcjonalności społecznościowej: rozdzielenie
    widoków sekcji \textit{social} dla właściciela profilu i osoby
    oglądającej, aktualizacja modelu danych i routingu oraz dodanie
    nawigacji z liczników profilu do odpowiednich list (friends,
    followers, followed), wraz z wprowadzeniem osobnego \gls{redux-slice}
    do zarządzania aktywnym typem listy social.
\end{zadaniaosoby}

\begin{zadaniaosoby}{Stanisław Oziemczuk}
    \item Refaktoryzacja komentarzy do \glslink{spot}{spotów} – zastąpienie ogólnego
    modelu szczegółowym, wprowadzenie nowych
    komponentów \gls{ui}, uporządkowanie warstwy \gls{api} oraz poprawa wyglądu
    i układu sekcji komentarzy.
\end{zadaniaosoby}

\subsection*{Lipiec 2025}

\begin{zadaniaosoby}{Mateusz Redosz}
    \item Dodanie nowej sekcji \textit{„Zdjęcia”} w panelu użytkownika,
    z możliwością sortowania i filtrowania po dacie, aktualizacja
    routingu i styli oraz dopisanie testów.
    \item Dodanie sekcji komentarzy w panelu konta (z grupowaniem
    po dacie i \glslink{spot}{spocie} oraz filtrowaniem/sortowaniem po dacie).
    \item Dodanie strony ustawień konta umożliwiającej edycję nazwy
    użytkownika, adresu e-mail i hasła. Aktualizacja routingu
    i modeli (nowe enumy/interfejsy, wariant przycisku), usunięcie
    starych \glslink{endpoint}{endpointów} edycji danych oraz dodanie zależności potrzebnych do obsługi formularzy.
    \item Ujednolicenie obsługi zdjęć i filmów do wspólnego modelu
    \textit{media}, dodanie nowej sekcji \textit{„Filmy”}
    w panelu użytkownika (routing, \glslink{endpoint}{endpointy}, komponenty UI) oraz
    refaktoryzacja istniejących widoków i DTO tak, aby korzystały
    z nowych, współdzielonych struktur.
\end{zadaniaosoby}

\begin{zadaniaosoby}{Adam Langmesser}
    \item Dodanie pobierania danych czatu i wysyłania wiadomości po \gls{websocket}.
    \item Uporządkowanie formatowania kodu z wykorzystaniem
    narzędzia \gls{prettier}, dodanie skryptu \texttt{format:check}
    oraz sprawdzania formatowania w potokach \gls{cicd}.
    \item Ujednolicenie modelu danych czatu, uproszczenie logiki komunikacji z
    \gls{api} i struktury store'a \gls{redux}, dostosowanie komponentów czatu
    do nowego modelu oraz poprawa wyglądu czatu.
    \item Poprawa danych deweloperskich w bazie (m.in. przykładowych
    \glslink{spot}{spotów} i użytkowników).
    \item Usprawnienie mechanizmu nieskończonego przewijania i nazewnictwa
    listy czatów.
    \item Stworzenie ogólnej logiki i komponentów pomocniczych
    umożliwiających wszystkim członkom zespołu wygodne korzystanie
    z \glslink{websocket}{websocketów} na froncie.
    \item Dodanie grupowania wiadomości na czacie po ich dacie wysłania,
    implementacja wielowierszowego pola tekstowego.
    \item Wprowadzenie drobnych poprawek w potokach \gls{cicd}.
\end{zadaniaosoby}

\begin{zadaniaosoby}{Stanisław Oziemczuk}
    \item Zastąpienie dotychczasowych filtrów wyszukiwania \glslink{spot}{spotów} na mapie nowym paskiem wyszukiwania po nazwie
    z bocznym panelem wyników i \glslink{paginacja}{paginacją} oraz dopracowanie interfejsu
    (gwiazdki ocen) i logiki \glslink{cache}{cache'owania} zapytań.
    \item Ujednolicenie obsługi mediów dla \glslink{spot}{spotów} i komentarzy oraz
    implementacja wyświetlania zdjęć i filmów dla \glslink{spot}{spotów}.
    \item Refaktoryzacja struktury plików na \glslink{backend}{backendzie} związanych
    z modułem mapy.
    \item Dodanie możliwości wyszukiwania \glslink{spot}{spotów} na mapie w obecnie widocznym dla użytkownika obszarze.
    \item Dodanie zdjęć i filmów do \glslink{spot}{spotów} w celach deweloperskich.
\end{zadaniaosoby}

\begin{zadaniaosoby}{Kacper Badek}
    \item Zapoznanie się z dokumentacją \gls{tinymce} \gls{rich-text-editor}.
\end{zadaniaosoby}

\subsection*{Sierpień 2025}

\begin{zadaniaosoby}{Adam Langmesser}
    \item Dodanie do czatu możliwości wyszukiwania i wysyłania animowanych
    obrazów (\glslink{gif}{gifów}) oraz wysyłania \gls{emoji}.
    \item Usprawnienie czatu poprzez wprowadzenie optymistycznego
    wysyłania wiadomości, dodanie nowych \glslink{endpoint}{endpointów}
    do stronicowanego pobierania starszych wiadomości.
\end{zadaniaosoby}

\begin{zadaniaosoby}{Stanisław Oziemczuk}
    \item Dodanie jednoznacznego „punktu środka” \glslink{spot}{spota} i konsekwentne wykorzystywanie
    go w \glslink{backend}{backendzie} i \glslink{frontend}{frontendzie}.
\end{zadaniaosoby}

\begin{zadaniaosoby}{Mateusz Redosz}
    \item Dodanie na stronie głównej karuzeli z najpopularniejszymi
    \glslink{spot}{spotami} oraz wyszukiwarki \glslink{spot}{spotów} po lokalizacji, z listą wyników i dystansem od użytkownika.
    \item Dodanie zaawansowanego wyszukiwania \glslink{spot}{spotów} na stronie głównej.
    \item Wprowadzenie \glslink{paginacja}{paginacji} do \glslink{endpoint}{endpointów} panelu użytkownika
    i przebudowa powiązanych komponentów \glslink{frontend}{frontendu} na nieskończone
    przewijanie.
    \item Dodanie sekcji \textit{„Zdjęcia”} w panelu społecznościowym
    użytkownika oraz uproszczenie logiki nieskończonego przewijania
    dla różnych zakładek social.
    \item Dodanie do panelu użytkownika funkcji dodawania własnych
    \glslink{spot}{spotów} oraz widoku listy dodanych \glslink{spot}{spotów}.
\end{zadaniaosoby}

\begin{zadaniaosoby}{Kacper Badek}
    \item Przygotowanie formularza dodawania postów z wykorzystaniem edytora \gls{tinymce}.
    \item Skonfigurowanie biblioteki \gls{jsoup} na \glslink{backend}{backendzie}.
\end{zadaniaosoby}

\subsection*{Wrzesień 2025}

\begin{zadaniaosoby}{Adam Langmesser}
    \item Wprowadzenie możliwości rozpoczynania lub kontynuowania
    prywatnych rozmów czatowych bezpośrednio z list znajomych
    i obserwujących w zakładce „Social”.
    \item Dodanie obsługi \gls{emoji} na czacie oraz poprawa wyglądu okna do wysyłania \glslink{gif}{gifów}.
\end{zadaniaosoby}

\begin{zadaniaosoby}{Mateusz Redosz}
    \item Refaktoryzacja systemu informacji o błędach i sukcesach na \glslink{frontend}{frontendzie}, umożliwiająca
    jednoczesne wyświetlanie wielu komunikatów oraz zwiększająca
    modularność i możliwość ponownego wykorzystania komponentów.
    \item Wprowadzenie \glslink{paginacja}{paginacji} dla wyszukiwarki \glslink{spot}{spotów}.
    \item Dodanie walidacji formularza dodawania \glslink{spot}{spota},
    a także wyświetlanie komunikatów błędów.
    \item Rozszerzenie komponentu przesyłania multimediów o podgląd
    wybranych obrazów i materiałów wideo.
    \item Wprowadzenie obsługi zmiany zdjęcia profilowego
    użytkownika.
    \item Refaktoryzacja struktury projektu części opisowej pracy inżynierskiej –
    zastąpienie dotychczasowej treści demonstracyjnej rzeczywistymi
    rozdziałami opisującymi projekt.
    \item Rozszerzenie zaawansowanych możliwości wyszukiwania \glslink{spot}{spotów}
    o sortowanie wyników według popularności lub oceny oraz filtrowanie
    po minimalnej ocenie.
\end{zadaniaosoby}

\begin{zadaniaosoby}{Stanisław Oziemczuk}
    \item Implementacja kompletnej funkcjonalności pogody dla \glslink{spot}{spotów}:
    dodanie podstawowego i szczegółowego modalu pogodowego
    (\gls{modal}) oraz zestawu komponentów interfejsu prezentujących
    m.in. temperaturę, prędkość wiatru, opady oraz dodatkowe
    parametry meteorologiczne.
    \item Przebudowa obsługi pogody tak, aby zamiast bezpośrednich
    wywołań publicznego \gls{api} wykorzystywany był \glslink{backend}{backend} jako warstwa
    pośrednia.
    \item Implementacja wyświetlania zdjęcia profilowego użytkownika w komentarzach \glslink{spot}{spota}.
\end{zadaniaosoby}

\begin{zadaniaosoby}{Kacper Badek}
    \item Zamiana edytora \gls{tinymce} na \gls{tiptap}.
    \item Implementacja przeglądania postów na forum oraz ich sortowania.
    \item Wprowadzenie czytelniejszych adresów \gls{url} opartych na \glslink{slug}{slugach},
    poprawa walidacji dodawania nowego posta.
    \item Poprawa ergonomii poruszania się po forum.
\end{zadaniaosoby}

\section{Etap 3 (październik 2025 – styczeń 2026)}
\label{sec:etap3}

Etap 3 obejmował finalizację prac nad systemem,
dopracowanie dokumentacji technicznej i tekstowej pracy inżynierskiej
oraz przygotowanie projektu do oddania.
W tym czasie większość nowych funkcjonalności była już zaimplementowana,
a nacisk położono na stabilizację, testy oraz spójny opis
w dokumentacji.

W ramach przedmiotu \gls{psem}, prowadzonego przez dr. hab. Marka
Bednarczyka, postępy w przygotowywaniu dokumentacji były na bieżąco
konsultowane, a na podstawie uzyskiwanej informacji zwrotnej wprowadzano
kolejne poprawki i uzupełnienia.
Analogiczny tryb pracy przyjęto w ramach przedmiotu \gls{prz2}, którego
opiekunem był promotor pracy, mgr. inż. Adam Urbanowicz.

\subsection*{Październik 2025}

\begin{zadaniaosoby}{Adam Langmesser}
    \item Wprowadzenie możliwości tworzenia czatów grupowych,
    w tym obsługi wyboru uczestników, komunikacji z \glslink{backend}{backendem}
    oraz integracji z istniejącą listą czatów.
    \item Dodanie funkcjonalności wysyłania i wyświetlania załączników
    w wiadomościach czatu (pliki oraz obrazy), wraz z logiką wyboru,
    podglądu i wysyłania samych plików bez treści tekstowej.
    \item Rozszerzenie istniejącej funkcjonalności czatów grupowych
    o możliwość edycji nazwy oraz obrazu czatu
    po stronie \glslink{backend}{backendu} i \glslink{frontend}{frontendu}.
    \item Dodanie możliwości dołączania nowych użytkowników
    do istniejących czatów grupowych.
\end{zadaniaosoby}

\begin{zadaniaosoby}{Mateusz Redosz}
    \item Rozbudowa systemu statusów znajomych w części społecznościowej
    aplikacji, w tym rozróżnienie zaproszeń wysłanych, otrzymanych
    oraz relacji zakończonych, a także dostosowanie interfejsu
    do prezentacji odpowiednich komunikatów i akcji.
    \item Wprowadzenie zaawansowanego zarządzania zaproszeniami
    do znajomych: dodanie modalnego widoku listy zaproszeń, obsługi
    akceptowania i odrzucania oraz integracji z dedykowanymi
    \glslink{endpoint}{endpointami} \glslink{backend}{backendowymi}.
    \item Dodanie funkcji \textit{„Dodaj znajomego”} w sekcji
    społecznościowej, obejmującej wyszukiwarkę użytkowników
    (\gls{paginacja}, wyszukiwanie po nazwie użytkownika) oraz spójny
    wygląd modalnego okna wyszukiwania.
    \item Rozbudowa komponentu przycisku przesyłania plików
    o możliwość podglądu wielu plików, nadawania im unikalnych
    identyfikatorów oraz usuwania pojedynczych plików przed wysłaniem.
    \item Przygotowanie struktury rozdziałów w \glslink{latex}{latexie}.
    \item Opracowanie rozdziału ~\ref{subsec:mateusz-redosz} \textit{\nameref{subsec:mateusz-redosz}}.
\end{zadaniaosoby}

\begin{zadaniaosoby}{Stanisław Oziemczuk}
    \item Wprowadzenie rozszerzonej galerii multimediów dla \glslink{spot}{spotów},
    obejmującej obsługę \glslink{paginacja}{paginacji}, podglądu w trybie pełnoekranowym
    oraz dodatkowych akcji (np. udostępnianie odnośnika do zasobu).
    \item Dostosowanie modułu mapy oraz widoku szczegółów \glslink{spot}{spotów}
    pod kątem responsywności i skalowania na dużych ekranach.
    \item Poprawa responsywności panelu pogodowego.
\end{zadaniaosoby}

\begin{zadaniaosoby}{Kacper Badek}
    \item Dodanie obsługi komentarzy do postów (dodawanie, edycję,
    usuwanie oraz głosowanie).
    \item Zastąpienie klasycznych wskaźników ładowania
    loaderami typu \gls{skeleton-loader} dla listy postów oraz paneli
    kategorii i tagów.
    \item Poprawa działania stanu formularza dodawania postów.
    \item Dodanie możliwości zgłaszania postów i komentarzy.
    \item Implementacja możliwości obserwowania postów.
\end{zadaniaosoby}

\subsection*{Listopad 2025}

%TODO: do uzupełnienia
TODO: do uzupełnienia

\subsection*{Grudzień 2025}

%TODO: do uzupełnienia
TODO: do uzupełnienia

\subsection*{Styczeń 2026}

%TODO: do uzupełnienia
TODO: do uzupełnienia

Na zakończenie prac projektowych przyjęto datę
\textbf{10 stycznia 2026 roku}.
