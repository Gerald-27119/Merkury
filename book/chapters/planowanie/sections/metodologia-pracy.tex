%! Author = mateusz
%! Date = 17/10/2025

\section{Metodologia pracy}
\label{sec:metodologia-pracy}

\subsection{Przegląd rozważanych podejść}
\label{subsec:przeglad-rozwazanych-podejsc}

Przy wyborze metodologii pracy rozważono trzy podejścia do prowadzenia projektu informatycznego:
\begin{itemize}
    \item klasyczny Agile (w praktyce: Scrum),
    \item model kaskadowy (waterfall),
    \item Disciplined Agile Delivery (DAD).
\end{itemize}

\subsection{Odrzucone podejścia}
\label{subsec:odrzucone-podejscia}

\paragraph{„Klasyczny Agile” (Scrum).}
Mimo elastyczności i popularności zakłada pracę w iteracjach 2--4 tygodni oraz stały zestaw ceremonii (planowanie, przegląd, retrospektywa).
Ze względu na nierównomierną dostępność zasobów w kolejnych miesiącach studiów nie zapewniono możliwości utrzymania stałej kadencji sprintów, dlatego z podejścia zrezygnowano.

\paragraph{Model kaskadowy (waterfall).}
Przewiduje sekwencyjne przechodzenie przez z góry określone etapy i ogranicza bieżącą weryfikację wymagań w trakcie prac deweloperskich.
W projekcie wymagano możliwości częstych rewizji założeń oraz wprowadzania istotnych zmian w docelowej wizji rozwiązania; dlatego z podejścia zrezygnowano.

\subsection{Wybrane podejście: Disciplined Agile Delivery (Lean Life Cycle)}
\label{subsec:wybrane-podejscie-dad-lean}

Podjęto decyzję o zastosowaniu \textbf{Disciplined Agile Delivery} w wariancie \textbf{Lean Life Cycle}, ponieważ podejście to łączy pożądane cechy Agile i waterfall, a jednocześnie eliminuje stałe sprinty na rzecz pracy w ciągłym przepływie.

\paragraph{Kluczowe argumenty wyboru:}
\begin{itemize}
    \item \textbf{Brak sprintów.} Zastosowano przepływ ciągły (flow) z ograniczeniami WIP, co pozwala dopasować tempo do zmiennej dostępności zespołu i unikać sztucznego „domykania” iteracji.
    \item \textbf{Rozbudowana faza startowa (Inception).} Na początku przewidziano większy wysiłek planistyczny: doprecyzowanie zakresu, wstępna wizja architektury, identyfikacja ryzyk, plan publikacji oraz kryteria jakości -- bez zamrażania szczegółów.
    \item \textbf{Ciągła weryfikacja wymagań.} W trakcie realizacji przewidziano bieżące doprecyzowywanie backlogu, regularny feedback promotora i interesariuszy oraz możliwość korygowania kierunku bez kosztów „przeskakiwania” między fazami.
    \item \textbf{Praktyki Lean i koncentracja na wartości.} Priorytetyzacja wartości biznesowej, wizualizacja pracy, małe partie dostaw, pomiar lead/cycle time oraz systematyczna optymalizacja przepływu.
    \item \textbf{Lekka governance i kamienie milowe.} Zastosowano lekkie mechanizmy nadzoru (np. peer review, przeglądy architektury, prezentacje postępów) zapewniające przejrzystość bez nadmiernej biurokracji.
    \item \textbf{Architektura ewolucyjna.} Wykorzystano podejście \emph{just-enough design} (spikes, prototypy) w celu wczesnego ograniczania ryzyk technologicznych i stopniowego uszczegóławiania rozwiązania.
    \item \textbf{Jakość wbudowana w proces.} Utrzymano ciągłą integrację i testowanie, jasną definicję „ukończenia” oraz standardy techniczne, co umożliwia bezpieczne i częste publikacje.
\end{itemize}

\subsection{Narzędzia i komunikacja}
\label{subsec:narzedzia-komunikacja}

Do zarządzania zadaniami zastosowana zostanie \textbf{Jira} (monitorowanie postępu prac oraz ewidencja zadań członków zespołu). Komunikację w zespole zaplanowano w formie regularnych spotkań oraz asynchronicznie z wykorzystaniem \textbf{Discorda}.

\subsection{Podział ról w zespole}
\label{subsec:podzial-rol}

\begin{itemize}
    \item \textbf{Adam} -- lider zespołu; kieruje przepływem pracy i nadaje priorytety; full-stack developer.
    \item \textbf{Stanisław} -- full-stack developer.
    \item \textbf{Kacper} -- full-stack developer.
    \item \textbf{Mateusz} -- full-stack developer.
\end{itemize}

Każdy z członków zespołu uczestniczy również w przygotowaniu dokumentacji.

\section{Harmonogram projektu}
\label{sec:harmonogram-projektu2}

W poniższym harmonogramie przedstawiono plan prac nad poszczególnymi częściami projektu, rozłożony na miesiące.

\begin{figure}[!htbp]
    \centering
    \fbox{\parbox{0.95\linewidth}{\centering \vspace{2.5cm}
    \textit{Miejsce na ilustrację harmonogramu (np. wykres Gantta).}
    \vspace{2.5cm}}}
    \caption{Wizualizacja harmonogramu (do uzupełnienia).}
    \label{fig:harmonogram-gantt}
\end{figure}

\subsection*{Rok 2024}
\begin{description}
    \item[Czerwiec] Zebranie zespołu; rozważenie potencjalnych pomysłów.
    \item[Lipiec] Wybór technologii; wstępne założenia architektoniczne.
    \item[Sierpień] Realizacja dokumentacji.
    \item[Wrzesień] \emph{(do uzupełnienia)}
    \item[Październik] \emph{(do uzupełnienia)}
    \item[Listopad] \emph{(do uzupełnienia)}
    \item[Grudzień] \emph{(do uzupełnienia)}
\end{description}

\subsection*{Rok 2025}
\begin{description}
    \item[Styczeń] \emph{(do uzupełnienia)}
    \item[Luty] \emph{(do uzupełnienia)}
    \item[Marzec] \emph{(do uzupełnienia)}
    \item[Kwiecień] \emph{(do uzupełnienia)}
    \item[Maj] \emph{(do uzupełnienia)}
    \item[Czerwiec] \emph{(do uzupełnienia)}
    \item[Lipiec] \emph{(do uzupełnienia)}
    \item[Sierpień] \emph{(do uzupełnienia)}
    \item[Wrzesień] \emph{(do uzupełnienia)}
    \item[Październik] \emph{(do uzupełnienia)}
    \item[Listopad] \emph{(do uzupełnienia)}
    \item[Grudzień] \emph{(do uzupełnienia)}
\end{description}

\subsection*{Rok 2026}
\begin{description}
    \item[Styczeń] \emph{(do uzupełnienia)}
\end{description}
