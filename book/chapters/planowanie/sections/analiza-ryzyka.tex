%! Author = kacper
%! Date = 31/12/2025

\section{Analiza ryzyka}
\label{sec:analiza-ryzyka}

Analiza ryzyka stanowi kluczowy element każdego projektu inżynierskiego.
W tabeli poniżej przedstawiono najważniejsze zidentyfikowane ryzyka wraz z ich
możliwymi konsekwencjami oraz sposobami minimalizacji.

\newcommand{\riskcard}[6]{
    \refstepcounter{integrationcard}
    \begin{center}
    \renewcommand{\arraystretch}{1.15}
    \begin{tabularx}{\textwidth}{|
            >{\columncolor{lightgray}\raggedright\arraybackslash}p{0.19\textwidth}|X|}
    \rowcolor{lightgray}
    \multicolumn{2}{|c|}{\textbf{KARTA ANALIZY RYZYKA}} \\ \hline
    \textbf{Kontekst zagrożenia:} & #2 \\ \hline
    \textbf{Czynnik zagrożenia:} & #3 \\ \hline
    \textbf{Prawdopodo\-bieństwo:} & #4 \\ \hline
    \textbf{Wpływ:} & #5 \\ \hline
    \textbf{Sposób minimalizacji:} & #6 \\ \hline
    \end{tabularx}
    \vspace{3pt}
    \textbf{Tabela \theintegrationcard:} Karta analizy ryzyka: #2\label{#1}
    \end{center}
    \addcontentsline{lot}{table}{Tabela \theintegrationcard: Karta analizy ryzyka: #2}
}

\Needspace{12\baselineskip}
\riskcard
{tab:risk-performance}
{Problemy z wydajnością}
{Spadek wydajności aplikacji przy zwiększonej liczbie użytkowników.}
{Niskie}
{Średni}
{Optymalizacja kodu oraz przeprowadzanie testów wydajnościowych.}

\Needspace{12\baselineskip}
\riskcard
{tab:risk-acceptance}
{Brak akceptacji użytkowników}
{Niewystarczające zainteresowanie aplikacją ze strony potencjalnych użytkowników.}
{Niskie}
{Wysoki}
{Projektowanie aplikacji z uwzględnieniem potrzeb użytkowników oraz analiza opinii społeczności.}

\Needspace{12\baselineskip}
\riskcard
{tab:risk-experience}
{Brak doświadczenia zespołu}
{Ograniczone doświadczenie w realizacji projektów o podobnej skali i złożoności.}
{Średnie}
{Średni}
{Stopniowe rozwijanie aplikacji oraz konsultacje z promotorem.}

\Needspace{12\baselineskip}
\riskcard
{tab:risk-content}
{Publikacja nieodpowiednich treści}
{Ryzyko zamieszczania przez użytkowników treści niezgodnych z regulaminem.}
{Średnie}
{Średni}
{Wprowadzenie mechanizmów moderacji treści oraz możliwości zgłaszania naruszeń.}

\Needspace{12\baselineskip}
\riskcard
{tab:risk-security}
{Błędy bezpieczeństwa}
{Możliwość wystąpienia podatności związanych z ochroną danych użytkowników.}
{Niskie}
{Wysoki}
{Stosowanie dobrych praktyk bezpieczeństwa oraz regularne testy zabezpieczeń.}

