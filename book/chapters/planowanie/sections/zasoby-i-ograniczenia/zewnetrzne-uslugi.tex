%! Author = Adam
%! Date = 11/2/2025

\subsection{Usługi zewnętrzne}
\label{subsec:uslugi-zewnetrzne}
\begingroup
\newcommand{\integrationcard}[5]{%
    \begin{table}[H]
    \centering
    {\renewcommand{\arraystretch}{1.15}%
    \begin{tabularx}{\textwidth}{|p{0.22\textwidth}|X|}
    \hline
    \textbf{Usługa} & #2 \\ \hline
    \textbf{Opis}   & #3 \\ \hline
    \textbf{Limit}  & #4 \\ \hline
    \end{tabularx}}
    \caption{Karta integracji: #5\label{#1}}
    \end{table}%
}

\integrationcard{tab:github-actions}
{GitHub Actions (CI)~\cite{github-actions-billing}}
{Uruchomienia pipeline’ów CI/CD dla repozytorium GitHub.}
{3000 min/mies. (wg planu GitHub Education)}
{GitHub Actions (CI)}

\integrationcard{tab:azure-blob}
{Azure Blob Storage~\cite{azure-blob-scalability}}
{Magazyn obiektów na pliki (np. zdjęcia spotów, załączniki z czatu).}
{TODO}
{Azure Blob Storage}

\integrationcard{tab:mailtrap}
{Mailtrap (Email Testing/API)~\cite{mailtrap-limits}}
{Środowisko testowe SMTP oraz Email API do wysyłki powiadomień na środowiskach nieprodukcyjnych.}
{TODO}
{Mailtrap (Email Testing/API)}

\integrationcard{tab:locationiq}
{LocationIQ (Geocoding)~\cite{locationiq-pricing}}
{Geokodowanie/adresy i autosugestia przy dodawaniu nowych spotów.}
{5\,000 zapytań/dzień (plan Free)}
{LocationIQ (Geocoding)}

\integrationcard{tab:google-maps}
{Google Maps Platform — nawigacja/URL~\cite{google-maps-pricing}}
{Otwieranie nawigacji w aplikacji Map (deep link/URL).}
{TODO}
{Google Maps Platform — nawigacja/URL}

\integrationcard{tab:openfreemap}
{OpenFreeMap (OSM tiles)~\cite{openfreemap-docs,openfreemap-quickstart}}
{Publiczny serwer kafelków do renderu mapy na froncie.}
{TODO}
{OpenFreeMap (OSM tiles)}

\integrationcard{tab:open-meteo}
{Open\mbox{-}Meteo (Weather API)~\cite{open-meteo-usage}}
{Prognozy pogody wyświetlane dla spotów.}
{10\,000 zapytań/dzień (plan Free)}
{Open-Meteo (Weather API)}

\integrationcard{tab:tenor}
{Tenor GIF API~\cite{tenor-docs}}
{Wyszukiwanie GIF-ów w czacie.}
{1 request na sekundę; brak ogólnego limitu}
{Tenor GIF API}

\integrationcard{tab:wheretheiss}
{„Where the ISS at?”~\cite{wheretheiss-docs}}
{Proste HTTP API udostępniające bieżące dane o pozycji; wykorzystywane pomocniczo.}
{TODO}
{„Where the ISS at?”}
\endgroup
