%! Author = Adam
%! Date = 11/2/2025

\subsection{Usługi zewnętrzne}
\label{subsec:uslugi-zewnetrzne}
\begingroup
\newcommand{\integrationcard}[5]{
    \refstepcounter{integrationcard}
    \begin{center}
    {\renewcommand{\arraystretch}{1.15}
    \begin{tabularx}{\textwidth}{|p{0.22\textwidth}|X|}
    \hline
    \textbf{Usługa} & #2 \\ \hline
    \textbf{Opis}   & #3 \\ \hline
    \textbf{Limit}  & #4 \\ \hline
    \end{tabularx}}\\[2pt]
    \textbf{Tabela \theintegrationcard:} Usługa zewnętrzna: #5%
    \label{#1}%
    % jeśli chcemy mieć je w spisie tabel:
    \addcontentsline{lot}{table}{Tabela \theintegrationcard: Usługa zewnętrzna: #5}
    \end{center}%
}

\integrationcard{tab:github-actions}
{GitHub Actions (CI)~\cite{github-actions-billing}}
{Uruchomienia pipeline’ów CI/CD dla repozytorium GitHub.}
{3000 min/mies.}
{GitHub Actions (CI)}

\integrationcard{tab:azure-blob}
{Azure Blob Storage~\cite{azure-blob-scalability}}
{Magazyn plików (m.in. zdjęcia spotów, załączniki z czatu).}
{1 GB/mies.}
{Azure Blob Storage}

\integrationcard{tab:mailtrap}
{Mailtrap~\cite{mailtrap-limits}}
{Środowisko testowe SMTP oraz Email API do wysyłki maili}
{150 maili/dzień}
{Mailtrap}

\integrationcard{tab:locationiq}
{LocationIQ~\cite{locationiq-pricing}}
{Geokodowanie/adresy i autosugestia przy dodawaniu nowych spotów.}
{5\,000 zapytań/dzień}
{LocationIQ)}

\integrationcard{tab:google-maps}
{Google Maps API~\cite{google-maps-pricing}}
{Otwieranie nawigacji w aplikacji Map (deep link/URL).}
{50\,000 zapytań/mies.}
{Google Maps API}

\integrationcard{tab:openfreemap}
{OpenFreeMap~\cite{openfreemap-docs,openfreemap-quickstart}}
{Publiczny serwer kafelków do renderu mapy na froncie.}
{30\,000 zapytań/mies.}
{OpenFreeMap}

\integrationcard{tab:open-meteo}
{Open\mbox{-}Meteo~\cite{open-meteo-usage}}
{Prognozy pogody wyświetlane dla spotów.}
{10\,000 zapytań/dzień}
{Open-Meteo}

\integrationcard{tab:tenor}
{Tenor GIF API~\cite{tenor-docs}}
{Wyszukiwanie GIF-ów w czacie.}
{1 request na sekundę; brak ogólnego limitu}
{Tenor GIF API}

\integrationcard{tab:wheretheiss}
{„Where the ISS at?”~\cite{wheretheiss-docs}}
{Proste HTTP API udostępniające bieżące dane o pozycji; wykorzystywane pomocniczo.}
{1 request na sekundę; brak ogólnego limitu}
{„Where the ISS at?”}
\endgroup
