%! Author = Adam
%! Date = 11/2/2025

\subsection{Usługi zewnętrzne}
\label{subsec:uslugi-zewnetrzne}

% Makro karty integracji
\newcommand{\integrationcard}[5]{%
    \refstepcounter{integrationcard}%
    \par\begin{center}
    \renewcommand{\arraystretch}{1.15}%
    \begin{tabularx}{\textwidth}{|p{0.22\textwidth}|X|}
    \hline
    \textbf{Usługa} & #2 \\ \hline
    \textbf{Opis}   & #3 \\ \hline
    \textbf{Limit}  & #4 \\ \hline
    \end{tabularx}
    \vspace{3pt}
    \textbf{Tabela \theintegrationcard:} Usługa zewnętrzna: #5\label{#1}
    \end{center}%
    \addcontentsline{lot}{table}{Tabela \theintegrationcard: Usługa zewnętrzna: #5}%
}

\integrationcard{tab:github-actions}
{GitHub Actions (CI)~\cite{github-actions-billing}}
{Uruchomienia pipeline’ów CI/CD dla repozytorium GitHub.}
{3000 min/mies.}
{GitHub Actions (CI)}

\integrationcard{tab:azure-blob}
{Azure Blob Storage~\cite{azure-blob-scalability}}
{Magazyn plików (m.in. zdjęcia spotów, załączniki z czatu).}
{1 GB/mies.}
{Azure Blob Storage}

\integrationcard{tab:mailtrap}
{Mailtrap~\cite{mailtrap-limits}}
{Środowisko testowe SMTP oraz Email API do wysyłki maili.}
{150 maili/dzień}
{Mailtrap}

\integrationcard{tab:locationiq}
{LocationIQ~\cite{locationiq-pricing}}
{Geokodowanie adresu przy dodawaniu nowych spotów.}
{5\,000 zapytań/dzień}
{LocationIQ}

\integrationcard{tab:google-maps}
{Google Maps (Maps URLs)~\cite{google-maps-urls}}
{Otwieranie nawigacji w aplikacji Map Google (deep link/URL).}
{Brak limitu w ramach dokumentowanego sposobu użycia.}
{Google Maps (Maps URLs)}

\integrationcard{tab:openfreemap}
{OpenFreeMap~\cite{openfreemap-docs,openfreemap-quickstart}}
{Publiczny serwer kafelków do renderu mapy na froncie.}
{30\,000 zapytań/mies.}
{OpenFreeMap}

\integrationcard{tab:open-meteo}
{Open\mbox{-}Meteo~\cite{open-meteo-usage}}
{Prognozy pogody wyświetlane dla spotów.}
{10\,000 zapytań/dzień}
{Open-Meteo}

\integrationcard{tab:tenor}
{Tenor GIF API~\cite{tenor-docs}}
{Wyszukiwanie GIF-ów w czacie.}
{1 zapytanie na sekundę; brak ogólnego limitu dziennego.}
{Tenor GIF API}

\integrationcard{tab:wheretheiss}
{Where the ISS at?~\cite{wheretheiss-docs}}
{HTTP API z bieżącą pozycją satelity, używane pomocniczo.}
{1 zapytanie na sekundę; brak ogólnego limitu dziennego.}
{Where the ISS at?}
