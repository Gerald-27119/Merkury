%! Author = Stanisław Oziemczuk
%! Date = 08/11/2025

\subsection{Narzędzia}
\label{subsec:narzedzia}

Do niektórych płatnych narzędzi mieliśmy bezpłatnhy dostęp za pośrednictwem uczelni, w innych mogliśmy założyć konta
edukacyjne, które oferowały dostęp do wszyskich funkcji narzędzia.
Gdy żadna z wymienionych opcji nie była udostępniona, wybieraliśmy rozwiązania darmowe.

\begin{itemize}
    \item \textbf{IntelliJ IDEA Ultimate}

    Jest to IDE od firmy JetBrains.
    Dzięki wielu dostępnym pluginom oferuje obsługę wielu języków programowania oraz innych składni.
    Pozwala również na integrację z repozytorium.
    Używamy go do programowania zarówno frontendu, jak i backendu oraz tworzenia dokumentacji w LaTeX.
    \item \textbf{Docker Desktop}
    \item \textbf{One Drive}
    \item \textbf{Jira}
    \item \textbf{GitHub}
    \item \textbf{GitHub Actions}
    \item \textbf{Discord}
    \item \textbf{Messenger}
    \item \textbf{Enterprise Architect}
    \item \textbf{Vertabelo}
    \item \textbf{Postman}
    \item \textbf{Azure Blob Storage}
    \item \textbf{GitHub Copilot}
\end{itemize}
