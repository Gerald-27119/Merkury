%! Author = Stanisław Oziemczuk
%! Date = 10/11/2025

\subsection{Narzędzia}
\label{subsec:narzedzia}

Do niektórych płatnych narzędzi mieliśmy bezpłatnhy dostęp za pośrednictwem uczelni, w innych mogliśmy założyć konta
edukacyjne, które oferowały dostęp do wszyskich funkcji narzędzia.
Gdy żadna z wymienionych opcji nie była udostępniona, wybieraliśmy rozwiązania darmowe.

\begin{itemize}
    \item \textbf{IntelliJ IDEA Ultimate}

    Jest to \gls{ide} od firmy JetBrains.
    Dzięki wielu dostępnym pluginom oferuje obsługę wielu języków programowania oraz innych składni.
    Pozwala również na integrację z repozytorium.
    Używamy go do programowania zarówno \gls{frontend}u, jak i \gls{backend}u oraz tworzenia dokumentacji w LaTeX.
    \item \textbf{Docker Desktop}

    To narzędzie do zarządzania obrazami, kontenerami oraz wolumenami Docker.
    Zawiera w sobie również silnik tej technologii.
    Wykorzystujemy je do lokalnego uruchamiana bazy danych oraz serwisu do cachowania.
    \item \textbf{One Drive}

    Usługa dysku chmurowego oferowana przez firmę Microsoft.
    Przechowujemy tam dokumenty oraz obrazy diagramów.
    \item \textbf{Azure Blob Storage}

    To rozwiązanie chmurowe Microsoft, służące do bezpiecznego przechowywania dużej ilości danych
    nieustruktyryzowanych, takich jak pliki multimedialne, dokumenty czy kopie zapasowe.
    Dane są dostępne poprzez interfejs REST API usługi Azure Storage.
    Wykorzystaliśmy go do przechowywania zdjęć profilowych użytkownika oraz multimedii (zdjęcia i flimy) ze spotów
    i forum.
    \item \textbf{Jira}

    To narzędzie firmy Atlassian do zarządzania pracami nad projektem w metodykach zwinnych.
    Do \Glslink{backlog}{Backlogu} wpisywaliśmy zadania, a na \Glslink{tablica_kanban}{tablicy Kanbanowej} rejestrowaliśmy ich statusy oraz poświęcony czas.
    \item \textbf{GitHub}

    Zdalne repozytorium służące do przechowywania i wersjonowania kodu aplikacji.
    Zamieściliśmy tam kod naszego projektu.
    Do każdego zadania tworzyliśmy osobną gałąź z właściwą nazwą, a po zakończeniu prac przeprowadzaliśmy \gls{review-kodu}.
    Następnie łączyliśmy ją do głównej gałęzi deweloperskiej.
    \item \textbf{GitHub Actions}

    To narzędzie do implementacji procesów \gls{cicd} na platformie GitHub, które
    umożliwiają automatyczne testowanie lub wdrażanie kodu.
    Uruchamiają się w reakcji na różne operacje w repozytorium, na przykład przesłanie zmian na wybraną gałąź.
    Stosowaliśmy je do automatycznego testowania i budowania projektu po każdorazowym wprowadzeniu zmian.
    \item \textbf{GitHub Copilot}

    To narzędzie sztucznej inteligencji będące asystentem programisty.
    W projekcie analizuje plik oraz pliki powiązane.
    Wykorzystywaliśmy go podczas \gls{review-kodu}.
    Copilot skanował wszystkie pliki i w komentarzach opisywał sugerowane zmiany lub potencjalne błędy.
    \item \textbf{Discord}

    Darmowa platforma komunikacyjna.
    Umożliwia udostępnienie obrazu z ekranu, komunikację głosową oraz tekstową, jak i również przesyłanie plików.
    Stosowaliśmy go do spotkań, na których omawialiśmy sprawy bieżące dotyczące projektu.
    \item \textbf{Messenger}

    Komunikator będący usługą Facebooka.
    Daje możliwość tworzenia czatów grupowych lub prywatnych, a także udostępniania plików.
    Używaliśmy go do ustalania spotkań na Discordzie oraz szybkiej komunikacji.
    \item \textbf{Postman}

    To narzędzie służące do testowania endpointów \gls{api}.
    Pozwala grupować zapytania w kolekcje, wysyłać ich różne typy oraz analizować odpowiedzi z serwera.
    Wykorzystywaliśmy go do testowania stworzonych endpointów oraz debugowania.
    \item \textbf{Figma}

    Narzędzie chmurowe do projektowania interfejsów użytkownika (\gls{ui}).
    Umożliwia zespołowe tworzenie w pełni interaktywnych prototypów.
    Wykonaliśmy w nim projekty ekranów naszej aplikacji.
\end{itemize}
