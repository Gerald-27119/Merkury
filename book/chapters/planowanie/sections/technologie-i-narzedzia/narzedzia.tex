%! Author = Stanisław Oziemczuk
%! Date = 09/11/2025

\subsection{Narzędzia}
\label{subsec:narzedzia}

Do niektórych płatnych narzędzi mieliśmy bezpłatnhy dostęp za pośrednictwem uczelni, w innych mogliśmy założyć konta
edukacyjne, które oferowały dostęp do wszyskich funkcji narzędzia.
Gdy żadna z wymienionych opcji nie była udostępniona, wybieraliśmy rozwiązania darmowe.

\begin{itemize}
    \item \textbf{IntelliJ IDEA Ultimate}

    Jest to IDE od firmy JetBrains.
    Dzięki wielu dostępnym pluginom oferuje obsługę wielu języków programowania oraz innych składni.
    Pozwala również na integrację z repozytorium.
    Używamy go do programowania zarówno frontendu, jak i backendu oraz tworzenia dokumentacji w LaTeX.
    \item \textbf{Docker Desktop}

    To narzędzie do zarządzania obrazami, kontenerami oraz wolumenami Docker.
    Zawiera w sobie również silnik tej technologii.
    Wykorzystujemy je do lokalnego uruchamiana bazy danych oraz serwisi do cachowania.
    \item \textbf{One Drive}

    Usługa dysku chmurowego oferowana przez firmę Microsoft.
    Przechowujemy tam dokumenty oraz obrazy diagramów.
    \item \textbf{Azure Blob Storage}

    To rozwiązanie chmurowe Microsoft, służące do bezpiecznego przechowywania dużej ilości danych
    nieustruktyryzowanych, takich jak pliki multimedialne, dokumenty czy kopie zapasowe.
    Dane są dostępne poprzez interfejs API REST usługi Azure Storage.
    Wykorzystujemy go do przechowywania zdjęć profilowych użytkownika oraz multimedii (zdjęcia i flimy) ze spotów
    i forum.
    \item \textbf{Jira}

    To narzędzie firmy Atlassian do zarządzania pracami nad projektem w metodykach zwinnych.
    Do Backlogu wpisywaliśmy zadania, a na tablicy Kanbanowej rejestrowaliśmy ich statusy oraz poświęcony czas.
    \item \textbf{GitHub}

    Zdalne repozytorium służące do przechowywania i wersjonowania kodu aplikacji.
    Zamieściliśmy tam kod naszego projektu.
    Do każdego zadania tworzyliśmy osobną gałąź z właściwą nazwą, a po zakończeniu prac przeprowadzaliśmy code review.
    Następnie łączyliśmy ją do głównej gałęzi deweloperskiej.
    \item \textbf{GitHub Actions}

    To narzędzie do implementacji procesów CI/CD na platformie GitHub, które
    umożliwiają automatyczne testowanie lub wdrażanie kodu.
    Uruchamiają się w reakcji na różne operacje w repozytorium, na przykład przesłanie zmian na wybraną gałąź.
    Stosowaliśmy je do automatycznego testowania i budowania projektu po każdorazowym wprowadzeniu zmian.
    \item \textbf{GitHub Copilot}

    To narzędzie sztucznej inteligencji będące asystentem programisty.
    W projekcie analizuje plik oraz pliki powiązane.
    Wykorzystywaliśmy go podczas code review.
    Copilot skanował wszystkie pliki i w komentarzach opisywał sugerowane zmiany lub potencjalne błędy.
    \item \textbf{Discord}
    \item \textbf{Messenger}
    \item \textbf{Postman}
\end{itemize}
