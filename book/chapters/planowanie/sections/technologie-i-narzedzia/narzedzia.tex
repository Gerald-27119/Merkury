%! Author = Stanisław Oziemczuk
%! Date = 10/11/2025

\subsection{Narzędzia}
\label{subsec:narzedzia}

Do niektórych płatnych narzędzi otrzymano bezpłatny dostęp za pośrednictwem uczelni, w innych istniała możliwość założenia konta
edukacyjnego, które oferowało dostęp do wszystkich funkcji narzędzia.
Gdy żadna z wymienionych opcji nie była udostępniona, wybierano rozwiązania darmowe.

\begin{itemize}
    \item \textbf{IntelliJ IDEA Ultimate}

    Jest to \gls{ide} od firmy JetBrains.
    Dzięki licznie dostępnym pluginom oferuje obsługę wielu języków programowania oraz innych składni.
    Pozwala również na integrację z repozytorium.
    Używano go do programowania zarówno \glslink{frontend}{frontendu}, jak i \glslink{backend}{backendu} oraz tworzenia dokumentacji w LaTeX.
    \item \textbf{Docker Desktop}

    To narzędzie do zarządzania obrazami, kontenerami oraz wolumenami Docker.
    Zawiera w sobie również silnik tej technologii.
    Wykorzystywano je do lokalnego uruchamiania bazy danych oraz serwisu do cachowania.
    \item \textbf{Docker Compose}

    Narzędzie, które pozwala definiować oraz uruchamiać aplikacje składające się z wielu kontenerów Docker.
    Konfiguracja serwisów, sieci i \glslink{wolumen}{wolumenów} jest ustawiana w pliku (lub plikach) YAML.
    Zastosowano je do skonfigurowania bazy danych i serwisu do \glslink{cache}{cache'owania} w środowisku deweloperskim.
    \item \textbf{One Drive}

    Usługa dysku chmurowego oferowana przez firmę Microsoft.
    Przechowywano tam dokumenty oraz obrazy diagramów.
    \item \textbf{Azure Blob Storage}

    To rozwiązanie chmurowe Microsoft, służące do bezpiecznego przechowywania dużej ilości danych
    nieustrukturyzowanych, takich jak pliki multimedialne, dokumenty czy kopie zapasowe.
    Dane są dostępne poprzez interfejs \gls{rest_api} usługi Azure Storage.
    Wykorzystywano je do przechowywania zdjęć profilowych użytkownika oraz multimedii (zdjęcia i filmy) ze \glslink{spot}{spotów}
    i forum.
    \item \textbf{Jira}

    To narzędzie firmy Atlassian do zarządzania pracami nad projektem w metodykach zwinnych.
    Do \glslink{backlog}{Backlogu} wpisywano zadania, a na \glslink{tablica_kanban}{tablicy Kanbanowej} rejestrowano ich statusy oraz poświęcony czas.
    \item \textbf{GitHub}

    Zdalne repozytorium służące do przechowywania i wersjonowania kodu aplikacji.
    Zamieszczono tam kod naszego projektu.
    Do każdego zadania tworzono osobną gałąź z właściwą nazwą, a po zakończeniu prac przeprowadzano \glslink{review-kodu}{review kodu}.
    Następnie łączono ją do głównej gałęzi deweloperskiej.
    \item \textbf{GitHub Actions}

    To narzędzie do implementacji procesów \gls{cicd} na platformie GitHub, które
    umożliwiają automatyczne testowanie lub wdrażanie kodu.
    Uruchamiają się w reakcji na różne operacje w repozytorium, na przykład przesłanie zmian na wybraną gałąź.
    Stosowano je do automatycznego testowania i budowania projektu po każdorazowym wprowadzeniu zmian.
    \item \textbf{GitHub Copilot}

    To narzędzie sztucznej inteligencji będące asystentem programisty.
    W projekcie analizuje plik oraz pliki powiązane.
    Wykorzystywano go podczas \glslink{review-kodu}{review kodu}.
    Copilot skanuje wszystkie pliki i w komentarzach opisuje sugerowane zmiany lub potencjalne błędy.
    \item \textbf{Discord}

    Darmowa platforma komunikacyjna.
    Umożliwia udostępnienie obrazu z ekranu, komunikację głosową oraz tekstową, jak i również przesyłanie plików.
    Stosowano go do spotkań, na których omawiano sprawy dotyczące projektu.
    \item \textbf{Messenger}

    Komunikator będący usługą Facebooka.
    Daje możliwość tworzenia czatów grupowych lub prywatnych, a także udostępniania plików.
    Używano go do ustalania spotkań na Discordzie oraz szybkiej komunikacji.
    \item \textbf{Postman}

    To narzędzie służące do testowania endpointów \gls{api}.
    Pozwala grupować zapytania w kolekcje, wysyłać ich różne typy oraz analizować odpowiedzi z serwera.
    Wykorzystywano go do testowania stworzonych endpointów oraz debugowania.
    \item \textbf{Figma}

    Narzędzie chmurowe do projektowania interfejsów użytkownika (\gls{ui}).
    Umożliwia zespołowe tworzenie w pełni interaktywnych prototypów.
    Wykonano w nim projekty ekranów naszej aplikacji.
    \item \textbf{Visual Paradigm}

    To narzędzie do tworzenia różnych diagramów stosowanych w inżynierii oprogramowania, takich jak \gls{uml}(~\cite{uml-def}) czy \gls{bpmn}(~\cite{bpmn-def}).
    Zrobiono w nim diagram przypadków użycia.
    \item \textbf{Xmind}

    Narzędzie służące do tworzenia mapy myśli.
    Wykorzystano je w celu lepszego zrozumienia problemów poprzez przeniesienie ich na diagram.
\end{itemize}
