%! Author = Stanisław Oziemczuk
%! Date = 02/12/2025

\subsection{Technologie}
\label{subsec:technologie}

Do realizacji projektu zespół wspólnie wytypował główne technologie części \glslink{backend}{backendowej}, \glslink{frontend}{frontendowej} oraz dokumentacji.
Natomiast poszczególne biblioteki i rozwiązania były wybierane indywidualnie lub po konsultacjach przez osobę wykonującą dane zadanie.
Poniżej przedstawiono stos technologiczny zastosowany w projekcie.

\begin{itemize}
    \item \textbf{\gls{backend}}

    Na główny \glslink{framework}{framework} został wybrany SpringBoot, ponieważ spośród innych dostępnych opcji, członkowie zespołu mieli z nim
    największe doświadczenie nabyte zarówno podczas studiów, jak i później pracę komercyjną.
    Językiem programowania wykorzystywanym w SpringBoot'cie jest Java, z którym zespół zapoznał się w ramach programu nauczania.

    \begin{itemize}
        \item \textbf{Java} \textendash \space obiektowy język programowania, cechujący się silnym typowaniem.
        Programy napisane w Javie są uruchamiane na maszynie writualnej Java (\gls{jvm}), dzięki czemu można je bezproblemowo
        przenosić między różnymi platformami wyposażonymi w to środowisko.
        \item \textbf{SpringBoot} \textendash \space \glslink{framework}{framework} służący do tworzenia aplikacji opartych na \glslink{spring-framework}{Spring Framework}.
        Wykorzystuje strategię \glslink{convention-over-configuration}{Convention Over Configuration}, która zmniejsza czas na konfigurowanie Springa, pozwalając skupić się na
        implementacji logiki.
        Służy do do tworzenia między innymi aplikacji internetowych czy mikroserwisów.
    \end{itemize}
\end{itemize}

\begin{longtable}{|
        >{\columncolor{lightgray}}p{0.25\textwidth}|
    p{0.65\textwidth}|
}
    \hline
    \rowcolor{lightgray}
    \multicolumn{2}{|c|}{\textbf{Zestawienie używanych bibliotek na backendzie}} \\ \hline
    \hline
    \rowcolor{lightgray}
    \textbf{Biblioteka} & \textbf{Opis} \\ \hline
    \endfirsthead

    \hline
    \rowcolor{lightgray}
    \textbf{Biblioteka} & \textbf{Opis} \\ \hline
    \endhead

    angus-mail & Wysyłanie i odbiór wiadomości e-mail w aplikacjach Java. \\ \hline
    azure-storage-blob & Operacje na blobach w Microsoft Azure Blob Storage, np. upload, download. \\ \hline
    GeographicLib-Java & Precyzyjne obliczenia geodezyjne i konwersja współrzędnych. \\ \hline
    h2 & Lekka baza danych H2 uruchamiana w pamięci RAM używana w testach jako zastępstwo prawdziwej. \\ \hline
    httpclient5 & Zaawansowany klient HTTP do wykonywania żądań i obsługi odpowiedzi. \\ \hline
    httpcore5 & Niskopoziomowe elementy HTTP wykorzystywane przez httpclient. \\ \hline
    jjwt-api & Tworzenir i parsowanie JWT. \\ \hline
    jjwt-impl & Implementacja funkcjonalna biblioteki JJWT \textendash \space konkretne mechanizmy tworzenia/parowania JWT. \\ \hline
    jjwt-jackson & Integracja JJWT z Jacksonem dla serializacji i deserializacji zawartości JWT. \\ \hline
    jsoup & Parsowanie, manipulacja i ekstrakcja danych z dokumentów HTML. \\ \hline
    junit-jupiter & Integracja biblioteki Testcontainers z frameworkiem testowym JUnit \textendash \space ułatwia pisanie testów integracyjnych z użyciem kontenerów. \\ \hline
    lombok & Biblioteka ułatwiająca generowanie kodu (gettery, settery, buildery, konstruktor itp.) przez adnotacje \textendash \space zmniejsza ilość boilerplate’u w kodzie. \\ \hline
    postgresql & Umożliwia połączenie i komunikację z bazą danych PostgreSQL. \\ \hline
    shedlock-spring & Zarządzanie zadaniami okresowymi (cron/scheduled). \\ \hline
    spring-boot-starter-websocket & Wsparcie Spring Boot dla komunikacji WebSocket \textendash \space konfiguracja i zależności umożliwiające dwukierunkową komunikację w czasie rzeczywistym w aplikacjach webowych. \\ \hline
    spring-security-messaging & Integracja Spring Security z warstwą messaging (np. STOMP/WebSocket) \textendash \space uwierzytelnianie i autoryzacja komunikatów. \\ \hline
    spring-boot-starter-cache & Abstrakcje i automatyczna konfiguracja cache w Spring Boot \textendash \space ułatwia włączenie mechanizmów cache'owania. \\ \hline
    spring-boot-starter-data-redis & Integracja Spring Data Redis \textendash \space klient Redis, repozytoria i konfiguracje dla współpracy z Redis. \\ \hline
    spring-boot-starter-data-jpa & Warstwa dostępu do relacyjnej bazy danych przez JPA. \\ \hline
    spring-boot-starter-web & Podstawowy starter webowy Spring Boot: Spring MVC, Jackson, wbudowany serwer aplikacyjny dla REST/HTTP. \\ \hline
    spring-boot-starter-validation & Wsparcie walidacji Bean Validation (Jakarta Validation / Hibernate Validator) dla danych wejściowych. \\ \hline
    spring-boot-starter-aop & Obsługa programowania aspektowego (AOP) w Springu \textendash \space aspekty, przechwytywanie wywołań, transakcyjne przekrojowe zachowania. \\ \hline
    spring-boot-starter-actuator & Monitoring, zbieranie metryk aplikacji Spring Boot. \\ \hline
    spring-boot-starter-oauth2-resource-server & Wsparcie serwera zasobów OAuth2 w Spring Boot \textendash \space walidacja tokenów i konfiguracja zabezpieczeń zasobów. \\ \hline
    spring-boot-configuration-processor & Procesor adnotacji dla konfiguracji Spring Boot \textendash \space generacja metadanych dla właściwości konfiguracyjnych. \\ \hline
    spring-boot-starter-test & Zestaw narzędzi testowych (JUnit, Mockito, AssertJ itp.) do testów jednostkowych i integracyjnych. \\ \hline
    spring-security-test & Narzędzia pomocnicze i rozszerzenia do testowania konfiguracji Spring Security. \\ \hline
    spring-boot-starter-oauth2-client & Wsparcie klienta OAuth2 w Spring Boot \textendash \space logowanie/połączenie z zewnętrznymi providerami OAuth2/OIDC. \\ \hline
    spring-boot-starter-security & Podstawowe komponenty Spring Security do zabezpieczania aplikacji. \\ \hline
    spring-boot-starter-webflux & Budowanie reaktywnych (asynchronicznych / nieblokujących) aplikacji webowych. \\ \hline
    spring-retry & Biblioteka do automatycznego ponawiania operacji z konfiguracją i adnotacjami. \\ \hline
    spring-boot-starter-thymeleaf & Integracja Thymeleaf z Spring Boot \textendash \space silnik szablonów do generowania widoków HTML po stronie serwera. \\ \hline
    testcontainers & Uruchamianie izolowanych kontenerów Docker w testach integracyjnych. \\ \hline

\end{longtable}

\refstepcounter{integrationcard}%
\par\noindent
\textbf{Tabela \theintegrationcard:} Zestawienie wszystkich bibliotek użytych na backendzie.\label{tab:all-technologies-backend}

\addcontentsline{lot}{table}{Tabela \theintegrationcard: Zestawienie wszystkich bibliotek użytych na backendzie.}%


\begin{itemize}
    \item \textbf{\gls{frontend}}

    Do realizacji tej części projektu wybrano \glslink{biblioteka}{bibliotekę} \gls{react} JavaScript, którą wszyscy członkowie zespołu poznali w trakcie studiów
    w ramach wybranej specjalizacji Aplikacje Internetowe.

    \begin{itemize}
        \item \textbf{\gls{react}} \textendash \space \glslink{biblioteka}{biblioteka} JavaScript służąca do budowania interaktywnych interfejsów użytkownika (\gls{ui}).
        Polega na programowaniu deklaratywnym oraz tworzeniu komponentów wielokrotnego użytku.
        Nie manipuluje bezpośrednio \gls{dom}, lecz tworzy swój wirtualny \gls{dom} i porównuje jego wersje.
        Po wykryciu zmian aktualizuje tylko te części \gls{dom}, które tego wymagają, co przekłada się na wydajną interakcję z apliklacją.
        Często jest wykorzystywany do tworzenia aplikacji typu \gls{spa}.
    \end{itemize}
\end{itemize}

\begin{longtable}{|
        >{\columncolor{lightgray}}p{0.25\textwidth}|
    p{0.65\textwidth}|
}
    \hline
    \rowcolor{lightgray}
    \multicolumn{2}{|c|}{\textbf{Zestawienie używanych bibliotek na frontendzie}} \\ \hline
    \hline
    \rowcolor{lightgray}
    \textbf{Biblioteka / plugin} & \textbf{Opis / przeznaczenie} \\ \hline
    \endfirsthead

    \hline
    \rowcolor{lightgray}
    \textbf{Biblioteka / plugin} & \textbf{Opis / przeznaczenie} \\ \hline
    \endhead

    @ferrucc-io/emoji-picker & Komponent umożliwiający wybór emotikon w aplikacji. \\ \hline
    @hookform/resolvers & Adaptery walidatorów dla \texttt{react-hook-form} \textendash \space ułatwia integrację z bibliotekami walidacji (np. Zod, Yup). \\ \hline
    @reduxjs/toolkit & Narzędzie do upraszczania konfiguracji i używania Redux. \\ \hline
    @stomp/stompjs & Klient protokołu STOMP do komunikacji poprzez WebSocket \textendash \space obsługa sesji, subskrypcji i wymiany komunikatów. \\ \hline
    @tailwindcss/vite & Wtyczka integrująca Tailwind CSS z bundlerem Vite. \\ \hline
    @tanstack/react-query & Zarządzanie asynchronicznymi danymi: pobieranie, cache’owanie, synchronizacja i obsługa błędów zapytań HTTP. \\ \hline
    @tiptap/extension-file-handler & Rozszerzenie edytora Tiptap dodające obsługę wstawiania i zarządzania plikami. \\ \hline
    @tiptap/extension-image & Rozszerzenie Tiptap pozwalające na wstawianie i obsługę obrazów w edytorze. \\ \hline
    @tiptap/extension-placeholder & Rozszerzenie Tiptap dodające placeholder w polach edytora. \\ \hline
    @tiptap/extension-text-align & Rozszerzenie Tiptap umożliwiające wyrównywanie tekstu. \\ \hline
    @tiptap/pm & Silnik edytora (ProseMirror) używany wewnętrznie przez Tiptap \textendash \space obsługa modelu dokumentu i transformacji. \\ \hline
    @tiptap/react & Integracja edytora Tiptap z Reactem \textendash \space komponenty i hooki do osadzania edytora w aplikacji. \\ \hline
    @tiptap/starter-kit & Zestaw podstawowych rozszerzeń i konfiguracji dla szybkiego startu z Tiptap. \\ \hline
    @vis.gl/react-maplibre & Komponenty i helpery do implementacji map w aplikacji React. \\ \hline
    antd & Biblioteka komponentów UI dla React o ustandaryzowanym wyglądzie. \\ \hline
    axios & Obiektowy klient HTTP do wykonywania zapytań oraz obsługi odpowiedzi i błędów. \\ \hline
    date-fns & Zestaw funkcji do manipulacji i formatowania dat. \\ \hline
    dotenv & Ładowanie zmiennych środowiskowych z pliku \texttt{.env}. \\ \hline
    maplibre-gl & Silnik renderowania map (WebGL) \textendash \space warstwy map, kafelki i interakcje geograficzne. \\ \hline
    media-chrome & Zestaw elementów UI do obsługi odtwarzania multimediów w przeglądarce. \\ \hline
    motion & Narzędzia do animacji interfejsu użytkownika. \\ \hline
    query-string & Narzędzia do parsowania i generowania parametrów zapytań w URL. \\ \hline
    react & Biblioteka do budowy deklaratywnych interfejsów użytkownika oparta na komponentach. \\ \hline
    react-dom & Pakiet odpowiedzialny za renderowanie elementów React w przeglądarkowym DOM. \\ \hline
    react-hook-form & Obsługa formularzy w React: zarządzanie stanem, walidacje i wydajność. \\ \hline
    react-icons & Zbiór ikon udostępnionych jako komponenty React. \\ \hline
    react-intersection-observer & Obserwowanie wejścia/wyjścia elementów z pola widzenia (lazy-loading, animacje). \\ \hline
    react-paginate & Komponent pomocniczy do paginacji list i tabel w aplikacji React. \\ \hline
    react-player & Odtwarzanie multimediów w aplikacji. \\ \hline
    react-redux & Integracja Redux i Reacta. \\ \hline
    react-router-dom & Biblioteka do routingu w aplikacjach React. \\ \hline
    react-select & Rozszerzony komponent \texttt{select} z opcjami wyszukiwania, grupowania i stylizacji. \\ \hline
    sockjs-client & Klient WebSocket z fallbackami umożliwiający stabilniejszą komunikację w czasie rzeczywistym. \\ \hline
    tailwindcss & Definiowanie wyglądu przy pomocy gotowych klas. \\ \hline
    uuid & Generator unikalnych identyfikatorów. \\ \hline
    victory & Tworzenie wykresów i wizualizacji danych w React. \\ \hline
    victory-chart & Moduł biblioteki Victory zawierający komponenty i narzędzia do rysowania wykresów. \\ \hline
    zod & Typowana walidacja danych. \\ \hline
    @testing-library/jest-dom & Rozszerzenia matcherów dla Jesta do asercji DOM. \\ \hline
    @testing-library/react & Narzędzia do testowania komponentów React. \\ \hline
    @testing-library/user-event & Symulacja zdarzeń użytkownika (np. kliknięcie) w testach integracyjnych. \\ \hline
    @types/react & Typy TypeScript dla biblioteki React. \\ \hline
    @types/react-dom & Typy TypeScript dla \texttt{react-dom}. \\ \hline
    @types/sockjs-client & Typy TypeScript dla klienta SockJS. \\ \hline
    @typescript-eslint/eslint-plugin & Zestaw reguł ESLint specyficznych dla TypeScript. \\ \hline
    @typescript-eslint/parser & Parser ESLint umożliwiający analizę kodu TypeScript. \\ \hline
    @vitejs/plugin-react & Wtyczka Vite zapewniająca obsługę JSX/React Fast Refresh i optymalizacje. \\ \hline
    cypress & Narzędzie do testów end-to-end. \\ \hline
    eslint & Narzędzie do analizy statycznej kodu na podstawie zdefiniowanych reguł. \\ \hline
    eslint-plugin-react & Wtyczka ESLint z regułami dedykowanymi dla aplikacji React. \\ \hline
    eslint-plugin-react-hooks & Reguły ESLint dotyczące poprawnego hooków React. \\ \hline
    eslint-plugin-react-refresh & Wtyczka wspomagająca integrację z React Fast Refresh. \\ \hline
    jest & Tworzenie testów jednostkowych JavaScript. \\ \hline
    jsdoc & Narzędzie do generowania dokumentacji API z adnotacji w kodzie źródłowym. \\ \hline
    jsdom & Implementacja DOM w Node.js używana w testach do symulacji środowiska przeglądarki. \\ \hline
    prettier & Formatowanie i ujednolicenie stylu kodu. \\ \hline
    prettier-plugin-tailwindcss & Wtyczka Prettiera sortująca klasy Tailwind CSS. \\ \hline
    tailwind-scrollbar & Plugin Tailwind CSS dodający klasy pomocnicze do stylizacji pasków przewijania. \\ \hline
    typescript & Nakładka JavaScript z systemem typów. \\ \hline
    vite & Bundler zoptymalizowany pod nowoczesne aplikacje webowe. \\ \hline
    vitest & Lekki runner testowy kompatybilny z Vite \textendash \space szybkie testy jednostkowe i integracyjne, obsługa TypeScript. \\ \hline

\end{longtable}

\refstepcounter{integrationcard}%
\par\noindent
\textbf{Tabela \theintegrationcard:} Zestawienie wszystkich bibliotek i pluginów użytych na frontendzie.\label{tab:all-technologies-frontend}

\addcontentsline{lot}{table}{Tabela \theintegrationcard: Zestawienie wszystkich bibliotek i pluginów użytych na frontendzie.}%

\begin{itemize}
    \item \textbf{\gls{cache}}
    \begin{itemize}
        \item \textbf{\gls{redis}} \textendash \space z ang. REmote DIctionary Server, nierelacyjna (NoSQL) baza danych
        przechowująca dane jako pary klucz - wartość.
        Działa w pamięci RAM, dzięki czemu pozwala na bardzo szybki odczyt danych.
    \end{itemize}
    \item \textbf{Konteneryzacja}
    \begin{itemize}
        \item \textbf{Docker} \textendash \space to silinik do konteneryzacji oprogramowania.
        Tworzy środowisko oddzielone od systemu hosta, w którym na podstawie obrazów programów tworzone są ich kontenery,
        czyli niezależne ich instancje.
        Docker pobiera wszystkie niezbędne dependnecje dla danego kontenera, bez potrzeby uruchamiania osobnego systemu operacyjnego,
        dzięki czemu kontenery są znacznie lżejsze niż maszyny wirtualne.
        Konteneryzacja pozwala na uruchomienie oprogramowania na różnych komputerach dokładnie w taki sam sposób, rozwiązuje to
        problem \enquote{na mojej maszynie działa}.
        Członkowie zespołu zapoznali się tą technologią w trakcie realizacji specjalizacji Aplikacje Internetowe.
    \end{itemize}
    \item \textbf{Baza danych}
    \begin{itemize}
        \item \textbf{PostgreSQL} \textendash \space system zarządzania relacyjnymi baz danych, zgodny ze standarden SQL.
        Wybrano relacyjną bazę danych, ponieważ doskonale wpisuje się ona w planowaną strukturę danych.
    \end{itemize}
\end{itemize}
