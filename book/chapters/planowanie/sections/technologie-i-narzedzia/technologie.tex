%! Author = Stanisław Oziemczuk
%! Date = 02/12/2025

\subsection{Technologie}
\label{subsec:technologie}

Do realizacji projektu zespół wspólnie wytypował główne technologie części \glslink{backend}{backendowej}, \glslink{frontend}{frontendowej} oraz dokumentacji.
Natomiast poszczególne biblioteki i rozwiązania były wybierane indywidualnie lub po konsultacjach przez osobę wykonującą dane zadanie.
Poniżej przedstawiono stos technologiczny zastosowany w projekcie.

\begin{itemize}
    \item \textbf{\gls{backend}}

    Na główny \glslink{framework}{framework} został wybrany SpringBoot, ponieważ spośród innych dostępnych opcji, członkowie zespołu mieli z nim
    największe doświadczenie nabyte zarówno podczas studiów, jak i później pracę komercyjną.
    Językiem programowania wykorzystywanym w SpringBoot'cie jest Java, z którym zespół zapoznał się w ramach programu nauczania.

    \begin{itemize}
        \item \textbf{Java} \textendash \space obiektowy język programowania, cechujący się silnym typowaniem.
        Programy napisane w Javie są uruchamiane na maszynie writualnej Java (\gls{jvm}), dzięki czemu można je bezproblemowo
        przenosić między różnymi platformami wyposażonymi w to środowisko.
        \item \textbf{SpringBoot} \textendash \space \glslink{framework}{framework} służący do tworzenia aplikacji opartych na \glslink{spring-framework}{Spring Framework}.
        Wykorzystuje strategię \glslink{convention-over-configuration}{Convention Over Configuration}, która zmniejsza czas na konfigurowanie Springa, pozwalając skupić się na
        implementacji logiki.
        Służy do do tworzenia między innymi aplikacji internetowych czy mikroserwisów.
    \end{itemize}
\end{itemize}

\begin{longtable}{|
        >{\columncolor{lightgray}}p{0.25\textwidth}|
    p{0.65\textwidth}|
}
    \hline
    \rowcolor{lightgray}
    \multicolumn{2}{|c|}{\textbf{Zestawienie używanych bibliotek / pakietów}} \\ \hline
    \hline
    \rowcolor{lightgray}
    \textbf{Biblioteka / pakiet} & \textbf{Opis / przeznaczenie} \\ \hline
    \endfirsthead

    \hline
    \rowcolor{lightgray}
    \textbf{Biblioteka / pakiet} & \textbf{Opis / przeznaczenie} \\ \hline
    \endhead

    spring-boot-starter-websocket & Wsparcie Spring Boot dla komunikacji WebSocket (w tym STOMP) — konfiguracja i zależności umożliwiające dwukierunkową komunikację w aplikacjach webowych. \\ \hline
    spring-security-messaging & Integracja Spring Security z warstwą messaging (np. STOMP/WebSocket) — uwierzytelnianie i autoryzacja komunikatów. \\ \hline
    testcontainers & Uruchamianie izolowanych kontenerów Docker w testach integracyjnych — ułatwia testowanie zależności infrastrukturalnych. \\ \hline
    GeographicLib-Java & Biblioteka do precyzyjnych obliczeń geodezyjnych i konwersji współrzędnych (np. obliczenia geodetyczne). \\ \hline
    junit-jupiter & Moduł integracyjny Testcontainers dla JUnit 5 — ułatwia użycie Testcontainers w testach JUnit Jupiter. \\ \hline
    spring-boot-starter-cache & Abstrakcje i automatyczna konfiguracja cache w Spring Boot — ułatwia włączenie mechanizmów cache'owania. \\ \hline
    spring-boot-starter-data-redis & Integracja Spring Data Redis — klient Redis, repozytoria i konfiguracje dla współpracy z Redis. \\ \hline
    spring-boot-starter-data-jpa & Spring Data JPA (Hibernate) — warstwa dostępu do relacyjnej bazy danych przez JPA. \\ \hline
    spring-boot-starter-web & Podstawowy starter webowy Spring Boot: Spring MVC, Jackson, wbudowany serwer aplikacyjny dla REST/HTTP. \\ \hline
    spring-boot-starter-validation & Wsparcie walidacji Bean Validation (Jakarta Validation / Hibernate Validator) dla danych wejściowych. \\ \hline
    postgresql & JDBC driver dla PostgreSQL — umożliwia połączenie i komunikację z serwerem PostgreSQL. \\ \hline
    spring-boot-starter-aop & Obsługa programowania aspektowego (AOP) w Springu — aspekty, przechwytywanie wywołań, transakcyjne przekrojowe zachowania. \\ \hline
    spring-boot-starter-actuator & Endpoints monitorujące, metryki i narzędzia operacyjne dla aplikacji Spring Boot. \\ \hline
    micrometer-core & Podstawowy moduł Micrometer do zbierania metryk aplikacji. \\ \hline
    micrometer-registry-prometheus & Rejestrator/metody eksportu metryk Micrometer do systemu Prometheus. \\ \hline
    spring-boot-starter-data-mongodb & Spring Data MongoDB — integracja z bazą dokumentową MongoDB, repozytoria i mapowanie dokumentów. \\ \hline
    h2 & Lekka wbudowana baza danych H2 używana w testach jako zastępstwo relacyjnej bazy. \\ \hline
    de.flapdoodle.embed.mongo & Wbudowany (embedded) serwer MongoDB do testów integracyjnych — uruchamianie lokalnej instancji MongoDB. \\ \hline
    de.flapdoodle.embed.mongo.spring3x & Integracja Flapdoodle Embedded Mongo z aplikacjami Spring 3.x — ułatwienia konfiguracji w środowisku Spring. \\ \hline
    lombok & Narzędzia generujące kod na etapie kompilacji (adnotacje) — redukcja boilerplate w klasach Java. \\ \hline
    spring-boot-starter-oauth2-resource-server & Wsparcie serwera zasobów OAuth2 w Spring Boot — walidacja tokenów i konfiguracja zabezpieczeń zasobów. \\ \hline
    jjwt-api & Interfejsy i API biblioteki JJWT do tworzenia i parsowania JWT. \\ \hline
    jjwt-impl & Implementacja funkcjonalna biblioteki JJWT — konkretne mechanizmy tworzenia/parowania JWT. \\ \hline
    jjwt-jackson & Integracja JJWT z Jacksonem dla serializacji i deserializacji zawartości JWT. \\ \hline
    spring-boot-configuration-processor & Procesor adnotacji dla konfiguracji Spring Boot — generacja metadanych dla właściwości konfiguracyjnych. \\ \hline
    spring-boot-starter-test & Zestaw narzędzi testowych (JUnit, Mockito, AssertJ itp.) do testów jednostkowych i integracyjnych. \\ \hline
    spring-security-test & Narzędzia pomocnicze i rozszerzenia do testowania konfiguracji Spring Security. \\ \hline
    httpclient5 & Apache HttpClient 5 — zaawansowany klient HTTP do wykonywania żądań i obsługi odpowiedzi. \\ \hline
    httpcore5 & Rdzeń Apache HttpComponents — niskopoziomowe elementy HTTP wykorzystywane przez httpclient. \\ \hline
    angus-mail & Implementacja JavaMail (Angus) — wysyłanie i odbiór wiadomości e-mail w aplikacjach Java. \\ \hline
    spring-boot-starter-oauth2-client & Wsparcie klienta OAuth2 w Spring Boot — logowanie/połączenie z zewnętrznymi providerami OAuth2/OIDC. \\ \hline
    spring-boot-starter-security & Podstawowe komponenty Spring Security do zabezpieczania aplikacji. \\ \hline
    spring-boot-starter-webflux & Reaktywny stos webowy (Spring WebFlux) — programowanie reaktywne oraz serwery oparte na Netty. \\ \hline
    shedlock-spring & ShedLock dla Spring — mechanizm rozproszonego blokowania zadań zadaniowych (cron/scheduled). \\ \hline
    shedlock-provider-mongo & Implementacja dostawcy blokad ShedLock wykorzystująca MongoDB jako magazyn blokad. \\ \hline
    spring-retry & Biblioteka do automatycznego ponawiania operacji (retry) z konfiguracją i adnotacjami. \\ \hline
    spring-boot-starter-thymeleaf & Integracja Thymeleaf z Spring Boot — silnik szablonów do generowania widoków HTML po stronie serwera. \\ \hline
    azure-storage-blob & Klient Microsoft Azure Blob Storage — operacje na blobach (upload, download, zarządzanie kontenerami). \\ \hline
    jsoup & Biblioteka do parsowania, manipulacji i ekstrakcji danych z dokumentów HTML. \\ \hline

\end{longtable}

\refstepcounter{integrationcard}%
\par\noindent
\textbf{Tabela \theintegrationcard:} Zestawienie wszystkich bibliotek i pakietów użytych w projekcie.\label{tab:all-technologies2}

\addcontentsline{lot}{table}{Tabela \theintegrationcard: Zestawienie wszystkich bibliotek i pakietów użytych w projekcie.}%


\begin{itemize}
    \item \textbf{\gls{frontend}}

    Do realizacji tej części projektu wybrano \glslink{biblioteka}{bibliotekę} \gls{react} JavaScript, którą wszyscy członkowie zespołu poznali w trakcie studiów
    w ramach wybranej specjalizacji Aplikacje Internetowe.

    \begin{itemize}
        \item \textbf{\gls{react}} \textendash \space \glslink{biblioteka}{biblioteka} JavaScript służąca do budowania interaktywnych interfejsów użytkownika (\gls{ui}).
        Polega na programowaniu deklaratywnym oraz tworzeniu komponentów wielokrotnego użytku.
        Nie manipuluje bezpośrednio \gls{dom}, lecz tworzy swój wirtualny \gls{dom} i porównuje jego wersje.
        Po wykryciu zmian aktualizuje tylko te części \gls{dom}, które tego wymagają, co przekłada się na wydajną interakcję z apliklacją.
        Często jest wykorzystywany do tworzenia aplikacji typu \gls{spa}.
    \end{itemize}
\end{itemize}

\begin{longtable}{|
        >{\columncolor{lightgray}}p{0.25\textwidth}|
    p{0.65\textwidth}|
}
    \hline
    \rowcolor{lightgray}
    \multicolumn{2}{|c|}{\textbf{Zestawienie używanych bibliotek / pakietów}} \\ \hline
    \hline
    \rowcolor{lightgray}
    \textbf{Biblioteka / pakiet} & \textbf{Opis / przeznaczenie} \\ \hline
    \endfirsthead

    \hline
    \rowcolor{lightgray}
    \textbf{Biblioteka / pakiet} & \textbf{Opis / przeznaczenie} \\ \hline
    \endhead


    @ferrucc-io/emoji-picker & Komponent do wyboru emoji — pozwala użytkownikowi wybierać emotki w interfejsie. \\ \hline
    @hookform/resolvers & Współpracuje z \texttt{react-hook-form} — ułatwia walidację formularzy. \\ \hline
    @reduxjs/toolkit & Uproszczona konfiguracja stanu globalnego w aplikacji React + Redux — ułatwia zarządzanie store, reducery, akcje itp. \\ \hline
    @stomp/stompjs & Biblioteka do komunikacji WebSocket / STOMP — umożliwia komunikację w czasie rzeczywistym między klientem a serwerem. \\ \hline
    @tailwindcss/vite & Integracja stylów \texttt{Tailwind CSS} z bundlerem Vite — ułatwia stosowanie Tailwinda w projekcie. \\ \hline
    @tanstack/react-query & (dawniej React Query) — zarządzanie danymi z API: fetch, cache, synchronizacja, obsługa loadingu i błędów. \\ \hline
    @tiptap/extension-file-handler & Rozszerzenie edytora tekstu: obsługa plików (np. dodawanie lub wstawianie plików). \\ \hline
    @tiptap/extension-image & Rozszerzenie edytora — umożliwia wstawianie i obsługę obrazów. \\ \hline
    @tiptap/extension-placeholder & Rozszerzenie edytora — placeholder przy pustym polu edycji. \\ \hline
    @tiptap/extension-text-align & Rozszerzenie edytora — umożliwia wyrównywanie tekstu (np. do lewej, środka, prawej). \\ \hline
    @tiptap/pm & Silnik edytora tekstów (np. ProseMirror) używany przez Tiptap. \\ \hline
    @tiptap/react & Integracja edytora Tiptap z Reactem — edytor jako komponent Reactowy. \\ \hline
    @tiptap/starter-kit & Podstawowy zestaw wtyczek/ustawień do edytora Tiptap — szybki start. \\ \hline
    @vis.gl/react-maplibre & React-owe wiązanie do map — pozwala wyświetlać i obsługiwać mapy w aplikacji. \\ \hline
    antd & Biblioteka komponentów UI (przyciski, formularze, kafelki itp.) dla React — ułatwia budowę interfejsu. \\ \hline
    axios & Klient HTTP — umożliwia wysyłanie zapytań do API (GET, POST itd.) i obsługę odpowiedzi/błędów. \\ \hline
    date-fns & Obsługa dat i czasu — formatowanie, porównywanie, manipulacja datami/czasem. \\ \hline
    dotenv & Zarządzanie zmiennymi środowiskowymi (np. klucze API, ustawienia środowiska). \\ \hline
    maplibre-gl & Biblioteka do renderowania map — wykorzystywana przy mapach i warstwach geograficznych. \\ \hline
    media-chrome & Komponenty multimedialne — do obsługi odtwarzania mediów w przeglądarce. \\ \hline
    motion & Biblioteka animacji / efektów UI — pozwala dodawać animacje do komponentów interfejsu. \\ \hline
    openmeteo & Klient lub wrapper API pogodowego — prawdopodobnie do pobierania danych meteorologicznych. \\ \hline
    query-string & Parsowanie i generowanie parametrów zapytań w URL (np. GET, query params). \\ \hline
    react & Biblioteka do budowy UI — komponenty, stan komponentów, renderowanie interfejsu. \\ \hline
    react-dom & Biblioteka do „renderowania” Reacta w DOM przeglądarki — podstawa działania React. \\ \hline
    react-hook-form & Ułatwia obsługę formularzy w React: stan formularza, walidacje, obsługa submitów. \\ \hline
    react-icons & Zestaw ikon jako komponentów Reactowych (“ikonografika” w UI). \\ \hline
    react-intersection-observer & Hook / narzędzie do obserwowania, kiedy komponent wchodzi/wychodzi z widoku — przydatne np. do lazy-loading lub animacji przy scrollu. \\ \hline
    react-player & Komponent do odtwarzania mediów (wideo/audio) w React — odtwarzacze osadzone w interfejsie. \\ \hline
    react-redux & Integracja React + Redux — pozwala użyć globalnego store Redux w aplikacji React. \\ \hline
    react-router-dom & Routing / nawigacja w aplikacji React — definiowanie ścieżek URL i odpowiadających im “stron” / komponentów. \\ \hline
    react-select & Komponent zaawansowanego selecta / dropdownu — lepszy niż natywny `<select>`, z opcjami, wyszukiwaniem, stylizacją. \\ \hline
    sockjs-client & Klient WebSocket / komunikacji w czasie rzeczywistym — np. używany razem z STOMP. \\ \hline
    tailwindcss & CSS‑framework typu utility-first — stylowanie komponentów interfejsu za pomocą klas, szybkie budowanie UI. \\ \hline
    uuid & Generowanie unikalnych identyfikatorów — przydatne np. przy kluczach elementów list, ID obiektów itp. \\ \hline
    victory / victory-chart & Biblioteka do rysowania wykresów / diagramów w React — wykresy liniowe, słupkowe, kołowe itp. \\ \hline
    zod & Biblioteka do walidacji danych i schematów — sprawdzanie kształtu obiektów danych (np. z formularzy/API). \\ \hline
\end{longtable}

\refstepcounter{integrationcard}%
\par\noindent
\textbf{Tabela \theintegrationcard:} Zestawienie wszystkich bibliotek i pakietów użytych w projekcie.\label{tab:all-technologies}

\addcontentsline{lot}{table}{Tabela \theintegrationcard: Zestawienie wszystkich bibliotek i pakietów użytych w projekcie.}%

\begin{itemize}
    \item \textbf{\gls{cache}}
    \begin{itemize}
        \item \textbf{\gls{redis}} \textendash \space z ang. REmote DIctionary Server, nierelacyjna (NoSQL) baza danych
        przechowująca dane jako pary klucz - wartość.
        Działa w pamięci RAM, dzięki czemu pozwala na bardzo szybki odczyt danych.
    \end{itemize}
    \item \textbf{Konteneryzacja}
    \begin{itemize}
        \item \textbf{Docker} \textendash \space to silinik do konteneryzacji oprogramowania.
        Tworzy środowisko oddzielone od systemu hosta, w którym na podstawie obrazów programów tworzone są ich kontenery,
        czyli niezależne ich instancje.
        Docker pobiera wszystkie niezbędne dependnecje dla danego kontenera, bez potrzeby uruchamiania osobnego systemu operacyjnego,
        dzięki czemu kontenery są znacznie lżejsze niż maszyny wirtualne.
        Konteneryzacja pozwala na uruchomienie oprogramowania na różnych komputerach dokładnie w taki sam sposób, rozwiązuje to
        problem \enquote{na mojej maszynie działa}.
        Członkowie zespołu zapoznali się tą technologią w trakcie realizacji specjalizacji Aplikacje Internetowe.
    \end{itemize}
    \item \textbf{Baza danych}
    \begin{itemize}
        \item \textbf{PostgreSQL} \textendash \space system zarządzania relacyjnymi baz danych, zgodny ze standarden SQL.
        Wybrano relacyjną bazę danych, ponieważ doskonale wpisuje się ona w planowaną strukturę danych.
    \end{itemize}
\end{itemize}
