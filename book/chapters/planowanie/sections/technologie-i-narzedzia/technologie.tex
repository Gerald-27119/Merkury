%! Author = Stanisław Oziemczuk
%! Date = 02/12/2025

\subsection{Technologie}
\label{subsec:technologie}

Do realizacji projektu zespół wspólnie wytypował główne technologie części \glslink{backend}{backendowej}, \glslink{frontend}{frontendowej} oraz dokumentacji.
Natomiast poszczególne biblioteki i rozwiązania były wybierane indywidualnie lub po konsultacjach przez osobę wykonującą dane zadanie.
Poniżej przedstawiono stos technologiczny zastosowany w projekcie.

\begin{itemize}
    \item \textbf{\gls{backend}}

    Na główny język programowania została wybrana Java, ponieważ spośród innych dostępnych opcji, członkowie zespołu mieli z nią
    największe doświadczenie nabyte zarówno podczas studiów, jak i później pracę komercyjną.
    Do stworzenia \glslink{backend}{backendu} wykorzystany został \glslink{framework}{framework} SpringBoot, z którym zespół zapoznał się w ramach
    przedmiotów prowadzonych podczas studiów.

    \begin{itemize}
        \item \textbf{Java} \textendash \space obiektowy język programowania, cechujący się silnym typowaniem.
        Programy napisane w Javie są uruchamiane na maszynie writualnej Java (\gls{jvm}), dzięki czemu można je bezproblemowo
        przenosić między różnymi platformami wyposażonymi w to środowisko.
        \item \textbf{SpringBoot} \textendash \space \glslink{framework}{framework} służący do tworzenia aplikacji opartych na \glslink{spring-framework}{Spring Framework}.
        Wykorzystuje strategię \glslink{convention-over-configuration}{Convention Over Configuration}, która zmniejsza czas na konfigurowanie Springa, pozwalając skupić się na
        implementacji logiki.
        Służy do do tworzenia między innymi aplikacji internetowych czy mikroserwisów.
    \end{itemize}
    \item \textbf{\gls{frontend}}

    Do realizacji tej części projektu wybrano \glslink{biblioteka}{bibliotekę} \gls{react} JavaScript, którą wszyscy członkowie zespołu poznali w trakcie studiów
    w ramach wybranej specjalizacji Aplikacje Internetowe.

    \begin{itemize}
        \item \textbf{\gls{react}} \textendash \space \glslink{biblioteka}{biblioteka} JavaScript służąca do budowania interaktywnych interfejsów użytkownika (\gls{ui}).
        Polega na programowaniu deklaratywnym oraz tworzeniu komponentów wielokrotnego użytku.
        Nie manipuluje bezpośrednio \gls{dom}, lecz tworzy swój wirtualny \gls{dom} i porównuje jego wersje.
        Po wykryciu zmian aktualizuje tylko te części \gls{dom}, które tego wymagają, co przekłada się na wydajną interakcję z apliklacją.
        Często jest wykorzystywany do tworzenia aplikacji typu \gls{spa}.
    \end{itemize}
\end{itemize}

%\newcounter{integrationcard}[chapter]
%\renewcommand{\thetechnologies}{\thesubsection.\arabic{technologie}}


\begin{longtable}{|
        >{\columncolor{lightgray}}p{0.25\textwidth}|
    p{0.65\textwidth}|
}
    \hline
    \rowcolor{lightgray}
    \multicolumn{2}{|c|}{\textbf{Zestawienie używanych bibliotek / pakietów}} \\ \hline
    \hline
    \rowcolor{lightgray}
    \textbf{Biblioteka / pakiet} & \textbf{Opis / przeznaczenie} \\ \hline
    \endfirsthead

    \hline
    \rowcolor{lightgray}
    \textbf{Biblioteka / pakiet} & \textbf{Opis / przeznaczenie} \\ \hline
    \endhead


    @ferrucc-io/emoji-picker & Komponent do wyboru emoji — pozwala użytkownikowi wybierać emotki w interfejsie. \\ \hline
    @hookform/resolvers & Współpracuje z \texttt{react-hook-form} — ułatwia walidację formularzy. \\ \hline
    @reduxjs/toolkit & Uproszczona konfiguracja stanu globalnego w aplikacji React + Redux — ułatwia zarządzanie store, reducery, akcje itp. \\ \hline
    @stomp/stompjs & Biblioteka do komunikacji WebSocket / STOMP — umożliwia komunikację w czasie rzeczywistym między klientem a serwerem. \\ \hline
    @tailwindcss/vite & Integracja stylów \texttt{Tailwind CSS} z bundlerem Vite — ułatwia stosowanie Tailwinda w projekcie. \\ \hline
    @tanstack/react-query & (dawniej React Query) — zarządzanie danymi z API: fetch, cache, synchronizacja, obsługa loadingu i błędów. \\ \hline
    @tiptap/extension-file-handler & Rozszerzenie edytora tekstu: obsługa plików (np. dodawanie lub wstawianie plików). \\ \hline
    @tiptap/extension-image & Rozszerzenie edytora — umożliwia wstawianie i obsługę obrazów. \\ \hline
    @tiptap/extension-placeholder & Rozszerzenie edytora — placeholder przy pustym polu edycji. \\ \hline
    @tiptap/extension-text-align & Rozszerzenie edytora — umożliwia wyrównywanie tekstu (np. do lewej, środka, prawej). \\ \hline
    @tiptap/pm & Silnik edytora tekstów (np. ProseMirror) używany przez Tiptap. \\ \hline
    @tiptap/react & Integracja edytora Tiptap z Reactem — edytor jako komponent Reactowy. \\ \hline
    @tiptap/starter-kit & Podstawowy zestaw wtyczek/ustawień do edytora Tiptap — szybki start. \\ \hline
    @vis.gl/react-maplibre & React-owe wiązanie do map — pozwala wyświetlać i obsługiwać mapy w aplikacji. \\ \hline
    antd & Biblioteka komponentów UI (przyciski, formularze, kafelki itp.) dla React — ułatwia budowę interfejsu. \\ \hline
    axios & Klient HTTP — umożliwia wysyłanie zapytań do API (GET, POST itd.) i obsługę odpowiedzi/błędów. \\ \hline
    date-fns & Obsługa dat i czasu — formatowanie, porównywanie, manipulacja datami/czasem. \\ \hline
    dotenv & Zarządzanie zmiennymi środowiskowymi (np. klucze API, ustawienia środowiska). \\ \hline
    maplibre-gl & Biblioteka do renderowania map — wykorzystywana przy mapach i warstwach geograficznych. \\ \hline
    media-chrome & Komponenty multimedialne — do obsługi odtwarzania mediów w przeglądarce. \\ \hline
    motion & Biblioteka animacji / efektów UI — pozwala dodawać animacje do komponentów interfejsu. \\ \hline
    openmeteo & Klient lub wrapper API pogodowego — prawdopodobnie do pobierania danych meteorologicznych. \\ \hline
    query-string & Parsowanie i generowanie parametrów zapytań w URL (np. GET, query params). \\ \hline
    react & Biblioteka do budowy UI — komponenty, stan komponentów, renderowanie interfejsu. \\ \hline
    react-dom & Biblioteka do „renderowania” Reacta w DOM przeglądarki — podstawa działania React. \\ \hline
    react-hook-form & Ułatwia obsługę formularzy w React: stan formularza, walidacje, obsługa submitów. \\ \hline
    react-icons & Zestaw ikon jako komponentów Reactowych (“ikonografika” w UI). \\ \hline
    react-intersection-observer & Hook / narzędzie do obserwowania, kiedy komponent wchodzi/wychodzi z widoku — przydatne np. do lazy-loading lub animacji przy scrollu. \\ \hline
    react-player & Komponent do odtwarzania mediów (wideo/audio) w React — odtwarzacze osadzone w interfejsie. \\ \hline
    react-redux & Integracja React + Redux — pozwala użyć globalnego store Redux w aplikacji React. \\ \hline
    react-router-dom & Routing / nawigacja w aplikacji React — definiowanie ścieżek URL i odpowiadających im “stron” / komponentów. \\ \hline
    react-select & Komponent zaawansowanego selecta / dropdownu — lepszy niż natywny `<select>`, z opcjami, wyszukiwaniem, stylizacją. \\ \hline
    sockjs-client & Klient WebSocket / komunikacji w czasie rzeczywistym — np. używany razem z STOMP. \\ \hline
    tailwindcss & CSS‑framework typu utility-first — stylowanie komponentów interfejsu za pomocą klas, szybkie budowanie UI. \\ \hline
    uuid & Generowanie unikalnych identyfikatorów — przydatne np. przy kluczach elementów list, ID obiektów itp. \\ \hline
    victory / victory-chart & Biblioteka do rysowania wykresów / diagramów w React — wykresy liniowe, słupkowe, kołowe itp. \\ \hline
    zod & Biblioteka do walidacji danych i schematów — sprawdzanie kształtu obiektów danych (np. z formularzy/API). \\ \hline
\end{longtable}

\refstepcounter{integrationcard}%
\par\noindent
\textbf{Tabela \theintegrationcard:} Zestawienie wszystkich bibliotek i pakietów użytych w projekcie.\label{tab:all-technologies}

\addcontentsline{lot}{table}{Tabela \theintegrationcard: Zestawienie wszystkich bibliotek i pakietów użytych w projekcie.}%

