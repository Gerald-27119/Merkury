%! Author = Stanisław Oziemczuk
%! Date = 02/12/2025

\subsection{Technologie}
\label{subsec:technologie}

Do realizacji projektu zespół wspólnie wytypował główne technologie części \glslink{backend}{backendowej}, \glslink{frontend}{frontendowej} oraz dokumentacji.
Natomiast poszczególne biblioteki i rozwiązania były wybierane indywidualnie lub po konsultacjach przez osobę wykonującą dane zadanie.
Poniżej przedstawiono stos technologiczny zastosowany w projekcie.

\begin{itemize}
    \item \textbf{\gls{backend}}

    Na główny język programowania została wybrana Java, ponieważ spośród innych dostępnych opcji, członkowie zespołu mieli z nią
    największe doświadczenie nabyte zarówno podczas studiów, jak i później pracę komercyjną.
    Do stworzenia \glslink{backend}{backendu} wykorzystany został \glslink{framework}{framework} SpringBoot, z którym zespół zapoznał się w ramach
    przedmiotów prowadzonych podczas studiów.

    \begin{itemize}
        \item \textbf{Java} \textendash \space obiektowy język programowania, cechujący się silnym typowaniem.
        Programy napisane w Javie są uruchamiane na maszynie writualnej Java (\gls{jvm}), dzięki czemu można je bezproblemowo
        przenosić między różnymi platformami wyposażonymi w to środowisko.
        \item \textbf{SpringBoot} \textendash \space \glslink{framework}{framework} służący do tworzenia aplikacji opartych na \glslink{spring-framework}{Spring Framework}.
        Wykorzystuje strategię \glslink{convention-over-configuration}{Convention Over Configuration}, która zmniejsza czas na konfigurowanie Springa, pozwalając skupić się na
        implementajcji logiki.
        Służy do do tworzenia między innymi aplikacji internetowych czy mikroserwisów.
    \end{itemize}
\end{itemize}
