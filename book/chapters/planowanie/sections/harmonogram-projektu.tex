%! Author = mateusz
%! Date = 17/10/2025

\section{Harmonogram projektu}
\label{sec:harmonogram-projektu}

W poniższym harmonogramie przedstawiono plan prac nad poszczególnymi częściami projektu, rozłożony na miesiące.

\begin{figure}[!htbp]
    \centering
    \fbox{\parbox{0.95\linewidth}{\centering \vspace{2.5cm}
    \textit{Miejsce na ilustrację harmonogramu (np. wykres Gantta).}
    \vspace{2.5cm}}}
    \caption{Wizualizacja harmonogramu (do uzupełnienia).}
    \label{fig:harmonogram-gantt}
\end{figure}

\subsection*{Rok 2024}
\begin{description}
    \item[Czerwiec] Zebranie zespołu; rozważenie potencjalnych pomysłów.
    \item[Lipiec] Wybór technologii; wstępne założenia architektoniczne.
    \item[Sierpień] Realizacja dokumentacji.
    \item[Wrzesień] \emph{(do uzupełnienia)}
    \item[Październik] \emph{(do uzupełnienia)}
    \item[Listopad] \emph{(do uzupełnienia)}
    \item[Grudzień] \emph{(do uzupełnienia)}
\end{description}

\subsection*{Rok 2025}
\begin{description}
    \item[Styczeń] \emph{(do uzupełnienia)}
    \item[Luty] \emph{(do uzupełnienia)}
    \item[Marzec] \emph{(do uzupełnienia)}
    \item[Kwiecień] \emph{(do uzupełnienia)}
    \item[Maj] \emph{(do uzupełnienia)}
    \item[Czerwiec] \emph{(do uzupełnienia)}
    \item[Lipiec] \emph{(do uzupełnienia)}
    \item[Sierpień] \emph{(do uzupełnienia)}
    \item[Wrzesień] \emph{(do uzupełnienia)}
    \item[Październik] \emph{(do uzupełnienia)}
    \item[Listopad] \emph{(do uzupełnienia)}
    \item[Grudzień] \emph{(do uzupełnienia)}
\end{description}

\subsection*{Rok 2026}
\begin{description}
    \item[Styczeń] \emph{(do uzupełnienia)}
\end{description}