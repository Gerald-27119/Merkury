%! Author = Mateusz
%! Date = 27/12/2025

\section{Napotkane wyzwania}
\label{sec:napotkane-wyzwania}

Jednym z pierwszych wyzwań był ograniczony poziom wcześniejszego doświadczenia w realizacji projektów
o porównywalnej skali i złożoności, co wymagało dopracowania sposobu planowania prac
oraz ujednolicenia podejścia do implementacji.

Istotnym obszarem trudności okazała się integracja zabezpieczeń po stronie \glslink{backend}{backendu}
z wykorzystaniem \glslink{biblioteka}{biblioteki} Spring Security, z którą wcześniej nie pracowano.
W konsekwencji pojawiały się nieporozumienia interpretacyjne oraz konieczność wprowadzania poprawek
w konfiguracji.

Kolejnym problemem była implementacja logowania z użyciem zewnętrznych dostawców (Google oraz GitHub).
Wystąpiły trudności związane z konfiguracją \glslink{oauth}{OAuth}, przekierowaniami oraz dopasowaniem integracji
do przyjętej architektury, przy jednoczesnym ograniczonym dostępie do materiałów opisujących analogiczny przypadek.

Dodatkowym wyzwaniem było zastosowanie w projekcie wielu nowych \glslink{biblioteka}{bibliotek}
zarówno po stronie \glslink{frontend}{frontendowej}, jak i \glslink{backend}{backendowej}.
Wymagało to analizy dokumentacji, weryfikacji kompatybilności wersji oraz doboru właściwych sposobów integracji.

Dodatkowym wyzwaniem była nauka języka \LaTeX{}, który wykorzystano do przygotowania dokumentacji projektu,
co wiązało się z koniecznością poznania składni oraz sposobów formatowania treści technicznej.

Trudności pojawiły się również podczas wdrażania usługi mapowej wykorzystywanej w aplikacji.
Było to pierwsze uruchomienie komponentu na zewnętrznym serwerze, co ujawniło problemy konfiguracyjne.
