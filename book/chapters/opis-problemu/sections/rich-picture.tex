%! Author = kacper
%! Date = 31/12/2025

\section{Rich picture}
\label{sec:rich-picture}

Rich picture jest techniką służącą do przedstawienia kontekstu systemu, jego interesariuszy, relacji między nimi oraz głównych potrzeb i wyzwań.
Pozwala na lepsze zrozumienie problemu przed przejściem do formalnych modeli, takich jak diagramy \glslink{uml}{UML}.

Na zaprezentowanym rich picture (rys. \ref{img:rich-picture}) przedstawiono kontekst działania aplikacji „SpotyNaDrony” oraz jej interesariuszy.

Centralnym elementem diagramu jest platforma łącząca użytkowników.
Z aplikacji korzystają różne grupy, w tym:
\begin{itemize}
    \item \textbf{początkujący \glslink{droniarz}{droniarz}} – korzysta z aplikacji w celu uzyskania informacji i wsparcia na początku swojej przygody z dronami
    \item \textbf{\glslink{droniarz}{droniarz}} – bardziej doświadczony użytkownik, wyszukujący nowe miejsca do latania oraz nawiązujący kontakty z innymi pasjonatami
    \item \textbf{społeczność} – tworzy i wymienia się treściami, dzieli się doświadczeniami oraz buduje relacje wokół wspólnego zainteresowania
    \item \textbf{administrator} – odpowiedzialny za moderowanie treści publikowanych w aplikacji oraz utrzymanie porządku na platformie
\end{itemize}

Diagram przedstawia również główne interakcje zachodzące w systemie:
\begin{itemize}
    \item tworzenie nowych treści oraz wyszukiwanie miejsc do latania przez użytkowników
    \item wymiana doświadczeń w obrębie społeczności
    \item moderowanie treści przez administratora
    \item generowanie przychodów przez aplikację poprzez reklamy
\end{itemize}

Całość stanowi wysokopoziomowy opis funkcjonowania platformy oraz przepływu interakcji pomiędzy użytkownikami a systemem.

\begin{figure}[H]
    \centering
    \includegraphics[width=1\textwidth]{attachments/rich-picture/rich_picture}
    \caption{Diagram Rich Picture przedstawiający kontekst systemu}
    \label{img:rich-picture}
\end{figure}