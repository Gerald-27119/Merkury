%! Author = kacper
%! Date = 31/12/2025

\section{Istniejące rozwiązania}
\label{sec:istniejace-rozwiazania}


W kontekście istniejących rozwiązań zidentyfikowano kilka narzędzi dostępnych na rynku, które w pewnym
stopniu odpowiadają funkcjonalnością projektowi „SpotyNaDrony”.
Żadne z nich nie oferuje jednak pełnego zestawu możliwości dostarczanych przez projektowaną aplikację,
dlatego zasadne jest omówienie najpopularniejszych rozwiązań.

\subsection{DroneMap PANSA}

DroneMap PANSA\footnote{\url{https://dronemap.pansa.pl/}} jest oficjalną mapą przestrzeni powietrznej
udostępnianą przez Polską Agencję Żeglugi Powietrznej (ang. \textit{Polish Air Navigation Services Agency}).
Narzędzie umożliwia sprawdzanie stref geograficznych, ograniczeń lotów oraz zasad obowiązujących
w strefach lotniczych na obszarze Polski.
Rozwiązanie to przeznaczone jest głównie do planowania lotów z punktu widzenia zgodności z przepisami prawa.

Aplikacja nie oferuje funkcji społecznościowych, takich jak możliwość dzielenia się doświadczeniami,
dodawania własnych lokalizacji do latania ani komunikacji pomiędzy użytkownikami.

\subsection{Dronestagram}

Dronestagram\footnote{\url{https://www.dronestagr.am/}} jest platformą internetową, która częściowo
odpowiada założeniom analizowanego projektu, jednak jej główny cel jest odmienny.
Użytkownicy mogą publikować zdjęcia oraz filmy wykonane z użyciem dronów, tworzyć posty oraz budować
portfolio swoich prac.
Platforma umożliwia również promowanie usług fotograficznych i filmowych, a także wyszukiwanie
zarejestrowanych pilotów oraz firm oferujących usługi związane z wykorzystaniem dronów, z użyciem mapy.

Rozwiązanie to nie oferuje jednak możliwości dodawania i oceniania spotów do latania,
prowadzenia forum dyskusyjnego ani komunikacji w czasie rzeczywistym,
co odróżnia je od koncepcji projektowanej aplikacji.