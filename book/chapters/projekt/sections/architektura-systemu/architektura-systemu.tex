%! Author = Mateusz
%! Date = 23/11/2025

\section{Architektura systemu}
\label{sec:architektura-systemu}

W niniejszym rozdziale przedstawiona zostanie architektura systemu, czyli sposób,
w jaki poszczególne komponenty komunikują się między sobą, a także technologie,
za pomocą których zostały stworzone.

Jednym z kluczowych etapów podczas realizacji projektu był wybór odpowiedniej architektury systemowej.
Ostatecznie zdecydowaliśmy się na oddzielenie poszczególnych warstw aplikacji, co pozwoliło uzyskać
większą elastyczność, skalowalność oraz prostszą możliwość rozwoju w przyszłości.
Przyjęte komponenty prezentują się następująco:

\begin{itemize}
\item \glslink{frontend}{frontend} – \gls{react} z wykorzystaniem \glslink{type-script}{TypeScriptu},
\item \glslink{backend}{backend} – Java Spring Boot,
\item \glslink{baza-danych}{baza danych} – PostgreSQL,
\item \glslink{redis}{redis} – wykorzystywany jako \glslink{baza-danych}{baza danych} klucz–wartość pełniąca rolę warstwy cache.
\end{itemize}

Jest to podejście, z którym zespół projektowy ma największe doświadczenie, dlatego zdecydowaliśmy
się na jego zastosowanie.
Pozwala ono również na tworzenie aplikacji responsywnej, dostępnej zarówno na komputerach,
jak i urządzeniach mobilnych.
Warstwa wizualna została przygotowana przy użyciu \gls{react} w wersji z \glslink{type-script}{TypeScriptem} oraz
\glslink{biblioteka}{biblioteki} Tailwind CSS, zapewniającej szybkie i wygodne stylowanie komponentów.
Z kolei za komunikację oraz logikę biznesową odpowiada \glslink{backend}{backend} oparty
na \glslink{framework}{frameworku} Spring Boot, realizujący założenia architektury
\glslink{rest_api}{REST API}.
Jako system zarządzania danymi wybraliśmy relacyjną bazę danych PostgreSQL, z
którą zespół posiada największe doświadczenie.
Relacyjny model danych doskonale sprawdza się w tym projekcie, zapewniając integralność danych,
możliwość tworzenia złożonych zapytań oraz wysoką stabilność.

\glslink{redis}{Redis} został wykorzystany jako warstwa \glslink{cache}{cache}, której zadaniem jest przyspieszenie działania aplikacji
poprzez ograniczenie liczby odwołań do głównej \glslink{baza-danych}{bazy danych}.
Dzięki przechowywaniu często wykorzystywanych danych w pamięci operacyjnej znacznie skraca się czas
odpowiedzi systemu, co pozytywnie wpływa na wydajność oraz skalowalność rozwiązania.
Zastosowanie \glslink{redis}{Redisa} okazało się szczególnie korzystne w przypadku operacji powtarzalnych i odczytowych,
które nie wymagają każdorazowego dostępu do relacyjnej \glslink{baza-danych}{bazy danych}.

\subimport{chapters/projekt/sections/architektura-systemu/}{diagram-architektury.tex}
\subimport{chapters/projekt/sections/architektura-systemu/}{komponenty-systemu.tex}
