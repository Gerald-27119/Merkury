%! Author = Mateusz
%! Date = 23/11/2025

\subsection{Komponenty systemu}
\label{subsec:komponenty-systemu}

System będzie składał się z kilku głównych komponentów, z których każdy pełni ściśle określoną rolę.

\begin{itemize}
    \item \glslink{frontend}{Frontend} – odpowiada za warstwę prezentacji oraz interfejs
    użytkownika dostępny dla wszystkich użytkowników systemu,
    \item \glslink{backend}{Backend} – odpowiada za autoryzację użytkowników oraz obsługę
    komunikacji między \glslink{frontend}{frontendem} a \glslink{baza-danych}{bazą danych},
    \item \glslink{baza-danych}{Baza danych} – przechowuje wszystkie dane aplikacji, w tym dane
    użytkowników, dane operacyjne oraz informacje potrzebne do działania systemu.
    \item \glslink{redis}{Redis} – wykorzystywany jako warstwa cache, przechowująca często
    odczytywane dane w pamięci operacyjnej, co znacząco przyspiesza działanie systemu
    oraz zmniejsza obciążenie głównej bazy danych.
\end{itemize}

\noindent Poniżej przedstawiono listę zewnętrznych API wraz z odwołaniami do odpowiadających im tabel:

\begin{itemize}
    \item Azure Blob Storage~--~\ref{tab:azure-blob}
    \item Mailtrap~--~\ref{tab:mailtrap}
    \item LocationIQ~--~\ref{tab:locationiq}
    \item Google Maps~--~\ref{tab:google-maps}
    \item OpenFreeMap~--~\ref{tab:openfreemap}
    \item Open Meteo~--~\ref{tab:open-meteo}
    \item Tenor Gif~--~\ref{tab:tenor}
    \item Where the ISS at?~--~\ref{tab:wheretheiss}
\end{itemize}
