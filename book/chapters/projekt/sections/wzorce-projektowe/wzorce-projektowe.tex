%! Author = Stanisław Oziemczuk
%! Date = 08/12/2025


\section{Wzorce projektowe}
\label{sec:wzorce-projektowe}

Podczas prac deweloperskich nad projektem skorzystano z różnych \glslink{wzorzec}{wzorców projektowych}.
Zarówno na \glslink{frontend}{frontendzie}, jak i \glslink{backend}{backendzie} istnieje wiele propozycji, z których członkowie zespołu starali się korzystać
w celu poszerzenia wiedzy, umiejętności oraz otrzymania wysokiej jakości pisanego kodu.
Poniżej przedstawiono listę zastosowanych rozwiązań.

\begin{itemize}
    \item \textbf{\glslink{backend}{Backend}}
    \begin{itemize}
        \item \textbf{MVC} \textendash \space Model View Controller, jest głównie wykorzystywany przy aplikacjach webowych.
        Polega na podzieleniu projektu na trzy główne części odpowiedzialne za:
        \begin{itemize}
            \item \textbf{Model} \textendash \space reprezentację danych i logikę biznesową.
            \item \textbf{View} \textendash \space prezentację danych użytkownikowi.
            \item \textbf{Controller} \textendash \space reagowanie na działania użytkownika poprzez uruchomienie logiki w Model i odświeżenie View.
        \end{itemize}
        \item \textbf{Chain of Responsibility}
        \item \textbf{Proxy} \textendash \space to strukturalny \glslink{wzorzec}{wzorzec} projektowy
        zakładający stworzenie obiektu pośredniczącego, który ma taki sam interfejs jak obiekt właściwy.
        Obiekt ten może zawierać dodatkową logikę, taką jak kontrola dostępu do zasobu, sprawdzenie uprawnień do wykonania zadanej operacji,
        utworzenie oryginalnego obiektu czy \glslink{cache}{cache'owanie}.
        Dla klienta korzystanie z takiego pośrednika jest identyczne jakby był to oryginalny zasób.
        \item \textbf{Inversion of Control}
        \item \textbf{Dependency Injection}
        \item \textbf{Singleton}
        \item \textbf{Builder}
        \item \textbf{Pagination}
        \item \textbf{Mapper}
        \item \textbf{Cache-Aside}
        \item \textbf{Iterator}
        \item \textbf{Factory method}
        \item \textbf{Interceptor}
        \item \textbf{Strategy}
        \item \textbf{Logging Facade}
        \item \textbf{Repository}
        \item \textbf{Facade}
    \end{itemize}
    \item \textbf{\glslink{frontend}{Frontend}}
    \begin{itemize}
        \item \textbf{Redux (Pattern)}
        \item \textbf{Command (Redux)}
        \item \textbf{Entity Adapter}
        \item \textbf{Optimistic UI}
        \item \textbf{Infinite Scroll}
        \item \textbf{Hooks Pattern}
        \item \textbf{Error Boundary}
        \item \textbf{Portal}
        \item \textbf{MVVM}
        \item \textbf{Protected route}
    \end{itemize}
\end{itemize}
