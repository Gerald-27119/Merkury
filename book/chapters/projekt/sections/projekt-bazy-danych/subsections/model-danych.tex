%! Author = Mateusz
%! Date = 22/12/2025

\subsection{Model danych}
\label{subsec:model-danych}

Model danych został zaprojektowany w sposób modułowy, zgodnie z obszarami funkcjonalnymi systemu.
Wyróżniono cztery główne części: użytkowników i relacje społeczne, moduł spotów, moduł forum oraz moduł czatu.
W celu ograniczenia duplikacji zastosowano klasy bazowe (\texttt{Comment} oraz \texttt{ForumReport}),
które agregują pola wspólne dla wielu encji.

\paragraph{Użytkownicy i relacje.}
Centralną encją systemu jest \texttt{UserEntity}, przechowująca dane identyfikacyjne użytkownika
(e-mail, nazwa użytkownika, hasło, rola oraz informacje o stanie konta).
Powiązane z użytkownikiem encje umożliwiają realizację funkcji społecznościowych:
\texttt{Friendship} przechowuje informacje o relacjach pomiędzy użytkownikami (status znajomości),
natomiast mechanizm obserwowania użytkowników realizowany jest poprzez relację wiele do wielu
pomiędzy użytkownikami (zbiór obserwujących oraz obserwowanych).

\paragraph{Moduł spotów.}
Encja \texttt{Spot} przechowuje informacje opisujące miejsce (nazwa, lokalizacja, opis, pole obszaru,
strefa czasowa oraz punkty graniczne definiujące obszar).
Dla spotów przewidziano system ocen i statystyk (liczba wyświetleń).
Komentowanie spotów realizuje encja \texttt{SpotComment}, dziedzicząca po \texttt{Comment}, co zapewnia wspólne pola
takie jak autor, data publikacji oraz liczniki głosów.
Dodatkowe treści multimedialne są przechowywane w \texttt{SpotMedia} (media spotu) oraz \texttt{SpotCommentMedia}
(media komentarza).
Funkcjonalność tagowania spotów realizuje encja \texttt{SpotTag} oraz relacja wiele do wielu pomiędzy \texttt{Spot}
a \texttt{SpotTag}, odwzorowana przy pomocy tabeli pośredniej.
Ulubione miejsca użytkownika przechowywane są w encji \texttt{FavoriteSpot}, która wiąże użytkownika ze spotem
oraz określa typ listy ulubionych.

\paragraph{Moduł forum.}
Forum bazuje na encji \texttt{Post}, przechowującej treść wpisu oraz metadane (data publikacji, liczba wyświetleń,
liczniki głosów i komentarzy).
Wpisy są przypisywane do kategorii poprzez \texttt{PostCategory}, a system tagów realizuje encja \texttt{Tag} oraz
relacja wiele do wielu z \texttt{Post}.
Komentarze do wpisów odwzorowuje \texttt{PostComment}, dziedziczący po \texttt{Comment}.
Obsługę zagnieżdżonych dyskusji realizuje relacja komentarz--rodzic (self-reference), umożliwiająca tworzenie odpowiedzi.
Media powiązane z wpisami i komentarzami przechowywane są odpowiednio w \texttt{PostMedia} oraz \texttt{PostCommentMedia}.
Mechanizm zgłoszeń naruszeń zrealizowano poprzez \texttt{ForumReport} jako klasę bazową oraz wyspecjalizowane encje
\texttt{PostReport} i \texttt{PostCommentReport}, które wskazują obiekt zgłoszenia oraz przechowują powód i szczegóły.

\paragraph{Moduł czatu.}
Komunikacja użytkowników realizowana jest poprzez encję \texttt{Chat}, przechowującą informacje o konwersacji
(typ czatu, nazwa, obraz, czas utworzenia).
Uczestnictwo w czacie odwzorowuje encja pośrednia \texttt{ChatParticipant}, która przechowuje rolę uczestnika oraz daty
związane z aktywnością.
Wiadomości są przechowywane w \texttt{ChatMessage} i zawierają treść oraz metadane (czas wysłania, status przeczytania).
Pliki dołączone do wiadomości zostały wydzielone do encji \texttt{ChatMessageAttachedFile}.
Proces zapraszania użytkowników do czatu obsługuje \texttt{ChatInvitation}, przechowująca nadawcę, odbiorcę, czat oraz status.

\paragraph{Relacje i integralność.}
Encje połączono za pomocą kluczy obcych, zapewniając spójność referencyjną danych.
Relacje jeden do wielu wykorzystano dla komentarzy i mediów, a relacje wiele do wielu
zrealizowano poprzez tabele pośrednie (tagi, polubienia, obserwowanie użytkowników).
W wybranych miejscach zastosowano kaskadowanie oraz usuwanie osieroconych rekordów,
co upraszcza utrzymanie spójności danych podczas modyfikacji obiektów powiązanych.
