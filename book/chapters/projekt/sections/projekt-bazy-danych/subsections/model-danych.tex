%! Author = Mateusz
%! Date = 22/12/2025

\subsection{Model danych}
\label{subsec:model-danych}

Poniżej zestawiono główne tabele (encje domenowe) oraz ich przeznaczenie:
\begin{itemize}
    \item \texttt{users} -- przechowuje dane kont użytkowników.
    \item \texttt{friendships} -- przechowuje informacje o relacjach pomiędzy użytkownikami.

    \item \texttt{spots} -- przechowuje informacje o spotach.
    \item \texttt{spots\_tags} -- przechowuje definicje tagów przypisywanych do spotów.
    \item \texttt{spot\_media} -- przechowuje media powiązane ze spotami.
    \item \texttt{spot\_comments} -- przechowuje komentarze do spotów.
    \item \texttt{spot\_comment\_media} -- przechowuje media dołączone do komentarzy spotów.
    \item \texttt{favorite\_spots} -- przechowuje spoty zapisane przez użytkownika.

    \item \texttt{posts} -- przechowuje wpisy forum.
    \item \texttt{forum\_categories} -- przechowuje kategorie wpisów forum.
    \item \texttt{forum\_tags} -- przechowuje tagi przypisywane do wpisów forum.
    \item \texttt{post\_comments} -- przechowuje komentarze do wpisów forum.
    \item \texttt{post\_reports} -- przechowuje zgłoszenia dotyczące wpisów forum.
    \item \texttt{post\_comment\_reports} -- przechowuje zgłoszenia dotyczące komentarzy wpisów forum.

    \item \texttt{chats} -- przechowuje informacje o czatach.
    \item \texttt{chat\_participants} -- przechowuje uczestników czatu.
    \item \texttt{chat\_messages} -- przechowuje wiadomości czatu.
    \item \texttt{chat\_message\_attached\_file} -- przechowuje pliki dołączone do wiadomości czatu.
    \item \texttt{chat\_invitations} -- przechowuje zaproszenia do czatów.
\end{itemize}

Relacje pomiędzy powyższymi tabelami zrealizowano za pomocą kluczy obcych.
Relacje typu wiele do wielu (tagi, polubienia, obserwowanie) odwzorowano przy użyciu tabel pośrednich.
Encje \texttt{Comment} oraz \texttt{ForumReport} pełnią rolę klas bazowych i nie stanowią osobnych tabel.
