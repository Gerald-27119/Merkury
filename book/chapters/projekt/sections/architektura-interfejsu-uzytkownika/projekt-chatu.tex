%! Author = Adam
%! Date = 01/01/2026

\subsection{Projekt czatu}
\label{subsec:projekt-chatu}

\newcommand{\chatfig}[1]{./attachments/projekt/architektura-interfejsu-uzytkownika/czat/#1}

\subsubsection{Główny widok modułu czatu}
\label{subsubsec:chat-main-view}

\begin{figure}[H]
    \centering
    \includegraphics[width=\textwidth]{\chatfig{ActiveChatPage}}
    \caption{Główny widok modułu czatu.}
    \label{fig:chat:main-view}
\end{figure}

Rysunek \ref{fig:chat:main-view} przedstawia główny widok modułu czatu.
Po lewej stronie widoczny jest pionowy panel z nagłówkiem \textit{Friends} oraz
sekcją \textit{Your chats}. Panel ten prezentuje listę istniejących rozmów,
w których użytkownik uczestniczy. Każda pozycja listy zawiera awatar
oraz nazwę rozmowy. Aktualnie wybrany czat jest wyróżniony
poprzez podświetlenie tła, co ułatwia orientację w kontekście prowadzonej rozmowy.
W dolnej części panelu umieszczono obszar informacyjny dotyczący
zalogowanego użytkownika (awatar oraz nazwa) wraz ze wskaźnikiem statusu aktywności.

Centralna część ekranu przedstawia zawartość bieżącej konwersacji.
W górnym pasku widoczne są informacje o rozmówcy (awatar i nazwa)
oraz ikona przypięcia wiadomości, pole wyszukiwania
w obrębie rozmowy oraz ikona związana z przeglądaniem multimediów wysłanych w ramach otwartego czatu.
Poniżej znajduje się historia wiadomości w układzie chronologicznym,
z rozróżnieniem autorów poprzez awatary i nazwy oraz z oznaczeniami czasu wysłania.
W widoku uwzględniono separator daty, który porządkuje dłuższą historię konwersacji.

Na dole okna czatu umieszczono pasek wprowadzania wiadomości z polem tekstowym oraz przyciskami akcji.
Widoczna jest opcja umożliwiająca dodanie załączników, wstawienie \glslink{gif}{GIF-a} oraz użycie \glslink{emoji}{emoji}.

\subsubsection{Oznaczenie nieprzeczytanej aktywności na liście rozmów}
\label{subsubsec:chat-unread-indicator}

\begin{figure}[H]
    \centering
    \includegraphics[width=\textwidth]{\chatfig{ActiveChatPageNewNotification}}
    \caption{Lista rozmów z oznaczeniem nieprzeczytanej aktywności.}
    \label{fig:chat:unread-indicator}
\end{figure}

Rysunek \ref{fig:chat:unread-indicator} przedstawia wariant głównego widoku modułu czatu,
w którym dla jednej z wylistowanych konwersacji wyświetlono plakietkę z liczbą nowych nieprzeczytanych wiadomości.
Oznaczenie to pozwala użytkownikowi szybko zidentyfikować nową aktywność.

\subsubsection{Podgląd przypiętych wiadomości}
\label{subsubsec:chat-pinned-messages}

\begin{figure}[H]
    \centering
    \includegraphics[width=\textwidth]{\chatfig{ChatPinnedMessages}}
    \caption{Podgląd przypiętych wiadomości w obrębie rozmowy.}
    \label{fig:chat:pinned-messages}
\end{figure}

Rysunek \ref{fig:chat:pinned-messages} przedstawia widok prezentacji przypiętych wiadomości,
stanowiący rozszerzenie głównego ekranu rozmowy o dodatkowy panel z listą przypiętych treści.

W górnej części obszaru konwersacji, na tle historii wiadomości, wyświetlono
\glslink{modal}{okno modalne} z nagłówkiem \textit{Pinned messages}.
Zawiera ono wydzielony obszar listy przypiętych treści.
W widocznym przykładzie przedstawiono pojedynczą pozycję z przypiętą wiadomością,
zawierającą awatar autora, nazwę użytkownika oraz znacznik daty i czasu.
Poniżej metadanych umieszczono podgląd treści wiadomości.

\subsubsection{Wyszukiwanie wiadomości w historii rozmowy}
\label{subsubsec:chat-search-messages}

\begin{figure}[H]
    \centering
    \includegraphics[width=\textwidth]{\chatfig{MessageFound}}
    \caption{Wyszukiwanie wiadomości w historii rozmowy.}
    \label{fig:chat:search-results}
\end{figure}

Rysunek \ref{fig:chat:search-results} przedstawia widok wyszukiwania wiadomości w obrębie aktywnej rozmowy.
W górnym pasku konwersacji pole wyszukiwania zawiera wprowadzoną frazę
oraz informację o liczbie dopasowań.
W treści rozmowy dopasowania są dodatkowo wyróżniane wizualnie poprzez
podświetlenie znalezionego fragmentu w wiadomości.

Widok prezentuje wyodrębniony obszar wyników po prawej stronie w formie nakładki,
z możliwością zamknięcia.
W obrębie tej nakładki widoczna jest lista wiadomości spełniających kryteria
wyszukiwania, przedstawionych wraz z informacjami o autorze i czasie wysłania.
Jedna z wiadomości jest dodatkowo podkreślona, co obrazuje aktualnie wybrane
dopasowanie. Równolegle, w głównym obszarze rozmowy zaznaczono odpowiadającą
wiadomość w historii konwersacji.

W dolnej części okna rozmowy widoczny jest przycisk \textit{Return to latest messages},
umożliwiający szybki powrót do najnowszych wiadomości po zakończeniu przeglądania
starszych wpisów.

\subsubsection{Podręczny pasek akcji dla wiadomości}
\label{subsubsec:chat-message-actions-hover}

\begin{figure}[H]
    \centering
    \includegraphics[width=\textwidth]{\chatfig{OptionsMessageHover}}
    \caption{Podręczny pasek akcji widoczny po wskazaniu wiadomości.}
    \label{fig:chat:message-actions-hover}
\end{figure}

Rysunek \ref{fig:chat:message-actions-hover} przedstawia zachowanie interfejsu po najechaniu kursorem
na pojedynczą wiadomość w oknie konwersacji.
Pojawiający się panel ma formę poziomego paska z ikonami, umieszczonego bezpośrednio przy wiadomości.
Zawiera zestaw działań dostępnych dla danej wiadomości: widoczna jest ikona reakcji,
ikony związane z interakcją na wiadomości (odpowiedź/przekazanie) oraz ikona wielokropka,
umożliwiająca dalsze operacje.

\subsubsection{Menu kontekstowe dla wiadomości}
\label{subsubsec:chat-message-more-menu}

\begin{figure}[H]
    \centering
    \includegraphics[width=\textwidth]{\chatfig{OptionsMessageHover3DotsMenu}}
    \caption{Menu kontekstowe dla wiadomości w rozmowie.}
    \label{fig:chat:message-more-menu}
\end{figure}

Rysunek \ref{fig:chat:message-more-menu} przedstawia stan interfejsu po rozwinięciu menu oferującego więcej opcji (ikona wielokropka)
dla wybranej wiadomości. Menu zawiera listę dostępnych działań.
Są to: \textit{Pin message}, \textit{Copy text} oraz \textit{Report}.
Ostatnia pozycja wyróżniona kolorem czerwonym, umożliwia zgłoszenie wiadomości do administracji aplikacji.

\subsubsection{Dodatkowe akcje dla wiadomości własnej}
\label{subsubsec:chat-own-message-actions}

\begin{figure}[H]
    \centering
    \includegraphics[width=\textwidth]{\chatfig{MyMessageHover3DotsMenu}}
    \caption{Opcje dostępne dla wiadomości własnej po najechaniu kursorem.}
    \label{fig:chat:own-message-hover}
\end{figure}

Rysunek \ref{fig:chat:own-message-hover} przedstawia podręczny pasek akcji wyświetlany po wskazaniu wiadomości
wysłanej przez zalogowanego użytkownika. W porównaniu do analogicznego widoku dla wiadomości innych uczestników
(rys.~\ref{fig:chat:message-actions-hover}), pasek został rozszerzony o dodatkową ikonę edycji (symbol ołówka),
umożliwiającą modyfikację własnego wpisu. Pozostałe skróty pozostają spójne.

\subsubsection{Menu „więcej opcji” dla wiadomości własnej}
\label{subsubsec:chat-own-message-more-menu}

\begin{figure}[H]
    \centering
    \includegraphics[width=\textwidth]{\chatfig{MyMessageHover}}
    \caption{Menu „więcej opcji” dla wiadomości własnej.}
    \label{fig:chat:own-message-more-menu}
\end{figure}

Rysunek \ref{fig:chat:own-message-more-menu} przedstawia rozwinięte menu „więcej opcji” dla wiadomości własnej.
Względem menu dostępnego dla wiadomości innych użytkowników (rys.~\ref{fig:chat:message-more-menu}) pojawia się
dodatkowa akcja \textit{Delete}, pozwalająca usunąć własną wiadomość z konwersacji. Pozycja ta jest wyróżniona kolorem
czerwonym oraz ikoną kosza. Pozostałe elementy menu pozostają zgodne z wariantem
standardowym (rys.~\ref{fig:chat:message-more-menu}).

\subsubsection{Przegląd multimediów powiązanych z rozmową}
\label{subsubsec:chat-media-grid}

\begin{figure}[H]
    \centering
    \includegraphics[width=\textwidth]{\chatfig{Images}}
    \caption{Przegląd multimediów powiązanych z rozmową.}
    \label{fig:chat:media-grid}
\end{figure}

Rysunek \ref{fig:chat:media-grid} przedstawia podgląd materiałów wizualnych powiązanych z obecną konwersacją.
W centralnej części ekranu wyświetlono siatkę miniatur plików.

\subsubsection{Menu kontekstowe dla obrazu w wiadomości}
\label{subsubsec:chat-image-context-menu}

\begin{figure}[H]
    \centering
    \includegraphics[width=\textwidth]{\chatfig{ImageMenu}}
    \caption{Menu kontekstowe dla obrazu w wiadomości.}
    \label{fig:chat:image-context-menu}
\end{figure}

Rysunek \ref{fig:chat:image-context-menu} przedstawia menu kontekstowe wyświetlane dla obrazu umieszczonego w wiadomości.
Menu pojawia się bezpośrednio przy obrazie i zawiera listę dostępnych operacji, takich jak zapis lub skopiowanie obrazu.
Dodatkowo widoczna jest opcja usunięcia wyróżniona kolorem czerwonym.

\subsubsection{Powiększony podgląd obrazu}
\label{subsubsec:chat-image-preview}

\begin{figure}[H]
    \centering
    \includegraphics[width=\textwidth]{\chatfig{ExpandedImage}}
    \caption{Powiększony podgląd obrazu w rozmowie.}
    \label{fig:chat:image-preview}
\end{figure}

Rysunek \ref{fig:chat:image-preview} przedstawia tryb powiększonego podglądu obrazu.
Tło aplikacji zostało przyciemnione, a na pierwszym planie wyświetlono obraz w większym rozmiarze, umieszczony centralnie.
W prawym górnym rogu widoczna jest ikona zamknięcia umożliwiająca powrót do standardowego widoku konwersacji.

\subsubsection{Odpowiedź na wcześniejszą wiadomość}
\label{subsubsec:chat-reply-message}

\begin{figure}[H]
    \centering
    \includegraphics[width=\textwidth]{\chatfig{ReplayMessageV3}}
    \caption{Wiadomość będąca odpowiedzią na wcześniejszy wpis.}
    \label{fig:chat:reply-message}
\end{figure}

Rysunek \ref{fig:chat:reply-message} przedstawia sposób prezentowania odpowiedzi odnoszącej się do wcześniejszej wiadomości.
W obszarze konwersacji, nad treścią nowej wiadomości, wyświetlony jest wydzielony pasek podglądu wiadomości,
do której nawiązano. Podgląd zawiera podstawowe informacje (awatar oraz nazwę autora i fragment treści).

\subsubsection{Tworzenie nowej rozmowy lub grupy}
\label{subsubsec:chat-create-modal}

\begin{figure}[H]
    \centering
    \includegraphics[width=\textwidth]{\chatfig{CreateNewChat}}
    \caption{Okno tworzenia nowej rozmowy lub grupy.}
    \label{fig:chat:create-modal}
\end{figure}

Rysunek \ref{fig:chat:create-modal} przedstawia widok tworzenia nowej rozmowy w formie \glslink{modal}{okna modalnego}.
W górnej części okna znajduje się nagłówek oraz pole wyszukiwania użytkowników.
Poniżej zaprezentowano listę użytkowników wraz z awatarami i nazwami, a po prawej stronie każdej pozycji
umieszczono pola wyboru umożliwiające zaznaczenie wybranych osób. W dolnej części okna widoczny jest przycisk
służący do utworzenia nowej rozmowy na podstawie dokonanych wyborów.

\subsubsection{Widok rozmowy grupowej}
\label{subsubsec:chat-group-view}

\begin{figure}[H]
    \centering
    \includegraphics[width=\textwidth]{\chatfig{ActiveGroupChatPage}}
    \caption{Widok rozmowy grupowej z panelem uczestników.}
    \label{fig:chat:group-view}
\end{figure}

Rysunek \ref{fig:chat:group-view} przedstawia ekran rozmowy grupowej.
W centralnej części widoczna jest zawartość aktywnej konwersacji,
natomiast w prawym panelu bocznym umieszczono listę uczestników oznaczoną nagłówkiem \textit{Group members}.
Lista prezentuje awatary oraz nazwy członków grupy w układzie pionowym.
W górnym pasku konwersacji widoczna jest nazwa grupy (oraz jej awatar),
a także elementy analogiczne do rozmów prywatnych, w tym wyszukiwanie oraz sekcja multimediów.

\subsubsection{Menu opcji dla rozmowy grupowej}
\label{subsubsec:chat-group-options-menu}

\begin{figure}[H]
    \centering
    \includegraphics[width=\textwidth]{\chatfig{GroupChatMenu}}
    \caption{Menu opcji dla rozmowy grupowej.}
    \label{fig:chat:group-options-menu}
\end{figure}

Rysunek \ref{fig:chat:group-options-menu} przedstawia menu opcji dostępne w kontekście rozmowy grupowej.
\glslink{modal}{Okno modalne} zawiera operacje związane z konfiguracją rozmowy, takie jak zmiana obrazu grupy lub jej nazwy.

\subsubsection{Oznaczanie użytkownika w wiadomości}
\label{subsubsec:chat-mention-suggestions}

\begin{figure}[H]
    \centering
    \includegraphics[width=\textwidth]{\chatfig{ActiveGroupChatPageUserTagged}}
    \caption{Oznaczanie użytkownika w wiadomości z podpowiedziami.}
    \label{fig:chat:mention-suggestions}
\end{figure}

Rysunek \ref{fig:chat:mention-suggestions} przedstawia mechanizm oznaczania użytkownika w wiadomości w ramach rozmowy grupowej.
W treści wiadomości widoczny jest wyróżniony fragment w formacie \texttt{@nazwaUzytkownika}.
W polu wprowadzania wiadomości użytkownik wpisuje ciąg rozpoczynający się od znaku \texttt{@},
a nad polem wyświetlono panel podpowiedzi z propozycjami dopasowań (awatar i nazwa),
umożliwiający szybkie wybranie osoby do oznaczenia.

\subsubsection{Menu kontekstowe dla członka grupy}
\label{subsubsec:chat-group-member-menu}

\begin{figure}[H]
    \centering
    \includegraphics[width=\textwidth]{\chatfig{ActiveGroupChatPageGroupMemberOptions}}
    \caption{Menu kontekstowe dla uczestnika rozmowy grupowej.}
    \label{fig:chat:group-member-menu}
\end{figure}

Rysunek \ref{fig:chat:group-member-menu} przedstawia menu kontekstowe dla wybranego uczestnika w panelu listy członków grupy.
Menu jest umieszczone bezpośrednio przy wskazanej pozycji użytkownika i zawiera operacje takie jak podgląd profilu
oraz rozpoczęcie rozmowy prywatnej. Widok ten umożliwia wykonanie działań względem konkretnego członka grupy
bez konieczności opuszczania ekranu rozmowy.
