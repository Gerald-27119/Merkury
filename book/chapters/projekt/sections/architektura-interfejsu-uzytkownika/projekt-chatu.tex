%! Author = Adam
%! Date = 01/01/2026

\subsection{Projekt chatu}
\label{subsec:projekt-chatu}

ActiveChatPage

Rysunek X przedstawia główny widok modułu czatu (ActiveChatPage),
wyświetlany po wejściu do sekcji rozmów. Interfejs utrzymano w ciemnej kolorystyce,
z wyraźnym podziałem na obszar nawigacyjny po lewej stronie oraz centralne okno aktualnie otwartej konwersacji.

Po lewej stronie widoczny jest pionowy panel z nagłówkiem \textit{Friends} oraz
sekcją \textit{Your chats}. Panel ten prezentuje listę istniejących rozmów,
w których użytkownik uczestniczy. Każda pozycja listy zawiera awatar
oraz nazwę rozmowy (np. użytkownika lub grupy). Aktualnie wybrany czat jest wyróżniony
poprzez podświetlenie tła, co ułatwia orientację w kontekście prowadzonej rozmowy.
W dolnej części panelu umieszczono obszar informacyjny dotyczący
zalogowanego użytkownika (awatar, nazwa) wraz ze wskaźnikiem statusu aktywności.

Centralna część ekranu przedstawia zawartość bieżącej konwersacji.
W górnym pasku widoczne są informacje o rozmówcy (awatar i nazwa)
oraz elementy sterujące, w tym ikona przypięcia, pole wyszukiwania
w obrębie rozmowy oraz ikona związana z przeglądaniem lub dodawaniem materiałów graficznych.
Poniżej znajduje się historia wiadomości w układzie chronologicznym,
z rozróżnieniem autorów poprzez awatary i nazwy oraz z oznaczeniami czasu wysłania.
W widoku uwzględniono separator daty (np. \textit{27 March 2025}),
co porządkuje dłuższą historię konwersacji.
Przykładowa wiadomość zawiera załączony obraz wraz z tekstem,
co wskazuje na obsługę treści multimedialnych w przebiegu rozmowy.

Na dole okna czatu umieszczono pasek wprowadzania wiadomości z polem tekstowym oraz przyciskami akcji.
Widoczne są m.in. kontrolki umożliwiające dodanie załączników, wstawienie GIF oraz użycie emotikon.
Taki układ zapewnia szybki dostęp do podstawowych funkcji komunikacji bez opuszczania bieżącej konwersacji.

ActiveChatPageNewNotification


