%! Author = Adam
%! Date = 01/01/2026

\subsection{Projekt chatu}
\label{subsec:projekt-chatu}

ActiveChatPage

Rysunek X przedstawia główny widok modułu czatu (ActiveChatPage),
wyświetlany po wejściu do sekcji rozmów. Interfejs utrzymano w ciemnej kolorystyce,
z wyraźnym podziałem na obszar nawigacyjny po lewej stronie oraz centralne okno aktualnie otwartej konwersacji.

Po lewej stronie widoczny jest pionowy panel z nagłówkiem \textit{Friends} oraz
sekcją \textit{Your chats}. Panel ten prezentuje listę istniejących rozmów,
w których użytkownik uczestniczy. Każda pozycja listy zawiera awatar
oraz nazwę rozmowy (np. użytkownika lub grupy). Aktualnie wybrany czat jest wyróżniony
poprzez podświetlenie tła, co ułatwia orientację w kontekście prowadzonej rozmowy.
W dolnej części panelu umieszczono obszar informacyjny dotyczący
zalogowanego użytkownika (awatar, nazwa) wraz ze wskaźnikiem statusu aktywności.

Centralna część ekranu przedstawia zawartość bieżącej konwersacji.
W górnym pasku widoczne są informacje o rozmówcy (awatar i nazwa)
oraz elementy sterujące, w tym ikona przypięcia, pole wyszukiwania
w obrębie rozmowy oraz ikona związana z przeglądaniem lub dodawaniem materiałów graficznych.
Poniżej znajduje się historia wiadomości w układzie chronologicznym,
z rozróżnieniem autorów poprzez awatary i nazwy oraz z oznaczeniami czasu wysłania.
W widoku uwzględniono separator daty (np. \textit{27 March 2025}),
co porządkuje dłuższą historię konwersacji.
Przykładowa wiadomość zawiera załączony obraz wraz z tekstem,
co wskazuje na obsługę treści multimedialnych w przebiegu rozmowy.

Na dole okna czatu umieszczono pasek wprowadzania wiadomości z polem tekstowym oraz przyciskami akcji.
Widoczne są m.in. kontrolki umożliwiające dodanie załączników, wstawienie GIF oraz użycie emotikon.
Taki układ zapewnia szybki dostęp do podstawowych funkcji komunikacji bez opuszczania bieżącej konwersacji.

ActiveChatPageNewNotification

Rysunek X przedstawia wariant głównego widoku modułu czatu oznaczony jako
\textit{ActiveChatPageNewNotification}. Układ ekranu pozostaje zgodny z widokiem
podstawowym: po lewej stronie znajduje się lista rozmów, natomiast w centralnej
części ekranu prezentowane jest okno aktualnie otwartej konwersacji. Interfejs
utrzymano w ciemnej kolorystyce, z jasną typografią oraz subtelnymi liniami
separującymi poszczególne obszary.

W porównaniu do wcześniejszego widoku, pojawia się dodatkowe
wyróżnienie przy jednej z pozycji listy istniejacych czatow: obok nazwy rozmowy widoczna jest
okrągła, fioletowa plakietka z liczbą „1”. Oznaczenie to sygnalizuje nową
aktywność w danej konwersacji i pozwala użytkownikowi szybko zidentyfikować
rozmowę wymagającą uwagi.

ChatPinnedMessages

Rysunek X przedstawia widok \textit{ChatPinnedMessages}, stanowiący rozszerzenie
głównego ekranu rozmowy o prezentację przypiętych wiadomości. Interfejs zachowuje
ten sam układ jak w pozostałych wariantach: po lewej stronie widoczny jest panel
z listą rozmów, natomiast w centralnej części znajduje się okno aktywnej konwersacji.
Dodatkowo, po skrajnej lewej stronie ekranu występuje pionowy pasek nawigacyjny
z ikonami, oddzielony wizualnie od listy czatów.

W górnej części obszaru konwersacji, na tle historii wiadomości, wyświetlono
dodatkowy panel w formie nakładki z nagłówkiem \textit{Pinned messages}.
Nakładka ma ciemne tło i zawiera wydzielony obszar listy przypiętych treści.
W widocznym przykładzie przedstawiono pojedynczą pozycję z przypiętą wiadomością,
zawierającą awatar autora, nazwę użytkownika oraz znacznik daty i czasu.
Poniżej metadanych umieszczono podgląd treści wiadomości w postaci krótkiego
fragmentu tekstu. Układ nakładki wskazuje na możliwość przeglądania przypiętych
elementów bez opuszczania aktualnie otwartej rozmowy, przy jednoczesnym
zachowaniu widoczności kontekstu konwersacji w tle.

MessageFound

Rysunek X przedstawia widok wyszukiwania wiadomości w obrębie aktywnej rozmowy,
oznaczony jako \textit{MessageFound}. Układ ekranu zachowuje podstawową strukturę
modułu czatu (panel nawigacyjny oraz lista rozmów po lewej stronie), natomiast
w centralnym obszarze rozmowy widoczny jest tryb przeglądania wyników wyszukiwania.

W górnym pasku konwersacji pole wyszukiwania zawiera wprowadzoną frazę
(\textit{amet}) oraz informację o liczbie dopasowań (np. \textit{99+ results}),
co wskazuje na możliwość wyszukiwania treści w historii czatu.
W treści rozmowy dopasowania są dodatkowo wyróżniane wizualnie poprzez
podświetlenie znalezionego fragmentu w wiadomości.

Widok prezentuje wyodrębniony obszar wyników po prawej stronie w formie nakładki,
z możliwością zamknięcia (ikona \textit{X} w prawym górnym rogu).
W obrębie tej nakładki widoczna jest lista wiadomości spełniających kryterium
wyszukiwania, przedstawionych wraz z informacjami o autorze i czasie wysłania.
Jedna z wiadomości jest dodatkowo podkreślona, co sugeruje aktualnie wybrane
dopasowanie. Równolegle, w głównym obszarze rozmowy zaznaczono odpowiadającą
wiadomość w historii konwersacji.

W dolnej części okna rozmowy widoczny jest przycisk \textit{Return to latest messages},
umożliwiający szybki powrót do najnowszych wiadomości po zakończeniu przeglądania
starszych wpisów. Takie rozwiązanie ułatwia odnajdywanie wcześniejszych treści
bez utraty kontekstu aktualnie prowadzonej rozmowy.

OptionsMessageHover

Rysunek X przedstawia widok \textit{OptionsMessageHover}, ilustrujący zachowanie
interfejsu po najechaniu kursorem na pojedynczą wiadomość w oknie konwersacji.
Układ ekranu pozostaje zgodny z głównym widokiem czatu, jednak nad wskazaną
wiadomością pojawia się dodatkowy, niewielki panel akcji.

Panel ma formę poziomego paska z ikonami, umieszczonego bezpośrednio przy
wiadomości (w jej górnej części). Zawiera zestaw skrótowych działań dostępnych
dla danej wiadomości: widoczna jest ikona reakcji (np. emotikona), ikony
powiązane z interakcją na wiadomości (np. odpowiedź/przekazanie) oraz ikona
menu wielokropka, sugerująca dostęp do dalszych opcji. Zastosowanie takiego
paska umożliwia szybkie wykonanie operacji na konkretnej wiadomości bez
zmiany kontekstu widoku i bez konieczności otwierania oddzielnych ekranów.

OptionsMessageHover3DotsMenu

Rysunek X przedstawia widok \textit{OptionsMessageHover3DotsMenu}, czyli stan
interfejsu po rozwinięciu menu „więcej opcji” dla wybranej wiadomości.
W górnej części wiadomości nadal widoczny jest podręczny pasek akcji,
natomiast po kliknięciu ikony trzech kropek wyświetlane jest dodatkowe menu
kontekstowe w formie niewielkiej nakładki.

Menu zawiera listę dostępnych działań opisanych tekstowo. W widocznym przykładzie
są to: \textit{Pin message} (z ikoną pinezki), \textit{Copy text} (z ikoną kopiowania)
oraz \textit{Report}. Ostatnia pozycja wyróżniona jest kolorem czerwonym, co
wizualnie wskazuje na działanie o charakterze zgłoszenia/nadużycia.
Nakładka pojawia się bezpośrednio przy miejscu interakcji, dzięki czemu użytkownik
może szybko wykonać wybraną operację na wiadomości bez opuszczania bieżącego widoku rozmowy.

MyMessageHover

Rysunek X przedstawia widok \textit{MyMessageHover}, czyli stan interfejsu po
najechaniu kursorem na wiadomość wysłaną przez zalogowanego użytkownika.
W obszarze konwersacji wyróżniona jest konkretna wiadomość (wraz z załączonym
obrazem), a w jej pobliżu wyświetlany jest zestaw opcji dostępnych dla tej treści.

Nad wiadomością widoczny jest podręczny pasek akcji w formie poziomego rzędu ikon,
a poniżej (po prawej stronie wiadomości) rozwinięte menu kontekstowe.
Menu prezentuje listę operacji przypisanych do wiadomości, w tym m.in.
\textit{Pin message} oraz \textit{Copy text}. Dodatkowo widoczne są pozycje
wyróżnione kolorem czerwonym: \textit{Report} oraz \textit{Delete}.
Pojawienie się opcji \textit{Delete} wskazuje na rozszerzony zakres działań
dla wiadomości własnych w porównaniu do wiadomości innych uczestników rozmowy.

Rozwiązanie to umożliwia wykonywanie działań na wiadomości bez opuszczania
aktualnie otwartej konwersacji, a jednocześnie zapewnia czytelne rozróżnienie
między standardowymi operacjami a akcjami potencjalnie nieodwracalnymi.

MyMessageHover3DotsMenu

Rysunek X przedstawia widok \textit{MyMessageHover3DotsMenu}, czyli stan
interfejsu po rozwinięciu menu „więcej opcji” (ikona trzech kropek) dla wiadomości
wysłanej przez zalogowanego użytkownika. W obszarze konwersacji widoczna jest
wiadomość własna zawierająca załączony obraz oraz fragment tekstu, a przy niej
wyświetlony jest podręczny pasek akcji.

Po aktywowaniu menu wielokropka pojawia się nakładka z listą dostępnych operacji.
Menu zawiera pozycje takie jak \textit{Pin message} oraz \textit{Copy text}, a także
akcje wyróżnione kolorem czerwonym: \textit{Report} i \textit{Delete}.
Obecność opcji \textit{Delete} wskazuje na możliwość usunięcia własnej wiadomości,
co odróżnia ten wariant od analogicznego menu dla wiadomości innych uczestników.
Nakładka jest umieszczona bezpośrednio przy wiadomości, dzięki czemu użytkownik
może wykonać operację bez opuszczania bieżącego widoku rozmowy.


Images

Rysunek X przedstawia widok \textit{Images}, w którym w obrębie aktywnej rozmowy
udostępniono podgląd materiałów wizualnych powiązanych z konwersacją.
W centralnej części ekranu wyświetlono siatkę miniatur, obejmującą co najmniej
dwa elementy. Pierwsza miniatura przedstawia obraz (fotografia drona), natomiast
druga zawiera nałożony symbol odtwarzania, co wizualnie odróżnia ją od zwykłego
obrazu. Widok pozostaje osadzony w kontekście modułu czatu: po lewej stronie
nadal widoczna jest lista rozmów, a w górnym pasku zachowano elementy sterujące
związane z konwersacją (w tym pole wyszukiwania) oraz wyróżnioną ikonę obrazów.

ImageMenu

Rysunek X przedstawia widok \textit{ImageMenu}, prezentujący menu kontekstowe
wyświetlane dla obrazu umieszczonego w wiadomości. Menu pojawia się bezpośrednio
przy obrazie i zawiera listę dostępnych operacji: \textit{Save image} oraz
\textit{Copy image}. Dodatkowo widoczna jest opcja \textit{Delete}, wyróżniona
kolorem czerwonym, co wskazuje na działanie o charakterze usunięcia.
Rozmieszczenie menu w pobliżu wskazanego obrazu umożliwia wykonanie operacji
bez opuszczania bieżącego widoku rozmowy.

ExpandedImage

Rysunek X przedstawia widok \textit{ExpandedImage}, czyli tryb powiększonego
podglądu obrazu. Tło aplikacji zostało przyciemnione, a na pierwszym planie
wyświetlono obraz w większym rozmiarze, umieszczony centralnie.
W prawym górnym rogu widoczna jest ikona zamknięcia (symbol \textit{X}),
umożliwiająca powrót do standardowego widoku konwersacji.
Rozwiązanie to pozwala użytkownikowi obejrzeć przesłaną grafikę w wysokiej
czytelności bez utraty kontekstu rozmowy w tle.

ReplayMessageV3

Rysunek X przedstawia widok \textit{ReplayMessageV3}, obrazujący sposób
prezentowania odpowiedzi odnoszącej się do wcześniejszej wiadomości.
W obszarze konwersacji, nad treścią nowej wiadomości, wyświetlony jest
wydzielony, węższy pasek podglądu wiadomości, do której nawiązano.
Podgląd zawiera podstawowe informacje widoczne na makiecie (awatar oraz nazwa
autora i fragment treści), dzięki czemu użytkownik może jednoznacznie powiązać
odpowiedź z konkretnym wpisem w historii rozmowy.

CreateNewChat

Rysunek X przedstawia widok \textit{CreateNewChat}, zrealizowany w formie
centralnie umieszczonego okna modalnego na przyciemnionym tle modułu czatu.
W górnej części okna znajduje się nagłówek \textit{Add user(s) to create new chat or group}
oraz pole wyszukiwania użytkowników. Poniżej zaprezentowano listę użytkowników
wraz z awatarami i nazwami, a po prawej stronie każdej pozycji umieszczono pola
wyboru umożliwiające zaznaczenie wybranych osób. W dolnej części okna widoczny
jest przycisk \textit{Create new chat}, służący do utworzenia nowej rozmowy
na podstawie dokonanych wyborów.

ActiveGroupChatPage

Rysunek X przedstawia widok \textit{ActiveGroupChatPage}, czyli ekran rozmowy
grupowej. W centralnej części widoczna jest zawartość aktywnej konwersacji,
natomiast w prawym panelu bocznym umieszczono listę uczestników oznaczoną
nagłówkiem \textit{Group members}. Lista prezentuje awatary oraz nazwy członków
grupy w układzie pionowym. W górnym pasku konwersacji widoczna jest nazwa grupy
(oraz jej awatar), a także elementy sterujące analogiczne do rozmów prywatnych,
w tym pole wyszukiwania w obrębie rozmowy oraz ikona związana z materiałami
wizualnymi.

GroupChatMenu

Rysunek X przedstawia widok \textit{GroupChatMenu}, w którym w kontekście
rozmowy grupowej wyświetlono dodatkowe menu opcji. Menu pojawia się jako niewielka
nakładka przy pozycji czatu grupowego i zawiera dwie operacje: \textit{Change picture}
oraz \textit{Change name}. Każda pozycja jest opatrzona ikoną, co ułatwia szybkie
rozpoznanie dostępnych działań bez potrzeby opuszczania widoku rozmowy.

ActiveGroupChatPageUserTagged

Rysunek X przedstawia widok \textit{ActiveGroupChatPageUserTagged}, ilustrujący
mechanizm oznaczania użytkownika w wiadomości w ramach rozmowy grupowej.
W treści wiadomości w obszarze konwersacji widoczny jest wyróżniony fragment
w formacie \texttt{@Username}, oznaczony kolorem odróżniającym go od reszty tekstu.
W dolnej części ekranu, w polu wprowadzania wiadomości, użytkownik wpisuje ciąg
zaczynający się od znaku \texttt{@}, a nad polem wyświetlono panel podpowiedzi
z propozycją użytkownika (awatar i nazwa), umożliwiający szybkie wybranie osoby
do oznaczenia. Po prawej stronie ekranu zachowano listę członków grupy w panelu
\textit{Group members}.

ActiveGroupChatPageGroupMemberOptions

Rysunek X przedstawia widok \textit{ActiveGroupChatPageGroupMemberOptions},
w którym dla wybranego uczestnika w panelu \textit{Group members} wyświetlono
menu kontekstowe. Menu jest umieszczone bezpośrednio przy wskazanej pozycji
użytkownika i zawiera dwie operacje: \textit{View profile} oraz \textit{Send message},
każda z nich opatrzona odpowiednią ikoną. Widok ten umożliwia wykonanie działań
względem konkretnego członka grupy bez konieczności opuszczania ekranu rozmowy.

