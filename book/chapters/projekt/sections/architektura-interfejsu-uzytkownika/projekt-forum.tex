%! Author = kacper
%! Date = 02/01/2026

\subsection{Projekt forum}
\label{subsec:projekt-forum}

Wygląd forum został zaprojektowany dla czterech podstron:
\begin{itemize}
    \item \textbf{Home page} – strona główna forum prezentująca listę postów,
    \item \textbf{Post details} – widok zawierający pełną treść posta wraz z obrazami i komentarzami,
    \item \textbf{Followed threads} – lista postów obserwowanych przez użytkownika,
    \item \textbf{Posts History} – lista ostatnio przeglądanych postów oraz postów ocenionych przez użytkownika.
\end{itemize}

Interfejs forum został zaprojektowany w ciemnym trybie kolorystycznym.

\subsubsection{Home page}
Strona główna (rys. \ref{img:forum-view}) forum składa się z trzech głównych obszarów funkcjonalnych.

\paragraph{Lista postów}
Centralna część strony zawiera listę postów posortowanych według wybranej opcji.
Nad listą znajduje się rozwijany przycisk umożliwiający zmianę typu sortowania.
Dostępne opcje sortowania to: \textit{Date} (data), \textit{Comments} (liczba komentarzy) oraz \textit{Views} (liczba wyświetleń).

Każdy post prezentowany jest w postaci kafelka zawierającego:
\begin{itemize}
    \item tytuł posta,
    \item kategorię,
    \item tagi,
    \item skróconą treść posta (bez obrazów),
    \item liczbę komentarzy,
    \item liczbę wyświetleń.
\end{itemize}

Dodatkowo każdy kafelek zawiera ikonę trzech poziomych kropek, która po kliknięciu wyświetla menu kontekstowe.
Menu to umożliwia zgłoszenie posta, a w przypadku gdy użytkownik jest jego autorem – także edycję lub usunięcie.

\begin{figure}[H]
    \centering
    \includegraphics[width=1\textwidth]{attachments/projekt/architektura-interfejsu-uzytkownika/forum/forum_home_page}
    \caption{Widok głównej strony forum}
    \label{img:forum-view}
\end{figure}

\paragraph{Lewy panel boczny}
Lewy panel boczny (rys. \ref{img:left-panel}) zawiera przycisk który otwiera \glslink{modal}{modal} zawierający formularz tworzenia nowego posta.
Poniżej przycisku wyświetlana jest lista dostępnych kategorii oraz tagów postów.

Formularz tworzenia posta (rys. \ref{img:create-discussion}) składa się z następujących pól:
\begin{itemize}
    \item \textit{title} – tytuł posta,
    \item \textit{category} – kategoria posta,
    \item \textit{tags} – lista tagów,
    \item \textit{content} – treść posta tworzona za pomocą rich text editor.
\end{itemize}

Na dole formularza znajdują się dwa przyciski:
\begin{itemize}
    \item \textit{create} – zapisanie posta,
    \item \textit{cancel} – anulowanie i zamknięcie formularza.
\end{itemize}

\begin{figure}[H]
    \centering
    \includegraphics[width=0.4\textwidth]{attachments/projekt/architektura-interfejsu-uzytkownika/forum/left_panel}
    \caption{Lewy panel boczny}
    \label{img:left-panel}
\end{figure}

\begin{figure}[H]
    \centering
    \includegraphics[width=1\textwidth]{attachments/projekt/architektura-interfejsu-uzytkownika/forum/forum_create_discussion}
    \caption{Formularz tworzenia posta}
    \label{img:create-discussion}
\end{figure}

\paragraph{Prawy panel boczny}
Prawy panel boczny (rys. \ref{img:right-panel}) składa się z trzech elementów.

\subparagraph{SearchBar}
Element umożliwia wyszukiwanie postów na forum. Po rozwinięciu paska (rys. \ref{img:searchbar_extended}) wyszukiwania dostępne są dodatkowe filtry:
\begin{itemize}
    \item kategorie,
    \item tagi,
    \item autor posta (\textit{posted by}),
    \item status posta (\textit{open}, \textit{closed}, \textit{archived}),
    \item zakres czasowy (\textit{before}, \textit{after}),
    \item konkretna data.
\end{itemize}

\begin{figure}[H]
    \centering
    \includegraphics[width=0.6\textwidth]{attachments/projekt/architektura-interfejsu-uzytkownika/forum/searchbar_extended}
    \caption{Rozwinięty searchBar}
    \label{img:searchbar_extended}
\end{figure}

\subparagraph{Top Posts}
Sekcja prezentuje listę najpopularniejszych postów na forum.
Każdy element listy zawiera tytuł posta, kategorię, tagi oraz skróconą treść.

\subparagraph{Recommended Friends}
Sekcja wyświetla listę polecanych znajomych w postaci zdjęcia profilowego oraz nazwy użytkownika.
Ikona przedstawiająca sylwetkę z symbolem plusa umożliwia dodanie użytkownika do listy znajomych.

\begin{figure}[H]
    \centering
    \includegraphics[width=0.6\textwidth]{attachments/projekt/architektura-interfejsu-uzytkownika/forum/right_panel}
    \caption{Prawy panel boczny}
    \label{img:right-panel}
\end{figure}

Lewy oraz prawy panel boczny są widoczne na wszystkich podstronach forum, co zapewnia spójną i wygodną nawigację.

\subsubsection{Post details}
Widok szczegółów posta (rys. \ref{img:post-details}) prezentuje pełną treść wraz z osadzonymi obrazami.
Nad jego kafelkiem znajdują się dwa przyciski:
\begin{itemize}
    \item po lewej stronie – przycisk nawigacyjny umożliwiający powrót do strony głównej forum,
    \item po prawej stronie – przycisk w postaci ikony dzwonka umożliwiający dodanie posta do listy obserwowanych.
\end{itemize}

Górna część kafelka zawiera zdjęcie profilowe autora oraz jego nazwę użytkownika, a po przeciwnej stronie datę utworzenia posta.
Poniżej wyświetlane są kategoria i tagi, następnie tytuł oraz pełna treść posta.

W dolnej części kafelka znajdują się elementy interakcji oraz informacje:
\begin{itemize}
    \item \textit{upvote} z liczbą polubień,
    \item \textit{downvote} z liczbą negatywnych ocen,
    \item liczba komentarzy,
    \item \textit{share} umożliwiający skopiowanie adresu URL posta,
    \item menu kontekstowe (trzy poziome kropki),
    \item dodanie nowego komentarza.
\end{itemize}

Pod kafelkiem posta znajduje się lista komentarzy.
Komentarze mają formę zbliżoną wizualnie do posta, z wyłączeniem funkcji udostępniania.
Każdy komentarz zawiera przycisk \textit{reply}.
W przypadku komentarzy posiadających odpowiedzi wyświetlana jest ikona strzałki wraz z liczbą odpowiedzi, która po kliknięciu rozwija ich listę.

\begin{figure}[H]
    \centering
    \includegraphics[width=1\textwidth]{attachments/projekt/architektura-interfejsu-uzytkownika/forum/forum_replies}
    \caption{Szczegóły posta}
    \label{img:post-details}
\end{figure}

\subsubsection{Followed threads}
Podstrona \textit{Followed threads} prezentuje listę postów obserwowanych (rys. \ref{img:followed-threads}) przez użytkownika.
Posty są wyświetlane w formie kafelków analogicznych do tych na stronie głównej forum.
Przejście do podstrony możliwe jest poprzez wybór odpowiedniej opcji w panelu bocznym.

\begin{figure}[H]
    \centering
    \includegraphics[width=1\textwidth]{attachments/projekt/architektura-interfejsu-uzytkownika/forum/forum_followed_threads}
    \caption{Obserwowane posty}
    \label{img:followed-threads}
\end{figure}

\subsubsection{Posts History}
Podstrona \textit{Posts history} umożliwia przeglądanie historii postów (rys. \ref{img:post-history}), z którymi użytkownik wchodził w interakcje. Dostęp do podstrony realizowany jest poprzez wybór opcji \textit{Posts history} w panelu bocznym.

Posty wyświetlane są jako lista kafelków analogiczna do strony głównej.
Nad listą znajduje się przycisk umożliwiający sortowanie postów według opcji:
\begin{itemize}
    \item \textit{last} – ostatnio przeglądane,
    \item \textit{upvoted} – polubione,
    \item \textit{downvoted} – negatywnie ocenione.
\end{itemize}

Po prawej stronie nad listą znajduje się ikona trzech pionowych kropek, która umożliwia wyczyszczenie historii interakcji użytkownika.

\begin{figure}[H]
    \centering
    \includegraphics[width=1\textwidth]{attachments/projekt/architektura-interfejsu-uzytkownika/forum/forum_posts_history}
    \caption{Historia przeglądanych postów}
    \label{img:post-history}
\end{figure}