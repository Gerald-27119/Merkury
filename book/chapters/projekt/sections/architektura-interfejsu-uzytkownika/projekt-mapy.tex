%! Author = Stanisław Oziemczuk
%! Date = 29/12/2025

\subsection{Projekt mapy}
\label{subsec:projekt-mapy}

W ramach przedmiotu \glslink{prz1}{PRZ1} wykonano projekt mapy umożlwiającej przeglądanie oraz wchodzenie w interakcje
ze znajdującymi się na niej \glslink{spot}{spotami}.

\subsubsection{Rodzaje mapy}
\label{subsubsec:rodzaje-mapy}

Mapę podzielono na trzy rodzaje:
\begin{itemize}
    \item ogólną (rys. \ref{fig:map-general}) \textendash \space zawiera informacje o obiektach (sklepy, restauracje, nazwy ulic, parków, rzek, itp.)
    \item ze zmniejszoną szczegółowością (rys. \ref{fig:map-less-details}) \textendash \space mapa ogólna, w której nazwy części obiektów
    zastąpiono ikonami lub nie są one wyświetlane
    \item strefami lotów \glslink{PANSA}{PANSA} (rys. \ref{fig:map-pansa}) \textendash \space zaznaczone strefy lotów informują
    gdzie można legalnie latać dronem
\end{itemize}

Podział ten zapewnia użytkownikowi swobodne przeglądanie mapy w zależności od potrzeb.
Na każdym typie zamieszczono przyciski do zmniejszana i zwiększania poziomu przybliżenia, a także pokazania obecnej
lokalizacji użytkownika.
Ponadto w projekcie pole do wyszukiwania \glslink{spot}{spotów} po nazwie, a także przycisk do wyświetlenia ich listy w
widocznym obszarze.
Prezentowane są również ogólne informacje pogodowe z możliwością otworzenia panelu zawierającego szczegółowe dane.

\begin{figure}[H]
    \centering
    \includegraphics[width=1\textwidth]{attachments/projekt/architektura-interfejsu-uzytkownika/mapa/mapa_ogolna}
    \caption{Mapa ogólna}
    \label{fig:map-general}
\end{figure}
\noindent

\begin{figure}[H]
    \centering
    \includegraphics[width=1\textwidth]{attachments/projekt/architektura-interfejsu-uzytkownika/mapa/mapa_malo_szczegolow}
    \caption{Mapa ze zmniejszoną szczegółowością}
    \label{fig:map-less-details}
\end{figure}
\noindent

\begin{figure}[H]
    \centering
    \includegraphics[width=1\textwidth]{attachments/projekt/architektura-interfejsu-uzytkownika/mapa/mapa_pansa}
    \caption{Mapa ze strefami lotów PANSA}
    \label{fig:map-pansa}
\end{figure}
\noindent

\subsubsection{Panel ze szczegółami spota}
\label{subsubsec:panel-ze-szczegolami-spota}

Panel ze szczegółami \glslink{spot}{spota} (rys. \ref{fig:map-spot-details-1}) zaprojektowano jako prostokąt pojawiający się po lewej stronie ekranu, dzięki
czemu umożliwia dalsze przeglądanie mapy.
W górnej jego części wyświetlane są informacje o lokalizacji wybranego miesjca oznaczone ikoną markera, a po ich prawej
stronie przycisk zamykający panel.
Poniżej prezentowane są nazwa wraz oceną \glslink{spot}{spota} w gwiazdkach wraz z liczbą opinii, jego tagi oraz opis.
Kolejną częścią panelu jest galeria multimediów.
Znajdują się w niej zdjęcia i filmy przewijane w formie karuzeli.
Pod nią umieszczone są przyciski akcji oznaczone ikonami ułatwiającymi rozpoznanie ich przeznaczenia.
Umożlwiają nawigację do \glslink{spot}{spota}, zapisanie go do listy ulubionych, udostępnienie lub dodanie multimedii.
Następny obszar przeznaczony jest na wyświetlanie komentarzy.
Na jego górze znajduje się przycisk otwierający \glslink{modal}{modal} z formularzem do dodania nowego komentarza.
Początkowo wyświetlane jest kilka pierwszych elementów, poniżej których znajduje się przycisk „Show More”.
Po jego kliknięciu prezentowane są wszyskie komentarze w formie przewijalnej listy.
Pojedynczy element zawiera dane o autorze (nazwę oraz zdjęcie profilowe), treść tekstową, zdjęcia i filmy oraz
przyciski do polubienia lub niepolubienia.
Komentarz może zawierać zero lub więcej multimedii, w przypadku gdy ich liczba nie przekracza trzech, wyświeltane są obok
siebie.
Natomiast jeśli jest ich więcej, widoczne są trzy pierwsze a na ostatnim z nich znajduje się przycisk „see more”
(rys. \ref{fig:map-spot-details-2}), którego kliknięcie powoduje ukazanie pozostałych elementów (rys. \ref{fig:map-spot-details-3}).

\begin{figure}[H]
    \centering
    \includegraphics[width=1\textwidth]{attachments/projekt/architektura-interfejsu-uzytkownika/mapa/mapa_szczegoly_spota1}
    \caption{Panel ze szczegółami spota}
    \label{fig:map-spot-details-1}
\end{figure}
\noindent

\begin{figure}[H]
    \centering
    \includegraphics[width=1\textwidth]{attachments/projekt/architektura-interfejsu-uzytkownika/mapa/mapa_szczegoly_spota2}
    \caption{Panel ze szczegółami spota (przycisk do pokazania wszyskich multimediów komentarza)}
    \label{fig:map-spot-details-2}
\end{figure}
\noindent

\begin{figure}[H]
    \centering
    \includegraphics[width=1\textwidth]{attachments/projekt/architektura-interfejsu-uzytkownika/mapa/mapa_szczegoly_spota3}
    \caption{Panel szczegółów spota (widoczne są wszystkie multimedia komentarza)}
    \label{fig:map-spot-details-3}
\end{figure}
\noindent

\subsubsection{Panel z pogodą spota}
\label{subsubsec:panel-z-pogoda-spota}

Wyświetlany jest po prawej stronie ekranu, dzięki czemu nie zasłania panelu ze szczegółami \glslink{spot}{spota} i umożliwia
dalsze przeglądanie mapy (rys. \ref{fig:map-weather}).
Znajdują się na nim szczegółowe dane pogodowe wybranego miejsca.
W górnej części panelu wyświetlane są podstawowe informacje: miasto, aktualny czas, temperatura oraz ikona wraz ze słownym opisem
stanu pogody.
Całość umieszczona jest na tle z gradientem odcieni niebieskiego, który zwraca pierwszą uwagę na element.
Poniżej w formie kafelek ułożonych w siatkę 2x2.
Z odpowiednimi ikonami prezentują one informacje o prawdopodobieństwie opadów, punkcie rosy, indeksie UV oraz
poziomie wilgotności.
Następnie znajduje się część wyświetlająca prędkość wiatru na różnych wysokościach.
Na jej lewej stronie prezentowana jest szybkość wiatru z możliwością zmiany jednostki.
Na prawo od niej umieszczone są przyciski umożliwiające wybór wysokości, dla której mają zostać zaprezentowane dane.
Ostatnim elementem jest wykres przedstawiający prognozę pogody ze zmianami oznaczonymi co godzinę.
Zawiera prawdopodobieństwo opadów, ikonę opisującą stan pogody oraz krzywą obrazującą wartości tempertury.

\begin{figure}[H]
    \centering
    \includegraphics[width=1\textwidth]{attachments/projekt/architektura-interfejsu-uzytkownika/mapa/mapa_pogoda_spota}
    \caption{Panel ze szczegółowymi danymi pogodowymi spota}
    \label{fig:map-weather}
\end{figure}
\noindent
