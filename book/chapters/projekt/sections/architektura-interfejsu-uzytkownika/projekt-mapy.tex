%! Author = Stanisław Oziemczuk
%! Date = 29/12/2025

\subsection{Projekt mapy}
\label{subsec:projekt-mapy}

W ramach przedmiotu \glslink{prz1}{PRZ1} wykonano projekt mapy umożlwiającej przeglądanie oraz wchodzenie w interakcje
ze znajdującymi się na niej \glslink{spot}{spotami}.

\subsubsection{Rodzaje mapy}
\label{subsubsec:rodzaje-mapy}

Mapę podzielono na trzy rodzaje:
\begin{itemize}
    \item ogólną (rys. \ref{fig:map-general}) \textendash \space zawiera informacje o obiektach (sklepy, restauracje, nazwy ulic, parków, rzek, itp.)
    \item ze zmniejszoną szczegółowością (rys. \ref{fig:map-less-details}) \textendash \space mapa ogólna, w której nazwy części obiektów
    zastąpiono ikonami lub nie są one wyświetlane
    \item strefami lotów \glslink{PANSA}{PANSA} (rys. \ref{fig:map-pansa}) \textendash \space zaznaczone strefy lotów informują
    gdzie można legalnie latać dronem
\end{itemize}

Podział ten zapewnia użytkownikowi swobodne przeglądanie mapy w zależności od potrzeb.
Na każdym typie zamieszczono przyciski do zmniejszana i zwiększania poziomu przybliżenia, a także pokazania obecnej
lokalizacji użytkownika.
Ponadto w projekcie pole do wyszukiwania \glslink{spot}{spotów} po nazwie, a także przycisk do wyświetlenia ich listy w
widocznym obszarze.
Prezentowane są również ogólne informacje pogodowe z możliwością otworzenia panelu zawierającego szczegółowe dane.

\begin{figure}[H]
    \centering
    \includegraphics[width=1\textwidth]{attachments/projekt/architektura-interfejsu-uzytkownika/mapa/mapa_ogolna}
    \caption{Mapa ogólna}
    \label{fig:map-general}
\end{figure}
\noindent

\begin{figure}[H]
    \centering
    \includegraphics[width=1\textwidth]{attachments/projekt/architektura-interfejsu-uzytkownika/mapa/mapa_malo_szczegolow}
    \caption{Mapa ze zmniejszoną szczegółowością}
    \label{fig:map-less-details}
\end{figure}
\noindent

\begin{figure}[H]
    \centering
    \includegraphics[width=1\textwidth]{attachments/projekt/architektura-interfejsu-uzytkownika/mapa/mapa_pansa}
    \caption{Mapa ze strefami lotów PANSA}
    \label{fig:map-pansa}
\end{figure}
\noindent
