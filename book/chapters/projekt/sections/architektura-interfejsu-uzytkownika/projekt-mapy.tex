%! Author = Stanisław Oziemczuk
%! Date = 29/12/2025

\subsection{Projekt mapy}
\label{subsec:projekt-mapy}

W ramach przedmiotu \glslink{prz1}{PRZ1} wykonano projekt mapy umożliwiającej przeglądanie oraz wchodzenie w interakcje
ze znajdującymi się na niej \glslink{spot}{spotami}.

\subsubsection{Rodzaje mapy}
\label{subsubsec:rodzaje-mapy}

Mapę podzielono na trzy rodzaje:
\begin{itemize}
    \item ogólną (rys. \ref{fig:map-general}) \textendash \space zawiera informacje o obiektach (sklepy, restauracje, nazwy ulic, parków, rzek, itp.)
    \item ze zmniejszoną szczegółowością (rys. \ref{fig:map-less-details}) \textendash \space mapa ogólna, w której nazwy części obiektów
    zastąpiono ikonami lub nie są one wyświetlane
    \item strefami lotów \glslink{PANSA}{PANSA} (rys. \ref{fig:map-pansa}) \textendash \space zaznaczone strefy lotów informują
    gdzie można legalnie latać dronem
\end{itemize}

Podział ten zapewnia użytkownikowi swobodne przeglądanie mapy w zależności od potrzeb.
Na każdym typie zamieszczono przyciski do zmniejszana i zwiększania poziomu przybliżenia, a także pokazania obecnej
lokalizacji użytkownika.
Ponadto w projekcie znajduje się pole do wyszukiwania \glslink{spot}{spotów} po nazwie, a także przycisk do wyświetlenia ich listy w
widocznym obszarze.
Prezentowane są również ogólne informacje pogodowe z możliwością otworzenia panelu zawierającego szczegółowe dane.

\begin{figure}[H]
    \centering
    \includegraphics[width=1\textwidth]{attachments/projekt/architektura-interfejsu-uzytkownika/mapa/mapa_ogolna}
    \caption{Mapa ogólna}
    \label{fig:map-general}
\end{figure}
\noindent

\begin{figure}[H]
    \centering
    \includegraphics[width=1\textwidth]{attachments/projekt/architektura-interfejsu-uzytkownika/mapa/mapa_malo_szczegolow}
    \caption{Mapa ze zmniejszoną szczegółowością}
    \label{fig:map-less-details}
\end{figure}
\noindent

\begin{figure}[H]
    \centering
    \includegraphics[width=1\textwidth]{attachments/projekt/architektura-interfejsu-uzytkownika/mapa/mapa_pansa}
    \caption{Mapa ze strefami lotów PANSA}
    \label{fig:map-pansa}
\end{figure}
\noindent

\subsubsection{Panel ze szczegółami spota}
\label{subsubsec:panel-ze-szczegolami-spota-projekt}

Panel ze szczegółami \glslink{spot}{spota} (rys. \ref{fig:map-spot-details-1}) zaprojektowano jako prostokąt pojawiający się po lewej stronie ekranu, dzięki
czemu umożliwia dalsze przeglądanie mapy.
W górnej jego części wyświetlane są informacje o lokalizacji wybranego miejsca oznaczone ikoną markera, a po ich prawej
stronie przycisk zamykający panel.
Poniżej prezentowane są nazwa oraz ocena \glslink{spot}{spota} w gwiazdkach wraz z liczbą opinii, jego tagi i opis.
Kolejną częścią panelu jest galeria multimediów.
Znajdują się w niej zdjęcia i filmy przewijane w formie karuzeli.
Pod nią umieszczone są przyciski akcji oznaczone ikonami ułatwiającymi rozpoznanie ich przeznaczenia.
Umożliwiają nawigację do \glslink{spot}{spota}, zapisanie go do listy ulubionych, udostępnienie lub dodanie multimedii.
Następny obszar przeznaczony jest na wyświetlanie komentarzy.
Na jego górze znajduje się przycisk otwierający \glslink{modal}{modal} z formularzem do dodania nowego komentarza.
Początkowo wyświetlane jest kilka pierwszych elementów, poniżej których znajduje się przycisk „Show More”.
Po jego kliknięciu prezentowane są wszystkie komentarze w formie przewijalnej listy.
Pojedynczy element zawiera dane o autorze (nazwę oraz zdjęcie profilowe), treść tekstową, zdjęcia i filmy oraz
przyciski do polubienia lub niepolubienia.
Komentarz może zawierać zero lub więcej multimedii, w przypadku gdy ich liczba nie przekracza trzech, wyświetlane są obok
siebie.
Natomiast jeśli jest ich więcej, widoczne są trzy pierwsze a na ostatnim z nich znajduje się przycisk „see more”
(rys. \ref{fig:map-spot-details-2}), którego kliknięcie powoduje ukazanie pozostałych elementów (rys. \ref{fig:map-spot-details-3}).

\begin{figure}[H]
    \centering
    \includegraphics[width=1\textwidth]{attachments/projekt/architektura-interfejsu-uzytkownika/mapa/mapa_szczegoly_spota1}
    \caption{Panel ze szczegółami spota}
    \label{fig:map-spot-details-1}
\end{figure}
\noindent

\begin{figure}[H]
    \centering
    \includegraphics[width=1\textwidth]{attachments/projekt/architektura-interfejsu-uzytkownika/mapa/mapa_szczegoly_spota2}
    \caption{Panel ze szczegółami spota (przycisk do pokazania wszystkich multimediów komentarza)}
    \label{fig:map-spot-details-2}
\end{figure}
\noindent

\begin{figure}[H]
    \centering
    \includegraphics[width=1\textwidth]{attachments/projekt/architektura-interfejsu-uzytkownika/mapa/mapa_szczegoly_spota3}
    \caption{Panel szczegółów spota (widoczne są wszystkie multimedia komentarza)}
    \label{fig:map-spot-details-3}
\end{figure}
\noindent

\subsubsection{Panel z pogodą spota}
\label{subsubsec:panel-z-pogoda-spota}

Wyświetlany jest po prawej stronie ekranu, dzięki czemu nie zasłania panelu ze szczegółami \glslink{spot}{spota} i umożliwia
dalsze przeglądanie mapy (rys. \ref{fig:map-weather}).
Znajdują się na nim szczegółowe dane pogodowe wybranego miejsca.
W górnej części panelu wyświetlane są podstawowe informacje: miasto, aktualny czas, temperatura oraz ikona wraz ze słownym opisem
stanu pogody.
Całość umieszczona jest na tle z gradientem odcieni niebieskiego, który zwraca pierwszą uwagę na element.
Poniżej w formie kafelek ułożonych w siatkę 2x2 prezentowane są dodatkowe wskaźniki.
Z odpowiednimi ikonami przedstawiają one informacje o prawdopodobieństwie opadów, punkcie rosy, indeksie UV oraz
poziomie wilgotności.
Następnie znajduje się część wyświetlająca prędkość wiatru na różnych wysokościach.
Na jej lewej stronie prezentowana jest szybkość wiatru z możliwością zmiany jednostki.
Na prawo od niej umieszczone są przyciski umożliwiające wybór wysokości, dla której mają zostać zaprezentowane dane.
Ostatnim elementem jest wykres przedstawiający prognozę pogody ze zmianami oznaczonymi co godzinę.
Zawiera prawdopodobieństwo opadów, ikonę opisującą stan pogody oraz krzywą obrazującą wartości temperatury.

\begin{figure}[H]
    \centering
    \includegraphics[width=1\textwidth]{attachments/projekt/architektura-interfejsu-uzytkownika/mapa/mapa_pogoda_spota}
    \caption{Panel ze szczegółowymi danymi pogodowymi spota}
    \label{fig:map-weather}
\end{figure}
\noindent

\subsubsection{Formularz dodania komentarza}
\label{subsubsec:formularz-dodania-komentarza}

Został zaprojektowany jako \glslink{modal}{modal} (rys. \ref{fig:spot-comment-modal}) otwierany dedykowanym przyciskiem akcji.
Zawiera formularz umożliwiający dodanie komentarza do wybranego \glslink{spot}{spota}.
Jego pola to ocena w gwiazdkach oraz tekst.
Panel nad polem tekstowym umożliwia jego stylizację, a także dodanie multimediów.
Nad formularzem znajduje się nazwa wskazanego miejsca, natomiast pod nim
umieszczone są dwa przyciski: do publikacji komentarza oraz zamknięcia \glslink{modal}{modalu}.
Wycentrowanie elementu oraz przyciemnienie tła pozwala skupić uwagę na wpisywanej treści.

\begin{figure}[H]
    \centering
    \includegraphics[width=1\textwidth]{attachments/projekt/architektura-interfejsu-uzytkownika/mapa/add_spot_comment_modal}
    \caption{Modal z formularzem do dodawania komentarza}
    \label{fig:spot-comment-modal}
\end{figure}
\noindent

\subsubsection{Lista z wynikami wyszukiwania}
\label{subsubsec:lista-z-wynikami-wyszukiwania}

Panel zawierający listę \glslink{spot}{spotów} będących wynikiem wyszukiwania po nazwie lub widocznego obszaru mapy
(rys. \ref{fig:spots-list}).
Zaprojektowano jego wyświetlanie po lewej stronie ekranu, co umożliwia dalsze przeglądanie mapy.
Elementy wyświetlane są w formie kafelek z czarnym, wyróżniającym się tłem.
Każdy z nich po swojej lewej stronie zawiera zdjęcie danego \glslink{spot}{spota}, a na prawo od niego
nazwę, ocenę w gwiazdkach i liczbę ocen oraz tagi.
W górnej części panelu umieszczone są pola do filtrowania oraz sortowania listy.

\begin{figure}[H]
    \centering
    \includegraphics[width=1\textwidth]{attachments/projekt/architektura-interfejsu-uzytkownika/mapa/spots_list}
    \caption{Lista z wynikami wyszukiwania}
    \label{fig:spots-list}
\end{figure}
\noindent

\subsubsection{Duża galeria multimediów}
\label{subsubsec:duza-galeria-multimediow}

Na projekt mapy składa się również duża galeria multimediów (rys. \ref{fig:spots-media-gallery}),
zawierająca zdjęcia i filmy dodane do wybranego \glslink{spot}{spota}.
Po jej lewej stronie znajduje się lista multimediów z umieszczonym nad nią panelem do sortowania i filtrowania.
Możliwe do wyboru opcje prezentowane są w formie przycisków zmieniających wygląd po ich kliknięciu.
Po prawej stronie listy wyświetlany jest powiększony obecnie wybrany element.
Tę część można rozszerzyć na resztę ekranu poprzez schowanie listy za pomocą przycisku ze strzałką wskazującą
kierunek jej ruchu po kliknięciu.
Pod wybranym multimedia znajduje się panel z dostępnymi akcjami: powiększeniem elementu na cały ekran, pobraniem go oraz
polubieniem.
Obok wyświetlana jest liczba polubień.
W lewym dolnym rogu umieszczony został licznik elementów wybranego typu (zdjęcia lub filmy) informujący o ich całkowitej
liczbie, jak i obecnym położeniu.
Natomiast na górze wyświetlany jest panel z wybranym typem multimediów, umożliwiający również jego zmianę.
Element jest częściowo przezroczysty, aby nie zasłaniać całkowicie wybranego elementu.
Prezentowany jest również przycisk do zakładki z komentarzami, wskazujący również ich ilość.
Po prawej stronie znajduje się przycisk udostępnienia elementu oraz informacja o liczbie jego wyświetleń.
Przechodzenie między multimediami jest możliwe za pomocą strzałek pojawiających się gdy użytkownik zbliży się do nich
myszką.

\begin{figure}[H]
    \centering
    \includegraphics[width=1\textwidth]{attachments/projekt/architektura-interfejsu-uzytkownika/mapa/spots_media_gallery}
    \caption{Duża galeria multimediów}
    \label{fig:spots-media-gallery}
\end{figure}
\noindent
