%! Author = kacper
%! Date = 12/10/2025

\subsection{Kacper Badek}
\label{subsec:kacper-badek}

Na pracę nad projektem poświęciłem łącznie XXX godzin, w tym:
\begin{itemize}
    \item 29 godzin \textendash \space zaprojektowanie interfejsu użytkownika modułu forum w figmie
    \item 264 godzin \textendash \space prace deweloperskie
    \item 68? godzin \textendash \space tworzenie dokumentacji
    \item XX godzin \textendash \space udział w spotkaniach zespołu projektowego oraz pomoc w rozwiązywaniu problemów związanych z realizowanymi zadaniami
    \item 30 godzin \textendash \space przeprowadzanie \glslink{review-kodu}{review kodu}
\end{itemize}

\subsubsection{Prace deweloperskie}

Na początku projektu odpowiadałem za implementację formularza logowania po stronie \glslink{frontend}{frontendu} oraz jego integrację z \glslink{backend}{backendowym} \glslink{api}{API}.
Następnie zaimplementowałem mechanizm resetowania hasła, obejmujący generowanie i weryfikację tokenów oraz ich cykliczne usuwanie.
Zajmowałem się również przygotowaniem i refaktoryzacją szablonów wiadomości e-mail.
Zmieniłem sposób wysyłania wiadomości e-mail w systemie na \glslink{asynchroniczność}{asynchroniczny}, dodając mechanizm ponawiania prób oraz logowanie błędów.

W ramach modułu mapy przygotowałem dane deweloperskie w postaci przykładowych spotów oraz zaimplementowałem możliwość dodawania ich do ulubionych i przeglądania zapisanych lokalizacji,
a także \glslink{backend}{backend} i \glslink{frontend}{frontend} komentarzy do spotów z możliwością głosowania.

Największą część projektu stanowiła praca nad modułem forum, za który byłem w pełni odpowiedzialny.
Rozpocząłem ją od zaprojektowania interfejsu w Figmie w ramach przedmiotu \glslink{prz1}{PRZ1}, a następnie zaimplementowałem \glslink{backend}{backend} oraz \glslink{frontend}{frontend} forum.
Zrealizowałem obsługę postów i komentarzy (\glslink{crud}{CRUD}), funkcję odpowiadania na komentarze, sortowanie i paginację, edycję treści z wykorzystaniem edytora rich text Tiptap, system głosowania, zgłaszania treści, obserwowania postów oraz wyszukiwarkę z filtrami.
Dodałem także widok postów obserwowanych przez użytkownika, sekcję prezentującą najpopularniejsze posty z ubiegłego miesiąca oraz regulamin forum.
Zadbałem również o obsługę jasnego i ciemnego motywu kolorystycznego.

Równolegle z rozwojem forum zintegrowałem \glslink{backend}{backend} z usługą \glslink{azure-blob-storage}{Azure Blob Storage}, przeznaczoną do przechowywania plików multimedialnych wykorzystywanych w systemie.

\subsubsection{Tworzenie dokumentacji}

W ramach dokumentacji pracy inżynierskiej opracowałem następujące rozdziały i podrozdziały:
\begin{itemize}
    \item \hyperref[ch:wstep]{\ref*{ch:wstep} Wstęp}
    \item \hyperref[ch:opis-problemu]{\ref*{ch:opis-problemu} Opis problemu}(z wyjątkiem opisu udziałowców)
    \item \hyperref[sec:analiza-ryzyka]{\ref*{sec:analiza-ryzyka} Analiza ryzyka}
    \item \hyperref[subsubsec:wymagania-ogolne-dla-forum]{\ref*{subsubsec:wymagania-ogolne-dla-forum} Wymagania ogólne dla forum}
    \item \hyperref[subsubsec:wymagania-funkcjonalne-dla-forum]{\ref*{subsubsec:wymagania-funkcjonalne-dla-forum} Wymagania funkcjonalne dla forum}
    \item \hyperref[subsubsec:wymagania-pozafunkcjonalne-dla-forum]{\ref*{subsubsec:wymagania-pozafunkcjonalne-dla-forum} Wymagania pozafunkcjonalne dla forum}
    \item \hyperref[subsec:forum-frontend]{\ref*{subsec:forum-frontend} Implementacja frontendu forum}
    \item \hyperref[subsec:projekt-forum]{\ref*{subsec:projekt-forum} Projekt forum}
    \item \hyperref[subsec:system-mailowy]{\ref*{subsec:system-mailowy} Implementacja systemu mailowego}
    \item \hyperref[sec:strona-forum]{\ref*{sec:strona-forum} Prezentacja systemu forum}
    \item \hyperref[subsec:kacper-badek]{\ref*{subsec:kacper-badek} ten rozdział}
\end{itemize}

Podczas opracowywania powyższych treści uzupełniłem również \texttt{Słownika pojęć i skrótów} oraz \texttt{Bibliografię} o niezbędne pozycje.