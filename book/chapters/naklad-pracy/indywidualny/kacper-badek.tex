%! Author = kacper
%! Date = 12/10/2025

\subsection{Kacper Badek}
\label{subsec:kacper-badek}

Na prace związane z realizacją projektu poświęciłem łącznie X godzin, w tym:
\begin{itemize}
    \item 29 godzin \textendash \space zaprojektowanie interfejsu użytkownika modułu forum w figmie
    \item B godzin \textendash \space prace deweloperskie
    \item C godzin \textendash \space tworzenie dokumentacji
    \item D godzin \textendash \space udział w spotkaniach zespołu projektowego oraz pomoc w rozwiązywaniu problemów związanych z realizowanymi zadaniami
    \item E godzin \textendash \space przeprowadzanie \glslink{review-kodu}{review kodu}
\end{itemize}

\subsubsection{Prace deweloperskie}

Jednym z moich pierwszych zadań było przygotowanie formularza logowania użytkownika po stronie \glslink{frontend}{frontendu} oraz jego integracja z \glslink{backend}{backendowym} API odpowiedzialnym za uwierzytelnianie użytkowników.
Następnie zaimplementowałem mechanizm resetowania hasła, obejmujący generowanie oraz weryfikację tokenów, formularz zmiany hasła, a także logikę cyklicznego usuwania przeterminowanych tokenów.
W dalszym etapie poprawiłem obsługę błędów po stronie \glslink{backend}{backendu} oraz dostosowałem treść i wygląd wiadomości e-mail wysyłanych przez system, dbając o ich czytelność i spójność wizualną.

Istotnym elementem projektu był również rozwój systemu mailowego.
Zmieniłem sposób wysyłania wiadomości po rejestracji użytkownika na asynchroniczny oraz dodałem mechanizm ponawiania prób wysyłania wiadomości e-maili z ograniczoną liczbą prób i logowaniem nieudanych operacji.
Przeprowadziłem także refaktoryzację szablonów wiadomości związanych z rejestracją i resetowaniem hasła, ujednolicając strukturę HTML oraz styl graficzny, w tym logo aplikacji.

W ramach prac nad modułem mapy przygotowałem dane deweloperskie, takie jak przykładowe \glslink{spot}{spoty} wykorzystywane do prezentacji i testów.
Zaimplementowałem funkcjonalność dodawania \glslink{spot}{spotów} do listy ulubionych po stronie \glslink{backend}{backendu} i \glslink{frontend}{frontendu}, a także widok umożliwiający użytkownikowi przeglądanie zapisanych lokalizacji.

Moją największą częścią projektu była praca nad modułem forum.
Rozpocząłem ją od przygotowania projektu interfejsu użytkownika w figmie (w ramach przedmiotu \glslink{pro}{PRO}), a następnie zaimplementowałem \glslink{backend}{backend} i \glslink{frontend}{frontend} obejmujący obsługę postów, kategorii, tagów oraz paginacji.

Równolegle do rozwoju forum zintegrowałem \glslink{backend}{backend} z usługą \glslink{azure-blob-storage}{Azure Blob Storage}, przeznaczoną do uploadu i przechowywania plików multimedialnych wykorzystywanych w całym systemie.

Przygotowałem formularz dodawania postów z wykorzystaniem edytora rich text TinyMCE, wraz z walidacją danych po stronie \glslink{backend}{backendu} i \glslink{frontend}{frontendu}.
W celu zabezpieczenia aplikacji skonfigurowałem bibliotekę jsoup do walidacji przesyłanej treści HTML, ograniczając dozwolone elementy i style.
W późniejszym etapie zdecydowałem się na zmianę edytora na Tiptap, który został skonfigurowany od podstaw, umożliwiając pełną kontrolę nad dostępnymi funkcjonalnościami edycji treści.

Forum zostało rozbudowane o mechanizmy sortowania oraz paginacji w formie infinite scroll, czytelne adresy \glslink{url}{URL} oparte na slugach tytułów oraz usprawnioną nawigację.
Wprowadzono możliwość edycji postów, a komentarze rozszerzono o funkcje dodawania, edycji i odpowiadania.
Dodatkowo zaimplementowano system głosowania i zgłaszania treści, dostępny zarówno dla wpisów, jak i komentarzy, a także funkcjonalność obserwowania wybranych dyskusji przez użytkowników.

W celu poprawy doświadczenia użytkownika dodałem struktury typu skeleton do ładowania danych, formularze dostępne globalnie poprzez stan aplikacji oraz pełne wsparcie dla jasnego i ciemnego motywu kolorystycznego.
Forum zostało również wyposażone w wyszukiwarkę umożliwiającą filtrowanie postów według tytułu, kategorii, tagów, zakresu dat oraz autora, a także stronę prezentującą wyniki wyszukiwania wraz z informacją o liczbie znalezionych elementów.

Dodatkowo przygotowałem strony listujące wszystkie kategorie i tagi alfabetycznie, stronę regulaminu forum, widok postów obserwowanych przez użytkownika oraz sekcję prezentującą najpopularniejsze posty z ubiegłego miesiąca.
Całość została zaprojektowana z myślą o czytelności, spójności wizualnej oraz wygodnej nawigacji.

Wyżej opisane zadania wymagały prac zarówno po stronie \glslink{frontend}{frontendu}, jak i \glslink{backend}{backendu}.

\subsubsection{Tworzenie dokumentacji}

W ramach tworzenia dokumentacji pracy inżynierskiej byłem odpowiedzialny za opracowanie następujących rozdziałów i podrozdziałów:
\begin{itemize}
    \item \hyperref[ch:wstep]{\ref*{ch:wstep} Wstęp}
    \item \hyperref[ch:opis-problemu]{\ref*{ch:opis-problemu} Opis problemu}(z wyjątkiem opisu udziałowców)
    \item \hyperref[sec:analiza-ryzyka]{\ref*{sec:analiza-ryzyka} Analiza ryzyka}
    \item Wymagania ogólne dla forum
    \item Wymagania funkcjonalne dla forum
    \item Wymagania pozafunkcjonalne dla forum
    \item Implementacja frontendu forum
    \item Projekt forum
    \item Implementacja systemu mailowego
    \item \hyperref[sec:strona-forum]{\ref*{sec:strona-forum} Prezentacja systemu forum}
    \item \hyperref[subsec:kacper-badek]{\ref*{subsec:kacper-badek} ninijeszy rozdział}
\end{itemize}

Podczas opracowywania powyższych treści uzupełniłem również \texttt{Słownika pojęć i skrótów} oraz \texttt{Bibliografię} o niezbędne pozycje.