%! Author = Stanisław Oziemczuk
%! Date = 05/01/2026

\subsection{Stanisław Oziemczuk}
\label{subsec:stanislaw-oziemczuk}

Na prace dotyczące projektu poświęciłem łącznie X godzin, w tym:
\begin{itemize}
    \item Y \textendash \space utworzenie projektu \glslink{ui}{UI} na figmie
    \item Z \textendash \space prace deweloperskie
    \item Q \textendash \space pisanie dokumentacji
    \item W \textendash \space spotkania zespołu projektowego
    \item V \textendash \space przeprowadzanie \glslink{review-kodu}{review kodu}
\end{itemize}

\subsubsection{Prace deweloperskie}
\label{subsubsec:prace-deweloperskie}

\subsubsection{Pisanie dokumentacji}
\label{subsubsec:pisanie-dokumentacji}

W ramach tworzenia dokumentacji naszej pracy inżynierskiej byłem odpowiedzialny za napisanie następujących rozdziałów oraz podrozdziałów:
\begin{itemize}
    \item Technologie i narzędzia
    \item Wymagania ogólne dla mapy
    \item Wymagania funkcjonalne dla mapy
    \item Wymagania pozafunkcjonalne dla mapy
    \item Projekt mapy
    \item Wzorce projektowe (oprócz wzorca \texttt{Chain of Responsibility})
    \item Mapa (jest to podrozdział rozdziału Implementacja frontendu)
    \item Implementacja OAuth
    \item Review kodu
    \item Strona mapy (podrozdział rozdziału Prezentacja systemu)
    \item ninijeszy rozdział
\end{itemize}

W ramach opracowywania wymienionych tematów dodałem do \texttt{Słownika pojęć i skrótów} oraz \texttt{Bibliografii}
niezbędne pozycje.
