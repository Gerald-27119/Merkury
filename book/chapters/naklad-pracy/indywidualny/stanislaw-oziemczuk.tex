%! Author = Stanisław Oziemczuk
%! Date = 05/01/2026

\subsection{Stanisław Oziemczuk}
\label{subsec:stanislaw-oziemczuk}

Na prace dotyczące projektu poświęciłem łącznie 592 godziny, w tym:
\begin{itemize}
    \item 38 godz. \textendash \space zaprojektowanie \glslink{ui}{UI} modułu czatu w figmie
    \item 375 godz. \textendash \space prace deweloperskie
    \item 92 godz. \textendash \space tworzenie dokumentacji
    \item 87 godz. \textendash \space przeprowadzanie \glslink{review-kodu}{review kodu}
\end{itemize}

\subsubsection{Prace deweloperskie}
\label{subsubsec:prace-deweloperskie}

Jednym z moich zadań była implementacja funkcjonalności logowania i rejestracji użytkowników za pomocą kont Google i \glslink{github}{GitHub}.
Polegało ono na integracji z serwisami autoryzacji tych provider'ów oraz ustawienia odpowiedniej
konfiguracji \glslink{oauth}{OAuth'a} w aplikacji \glslink{spring-boot}{Spring Boot}.
W ramach przedmiotu \glslink{pro}{PRO} wykonałem panel wyświetlający szczegóły wybranego \glslink{spot}{spota}, filtrowanie ich na mapie,
a także zakładkę umożlwiającą użytkownikowi zmianę danych jego konta.
Usprawniłem również plik \glslink{docker-compose}{docker-compose} oraz stworzyłem skrypty uruchamiające wszystkie niezbędne \glslink{kontener}{kontenery}.

Po zaprojektowaniu interfejsu użytkownika byłem odpowiedzialny za całościową implementację modułu mapy.
W ramach prac wybrałem nowego dostawcę \glslink{map-tiles}{kafelków mapy}, na którą następnie naniosłem \glslink{spot}{spoty} wraz z możliwością
przybliżania lub oddalania widoku, a także zaznaczeniem obecnej lokalizacji użytkownika.
Kolejnym zadaniem była implementacja panelu prezentującego informacje szczegółowe wybranego miejsca, w tym galerię
multimediów czy listę komentarzy.
Do każdego \glslink{spot}{spota} dodałem przyciski akcji umożliwiające nawigację, udostępnienie, dodanie do listy
ulubionych oraz dołączenie zdjęć i filmów.
Zaimplementowałem również \glslink{modal}{okno modalne} zawierające formularz umożliwiający opublikowanie komentarza.
Zadbałem o czytelne komunikaty w przypadku wprowadzenia niepoprawnych danych, a także poinformowanie użytkownika o
pomyślnym przebiegu operacji.
Stworzyłem też panele z pogodą \glslink{spot}{spota} \textendash \space ogólną i szczegółową.
Pierwszy z nich wyświetlany jest po kliknięciu w wybrany element i zawiera przycisk otwierający drugi.
Do pobrania informacji pogodowych wybrałem popularne \glslink{api}{API} umożliwiające wskazanie dowolnego miejsca
poprzez przekazanie jego koordynatów.
W ramach tego zadania przygotowałem wykres prezentujący prognozę pogody na trzy kolejne dni.
W celu poprawienia wydajności zadbałem o \glslink{cache}{cache'owanie} danych na \glslink{backend}{backendzie}.
Oprócz tego byłem odpowiedzialny za implementację dużej galerii multimediów \glslink{spot}{spota}, umożliwiającej wygodne przeglądanie zdjęć i filmów dodanych
poprzez komentarze lub odpowiedni formularz.
Chowana lista po lewej daje możliwość sortowania oraz filtrowania dostępnych multimediów.
Natomiast powiększenie wybranego elementu na pozostałej części ekranu pozwala przyjrzeć się szczegółom, a także wykonać akcje polubienia,
pobrania, powiększenia czy udostępnienia pliku.
Zaimplementowałem również funkcjonalności dotyczące przeszukiwania \glslink{spot}{spotów} po nazwie, a także
znalezienie położonych w widocznym obszarze mapy.
W celu ułatwienia przeglądania wyników z obydwóch wyszukiwań, ich listy mają opcje sortowania lub filtrowania.
Skonfigurowałem również \glslink{serwer}{serwer}, na którym uruchomiłem skrypt przystosowujący go do hostowania \glslink{map-tiles}{kafelków mapy}.


Podczas wykonywania prac starałem się stosować najlepsze rozwiązania, a także dbałem o przystosowanie modułu do
ekranów o różnych rozmiarach, a powstałe problemy czy niejasności konsultowałem z promotorem oraz członkami zespołu.
Wszystkie elementy interfejsu użytkownika przygotowałem w ciemnym i jasnym motywie kolorystycznym.

Wyżej opisane zadania wymagały prac zarówno po stronie \glslink{frontend}{frontendu}, jak i \glslink{backend}{backendu}.

\subsubsection{Tworzenie dokumentacji}
\label{subsubsec:tworzenie-dokumentacji}

W ramach tworzenia dokumentacji naszej pracy inżynierskiej byłem odpowiedzialny za napisanie następujących rozdziałów oraz podrozdziałów:
\begin{itemize}
    \item \hyperref[sec:technologie-i-narzedzia]{\ref*{sec:technologie-i-narzedzia} Technologie i narzędzia}
    \item \hyperref[subsubsec:wymagania-ogolne-dla-mapy]{\ref*{subsubsec:wymagania-ogolne-dla-mapy} Wymagania ogólne dla mapy}
    \item \hyperref[subsubsec:wymagania-funkcjonalne-dla-mapy]{\ref*{subsubsec:wymagania-funkcjonalne-dla-mapy} Wymagania funkcjonalne dla mapy}
    \item \hyperref[subsubsec:wymagania-pozafunkcjonalne-dla-mapy]{\ref*{subsubsec:wymagania-pozafunkcjonalne-dla-mapy} Wymagania pozafunkcjonalne dla mapy}
    \item \hyperref[subsec:projekt-mapy]{\ref*{subsec:projekt-mapy} Projekt mapy}
    \item \hyperref[sec:wzorce-projektowe]{\ref*{sec:wzorce-projektowe} Wzorce projektowe} (oprócz wzorca \texttt{Chain of Responsibility})
    \item \hyperref[subsec:mapa-frontend]{\ref*{subsec:mapa-frontend} Mapa}
    \item \hyperref[sec:implementacja-oauth]{\ref*{sec:implementacja-oauth} Implementacja OAuth}
    \item \hyperref[sec:review-kodu]{\ref*{sec:review-kodu} Review kodu}
    \item \hyperref[sec:strona-mapy]{\ref*{sec:strona-mapy} Strona mapy}
    \item \hyperref[subsec:stanislaw-oziemczuk]{\ref*{subsec:stanislaw-oziemczuk} ninijeszy rozdział}
\end{itemize}

W ramach opracowywania wymienionych tematów dodałem do \texttt{Słownika pojęć i skrótów} oraz \texttt{Bibliografii}
niezbędne pozycje.
