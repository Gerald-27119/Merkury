%! Author = Stanisław Oziemczuk
%! Date = 05/01/2026

\subsection{Stanisław Oziemczuk}
\label{subsec:stanislaw-oziemczuk}

Na prace dotyczące projektu poświęciłem łącznie X godzin, w tym:
\begin{itemize}
    \item 37 godz. 11 min. \textendash \space utworzenie projektu modułu czatu \glslink{ui}{UI} na figmie
    \item Z \textendash \space prace deweloperskie
    \item Q \textendash \space tworzenie dokumentacji
    \item W \textendash \space spotkania zespołu projektowego oraz pomoc w rozwiązywaniu problemów związanych wykonywanymi zadaniami
    \item V \textendash \space przeprowadzanie \glslink{review-kodu}{review kodu}
\end{itemize}

\subsubsection{Prace deweloperskie}
\label{subsubsec:prace-deweloperskie}

Jednym z moich zadań była implementacja funkcjonalności logowania i rejestracji użytkowników za pomocą kont Google i \glslink{github}{GitHub}.
Polegało ono na integracji z serwisami autoryzacji tych provider'ów oraz ustawienia odpowiedniej
konfiguracji \glslink{oauth}{OAuth'a} w aplikacji \glslink{SpringBoot}{Spring Boot}.
W ramach przedmiotu \glslink{pro}{PRO} wykonałem panel wyświetlający szczegóły wybranego \glslink{spot}{spota}, filtrowanie ich na mapie,
a także zakładkę umożlwiającą użytkownikowi zmianę danych jego konta.
Usprawniłem również plik \glslink{docker-compose}{docker-compose} oraz stworzyłem skrypty uruchamiające wszystkie niezbędne \glslink{kontener}{kontenery}.

Po zaprojektowaniu inferfesju użytkownika zostałem odpowiedzialny za całościową implementację modułu mapy.
W ramach prac wybrałem nowego dostawcę kafelków mapy, na którą następnie naniosłem \glslink{spot}{spoty}.
Kolejnym zadaniem była implementacja panelu prezentującego informacje szczegółowe wybranego miejsca, w tym galerię
multimediów czy listę komentarzy.
Do każdego \glslink{spot}{spota} dodałem przyciski akcji umożliwiające nawigację, udostępnienie, dodanie do listy
ulubionych oraz dodanie zdjęć i filmów do danego miejsca.
Zaimplementowałem też \glslink{modal}{okno modalne} zawierające formularz umożliwiający dodadanie komentarza.
Zadbałem o czytelne komunikaty w przypadku wprowadzenia niepoprawnych danych, a także poinformowanie użytkownika o
pomyślnym przebiegu operacji.
Stworzłem też panele z pogodą \glslink{spot}{spota} \textendash \space ogólną i szczegółową.
Pierwszy z nich wyświetlany jest po kliknięciu w wybrany element i zawiera przycisk otwierający drugi.
Do pobrania informacji pogodowych wybrałem popularne \glslink{api}{API} umożlwiające wskazanie dowolnego miejsca
poprzez przekazanie jego koordynatów.
W celu poprawienia wydajności zadbałem o \glslink{cache}{cache'owanie} danych na \glslink{backend}{backendzie}.
W ramach tego zadania przygotowałem wykres prezentujący prognozę pogody na trzy kolejne dni.
Zaimplementowełem dużą galerię multimediów \glslink{spot}{spota}, umożliwiającą wygodne przeglądanie zdjęć i filmów dodanych
poprzez komentarze lub odpowiedni formularz.
Chowana lista po lewej daje możliwość sortowania oraz filtrowania dostępnych multimediów.
Natomiast powiększenie wybranego elementu na pozostałej części ekranu pozwala przyjrzeć się szczegółom, a także wykonać akcje polubienia,
pobrania, powiększenia czy udostępnienia pliku.
Zaimplementowałem również funkcjonalności dotyczące przeszukiwania \glslink{spot}{spotów} po nazwie, a także
znalezienie widocznych w widocznym obszarze mapy.
W celu ułatwienia przeglądania wyników z obydwóch wyszukiwań, ich listy mają opcje sortowania lub filtrowania.
Skonfigurowałem również \glslink{serwer}{serwer}, na którym uruchomiłem skrypt przystosowujący go do hostowania kafelków mapy.


Podczas wykonywania prac starałem się stosować najlepsze rozwiązania, a także dbałem o przystosowanie modułu do
ekranów o różnych rozmiarów, a powstałe problemy czy niejasności konsultowałem z promotorem oraz członkami zespołu.
Wszystkie elementy interfejsu użytkownika przygotowałem w ciemnym i jasnym motywie kolorystycznym.

Wyżej opisane zadania wymagały prac zarówno po stronie \glslink{frontend}{frontendu}, jak i \glslink{backend}{backendu}.

\subsubsection{Tworzenie dokumentacji}
\label{subsubsec:tworzenie-dokumentacji}

W ramach tworzenia dokumentacji naszej pracy inżynierskiej byłem odpowiedzialny za napisanie następujących rozdziałów oraz podrozdziałów:
\begin{itemize}
    \item Technologie i narzędzia
    \item Wymagania ogólne dla mapy
    \item Wymagania funkcjonalne dla mapy
    \item Wymagania pozafunkcjonalne dla mapy
    \item Projekt mapy
    \item Wzorce projektowe (oprócz wzorca \texttt{Chain of Responsibility})
    \item Mapa (jest to podrozdział rozdziału Implementacja frontendu)
    \item Implementacja OAuth
    \item Review kodu
    \item Strona mapy (podrozdział rozdziału Prezentacja systemu)
    \item ninijeszy rozdział
\end{itemize}

W ramach opracowywania wymienionych tematów dodałem do \texttt{Słownika pojęć i skrótów} oraz \texttt{Bibliografii}
niezbędne pozycje.
