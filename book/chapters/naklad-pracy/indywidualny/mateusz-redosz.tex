%! Author = Mateusz
%! Date = 12/10/2025

\subsection{Mateusz Redosz}
\label{subsec:mateusz-redosz}

\newcommand{\linkitem}[1]{%
    \item \hyperref[#1]{\ref*{#1}~\nameref*{#1}}%
}

Na realizację projektu poświęciłem łącznie XXX godzin, z czego:
\begin{itemize}
    \item 270 godz. \textendash \space prace deweloperskie
    \item XXX godz. \textendash \space przygotowanie dokumentacji
    \item XX godz. \textendash \space \glslink{review-kodu}{review kodu}
    \item XX godz. \textendash \space spotkania dotyczące omówienia dalszych prac projektowych oraz pomocy innym członkom zespołu
    \item 49 godz. \textendash \space przygotowanie widoków w Figmie
\end{itemize}

\subsubsection{Prace deweloperskie}
\label{subsubsec:prace-deweloperskie-mateusz}

W ramach projektu odpowiadałem za implementację rejestracji użytkownika oraz częściową obsługę \glslink{token}{tokenu} \gls{jwt}
w tym automatycznym wylogowaniem użytkownika z systemu.
Pracowałem również nad wybranymi elementami \glslink{pipeline}{pipeline’u} \gls{cicd}, wspierając przygotowanie
automatyzacji budowania oraz testowania aplikacji.
Dużą część czasu poświęciłem na rozwój głównych modułów systemu, w szczególności wyszukiwarki \glslink{spot}{spotów} oraz panelu
użytkownika, zarówno na \glslink{frontend}{frontendzie}, jak i \glslink{backend}{backendzie}.
Oba moduły są \glslink{responsywnosc}{responsywne} na dużych i małych ekranach oraz zostały przygotowane w motywie jasnym i ciemnym.
Panel użytkownika przetestowałem testami \glslink{unit-tests}{jednostkowymi},
\glslink{integration-tests}{integracyjnymi} oraz \glslink{e2e-tests}{E2E}.

Dodatkowo zaimplementowałem \glslink{sidebar}{sidebar}, dbając o \glslink{responsywnosc}{responsywność} i
dopasowanie układu do różnych rozdzielczości, tak aby korzystanie z aplikacji było wygodne zarówno na
urządzeniach mobilnych, jak i na komputerach.
Na początku przedmiotu \gls{prz1} przygotowałem w Figmie projekt widoków dla panelu użytkownika,
wyszukiwarki \glslink{spot}{spotów} oraz panelu logowania i rejestracji, dbając o spójność stylu, czytelność interfejsu
oraz zgodność z założeniami funkcjonalnymi.

W trakcie prac regularnie konsultowałem postępy z zespołem, brałem udział w spotkaniach oraz pomagałem
innym osobom w rozwiązywaniu problemów związanych z implementacją i konfiguracją środowiska.
Po wykonaniu kluczowych funkcjonalności wprowadzałem poprawki wynikające z testów oraz uwag z \glslink{review-kodu}{review kodu},
skupiając się na jakości, stabilności oraz utrzymaniu jednolitego standardu kodu w projekcie.

\subsubsection{Tworzenie dokumentacji}
\label{subsubsec:tworzenie-dokumentacji-mateusz}

W ramach tworzenia dokumentacji naszej pracy inżynierskiej byłem odpowiedzialny za napisanie następujących rozdziałów oraz podrozdziałów:

\begin{itemize}
    % --- Wymagania ogólne ---
    \linkitem{subsubsec:wymagania-ogolne-dla-panelu-uzytkownika}
    \linkitem{subsubsec:wymagania-ogolne-dla-wyszukiwarki-spotow}
    \linkitem{subsubsec:wymagania-ogolne-dla-logowania-i-rejestracji}
    \linkitem{subsubsec:wymagania-ogolne-dla-motywu}

    % --- Wymagania funkcjonalne ---
    \linkitem{subsubsec:wymagania-funkcjonalne-dla-panelu-uzytkownika}
    \linkitem{subsubsec:wymagania-funkcjonalne-dla-wyszukiwarki-spotow}
    \linkitem{subsubsec:wymagania-funkcjonalne-dla-logowania-rejestracji}
    \linkitem{subsubsec:wymagania-funkcjonalne-dla-motywu}

    % --- Wymagania pozafunkcjonalne ---
    \linkitem{subsubsec:wymagania-pozafunkcjonalne-dla-panelu-uzytkownika}
    \linkitem{subsubsec:wymagania-pozafunkcjonalne-dla-wyszukiwarki-spotow}
    \linkitem{subsubsec:wymagania-pozafunkcjonalne-dla-panelu-logowania-i-rejestracji}
    \linkitem{subsubsec:wymagania-pozafunkcjonalne-dla-motywu}

    % --- Reszta rozdziałów/sekcji ---
    \linkitem{sec:architektura-systemu}
    \linkitem{sec:projekt-bazy-danych}
    \linkitem{subsec:projekt-panelu-logowania-i-rejestracji}
    \linkitem{subsec:projekt-strony-glownej}
    \linkitem{subsec:projekt-panelu-uzytkownika}
    \linkitem{subsec:struktura-projektu}
    \linkitem{subsec:endpointy-systemu}
    \linkitem{subsec:integracja-z-baza-danych}
    \linkitem{subsec:konteneryzacja}
    \linkitem{subsec:cache}
    \linkitem{subsec:struktura-aplikacji}
    \linkitem{subsec:zarzadzanie-stanem-i-przeplyw-danych}
    \linkitem{subsec:integracja-i-komunikacja-z-backendem}
    \linkitem{subsec:style}
    \linkitem{subsec:strona-glowna-frontend}
    \linkitem{subsec:panel-uzytkownika-frontend}
    \linkitem{subsec:panel-logowania-frontend}
    \linkitem{subsec:sidebar-frontend}
    \linkitem{sec:implementacja-ci-cd}
    \linkitem{ch:testy}
    \linkitem{sec:wyszukiwarka-spotow}
    \linkitem{sec:panel-logowania-i-rejestracji}
    \linkitem{sec:sidebar}
    \linkitem{sec:notification}
    \linkitem{sec:panel-uzytkownika}
    \linkitem{sec:ogolny-naklad-pracy}
    \linkitem{subsec:mateusz-redosz}
    \linkitem{ch:podsumowanie}
\end{itemize}

W trakcie tworzenia powyższych rozdziałów dodałem/am wymagane hasła do słowniczka oraz odpowiednie pozycje do
bibliografii.