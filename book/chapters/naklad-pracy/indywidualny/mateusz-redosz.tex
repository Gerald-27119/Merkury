%! Author = Mateusz
%! Date = 12/10/2025

\subsection{Mateusz Redosz}
\label{subsec:mateusz-redosz}

Na realizację projektu poświęciłem łącznie XXX godzin, z czego 256 przeznaczyłem na prace deweloperskie,
XXX na przygotowanie dokumentacji, XX godzin na \glslink{review-kodu}{review kodu}, XX na spotkania dotyczące
omówienia dalszych prac projektowych oraz pomocy innym członkom zespołu, a także 49 godzin na przygotowanie
widoków w Figmie.
Prace nad częścią deweloperską rozpocząłem 04.08.2024, a zakończyłem 08.09.2025.

W ramach projektu odpowiadałem za implementację rejestracji użytkownika oraz częściową obsługę tokenu \gls{jwt}.
Pracowałem również nad wybranymi elementami \glslink{pipeline}{pipeline’u} \gls{cicd}, wspierając przygotowanie
automatyzacji budowania oraz testowania aplikacji.
Dużą część czasu poświęciłem na rozwój głównych modułów systemu, w szczególności wyszukiwarki spotów oraz panelu
użytkownika, zarówno na \glslink{frontend}{frontendzie}, jak i \glslink{backend}{backendzie}.
Oba moduły są responsywne na dużych i małych ekranach oraz zostały przygotowane w motywie jasnym i ciemnym.
Panel użytkownika przetestowałem testami jednostkowymi, integracyjnymi oraz E2E.

Dodatkowo zaimplementowałem \glslink{sidebar}{sidebar}, dbając o \glslink{responsywnosc}{responsywność} i
dopasowanie układu do różnych rozdzielczości, tak aby korzystanie z aplikacji było wygodne zarówno na
urządzeniach mobilnych, jak i na komputerach.
Na początku przedmiotu \gls{prz1} przygotowałem w Figmie projekt widoków dla panelu użytkownika,
wyszukiwarki spotów oraz panelu logowania i rejestracji, dbając o spójność stylu, czytelność interfejsu
oraz zgodność z założeniami funkcjonalnymi.

W trakcie prac regularnie konsultowałem postępy z zespołem, brałem udział w spotkaniach oraz pomagałem
innym osobom w rozwiązywaniu problemów związanych z implementacją i konfiguracją środowiska.
Po wykonaniu kluczowych funkcjonalności wprowadzałem poprawki wynikające z testów oraz uwag z \gls{review-kodu},
skupiając się na jakości, stabilności oraz utrzymaniu jednolitego standardu kodu w projekcie.

Przy pisaniu pracy dyplomowej odpowiadałem za następujące rozdziały:
-
