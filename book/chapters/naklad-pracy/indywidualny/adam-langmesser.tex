%! Author = Adam
%! Date = 02/02/2026

\subsection{Adam Langmesser}
\label{subsec:adam-langmesser}

Na prace dotyczące projektu poświęciłem łącznie 608 godzin, w tym:
\begin{itemize}
    \item 34 godz. \textendash \space zaprojektowanie \glslink{ui}{UI} modułu mapy w Figmie
    \item 293 godz. \textendash \space prace deweloperskie
    \item 215 godz. \textendash \space tworzenie dokumentacji
    \item 66 godz. \textendash \space przeprowadzanie \glslink{review-kodu}{review kodu}
\end{itemize}

\subsubsection{Rola lidera}
\label{subsubsec:rola-lidera}

Jako lider zespołu byłem odpowiedzialny za koordynację prac, podział zadań oraz kontrolę postępów realizacji projektu.
Ponadto wypracowałem procedury włączania nowego kodu do repozytorium, obejmującą \glslink{review-kodu}{review kodu}.

\subsubsection{Prace deweloperskie}
\label{subsubsec:prace-deweloperskie-adam}

%TODO

Przygotowałem lokalne środowisko deweloperskie dla \glslink{backend}{backendu} i \glslink{frontend}{frontendu} (konfiguracja \gls{ide},
weryfikacja narzędzi budujących oraz uruchomienie projektów testowych), co ułatwiło płynne rozpoczęcie właściwych prac implementacyjnych.
Następnie wdrożyłem podstawową stronę powitalną po stronie \glslink{frontend}{frontendu} oraz zaimplementowałem fundamenty logiki logowania
i rejestracji użytkownika w \glslink{backend}{backendzie}, uwzględniając scenariusze brzegowe, takie jak próba rejestracji z zajętą nazwą użytkownika.
Rozszerzyłem też środowisko uruchomieniowe o konfigurację \gls{docker-compose} dla \glslink{backend}{backendu}
i bazy danych oraz dodałem cykliczny test weryfikujący poprawność uruchamiania aplikacji w kontenerach i jej podstawową responsywność.
W dalszej kolejności zaprezentowałem zespołowi przykładowe podejście do \glslink{integration-tests}{testów integracyjnych}
oraz \glslink{e2e-tests}{testów E2E} na \glslink{backend}{backendzie}, a równolegle przygotowałem demonstracyjny prototyp mapy
z wykorzystaniem biblioteki \gls{leaflet}, aby pokazać możliwości interaktywnego widoku mapowego,
oraz dopracowałem konfigurację \gls{eslint} po stronie \glslink{frontend}{frontendu}.
Kontynuowałem prace w obszarze bezpieczeństwa, doprecyzowując konfigurację \gls{spring-security} (role i uprawnienia),
przygotowując dane deweloperskie użytkowników oraz dodając automatyczne testy obejmujące kluczowe scenariusze logowania i rejestracji,
a następnie ustabilizowałem zestaw testów poprzez ujednolicenie asercji i rozwiązanie problemów z relacjami między
obiektami wykorzystywanymi w testach integracyjnych. Wprowadziłem kolejne
usprawnienia konfiguracji \gls{spring-security} (reguły autoryzacji i filtry),
uporządkowałem plik \texttt{.gitignore} o nowe artefakty generowane przez narzędzia
oraz poprawiłem działanie potoków \gls{cicd} dla \glslink{backend}{backendu};
jednocześnie uporządkowałem strukturę projektu po stronie serwera,
wdrożyłem mechanizm cachowania z użyciem \gls{redis}
oraz doprecyzowałem konfigurację poziomów logowania błędów,
co usprawniło diagnozowanie problemów w środowiskach deweloperskich.
Równolegle rozpocząłem implementację modułu czatu zarówno po stronie \glslink{backend}{backendowej},
jak i \glslink{frontend}{frontendowej}, dostosowując układ aplikacji oraz elementy nawigacyjne do nowej sekcji,
a także rozwinąłem procesy \gls{cicd} dla serwera, optymalizując czas budowania i doprecyzowując kroki uruchamiania testów.
Następnie znacząco rozbudowałem czat, dodając pobieranie danych i wysyłanie wiadomości z użyciem \gls{websocket},
ujednolicając model danych, upraszczając komunikację z \gls{api} oraz przebudowując strukturę \gls{redux} i komponentów,
co przełożyło się na czytelniejszy kod i spójniejsze działanie interfejsu; dodatkowo wdrożyłem usprawnienia narzędziowe
w postaci formatowania przez \gls{prettier}, skryptu \texttt{format:check} oraz weryfikacji formatowania w \gls{cicd}.
Usprawniłem mechanizm \glslink{infinite-scroll}{nieskończonego przewijania} listy czatów, przygotowałem wspólną logikę
i komponenty pomocnicze ułatwiające zespołowi korzystanie z \glslink{websocket}{websocketów} na froncie,
dodałem grupowanie wiadomości po dacie oraz obsługę wielowierszowego pola tekstowego,
a także wprowadziłem drobne korekty w potokach \gls{cicd}. Kolejnym krokiem było rozszerzenie czatu o wyszukiwanie
i wysyłanie \glslink{gif}{GIF-ów} oraz wysyłanie \gls{emoji}, a także usprawnienie komunikacji poprzez optymistyczne
wysyłanie wiadomości i dodanie nowych \glslink{endpoint}{endpointów} do stronicowanego pobierania starszych wiadomości.
Na koniec umożliwiłem rozpoczynanie i kontynuowanie prywatnych rozmów bezpośrednio z widoków społecznościowych
(list znajomych oraz obserwujących), dopracowałem obsługę \gls{emoji} i ergonomię okna wyboru \glslink{gif}{GIF-ów},
a także zaimplementowałem funkcjonalność czatów grupowych obejmującą tworzenie rozmów z wyborem uczestników,
edycję nazwy i obrazu czatu, dodawanie nowych członków oraz pełną obsługę załączników (pliki i obrazy) wraz z logiką wyboru,
podglądu i wysyłania plików również bez treści tekstowej.


\subsubsection{Tworzenie dokumentacji}
\label{subsubsec:tworzenie-dokumentacji-adam}

W ramach tworzenia dokumentacji naszej pracy inżynierskiej byłem odpowiedzialny za napisanie następujących rozdziałów oraz podrozdziałów:
\begin{itemize}
    \item \hyperref[sec:udzialowcy]{\ref*{sec:udzialowcy} Udziałowcy}
    \item \hyperref[sec:metodologia-pracy]{\ref*{sec:metodologia-pracy} Metodologia pracy}
    \item \hyperref[sec:harmonogram-projektu]{\ref*{sec:harmonogram-projektu} Harmonogram projektu}
    \item \hyperref[sec:zasoby-i-ograniczenia]{\ref*{sec:zasoby-i-ograniczenia} Zasoby i ograniczenia}
    \item \hyperref[sec:przypadki-uzycia]{\ref*{sec:przypadki-uzycia} Przypadki użycia}
    \item \hyperref[subsubsec:wymagania-ogolne-dla-chatu]{\ref*{subsubsec:wymagania-ogolne-dla-chatu} Wymagania ogólne dla czatu}
    \item \hyperref[subsubsec:wymagania-funkcjonalne-dla-chatu]{\ref*{subsubsec:wymagania-funkcjonalne-dla-chatu} Wymagania funkcjonalne dla czatu}
    \item \hyperref[subsubsec:wymagania-pozafunkcjonalne-dla-czatu]{\ref*{subsubsec:wymagania-pozafunkcjonalne-dla-czatu} Wymagania pozafunkcjonalne dla czatu}
    \item \hyperref[subsec:projekt-chatu]{\ref*{subsec:projekt-chatu} Projekt czatu}
    \item \hyperref[ch:przebieg-realizacji-projektu]{\ref*{ch:przebieg-realizacji-projektu} Przebieg realizacji projektu}
    \item \hyperref[subsubsec:chain-of-responsibility]{\ref*{subsubsec:chain-of-responsibility} Chain of Responsibility}
    \item \hyperref[subsec:bezpieczenstwo-aplikacji]{\ref*{subsec:bezpieczenstwo-aplikacji} Bezpieczeństwo aplikacji}
    \item TODO: Integracja z Tenor na backendzie
    \item TODO: czat frontend
    \item \hyperref[sec:implementacja-websocket]{\ref*{sec:implementacja-websocket} Implementacja WebSocket}
    \item \hyperref[sec:strona-chatu]{\ref*{sec:strona-chatu} Strona czatu}
    \item \hyperref[subsec:adam-langmesser]{\ref*{subsec:adam-langmesser} Adam Langmesser}
\end{itemize}
