%! Author = Adam
%! Date = 10/01/2026

\subsection{Adam Langmesser}
\label{subsec:adam-langmesser}

Na prace dotyczące projektu poświęciłem łącznie 608 godzin, w tym:
\begin{itemize}
    \item 34 godz. \textendash \space zaprojektowanie \glslink{ui}{UI} modułu mapy w Figmie
    \item 293 godz. \textendash \space prace deweloperskie
    \item 215 godz. \textendash \space tworzenie dokumentacji
    \item 66 godz. \textendash \space przeprowadzanie \glslink{review-kodu}{review kodu}
\end{itemize}

\subsubsection{Rola lidera}
\label{subsubsec:rola-lidera}

Jako lider zespołu byłem odpowiedzialny za koordynację prac, podział zadań oraz kontrolę postępów realizacji projektu.
Ponadto wypracowałem procedurę włączania nowego kodu do repozytorium, obejmującą \glslink{review-kodu}{review kodu}.
Organizowałem spotkania zespołu, konsultowałem postępy i priorytety oraz wspierałem członków zespołu
w rozwiązywaniu problemów implementacyjnych i konfiguracyjnych. W razie potrzeby uzgadniałem kluczowe decyzje projektowe
podczas konsultacji wewnątrz zespołu oraz z promotorem.

\subsubsection{Prace deweloperskie}
\label{subsubsec:prace-deweloperskie-adam}

Przygotowałem repozytorium projektu oraz zainicjalizowałem jego podstawową strukturę, tworząc odrębne moduły dla \glslink{backend}{backendu} i \glslink{frontend}{frontendu},
a następnie przygotowałem konfigurację \gls{docker-compose}, umożliwiającą uruchamianie lokalnego środowiska projektu w \glslink{kontener}{kontenerach}.
Zaprezentowałem zespołowi przykładowe podejście do \glslink{integration-tests}{testów integracyjnych} oraz \glslink{e2e-tests}{testów E2E} na \glslink{backend}{backendzie},
równolegle przygotowując demonstracyjny prototyp mapy z wykorzystaniem biblioteki \gls{leaflet}. Skonfigurowałem zabezpieczenia po stronie \glslink{serwer}{serwera},
opracowując konfigurację \gls{spring-security} w zakresie reguł autoryzacji oraz filtrów. Wdrożyłem mechanizm cachowania z użyciem \gls{redis}.

Równolegle przygotowałem podstawy pod procesy \gls{cicd} dla części \glslink{backend}{backendowej} (w tym kluczowe kroki budowania i uruchamiania testów).

Po konfiguracji bezpieczeństwa na \glslink{backend}{backendzie} moim głównym obszarem odpowiedzialności stał się moduł czatu.
Zaimplementowałem funkcjonalności: tworzenie czatu prywatnego, przeglądanie historii konwersacji, grupowanie wiadomości według daty,
wysyłanie \glslink{gif}{GIF-ów}, plików oraz \gls{emoji}, a także tworzenie czatów grupowych i dodawanie użytkowników do istniejących rozmów grupowych
— całość zrealizowano w oparciu o komunikację w czasie rzeczywistym z użyciem \glslink{websocket}{WebSocket}.

\subsubsection{Tworzenie dokumentacji}
\label{subsubsec:tworzenie-dokumentacji-adam}

W ramach tworzenia dokumentacji naszej pracy inżynierskiej byłem odpowiedzialny za napisanie następujących rozdziałów oraz podrozdziałów:
\begin{itemize}
    \item \hyperref[sec:udzialowcy]{\ref*{sec:udzialowcy} Udziałowcy}
    \item \hyperref[sec:metodologia-pracy]{\ref*{sec:metodologia-pracy} Metodologia pracy}
    \item \hyperref[sec:harmonogram-projektu]{\ref*{sec:harmonogram-projektu} Harmonogram projektu}
    \item \hyperref[sec:zasoby-i-ograniczenia]{\ref*{sec:zasoby-i-ograniczenia} Zasoby i ograniczenia}
    \item \hyperref[sec:przypadki-uzycia]{\ref*{sec:przypadki-uzycia} Przypadki użycia}
    \item \hyperref[subsubsec:wymagania-ogolne-dla-chatu]{\ref*{subsubsec:wymagania-ogolne-dla-chatu} Wymagania ogólne dla czatu}
    \item \hyperref[subsubsec:wymagania-funkcjonalne-dla-chatu]{\ref*{subsubsec:wymagania-funkcjonalne-dla-chatu} Wymagania funkcjonalne dla czatu}
    \item \hyperref[subsubsec:wymagania-pozafunkcjonalne-dla-czatu]{\ref*{subsubsec:wymagania-pozafunkcjonalne-dla-czatu} Wymagania pozafunkcjonalne dla czatu}
    \item \hyperref[subsec:projekt-chatu]{\ref*{subsec:projekt-chatu} Projekt czatu}
    \item \hyperref[ch:przebieg-realizacji-projektu]{\ref*{ch:przebieg-realizacji-projektu} Przebieg realizacji projektu}
    \item \hyperref[subsubsec:chain-of-responsibility]{\ref*{subsubsec:chain-of-responsibility} Chain of Responsibility}
    \item \hyperref[subsec:bezpieczenstwo-aplikacji]{\ref*{subsec:bezpieczenstwo-aplikacji} Bezpieczeństwo aplikacji}
    \item \hyperref[subsec:integracja-z-dostawca-gif-ow]{\ref*{subsec:integracja-z-dostawca-gif-ow} Integracja z dostawcą GIF-ów}
    \item \hyperref[subsec:chat-frontend]{\ref*{subsec:chat-frontend} Czat}
    \item \hyperref[sec:implementacja-websocket]{\ref*{sec:implementacja-websocket} Implementacja WebSocket}
    \item \hyperref[sec:strona-chatu]{\ref*{sec:strona-chatu} Strona czatu}
    \item \hyperref[subsec:adam-langmesser]{\ref*{subsec:adam-langmesser} Adam Langmesser}
\end{itemize}

W trakcie opracowywania powyższych rozdziałów uzupełniłem również \texttt{Słownik pojęć i skrótów}
oraz dodałem wymagane pozycje do \texttt{Bibliografii}.
