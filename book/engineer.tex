\documentclass{sprz}

\usepackage[hidelinks]{hyperref}
\usepackage{glossaries}
\usepackage[polish]{babel}
\usepackage{import}

%HTTP Methods Coloring
\lstdefinelanguage{http}{
    alsoletter={:},
    literate=
        {GET}{{{\color{blue}\textbf{GET}}}}3
        {POST}{{{\color{green!60!black}\textbf{POST}}}}4
        {DELETE}{{{\color{red}\textbf{DELETE}}}}6
        {PUT}{{{\color{orange!90!black}\textbf{PUT}}}}3
        {PATCH}{{{\color{teal}\textbf{PATCH}}}}5
        {OPTIONS}{{{\color{magenta!80!black}\textbf{OPTIONS}}}}7
}

%JSON Coloring
\lstdefinelanguage{json}{
    basicstyle=\footnotesize\ttfamily,
    showstringspaces=false,
    breaklines=true,
    frame=single,
    framerule=0pt,
    backgroundcolor=\color{gray!3},
    literate=
    *{true}{{{\color{blue}\bfseries true}}}4
        {false}{{{\color{blue}\bfseries false}}}5
        {null}{{{\color{magenta}\bfseries null}}}4
        {:}{{{\color{black}:{}}}}1
        {,}{{{\color{black},{}}}}1
        {"}{{{\color{gray}"}}}1
        {0}{{{\color{purple!70!black}0}}}1
        {1}{{{\color{purple!70!black}1}}}1
        {2}{{{\color{purple!70!black}2}}}1
        {3}{{{\color{purple!70!black}3}}}1
        {4}{{{\color{purple!70!black}4}}}1
        {5}{{{\color{purple!70!black}5}}}1
        {6}{{{\color{purple!70!black}6}}}1
        {7}{{{\color{purple!70!black}7}}}1
        {8}{{{\color{purple!70!black}8}}}1
        {9}{{{\color{purple!70!black}9}}}1,
    morestring=[b]",
    stringstyle=\color{green!40!black}
}

% Http Errors coloring
\lstdefinelanguage{httperror}{
    basicstyle=\footnotesize\ttfamily,
    showstringspaces=false,
    breaklines=true,
    frame=single,
    backgroundcolor=\color{gray!3},
    literate=*
        {200\ OK}{{{\color{green!50!black}\textbf{200 OK}}}}6
        {201\ Created}{{{\color{green!40!black}\textbf{201 Created}}}}12
        {204\ No\ Content}{{{\color{green!30!black}\textbf{204 No Content}}}}14
        {400\ Bad\ Request}{{{\color{orange!80!black}\textbf{400 Bad Request}}}}15
        {401\ Unauthorized}{{{\color{orange!70!black}\textbf{401 Unauthorized}}}}16
        {403\ Forbidden}{{{\color{orange!90!black}\textbf{403 Forbidden}}}}13
        {404\ Not\ Found}{{{\color{red!80!black}\textbf{404 Not Found}}}}13
        {409\ Conflict}{{{\color{red!70!black}\textbf{409 Conflict}}}}12
        {500\ Internal\ Server\ Error}{{{\color{red!90!black}\textbf{500 Internal Server Error}}}}25,
}





\makeglossaries
\addbibresource{bibliography.bib}

%! Author = Mateusz Redosz
%! Date = 20/09/2025

% Słownik pojęć

\newglossaryentry{backend}
{
    name={Backend},
    description={Część aplikacji odpowiedzialna za logikę biznesową, przetwarzanie danych i komunikację z bazą danych. Działa po stronie serwera i obsługuje żądania wysyłane przez frontend}
}

\newglossaryentry{frontend}
{
    name={Frontend},
    description={Warstwa aplikacji odpowiedzialna za interfejs użytkownika oraz interakcję z użytkownikiem. Zazwyczaj tworzona przy użyciu technologii takich jak HTML, CSS i JavaScript}
}

\newglossaryentry{baza-danych}
{
    name={Baza danych},
    description={Zbiór uporządkowanych danych przechowywanych w sposób umożliwiający ich łatwe wyszukiwanie, modyfikowanie i analizowanie. W aplikacjach najczęściej wykorzystywane są relacyjne lub nierelacyjne bazy danych}
}

\newglossaryentry{framework}
{
    name={Framework},
    description={Zestaw narzędzi, bibliotek i struktur wspomagających tworzenie aplikacji. Ułatwia programowanie poprzez dostarczenie gotowych komponentów oraz określenie zasad organizacji kodu}
}

\newglossaryentry{review-kodu}
{
    name={Review kodu},
    description={Proces polegający na wzajemnym przeglądzie kodu źródłowego przez programistów w celu wykrycia błędów, poprawy jakości oraz zwiększenia spójności projektu}
}

\newglossaryentry{jwt}
{
    name={JWT},
    description={Skrót od \textit{JSON Web Token}. Standard służący do bezpiecznego przekazywania informacji między stronami w formacie JSON, często używany w procesach autoryzacji użytkowników}
}

\newglossaryentry{oauth}
{
    name={OAuth},
    description={Standard autoryzacji umożliwiający aplikacjom zewnętrznym uzyskanie dostępu do zasobów użytkownika bez przekazywania jego hasła, często wykorzystywany przy logowaniu za pomocą dostawców takich jak Google czy GitHub}
}

\newglossaryentry{cicd}
{
    name={CI/CD},
    description={Skrót od \textit{Continuous Integration/Continuous Deployment}. Praktyka programistyczna polegająca na automatyzacji procesu budowania, testowania i wdrażania oprogramowania}
}

\newglossaryentry{github}
{
    name={GitHub},
    description={Platforma hostingu repozytoriów \textit{Git} w chmurze, oferująca m.in. pull requesty, system zgłoszeń (issues), zarządzanie wersjami oraz integrację z narzędziami CI/CD}
}

\newglossaryentry{jira}
{
    name={Jira},
    description={Narzędzie firmy Atlassian do zarządzania projektami i zadaniami, szeroko stosowane w metodykach zwinnych. Umożliwia pracę z epikami, taskami, podtaskami oraz tablicami Scrum i Kanban}
}

\newglossaryentry{spring-boot}
{
    name={Spring Boot},
    description={Framework w ekosystemie Spring dla języka Java, ułatwiający tworzenie aplikacji backendowych dzięki automatycznej konfiguracji, wbudowanemu serwerowi aplikacyjnemu oraz zestawowi gotowych starterów}
}

\newglossaryentry{spring-security}
{
    name={Spring Security},
    description={Moduł bezpieczeństwa w ekosystemie Spring odpowiedzialny za uwierzytelnianie i autoryzację użytkowników. Zapewnia obsługę różnych mechanizmów logowania, ról i uprawnień oraz integrację z różnymi źródłami danych}
}

\newglossaryentry{docker}
{
    name={Docker},
    description={Platforma do konteneryzacji aplikacji. Pozwala uruchamiać oprogramowanie w lekkich, izolowanych kontenerach tworzonych na podstawie obrazów, co upraszcza wdrażanie i utrzymanie spójnego środowiska}
}

\newglossaryentry{cors}
{
    name={CORS},
    description={Skrót od \textit{Cross-Origin Resource Sharing}. Mechanizm bezpieczeństwa w przeglądarkach, który kontroluje, czy aplikacja z jednej domeny może wykonywać zapytania HTTP do serwera w innej domenie; konfigurowany za pomocą nagłówków HTTP}
}

\newglossaryentry{http-only-cookie}
{
    name={Ciasteczko HttpOnly},
    description={Ciasteczko HTTP ustawione z flagą \texttt{HttpOnly}, dzięki czemu nie jest dostępne z poziomu JavaScriptu. Zmniejsza ryzyko kradzieży tokenów (np. JWT) w przypadku ataków typu XSS}
}

\newglossaryentry{tailwind-css}
{
    name={Tailwind CSS},
    description={Framework CSS typu \textit{utility-first}, dostarczający gotowe klasy narzędziowe do określania wyglądu (kolory, odstępy, layout). Umożliwia szybkie prototypowanie i spójne stylowanie komponentów bez pisania rozbudowanych arkuszy CSS}
}

\newglossaryentry{prettier}
{
    name={Prettier},
    description={Narzędzie do automatycznego formatowania kodu (np. JavaScript, TypeScript, CSS, HTML). Narzuca spójny styl formatowania, zastępując ręczne ustawianie wcięć i łamań linii}
}

\newglossaryentry{eslint}
{
    name={ESLint},
    description={Statyczny analizator kodu JavaScript/TypeScript. Umożliwia wykrywanie błędów, niespójności stylu oraz potencjalnych problemów poprzez zestaw reguł, które można dostosować do projektu}
}

\newglossaryentry{tanstack-query}
{
    name={TanStack Query},
    description={Biblioteka do obsługi zapytań do serwera i cachowania danych w aplikacjach frontendowych (m.in. React). Ułatwia zarządzanie stanem danych z backendu: pobieranie, odświeżanie, invalidację i obsługę błędów}
}

\newglossaryentry{leaflet}
{
    name={Leaflet},
    description={Lekka biblioteka JavaScript do tworzenia interaktywnych map w przeglądarce, często używana z danymi z OpenStreetMap. Umożliwia dodawanie znaczników, warstw oraz obsługę interakcji użytkownika}
}

\newglossaryentry{e2e-tests}
{
    name={Testy E2E},
    description={Testy \textit{end-to-end}, które sprawdzają działanie systemu od strony użytkownika, przechodząc przez wszystkie warstwy aplikacji (frontend, backend, baza danych) i symulując rzeczywiste scenariusze użycia}
}

\newglossaryentry{dto}
{
    name={DTO},
    description={Skrót od \textit{Data Transfer Object}. Prosty obiekt przenoszący dane między warstwami systemu lub między usługami. Zawiera pola danych, zazwyczaj bez logiki biznesowej}
}

\newglossaryentry{modal}
{
    name={Modal},
    description={Okno dialogowe (okno modalne), które pojawia się na wierzchu interfejsu i blokuje interakcję z resztą aplikacji, dopóki użytkownik go nie zamknie. Służy do prezentowania ważnych komunikatów lub formularzy}
}

\newglossaryentry{skeleton-loader}
{
    name={Skeleton loader},
    description={Wzorzec prezentowania stanu ładowania, w którym zamiast klasycznego „spinnera” wyświetlane są szare prostokąty imitujące docelowy układ treści. Poprawia subiektywne odczucie szybkości działania aplikacji}
}

\newglossaryentry{z-index}
{
    name={z-index},
    description={Właściwość CSS określająca kolejność nakładania się elementów (oś Z). Wyższa wartość powoduje wyświetlenie elementu „nad” elementami o niższych wartościach}
}

\newglossaryentry{intersection-observer}
{
    name={Intersection Observer},
    description={API przeglądarkowe umożliwiające reagowanie na momenty, gdy dany element pojawia się w polu widzenia użytkownika (viewport) lub opuszcza je. Wykorzystywane m.in. do implementacji \gls{infinite-scroll} i lazy loadingu}
}

\newglossaryentry{latex}
{
    name={LaTeX},
    description={System składu tekstu wykorzystywany do przygotowywania profesjonalnych dokumentów technicznych i naukowych. Umożliwia precyzyjne formatowanie, zarządzanie odwołaniami, bibliografią i wzorami matematycznymi}
}

\newglossaryentry{commit}
{
    name={Commit},
    @@ -60,7 +180,7 @@
\newglossaryentry{spot}
{
    name={Spot},
    description={Potencjalne miejsce do latania dronem, zaznaczone na mapie.}
}

\newglossaryentry{sidebar}
@@ -300,7 +420,197 @@
description={(ang. \textit{Business Process Model and Notation});
standardowa notacja graficzna, która umożliwia szczegółowe przedstawienie i dokumentowanie procesów biznesowych.}
}

\newglossaryentry{infinite-scroll}{
    name={Infinite scroll},
    description={Wzorzec interfejsu użytkownika, w którym kolejne porcje treści są automatycznie doładowywane podczas przewijania strony w dół, zamiast być podzielone na odrębne, ręcznie przełączane strony}
}
\newglossaryentry{cdn}
{
    name={CDN},
    description={Skrót od \textit{Content Delivery Network}. Rozproszona sieć serwerów
    służąca do szybkiego dostarczania statycznych zasobów (np. obrazów, arkuszy CSS,
    skryptów JavaScript) z węzłów geograficznie najbliższych użytkownikowi, co zmniejsza
    opóźnienia i odciąża serwer aplikacji}
}

\newglossaryentry{react-maplibre}
{
    name={React-MapLibre},
    description={Otwartoźródłowa biblioteka do renderowania interaktywnych map
    wektorowych w przeglądarce, rozwijana jako niezależna kontynuacja Mapbox GL JS.
    Umożliwia wyświetlanie kafelków mapowych, znaczników i warstw z danymi
    geoprzestrzennymi}
}

\newglossaryentry{websocket}
{
    name={WebSocket},
    description={Protokół komunikacyjny umożliwiający dwukierunkową komunikację
    w czasie rzeczywistym między przeglądarką a serwerem po pojedynczym,
    utrzymywanym połączeniu TCP. Często wykorzystywany m.in. w czatach i aplikacjach
    działających w czasie rzeczywistym.}
}

\newglossaryentry{docker-compose}
{
    name={Docker Compose},
    description={Narzędzie do definiowania i uruchamiania wielokontenerowych aplikacji \gls{docker}
    za pomocą pliku konfiguracyjnego (np. \texttt{docker-compose.yml}). Umożliwia jednoczesne
    uruchamianie powiązanych usług (np. \gls{backend}, baza danych, usługi pomocnicze) jednym poleceniem}
}

\newglossaryentry{pro}
{
    name={PRO},
    description={Przedmiot realizowany na 5. semestrze studiów, prowadzony przez dr. inż. Martę Łabudę. W ramach przedmiotu
    wybrano temat projektu oraz
    wytworzono wstępną dokumentację projektu w tym m.in. wymagania.}
}

\newglossaryentry{prz1}
{
    name={PRZ 1},
    description={Przedmiot realizowany na 6. semestrze studiów, prowadzony w przypadku zespołu projektowego przez mgr. inż. Adama Urbanowicza. W ramach przedmiotu
    wytworzono projekt interfejsu użytkownika.}
}

\newglossaryentry{prz2}
{
    name={PRZ 2},
    description={Przedmiot realizowany na 7. semestrze studiów, prowadzony w przypadku zespołu projektowego przez mgr. inż. Adama Urbanowicza. W ramach przedmiotu
    dokończono prace nad pracą inżynierską. Pan Adam Urbanowicz jako promotor doradzał zespołowi projektowemu.}
}

\newglossaryentry{psem}
{
    name={PSEM},
    description={Przedmiot realizowany na 7. semestrze studiów, prowadzony w przypadku zespołu projektowego przez dr. inż. Marka Bednarczyka. W ramach przedmiotu
    dokończono wytwarzanie dokumentacji.}
}

\newglossaryentry{spa}
{
    name={SPA},
    description={SPA (Single Page Application) to aplikacja webowa, w której cała strona ładuje się raz,
    a późniejsze zmiany widoku odbywają się dynamicznie po stronie przeglądarki bez pełnego przeładowania strony.}
}

\newglossaryentry{routing}
{
    name={routing},
    description={Routing w \gls{spa} to warstwa w kliencie odpowiedzialna za zarządzanie stanem “aktualnej strony” na podstawie URL-a,
    zwykle z wykorzystaniem historii przeglądarki,
    tak aby interfejs reagował na zmianę ścieżki bez przeładowań z serwera.}
}



\newglossaryentry{unit-tests}
{
    name={testy jednostkowe},
    description={Testy sprawdzające poprawność działania pojedynczych, małych fragmentów kodu (np. funkcji, metod, klas) w izolacji od reszty systemu.}
}

\newglossaryentry{jakarta-validation}
{
    name={jakarta validation},
    description={Jakarta Validation to specyfikacja (i zestaw adnotacji, typu @NotNull, @Size itd.) służąca do automatycznego sprawdzania poprawności danych w aplikacjach stworzonych za pomocą Java/Jakarta EE/Spring, np. przy walidacji pól DTO, encji czy parametrów metod.}
}
\newglossaryentry{intellij-idea}
{
    name={IntelliJ IDEA},
    description={Zintegrowane środowisko programistyczne (IDE) firmy JetBrains, szeroko stosowane przy tworzeniu aplikacji backendowych w ekosystemie Spring. Oferuje m.in. podpowiedzi składni, refaktoryzację kodu, debugger oraz integrację z systemami kontroli wersji}
}

\newglossaryentry{dockerfile}
{
    name={Dockerfile},
    description={Plik tekstowy zawierający instrukcje opisujące, jak zbudować obraz Dockera (jakiej podstawy użyć, jakie pliki skopiować, jakie polecenia uruchomić). Na jego podstawie narzędzie Docker tworzy gotowy obraz kontenera}
}

\newglossaryentry{redis}
{
    name={Redis},
    description={Szybka baza danych typu klucz–wartość przechowywana głównie w pamięci operacyjnej. Często wykorzystywana jako pamięć podręczna (cache), magazyn sesji lub prosty mechanizm komunikatów między usługami}
}

\newglossaryentry{gif}
{
    name={GIF},
    description={Format graficzny \textit{Graphics Interchange Format} obsługujący krótkie, zapętlone animacje. W aplikacjach czatowych wykorzystywany do wysyłania „reakcji” w postaci ruchomych obrazków}
}

\newglossaryentry{emoji}{
    name={emoji},
    description={Małe graficzne ikonki używane do wyrażania emocji
    lub pojęć w komunikacji cyfrowej (np. uśmiechnięta buźka, kciuk w górę,
    symbol serca).}
}


\newglossaryentry{url}
{
    name={URL},
    description={Adres zasobu w internecie (ang. \textit{Uniform Resource Locator}), np. adres strony, widoku w aplikacji webowej lub konkretnego posta na forum}
}

\newglossaryentry{slug}
{
    name={Slug},
    description={Przyjazny dla użytkownika fragment adresu URL, zwykle oparty na tytule (np. \texttt{/post/jak-zaczac-latac-dronem}), ułatwiający identyfikację treści i pozycjonowanie w wyszukiwarkach}
}

\newglossaryentry{tinymce}
{
    name={TinyMCE},
    description={Popularny edytor \textit{rich text} osadzany w przeglądarce. Pozwala użytkownikowi formatować tekst (pogrubienia, listy, nagłówki, linki) w sposób przypominający klasyczny edytor tekstu, zapisując wynik zwykle w HTML}
}

\newglossaryentry{rich-text-editor}
{
    name={Rich text editor},
    description={Edytor treści, który zamiast „surowego” tekstu umożliwia stosowanie formatowania (np. pogrubienie, kursywa, listy, nagłówki, linki), dzięki czemu użytkownik może tworzyć czytelne, sformatowane wpisy}
}

\newglossaryentry{tiptap}
{
    name={Tiptap},
    description={Nowoczesny, rozszerzalny edytor \textit{rich text} dla aplikacji webowych oparty na silniku ProseMirror. Umożliwia budowanie rozbudowanych, modularnych edytorów treści, np. do postów na forum}
}

\newglossaryentry{integration-tests}
{
    name={Testy integracyjne},
    description={Testy sprawdzające, czy połączone ze sobą moduły lub usługi współpracują poprawnie — na przykład czy warstwa backendowa poprawnie komunikuje się z bazą danych, warstwą sieciową i pozostałymi komponentami systemu}
}
\newglossaryentry{endpoint}
{
    name={endpoint},
    description={Endpoint to konkretny adres (np. \gls{url}) i metoda protokołu HTTP
    w \gls{api}, które razem odpowiadają za realizację jednej, dobrze zdefiniowanej
    operacji (np. pobrania listy spotów, dodania komentarza, wyszukania spotów).}
}

\newglossaryentry{redux-slice}
{
    name={slice Redux},
    description={Slice Redux to wydzielona część globalnego stanu w \gls{redux},
    wraz z powiązanymi akcjami i reduktorami, odpowiedzialna za jeden obszar domeny
    (np. konto użytkownika, czat, mapę czy listę znajomych).}
}
\newglossaryentry{jsoup}{
    name={jsoup},
    description={Biblioteka \textit{Java} do przetwarzania dokumentów HTML,
    umożliwiająca parsowanie, przeszukiwanie i modyfikowanie struktury dokumentu
    w sposób zbliżony do pracy z DOM-em i selektorami CSS.}
}
\newglossaryentry{paginacja}
{
    name={paginacja},
    description={Mechanizm dzielenia dużych zbiorów danych
    (np. list postów, wyników wyszukiwania, komentarzy)
    na mniejsze strony, które są pobierane i wyświetlane stopniowo,
    zamiast ładowania wszystkich elementów jednocześnie.}
}


\studfield{Informatyka}
\studtype{Stacjonarne}
\title{Aplikacja webowa: spoty-na-drony.pl}
\engtitle{Web application: spoty-na-drony.pl}
\acronym{Merkury}
\titledate{2023-10-10} %todo jaka data tutaj ma byc
\supervisor{mgr Adam Urbanowicz}
\reviewer{--- brak ---} %todo dodać

\author{Langmesser Adam}{s27119}{Aplikacje Internetowe}{Stacjonarny}
\author{Redosz Mateusz}{s27094}{Aplikacje Internetowe}{Stacjonarny}
\author{Oziemczuk Stanisław}{s26982}{Aplikacje Internetowe}{Stacjonarny}
\author{Badek Kacper}{s29168}{Aplikacje Internetowe}{Stacjonarny}

\consultant{--- brak ---} % Koniecznie trzeba podać brak, albo wpisać konsultantów tak jak przy autorach
\projectgoals{Stworzenie w pełni funkcjonalnej aplikacji internetowej do rozwijania hobby(latania dronem).} %todo do poprawy
\productsandservices{Aplikacja Internetowa, Dokumentacja}

\mainfunctionalities{
    Interaktywna mapa z wyśiwetlanymi spotami oraz pogodą.\newline
    Zaawansowana wyszukiwarka spotów.\newline
    Forum do dzielenia się informacjami na temat dronów.\newline
    Chat jednoosobowy oraz grupowy.\newline
    Konto użytkownika z możliwością zapisania ulubionych spotów.
}

\successmeasure{
    Gotowa do wdrożenia aplikacja.\newline
    Realizacja w terminie zgodnym z wymaganiami.
}

\projlimitations{
    Budżetowe: brak środków na wdrożenie.\newline
    Zawodowe: brak doświadczenia.\newline
    Czasowe: trzy semestry (09.2024 - 02.2026).\newline
    Ludzkie: czteroosobowy zespół.
}

\date{miesiąc, 2100 obrony} %todo jaka to ma być data
\finishdate{\today}

%todo poprawić podejście
\nabstract{
    Celem niniejszej pracy było stworzenie w pełni funkcjonalnej i działającej aplikacji internetowej pozwalającej na szybkie wyszukiwanie spotów w okolicy oraz dzielenie się zdjęciami, filmami oraz doświdczeniem z innymi użytkownikami.
    W ramach pracy stworzono system składający się z trzech komponentów: \gls{frontend}u, \gls{backend}u oraz bazy-danych.
    Aplikacja internetowa została wykonana przy pomocy \gls{framework}a React w językach Javascript oraz Typescript, do styli został użyty Tailwind.
    Serwis backendowy został stworzony w języku Java oraz biblioteki Spring Boot.
    Baza danych to PostgreSql.\newline
    Komunikacja między komponentami odbywała się zgodnie ze standardem REST.
    Projekt został zrealizowany w podejściu ewolucyjno-przyrostowym~z~elementami Kanban.
}

\keyword{--- brak ---}%todo nie wiem czy bedzie potrzebne ale zostawiam żeby było




\begin{document}

    \maketitle
    \makeprojectcard

    %Spis treści
    \tableofcontents
    \clearpage

    %! Author = mateusz
%! Date = 20/09/2025


\chapter{Wstęp}
\label{ch:wstep}

Rozdział ten ma na celu przedstawienie ogólnego kontekstu projektu.
Wyjaśnia, czym jest projekt, po co powstał, dla kogo jest przeznaczony oraz skąd wziął się pomysł jego stworzenia.

%! Author = mateusz
%! Date = 17/10/2025

\section{O projekcie}
\label{sec:o-projekcie}
%! Author = kacper
%! Date = 31/12/2025

\section{Cel i zakres prac}
\label{sec:cel-i-zakres-prac}

Głównym celem pracy jest opracowanie i implementacja aplikacji webowej dla społeczności pasjonatów dronów.
Zakres pracy obejmuje:

\begin{itemize}
    \item analizę wymagań aplikacji, obejmującą określenie potrzeb użytkowników oraz zdefiniowanie funkcjonalności systemu
    \item projektowanie struktury aplikacji, w tym struktury \glslink{baza-danych}{bazy danych} oraz interfejsu użytkownika
    \item implementację modułu mapy umożliwiającego przeglądanie, ocenianie oraz komentowanie \glslink{spot}{spotów} do latania dronem
    \item implementację modułu forum stanowiącego przestrzeń do wymiany wiedzy i doświadczeń
    \item implementację modułu czatu zapewniającego komunikację w czasie rzeczywistym pomiędzy użytkownikami
    \item implementację modułu profilu użytkownika umożliwiającego tworzenie i edycję profilu, dodawanie własnych \glslink{spot}{spotów}
    oraz przeglądanie historii aktywności
    \item testowanie i weryfikację działania aplikacji pod kątem poprawności, wydajności oraz bezpieczeństwa
\end{itemize}
%! Author = mateusz
%! Date = 17/10/2025

\section{Geneza pomysłu}
\label{sec:geneza-pomysłu}


    % Słownik pojęć i skrótów
%    \printglossary[type=\acronymtype]
    \printglossary[title={Słownik pojęć i skrótów},toctitle={Słownik pojęć i skrótów}]

    %! Author = mateusz
%! Date = 20/09/2025


\chapter{Opis problemu}
\label{ch:opis-problemu}

Niniejszy rozdział przedstawia kontekst projektowanego systemu.
Opisuje problemy i potrzeby użytkowników, identyfikuje interesariuszy oraz omawia istniejące rozwiązania.
Zawiera także wizję systemu oraz jego aspekty społeczno-biznesowe.

%! Author = kacper
%! Date = 31/12/2025

\section{Rich picture}
\label{sec:rich-picture}

Rich picture jest techniką służącą do przedstawienia kontekstu systemu, jego interesariuszy, relacji między nimi oraz głównych potrzeb i wyzwań.
Pozwala na lepsze zrozumienie problemu przed przejściem do formalnych modeli, takich jak diagramy \glslink{uml}{UML}.

Na zaprezentowanym rich picture (rys. \ref{img:rich-picture}) przedstawiono kontekst działania aplikacji „SpotyNaDrony” oraz jej interesariuszy.

Centralnym elementem diagramu jest platforma łącząca użytkowników.
Z aplikacji korzystają różne grupy, w tym:
\begin{itemize}
    \item \textbf{początkujący \glslink{droniarz}{droniarz}} – korzysta z aplikacji w celu uzyskania informacji i wsparcia na początku swojej przygody z dronami
    \item \textbf{\glslink{droniarz}{droniarz}} – bardziej doświadczony użytkownik, wyszukujący nowe miejsca do latania oraz nawiązujący kontakty z innymi pasjonatami
    \item \textbf{społeczność} – tworzy i wymienia się treściami, dzieli się doświadczeniami oraz buduje relacje wokół wspólnego zainteresowania
    \item \textbf{administrator} – odpowiedzialny za moderowanie treści publikowanych w aplikacji oraz utrzymanie porządku na platformie
\end{itemize}

Diagram przedstawia również główne interakcje zachodzące w systemie:
\begin{itemize}
    \item tworzenie nowych treści oraz wyszukiwanie miejsc do latania przez użytkowników
    \item wymiana doświadczeń w obrębie społeczności
    \item moderowanie treści przez administratora
    \item generowanie przychodów przez aplikację poprzez reklamy
\end{itemize}

Całość stanowi wysokopoziomowy opis funkcjonowania platformy oraz przepływu interakcji pomiędzy użytkownikami a systemem.

\begin{figure}[H]
    \centering
    \includegraphics[width=1\textwidth]{attachments/rich-picture/rich_picture}
    \caption{Diagram Rich Picture przedstawiający kontekst systemu}
    \label{img:rich-picture}
\end{figure}
%! Author = Adam
%! Date = 01/11/2025

\section{Udziałowcy}
\label{sec:udzialowcy}

%zespół projektowy
%droniarze  TODO:daj do slowniczka
%promotor

\begin{table}[htbp]
    \centering
    \renewcommand{\arraystretch}{1.15} % opcjonalnie: wyższe wiersze
    \begin{tabularx}{\textwidth}{|L{0.33\textwidth}|X|}
        \hline
        \textbf{Nazwa udziałowca} & \textbf{Opis} \\ \hline
        Zespół projektowy & Odpowiedzialny za zaprojektowanie, implementację i testowanie systemu. \\ \hline
        Promotor & Weryfikuje i nadzoruje pracę zespołu projektowego. \\ \hline
        Droniarze & Główni użytkownicy systemu. \\ \hline
    \end{tabularx}
    \caption{Przedstawienie poszczególnych udziałowców wraz z ich znaczeniem}\label{tab:udzialowcy}
\end{table}

%! Author = mateusz
%! Date = 17/10/2025

\section{Istniejące rozwiązania}
\label{sec:istniejace-rozwiazania}
%! Author = kacper
%! Date = 31/12/2025

\section{Wizja rozwiązania}
\label{sec:wizja-rozwiązania}

Celem aplikacji jest wspieranie społeczności pasjonatów dronów poprzez usprawnienie dostępu do informacji
oraz skupienie użytkowników na jednej platformie.
Rozwiązanie ma umożliwiać szybki dostęp do sprawdzonych lokalizacji oraz informacji,
których dotychczasowe pozyskiwanie jest niewydajne i czasochłonne.

Projekt zakłada stworzenie platformy internetowej integrującej funkcjonalności istotne
z punktu widzenia operatorów dronów w jednym środowisku.
Aplikacja ma pełnić rolę narzędzia wspomagającego planowanie aktywności lotniczych
oraz budowanie społeczności poprzez umożliwienie użytkownikom wzajemnego dzielenia się doświadczeniami.

Przyjęto, że użytkownikiem końcowym aplikacji będzie zarówno doświadczony, jak i początkujący operator drona,
nieposiadający specjalistycznej wiedzy.
Z tego powodu interfejs aplikacji zostanie uproszczony.
Rozwiązanie ma charakter wspierający i nie zastępuje oficjalnych narzędzi do planowania lotów
ani systemów regulacyjnych.
Aplikacja stanowi uzupełnienie, oferując społecznościową warstwę wymiany informacji.


%! Author = kacper
%! Date = 31/12/2025

\section{Aspekty społeczne i biznesowe}
\label{sec:aspekty-spoleczne-i-biznesowe}

\newcounter{card}[chapter]
\renewcommand{\thecard}{\thechapter.\arabic{card}}

Poniżej przedstawiono najważniejsze aspekty społeczne oraz biznesowe projektu.

%! Author = mateusz
%! Date = 26/10/2025

\subsection{Aspekty społeczne}
\label{subsec:aspekty-spoleczne}
%! Author = kacper
%! Date = 31/12/2025

\subsection{Aspekty biznesowe}
\label{subsec:aspekty-biznesowe}

\begin{longtable}{|
        >{\columncolor{lightgray}}p{0.3\textwidth}|
    p{0.65\textwidth}|}
    \hline
    \rowcolor{lightgray}
    \multicolumn{2}{|c|}{\textbf{ASPEKTY BIZNESOWE PROJEKTU}} \\ \hline
    \endfirsthead

    \hline
    \rowcolor{lightgray}
    \multicolumn{2}{|c|}{\textbf{ASPEKTY BIZNESOWE PROJEKTU (cd.)}} \\ \hline
    \endhead

    \textbf{Model przychodowy} &
    Podstawowym źródłem przychodu aplikacji będą reklamy wyświetlane użytkownikom platformy. \\ \hline

    \textbf{Potencjał rynkowy} &
    Rosnąca popularność technologii dronowych zwiększa zapotrzebowanie
    na dedykowaną platformę społecznościową. \\ \hline

    \textbf{Możliwość monetyzacji produktu} &
    Aplikacja może zostać rozszerzona o funkcje premium
    dostępne dla użytkowników za opłatą. \\ \hline

    \textbf{Współpraca z branżą} &
    Istnieje możliwość nawiązania współpracy z producentami dronów
    oraz ich akcesoriów. \\ \hline

\end{longtable}

    %! Author = mateusz
%! Date = 20/09/2025


\chapter{Planowanie}
\label{ch:planowanie}

\newcounter{integrationcard}[chapter]
\renewcommand{\theintegrationcard}{\thechapter.\arabic{integrationcard}}

W niniejszym rozdziale przedstawiono sposób zaplanowania realizacji projektu oraz założenia organizacyjne.
W pierwszej kolejności opisano proces wyboru metodyki pracy oraz uzasadniono przyjęte podejście spośród rozważanych alternatyw.
Następnie zaprezentowano harmonogram projektu, obejmujący podział działań w czasie oraz kluczowe etapy realizacji.
Kolejna część rozdziału przedstawia technologie i narzędzia wykorzystywane do komunikacji w zespole oraz zarządzania przepływem pracy.
Rozdział uzupełniają informacje o dostępnych zasobach i ograniczeniach projektowych, a także analiza ryzyka wraz z omówieniem potencjalnych zagrożeń i sposobów ich ograniczania.

%! Author = mateusz
%! Date = 17/10/2025

\section{Metodologia pracy}
\label{sec:metodologia-pracy}

\subsection{Przegląd rozważanych podejść}
\label{subsec:przeglad-rozwazanych-podejsc}

Przy wyborze metodologii pracy rozważono trzy podejścia do prowadzenia projektu informatycznego:
\begin{itemize}
    \item klasyczny Agile (w praktyce: Scrum),
    \item model kaskadowy (waterfall),
    \item Disciplined Agile Delivery (DAD).
\end{itemize}

\subsection{Odrzucone podejścia}
\label{subsec:odrzucone-podejscia}

\paragraph{„Klasyczny Agile” (Scrum).}
Mimo elastyczności i popularności zakłada pracę w iteracjach 2--4 tygodni oraz stały zestaw ceremonii (planowanie, przegląd, retrospektywa).
Ze względu na nierównomierną dostępność zasobów w kolejnych miesiącach studiów nie zapewniono możliwości utrzymania stałej kadencji sprintów, dlatego z podejścia zrezygnowano.

\paragraph{Model kaskadowy (waterfall).}
Przewiduje sekwencyjne przechodzenie przez z góry określone etapy i ogranicza bieżącą weryfikację wymagań w trakcie prac deweloperskich.
W projekcie wymagano możliwości częstych rewizji założeń oraz wprowadzania istotnych zmian w docelowej wizji rozwiązania; dlatego z podejścia zrezygnowano.

\subsection{Wybrane podejście: Disciplined Agile Delivery (Lean Life Cycle)}
\label{subsec:wybrane-podejscie-dad-lean}

Podjęto decyzję o zastosowaniu \textbf{Disciplined Agile Delivery} w wariancie \textbf{Lean Life Cycle}, ponieważ podejście to łączy pożądane cechy Agile i waterfall, a jednocześnie eliminuje stałe sprinty na rzecz pracy w ciągłym przepływie.

\paragraph{Kluczowe argumenty wyboru:}
\begin{itemize}
    \item \textbf{Brak sprintów.} Zastosowano przepływ ciągły (flow) z ograniczeniami WIP, co pozwala dopasować tempo do zmiennej dostępności zespołu i unikać sztucznego „domykania” iteracji.
    \item \textbf{Rozbudowana faza startowa (Inception).} Na początku przewidziano większy wysiłek planistyczny: doprecyzowanie zakresu, wstępna wizja architektury, identyfikacja ryzyk, plan publikacji oraz kryteria jakości -- bez zamrażania szczegółów.
    \item \textbf{Ciągła weryfikacja wymagań.} W trakcie realizacji przewidziano bieżące doprecyzowywanie backlogu, regularny feedback promotora i interesariuszy oraz możliwość korygowania kierunku bez kosztów „przeskakiwania” między fazami.
    \item \textbf{Praktyki Lean i koncentracja na wartości.} Priorytetyzacja wartości biznesowej, wizualizacja pracy, małe partie dostaw, pomiar lead/cycle time oraz systematyczna optymalizacja przepływu.
    \item \textbf{Lekka governance i kamienie milowe.} Zastosowano lekkie mechanizmy nadzoru (np. peer review, przeglądy architektury, prezentacje postępów) zapewniające przejrzystość bez nadmiernej biurokracji.
    \item \textbf{Architektura ewolucyjna.} Wykorzystano podejście \emph{just-enough design} (spikes, prototypy) w celu wczesnego ograniczania ryzyk technologicznych i stopniowego uszczegóławiania rozwiązania.
    \item \textbf{Jakość wbudowana w proces.} Utrzymano ciągłą integrację i testowanie, jasną definicję „ukończenia” oraz standardy techniczne, co umożliwia bezpieczne i częste publikacje.
\end{itemize}

\subsection{Narzędzia i komunikacja}
\label{subsec:narzedzia-komunikacja}

Do zarządzania zadaniami zastosowana zostanie \textbf{Jira} (monitorowanie postępu prac oraz ewidencja zadań członków zespołu). Komunikację w zespole zaplanowano w formie regularnych spotkań oraz asynchronicznie z wykorzystaniem \textbf{Discorda}.

\subsection{Podział ról w zespole}
\label{subsec:podzial-rol}

\begin{itemize}
    \item \textbf{Adam} -- lider zespołu; kieruje przepływem pracy i nadaje priorytety; full-stack developer.
    \item \textbf{Stanisław} -- full-stack developer.
    \item \textbf{Kacper} -- full-stack developer.
    \item \textbf{Mateusz} -- full-stack developer.
\end{itemize}

Każdy z członków zespołu uczestniczy również w przygotowaniu dokumentacji.

\section{Harmonogram projektu}
\label{sec:harmonogram-projektu2}

W poniższym harmonogramie przedstawiono plan prac nad poszczególnymi częściami projektu, rozłożony na miesiące.

\begin{figure}[!htbp]
    \centering
    \fbox{\parbox{0.95\linewidth}{\centering \vspace{2.5cm}
    \textit{Miejsce na ilustrację harmonogramu (np. wykres Gantta).}
    \vspace{2.5cm}}}
    \caption{Wizualizacja harmonogramu (do uzupełnienia).}
    \label{fig:harmonogram-gantt}
\end{figure}

\subsection*{Rok 2024}
\begin{description}
    \item[Czerwiec] Zebranie zespołu; rozważenie potencjalnych pomysłów.
    \item[Lipiec] Wybór technologii; wstępne założenia architektoniczne.
    \item[Sierpień] Realizacja dokumentacji.
    \item[Wrzesień] \emph{(do uzupełnienia)}
    \item[Październik] \emph{(do uzupełnienia)}
    \item[Listopad] \emph{(do uzupełnienia)}
    \item[Grudzień] \emph{(do uzupełnienia)}
\end{description}

\subsection*{Rok 2025}
\begin{description}
    \item[Styczeń] \emph{(do uzupełnienia)}
    \item[Luty] \emph{(do uzupełnienia)}
    \item[Marzec] \emph{(do uzupełnienia)}
    \item[Kwiecień] \emph{(do uzupełnienia)}
    \item[Maj] \emph{(do uzupełnienia)}
    \item[Czerwiec] \emph{(do uzupełnienia)}
    \item[Lipiec] \emph{(do uzupełnienia)}
    \item[Sierpień] \emph{(do uzupełnienia)}
    \item[Wrzesień] \emph{(do uzupełnienia)}
    \item[Październik] \emph{(do uzupełnienia)}
    \item[Listopad] \emph{(do uzupełnienia)}
    \item[Grudzień] \emph{(do uzupełnienia)}
\end{description}

\subsection*{Rok 2026}
\begin{description}
    \item[Styczeń] \emph{(do uzupełnienia)}
\end{description}

%! Author = Adam
%! Date = 30/10/2025


\section{Harmonogram projektu}
\label{sec:harmonogram-projektu}

W poniższym harmonogramie przedstawiono plan prac nad poszczególnymi częściami projektu, rozłożony na miesiące.

\subsection*{Rok 2024}
\begin{description}
    \item[Czerwiec]
    \begin{itemize}
        \item Zebranie zespołu.
        \item Rozważenie potencjalnych pomysłów.
    \end{itemize}

    \item[Lipiec]
    \begin{itemize}
        \item Wybór technologii.
        \item Wstępne założenia architektoniczne.
    \end{itemize}

    \item[Sierpień]
    \begin{itemize}
        \item \emph{(do uzupełnienia)}
        \item \emph{(do uzupełnienia)}
    \end{itemize}

    \item[Wrzesień]
    \begin{itemize}
        \item \emph{(do uzupełnienia)}
        \item \emph{(do uzupełnienia)}
    \end{itemize}

    \item[Październik]
    \begin{itemize}
        \item \emph{(do uzupełnienia)}
        \item \emph{(do uzupełnienia)}
    \end{itemize}

    \item[Listopad]
    \begin{itemize}
        \item \emph{(do uzupełnienia)}
        \item \emph{(do uzupełnienia)}
    \end{itemize}

    \item[Grudzień]
    \begin{itemize}
        \item \emph{(do uzupełnienia)}
        \item \emph{(do uzupełnienia)}
    \end{itemize}
\end{description}

\subsection*{Rok 2025}
\begin{description}
    \item[Styczeń]
    \begin{itemize}
        \item \emph{(do uzupełnienia)}
        \item \emph{(do uzupełnienia)}
    \end{itemize}

    \item[Luty]
    \begin{itemize}
        \item \emph{(do uzupełnienia)}
        \item \emph{(do uzupełnienia)}
    \end{itemize}

    \item[Marzec]
    \begin{itemize}
        \item \emph{(do uzupełnienia)}
        \item \emph{(do uzupełnienia)}
    \end{itemize}

    \item[Kwiecień]
    \begin{itemize}
        \item \emph{(do uzupełnienia)}
        \item \emph{(do uzupełnienia)}
    \end{itemize}

    \item[Maj]
    \begin{itemize}
        \item \emph{(do uzupełnienia)}
        \item \emph{(do uzupełnienia)}
    \end{itemize}

    \item[Czerwiec]
    \begin{itemize}
        \item \emph{(do uzupełnienia)}
        \item \emph{(do uzupełnienia)}
    \end{itemize}

    \item[Lipiec]
    \begin{itemize}
        \item \emph{(do uzupełnienia)}
        \item \emph{(do uzupełnienia)}
    \end{itemize}

    \item[Sierpień]
    \begin{itemize}
        \item \emph{(do uzupełnienia)}
        \item \emph{(do uzupełnienia)}
    \end{itemize}

    \item[Wrzesień]
    \begin{itemize}
        \item \emph{(do uzupełnienia)}
        \item \emph{(do uzupełnienia)}
    \end{itemize}

    \item[Październik]
    \begin{itemize}
        \item \emph{(do uzupełnienia)}
        \item \emph{(do uzupełnienia)}
    \end{itemize}

    \item[Listopad]
    \begin{itemize}
        \item \emph{(do uzupełnienia)}
        \item \emph{(do uzupełnienia)}
    \end{itemize}

    \item[Grudzień]
    \begin{itemize}
        \item \emph{(do uzupełnienia)}
        \item \emph{(do uzupełnienia)}
    \end{itemize}
\end{description}

\subsection*{Rok 2026}
\begin{description}
    \item[Styczeń]
    \begin{itemize}
        \item \emph{(do uzupełnienia)}
        \item \emph{(do uzupełnienia)}
    \end{itemize}
\end{description}

%! Author = mateusz
%! Date = 17/10/2025

\section{Technologie i narzędzia}
\label{sec:technologie-i-narzedzia}

%! Author = Stanisław Oziemczuk
%! Date = 02/12/2025

\subsection{Technologie}
\label{subsec:technologie}

Do realizacji projektu zespół wspólnie wytypował główne technologie części \glslink{backend}{backendowej}, \glslink{frontend}{frontendowej} oraz dokumentacji.
Natomiast poszczególne biblioteki i rozwiązania były wybierane indywidualnie lub po konsultacjach przez osobę wykonującą dane zadanie.
Poniżej przedstawiono stos technologiczny zastosowany w projekcie.

%! Author = Stanisław Oziemczuk
%! Date = 10/11/2025

\subsection{Narzędzia}
\label{subsec:narzedzia}

Do niektórych płatnych narzędzi otrzymano bezpłatny dostęp za pośrednictwem uczelni, w innych istniała możliwość założenia konta
edukacyjnego, które oferowało dostęp do wszystkich funkcji narzędzia.
Gdy żadna z wymienionych opcji nie była udostępniona, wybierano rozwiązania darmowe.

\begin{itemize}
    \item \textbf{IntelliJ IDEA Ultimate}

    Jest to \gls{ide} od firmy JetBrains.
    Dzięki licznie dostępnym pluginom oferuje obsługę wielu języków programowania oraz innych składni.
    Pozwala również na integrację z repozytorium.
    Używano go do programowania zarówno \glslink{frontend}{frontendu}, jak i \glslink{backend}{backendu} oraz tworzenia dokumentacji w LaTeX.
    \item \textbf{Docker Desktop}

    To narzędzie do zarządzania obrazami, kontenerami oraz wolumenami Docker.
    Zawiera w sobie również silnik tej technologii.
    Wykorzystywano je do lokalnego uruchamiania bazy danych oraz serwisu do cachowania.
    \item \textbf{Docker Compose}

    Narzędzie, które pozwala definiować oraz uruchamiać aplikacje składające się z wielu kontenerów Docker.
    Konfiguracja serwisów, sieci i \glslink{wolumen}{wolumenów} jest ustawiana w pliku (lub plikach) YAML.
    Zastosowano je do skonfigurowania bazy danych i serwisu do \glslink{cache}{cache'owania} w środowisku deweloperskim.
    \item \textbf{One Drive}

    Usługa dysku chmurowego oferowana przez firmę Microsoft.
    Przechowywano tam dokumenty oraz obrazy diagramów.
    \item \textbf{Azure Blob Storage}

    To rozwiązanie chmurowe Microsoft, służące do bezpiecznego przechowywania dużej ilości danych
    nieustrukturyzowanych, takich jak pliki multimedialne, dokumenty czy kopie zapasowe.
    Dane są dostępne poprzez interfejs \gls{rest_api} usługi Azure Storage.
    Wykorzystywano je do przechowywania zdjęć profilowych użytkownika oraz multimedii (zdjęcia i filmy) ze \glslink{spot}{spotów}
    i forum.
    \item \textbf{Jira}

    To narzędzie firmy Atlassian do zarządzania pracami nad projektem w metodykach zwinnych.
    Do \glslink{backlog}{Backlogu} wpisywano zadania, a na \glslink{tablica_kanban}{tablicy Kanbanowej} rejestrowano ich statusy oraz poświęcony czas.
    \item \textbf{GitHub}

    Zdalne repozytorium służące do przechowywania i wersjonowania kodu aplikacji.
    Zamieszczono tam kod naszego projektu.
    Do każdego zadania tworzono osobną gałąź z właściwą nazwą, a po zakończeniu prac przeprowadzano \glslink{review-kodu}{review kodu}.
    Następnie łączono ją do głównej gałęzi deweloperskiej.
    \item \textbf{GitHub Actions}

    To narzędzie do implementacji procesów \gls{cicd} na platformie GitHub, które
    umożliwiają automatyczne testowanie lub wdrażanie kodu.
    Uruchamiają się w reakcji na różne operacje w repozytorium, na przykład przesłanie zmian na wybraną gałąź.
    Stosowano je do automatycznego testowania i budowania projektu po każdorazowym wprowadzeniu zmian.
    \item \textbf{GitHub Copilot}

    To narzędzie sztucznej inteligencji będące asystentem programisty.
    W projekcie analizuje plik oraz pliki powiązane.
    Wykorzystywano go podczas \glslink{review-kodu}{review kodu}.
    Copilot skanuje wszystkie pliki i w komentarzach opisuje sugerowane zmiany lub potencjalne błędy.
    \item \textbf{Discord}

    Darmowa platforma komunikacyjna.
    Umożliwia udostępnienie obrazu z ekranu, komunikację głosową oraz tekstową, jak i również przesyłanie plików.
    Stosowano go do spotkań, na których omawiano sprawy dotyczące projektu.
    \item \textbf{Messenger}

    Komunikator będący usługą Facebooka.
    Daje możliwość tworzenia czatów grupowych lub prywatnych, a także udostępniania plików.
    Używano go do ustalania spotkań na Discordzie oraz szybkiej komunikacji.
    \item \textbf{Postman}

    To narzędzie służące do testowania endpointów \gls{api}.
    Pozwala grupować zapytania w kolekcje, wysyłać ich różne typy oraz analizować odpowiedzi z serwera.
    Wykorzystywano go do testowania stworzonych endpointów oraz debugowania.
    \item \textbf{Figma}

    Narzędzie chmurowe do projektowania interfejsów użytkownika (\gls{ui}).
    Umożliwia zespołowe tworzenie w pełni interaktywnych prototypów.
    Wykonano w nim projekty ekranów naszej aplikacji.
    \item \textbf{Visual Paradigm}

    To narzędzie do tworzenia różnych diagramów stosowanych w inżynierii oprogramowania, takich jak \gls{uml}(~\cite{uml-def}) czy \gls{bpmn}(~\cite{bpmn-def}).
    Zrobiono w nim diagram przypadków użycia.
    \item \textbf{Xmind}

    Narzędzie służące do tworzenia mapy myśli.
    Wykorzystano je w celu lepszego zrozumienia problemów poprzez przeniesienie ich na diagram.
\end{itemize}


%! Author = mateusz
%! Date = 17/10/2025

\section{Zasoby i ograniczenia}
\label{sec:zasoby-i-ograniczenia}
%! Author = mateusz
%! Date = 17/10/2025

\section{Analiza ryzyka}
\label{sec:analiza-ryzyka}

    %! Author = mateusz
%! Date = 20/09/2025


\chapter{Analiza wymagań}
\label{ch:analiza-wymagan}

%! Author = mateusz
%! Date = 17/10/2025

\section{Przypadki użycia}
\label{sec:przypadki-uzycia}

%! Author = Adam
%! Date = 01/11/2025

\subsection{Aktorzy}
\label{subsec:aktorzy}

\begin{description}
    \item[\textbf{Użytkownik niezalogowany}] Gość przeglądający publiczne treści (mapa, spoty, forum); może się zarejestrować/zalogować.
    \item[\textbf{Użytkownik (nie premium)}] Zarejestrowany użytkownik podstawowy: zarządza kontem i ulubionymi, dodaje treści/komentarze, korzysta z czatu.
    \item[\textbf{Użytkownik premium}] Użytkownik z wykupioną subskrypcją; ma dostęp do funkcji premium (np. widok premium, rozszerzone wyszukiwanie).
    \item[\textbf{Moderator}] Nadzór treści: przegląda zgłoszenia, akceptuje/usuwa posty i komentarze, blokuje nadużycia.
    \item[\textbf{Deweloper}] Osoba utrzymująca system: inicjuje pipeline’y, monitoruje wdrożenia i infrastrukturę.
\end{description}

\paragraph{Aktorzy będący zewnętrznymi usługami}
Opisy znajdują się w rozdziale~\ref{subsec:uslugi-zewnetrzne}.
Poniżej jedynie lista nazw:
\begin{itemize}
    \item Usługa mailowa (Mailtrap)
    \item Dostawca API do map (OpenFreeMap)
    \item Nawigacja (Google Maps)
    \item Dostawca API pogodowego (Open-Meteo)
    \item Dostawca API GIFów (Tenor)
    \item Dostawca API do określania strefy czasowej spota (``Where the ISS at?'')
    \item Dostawca API do geolokalizacji (LocationQ)
    \item Azure Blob Storage
    \item Dostawca OAuth (Google)
    \item Dostawca OAuth (GitHub)
\end{itemize}

%! Author = mateusz
%! Date = 18/10/2025

\subsection{Diagram przypadków użycia}
\label{subsec:diagram-przypadkow-uzycia}
%%! Author = Adam
%%! Date = 22/11/2025

\subsection{Scenariusze przypadków użycia}
\label{subsec:scenariusz-przypadkow-uzycia}

Niniejszy rozdział zawiera scenariusze przypadków użycia.
Zostały one wykonane dla wybranych przypadków użycia.
\\
Każdy ze scenariuszy posiada jeden z trzech możliwych priorytetów implementacji: wysoki, średni albo niski.

% --- Licznik dla scenariuszy przypadków użycia ---
\newcounter{usecase}[chapter]
\renewcommand{\theusecase}{\thechapter.\arabic{usecase}}

\newcommand{\ucpriority}[1]{\textbf{Priorytet:} & #1 \\ \hline}
\newcommand{\ucactors}[1]{\textbf{Aktorzy:} & #1 \\ \hline}
\newcommand{\ucdesc}[1]{\textbf{Opis:} & #1 \\ \hline}
\newcommand{\ucpre}[1]{\textbf{Warunki\newline wstępne:} & #1 \\ \hline}
\newcommand{\ucpost}[1]{\textbf{Warunki\newline końcowe:} & #1 \\ \hline}
\newcommand{\ucmain}[1]{\textbf{Główny\newline przepływ\newline zdarzeń:} & #1 \\ \hline}
\newcommand{\ucalt}[1]{\textbf{Alternatywne\newline przepływy\newline zdarzeń:} & #1 \\ \hline}

% --- Karta scenariusza PU ---
\newcommand{\usecasecard}[4]{%
    \refstepcounter{usecase}%
    \begin{center}
    \renewcommand{\arraystretch}{1.15}%
    \begin{tabularx}{\textwidth}{|>{\columncolor{lightgray}}p{0.20\textwidth}|X|}
    \rowcolor{lightgray}
    \multicolumn{2}{|c|}{\textbf{KARTA SCENARIUSZA PRZYPADKU UŻYCIA}} \\ \hline
    \textbf{Identyfikator:} & #3 \\ \hline
    \textbf{Nazwa:}         & #2 \\ \hline
    #4
    \end{tabularx}
    \vspace{3pt}
    \textbf{Tabela \theusecase:} Scenariusz przypadku użycia: #2\label{#1}
    \end{center}%
    \addcontentsline{lot}{table}{Tabela \theusecase: Scenariusz przypadku użycia: #2}%
}

% Scenariusze PU:

\usecasecard{tab:pu1-rejestracja}{Rejestracja użytkownika}{PU1}{%
    \ucpriority{Wysoki}
    \ucactors{Użytkownik niezalogowany}
    \ucdesc{Użytkownik zakłada konto poprzez formularz rejestracji.}
    \ucpre{Użytkownik znajduje się na stronie z formularzem rejestracji.}
    \ucpost{Użytkownik posiada konto w systemie.}
    \ucmain{%
        \begin{enumerate}[nosep,leftmargin=16pt,labelindent=0pt]
            \item Użytkownik wypełnia formularz rejestracyjny.
            \item Użytkownik naciska przycisk rejestracji.
            \item System tworzy konto użytkownika.
            \item System loguje użytkownika i przenosi go na stronę główną aplikacji.
        \end{enumerate}
    }
    \ucalt{%
        \begin{enumerate}[nosep,leftmargin=21pt,labelindent=0pt,label={}]
            \item[1a.] Podane dane są niepoprawne – system wyświetla
            komunikat o błędzie oraz podświetla pola wymagające poprawy.
            \item[2a.] Nazwa użytkownika jest już zajęta – system wyświetla
            komunikat o błędzie.
        \end{enumerate}
    }
}

\usecasecard{tab:pu2-logowanie}{Logowanie użytkownika}{PU2}{%
    \ucpriority{Wysoki}
    \ucactors{Użytkownik niezalogowany}
    \ucdesc{Użytkownik loguje się do systemu, podając login i hasło.}
    \ucpre{Użytkownik znajduje się na stronie logowania.}
    \ucpost{Użytkownik jest zalogowany i przeniesiony na stronę główną aplikacji.}
    \ucmain{%
        \begin{enumerate}[nosep,leftmargin=16pt,labelindent=0pt]
            \item Użytkownik wypełnia formularz logowania.
            \item Użytkownik naciska przycisk logowania.
            \item System loguje użytkownika i przenosi go na ostatnią odwiedzoną przez niego stronę aplikacji.
        \end{enumerate}
    }
    \ucalt{%
        \begin{enumerate}[nosep,leftmargin=21pt,labelindent=0pt,label={}]
            \item[2a.] Podane dane są niepoprawne – system wyświetla komunikat o błędzie.
        \end{enumerate}
    }
}

\usecasecard{tab:pu3-reset-hasla}{Resetowanie hasła}{PU3}{%
    \ucpriority{Wysoki}
    \ucactors{Użytkownik niezalogowany, Usługa SMTP}
    \ucdesc{Użytkownik inicjuje reset hasła, aby odzyskać dostęp do konta.}
    \ucpre{Użytkownik znajduje się na ekranie resetu hasła.}
    \ucpost{Użytkownik otrzymuje wiadomość e-mail z linkiem do ustawienia nowego hasła.}
    \ucmain{%
        \begin{enumerate}[nosep,leftmargin=16pt,labelindent=0pt]
            \item Użytkownik wpisuje adres e-mail powiązany z kontem.
            \item Użytkownik zatwierdza żądanie resetu hasła.
            \item System generuje token resetu hasła.
            \item System wysyła e-mail z linkiem do zmiany hasła.
        \end{enumerate}
    }
    \ucalt{%
        \begin{enumerate}[nosep,leftmargin=21pt,labelindent=0pt,label={}]
            \item[2a.] Nie istnieje konto dla podanego adresu – system wyświetla komunikat o błędzie.
            \item[4a.] Występuje błąd połączenia z usługą SMTP – system informuje użytkownika o problemie technicznym.
        \end{enumerate}
    }
}

%TODO sprwawdz porpawnosc
\usecasecard{tab:pu4-zmiana-hasla}{Zmiana hasła w ustawieniach konta}{PU4}{%
    \ucpriority{Wysoki}
    \ucactors{Użytkownik zalogowany}
    \ucdesc{Użytkownik zmienia hasło do konta z poziomu ustawień profilu.}
    \ucpre{Użytkownik jest zalogowany i znajduje się na ekranie zmiany danych konta.}
    \ucpost{Hasło do konta użytkownika zostało zaktualizowane.}
    \ucmain{%
        \begin{enumerate}[nosep,leftmargin=16pt,labelindent=0pt]
            \item Użytkownik wpisuje aktualne hasło.
            \item Użytkownik wpisuje nowe hasło i powtarza je.
            \item Użytkownik zatwierdza formularz zmiany hasła.
            \item System zapisuje nowe hasło i informuje o powodzeniu operacji.
        \end{enumerate}
    }
    \ucalt{%
        \begin{enumerate}[nosep,leftmargin=21pt,labelindent=0pt,label={}]
            \item[3a.] Aktualne hasło jest nieprawidłowe – system wyświetla komunikat i nie zapisuje zmian.
            \item[3b.] Nowe hasło nie spełnia wymagań bezpieczeństwa – system informuje o błędzie i podświetla pola do poprawy.
        \end{enumerate}
    }
}

\usecasecard{tab:pu5-wylogowanie}{Wylogowanie użytkownika}{PU5}{%
    \ucpriority{Wysoki}
    \ucactors{Użytkownik zalogowany}
    \ucdesc{Użytkownik wylogowuje się z aplikacji.}
    \ucpre{Użytkownik jest zalogowany.}
    \ucpost{Sesja użytkownika została zakończona, użytkownik widzi stronę główną dla niezalogowanych.}
    \ucmain{%
        \begin{enumerate}[nosep,leftmargin=16pt,labelindent=0pt]
            \item Użytkownik wybiera opcję wylogowania z menu.
            \item System unieważnia token dostępu użytkownika.
            \item System przenosi użytkownika na stronę główną aplikacji.
        \end{enumerate}
    }
    \ucalt{Brak istotnych alternatywnych przepływów.}
}

\usecasecard{tab:pu6-powiadomienia}{Przeglądanie powiadomień}{PU6}{%
    \ucpriority{Niski}
    \ucactors{Użytkownik zalogowany}
    \ucdesc{Użytkownik przegląda listę powiadomień.}
    \ucpre{Użytkownik jest na ekranie centra powiadomień.}
    \ucpost{Powiadomienia zostały wyświetlone, a wybrane oznaczone jako przeczytane.}
    \ucmain{%
        \begin{enumerate}[nosep,leftmargin=16pt,labelindent=0pt]
            \item System wyświetla powiadomienia w odwróconym porządku chronologicznym.
            \item Użytkownik otwiera wybrane powiadomienie.
            \item System oznacza powiadomienie jako przeczytane i ewentualnie przenosi użytkownika do powiązanego widoku.
        \end{enumerate}
    }
    \ucalt{%
        \begin{enumerate}[nosep,leftmargin=21pt,labelindent=0pt,label={}]
            \item[1a.] System nie może pobrać powiadomień (błąd serwera) – użytkownik otrzymuje komunikat o błędzie i może spróbować ponownie.
        \end{enumerate}
    }
}

\usecasecard{tab:pu7-subskrypcja}{Wykupienie subskrypcji premium}{PU7}{%
    \ucpriority{Niski}
    \ucactors{Użytkownik zalogowany, Bramka płatnicza, System finansowo-księgowy}
    \ucdesc{Użytkownik opłaca subskrypcję premium w celu uzyskania dodatkowych funkcji.}
    \ucpre{Użytkownik jest zalogowany i znajduje się w module subskrypcji.}
    \ucpost{Subskrypcja premium jest aktywna, a użytkownik ma dostęp do funkcji premium.}
    \ucmain{%
        \begin{enumerate}[nosep,leftmargin=16pt,labelindent=0pt]
            \item Użytkownik wybiera plan subskrypcji.
            \item Użytkownik przechodzi do bramki płatniczej.
            \item Użytkownik podaje dane płatnicze i zatwierdza transakcję.
            \item Bramka płatnicza przetwarza płatność i zwraca wynik do systemu.
            \item System zapisuje informację o opłaconej subskrypcji i aktualizuje uprawnienia.
            \item System generuje wpis w systemie finansowo-księgowym.
        \end{enumerate}
    }
    \ucalt{%
        \begin{enumerate}[nosep,leftmargin=21pt,labelindent=0pt,label={}]
            \item[4a.] Płatność nie powiodła się – system informuje użytkownika i umożliwia ponowną próbę.
            \item[5a.] W czasie aktualizacji subskrypcji wystąpił błąd – system cofa zmiany i wyświetla komunikat o problemie.
        \end{enumerate}
    }
}

\usecasecard{tab:pu8-mapa}{Przeglądanie mapy spotów}{PU8}{%
    \ucpriority{Wysoki}
    \ucactors{Użytkownik niezalogowany, Usługa do wyświetlania mapy}
    \ucdesc{Użytkownik przegląda mapę spotów.}
    \ucpre{Użytkownik znajduje się w module mapy.}
    \ucpost{Mapa ze spotami została wyświetlona, a użytkownik może przybliżać, oddalać i przesuwać widok.}
    \ucmain{%
        \begin{enumerate}[nosep,leftmargin=16pt,labelindent=0pt]
            \item System inicjuje widok mapy z domyślnym obszarem.
            \item System pobiera listę spotów w aktualnym zakresie mapy.
            \item System rysuje znaczniki spotów na mapie.
            \item Użytkownik przesuwa lub skaluje mapę.
            \item System pobiera spoty dla nowego zakresu.
        \end{enumerate}
    }
    \ucalt{%
        \begin{enumerate}[nosep,leftmargin=21pt,labelindent=0pt,label={}]
            \item[2a.] Usługa mapy jest niedostępna – system wyświetla komunikat o błędzie.
        \end{enumerate}
    }
}

\usecasecard{tab:pu9-szukaj-na-mapie}{Wyszukiwanie spota na mapie}{PU9}{%
    \ucpriority{Wysoki}
    \ucactors{Użytkownik niezalogowany}
    \ucdesc{Użytkownik wyszukuje spota po nazwie korzystając z pola wyszukiwania na mapie.}
    \ucpre{Użytkownik widzi mapę spotów.}
    \ucpost{Mapa zostaje ustawiona na wybranego spota lub listę dopasowań.}
    \ucmain{%
        \begin{enumerate}[nosep,leftmargin=16pt,labelindent=0pt]
            \item Użytkownik wpisuje frazę w polu wyszukiwania na mapie.
            \item System podpowiada listę pasujących spotów.
            \item Użytkownik wybiera spota z listy.
            \item System przybliża mapę do wybranego spota i podświetla jego znacznik.
        \end{enumerate}
    }
    \ucalt{%
        \begin{enumerate}[nosep,leftmargin=21pt,labelindent=0pt,label={}]
            \item[2a.] Brak wyników dla podanej frazy – system informuje użytkownika o braku dopasowań.
        \end{enumerate}
    }
}

\usecasecard{tab:pu10-globalna-wyszukiwarka}{Wyszukiwanie spota w globalnej wyszukiwarce}{PU10}{%
    \ucpriority{Wysoki}
    \ucactors{Użytkownik niezalogowany, Usługa do wyświetlania mapy, Usługa do pogody}
    \ucdesc{Użytkownik wyszukuje spoty za pomocą globalnej wyszukiwarki w aplikacji.}
    \ucpre{Użytkownik znajduje sie na stronie głównej z wyszukwiarką.}
    \ucpost{Użytkownik otrzymuje listę znalezionych spotów.}
    \ucmain{%
        \begin{enumerate}[nosep,leftmargin=16pt,labelindent=0pt]
            \item Użytkownik wpisuje frazę w globalnej wyszukiwarce.
            \item System wyszukuje spoty spełniające kryteria.
            \item System wyświetla listę wyników.
        \end{enumerate}
    }
    \ucalt{%
        \begin{enumerate}[nosep,leftmargin=21pt,labelindent=0pt,label={}]
            \item[3a.] Brak wyników – system wyświetla komunikat i proponuje zmianę kryteriów wyszukiwania.
        \end{enumerate}
    }
}

\usecasecard{tab:pu11-przejscie-z-wynikow}{Przejście do spota na mapie z wyszukiwarki}{PU11}{%
    \ucpriority{Wysoki}
    \ucactors{Użytkownik niezalogowany}
    \ucdesc{Użytkownik przechodzi z wyników wyszukiwarki do widoku mapy ustawionego na konkretny spot.}
    \ucpre{Wyświetlona jest lista wyników wyszukiwania spotów.}
    \ucpost{Mapa jest przybliżona do wybranego spota, a jego szczegóły są dostępne.}
    \ucmain{%
        \begin{enumerate}[nosep,leftmargin=16pt,labelindent=0pt]
            \item Użytkownik wybiera spota z listy wyników.
            \item System przełącza widok na moduł mapy.
            \item System ustawia mapę na lokalizację spota i otwiera jego szczegóły.
        \end{enumerate}
    }
    \ucalt{Brak istotnych alternatywnych przepływów.}
}

\usecasecard{tab:pu12-komentarze-spota}{Przeglądanie komentarzy do spota}{PU12}{%
    \ucpriority{Średni}
    \ucactors{Użytkownik niezalogowany}
    \ucdesc{Użytkownik czyta komentarze pod wybranym spotem.}
    \ucpre{Wyświetlany jest widok szczegółów spota.}
    \ucpost{Lista komentarzy do spota została wyświetlona.}
    \ucmain{%
        \begin{enumerate}[nosep,leftmargin=16pt,labelindent=0pt]
            \item System pobiera komentarze powiązane ze spotem.
            \item System wyświetla komentarze w kolejności chronologicznej lub według popularności.
            \item Użytkownik przewija listę komentarzy.
        \end{enumerate}
    }
    \ucalt{%
        \begin{enumerate}[nosep,leftmargin=21pt,labelindent=0pt,label={}]
            \item[1a.] Spot nie ma jeszcze komentarzy – system wyświetla odpowiednią informację.
        \end{enumerate}
    }
}

\usecasecard{tab:pu13-pogoda}{Przeglądanie pogody na spocie}{PU13}{%
    \ucpriority{Średni}
    \ucactors{Użytkownik zalogowany, Usługa danych pogodowych}
    \ucdesc{Użytkownik sprawdza prognozę pogody dla lokalizacji spota.}
    \ucpre{Wyświetlany jest widok szczegółów spota.}
    \ucpost{Prognoza pogody dla spota została wyświetlona.}
    \ucmain{%
        \begin{enumerate}[nosep,leftmargin=16pt,labelindent=0pt]
            \item Użytkownik otwiera zakładkę pogody.
            \item System wysyła zapytanie do usługi pogodowej z lokalizacją spota.
            \item System odbiera prognozę i prezentuje ją (temperatura, prędkość wiatru, opady).
        \end{enumerate}
    }
    \ucalt{%
        \begin{enumerate}[nosep,leftmargin=21pt,labelindent=0pt,label={}]
            \item[2a.] Usługa pogodowa jest niedostępna – system wyświetla komunikat o braku danych pogodowych.
        \end{enumerate}
    }
}

\usecasecard{tab:pu14-posty-forum}{Przeglądanie postów na forum}{PU14}{%
    \ucpriority{Wysoki}
    \ucactors{Użytkownik niezalogowany}
    \ucdesc{Użytkownik przegląda listę postów na forum.}
    \ucpre{Użytkownik znajduje się w module forum.}
    \ucpost{Lista postów forum jest wyświetlona, a użytkownik może przechodzić do szczegółów.}
    \ucmain{%
        \begin{enumerate}[nosep,leftmargin=16pt,labelindent=0pt]
            \item System pobiera listę postów.
            \item System wyświetla posty z podstawowymi informacjami.
            \item Użytkownik wybiera post, który chce przeczytać.
            \item System otwiera szczegółowy widok posta.
        \end{enumerate}
    }
    \ucalt{%
        \begin{enumerate}[nosep,leftmargin=21pt,labelindent=0pt,label={}]
            \item[3a.] System nie może pobrać szczegółów psota – system wyświetla komunikat o błędzie.
        \end{enumerate}
    }
}

\usecasecard{tab:pu15-dodaj-post}{Dodanie posta na forum}{PU15}{%
    \ucpriority{Wysoki}
    \ucactors{Użytkownik zalogowany, Usługa do przechowywania plików w chmurze}
    \ucdesc{Użytkownik publikuje nowy post na forum.}
    \ucpre{Użytkownik znajduje się w module forum.}
    \ucpost{Nowy post jest widoczny na forum.}
    \ucmain{%
        \begin{enumerate}[nosep,leftmargin=16pt,labelindent=0pt]
            \item Użytkownik wybiera opcję dodania nowego posta.
            \item Użytkownik wpisuje tytuł i treść posta.
            \item (Opcjonalnie) Użytkownik dodaje załączniki (zdjęcia/filmy) do posta.
            \item Użytkownik publikuje posta.
            \item System zapisuje posta (oraz załączniki w chmurze) i wyświetla go na liście postów.
        \end{enumerate}
    }
    \ucalt{%
        \begin{enumerate}[nosep,leftmargin=21pt,labelindent=0pt,label={}]
            \item[3a.] Załącznik nie może zostać zapisany – system informuje o błędzie i pozwala opublikować posta bez pliku.
            \item[4a.] Formularz zawiera błędne lub niekompletne dane – system wyświetla komunikat i prosi o poprawę.
        \end{enumerate}
    }
}

\usecasecard{tab:pu16-dodaj-komentarz}{Dodanie komentarza na forum}{PU16}{%
    \ucpriority{Wysoki}
    \ucactors{Użytkownik zalogowany}
    \ucdesc{Użytkownik dodaje komentarz pod postem na forum.}
    \ucpre{Użytkownik jest zalogowany i widzi szczegóły posta.}
    \ucpost{Nowy komentarz został zapisany i widoczny pod postem.}
    \ucmain{%
        \begin{enumerate}[nosep,leftmargin=16pt,labelindent=0pt]
            \item Użytkownik wpisuje treść komentarza w formularzu pod postem.
            \item Użytkownik publikuje komentarz.
            \item System zapisuje komentarz i odświeża listę komentarzy.
        \end{enumerate}
    }
    \ucalt{%
        \begin{enumerate}[nosep,leftmargin=21pt,labelindent=0pt,label={}]
            \item[2a.] Treść komentarza jest niepoprawa – system wyświetla komunikat o błędzie.
        \end{enumerate}
    }
}

\usecasecard{tab:pu17-historia-postow}{Przeglądanie historii interakcji z postami}{PU17}{%
    \ucpriority{Średni}
    \ucactors{Użytkownik zalogowany}
    \ucdesc{Użytkownik przegląda historię swoich aktywności na forum (dodane posty, komentarze, reakcje).}
    \ucpre{Użytkownik jest zalogowany.}
    \ucpost{Lista interakcji użytkownika z postami jest wyświetlona.}
    \ucmain{%
        \begin{enumerate}[nosep,leftmargin=16pt,labelindent=0pt]
            \item Użytkownik przechodzi do sekcji historii aktywności.
            \item System pobiera historię interakcji użytkownika.
            \item System wyświetla listę interakcji z możliwością filtrowania.
        \end{enumerate}
    }
    \ucalt{Brak istotnych alternatywnych przepływów.}
}

\usecasecard{tab:pu18-czat-prywatny}{Utworzenie prywatnego czatu}{PU18}{%
    \ucpriority{Wysoki}
    \ucactors{Użytkownik zalogowany}
    \ucdesc{Użytkownik tworzy prywatną konwersację z innym użytkownikiem.}
    \ucpre{Użytkownik jest zalogowany i znajduje się w zakładce społeczność.}
    \ucpost{Nowy czat prywatny został utworzony i wyświetlony użytkownikowi.}
    \ucmain{%
        \begin{enumerate}[nosep,leftmargin=16pt,labelindent=0pt]
            \item Użytkownik wybiera opcję utworzenia nowego czatu.
            \item System tworzy nowy czat (jeśli nie istnieje).
            \item System otwiera widok nowego czatu.
        \end{enumerate}
    }
    \ucalt{%
        \begin{enumerate}[nosep,leftmargin=21pt,labelindent=0pt,label={}]
            \item[1a.] Taki czat już istnieje – system zamiast tworzyć nowy, otwiera istniejącą konwersację.
        \end{enumerate}
    }
}

\usecasecard{tab:pu19-czat-grupowy}{Utworzenie czatu grupowego}{PU19}{%
    \ucpriority{Średni}
    \ucactors{Użytkownik zalogowany}
    \ucdesc{Użytkownik tworzy nowy czat grupowy z kilkoma uczestnikami.}
    \ucpre{Użytkownik jest zalogowany i znajduje się na dowolnym czacie prywatnym.}
    \ucpost{Czat grupowy został utworzony i otwarty.}
    \ucmain{%
        \begin{enumerate}[nosep,leftmargin=16pt,labelindent=0pt]
            \item Użytkownik wybiera opcję utworzenia czatu grupowego.
            \item Użytkownik wybiera uczestników grupy.
            \item Użytkownik zatwierdza utworzenie czatu.
            \item System tworzy czat grupowy i dodaje do niego wskazanych użytkowników.
            \item System otwiera widok nowego czatu grupowego.
        \end{enumerate}
    }
    \ucalt{%
        \begin{enumerate}[nosep,leftmargin=21pt,labelindent=0pt,label={}]
            \item[3a.] System nie może utworzyć czatu – aplikacja informuje o błędzie.
        \end{enumerate}
    }
}

\usecasecard{tab:pu20-lista-czatow}{Przeglądanie listy czatów}{PU20}{%
    \ucpriority{Wysoki}
    \ucactors{Użytkownik zalogowany}
    \ucdesc{Użytkownik przegląda listę swoich czatów prywatnych i grupowych.}
    \ucpre{Użytkownik jest zalogowany i otwiera moduł czatu.}
    \ucpost{Lista czatów użytkownika została wyświetlona.}
    \ucmain{%
        \begin{enumerate}[nosep,leftmargin=16pt,labelindent=0pt]
            \item System pobiera listę czatów użytkownika.
            \item System wyświetla listę czatów z podstawowymi informacjami.
            \item Użytkownik wybiera czat z listy.
            \item System otwiera widok wybranego czatu.
        \end{enumerate}
    }
    \ucalt{Brak istotnych alternatywnych przepływów.}
}

\usecasecard{tab:pu20-wyslij-wiadomosc}{Wysyłanie wiadomości na czacie}{PU20}{%
    \ucpriority{Wysoki}
    \ucactors{Użytkownik zalogowany}
    \ucdesc{Użytkownik wysyła wiadomość tekstową na czacie.}
    \ucpre{Użytkownik jest zalogowany i znajduje się w widoku konkretnego czatu.}
    \ucpost{Nowa wiadomość jest zapisana i widoczna w historii czatu.}
    \ucmain{%
        \begin{enumerate}[nosep,leftmargin=16pt,labelindent=0pt]
            \item Użytkownik wpisuje treść wiadomości.
            \item Użytkownik wysyła wiadomość.
            \item System zapisuje wiadomość i dostarcza ją do uczestników czatu.
            \item System wyświetla wiadomość na liście wiadomości.
        \end{enumerate}
    }
    \ucalt{%
        \begin{enumerate}[nosep,leftmargin=21pt,labelindent=0pt,label={}]
            \item[2a.] Treść wiadomości jest pusta – system blokuje wysłanie i pozostaje w tym samym widoku.
        \end{enumerate}
    }
}

\usecasecard{tab:pu21-wyslij-gifa}{Wysyłanie GIF-a na czacie}{PU25}{%
    \ucpriority{Średni}
    \ucactors{Użytkownik zalogowany, Usługa GIF-ów}
    \ucdesc{Użytkownik wysyła animację GIF w konwersacji czatowej.}
    \ucpre{Użytkownik jest zalogowany i znajduje się w widoku czatu.}
    \ucpost{Wybrany GIF został dodany jako wiadomość w czacie.}
    \ucmain{%
        \begin{enumerate}[nosep,leftmargin=16pt,labelindent=0pt]
            \item Użytkownik wybiera opcję dodania GIF-a.
            \item System otwiera okno wyszukiwarki GIF-ów.
            \item Użytkownik wybiera lub wyszukuje GIF-a.
            \item Użytkownik zatwierdza wysłanie GIF-a.
            \item System dodaje GIF-a jako wiadomość na czacie.
        \end{enumerate}
    }
    \ucalt{%
        \begin{enumerate}[nosep,leftmargin=21pt,labelindent=0pt,label={}]
            \item[2a.] Usługa GIF-ów jest niedostępna – system informuje o braku możliwości wysłania GIF-a.
        \end{enumerate}
    }
}

\usecasecard{tab:pu22-wyslij-plik}{Wysyłanie pliku na czacie}{PU22}{%
    \ucpriority{Średni}
    \ucactors{Użytkownik zalogowany, Usługa do przechowywania plików w chmurze}
    \ucdesc{Użytkownik wysyła plik (np. zdjęcie, film) w czacie.}
    \ucpre{Użytkownik jest zalogowany i znajduje się w widoku czatu.}
    \ucpost{Plik został zapisany w chmurze i powiązany z wiadomością na czacie.}
    \ucmain{%
        \begin{enumerate}[nosep,leftmargin=16pt,labelindent=0pt]
            \item Użytkownik wybiera opcję dodania pliku.
            \item Użytkownik wybiera plik z urządzenia.
            \item System przesyła plik do usługi przechowywania w chmurze.
            \item System tworzy wiadomość z odnośnikiem do pliku.
            \item System wyświetla wiadomość na liście czatu.
        \end{enumerate}
    }
    \ucalt{%
        \begin{enumerate}[nosep,leftmargin=21pt,labelindent=0pt,label={}]
            \item[3a.] Przesyłanie pliku nie powiodło się – system informuje użytkownika i umożliwia ponowną próbę.
        \end{enumerate}
    }
}

\usecasecard{tab:pu23-edycja-czatu}{Edycja ustawień czatu}{PU23}{%
    \ucpriority{Niski}
    \ucactors{Użytkownik zalogowany}
    \ucdesc{Użytkownik modyfikuje ustawienia czatu (np. nazwę, avatar, tryb powiadomień).}
    \ucpre{Użytkownik jest zalogowany i ma uprawnienia do edycji danego czatu.}
    \ucpost{Zaktualizowane ustawienia czatu są zapisane i widoczne dla uczestników.}
    \ucmain{%
        \begin{enumerate}[nosep,leftmargin=16pt,labelindent=0pt]
            \item Użytkownik otwiera panel ustawień czatu.
            \item Użytkownik wprowadza zmiany (np. nazwę, opis, avatar).
            \item Użytkownik zapisuje zmiany.
            \item System waliduje dane i aktualizuje konfigurację czatu.
        \end{enumerate}
    }
    \ucalt{Brak istotnych alternatywnych przepływów poza walidacją pól.}
}

\usecasecard{tab:pu24-dodaj-czlonka}{Dodanie członka do czatu grupowego}{PU24}{%
    \ucpriority{Średni}
    \ucactors{Użytkownik zalogowany}
    \ucdesc{Użytkownik z uprawnieniami administratora dodaje nowego uczestnika do czatu grupowego.}
    \ucpre{Użytkownik jest zalogowany, znajduje się w czacie grupowym i ma prawo zarządzać członkami.}
    \ucpost{Nowy uczestnik został dodany do czatu grupowego.}
    \ucmain{%
        \begin{enumerate}[nosep,leftmargin=16pt,labelindent=0pt]
            \item Użytkownik otwiera listę uczestników czatu grupowego.
            \item Użytkownik wybiera opcję dodania nowego członka.
            \item Użytkownik wskazuje użytkownika do dodania i zatwierdza wybór.
            \item System dodaje wskazanego użytkownika do czatu grupowego.
        \end{enumerate}
    }
    \ucalt{%
        \begin{enumerate}[nosep,leftmargin=21pt,labelindent=0pt,label={}]
            \item[3a.] Operacja nie powiodła się – system informuje o błędzie.
        \end{enumerate}
    }
}

\usecasecard{tab:pu25-historia-czatu}{Przeszukiwanie historii czatu}{PU25}{%
    \ucpriority{Niski}
    \ucactors{Użytkownik premium}
    \ucdesc{Użytkownik wyszukuje konkretne wiadomości w historii czatu.}
    \ucpre{Użytkownik jest zalogowany jako użytkownik premium i znajduje się w widoku czatu.}
    \ucpost{Wiadomości spełniające kryteria wyszukiwania zostały wyświetlone.}
    \ucmain{%
        \begin{enumerate}[nosep,leftmargin=16pt,labelindent=0pt]
            \item Użytkownik otwiera pole wyszukiwania historii w czacie.
            \item Użytkownik wpisuje frazę lub filtr (np. zakres dat, autor).
            \item System filtruje wiadomości zgodnie z kryteriami.
            \item System prezentuje listę dopasowanych fragmentów rozmowy.
        \end{enumerate}
    }
    \ucalt{Brak istotnych alternatywnych przepływów.}
}

\usecasecard{tab:pu26-wyslane-pliki}{Przeglądanie wysłanych plików na czacie}{PU26}{%
    \ucpriority{Niski}
    \ucactors{Użytkownik premium, Usługa do przechowywania plików w chmurze}
    \ucdesc{Użytkownik przegląda listę plików wysłanych w ramach czatów.}
    \ucpre{Użytkownik jest zalogowany jako użytkownik premium.}
    \ucpost{Użytkownik widzi listę wysłanych plików i może przechodzić do powiązanych czatów.}
    \ucmain{%
        \begin{enumerate}[nosep,leftmargin=16pt,labelindent=0pt]
            \item Użytkownik otwiera sekcję „Wysłane pliki”.
            \item System pobiera metadane plików z usługi przechowywania.
            \item System wyświetla listę plików z podstawowymi informacjami (nazwa, typ, data).
            \item Użytkownik wybiera plik, aby otworzyć go lub przejść do powiązanego czatu.
        \end{enumerate}
    }
    \ucalt{Brak istotnych alternatywnych przepływów.}
}

\usecasecard{tab:pu27-dodaj-spota}{Dodanie spota w profilu użytkownika}{PU27}{%
    \ucpriority{Wysoki}
    \ucactors{Użytkownik zalogowany, Usługa do wyświetlania mapy, Usługa do przechowywania plików w chmurze}
    \ucdesc{Użytkownik dodaje nowy spot poprzez swój profil.}
    \ucpre{Użytkownik jest zalogowany i znajduje się w widoku swojego profilu.}
    \ucpost{Nowy spot został zapisany i widoczny na mapie oraz w profilu użytkownika.}
    \ucmain{%
        \begin{enumerate}[nosep,leftmargin=16pt,labelindent=0pt]
            \item Użytkownik wybiera opcję „Dodaj spota”.
            \item Użytkownik uzupełnia podstawowe informacje o spocie (nazwa, opis, typ).
            \item Użytkownik wskazuje lokalizację spota na mapie.
            \item (Opcjonalnie) Użytkownik dodaje zdjęcia/filmy do spota.
            \item Użytkownik zapisuje spota.
            \item System zapisuje dane spota (oraz pliki w chmurze) i aktualizuje mapę oraz profil użytkownika.
        \end{enumerate}
    }
    \ucalt{%
        \begin{enumerate}[nosep,leftmargin=21pt,labelindent=0pt,label={}]
            \item[2a.] Formularz zawiera błędy – system wyświetla komunikat i zaznacza wymagające poprawy pola.
        \end{enumerate}
    }
}

\usecasecard{tab:pu28-profil-wlasny}{Przeglądanie profilu użytkownika}{PU28}{%
    \ucpriority{Wysoki}
    \ucactors{Użytkownik zalogowany}
    \ucdesc{Użytkownik przegląda swój profil (lista spotów, media, podstawowe dane).}
    \ucpre{Użytkownik jest zalogowany.}
    \ucpost{Wyświetlony jest widok profilu użytkownika wraz z jego zawartością.}
    \ucmain{%
        \begin{enumerate}[nosep,leftmargin=16pt,labelindent=0pt]
            \item Użytkownik otwiera swój profil.
            \item System pobiera dane profilu (informacje podstawowe, spoty, media).
            \item System wyświetla dane w odpowiednich sekcjach (spoty, zdjęcia, filmy, komentarze).
        \end{enumerate}
    }
    \ucalt{Brak istotnych alternatywnych przepływów.}
}

\usecasecard{tab:pu29-profil-innego}{Przeglądanie profilu innego użytkownika}{PU29}{%
    \ucpriority{Średni}
    \ucactors{Użytkownik zalogowany}
    \ucdesc{Użytkownik ogląda profil innego użytkownika (np. z mapy, forum lub społeczności).}
    \ucpre{Użytkownik jest zalogowany i ma dostęp do odnośnika do profilu innego użytkownika.}
    \ucpost{Profil innego użytkownika został wyświetlony.}
    \ucmain{%
        \begin{enumerate}[nosep,leftmargin=16pt,labelindent=0pt]
            \item Użytkownik wybiera odnośnik do profilu innego użytkownika.
            \item System pobiera dane profilu docelowego użytkownika.
            \item System wyświetla profil (media, podstawowe informacje).
        \end{enumerate}
    }
    \ucalt{%
        \begin{enumerate}[nosep,leftmargin=21pt,labelindent=0pt,label={}]
            \item[2a.] Wystąpił błąd podczas pobierania danych użytkownika  – system wyświetla informację o błędzie.
        \end{enumerate}
    }
}


\usecasecard{tab:pu30-dodaj-znajomego}{Dodanie użytkownika do znajomych}{PU30}{%
    \ucpriority{Średni}
    \ucactors{Użytkownik zalogowany}
    \ucdesc{Użytkownik wysyła lub akceptuje zaproszenie do znajomych.}
    \ucpre{Użytkownik jest zalogowany i przegląda profil innego użytkownika.}
    \ucpost{Relacja „znajomy” została utworzona lub zaproszenie czeka na akceptację.}
    \ucmain{%
        \begin{enumerate}[nosep,leftmargin=16pt,labelindent=0pt]
            \item Użytkownik klika przycisk „Dodaj do znajomych”.
            \item System sprawdza, czy relacja już istnieje.
            \item System tworzy nowe zaproszenie.
            \item System informuje o statusie o wysłaniu zaproszenia.
        \end{enumerate}
    }
    \ucalt{Brak istotnych alternatywnych przepływów.}
    }
}

\usecasecard{tab:pu31-spolecznosci}{Przeglądanie społeczności}{PU31}{%
    \ucpriority{Średni}
    \ucactors{Użytkownik zalogowany}
    \ucdesc{Użytkownik przegląda społeczności, grupy lub listy znajomych powiązane z aplikacją.}
    \ucpre{Użytkownik jest zalogowany.}
    \ucpost{Lista społeczności lub znajomych została wyświetlona.}
    \ucmain{%
        \begin{enumerate}[nosep,leftmargin=16pt,labelindent=0pt]
            \item Użytkownik przechodzi do sekcji społeczności.
            \item System pobiera listę społeczności i znajomych użytkownika.
            \item System wyświetla listę z możliwością przechodzenia do profili i czatów.
        \end{enumerate}
    }
    \ucalt{Brak istotnych alternatywnych przepływów.}
}

\usecasecard{tab:pu32-zarzadzaj-komentarzami-spot}{Zarządzanie komentarzami do spotów}{PU32}{%
    \ucpriority{Niski}
    \ucactors{Użytkownik zalogowany (właściciel spota lub moderator)}
    \ucdesc{Użytkownik zarządza komentarzami dodanymi do spota (edycja, usuwanie, ukrywanie).}
    \ucpre{Użytkownik jest zalogowany i wyświetla szczegóły spota.}
    \ucpost{Wybrane komentarze zostały zaktualizowane lub ukryte zgodnie z działaniem użytkownika.}
    \ucmain{%
        \begin{enumerate}[nosep,leftmargin=16pt,labelindent=0pt]
            \item Użytkownik otwiera panel zarządzania komentarzami dla danego spota.
            \item System pobiera listę komentarzy wraz z możliwymi akcjami.
            \item Użytkownik wybiera komentarz i akcję (np. edytuj, usuń, ukryj).
            \item System wykonuje wybraną akcję na komentarzu.
            \item System odświeża listę komentarzy.
        \end{enumerate}
    }
    \ucalt{%
        \begin{enumerate}[nosep,leftmargin=21pt,labelindent=0pt,label={}]
            \item[3a.] Użytkownik nie ma uprawnień do zarządzania komentarzem – system wyświetla komunikat o braku uprawnień.
        \end{enumerate}
    }
}

\usecasecard{tab:pu33-zarzadzaj-komentarzami-forum}{Zarządzanie komentarzami na forum}{PU33}{%
    \ucpriority{Niski}
    \ucactors{Użytkownik zalogowany (autor posta lub moderator)}
    \ucdesc{Użytkownik zarządza komentarzami pod postami forum (edycja, usuwanie, przypinanie).}
    \ucpre{Użytkownik jest zalogowany i ma dostęp do danego wątku forum.}
    \ucpost{Komentarze zostały zaktualizowane zgodnie z działaniami użytkownika.}
    \ucmain{%
        \begin{enumerate}[nosep,leftmargin=16pt,labelindent=0pt]
            \item Użytkownik otwiera widok komentarzy pod postem.
            \item Użytkownik wybiera komentarz i odpowiednią akcję.
            \item System weryfikuje uprawnienia użytkownika.
            \item System wykonuje wybraną akcję i aktualizuje widok.
        \end{enumerate}
    }
    \ucalt{%
        \begin{enumerate}[nosep,leftmargin=21pt,labelindent=0pt,label={}]
            \item[3a.] Użytkownik nie ma wymaganych uprawnień – system blokuje operację i informuje o tym.
        \end{enumerate}
    }
}

\usecasecard{tab:pu34-zarzadzaj-postami}{Zarządzanie postami na forum}{PU34}{%
    \ucpriority{Niski}
    \ucactors{Użytkownik zalogowany (autor posta lub moderator)}
    \ucdesc{Użytkownik edytuje, archiwizuje lub usuwa własne posty na forum.}
    \ucpre{Użytkownik jest zalogowany i otwiera listę swoich postów lub moderowany dział forum.}
    \ucpost{Status wybranych postów został zaktualizowany.}
    \ucmain{%
        \begin{enumerate}[nosep,leftmargin=16pt,labelindent=0pt]
            \item Użytkownik przechodzi do sekcji zarządzania postami.
            \item System pobiera listę postów użytkownika (lub działu).
            \item Użytkownik wybiera post i żądaną akcję (edycja, archiwizacja, usunięcie).
            \item System zapisuje zmiany i aktualizuje listę postów.
        \end{enumerate}
    }
    \ucalt{%
        \begin{enumerate}[nosep,leftmargin=21pt,labelindent=0pt,label={}]
            \item[3a.] Użytkownik próbuje usunąć post z zablokowanego wątku – system odmawia wykonania operacji.
        \end{enumerate}
    }
}

\usecasecard{tab:pu35-zglos-komentarz}{Zgłoszenie komentarza naruszającego regulamin}{PU35}{%
    \ucpriority{Średni}
    \ucactors{Użytkownik zalogowany}
    \ucdesc{Użytkownik zgłasza komentarz na forum.}
    \ucpre{Użytkownik widzi komentarz w aplikacji.}
    \ucpost{Zgłoszenie komentarza zostało zapisane i trafiło do kolejki moderacyjnej.}
    \ucmain{%
        \begin{enumerate}[nosep,leftmargin=16pt,labelindent=0pt]
            \item Użytkownik wybiera opcję „Zgłoś komentarz”.
            \item Użytkownik określa powód zgłoszenia.
            \item System zapisuje zgłoszenie i wiąże je z komentarzem i zgłaszającym.
        \end{enumerate}
    }
    \ucalt{Brak istotnych alternatywnych przepływów.}
}

\usecasecard{tab:pu36-zglos-posta}{Zgłoszenie posta na forum}{PU36}{%
    \ucpriority{Średni}
    \ucactors{Użytkownik zalogowany}
    \ucdesc{Użytkownik zgłasza post forum jako naruszający regulamin lub tematykę.}
    \ucpre{Wyświetlony jest widok posta na forum.}
    \ucpost{Zgłoszenie posta zostało zapisane i przekazane moderatorom.}
    \ucmain{%
        \begin{enumerate}[nosep,leftmargin=16pt,labelindent=0pt]
            \item Użytkownik wybiera opcję „Zgłoś post”.
            \item Użytkownik wybiera kategorię naruszenia i potwierdza zgłoszenie.
            \item System zapisuje zgłoszenie i oznacza post jako zgłoszony.
        \end{enumerate}
    }
    \ucalt{Brak istotnych alternatywnych przepływów.}
}

\usecasecard{tab:pu37-zmien-typ-mapy}{Zmiana typu mapy}{PU37}{%
    \ucpriority{Niski}
    \ucactors{Użytkownik premium, Usługa do wyświetlania mapy}
    \ucdesc{Użytkownik zmienia typ mapy (np. standardowa, satelitarna, hybrydowa).}
    \ucpre{Użytkownik premium jest na ekranie mapy.}
    \ucpost{Mapa jest wyświetlana w wybranym typie.}
    \ucmain{%
        \begin{enumerate}[nosep,leftmargin=16pt,labelindent=0pt]
            \item Użytkownik otwiera ustawienia widoku mapy.
            \item Użytkownik wybiera typ mapy z dostępnej listy.
            \item System przełącza widok mapy na wybrany typ.
        \end{enumerate}
    }
    \ucalt{%
        \begin{enumerate}[nosep,leftmargin=21pt,labelindent=0pt,label={}]
            \item[3a.] Wybrany typ mapy nie jest dostępny (błąd usługi mapowej) – system przywraca poprzedni typ i informuje o błędzie.
        \end{enumerate}
    }
}

\usecasecard{tab:pu38-strefy-pansa}{Przeglądanie stref PANSA}{PU38}{%
    \ucpriority{Niski}
    \ucactors{Użytkownik premium, Usługa do wyświetlania mapy}
    \ucdesc{Użytkownik wyświetla na mapie strefy przestrzeni powietrznej \gls{PANSA}.}
    \ucpre{Użytkownik premium ma otwarty moduł mapy.}
    \ucpost{Strefy PANSA zostały zwizualizowane na mapie.}
    \ucmain{%
        \begin{enumerate}[nosep,leftmargin=16pt,labelindent=0pt]
            \item Użytkownik włącza warstwę „Strefy PANSA”.
            \item System pobiera dane o strefach.
            \item System nakłada kontury stref na mapę.
        \end{enumerate}
    }
    \ucalt{%
        \begin{enumerate}[nosep,leftmargin=21pt,labelindent=0pt,label={}]
            \item[2a.] Dane o strefach są chwilowo niedostępne – system komunikuje problem i nie włącza warstwy.
        \end{enumerate}
    }
}


%! Author = mateusz
%! Date = 17/10/2025

\section{Wymagania ogólne i dziedzinowe}
\label{sec:wymagania-ogolne-i-dziedzinowe}
%! Author = mateusz
%! Date = 17/10/2025

\section{Wymagania funkcjonalne}
\label{sec:wymagania-funkcjonalne}
%! Author = mateusz
%! Date = 17/10/2025

\subsection{Wymagania pozafunkcjonalne}
\label{subsec:wymagania-pozafunkcjonalne}

Niniejszy rozdział zawiera wymagania pozafunkcjonalne postawione systemowi.
Został on podzielony tematycznie.

\subimport{subsections/}{czat.tex}
\subimport{subsections/}{mapa.tex}
\subimport{subsections/}{forum.tex}
\subimport{subsections/}{panel-uzytkownika.tex}
\subimport{subsections/}{wyszukiwarka-spotow.tex}
\subimport{subsections/}{panel-logowania.tex}
\subimport{subsections/}{motyw.tex}
%! Author = mateusz
%! Date = 17/10/2025

\section{Wymagania interfejs z otoczeniem}
\label{sec:wymagania-interfejs-z-otoczeniem}
%! Author = mateusz
%! Date = 17/10/2025

\section{Wymagania na środowisko docelowe}
\label{sec:wymagania-na-srodowisko-docelowe}

    %! Author = mateusz
%! Date = 20/09/2025


\chapter{Projekt}
\label{ch:projekt}

%! Author = Stanisław Oziemczuk
%! Date = 08/12/2025


\section{Wzorce projektowe}
\label{sec:wzorce-projektowe}

Podczas prac deweloperskich nad projektem skorzystano z różnych \glslink{wzorzec}{wzorców projektowych}.
Zarówno na \glslink{frontend}{frontendzie}, jak i \glslink{backend}{backendzie} istnieje wiele propozycji, z których członkowie zespołu starali się korzystać
w celu poszerzenia wiedzy, umiejętności oraz otrzymania wysokiej jakości pisanego kodu.
Poniżej przedstawiono wybrane rozwiązania.

\begin{itemize}
    \item \textbf{\glslink{backend}{Backend}}
    \begin{itemize}
        \item \textbf{Chain of Responsibility}
        \item \textbf{Fasada} \textemdash \space jest strukturalnym \glslink{wzorzec}{wzorcem} projektowym, który nakłada na
        bibliotekę lub zestaw klas interfejs ułatwiający korzystanie z zawartych w nich operacji.
        \begin{figure}[H]
            \centering
            \includegraphics[width=1\textwidth]{attachments/wzorce-projektowe/facade}
            \caption{Diagram klas wzorca projektowego Fasada}
            \label{fig:diagram-facade}
        \end{figure}
        \noindent

        W projekcie Fasada została zastosowana zaimplementowana jako \emph{PolygonAreaCalculator}.
        To klasa odpowiedzialna za obliczanie pola powierzchni \glslink{spot}{spota} na podstawie ograniczających go punktów.
        Do wykonywania konkretnych obliczeń wykorzystano \glslink{biblioteka}{bibliotekę} \emph{geographiclib}, której komponenty
        są używane podczas wywołana metody \emph{calculateArea}.
        Dzięki zastosowaniu Fasady, gdy zajdzie potrzeba zmiany biblioteki, ponownej implementacji będzie wymagać tylko
        metoda \emph{calculateArea} \textendash \space sposób jej wywoływania pozostanie bez zmian.

        Poniżej przedstawiono implentację klasy \emph{PolygonAreaCalculator} oraz jej przykładowe użycie podczas operacji dodawania nowego \glslink{spot}{spota}.
        \begin{figure}[H]
            \centering
            \includegraphics[width=1\textwidth]{attachments/wzorce-projektowe/facade_implementation}
            \caption{Implementacja wzorca Fasada}
            \label{fig:facade-implementation}
        \end{figure}
        \noindent
        \begin{figure}[H]
            \centering
            \includegraphics[width=1\textwidth]{attachments/wzorce-projektowe/facade_usage}
            \caption{Przykładowe użycie klasy PolygonAreaCalculator}
            \label{fig:facade-usage}
        \end{figure}
        \noindent
        \item \textbf{Singleton} \textemdash \space to kreacyjny \glslink{wzorzec}{wzorzec} projektowy zapewniający
        istnienie dokładnie jednej instancji danego obiektu, która jest dostępna globalnie.
        Kontruktor takiego obiektu jest prywatny, a dostęp do niego odbywa się poprzez statyczną metodę zwracającą
        istniejącą instancję lub jeśli jej nie ma, tworzącą nową.
        Używany jest między innymi do zarządzania konfiguracjami czy połączeniami do bazy danych.
        \glslink{wzorzec}{Wzorzec} Singleton łamie zasadę \glslink{srp}{Single Responsibilty}, ponieważ taki obiekt oprócz wykonywania swojej logiki,
        dba o swoją unikatowość.
        \begin{figure}[H]
            \centering
            \includegraphics[width=1\textwidth]{attachments/wzorce-projektowe/singleton}
            \caption{Diagram klas wzorca projektowego Singleton}
            \label{fig:diagram-singleton}
        \end{figure}
        \noindent
        \item \textbf{Builder} \textemdash \space kreacyjny \glslink{wzorzec}{wzorzec} projektowy, który ułatwia tworzenie
        skomplikowanych obiektów poprzez rozbicie tego procesu na mniejsze, konfigurowalne etapy.
        Eliminuje potrzebę korzystania z kontruktorów zawierjących wiele parametrów.
        Stworzony zostaje obiekt budujący (Budowniczy), który implemnetuje poszczególne kroki kontrukcji obiektu, a na końcu
        wywoływana jest metoda inicjalizująca go.
        Nie jest wymagane wywołanie wszystkich kroków, ponadto można stworzyć wielu Budowniczych, kreujących różne warianty obiektu.
        \begin{figure}[H]
            \centering
            \includegraphics[width=1\textwidth]{attachments/wzorce-projektowe/builder}
            \caption{Diagram klas wzorca projektowego Builder}
            \label{fig:diagram-builder}
        \end{figure}
        \noindent
    \end{itemize}
    \item \textbf{\glslink{frontend}{Frontend}}
    \begin{itemize}
        \item \textbf{Hooks Pattern}
        \item \textbf{Error Boundary}
        \item \textbf{Portal}
        \item \textbf{Protected route}
    \end{itemize}
\end{itemize}

Opisy \glslink{wzorzec}{wzorców} projektowych użytych na \glslink{backend}{backendzie} (oprócz MVC) zostały
wykonane na podstawie treści zawartych w książce \cite{wzorce-projektowe}.
%! Author = Mateusz
%! Date = 23/11/2025

\section{Architektura systemu}
\label{sec:architektura-systemu}

W niniejszym rozdziale przedstawiona zostanie architektura systemu, czyli sposób,
w jaki poszczególne komponenty komunikują się między sobą, a także technologie,
za pomocą których zostały stworzone.

Jednym z kluczowych etapów podczas realizacji projektu był wybór odpowiedniej architektury systemowej.
Ostatecznie zdecydowaliśmy się na oddzielenie poszczególnych warstw aplikacji, co pozwoliło uzyskać
większą elastyczność, skalowalność oraz prostszą możliwość rozwoju w przyszłości.
Przyjęte komponenty prezentują się następująco:

\begin{itemize}
\item \glslink{frontend}{frontend} – \gls{react} z wykorzystaniem \glslink{type-script}{TypeScriptu},
\item \glslink{backend}{backend} – Java Spring Boot,
\item \glslink{baza-danych}{baza danych} – PostgreSQL,
\item \glslink{redis}{redis} – wykorzystywany jako \glslink{baza-danych}{baza danych} klucz–wartość pełniąca rolę warstwy cache.
\end{itemize}

Jest to podejście, z którym zespół projektowy ma największe doświadczenie, dlatego zdecydowaliśmy
się na jego zastosowanie.
Pozwala ono również na tworzenie aplikacji responsywnej, dostępnej zarówno na komputerach,
jak i urządzeniach mobilnych.
Warstwa wizualna została przygotowana przy użyciu \gls{react} w wersji z \glslink{type-script}{TypeScriptem} oraz
\glslink{biblioteka}{biblioteki} Tailwind CSS, zapewniającej szybkie i wygodne stylowanie komponentów.
Z kolei za komunikację oraz logikę biznesową odpowiada \glslink{backend}{backend} oparty
na \glslink{framework}{frameworku} Spring Boot, realizujący założenia architektury
\glslink{rest_api}{REST API}.
Jako system zarządzania danymi wybraliśmy relacyjną bazę danych PostgreSQL, z
którą zespół posiada największe doświadczenie.
Relacyjny model danych doskonale sprawdza się w tym projekcie, zapewniając integralność danych,
możliwość tworzenia złożonych zapytań oraz wysoką stabilność.

\glslink{redis}{Redis} został wykorzystany jako warstwa \glslink{cache}{cache}, której zadaniem jest przyspieszenie działania aplikacji
poprzez ograniczenie liczby odwołań do głównej \glslink{baza-danych}{bazy danych}.
Dzięki przechowywaniu często wykorzystywanych danych w pamięci operacyjnej znacznie skraca się czas
odpowiedzi systemu, co pozytywnie wpływa na wydajność oraz skalowalność rozwiązania.
Zastosowanie \glslink{redis}{Redisa} okazało się szczególnie korzystne w przypadku operacji powtarzalnych i odczytowych,
które nie wymagają każdorazowego dostępu do relacyjnej \glslink{baza-danych}{bazy danych}.

\subimport{chapters/projekt/sections/architektura-systemu/}{diagram-architektury.tex}
\subimport{chapters/projekt/sections/architektura-systemu/}{komponenty-systemu.tex}

%! Author = mateusz
%! Date = 17/10/2025

\section{Projekt bazy danych}
\label{sec:projekt-bazy-danych}
%! Author = Mateusz
%! Date = 17/10/2025

\section{Architektura interfejsu użytkownika}
\label{sec:architektura-interfejsu-uzytkownika}

Niniejszy rozdział opisuje projekt interfejsu użytkownika stworzony w figmie w ramach przedmiotu PRZ1.

\subimport{chapters/projekt/sections/architektura-interfejsu-uzytkownika/}{projekt-strony-glownej.tex}
\subimport{chapters/projekt/sections/architektura-interfejsu-uzytkownika/}{projekt-panelu-logowania.tex}
\subimport{chapters/projekt/sections/architektura-interfejsu-uzytkownika/}{projekt-mapy.tex}
\subimport{chapters/projekt/sections/architektura-interfejsu-uzytkownika/}{projekt-chatu.tex}
\subimport{chapters/projekt/sections/architektura-interfejsu-uzytkownika/}{projekt-forum.tex}
\setcounter{secnumdepth}{4}
\setcounter{tocdepth}{4}

\subimport{chapters/projekt/sections/architektura-interfejsu-uzytkownika/}{projekt-panelu-uzytkownika.tex}

\setcounter{secnumdepth}{3}
\setcounter{tocdepth}{3}


    %! Author = Mateusz Redosz
%! Date = 20/09/2025


\chapter{Realizacja Projektu}
\label{ch:realizacja}

W niniejszym rozdziale przedstawiono kluczowe elementy implementacji systemu.
Opis obejmuje realizację poszczególnych modułów aplikacji po stronie frontendu oraz backendu,
a także zagadnienia związane z architekturą, zastosowanymi wzorcami projektowymi oraz automatyzacją
procesu dostarczania nowego kodu.

Ze względu na złożoność projektu i objętość kodu źródłowego, opis skupiono na komponentach mających
największy wpływ na działanie systemu.
Pominięto natomiast szczegółowe omówienie
elementów o charakterze powtarzalnym lub wspierającym, takich jak definicje encji, klasy \gls{dto},
mappery czy pomocnicze klasy narzędziowe, o ile nie wnosiły one istotnych informacji z perspektywy
zaprojektowanych mechanizmów i funkcjonalności.

%! Author = Stanisław Oziemczuk
%! Date = 08/12/2025


\section{Wzorce projektowe}
\label{sec:wzorce-projektowe}

Podczas prac deweloperskich nad projektem skorzystano z różnych \glslink{wzorzec}{wzorców projektowych}.
Zarówno na \glslink{frontend}{frontendzie}, jak i \glslink{backend}{backendzie} istnieje wiele propozycji, z których członkowie zespołu starali się korzystać
w celu poszerzenia wiedzy, umiejętności oraz otrzymania wysokiej jakości pisanego kodu.
Poniżej przedstawiono wybrane rozwiązania.

\begin{itemize}
    \item \textbf{\glslink{backend}{Backend}}
    \begin{itemize}
        \item \textbf{Chain of Responsibility}
        \item \textbf{Fasada} \textemdash \space jest strukturalnym \glslink{wzorzec}{wzorcem} projektowym, który nakłada na
        bibliotekę lub zestaw klas interfejs ułatwiający korzystanie z zawartych w nich operacji.
        \begin{figure}[H]
            \centering
            \includegraphics[width=1\textwidth]{attachments/wzorce-projektowe/facade}
            \caption{Diagram klas wzorca projektowego Fasada}
            \label{fig:diagram-facade}
        \end{figure}
        \noindent

        W projekcie Fasada została zastosowana zaimplementowana jako \emph{PolygonAreaCalculator}.
        To klasa odpowiedzialna za obliczanie pola powierzchni \glslink{spot}{spota} na podstawie ograniczających go punktów.
        Do wykonywania konkretnych obliczeń wykorzystano \glslink{biblioteka}{bibliotekę} \emph{geographiclib}, której komponenty
        są używane podczas wywołana metody \emph{calculateArea}.
        Dzięki zastosowaniu Fasady, gdy zajdzie potrzeba zmiany biblioteki, ponownej implementacji będzie wymagać tylko
        metoda \emph{calculateArea} \textendash \space sposób jej wywoływania pozostanie bez zmian.

        Poniżej przedstawiono implentację klasy \emph{PolygonAreaCalculator} oraz jej przykładowe użycie podczas operacji dodawania nowego \glslink{spot}{spota}.
        \begin{figure}[H]
            \centering
            \includegraphics[width=1\textwidth]{attachments/wzorce-projektowe/facade_implementation}
            \caption{Implementacja wzorca Fasada}
            \label{fig:facade-implementation}
        \end{figure}
        \noindent
        \begin{figure}[H]
            \centering
            \includegraphics[width=1\textwidth]{attachments/wzorce-projektowe/facade_usage}
            \caption{Przykładowe użycie klasy PolygonAreaCalculator}
            \label{fig:facade-usage}
        \end{figure}
        \noindent
        \item \textbf{Singleton} \textemdash \space to kreacyjny \glslink{wzorzec}{wzorzec} projektowy zapewniający
        istnienie dokładnie jednej instancji danego obiektu, która jest dostępna globalnie.
        Kontruktor takiego obiektu jest prywatny, a dostęp do niego odbywa się poprzez statyczną metodę zwracającą
        istniejącą instancję lub jeśli jej nie ma, tworzącą nową.
        Używany jest między innymi do zarządzania konfiguracjami czy połączeniami do bazy danych.
        \glslink{wzorzec}{Wzorzec} Singleton łamie zasadę \glslink{srp}{Single Responsibilty}, ponieważ taki obiekt oprócz wykonywania swojej logiki,
        dba o swoją unikatowość.
        \begin{figure}[H]
            \centering
            \includegraphics[width=1\textwidth]{attachments/wzorce-projektowe/singleton}
            \caption{Diagram klas wzorca projektowego Singleton}
            \label{fig:diagram-singleton}
        \end{figure}
        \noindent
        \item \textbf{Builder} \textemdash \space kreacyjny \glslink{wzorzec}{wzorzec} projektowy, który ułatwia tworzenie
        skomplikowanych obiektów poprzez rozbicie tego procesu na mniejsze, konfigurowalne etapy.
        Eliminuje potrzebę korzystania z kontruktorów zawierjących wiele parametrów.
        Stworzony zostaje obiekt budujący (Budowniczy), który implemnetuje poszczególne kroki kontrukcji obiektu, a na końcu
        wywoływana jest metoda inicjalizująca go.
        Nie jest wymagane wywołanie wszystkich kroków, ponadto można stworzyć wielu Budowniczych, kreujących różne warianty obiektu.
        \begin{figure}[H]
            \centering
            \includegraphics[width=1\textwidth]{attachments/wzorce-projektowe/builder}
            \caption{Diagram klas wzorca projektowego Builder}
            \label{fig:diagram-builder}
        \end{figure}
        \noindent
    \end{itemize}
    \item \textbf{\glslink{frontend}{Frontend}}
    \begin{itemize}
        \item \textbf{Hooks Pattern}
        \item \textbf{Error Boundary}
        \item \textbf{Portal}
        \item \textbf{Protected route}
    \end{itemize}
\end{itemize}

Opisy \glslink{wzorzec}{wzorców} projektowych użytych na \glslink{backend}{backendzie} (oprócz MVC) zostały
wykonane na podstawie treści zawartych w książce \cite{wzorce-projektowe}.
%! Author = Mateusz Redosz
%! Date = 26/11/2025


\section{Implementacja backendu}
\label{sec:implementacja-backendu}

W niniejszym rozdziale przedstawiono strukturę backendu aplikacji, opis implementowanych endpointów,
integrację z bazą danych, mechanizmy uwierzytelniania oraz proces konteneryzacji systemu.

\subimport{chapters/realizacja-projektu/sections/backend/}{struktura-projektu.tex}
\subimport{chapters/realizacja-projektu/sections/backend/endpoints/}{endpoints.tex}
\subimport{chapters/realizacja-projektu/sections/backend/}{integracja-z-baza-danych.tex}
\subimport{chapters/realizacja-projektu/sections/backend/}{obsluga-uwierzytelnienia.tex}
\subimport{chapters/realizacja-projektu/sections/backend/}{modul-czatu.tex}
\subimport{chapters/realizacja-projektu/sections/backend/}{konteneryzacja.tex}

%! Author = Mateusz Redosz
%! Date = 14/10/2025

\section{Implementacja frontendu}
\label{sec:implementacja-frontendu}

W niniejszym rozdziale przedstawiono proces implementacji części \glslink{frontend}{frontendowej} aplikacji. \newline

\subimport{chapters/realizacja-projektu/sections/frontend/}{struktura-aplikacji.tex}
\subimport{chapters/realizacja-projektu/sections/frontend/}{zarzadzanie-przeplywem-danych.tex}
\subimport{chapters/realizacja-projektu/sections/frontend/}{integracja-z-backendem.tex}
\subimport{chapters/realizacja-projektu/sections/frontend/}{style.tex}
\subimport{chapters/realizacja-projektu/sections/frontend/}{wyszukiwarka-spotow.tex}
\subimport{chapters/realizacja-projektu/sections/frontend/}{mapa.tex}
\subimport{chapters/realizacja-projektu/sections/frontend/}{chat.tex}
\subimport{chapters/realizacja-projektu/sections/frontend/}{forum.tex}
\setcounter{secnumdepth}{4}
\setcounter{tocdepth}{4}

\subimport{chapters/realizacja-projektu/sections/frontend/}{panel-uzytkownika.tex}

\setcounter{secnumdepth}{3}
\setcounter{tocdepth}{3}
\subimport{chapters/realizacja-projektu/sections/frontend/}{panel-logowania.tex}

%! Author = Mateusz
%! Date = 18/12/2025

\section{Implementacja CI/CD}
\label{sec:implementacja-ci-cd}

W projekcie zastosowano mechanizmy \gls{cicd} z wykorzystaniem \glslink{github-actions}{GitHub Actions}, ponieważ narzędzie to jest
zintegrowane z repozytorium i umożliwia automatyczne uruchamianie procesu budowania oraz testowania po każdej zmianie w kodzie.
W ramach repozytorium przygotowano dwa niezależne \glslink{workflow}{workflow}: dla \glslink{backend}{backendu} (\texttt{Java CI}) oraz
dla \glslink{frontend}{frontendu} (\texttt{React CI}).
Rozdzielenie \glslink{pipeline}{pipeline'ów} pozwoliło ograniczyć liczbę uruchomień tylko do przypadków, gdy modyfikacje
dotyczą danej części systemu.

\subsection{Pipeline backendu}
\label{subsec:pipeline-backendu}

\glslink{workflow}{Workflow} \glslink{backend}{backendu} zdefiniowano w pliku \texttt{java-ci.yml}.
Jest on uruchamiany:
\begin{itemize}
    \item przy zdarzeniu \glslink{push}{\texttt{push}} dla zmian w katalogu \texttt{vulcanus/**},
    \item przy otwarciu \glslink{pull-request}{\textit{pull request}} do gałęzi \texttt{master} oraz \texttt{develop},
    \item w trybie \glslink{merge}{\texttt{merge\_group}} (kolejka scalania) dla gałęzi \texttt{master} oraz \texttt{develop},
\end{itemize}
z jednoczesnym pomijaniem zmian w katalogu \texttt{book/**}.
Fragment konfiguracji wyzwalaczy przedstawiono (rys. \ref{img:cicd-backend-on}).

\paragraph{Wyzwalacze \glslink{workflow}{workflow} (\texttt{on:}).}
W plikach konfiguracyjnych \glslink{github-actions}{GitHub Actions} zastosowano wyzwalacze (\textit{triggery}),
które określają moment uruchomienia \glslink{workflow}{workflow}.
W przypadku zdarzenia \glslink{push}{\texttt{push}} pipeline uruchamia się po wypchnięciu commitów do repozytorium,
przy czym użyto filtru \texttt{paths}, aby wykonywać \glslink{workflow}{workflow} wyłącznie dla zmian w odpowiednich
katalogach (\texttt{vulcanus/**} lub \texttt{venus/**}).
Zdarzenie \glslink{pull-request}{\texttt{pull\_request}} powoduje uruchomienie \glslink{workflow}{workflow} przy otwarciu
\glslink{pull-request}{pull requesta} do gałęzi \texttt{master} lub \texttt{develop}, co umożliwia weryfikację
jakości zmian przed ich scaleniem.
Dodatkowo wykorzystano \texttt{paths-ignore}, aby pomijać uruchomienia związane wyłącznie ze zmianami
w katalogu \texttt{book/**}.
Zdarzenie \glslink{merge}{\texttt{merge\_group}} odpowiada za uruchamianie \glslink{workflow}{workflow} w kontekście kolejki
scalania (\textit{merge queue}) dla gałęzi \texttt{master} oraz \texttt{develop}, co pozwala testować
zmiany w warunkach zbliżonych do rzeczywistego scalenia.

\begin{figure}[H]
    \centering
    \includegraphics[width=1\textwidth]{attachments/implementacja-cicd/github-backend-job1}
    \caption{Fragment pliku \texttt{java-ci.yml}: konfiguracja wyzwalaczy workflow (\texttt{on:}).}
    \label{img:cicd-backend-on}
\end{figure}

\glslink{workflow}{Workflow} składa się z trzech jobów uruchamianych sekwencyjnie: \texttt{setup}, \texttt{build} oraz \texttt{test}.
Zależności pomiędzy \glslink{job}{jobami} (kolejność wykonania) oraz przykładowe czasy trwania
przedstawiono (rys. \ref{img:cicd-backend-run-summary}).

\begin{figure}[H]
    \centering
    \includegraphics[width=1\textwidth]{attachments/implementacja-cicd/backend-summary}
    \caption{Podsumowanie wykonania workflow backendu (\texttt{java-ci.yml}) w GitHub Actions.}
    \label{img:cicd-backend-run-summary}
\end{figure}

\subsubsection{Job \texttt{setup}}
\label{subsubsec:cicd-backend-setup}

\glslink{job}{Job} \texttt{setup} odpowiada za przygotowanie środowiska uruchomieniowego na runnerze:
pobiera repozytorium, ustawia wersję JDK 21, inicjalizuje środowisko \glslink{docker}{Docker} oraz obsługuje
\glslink{cache}{cache} narzędzia \glslink{docker-compose}{Docker Compose} (instalacja wykonywana jest wyłącznie w
przypadku braku trafienia w \glslink{cache}{cache}).
Konfigurację tego joba przedstawiono (rys. \ref{img:cicd-backend-setup-config}), a przykładowy
log wykonania (rys. \ref{img:cicd-backend-setup-log}).

\begin{figure}[H]
    \centering
    \includegraphics[width=1\textwidth]{attachments/implementacja-cicd/backend-job1}
    \caption{Fragment pliku \texttt{java-ci.yml}: definicja joba \texttt{setup}.}
    \label{img:cicd-backend-setup-config}
\end{figure}

\begin{figure}[H]
    \centering
    \includegraphics[width=1\textwidth]{attachments/implementacja-cicd/github-backend-job1}
    \caption{Przykładowy log wykonania joba \texttt{setup} w GitHub Actions.}
    \label{img:cicd-backend-setup-log}
\end{figure}

\subsubsection{Job \texttt{build}}
\label{subsubsec:cicd-backend-build}

\glslink{job}{Job} \texttt{build} jest uruchamiany po poprawnym zakończeniu \texttt{setup}.
W jego ramach:
\begin{itemize}
    \item ustawiana jest wersja JDK 21,
    \item przywracany jest \glslink{cache}{cache} repozytorium \glslink{maven}{Mavena} (przyspieszenie pobierania zależności),
    \item uruchamiane są wymagane usługi poprzez \glslink{docker-compose}{\texttt{docker compose}},
    \item wykonywany jest build aplikacji z pominięciem testów (\texttt{-DskipTests}).
\end{itemize}
Konfigurację joba \texttt{build} przedstawiono (rys. \ref{img:cicd-backend-build-config}), natomiast
przykładowy przebieg wykonania w \glslink{github-actions}{GitHub Actions} (rys. \ref{img:cicd-backend-build-log}).

\begin{figure}[H]
    \centering
    \includegraphics[width=1\textwidth]{attachments/implementacja-cicd/backend-job2}
    \caption{Fragment pliku \texttt{java-ci.yml}: definicja joba \texttt{build}.}
    \label{img:cicd-backend-build-config}
\end{figure}

\begin{figure}[H]
    \centering
    \includegraphics[width=1\textwidth]{attachments/implementacja-cicd/github-backend-job2}
    \caption{Przykładowy log wykonania joba \texttt{build} w GitHub Actions.}
    \label{img:cicd-backend-build-log}
\end{figure}

\subsubsection{Job \texttt{test}}
\label{subsubsec:cicd-backend-test}

\glslink{job}{Job} \texttt{test} jest uruchamiany po zakończeniu \texttt{build}.
W jego ramach przywracany jest \glslink{cache}{cache} \glslink{maven}{Mavena}, a następnie uruchamiane
są testy \glslink{backend}{backendu}.
W \glslink{job}{jobie} wykorzystywane są również zmienne środowiskowe przekazywane z \texttt{secrets} repozytorium
(dane dostępowe do usług zewnętrznych), co pozwala na bezpieczne wykonywanie testów bez umieszczania wrażliwych danych w kodzie.
Konfigurację joba przedstawiono (rys. \ref{img:cicd-backend-test-config}), a przykładowy
log wykonania (rys. \ref{img:cicd-backend-test-log}).

\begin{figure}[H]
    \centering
    \includegraphics[width=1\textwidth]{attachments/implementacja-cicd/backend-job3}
    \caption{Fragment pliku \texttt{java-ci.yml}: definicja joba \texttt{test} (wraz z użyciem \texttt{secrets}).}
    \label{img:cicd-backend-test-config}
\end{figure}

\begin{figure}[H]
    \centering
    \includegraphics[width=1\textwidth]{attachments/implementacja-cicd/github-backend-job3}
    \caption{Przykładowy log wykonania joba \texttt{test} w GitHub Actions.}
    \label{img:cicd-backend-test-log}
\end{figure}

\subsection{Pipeline frontendu}
\label{subsec:pipeline-frontendu}

\glslink{workflow}{Workflow} \glslink{frontend}{frontendu} zdefiniowano w pliku \texttt{react-ci.yml}.
Analogicznie do backendu uruchamiany jest dla zmian w katalogu \texttt{venus/**},
dla otwieranych \glslink{pull-request}{\textit{pull request}} do gałęzi \texttt{master} i \texttt{develop} oraz w trybie
\glslink{merge}{\texttt{merge\_group}}, z pominięciem zmian w \texttt{book/**}.
Fragment konfiguracji (wyzwalacze oraz \glslink{job}{job}) przedstawiono (rys. \ref{img:cicd-frontend-config}).

\begin{figure}[H]
    \centering
    \includegraphics[width=1\textwidth]{attachments/implementacja-cicd/github-frontend-job}
    \caption{Fragment pliku \texttt{react-ci.yml}: konfiguracja workflow oraz joba \texttt{formatting-and-test}.}
    \label{img:cicd-frontend-config}
\end{figure}

\glslink{workflow}{Workflow} \glslink{frontend}{frontendu} zawiera pojedynczy \glslink{job}{job} \texttt{formatting-and-test}.
Najpierw pobierany jest kod i ustawiana jest odpowiednia wersja \glslink{node}{Node.js}, następnie instalowane są zależności.
Kolejnym krokiem jest weryfikacja formatowania kodu (\glslink{prettier}{prettier}), a na końcu uruchamiane są testy
jednostkowe oraz integracyjne.
Podsumowanie przebiegu \glslink{workflow}{workflow} przedstawiono (rys. \ref{img:cicd-frontend-run-summary}),
natomiast przykładowy log wykonania \glslink{job}{joba} (rys. \ref{img:cicd-frontend-log}).

\begin{figure}[H]
    \centering
    \includegraphics[width=1\textwidth]{attachments/implementacja-cicd/frontend-summary}
    \caption{Podsumowanie wykonania workflow frontendu (\texttt{react-ci.yml}) w GitHub Actions.}
    \label{img:cicd-frontend-run-summary}
\end{figure}

\begin{figure}[H]
    \centering
    \includegraphics[width=1\textwidth]{attachments/implementacja-cicd/github-frontend-job}
    \caption{Przykładowy log wykonania joba \texttt{formatting-and-test} w GitHub Actions.}
    \label{img:cicd-frontend-log}
\end{figure}


    %! Author = mateusz
%! Date = 20/09/2025


\chapter{Testy}
\label{ch:testy}

%! Author = Mateusz
%! Date = 13/12/2025

\section{Testy jednostkowe}
\label{sec:testy-jednostkowe}

Do automatycznego uruchamiania testów jednostkowych wykorzystano
\gls{github-actions}, co umożliwiło ich cykliczne wykonywanie w ramach procesu \gls{cicd}
(przy każdym \textit{push} lub \textit{pull request}).
Łącznie przygotowano 273 testy jednostkowe, w tym 211 dla warstwy \glslink{frontend}{frontendowej}
oraz 62 dla warstwy \glslink{backend}{backendowej}.
Wszystkie przygotowane testy zakończyły się powodzeniem (rys.~\ref{fig:unit-tests-frontend}
oraz rys.~\ref{fig:unit-tests-backend-suite}--\ref{fig:unit-tests-backend-summary}).

W warstwie \glslink{frontend}{frontendu} utworzono łącznie 211 testów jednostkowych
(w 23 plikach testowych).
Testy opracowano dla następujących modułów i komponentów:
\begin{itemize}
    \item AddedSpots,
    \item Comments,
    \item FavoriteSpots,
    \item Movies,
    \item Photos (w tym DateChooser oraz SortDropdown),
    \item Profile (ProfileForViewer, UserOwnProfile),
    \item Settings,
    \item Social (SocialCard, SocialForViewer, UserOwnSocial),
    \item Login,
    \item Register,
    \item CurrentViewSpotsList,
    \item SearchedSpotsList,
    \item SearchedSpotsSortingForm,
    \item Sidebar,
    \item SpotDetails,
    \item SpotsNameSearchBar,
    \item UserLocationPanel,
    \item ZoomPanelControl.
\end{itemize}
W testach weryfikowano poprawność renderowania komponentów, obecność i treść kluczowych
elementów interfejsu.
Przykładowy wynik uruchomienia testów jednostkowych frontendu przedstawiono (rys.~\ref{fig:unit-tests-frontend}).

\begin{figure}[H]
    \centering
    \includegraphics[width=1\textwidth]{attachments/testy/unit-frontend}
    \caption{Wynik uruchomienia testów jednostkowych warstwy frontendowej}
    \label{fig:unit-tests-frontend}
\end{figure}

Dla \glslink{backend}{backendu} przygotowano łącznie 62 testy jednostkowe, napisane dla serwisów:
\begin{itemize}
    \item FollowersService,
    \item CommentsService,
    \item MediaService,
    \item AddSpotService,
    \item SettingsService,
    \item ProfileService,
    \item FavoriteSpotService,
    \item FriendsService,
    \item RegisterService.
\end{itemize}
Testy te służyły do potwierdzenia poprawnego działania metod serwisowych, w tym obsługi
przypadków brzegowych oraz walidacji danych wejściowych.
Dodatkowo sprawdzano poprawność współpracy z wybranymi zewnętrznymi interfejsami \gls{api},
przy zachowaniu izolacji logiki aplikacyjnej (poprzez zastępowanie zależności atrapami).
Zestaw uruchomionych testów backendu pokazano (rys.~\ref{fig:unit-tests-backend-suite}),
natomiast podsumowanie ich wykonania przedstawiono (rys.~\ref{fig:unit-tests-backend-summary}).

\begin{figure}[H]
    \centering
    \includegraphics[width=1\textwidth]{attachments/testy/unit-backend}
    \caption{Zestaw testów jednostkowych uruchomionych dla warstwy backendowej}
    \label{fig:unit-tests-backend-suite}
\end{figure}

\begin{figure}[H]
    \centering
    \includegraphics[width=1\textwidth]{attachments/testy/unit-backend-liczba}
    \caption{Podsumowanie uruchomienia testów jednostkowych warstwy backendowej}
    \label{fig:unit-tests-backend-summary}
\end{figure}

%! Author = Mateusz
%! Date = 13/12/2025

\section{Testy integracyjne}
\label{sec:testy-integracyjne}

Testy integracyjne (ang. \textit{integration tests}) to testy automatyczne służące do weryfikacji
poprawnej współpracy kilku modułów lub warstw aplikacji, które w testach jednostkowych były badane
oddzielnie.
Ich celem jest potwierdzenie, że komponenty poprawnie komunikują się ze sobą, przekazują dane, obsługują
zależności oraz że cały fragment funkcjonalności działa spójnie w warunkach zbliżonych do rzeczywistych.

W przeciwieństwie do testów jednostkowych, testy integracyjne zwykle obejmują większy zakres systemu
i mogą wykorzystywać rzeczywiste implementacje wybranych zależności (warstwę dostępu do danych,
konfigurację kontenera zależności lub uruchomienie kontekstu aplikacji), ewentualnie częściowo
zastępowane atrapami w celu kontroli środowiska testowego.
Dzięki temu umożliwiają wykrycie problemów wynikających z integracji (błędnej konfiguracji,
niezgodnych kontraktów lub niepoprawnych interakcji pomiędzy komponentami).

Do automatycznego uruchamiania testów integracyjnych wykorzystano \gls{github-actions},
co umożliwiło ich cykliczne wykonywanie w ramach procesu \gls{cicd}
(przy każdym \textit{push} lub \textit{pull request}).
Łącznie przygotowano 96 testy integracyjne, w tym 52 dla \glslink{frontend}{frontendu}
oraz 44 dla \glslink{backend}{backendu}.
Wszystkie przygotowane testy zakończyły się powodzeniem (rys. \ref{fig:integration-tests-frontend}
oraz rys.~\ref{fig:integration-tests-backend-suite}--\ref{fig:integration-tests-backend-summary}).

W warstwie \glslink{frontend}{frontendu} utworzono łącznie 52 testy integracyjne
(w 11 plikach testowych).
Testy te pozwalały zweryfikować poprawną współpracę wybranych komponentów w ramach większych
fragmentów interfejsu oraz spójność przepływu danych pomiędzy nimi.
Dodatkowo sprawdzano poprawność zmian stanu aplikacji w odpowiedzi na interakcje użytkownika
(kliknięcia) oraz działanie mechanizmu \glslink{infinite-scroll}{\textit{infinite scroll}},
w tym poprawne dociąganie i prezentację kolejnych elementów listy.

Przykładowy wynik uruchomienia testów integracyjnych \glslink{frontend}{frontendu} przedstawiono
(rys. \ref{fig:integration-tests-frontend}).

\begin{figure}[H]
    \centering
    \includegraphics[width=1\textwidth]{attachments/testy/integration-frontned}
    \caption{Wynik uruchomienia testów integracyjnych warstwy frontendowej}
    \label{fig:integration-tests-frontend}
\end{figure}

Dla \glslink{backend}{backendu} przygotowano łącznie 44 testów integracyjnych.
Testy te służyły do potwierdzenia poprawnej współpracy kluczowych warstw aplikacji,
w tym poprawnego uruchomienia kontekstu aplikacji.
Zestaw uruchomionych testów \glslink{backend}{backendu} pokazano (rys. \ref{fig:integration-tests-backend-suite}),
natomiast podsumowanie ich wykonania przedstawiono (rys. \ref{fig:integration-tests-backend-summary}).

\begin{figure}[H]
    \centering
    \includegraphics[width=1\textwidth]{attachments/testy/integration-backend}
    \caption{Zestaw testów integracyjnych uruchomionych dla warstwy backendowej}
    \label{fig:integration-tests-backend-suite}
\end{figure}

\begin{figure}[H]
    \centering
    \includegraphics[width=1\textwidth]{attachments/testy/integration-backend-liczba}
    \caption{Podsumowanie uruchomienia testów integracyjnych warstwy backendowej}
    \label{fig:integration-tests-backend-summary}
\end{figure}

%! Author = Mateusz
%! Date = 13/12/2025

\section{Testy end-to-end (E2E)}
\label{sec:testy-e2e}

Testy end-to-end (E2E) zrealizowano z wykorzystaniem narzędzia Cypress.
W przeciwieństwie do testów jednostkowych i integracyjnych, testy E2E nie były uruchamiane w ramach
\gls{github-actions} (procesu \gls{cicd}), lecz wykonywano je lokalnie w środowisku deweloperskim.
Testy uruchamiano na działającej aplikacji, symulując rzeczywiste działania użytkownika w przeglądarce,
co pozwoliło zweryfikować pełny przepływ od interfejsu \glslink{frontend}{frontendowego}
do warstwy \glslink{backend}{backendowej}.

Łącznie przygotowano 40 testów E2E (w 9 plikach \textit{spec}), a wszystkie zakończyły się powodzeniem
(rys.~\ref{fig:e2e-tests-frontend-summary}).
Zaimplementowane testy stanowią bezpośrednią realizację scenariuszy testowych opisanych w sekcji
\ref{sec:scenariusze-testow-e2e}, obejmujących zarówno przypadki z użyciem danych mockowanych
(poprzez przechwytywanie żądań HTTP), jak i scenariusze wykonywane na rzeczywistym backendzie.

Testy E2E opracowano dla kluczowych obszarów aplikacji:
\begin{itemize}
    \item account (logowanie, rejestracja),
    \item user-dashboard/add-spot (lista, \glslink{infinite-scroll}{\textit{infinite scroll}}, dodawanie miejsca),
    \item user-dashboard/comments (lista, sortowanie),
    \item user-dashboard/favorite-spots (listy, przełączanie typów, \glslink{infinite-scroll}{\textit{infinite scroll}}),
    \item user-dashboard/movies (lista, sortowanie, \glslink{infinite-scroll}{\textit{infinite scroll}}),
    \item user-dashboard/photos (lista, sortowanie, \glslink{infinite-scroll}{\textit{infinite scroll}}),
    \item user-dashboard/profile (widok profilu, nawigacja, akcje społecznościowe),
    \item user-dashboard/settings (edycja danych konta oraz ograniczenia dla kont OAuth),
    \item user-dashboard/social (listy: friends/followed/followers, zaproszenia, \glslink{infinite-scroll}{\textit{infinite scroll}}).
\end{itemize}

W testach weryfikowano poprawność realizacji scenariuszy użytkownika, w tym nawigację pomiędzy widokami,
wykonywanie operacji w interfejsie (kliknięcia i wypełnianie formularzy) oraz poprawną aktualizację
stanu aplikacji po wykonanych akcjach.
Dodatkowo sprawdzano działanie mechanizmu przewijania z dynamicznym doładowywaniem danych
(\glslink{infinite-scroll}{\textit{infinite scroll}}) w warunkach zbliżonych do rzeczywistego użycia aplikacji.

\begin{figure}[H]
    \centering
    \includegraphics[width=1\textwidth]{attachments/testy/e2e}
    \caption{Podsumowanie uruchomienia testów end-to-end (E2E) w Cypress}
    \label{fig:e2e-tests-frontend-summary}
\end{figure}

%! Author = Mateusz
%! Date = 13/12/2025

\section{Wyniki testów i wnioski}
\label{sec:wyniki-testow-i-wnioski}

Przeprowadzone testy jednostkowe, integracyjne oraz end-to-end potwierdziły poprawność działania
kluczowych funkcjonalności aplikacji zarówno po stronie \glslink{frontend}{frontendu}, jak i
\glslink{backend}{backendu}.
Uzyskane wyniki wskazują, że zaimplementowana logika biznesowa, warstwa prezentacji oraz
komunikacja z \gls{api} działają zgodnie z założeniami projektowymi.

Testy jednostkowe pozwoliły zweryfikować poprawność działania pojedynczych komponentów i usług,
w tym obsługę przypadków brzegowych oraz walidację danych wejściowych.
Testy integracyjne umożliwiły potwierdzenie współpracy większych fragmentów systemu, w szczególności przepływu danych
pomiędzy komponentami oraz reakcji aplikacji na działania użytkownika (zmianę stanu po interakcji).
Z kolei testy E2E, uruchamiane w środowisku zbliżonym do rzeczywistego, potwierdziły poprawną realizację
pełnych scenariuszy użytkownika, obejmujących logowanie, edycję ustawień konta, przeglądanie treści
oraz dynamiczne doładowywanie danych (\glslink{infinite-scroll}{\textit{infinite scroll}}).

Na podstawie wyników testów sformułowano następujące wnioski:
\begin{itemize}
    \item Zapewniono wysoką stabilność kluczowych modułów aplikacji dzięki szerokiemu pokryciu testami jednostkowymi.
    \item Mechanizmy odpowiedzialne za paginację i doładowywanie danych (\glslink{infinite-scroll}{\textit{infinite scroll}}) działają poprawnie
    w testowanych widokach oraz nie powodują błędów w interfejsie.
    \item Integracja frontendu z backendem została potwierdzona zarówno w testach integracyjnych, jak i E2E,
    co minimalizuje ryzyko regresji w przypadku dalszego rozwoju aplikacji.
    \item Zastosowanie mockowania zapytań w testach ułatwiło deterministyczne odtwarzanie scenariuszy
    oraz testowanie stanów trudnych do uzyskania w środowisku rzeczywistym (puste listy, konkretne warianty sortowania).
\end{itemize}


    %! Author = mateusz
%! Date = 20/09/2025


\chapter{Prezentacja systemu}
\label{ch:prezentacja-systemu}

%! Author = Mateusz
%! Date = 13/11/2025

\subsection{Strona główna}
\label{subsec:strona-glowna-frontend}

Jednym z głównych modółów aplikacji jest strona główna.
Pełni ona rolę głównej wyszukiwarki spotów, dzięki której użytkownik może w
łatwy sposób znaleść interesujące go lokacje.
Posiadan ona dwa tryby prosty i zaawansowany, dzięki przyciskowi na samej górze strnony można się łatwo
przełączyć między tymi widokami.
Na prostym znajduje się karuzela z 12 najpopularniejszymi spotami w całej aplikacji, użytkownik może tutaj
wyszukać spoty po lokalizacji ( kraj, region, miasto).
Na zaawansowanym widoku jest wyszukiwarka która filtruje po mieście, tagach oraz ocenie dodatkowo
sortuje po popularności i ocenach.

Strona główna została zbudowana z dwóch głównych komponentów HomePaga oraz AdvanceHomePage.
W skład prostej wersji wchodzą następujące komponenty:
\begin{itemize}
  \item Switch - służy do przełączania widoku między trybem podstawowym a zaawansowanym
  \item SearchBar - wyszukiwarka spotów
  \item Carousel - wyświetla najpopularnejsze spoty
  \item SearchSpotList - wyświetla wyszukane spoty
\end{itemize}

W skład zaawansowanej wersji wchodzą następujące komponenty:
\begin{itemize}
    \item Switch - służy do przełączania widoku między trybem podstawowym a zaawansowanym
    \item AdvanceSearchBar - wyszukiwarka spotów
    \item SearchSpotList - wyświetla wyszukane spoty
\end{itemize}

Komponent Switch zawiera w sobie dwa NavLink z biblioteki React Router dzięki temu można przełącznyć widok bez niepotrzebnych
odświeżeń strony.

W koponencie SearchBar po wpisaniu conajmniej 2 znaków pojawi się lista z podpowiedziami
do kraju, regionu oraz miasta w zależności od tego które aktualnie uzupełniamy.
Po pokazaniu się tej listy można wybrać interesujące nas miejsce dzięki czemu
wiemy w jakich lokalizacjach znajdują się spoty.

SearchSpotList zawiera listę komponentów SpotTile, LoadingSpiner oraz komunikat który
wyświetli się jeżeli nie zostanie wyświetlony żaden spot.

Spot tile zawiera informacje takie jak:
\begin{itemize}
    \item Zdjęcie spota
    \item Miasto w którym  się znajduje
    \item Nazwę
    \item Oceny i ich liczbę
    \item Tagi
    \item Podstawowe informacje pogodowe (temperatura i typ)
    \item Dwa przyciski, jeden do przejścia do szczegółów a drugi z informacją jak daleko znajduje
    się dany spot, po kliknięciu pokazuje go na mapie.
\end{itemize}

Komponent AdvanceSearchBar wygląda bardzo podobnie tylko z lokalizacji można podać tylko miasto,
dodatko jest mozżliwość dodania tagów z listy.
Jest też filtrowanie po ocenie oraz sortowanie po ocenie i popularności (komponenty Dropdown).

%\begin{figure}[H]
%    \centering
%    \includegraphics[width=1\textwidth]{attachments/implementacja-strona-glowna1}
%    \caption{Implementacja strony głównej}
%    \label{img:implementacja-strona-glowna1}
%\end{figure}
\subimport{chapters/prezentacja-systemu/sections/}{strona-mapy.tex}
\subimport{chapters/prezentacja-systemu/sections/}{strona-chatu.tex}
\subimport{chapters/prezentacja-systemu/sections/}{strona-forum.tex}
%! Author = mateusz
%! Date = 20/10/2025

\subsection{Panel logowania}
\label{subsec:panel-logowania-frontend}

\subimport{chapters/prezentacja-systemu/sections/}{panel-konta-uzytkownika.tex}
%! Author = Mateusz
%! Date = 19/12/2025

\subsection{Sidebar}
\label{subsec:sidebar-frontend}

\glslink{sidebar}{Sidebar} stanowi główny komponent nawigacyjny aplikacji.
Implementacja obejmuje:
\begin{itemize}
    \item zestaw komponentów interfejsu odpowiedzialnych za renderowanie pozycji, podmenu, tooltipów oraz obsługę
    interakcji użytkownika,
    \item pliki pomocnicze służące do budowania listy linków i rozpoznawania typów elementów nawigacji
    (\texttt{functions}, \texttt{sidebarLinks}),
    \item definicje typów i interfejsów opisujących strukturę pozycji paska bocznego (\texttt{link}),
    \item moduł \texttt{Redux} przechowujący stan rozwinięcia paska (\texttt{sidebar.ts}).
\end{itemize}

\subsubsection{Komponenty}

\textbf{\texttt{Sidebar}} \\
Komponent odpowiada za złożenie całej nawigacji oraz sterowanie jej zachowaniem.
Wykorzystywane są \glslink{hook}{hooki} \texttt{useDarkMode} (obsługa motywu), \newline \texttt{useSelectorTyped}
(odczyt \texttt{isLogged} oraz \texttt{isSidebarOpen}) oraz \texttt{useLocation} (sprawdzenie aktualnej ścieżki).
Na podstawie lokalizacji ustalany jest tryb pozycjonowania: dla stron o układzie „sticky”
(panel użytkownika, czat, strona główna) stosowana jest klasa \texttt{xl:sticky}, w pozostałych przypadkach \texttt{absolute}.
Zestawy linków nawigacyjnych budowane są w pliku \texttt{sidebarLinks} (opis w podrozdz. \ref{subsubsec:sidebar-links-files}).

W strukturze komponentu renderowane są:
\begin{itemize}
    \item \texttt{SidebarToggleButton},
    \item sekcja nawigacyjna: \texttt{SidebarSection} $\rightarrow$ \texttt{SidebarList} (linki główne),
    \item sekcja opcji: \texttt{SidebarSection} $\rightarrow$ \texttt{SidebarList} (akcje i opcje).
\end{itemize}

\textbf{\texttt{Tooltip}} \\
Komponent służy do prezentacji nazwy pozycji oraz (w przypadku submenu) listy pozycji podrzędnych, gdy pasek
boczny pozostaje zwinięty.
Po kliknięciu w element następuje przejście do przypisanej podstrony.

\textbf{\texttt{SidebarToggleButton}} \\
Komponent wyświetla przycisk zwijania/rozwijania paska bocznego.
Po kliknięciu wywoływana jest akcja \texttt{toggleSidebar} z \glslink{redux}{\texttt{Redux}}.
Na mniejszych ekranach (poniżej \texttt{xl} czyli 1280px) prezentowany jest również napis „Merkury”.

\textbf{\texttt{SidebarSection}} \\
Komponent grupuje listy nawigacyjne przekazane przez \texttt{children} oraz opcjonalnie dodaje separator u
góry lub na dole.

\textbf{\texttt{SidebarList}} \\
Komponent renderuje listę elementów nawigacji poprzez mapowanie na \texttt{SidebarItem}.

\textbf{\texttt{SidebarLabel}} \\
Komponent odpowiada za wyświetlenie etykiety pozycji, gdy pasek jest rozwinięty.
Dla płynnej animacji zastosowano \texttt{AnimatePresence} oraz \texttt{motion.p}.
W przypadku \texttt{submenu} z niepustą listą dzieci prezentowana jest dodatkowo ikona
strzałki.

\textbf{\texttt{SidebarIcon}} \\
Komponent odpowiada za prezentację ikony danej pozycji.
W przypadku zwiniętego paska i pozycji typu \texttt{submenu} ikona stanowi \texttt{NavLink}
(umożliwia przejście do strony nadrzędnej), natomiast dla \texttt{link} i \texttt{action}
zwracana jest sama ikona.

\textbf{\texttt{SidebarItemSubmenuLink}} \\
Komponent renderuje elementy podrzędne submenu jako \texttt{NavLink}.
Przy każdym kliknięciu wywoływana jest akcja \texttt{closeSidebar}, która zwija pasek wyłącznie na
ekranach o szerokości mniejszej niż \texttt{1280px}.

\textbf{\texttt{SidebarItemSubmenu}} \\
Komponent odpowiada za obsługę pozycji typu \texttt{submenu}.
Zastosowano:
\begin{itemize}
    \item \texttt{useToggleState} do przełączania stanu rozwinięcia,
    \item \texttt{useState} do przechowywania nazwy aktualnie otwartego submenu (w celu utrzymania zasady „jedno submenu otwarte naraz”),
    \item \texttt{useLocation} oraz \texttt{useEffect} do automatycznego rozwinięcia submenu, gdy aktywna jest jedna z podstron potomnych.
\end{itemize}
Pozycja submenu renderowana jest jako przycisk zawierający \texttt{SidebarItemContent}, a lista elementów potomnych
pojawia się animacyjnie (przez \texttt{AnimatePresence} i \texttt{motion.div}) tylko wtedy, gdy pasek jest rozwinięty.

\textbf{\texttt{SidebarItemLink}} \\
Komponent renderuje pojedynczą pozycję typu \texttt{link} jako \texttt{NavLink} zawierający \newline \texttt{SidebarItemContent}.
Dodatkowo, po kliknięciu wywoływana jest akcja \texttt{closeSidebar} (działająca tylko dla szerokości poniżej
\texttt{1280px}).

\textbf{\texttt{SidebarItemContent}} \\
Komponent stanowi wspólne „wnętrze” elementów \texttt{SidebarItem}.
Zawiera \texttt{SidebarIcon}, warunkowo \texttt{Tooltip} (gdy pasek jest zwinięty i aktywny jest stan tooltipa)
oraz \texttt{SidebarLabel}.

\textbf{\texttt{SidebarItemAction}} \\
Komponent obsługuje pozycje typu \texttt{action}, które nie prowadzą do \glslink{routing}{routingu}, lecz uruchamiają funkcje.
Zaimplementowano akcję wylogowania (wywołanie \texttt{logout} oraz aktualizacja stanu konta)
oraz przełączanie motywu poprzez \texttt{onChangeTheme}.
Renderowany jest przycisk zawierający \texttt{SidebarItemContent}.

\textbf{\texttt{SidebarItem}} \\
Komponent wybiera właściwy podtyp elementu na podstawie pola \texttt{type}.
Wykorzystywany jest \glslink{hook}{hook} \texttt{useBoolean} do sterowania widocznością tooltipa oraz \texttt{useLocation} i
\texttt{useEffect} do ukrycia tooltipa po zmianie ścieżki.
Dobór implementacji realizowany jest przez instrukcję \texttt{switch}:
\texttt{submenu} $\rightarrow$ \texttt{SidebarItemSubmenu}, \texttt{action} $\rightarrow$ \texttt{SidebarItemAction},
\texttt{link} $\rightarrow$ \texttt{SidebarItemLink}.

\subsubsection{Pliki pomocnicze}
\label{subsubsec:sidebar-links-files}

\textbf{\texttt{functions}} \\
Plik zawiera funkcje typu \emph{type guard} umożliwiające rozróżnienie wariantów linków:
\texttt{isSidebarSubmenu}, \texttt{isSidebarAction} oraz \texttt{isSidebarLink}.

\textbf{\texttt{sidebarLinks}} \\
Plik definiuje strukturę nawigacji w postaci obiektów:
\texttt{staticLinks}, \texttt{userLoggedLinks} oraz \texttt{getOptionsLinks}.
Zależnie od stanu zalogowania (\texttt{isLogged}) generowane są różne zestawy pozycji:
\begin{itemize}
    \item dla użytkownika niezalogowanego: strona główna, mapa oraz forum (strona główna forum, regulamin),
    \item dla użytkownika zalogowanego: dodatkowo lista obserwowanych postów na forum, czat oraz panel użytkownika
    \item (submenu „account” z podstronami profilu, ustawień itd.).
\end{itemize}
Sekcja opcji (\texttt{getOptionsLinks}) zawiera:
\begin{itemize}
    \item pozycję logowania lub wylogowania (zależnie od \texttt{isLogged}),
    \item przełączenie motywu jasny/ciemny (zależnie od \texttt{isDark}).
\end{itemize}

\textbf{\texttt{link}} \\
Plik z typami definiuje \texttt{LinkType} oraz interfejsy: \texttt{BaseLink}, \texttt{SidebarLink},
\newline \texttt{SidebarSubmenuLink}, \texttt{SidebarSubmenu}, \texttt{SidebarAction}.

\subsubsection{Redux}

\glslink{stan}{Stan} paska bocznego utrzymywany jest w module \texttt{Redux} w pliku \texttt{sidebar.ts}.
Przechowywana jest wyłącznie flaga \texttt{isOpen} informująca o stanie rozwinięcia.
Zdefiniowano:
\begin{itemize}
    \item reducer \texttt{setIsSidebarOpen} (jawne ustawienie wartości),
    \item reducer \texttt{toggleSidebar} (przełączenie stanu),
    \item thunk \texttt{closeSidebar}, który zwija pasek wyłącznie dla szerokości mniejszej niż \texttt{1280px}.
\end{itemize}
Rozwiązanie to umożliwia automatyczne zwijanie paska po kliknięciu w link na urządzeniach mobilnych, bez wpływu na
zachowanie na dużych ekranach.

\subsubsection{Użycie w układzie aplikacji}

\textbf{\texttt{Layout}} \\
Stanowi szkielet widoków aplikacji: renderuje \texttt{Sidebar} oraz część
główną (\texttt{main}) zawierającą \texttt{MobileBar}, listę powiadomień oraz \texttt{Outlet}
(miejsce wstrzyknięcia aktualnej podstrony przez router).
Dla strony mapy ustawiany jest układ \texttt{relative}, a dla pozostałych widoków układ elastyczny (\texttt{flex}).
Dodatkowo, w \texttt{useEffect} wykrywana jest nawigacja do ścieżek panelu konta; na ekranach o szerokości
co najmniej \texttt{1280px} pasek boczny jest wówczas automatycznie rozwijany poprzez
\newline \texttt{dispatch(sidebarAction.setIsSidebarOpen(true))}.

\textbf{\texttt{MobileBar}} \\
Wyświetlany jest wyłącznie poniżej progu \texttt{xl} (1280px) (klasa \texttt{xl:hidden}).
Stanowi górny pasek nawigacyjny na urządzeniach mobilnych: zawiera przycisk z ikoną menu, który przełącza
\glslink{stan}{stan} paska bocznego (\texttt{toggleSidebar}), oraz tytuł aplikacji „Merkury”.
Dzięki umieszczeniu w części głównej (\texttt{main}) komponent pozostaje widoczny nad zawartością strony,
jednocześnie nie wpływając na układ desktopowy.

\input{chapters/prezentacja-systemu/sections/nottification}

    %! Author = Mateusz Redosz
%! Date = 12/10/2025


\chapter{Nakład pracy}
\label{ch:naklad-pracy}

\section{Indywidualne nakłady pracy}
\label{sec:indywidualne-naklady-pracy}

%! Author = Adam
%! Date = 02/02/2026

\subsection{Adam Langmesser}
\label{subsec:adam-langmesser}

Na prace dotyczące projektu poświęciłem łącznie 608 godzin, w tym:
\begin{itemize}
    \item 34 godz. \textendash \space zaprojektowanie \glslink{ui}{UI} modułu mapy w Figmie
    \item 293 godz. \textendash \space prace deweloperskie
    \item 215 godz. \textendash \space tworzenie dokumentacji
    \item 66 godz. \textendash \space przeprowadzanie \glslink{review-kodu}{review kodu}
\end{itemize}

\subsubsection{Rola lidera}
\label{subsubsec:rola-lidera}

Jako lider zespołu byłem odpowiedzialny za koordynację prac, podział zadań oraz kontrolę postępów realizacji projektu.
Ponadto wypracowałem procedury włączania nowego kodu do repozytorium, obejmującą \glslink{review-kodu}{review kodu}.

\subsubsection{Prace deweloperskie}
\label{subsubsec:prace-deweloperskie-adam}

%TODO

Przygotowałem lokalne środowisko deweloperskie dla \glslink{backend}{backendu} i \glslink{frontend}{frontendu} (konfiguracja \gls{ide},
weryfikacja narzędzi budujących oraz uruchomienie projektów testowych), co ułatwiło płynne rozpoczęcie właściwych prac implementacyjnych.
Następnie wdrożyłem podstawową stronę powitalną po stronie \glslink{frontend}{frontendu} oraz zaimplementowałem fundamenty logiki logowania
i rejestracji użytkownika w \glslink{backend}{backendzie}, uwzględniając scenariusze brzegowe, takie jak próba rejestracji z zajętą nazwą użytkownika.
Rozszerzyłem też środowisko uruchomieniowe o konfigurację \gls{docker-compose} dla \glslink{backend}{backendu}
i bazy danych oraz dodałem cykliczny test weryfikujący poprawność uruchamiania aplikacji w kontenerach i jej podstawową responsywność.
W dalszej kolejności zaprezentowałem zespołowi przykładowe podejście do \glslink{integration-tests}{testów integracyjnych}
oraz \glslink{e2e-tests}{testów E2E} na \glslink{backend}{backendzie}, a równolegle przygotowałem demonstracyjny prototyp mapy
z wykorzystaniem biblioteki \gls{leaflet}, aby pokazać możliwości interaktywnego widoku mapowego,
oraz dopracowałem konfigurację \gls{eslint} po stronie \glslink{frontend}{frontendu}.
Kontynuowałem prace w obszarze bezpieczeństwa, doprecyzowując konfigurację \gls{spring-security} (role i uprawnienia),
przygotowując dane deweloperskie użytkowników oraz dodając automatyczne testy obejmujące kluczowe scenariusze logowania i rejestracji,
a następnie ustabilizowałem zestaw testów poprzez ujednolicenie asercji i rozwiązanie problemów z relacjami między
obiektami wykorzystywanymi w testach integracyjnych. Wprowadziłem kolejne
usprawnienia konfiguracji \gls{spring-security} (reguły autoryzacji i filtry),
uporządkowałem plik \texttt{.gitignore} o nowe artefakty generowane przez narzędzia
oraz poprawiłem działanie potoków \gls{cicd} dla \glslink{backend}{backendu};
jednocześnie uporządkowałem strukturę projektu po stronie serwera,
wdrożyłem mechanizm cachowania z użyciem \gls{redis}
oraz doprecyzowałem konfigurację poziomów logowania błędów,
co usprawniło diagnozowanie problemów w środowiskach deweloperskich.
Równolegle rozpocząłem implementację modułu czatu zarówno po stronie \glslink{backend}{backendowej},
jak i \glslink{frontend}{frontendowej}, dostosowując układ aplikacji oraz elementy nawigacyjne do nowej sekcji,
a także rozwinąłem procesy \gls{cicd} dla serwera, optymalizując czas budowania i doprecyzowując kroki uruchamiania testów.
Następnie znacząco rozbudowałem czat, dodając pobieranie danych i wysyłanie wiadomości z użyciem \gls{websocket},
ujednolicając model danych, upraszczając komunikację z \gls{api} oraz przebudowując strukturę \gls{redux} i komponentów,
co przełożyło się na czytelniejszy kod i spójniejsze działanie interfejsu; dodatkowo wdrożyłem usprawnienia narzędziowe
w postaci formatowania przez \gls{prettier}, skryptu \texttt{format:check} oraz weryfikacji formatowania w \gls{cicd}.
Usprawniłem mechanizm \glslink{infinite-scroll}{nieskończonego przewijania} listy czatów, przygotowałem wspólną logikę
i komponenty pomocnicze ułatwiające zespołowi korzystanie z \glslink{websocket}{websocketów} na froncie,
dodałem grupowanie wiadomości po dacie oraz obsługę wielowierszowego pola tekstowego,
a także wprowadziłem drobne korekty w potokach \gls{cicd}. Kolejnym krokiem było rozszerzenie czatu o wyszukiwanie
i wysyłanie \glslink{gif}{GIF-ów} oraz wysyłanie \gls{emoji}, a także usprawnienie komunikacji poprzez optymistyczne
wysyłanie wiadomości i dodanie nowych \glslink{endpoint}{endpointów} do stronicowanego pobierania starszych wiadomości.
Na koniec umożliwiłem rozpoczynanie i kontynuowanie prywatnych rozmów bezpośrednio z widoków społecznościowych
(list znajomych oraz obserwujących), dopracowałem obsługę \gls{emoji} i ergonomię okna wyboru \glslink{gif}{GIF-ów},
a także zaimplementowałem funkcjonalność czatów grupowych obejmującą tworzenie rozmów z wyborem uczestników,
edycję nazwy i obrazu czatu, dodawanie nowych członków oraz pełną obsługę załączników (pliki i obrazy) wraz z logiką wyboru,
podglądu i wysyłania plików również bez treści tekstowej.


\subsubsection{Tworzenie dokumentacji}
\label{subsubsec:tworzenie-dokumentacji-adam}

W ramach tworzenia dokumentacji naszej pracy inżynierskiej byłem odpowiedzialny za napisanie następujących rozdziałów oraz podrozdziałów:
\begin{itemize}
    \item \hyperref[sec:udzialowcy]{\ref*{sec:udzialowcy} Udziałowcy}
    \item \hyperref[sec:metodologia-pracy]{\ref*{sec:metodologia-pracy} Metodologia pracy}
    \item \hyperref[sec:harmonogram-projektu]{\ref*{sec:harmonogram-projektu} Harmonogram projektu}
    \item \hyperref[sec:zasoby-i-ograniczenia]{\ref*{sec:zasoby-i-ograniczenia} Zasoby i ograniczenia}
    \item \hyperref[sec:przypadki-uzycia]{\ref*{sec:przypadki-uzycia} Przypadki użycia}
    \item \hyperref[subsubsec:wymagania-ogolne-dla-chatu]{\ref*{subsubsec:wymagania-ogolne-dla-chatu} Wymagania ogólne dla czatu}
    \item \hyperref[subsubsec:wymagania-funkcjonalne-dla-chatu]{\ref*{subsubsec:wymagania-funkcjonalne-dla-chatu} Wymagania funkcjonalne dla czatu}
    \item \hyperref[subsubsec:wymagania-pozafunkcjonalne-dla-czatu]{\ref*{subsubsec:wymagania-pozafunkcjonalne-dla-czatu} Wymagania pozafunkcjonalne dla czatu}
    \item \hyperref[subsec:projekt-chatu]{\ref*{subsec:projekt-chatu} Projekt czatu}
    \item \hyperref[ch:przebieg-realizacji-projektu]{\ref*{ch:przebieg-realizacji-projektu} Przebieg realizacji projektu}
    \item \hyperref[subsubsec:chain-of-responsibility]{\ref*{subsubsec:chain-of-responsibility} Chain of Responsibility}
    \item \hyperref[subsec:bezpieczenstwo-aplikacji]{\ref*{subsec:bezpieczenstwo-aplikacji} Bezpieczeństwo aplikacji}
    \item TODO: Integracja z Tenor na backendzie
    \item TODO: czat frontend
    \item \hyperref[sec:implementacja-websocket]{\ref*{sec:implementacja-websocket} Implementacja WebSocket}
    \item \hyperref[sec:strona-chatu]{\ref*{sec:strona-chatu} Strona czatu}
    \item \hyperref[subsec:adam-langmesser]{\ref*{subsec:adam-langmesser} Adam Langmesser}
\end{itemize}

%! Author = Mateusz
%! Date = 12/10/2025

\subsection{Mateusz Redosz}
\label{subsec:mateusz-redosz}

Na realizację projektu poświęciłem łącznie XXX godzin, z czego 256 przeznaczyłem na prace deweloperskie,
XXX na przygotowanie dokumentacji, XX godzin na \glslink{review-kodu}{review kodu}, XX na spotkania dotyczące
omówienia dalszych prac projektowych oraz pomocy innym członkom zespołu, a także 49 godzin na przygotowanie
widoków w Figmie.
Prace nad częścią deweloperską rozpocząłem 04.08.2024, a zakończyłem 08.09.2025.

W ramach projektu odpowiadałem za implementację rejestracji użytkownika oraz częściową obsługę tokenu \gls{jwt}.
Pracowałem również nad wybranymi elementami \glslink{pipeline}{pipeline’u} \gls{cicd}, wspierając przygotowanie
automatyzacji budowania oraz testowania aplikacji.
Dużą część czasu poświęciłem na rozwój głównych modułów systemu, w szczególności wyszukiwarki spotów oraz panelu
użytkownika, zarówno na \glslink{frontend}{frontendzie}, jak i \glslink{backend}{backendzie}.
Oba moduły są responsywne na dużych i małych ekranach oraz zostały przygotowane w motywie jasnym i ciemnym.
Panel użytkownika przetestowałem testami jednostkowymi, integracyjnymi oraz E2E.

Dodatkowo zaimplementowałem \glslink{sidebar}{sidebar}, dbając o \glslink{responsywnosc}{responsywność} i
dopasowanie układu do różnych rozdzielczości, tak aby korzystanie z aplikacji było wygodne zarówno na
urządzeniach mobilnych, jak i na komputerach.
Na początku przedmiotu \gls{prz1} przygotowałem w Figmie projekt widoków dla panelu użytkownika,
wyszukiwarki spotów oraz panelu logowania i rejestracji, dbając o spójność stylu, czytelność interfejsu
oraz zgodność z założeniami funkcjonalnymi.

W trakcie prac regularnie konsultowałem postępy z zespołem, brałem udział w spotkaniach oraz pomagałem
innym osobom w rozwiązywaniu problemów związanych z implementacją i konfiguracją środowiska.
Po wykonaniu kluczowych funkcjonalności wprowadzałem poprawki wynikające z testów oraz uwag z \gls{review-kodu},
skupiając się na jakości, stabilności oraz utrzymaniu jednolitego standardu kodu w projekcie.

Przy pisaniu pracy dyplomowej odpowiadałem za następujące rozdziały:
-

%! Author = mateusz
%! Date = 12/10/2025

\subsection{Stanisław Oziemczuk}
\label{subsec:stanislaw-oziemczuk}
%! Author = kacper
%! Date = 12/10/2025

\subsection{Kacper Badek}
\label{subsec:kacper-badek}

Na prace związane z realizacją projektu poświęciłem łącznie X godzin, w tym:
\begin{itemize}
    \item 29 godzin \textendash \space zaprojektowanie interfejsu użytkownika modułu forum w figmie
    \item B godzin \textendash \space prace deweloperskie
    \item C godzin \textendash \space tworzenie dokumentacji
    \item D godzin \textendash \space udział w spotkaniach zespołu projektowego oraz pomoc w rozwiązywaniu problemów związanych z realizowanymi zadaniami
    \item E godzin \textendash \space przeprowadzanie \glslink{review-kodu}{review kodu}
\end{itemize}

\subsubsection{Prace deweloperskie}

Jednym z moich pierwszych zadań było przygotowanie formularza logowania użytkownika po stronie \glslink{frontend}{frontendu} oraz jego integracja z \glslink{backend}{backendowym} API odpowiedzialnym za uwierzytelnianie użytkowników.
Następnie zaimplementowałem mechanizm resetowania hasła, obejmujący generowanie oraz weryfikację tokenów, formularz zmiany hasła, a także logikę cyklicznego usuwania przeterminowanych tokenów.
W dalszym etapie poprawiłem obsługę błędów po stronie \glslink{backend}{backendu} oraz dostosowałem treść i wygląd wiadomości e-mail wysyłanych przez system, dbając o ich czytelność i spójność wizualną.

Istotnym elementem projektu był również rozwój systemu mailowego.
Zmieniłem sposób wysyłania wiadomości po rejestracji użytkownika na asynchroniczny oraz dodałem mechanizm ponawiania prób wysyłania wiadomości e-maili z ograniczoną liczbą prób i logowaniem nieudanych operacji.
Przeprowadziłem także refaktoryzację szablonów wiadomości związanych z rejestracją i resetowaniem hasła, ujednolicając strukturę HTML oraz styl graficzny, w tym logo aplikacji.

W ramach prac nad modułem mapy przygotowałem dane deweloperskie, takie jak przykładowe \glslink{spot}{spoty} wykorzystywane do prezentacji i testów.
Zaimplementowałem funkcjonalność dodawania \glslink{spot}{spotów} do listy ulubionych po stronie \glslink{backend}{backendu} i \glslink{frontend}{frontendu}, a także widok umożliwiający użytkownikowi przeglądanie zapisanych lokalizacji.

Moją największą częścią projektu była praca nad modułem forum.
Rozpocząłem ją od przygotowania projektu interfejsu użytkownika w figmie (w ramach przedmiotu \glslink{pro}{PRO}), a następnie zaimplementowałem \glslink{backend}{backend} i \glslink{frontend}{frontend} obejmujący obsługę postów, kategorii, tagów oraz paginacji.

Równolegle do rozwoju forum zintegrowałem \glslink{backend}{backend} z usługą \glslink{azure-blob-storage}{Azure Blob Storage}, przeznaczoną do uploadu i przechowywania plików multimedialnych wykorzystywanych w całym systemie.

Przygotowałem formularz dodawania postów z wykorzystaniem edytora rich text TinyMCE, wraz z walidacją danych po stronie \glslink{backend}{backendu} i \glslink{frontend}{frontendu}.
W celu zabezpieczenia aplikacji skonfigurowałem bibliotekę jsoup do walidacji przesyłanej treści HTML, ograniczając dozwolone elementy i style.
W późniejszym etapie zdecydowałem się na zmianę edytora na Tiptap, który został skonfigurowany od podstaw, umożliwiając pełną kontrolę nad dostępnymi funkcjonalnościami edycji treści.

Forum zostało rozbudowane o mechanizmy sortowania oraz paginacji w formie infinite scroll, czytelne adresy \glslink{url}{URL} oparte na slugach tytułów oraz usprawnioną nawigację.
Wprowadzono możliwość edycji postów, a komentarze rozszerzono o funkcje dodawania, edycji i odpowiadania.
Dodatkowo zaimplementowano system głosowania i zgłaszania treści, dostępny zarówno dla wpisów, jak i komentarzy, a także funkcjonalność obserwowania wybranych dyskusji przez użytkowników.

W celu poprawy doświadczenia użytkownika dodałem struktury typu skeleton do ładowania danych, formularze dostępne globalnie poprzez stan aplikacji oraz pełne wsparcie dla jasnego i ciemnego motywu kolorystycznego.
Forum zostało również wyposażone w wyszukiwarkę umożliwiającą filtrowanie postów według tytułu, kategorii, tagów, zakresu dat oraz autora, a także stronę prezentującą wyniki wyszukiwania wraz z informacją o liczbie znalezionych elementów.

Dodatkowo przygotowałem strony listujące wszystkie kategorie i tagi alfabetycznie, stronę regulaminu forum, widok postów obserwowanych przez użytkownika oraz sekcję prezentującą najpopularniejsze posty z ubiegłego miesiąca.
Całość została zaprojektowana z myślą o czytelności, spójności wizualnej oraz wygodnej nawigacji.

Wyżej opisane zadania wymagały prac zarówno po stronie \glslink{frontend}{frontendu}, jak i \glslink{backend}{backendu}.

\subsubsection{Tworzenie dokumentacji}

W ramach tworzenia dokumentacji pracy inżynierskiej byłem odpowiedzialny za opracowanie następujących rozdziałów i podrozdziałów:
\begin{itemize}
    \item \hyperref[ch:wstep]{\ref*{ch:wstep} Wstęp}
    \item \hyperref[ch:opis-problemu]{\ref*{ch:opis-problemu} Opis problemu}(z wyjątkiem opisu udziałowców)
    \item \hyperref[sec:analiza-ryzyka]{\ref*{sec:analiza-ryzyka} Analiza ryzyka}
    \item Wymagania ogólne dla forum
    \item Wymagania funkcjonalne dla forum
    \item Wymagania pozafunkcjonalne dla forum
    \item Implementacja frontendu forum
    \item Projekt forum
    \item Implementacja systemu mailowego
    \item \hyperref[sec:strona-forum]{\ref*{sec:strona-forum} Prezentacja systemu forum}
    \item \hyperref[subsec:kacper-badek]{\ref*{subsec:kacper-badek} ninijeszy rozdział}
\end{itemize}

Podczas opracowywania powyższych treści uzupełniłem również \texttt{Słownika pojęć i skrótów} oraz \texttt{Bibliografię} o niezbędne pozycje.

    %! Author = Mateusz Redosz
%! Date = 20/09/2025


\chapter{Podsumowanie}
\label{ch:podsumowanie}

%! Author = mateusz
%! Date = 17/10/2025

\section{Osiągnięte rezultaty}
\label{sec:osiagniete-rezultaty}
%! Author = mateusz
%! Date = 17/10/2025

\section{Napotkane wyzwania}
\label{sec:napotkane wyzwania}
%! Author = mateusz
%! Date = 17/10/2025

\section{Plany na przyszłość}
\label{sec:plany-na-przyszloasc}


    \printbibliography[title={Bibliografia}, heading=bibintoc]
    \makethesisattachments

\end{document}