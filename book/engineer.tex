\documentclass{sprz}

\usepackage[hidelinks]{hyperref}
\usepackage{glossaries}
\usepackage[polish]{babel}


\makeglossaries
\addbibresource{bibliography.bib}

%! Author = Mateusz Redosz
%! Date = 20/09/2025

% Słownik pojęć

\newglossaryentry{backend}
{
    name={Backend},
    description={Część aplikacji odpowiedzialna za logikę biznesową, przetwarzanie danych i komunikację z bazą danych. Działa po stronie serwera i obsługuje żądania wysyłane przez frontend}
}

\newglossaryentry{frontend}
{
    name={Frontend},
    description={Warstwa aplikacji odpowiedzialna za interfejs użytkownika oraz interakcję z użytkownikiem. Zazwyczaj tworzona przy użyciu technologii takich jak HTML, CSS i JavaScript}
}

\newglossaryentry{baza-danych}
{
    name={Baza danych},
    description={Zbiór uporządkowanych danych przechowywanych w sposób umożliwiający ich łatwe wyszukiwanie, modyfikowanie i analizowanie. W aplikacjach najczęściej wykorzystywane są relacyjne lub nierelacyjne bazy danych}
}

\newglossaryentry{framework}
{
    name={Framework},
    description={Zestaw narzędzi, bibliotek i struktur wspomagających tworzenie aplikacji. Ułatwia programowanie poprzez dostarczenie gotowych komponentów oraz określenie zasad organizacji kodu}
}

\newglossaryentry{review-kodu}
{
    name={Review kodu},
    description={Proces polegający na wzajemnym przeglądzie kodu źródłowego przez programistów w celu wykrycia błędów, poprawy jakości oraz zwiększenia spójności projektu}
}

\newglossaryentry{jwt}
{
    name={JWT},
    description={Skrót od \textit{JSON Web Token}. Standard służący do bezpiecznego przekazywania informacji między stronami w formacie JSON, często używany w procesach autoryzacji użytkowników}
}

\newglossaryentry{oauth}
{
    name={OAuth},
    description={Standard autoryzacji umożliwiający aplikacjom zewnętrznym uzyskanie dostępu do zasobów użytkownika bez przekazywania jego hasła, często wykorzystywany przy logowaniu za pomocą dostawców takich jak Google czy GitHub}
}

\newglossaryentry{cicd}
{
    name={CI/CD},
    description={Skrót od \textit{Continuous Integration/Continuous Deployment}. Praktyka programistyczna polegająca na automatyzacji procesu budowania, testowania i wdrażania oprogramowania}
}

\newglossaryentry{github}
{
    name={GitHub},
    description={Platforma hostingu repozytoriów \textit{Git} w chmurze, oferująca m.in. pull requesty, system zgłoszeń (issues), zarządzanie wersjami oraz integrację z narzędziami CI/CD}
}

\newglossaryentry{jira}
{
    name={Jira},
    description={Narzędzie firmy Atlassian do zarządzania projektami i zadaniami, szeroko stosowane w metodykach zwinnych. Umożliwia pracę z epikami, taskami, podtaskami oraz tablicami Scrum i Kanban}
}

\newglossaryentry{spring-boot}
{
    name={Spring Boot},
    description={Framework w ekosystemie Spring dla języka Java, ułatwiający tworzenie aplikacji backendowych dzięki automatycznej konfiguracji, wbudowanemu serwerowi aplikacyjnemu oraz zestawowi gotowych starterów}
}

\newglossaryentry{spring-security}
{
    name={Spring Security},
    description={Moduł bezpieczeństwa w ekosystemie Spring odpowiedzialny za uwierzytelnianie i autoryzację użytkowników. Zapewnia obsługę różnych mechanizmów logowania, ról i uprawnień oraz integrację z różnymi źródłami danych}
}

\newglossaryentry{docker}
{
    name={Docker},
    description={Platforma do konteneryzacji aplikacji. Pozwala uruchamiać oprogramowanie w lekkich, izolowanych kontenerach tworzonych na podstawie obrazów, co upraszcza wdrażanie i utrzymanie spójnego środowiska}
}

\newglossaryentry{cors}
{
    name={CORS},
    description={Skrót od \textit{Cross-Origin Resource Sharing}. Mechanizm bezpieczeństwa w przeglądarkach, który kontroluje, czy aplikacja z jednej domeny może wykonywać zapytania HTTP do serwera w innej domenie; konfigurowany za pomocą nagłówków HTTP}
}

\newglossaryentry{http-only-cookie}
{
    name={Ciasteczko HttpOnly},
    description={Ciasteczko HTTP ustawione z flagą \texttt{HttpOnly}, dzięki czemu nie jest dostępne z poziomu JavaScriptu. Zmniejsza ryzyko kradzieży tokenów (np. JWT) w przypadku ataków typu XSS}
}

\newglossaryentry{tailwind-css}
{
    name={Tailwind CSS},
    description={Framework CSS typu \textit{utility-first}, dostarczający gotowe klasy narzędziowe do określania wyglądu (kolory, odstępy, layout). Umożliwia szybkie prototypowanie i spójne stylowanie komponentów bez pisania rozbudowanych arkuszy CSS}
}

\newglossaryentry{prettier}
{
    name={Prettier},
    description={Narzędzie do automatycznego formatowania kodu (np. JavaScript, TypeScript, CSS, HTML). Narzuca spójny styl formatowania, zastępując ręczne ustawianie wcięć i łamań linii}
}

\newglossaryentry{eslint}
{
    name={ESLint},
    description={Statyczny analizator kodu JavaScript/TypeScript. Umożliwia wykrywanie błędów, niespójności stylu oraz potencjalnych problemów poprzez zestaw reguł, które można dostosować do projektu}
}

\newglossaryentry{tanstack-query}
{
    name={TanStack Query},
    description={Biblioteka do obsługi zapytań do serwera i cachowania danych w aplikacjach frontendowych (m.in. React). Ułatwia zarządzanie stanem danych z backendu: pobieranie, odświeżanie, invalidację i obsługę błędów}
}

\newglossaryentry{leaflet}
{
    name={Leaflet},
    description={Lekka biblioteka JavaScript do tworzenia interaktywnych map w przeglądarce, często używana z danymi z OpenStreetMap. Umożliwia dodawanie znaczników, warstw oraz obsługę interakcji użytkownika}
}

\newglossaryentry{e2e-tests}
{
    name={Testy E2E},
    description={Testy \textit{end-to-end}, które sprawdzają działanie systemu od strony użytkownika, przechodząc przez wszystkie warstwy aplikacji (frontend, backend, baza danych) i symulując rzeczywiste scenariusze użycia}
}

\newglossaryentry{dto}
{
    name={DTO},
    description={Skrót od \textit{Data Transfer Object}. Prosty obiekt przenoszący dane między warstwami systemu lub między usługami. Zawiera pola danych, zazwyczaj bez logiki biznesowej}
}

\newglossaryentry{modal}
{
    name={Modal},
    description={Okno dialogowe (okno modalne), które pojawia się na wierzchu interfejsu i blokuje interakcję z resztą aplikacji, dopóki użytkownik go nie zamknie. Służy do prezentowania ważnych komunikatów lub formularzy}
}

\newglossaryentry{skeleton-loader}
{
    name={Skeleton loader},
    description={Wzorzec prezentowania stanu ładowania, w którym zamiast klasycznego „spinnera” wyświetlane są szare prostokąty imitujące docelowy układ treści. Poprawia subiektywne odczucie szybkości działania aplikacji}
}

\newglossaryentry{z-index}
{
    name={z-index},
    description={Właściwość CSS określająca kolejność nakładania się elementów (oś Z). Wyższa wartość powoduje wyświetlenie elementu „nad” elementami o niższych wartościach}
}

\newglossaryentry{intersection-observer}
{
    name={Intersection Observer},
    description={API przeglądarkowe umożliwiające reagowanie na momenty, gdy dany element pojawia się w polu widzenia użytkownika (viewport) lub opuszcza je. Wykorzystywane m.in. do implementacji \gls{infinite-scroll} i lazy loadingu}
}

\newglossaryentry{latex}
{
    name={LaTeX},
    description={System składu tekstu wykorzystywany do przygotowywania profesjonalnych dokumentów technicznych i naukowych. Umożliwia precyzyjne formatowanie, zarządzanie odwołaniami, bibliografią i wzorami matematycznymi}
}

\newglossaryentry{commit}
{
    name={Commit},
    @@ -60,7 +180,7 @@
\newglossaryentry{spot}
{
    name={Spot},
    description={Potencjalne miejsce do latania dronem, zaznaczone na mapie.}
}

\newglossaryentry{sidebar}
@@ -300,7 +420,197 @@
description={(ang. \textit{Business Process Model and Notation});
standardowa notacja graficzna, która umożliwia szczegółowe przedstawienie i dokumentowanie procesów biznesowych.}
}

\newglossaryentry{infinite-scroll}{
    name={Infinite scroll},
    description={Wzorzec interfejsu użytkownika, w którym kolejne porcje treści są automatycznie doładowywane podczas przewijania strony w dół, zamiast być podzielone na odrębne, ręcznie przełączane strony}
}
\newglossaryentry{cdn}
{
    name={CDN},
    description={Skrót od \textit{Content Delivery Network}. Rozproszona sieć serwerów
    służąca do szybkiego dostarczania statycznych zasobów (np. obrazów, arkuszy CSS,
    skryptów JavaScript) z węzłów geograficznie najbliższych użytkownikowi, co zmniejsza
    opóźnienia i odciąża serwer aplikacji}
}

\newglossaryentry{react-maplibre}
{
    name={React-MapLibre},
    description={Otwartoźródłowa biblioteka do renderowania interaktywnych map
    wektorowych w przeglądarce, rozwijana jako niezależna kontynuacja Mapbox GL JS.
    Umożliwia wyświetlanie kafelków mapowych, znaczników i warstw z danymi
    geoprzestrzennymi}
}

\newglossaryentry{websocket}
{
    name={WebSocket},
    description={Protokół komunikacyjny umożliwiający dwukierunkową komunikację
    w czasie rzeczywistym między przeglądarką a serwerem po pojedynczym,
    utrzymywanym połączeniu TCP. Często wykorzystywany m.in. w czatach i aplikacjach
    działających w czasie rzeczywistym.}
}

\newglossaryentry{docker-compose}
{
    name={Docker Compose},
    description={Narzędzie do definiowania i uruchamiania wielokontenerowych aplikacji \gls{docker}
    za pomocą pliku konfiguracyjnego (np. \texttt{docker-compose.yml}). Umożliwia jednoczesne
    uruchamianie powiązanych usług (np. \gls{backend}, baza danych, usługi pomocnicze) jednym poleceniem}
}

\newglossaryentry{pro}
{
    name={PRO},
    description={Przedmiot realizowany na 5. semestrze studiów, prowadzony przez dr. inż. Martę Łabudę. W ramach przedmiotu
    wybrano temat projektu oraz
    wytworzono wstępną dokumentację projektu w tym m.in. wymagania.}
}

\newglossaryentry{prz1}
{
    name={PRZ 1},
    description={Przedmiot realizowany na 6. semestrze studiów, prowadzony w przypadku zespołu projektowego przez mgr. inż. Adama Urbanowicza. W ramach przedmiotu
    wytworzono projekt interfejsu użytkownika.}
}

\newglossaryentry{prz2}
{
    name={PRZ 2},
    description={Przedmiot realizowany na 7. semestrze studiów, prowadzony w przypadku zespołu projektowego przez mgr. inż. Adama Urbanowicza. W ramach przedmiotu
    dokończono prace nad pracą inżynierską. Pan Adam Urbanowicz jako promotor doradzał zespołowi projektowemu.}
}

\newglossaryentry{psem}
{
    name={PSEM},
    description={Przedmiot realizowany na 7. semestrze studiów, prowadzony w przypadku zespołu projektowego przez dr. inż. Marka Bednarczyka. W ramach przedmiotu
    dokończono wytwarzanie dokumentacji.}
}

\newglossaryentry{spa}
{
    name={SPA},
    description={SPA (Single Page Application) to aplikacja webowa, w której cała strona ładuje się raz,
    a późniejsze zmiany widoku odbywają się dynamicznie po stronie przeglądarki bez pełnego przeładowania strony.}
}

\newglossaryentry{routing}
{
    name={routing},
    description={Routing w \gls{spa} to warstwa w kliencie odpowiedzialna za zarządzanie stanem “aktualnej strony” na podstawie URL-a,
    zwykle z wykorzystaniem historii przeglądarki,
    tak aby interfejs reagował na zmianę ścieżki bez przeładowań z serwera.}
}



\newglossaryentry{unit-tests}
{
    name={testy jednostkowe},
    description={Testy sprawdzające poprawność działania pojedynczych, małych fragmentów kodu (np. funkcji, metod, klas) w izolacji od reszty systemu.}
}

\newglossaryentry{jakarta-validation}
{
    name={jakarta validation},
    description={Jakarta Validation to specyfikacja (i zestaw adnotacji, typu @NotNull, @Size itd.) służąca do automatycznego sprawdzania poprawności danych w aplikacjach stworzonych za pomocą Java/Jakarta EE/Spring, np. przy walidacji pól DTO, encji czy parametrów metod.}
}
\newglossaryentry{intellij-idea}
{
    name={IntelliJ IDEA},
    description={Zintegrowane środowisko programistyczne (IDE) firmy JetBrains, szeroko stosowane przy tworzeniu aplikacji backendowych w ekosystemie Spring. Oferuje m.in. podpowiedzi składni, refaktoryzację kodu, debugger oraz integrację z systemami kontroli wersji}
}

\newglossaryentry{dockerfile}
{
    name={Dockerfile},
    description={Plik tekstowy zawierający instrukcje opisujące, jak zbudować obraz Dockera (jakiej podstawy użyć, jakie pliki skopiować, jakie polecenia uruchomić). Na jego podstawie narzędzie Docker tworzy gotowy obraz kontenera}
}

\newglossaryentry{redis}
{
    name={Redis},
    description={Szybka baza danych typu klucz–wartość przechowywana głównie w pamięci operacyjnej. Często wykorzystywana jako pamięć podręczna (cache), magazyn sesji lub prosty mechanizm komunikatów między usługami}
}

\newglossaryentry{gif}
{
    name={GIF},
    description={Format graficzny \textit{Graphics Interchange Format} obsługujący krótkie, zapętlone animacje. W aplikacjach czatowych wykorzystywany do wysyłania „reakcji” w postaci ruchomych obrazków}
}

\newglossaryentry{emoji}{
    name={emoji},
    description={Małe graficzne ikonki używane do wyrażania emocji
    lub pojęć w komunikacji cyfrowej (np. uśmiechnięta buźka, kciuk w górę,
    symbol serca).}
}


\newglossaryentry{url}
{
    name={URL},
    description={Adres zasobu w internecie (ang. \textit{Uniform Resource Locator}), np. adres strony, widoku w aplikacji webowej lub konkretnego posta na forum}
}

\newglossaryentry{slug}
{
    name={Slug},
    description={Przyjazny dla użytkownika fragment adresu URL, zwykle oparty na tytule (np. \texttt{/post/jak-zaczac-latac-dronem}), ułatwiający identyfikację treści i pozycjonowanie w wyszukiwarkach}
}

\newglossaryentry{tinymce}
{
    name={TinyMCE},
    description={Popularny edytor \textit{rich text} osadzany w przeglądarce. Pozwala użytkownikowi formatować tekst (pogrubienia, listy, nagłówki, linki) w sposób przypominający klasyczny edytor tekstu, zapisując wynik zwykle w HTML}
}

\newglossaryentry{rich-text-editor}
{
    name={Rich text editor},
    description={Edytor treści, który zamiast „surowego” tekstu umożliwia stosowanie formatowania (np. pogrubienie, kursywa, listy, nagłówki, linki), dzięki czemu użytkownik może tworzyć czytelne, sformatowane wpisy}
}

\newglossaryentry{tiptap}
{
    name={Tiptap},
    description={Nowoczesny, rozszerzalny edytor \textit{rich text} dla aplikacji webowych oparty na silniku ProseMirror. Umożliwia budowanie rozbudowanych, modularnych edytorów treści, np. do postów na forum}
}

\newglossaryentry{integration-tests}
{
    name={Testy integracyjne},
    description={Testy sprawdzające, czy połączone ze sobą moduły lub usługi współpracują poprawnie — na przykład czy warstwa backendowa poprawnie komunikuje się z bazą danych, warstwą sieciową i pozostałymi komponentami systemu}
}
\newglossaryentry{endpoint}
{
    name={endpoint},
    description={Endpoint to konkretny adres (np. \gls{url}) i metoda protokołu HTTP
    w \gls{api}, które razem odpowiadają za realizację jednej, dobrze zdefiniowanej
    operacji (np. pobrania listy spotów, dodania komentarza, wyszukania spotów).}
}

\newglossaryentry{redux-slice}
{
    name={slice Redux},
    description={Slice Redux to wydzielona część globalnego stanu w \gls{redux},
    wraz z powiązanymi akcjami i reduktorami, odpowiedzialna za jeden obszar domeny
    (np. konto użytkownika, czat, mapę czy listę znajomych).}
}
\newglossaryentry{jsoup}{
    name={jsoup},
    description={Biblioteka \textit{Java} do przetwarzania dokumentów HTML,
    umożliwiająca parsowanie, przeszukiwanie i modyfikowanie struktury dokumentu
    w sposób zbliżony do pracy z DOM-em i selektorami CSS.}
}
\newglossaryentry{paginacja}
{
    name={paginacja},
    description={Mechanizm dzielenia dużych zbiorów danych
    (np. list postów, wyników wyszukiwania, komentarzy)
    na mniejsze strony, które są pobierane i wyświetlane stopniowo,
    zamiast ładowania wszystkich elementów jednocześnie.}
}


\studfield{Informatyka}
\studtype{Stacjonarne}
\title{Aplikacja webowa: spoty-na-drony.pl}
\engtitle{Web application: spoty-na-drony.pl}
\acronym{Merkury}
\titledate{2023-10-10} %todo jaka data tutaj ma byc
\supervisor{mgr Adam Urbanowicz}
\reviewer{--- brak ---} %todo dodać

\author{Langmesser Adam}{s27119}{Aplikacje Internetowe}{Stacjonarny}
\author{Redosz Mateusz}{s27094}{Aplikacje Internetowe}{Stacjonarny}
\author{Oziemczuk Stanisław}{s26982}{Aplikacje Internetowe}{Stacjonarny}
\author{Badek Kacper}{s29168}{Aplikacje Internetowe}{Stacjonarny}

\consultant{--- brak ---} % Koniecznie trzeba podać brak, albo wpisać konsultantów tak jak przy autorach
\projectgoals{Stworzenie w pełni funkcjonalnej aplikacji internetowej do rozwijania hobby(latania dronem).} %todo do poprawy
\productsandservices{Aplikacja Internetowa, Dokumentacja}

\mainfunctionalities{
    Interaktywna mapa z wyśiwetlanymi spotami oraz pogodą.\newline
    Zaawansowana wyszukiwarka spotów.\newline
    Forum do dzielenia się informacjami na temat dronów.\newline
    Chat jednoosobowy oraz grupowy.\newline
    Konto użytkownika z możliwością zapisania ulubionych spotów.
}

\successmeasure{
    Gotowa do wdrożenia aplikacja.\newline
    Realizacja w terminie zgodnym z wymaganiami.
}

\projlimitations{
    Budżetowe: brak środków na wdrożenie.\newline
    Zawodowe: brak doświadczenia.\newline
    Czasowe: trzy semestry (09.2024 - 02.2026).\newline
    Ludzkie: czteroosobowy zespół.
}

\date{miesiąc, 2100 obrony} %todo jaka to ma być data
\finishdate{\today}

%todo poprawić podejście
\nabstract{
    Celem niniejszej pracy było stworzenie w pełni funkcjonalnej i działającej aplikacji internetowej pozwalającej na szybkie wyszukiwanie spotów w okolicy oraz dzielenie się zdjęciami, filmami oraz doświdczeniem z innymi użytkownikami.
    W ramach pracy stworzono system składający się z trzech komponentów: \gls{frontend}u, \gls{backend}u oraz bazy-danych.
    Aplikacja internetowa została wykonana przy pomocy \gls{framework}a React w językach Javascript oraz Typescript, do styli został użyty Tailwind.
    Serwis backendowy został stworzony w języku Java oraz biblioteki Spring Boot.
    Baza danych to PostgreSql.\newline
    Komunikacja między komponentami odbywała się zgodnie ze standardem REST.
    Projekt został zrealizowany w podejściu ewolucyjno-przyrostowym~z~elementami Kanban.
}

\keyword{--- brak ---}%todo nie wiem czy bedzie potrzebne ale zostawiam żeby było




\begin{document}

    \maketitle
    \makeprojectcard

    %Spis treści
    \tableofcontents
    \clearpage

    %! Author = mateusz
%! Date = 20/09/2025


\chapter{Wstęp}
\label{ch:wstep}


    % Słownik pojęć i skrótów
%    \printglossary[type=\acronymtype]
    \printglossary[title={Słownik pojęć i skrótów},toctitle={Słownik pojęć i skrótów}]

    %! Author = mateusz
%! Date = 20/09/2025


\chapter{Opis problemu}
\label{ch:opis-problemu}

    %! Author = mateusz
%! Date = 20/09/2025


\chapter{Kontekst projektu}
\label{ch:kontekst-projektu}

    %! Author = mateusz
%! Date = 20/09/2025


\chapter{Analiza wymagań}
\label{ch:analiza-wymagan}

    %! Author = mateusz
%! Date = 20/09/2025


\chapter{Decyzje projektowe}
\label{ch:decyzje-projektowe}

    %! Author = mateusz
%! Date = 20/09/2025


\chapter{Projekt}
\label{ch:projekt}

    %! Author = mateusz
%! Date = 20/09/2025


\chapter{Realizacja Projektu}
\label{ch:realizacja}

    %! Author = mateusz
%! Date = 20/09/2025


\chapter{Testy}
\label{ch:testy}

    %! Author = mateusz
%! Date = 20/09/2025


\chapter{Prezentacja systemu}
\label{ch:prezentacja-systemu}

    %! Author = Mateusz Redosz
%! Date = 12/10/2025


\chapter{Nakład pracy}
\label{ch:naklad-pracy}


\section{Indywidualne nakłady pracy}
\label{sec:indywidualne-naklady-pracy}

%! Author = Adam
%! Date = 02/02/2026

\subsection{Adam Langmesser}
\label{subsec:adam-langmesser}

Na prace dotyczące projektu poświęciłem łącznie 608 godzin, w tym:
\begin{itemize}
    \item 34 godz. \textendash \space zaprojektowanie \glslink{ui}{UI} modułu mapy w Figmie
    \item 293 godz. \textendash \space prace deweloperskie
    \item 215 godz. \textendash \space tworzenie dokumentacji
    \item 66 godz. \textendash \space przeprowadzanie \glslink{review-kodu}{review kodu}
\end{itemize}

\subsubsection{Rola lidera}
\label{subsubsec:rola-lidera}

Jako lider zespołu byłem odpowiedzialny za koordynację prac, podział zadań oraz kontrolę postępów realizacji projektu.
Ponadto wypracowałem procedury włączania nowego kodu do repozytorium, obejmującą \glslink{review-kodu}{review kodu}.

\subsubsection{Prace deweloperskie}
\label{subsubsec:prace-deweloperskie-adam}

%TODO

Przygotowałem lokalne środowisko deweloperskie dla \glslink{backend}{backendu} i \glslink{frontend}{frontendu} (konfiguracja \gls{ide},
weryfikacja narzędzi budujących oraz uruchomienie projektów testowych), co ułatwiło płynne rozpoczęcie właściwych prac implementacyjnych.
Następnie wdrożyłem podstawową stronę powitalną po stronie \glslink{frontend}{frontendu} oraz zaimplementowałem fundamenty logiki logowania
i rejestracji użytkownika w \glslink{backend}{backendzie}, uwzględniając scenariusze brzegowe, takie jak próba rejestracji z zajętą nazwą użytkownika.
Rozszerzyłem też środowisko uruchomieniowe o konfigurację \gls{docker-compose} dla \glslink{backend}{backendu}
i bazy danych oraz dodałem cykliczny test weryfikujący poprawność uruchamiania aplikacji w kontenerach i jej podstawową responsywność.
W dalszej kolejności zaprezentowałem zespołowi przykładowe podejście do \glslink{integration-tests}{testów integracyjnych}
oraz \glslink{e2e-tests}{testów E2E} na \glslink{backend}{backendzie}, a równolegle przygotowałem demonstracyjny prototyp mapy
z wykorzystaniem biblioteki \gls{leaflet}, aby pokazać możliwości interaktywnego widoku mapowego,
oraz dopracowałem konfigurację \gls{eslint} po stronie \glslink{frontend}{frontendu}.
Kontynuowałem prace w obszarze bezpieczeństwa, doprecyzowując konfigurację \gls{spring-security} (role i uprawnienia),
przygotowując dane deweloperskie użytkowników oraz dodając automatyczne testy obejmujące kluczowe scenariusze logowania i rejestracji,
a następnie ustabilizowałem zestaw testów poprzez ujednolicenie asercji i rozwiązanie problemów z relacjami między
obiektami wykorzystywanymi w testach integracyjnych. Wprowadziłem kolejne
usprawnienia konfiguracji \gls{spring-security} (reguły autoryzacji i filtry),
uporządkowałem plik \texttt{.gitignore} o nowe artefakty generowane przez narzędzia
oraz poprawiłem działanie potoków \gls{cicd} dla \glslink{backend}{backendu};
jednocześnie uporządkowałem strukturę projektu po stronie serwera,
wdrożyłem mechanizm cachowania z użyciem \gls{redis}
oraz doprecyzowałem konfigurację poziomów logowania błędów,
co usprawniło diagnozowanie problemów w środowiskach deweloperskich.
Równolegle rozpocząłem implementację modułu czatu zarówno po stronie \glslink{backend}{backendowej},
jak i \glslink{frontend}{frontendowej}, dostosowując układ aplikacji oraz elementy nawigacyjne do nowej sekcji,
a także rozwinąłem procesy \gls{cicd} dla serwera, optymalizując czas budowania i doprecyzowując kroki uruchamiania testów.
Następnie znacząco rozbudowałem czat, dodając pobieranie danych i wysyłanie wiadomości z użyciem \gls{websocket},
ujednolicając model danych, upraszczając komunikację z \gls{api} oraz przebudowując strukturę \gls{redux} i komponentów,
co przełożyło się na czytelniejszy kod i spójniejsze działanie interfejsu; dodatkowo wdrożyłem usprawnienia narzędziowe
w postaci formatowania przez \gls{prettier}, skryptu \texttt{format:check} oraz weryfikacji formatowania w \gls{cicd}.
Usprawniłem mechanizm \glslink{infinite-scroll}{nieskończonego przewijania} listy czatów, przygotowałem wspólną logikę
i komponenty pomocnicze ułatwiające zespołowi korzystanie z \glslink{websocket}{websocketów} na froncie,
dodałem grupowanie wiadomości po dacie oraz obsługę wielowierszowego pola tekstowego,
a także wprowadziłem drobne korekty w potokach \gls{cicd}. Kolejnym krokiem było rozszerzenie czatu o wyszukiwanie
i wysyłanie \glslink{gif}{GIF-ów} oraz wysyłanie \gls{emoji}, a także usprawnienie komunikacji poprzez optymistyczne
wysyłanie wiadomości i dodanie nowych \glslink{endpoint}{endpointów} do stronicowanego pobierania starszych wiadomości.
Na koniec umożliwiłem rozpoczynanie i kontynuowanie prywatnych rozmów bezpośrednio z widoków społecznościowych
(list znajomych oraz obserwujących), dopracowałem obsługę \gls{emoji} i ergonomię okna wyboru \glslink{gif}{GIF-ów},
a także zaimplementowałem funkcjonalność czatów grupowych obejmującą tworzenie rozmów z wyborem uczestników,
edycję nazwy i obrazu czatu, dodawanie nowych członków oraz pełną obsługę załączników (pliki i obrazy) wraz z logiką wyboru,
podglądu i wysyłania plików również bez treści tekstowej.


\subsubsection{Tworzenie dokumentacji}
\label{subsubsec:tworzenie-dokumentacji-adam}

W ramach tworzenia dokumentacji naszej pracy inżynierskiej byłem odpowiedzialny za napisanie następujących rozdziałów oraz podrozdziałów:
\begin{itemize}
    \item \hyperref[sec:udzialowcy]{\ref*{sec:udzialowcy} Udziałowcy}
    \item \hyperref[sec:metodologia-pracy]{\ref*{sec:metodologia-pracy} Metodologia pracy}
    \item \hyperref[sec:harmonogram-projektu]{\ref*{sec:harmonogram-projektu} Harmonogram projektu}
    \item \hyperref[sec:zasoby-i-ograniczenia]{\ref*{sec:zasoby-i-ograniczenia} Zasoby i ograniczenia}
    \item \hyperref[sec:przypadki-uzycia]{\ref*{sec:przypadki-uzycia} Przypadki użycia}
    \item \hyperref[subsubsec:wymagania-ogolne-dla-chatu]{\ref*{subsubsec:wymagania-ogolne-dla-chatu} Wymagania ogólne dla czatu}
    \item \hyperref[subsubsec:wymagania-funkcjonalne-dla-chatu]{\ref*{subsubsec:wymagania-funkcjonalne-dla-chatu} Wymagania funkcjonalne dla czatu}
    \item \hyperref[subsubsec:wymagania-pozafunkcjonalne-dla-czatu]{\ref*{subsubsec:wymagania-pozafunkcjonalne-dla-czatu} Wymagania pozafunkcjonalne dla czatu}
    \item \hyperref[subsec:projekt-chatu]{\ref*{subsec:projekt-chatu} Projekt czatu}
    \item \hyperref[ch:przebieg-realizacji-projektu]{\ref*{ch:przebieg-realizacji-projektu} Przebieg realizacji projektu}
    \item \hyperref[subsubsec:chain-of-responsibility]{\ref*{subsubsec:chain-of-responsibility} Chain of Responsibility}
    \item \hyperref[subsec:bezpieczenstwo-aplikacji]{\ref*{subsec:bezpieczenstwo-aplikacji} Bezpieczeństwo aplikacji}
    \item TODO: Integracja z Tenor na backendzie
    \item TODO: czat frontend
    \item \hyperref[sec:implementacja-websocket]{\ref*{sec:implementacja-websocket} Implementacja WebSocket}
    \item \hyperref[sec:strona-chatu]{\ref*{sec:strona-chatu} Strona czatu}
    \item \hyperref[subsec:adam-langmesser]{\ref*{subsec:adam-langmesser} Adam Langmesser}
\end{itemize}

%! Author = Mateusz
%! Date = 12/10/2025

\subsection{Mateusz Redosz}
\label{subsec:mateusz-redosz}

Na realizację projektu poświęciłem łącznie XXX godzin, z czego 256 przeznaczyłem na prace deweloperskie,
XXX na przygotowanie dokumentacji, XX godzin na \glslink{review-kodu}{review kodu}, XX na spotkania dotyczące
omówienia dalszych prac projektowych oraz pomocy innym członkom zespołu, a także 49 godzin na przygotowanie
widoków w Figmie.
Prace nad częścią deweloperską rozpocząłem 04.08.2024, a zakończyłem 08.09.2025.

W ramach projektu odpowiadałem za implementację rejestracji użytkownika oraz częściową obsługę tokenu \gls{jwt}.
Pracowałem również nad wybranymi elementami \glslink{pipeline}{pipeline’u} \gls{cicd}, wspierając przygotowanie
automatyzacji budowania oraz testowania aplikacji.
Dużą część czasu poświęciłem na rozwój głównych modułów systemu, w szczególności wyszukiwarki spotów oraz panelu
użytkownika, zarówno na \glslink{frontend}{frontendzie}, jak i \glslink{backend}{backendzie}.
Oba moduły są responsywne na dużych i małych ekranach oraz zostały przygotowane w motywie jasnym i ciemnym.
Panel użytkownika przetestowałem testami jednostkowymi, integracyjnymi oraz E2E.

Dodatkowo zaimplementowałem \glslink{sidebar}{sidebar}, dbając o \glslink{responsywnosc}{responsywność} i
dopasowanie układu do różnych rozdzielczości, tak aby korzystanie z aplikacji było wygodne zarówno na
urządzeniach mobilnych, jak i na komputerach.
Na początku przedmiotu \gls{prz1} przygotowałem w Figmie projekt widoków dla panelu użytkownika,
wyszukiwarki spotów oraz panelu logowania i rejestracji, dbając o spójność stylu, czytelność interfejsu
oraz zgodność z założeniami funkcjonalnymi.

W trakcie prac regularnie konsultowałem postępy z zespołem, brałem udział w spotkaniach oraz pomagałem
innym osobom w rozwiązywaniu problemów związanych z implementacją i konfiguracją środowiska.
Po wykonaniu kluczowych funkcjonalności wprowadzałem poprawki wynikające z testów oraz uwag z \gls{review-kodu},
skupiając się na jakości, stabilności oraz utrzymaniu jednolitego standardu kodu w projekcie.

Przy pisaniu pracy dyplomowej odpowiadałem za następujące rozdziały:
-

%! Author = mateusz
%! Date = 12/10/2025

\subsection{Stanisław Oziemczuk}
\label{subsec:stanislaw-oziemczuk}
%! Author = kacper
%! Date = 12/10/2025

\subsection{Kacper Badek}
\label{subsec:kacper-badek}

Na prace związane z realizacją projektu poświęciłem łącznie X godzin, w tym:
\begin{itemize}
    \item 29 godzin \textendash \space zaprojektowanie interfejsu użytkownika modułu forum w figmie
    \item B godzin \textendash \space prace deweloperskie
    \item C godzin \textendash \space tworzenie dokumentacji
    \item D godzin \textendash \space udział w spotkaniach zespołu projektowego oraz pomoc w rozwiązywaniu problemów związanych z realizowanymi zadaniami
    \item E godzin \textendash \space przeprowadzanie \glslink{review-kodu}{review kodu}
\end{itemize}

\subsubsection{Prace deweloperskie}

Jednym z moich pierwszych zadań było przygotowanie formularza logowania użytkownika po stronie \glslink{frontend}{frontendu} oraz jego integracja z \glslink{backend}{backendowym} API odpowiedzialnym za uwierzytelnianie użytkowników.
Następnie zaimplementowałem mechanizm resetowania hasła, obejmujący generowanie oraz weryfikację tokenów, formularz zmiany hasła, a także logikę cyklicznego usuwania przeterminowanych tokenów.
W dalszym etapie poprawiłem obsługę błędów po stronie \glslink{backend}{backendu} oraz dostosowałem treść i wygląd wiadomości e-mail wysyłanych przez system, dbając o ich czytelność i spójność wizualną.

Istotnym elementem projektu był również rozwój systemu mailowego.
Zmieniłem sposób wysyłania wiadomości po rejestracji użytkownika na asynchroniczny oraz dodałem mechanizm ponawiania prób wysyłania wiadomości e-maili z ograniczoną liczbą prób i logowaniem nieudanych operacji.
Przeprowadziłem także refaktoryzację szablonów wiadomości związanych z rejestracją i resetowaniem hasła, ujednolicając strukturę HTML oraz styl graficzny, w tym logo aplikacji.

W ramach prac nad modułem mapy przygotowałem dane deweloperskie, takie jak przykładowe \glslink{spot}{spoty} wykorzystywane do prezentacji i testów.
Zaimplementowałem funkcjonalność dodawania \glslink{spot}{spotów} do listy ulubionych po stronie \glslink{backend}{backendu} i \glslink{frontend}{frontendu}, a także widok umożliwiający użytkownikowi przeglądanie zapisanych lokalizacji.

Moją największą częścią projektu była praca nad modułem forum.
Rozpocząłem ją od przygotowania projektu interfejsu użytkownika w figmie (w ramach przedmiotu \glslink{pro}{PRO}), a następnie zaimplementowałem \glslink{backend}{backend} i \glslink{frontend}{frontend} obejmujący obsługę postów, kategorii, tagów oraz paginacji.

Równolegle do rozwoju forum zintegrowałem \glslink{backend}{backend} z usługą \glslink{azure-blob-storage}{Azure Blob Storage}, przeznaczoną do uploadu i przechowywania plików multimedialnych wykorzystywanych w całym systemie.

Przygotowałem formularz dodawania postów z wykorzystaniem edytora rich text TinyMCE, wraz z walidacją danych po stronie \glslink{backend}{backendu} i \glslink{frontend}{frontendu}.
W celu zabezpieczenia aplikacji skonfigurowałem bibliotekę jsoup do walidacji przesyłanej treści HTML, ograniczając dozwolone elementy i style.
W późniejszym etapie zdecydowałem się na zmianę edytora na Tiptap, który został skonfigurowany od podstaw, umożliwiając pełną kontrolę nad dostępnymi funkcjonalnościami edycji treści.

Forum zostało rozbudowane o mechanizmy sortowania oraz paginacji w formie infinite scroll, czytelne adresy \glslink{url}{URL} oparte na slugach tytułów oraz usprawnioną nawigację.
Wprowadzono możliwość edycji postów, a komentarze rozszerzono o funkcje dodawania, edycji i odpowiadania.
Dodatkowo zaimplementowano system głosowania i zgłaszania treści, dostępny zarówno dla wpisów, jak i komentarzy, a także funkcjonalność obserwowania wybranych dyskusji przez użytkowników.

W celu poprawy doświadczenia użytkownika dodałem struktury typu skeleton do ładowania danych, formularze dostępne globalnie poprzez stan aplikacji oraz pełne wsparcie dla jasnego i ciemnego motywu kolorystycznego.
Forum zostało również wyposażone w wyszukiwarkę umożliwiającą filtrowanie postów według tytułu, kategorii, tagów, zakresu dat oraz autora, a także stronę prezentującą wyniki wyszukiwania wraz z informacją o liczbie znalezionych elementów.

Dodatkowo przygotowałem strony listujące wszystkie kategorie i tagi alfabetycznie, stronę regulaminu forum, widok postów obserwowanych przez użytkownika oraz sekcję prezentującą najpopularniejsze posty z ubiegłego miesiąca.
Całość została zaprojektowana z myślą o czytelności, spójności wizualnej oraz wygodnej nawigacji.

Wyżej opisane zadania wymagały prac zarówno po stronie \glslink{frontend}{frontendu}, jak i \glslink{backend}{backendu}.

\subsubsection{Tworzenie dokumentacji}

W ramach tworzenia dokumentacji pracy inżynierskiej byłem odpowiedzialny za opracowanie następujących rozdziałów i podrozdziałów:
\begin{itemize}
    \item \hyperref[ch:wstep]{\ref*{ch:wstep} Wstęp}
    \item \hyperref[ch:opis-problemu]{\ref*{ch:opis-problemu} Opis problemu}(z wyjątkiem opisu udziałowców)
    \item \hyperref[sec:analiza-ryzyka]{\ref*{sec:analiza-ryzyka} Analiza ryzyka}
    \item Wymagania ogólne dla forum
    \item Wymagania funkcjonalne dla forum
    \item Wymagania pozafunkcjonalne dla forum
    \item Implementacja frontendu forum
    \item Projekt forum
    \item Implementacja systemu mailowego
    \item \hyperref[sec:strona-forum]{\ref*{sec:strona-forum} Prezentacja systemu forum}
    \item \hyperref[subsec:kacper-badek]{\ref*{subsec:kacper-badek} ninijeszy rozdział}
\end{itemize}

Podczas opracowywania powyższych treści uzupełniłem również \texttt{Słownika pojęć i skrótów} oraz \texttt{Bibliografię} o niezbędne pozycje.

    %! Author = mateusz
%! Date = 20/09/2025


\chapter{Podsumowanie}
\label{ch:podsumowanie}


    \printbibliography[title={Bibliografia}, heading=bibintoc]
    \makethesisattachments

\end{document}